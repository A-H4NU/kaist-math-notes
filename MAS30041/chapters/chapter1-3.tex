\documentclass[../complex_variables_1.tex]{subfiles}

\begin{document}

\section{Polar Representation}

Given \(z \in \CC\), \(\abs{z}\) is unique.
\(\arg z = \theta + 2k \pi\) (\(k \in \ZZ\))
(Or \(\arg z = \theta \pmod{2\pi}\))

\begin{Definition}{}[]
    If \(z = |z| \cdot (\cos\theta+i\sin\theta)\), \(\theta\) is called an \emph{argument} of \(z\)
    and is written \(\arg z = \theta \pmod{2\pi}\) (as \(\theta + 2k\pi\) for \(k \in \ZZ\)
    is an argument of \(z\) as well).
    If \(\arg z = \theta^\ast \pmod{2\pi}\), and if \(-\pi < \theta^\ast \le \pi\), then we define
    \(\Arg z = \theta^\ast\) and it is called the \emph{principal argument} of \(z\).
\end{Definition}

\begin{Theorem}{}[]
    For \(z_1, z_2 \in \CC\) with \(z_1, z_2 \neq 0\),
    \(\arg z_1z_2 = \arg z_1 + \arg z_2 \pmod{2\pi}\).
\end{Theorem}
\begin{myproof}[Proof]
    Let \(\arg z_1 = \theta_1 \pmod{2\pi}\) and \(\arg z_2 = \theta_2 \pmod{2\pi}\)
    Then, \(z_1 = \abs{z_1} (\cos\theta_1 + i \sin\theta_1)\)
    and \(z_2 = \abs{z_2} (\cos\theta_2 + i \sin\theta_2)\).
    Now, we have \(z_1 \cdot z_2 = \abs{z_1}\abs{z_2} \bigl( \cos(\theta_1+\theta_2)+i
    \sin(\theta_1+\theta_2) \bigr)\).
\end{myproof}

\end{document}
