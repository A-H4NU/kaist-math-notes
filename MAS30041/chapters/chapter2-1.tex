\documentclass[../complex_variables_1.tex]{subfiles}

\begin{document}

\section{Exponential Functions}

\begin{Definition}{Exponential Function}[]
    For each \(z = x + iy\) where \(x, y \in \RR\), we define
    \(e^z \coloneqq e^x \cdot (\cos y + i \sin y)\).
\end{Definition}

\begin{Theorem}{}[]
    For each \(z \in \CC\), \(e^z = \sum_{j=1}^\infty \frac{z^j}{j!}\).
\end{Theorem}
\begin{myproof}[Proof]
    Proved later using complex integral.
\end{myproof}

\begin{Theorem}{}[basicExpo]
    For each \(z, z' \in \CC\),
    \begin{enumerate}[label=(\alph*)]
        \ii \(e^{z+z'} = e^z \cdot e^{z'}\),
        \ii \(e^{-z} = \frac{1}{e^z}\), and
        \ii \(e^{z+2k\pi i} = e^z\) for all \(k \in \ZZ\).
    \end{enumerate}
\end{Theorem}

\begin{Definition}{}[]
    For each \(z \in \CC\),
    \begin{enumerate}[label=(\arabic*)]
        \ii \(\cos z \coloneqq \frac{e^{iz}+e^{-iz}}{2}\)
        \ii \(\sin z \coloneqq \frac{e^{iz}-e^{-iz}}{2i}\)
        \ii \(\cosh z = \frac{e^z + e^{-z}}{2}\)
        \ii \(\sinh z = \frac{e^z - e^{-z}}{2}\)
    \end{enumerate}
\end{Definition}

\begin{Theorem}{}[]
    For each \(z \in \CC\), we have \(\cosh z = \cos(iz)\)
    and \(\sinh z = -i \sin(iz)\).
\end{Theorem}

\begin{Example}{}[]
    Let us solve \(\cos z = 2\).
    Let \(t \coloneqq e^{iz}\) to obtain \(t^2 - 4t + 1 = 0\),
    which gives \(t = 2 \pm \sqrt{3}\).
    Write \(z = x + iy\) where \(x, y \in \RR\)
    to have \(e^{ix} e^{-y} = 2 \pm\sqrt{3}\).
    Taking modulus to both sides gives \(e^{-y} = 2 \pm \sqrt{3}\),
    i.e., \(y = -\ln (2 \pm \sqrt{3})\).
    Taking argument to both sides gives \(x = 2k\pi\)
    for \(k \in \ZZ\).
    Thus, \(z = 2k\pi - i \ln(2\pm\sqrt{3})\)
    for \(k \in \ZZ\).
\end{Example}

\end{document}
