\documentclass[../complex_variables_1.tex]{subfiles}

\begin{document}

\section{Complex Plane}

\begin{Definition}{Complex Number}[]
    \(i \coloneqq \sqrt{-1}\)
    is called the \emph{imaginary unit}.
    \(\CC \coloneqq \{\,x + iy \mid x, y \in \RR\,\}\) is the set of complex numbers
    where \(\RR\) is the set of real numbers.
\end{Definition}

\begin{Definition}{Algebras of \(\CC\)}[]
    For \(z_k \coloneqq x_k + iy_k\) where \(k \in \ZZ_+\) and \(x_k, y_k \in \RR\),
    \begin{itemize}
        \ii \(z_1 + z_2 \coloneqq (x_1 + x_2) + i(y_1 + y_2)\)
        \ii \(z_1 \cdot z_2 \coloneqq (x_1x_2-y_1y_2)+i(x_1y_2+x_2y_1)\).
    \end{itemize}
\end{Definition}

\begin{Theorem}{}[]
    \(\CC\) is a field.
\end{Theorem}
\begin{myproof}[Proof]
    Trivial.
\end{myproof}

\begin{note}
    \(z = a + ib\), \(a, b \in \RR\) with \(z \neq 0\).
    Then, \(z\inv = \frac{1}{a+ib} = \frac{a-ib}{a^2+b^2}\).
\end{note}

% \begin{Definition}{Conjugation}[]
%     For \(z = a+ib\), define \(\ol{z} \coloneqq a-ib\).
% \end{Definition}


\end{document}
