\documentclass[../introduction_to_set_theory_Note.tex]{subfiles}

\begin{document}

\subsection*{Selected Problems}

\setexernumber{3}

\exer[5.1.4]{}{
    For every cardinal number \(\kappa\), \(\kappa^\kappa \le 2^{\kappa \cdot \kappa}\).
}
\pf{Proof}{
    As \(\kappa \le 2^\kappa\),
    we have \(\kappa^\kappa \le (2^\kappa)^\kappa = 2^{\kappa \cdot \kappa}\)
    by \ref{itm:cardExp.ii}.
}

\exer[5.1.5]{}{
    If \(|A| \le |B|\) and \(A \neq \OO\), then there exists \(f \colon B \twoheadrightarrow A\).
}
\pf{Proof}{
    Fix some \(a \in A\).
    Let \(g \colon A \hookrightarrow B\).
    Then, define \(f \colon B \twoheadrightarrow A\) by
    \[
        f(b) \triangleq \begin{cases}
            g\inv(b) & \text{if } b \in \ran g \\
            a        & \text{otherwise.}
        \end{cases}
    \]
}

\exer[5.1.6]{}{
    If there exists \(g \colon B \twoheadrightarrow A\),
    then \(2^{|A|} \le 2^{|B|}\).
}
\pf{Proof}{
    Let \(g \colon B \twoheadrightarrow A\).
    Define \(f \colon \mcal P(A) \to \mcal P(B)\)
    by \(X \mapsto g\inv[X]\).
    Then, \(f\) is injective.
}

\dfn[dedekindInfinite]{Dedekind Infinite Set}{
    \begin{itemize}[nolistsep, leftmargin=*, listparindent=\parindent]
        \ii
        A set \(X\) is called \textit{Dedekind infinite} if
        there exists an injection on \(X\) onto its proper subset.
        \ii
        A set \(X\) is called \textit{Dedekind finite} if
        \(X\) is not Dedekind infinite.
    \end{itemize}
}

\setexernumber{7}

\exer[5.1.8]{}{
    A Dedekind infinite set is infinite.
}
\pf{Proof}{
    Let \(X\) be a set and let \(f \colon X \hookrightarrow X\) with \(f[X] \subsetneq X\).
    Suppose \(|X| = n\) for some \(n \in \NN\) for the sake of contradiction.
    Then, \(|f[X]| = |X| = n\) as \(f\) is injective,
    but this is impossible by \Cref{lem:finiteProperSubset}.
}

\exer[5.1.9]{}{
    Every countably infinite set is Dedekind infinite.
}
\pf{Proof}{
    It is enough to show that \(\NN\) is Dedekind infinite.
    By \Cref{exer:3.2.2}, \(f \colon \NN \to \NN\)
    defined by \(n \mapsto n + 1\) is injective
    but \(0 \notin \ran f\).
    Hence, \(\NN\) is Dedekind infinite.
}

\exer[5.1.10]{}{
    If \(X\) has a countably infinite subset, then \(X\) is Dedekind infinite.
}
\pf{Proof}{
    Let \(Y \subseteq X\) and \(|Y| = \aleph_0\).
    By \Cref{exer:5.1.9}, there exists \(f \colon Y \hookrightarrow Y\)
    such that \(\ran f \subsetneq Y\).
    Let \(g \triangleq f \cup \mrm{Id}_{X \setminus Y}\).
    Then, \(g \colon X \hookrightarrow X\)
    and \(Y \setminus \ran f \subseteq X \setminus \ran g\).
    Therefore, \(X\) is Dedekind infinite.
}

\exer[5.1.11]{}{
    If \(X\) is Dedekind infinite, then \(X\) has a countably infinite subset.
}
\pf{Proof}{
    Let \(f \colon X \hookrightarrow X\) with \(\ran f \subsetneq X\).
    Fix any \(x \in X \setminus \ran f\).
    Define \(\lang x_n \rang_{n \in \NN}\) by
    \begin{alignat*}{2}
        && x_0 &\triangleq x \\
        \fall n \in \NN,&\quad& x_{n+1} &\triangleq f(x_n).
    \end{alignat*}

    Let \(\mbf{P}(n)\) be the property ``\(\fall m < n,\: x_m \neq x_n\).''
    \(\mbf{P}(0)\) is vacuously true, and
    \(\mbf{P}(1)\) follows from \(x_0 \notin \ran f\).

    Fix \(n \ge 1\) and assume \(\mbf{P}(n)\) for the sake of induction.
    Then, for each \(0 < m < n\),
    \(x_{n} = f(x_{n-1}) \neq f(x_{m-1}) = x_{m}\) as \(f\) is injective.
    \(x_0 \neq x_{n}\) since \(x_0 \notin \ran f\).
    Hence, \(\mbf{P}(n+1)\) holds.
    By \nameref{th:induction}, \(\lang x_n \rang_{n \in \NN}\) is injective,
    and thus \(\{\,x_n \mid n \in \NN\,\}\) is a countably infinite
    subset of \(X\).
}

\nt{
    \Cref{exer:5.1.10} and \Cref{exer:5.1.11}
    say that \(X\) is Dedekind infinite if and only if
    \(X\) has a countably infinite subset.
    In \Cref{chap:aoc}, we will show that
    a set is Dedekind infinite if and only if it is infinite
    using \nameref{ax:choice}. (See \Cref{th:infIffDedekindInf}.)
}

\exer[5.1.12]{}{
    If \(A\) and \(B\) are Dedekind finite, then \(A \cup B\) is Dedekind finite.
}
\pf{Proof}{
    Suppose \(A \cup B\) is Dedekind infinite for the sake of contradiction.
    Then, by \Cref{exer:5.1.11}, there exists \(C \subseteq A \cup B\)
    such that \(C\) is countably infinite.
    Noting that at least one of \(A \cap C\) and \(B \cap C\) is countably infinite,
    by \Cref{exer:5.1.10}, we conclude \(A\) or \(B\) is Dedekind infinite,
    which is a contradiction.
}

\exer[5.1.13]{}{
    If \(A\) and \(B\) are Dedekind finite, then \(A \times B\) is Dedekind finite.
}
\pf{Proof}{
    Suppose \(A \times B\) is Dedekind infinite for the sake of contradiction.
    Then, by \Cref{exer:5.1.11}, there exists \(C \subseteq A \times B\)
    such that \(C\) is countably infinite.
    Let \(A' \triangleq \dom C\) and \(B' \triangleq \ran C\).
    If \(A'\) and \(B'\) were both finite,
    then \(C \subseteq A' \times B'\) would be finite by
    \Cref{exer:4.2.2} and \Cref{th:subsetOfFiniteIsFinite}.

    \WLOG, \(A'\) is infinite.
    Let \(f \colon \NN \twoheadrightarrow C\).
    Then, define \(g \colon \NN \twoheadrightarrow A'\) by
    \(n \mapsto a'\) where \((a', b') = f(n)\).
    Hence, by \Cref{th:imageOfCountable}, \(\ran g = A'\) is countably infinite.
    Therefore, by \Cref{exer:5.1.10}, \(A\) is Dedekind infinte,
    which is a contradiction.
}

\end{document}
