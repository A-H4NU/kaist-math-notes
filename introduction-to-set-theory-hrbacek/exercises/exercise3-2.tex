\documentclass[../introduction_to_set_theory.tex]{subfiles}

\begin{document}

\subsection*{Selected Problems}

\setexernumber{1}

\exer[3.2.2]{}{
    \(\fall m, n \in \NN,\: (m < n \implies m + 1 < n + 1)\).
    Hence, \(S \colon \NN \to \NN\) where \(n \mapsto n+1\) defines a one-to-one function on \(\NN\).
}
\pf{Proof}{
    By \Cref{clm:nPlusOne} in the proof of \nameref{th:NisTotallyOrdered},
    we have \(m+1 \le n\). Together with \(n < n + 1\), we have
    \(m + 1 < n + 1\).

    Now, take any \(m, n \in \NN\) with \(m \neq n\).
    Then, by \nameref{th:NisTotallyOrdered}, we have \(m < n\) or \(n < m\),
    i.e., \(S(m) < S(n)\) or \(S(n) < S(m)\).
    In both cases, \(S(m) \neq S(n)\).
    Therefore, \(S\) is one-to-one.
}

\exer[3.2.3]{}{
    There exists \(X \subsetneq \NN\) and \(f \colon \NN \to X\)
    such that \(f\) is injective.
}
\pf{Proof}{
    Let \(S \colon \NN \to \NN\) where \(n \mapsto n + 1\).
    Then, \(S\) is injective by \Cref{exer:3.2.2}.
    Since there exists no \(n \in \NN\) such that \(n \cup \{n\} = \OO\),
    \(0 \notin \ran S\); \(\ran S \subsetneq \NN\).
    Therefore, \(S \colon \NN \to \ran S\) is the function we are looking for.
}

\exer[3.2.4]{}{
    \(\fall n \in \NN \setminus \{0\},\: \exs! k \in \NN,\: n = k + 1\)
}
\pf{Proof}{
    Let \(\mbf{P}(x)\) be the property ``\(x = 0 \lor \exs! k \in \NN,\: x = k + 1\).''
    \(\mbf{P}(0)\) holds by definition.

    Now, assume \(\mbf{P}(n)\) where \(n \in \NN\).
    There exists \(k \in \NN\) such that \(n + 1 = k + 1\), namely, \(k = n\).
    If \(k'\) is another natural number such that \(n + 1 = k' + 1\),
    then by \Cref{exer:3.2.2}, we have \(k = k'\).
    Hence, \(\mbf{P}(n+1)\) holds.
    The result follows from \nameref{th:induction}.
}

\setexernumber{5}

\exer[3.2.6]{}{
    \(\fall n \in \NN,\: n = \{\,m \in \NN \mid m < n\,\}\)
}
\pf{Proof}{
    Let \(\mbf{P}(x)\) be the property ``\(x = \{\,m \in \NN \mid m < x\,\}\).''
    We have \(\mbf{P}(0)\) since there exists no \(m \in \NN\) with \(m < 0\).

    Now, assume \(\mbf{P}(n)\) where \(n \in \NN\).
    Then, \(n + 1 = \{\,m \in \NN \mid m < n\,\} \cup \{n\}\).
    By \ref{itm:basicLess.ii}, \(m < n + 1\) if and only if \(m < n\) or \(m = n\).
    Therefore,
    \(\{\,m \in \NN \mid m < n\,\} \cup \{n\} = \{\,m \in \NN \mid m < n \lor m = n\,\}
    = \{\,m \in \NN \mid m < n + 1\,\}\); \(\mbf{P}(n+1)\) holds.
    The result follows from \nameref{th:induction}.
}

\setexernumber{7}

\exer[3.2.7]{}{
    There is no function \(f \colon \NN \to \NN\) such that
    \(\fall n \in \NN,\: f(n + 1) < f(n)\).
}
\pf{Proof}{
    Let \(\mbf{P}(x)\) be the property ``there is no function \(f \colon \NN \to \NN\) such that
    \(f(0) = x\) and \(\fall n \in \NN,\: f(n + 1) < f(n)\).''

    For the sake of induction, assume \(\fall k < n,\: P(k)\) where \(n \in \NN\).
    Suppose there exists \(f \colon \NN \to \NN\) such that
    \(f(0) = n\) and \(\fall k \in \NN,\: f(k + 1) < f(k)\).
    Now, define \(g \colon \NN \to \NN\) by \(g(k) = f(k + 1)\).
    Then, \(g(0) = f(1) < n\) and \(\fall k \in \NN,\: g(k + 1) = f((k + 1) + 1) < f(k + 1) = g(k)\).
    However, by \(\mbf{P}(g(0))\), such \(g\) cannot exist;
    by contradiction, \(\mbf{P}(n)\) holds.
    Hence, \(\fall m \in \NN,\: \mbf{P}(m)\) by \nameref{th:strongInduction}.

    Finally, suppose there exists \(f \colon \NN \to \NN\) such that \(\fall n \in \NN,\: f(n + 1) < f(n)\).
    Then, by \(\mbf{P}(f(0))\), such \(f\) may not exist.
}

\setexernumber{10}

\exer[3.2.11]{}{
    Let \(\mbf{P}(x)\) be a property and let \(k \in \NN\).
    \[
        \mbf{P}(k) \land \fall n \ge k,\: (\mbf{P}(n) \implies \mbf{P}(n+1))
        \implies \fall n \ge k,\: \mbf{P}(n)
    \]
}
\pf{Proof}{
    Let \(\mbf{Q}(x)\) be the property ``\(x < k \lor \mbf{P}(x)\).''
    If \(k = 0\), then \(\mbf{P}(0)\) holds. If \(k > 0\), then \(0 < k\) holds.
    Hence, in both cases, \(\mbf{Q}(0)\) holds.

    Now assume \(\mbf{Q}(n)\) holds where \(n \in \NN\).
    Then, by \nameref{th:NisTotallyOrdered}, we have \(n + 1 < k\), \(n + 1 = k\), or \(n + 1 > k\).
    If \(n + 1 < k\) or \(n + 1 = k\), we immediately have \(\mbf{Q}(n + 1)\).
    If \(n + 1 > k\), we have \(n \ge k\) by \ref{itm:basicLess.ii}.
    Therefore, \(\mbf{P}(n)\) holds, and thus \(\mbf{P}(n+1)\) holds by assumption.
    Hence, \(\mbf{Q}(n+1)\).
    By \nameref{th:induction}, \(\fall n \in \NN,\: n < k \lor \mbf{P}(n)\).
    In other words, \(\fall n \ge k,\: \mbf{P}(n)\).
}

\exer[3.2.12]{The Finite Induction Principle}{
    Let \(\mbf{P}(x)\) be a property and let \(k \in \NN\).
    \[
        \mbf{P}(0) \land \fall n < k,\: (\mbf{P}(n) \implies \mbf{P}(n+1))
        \implies \fall n \le k,\: \mbf{P}(n)
    \]
}
\pf{Proof}{
    Let \(\mbf{Q}(x)\) be the property ``\(x > k \lor \mbf{P}(x)\).''
    \(\mbf{Q}(0)\) holds as \(\mbf{P}(0)\).

    Now, assume \(\mbf{Q}(n)\) holds where \(n \in \NN\).
    Then, by \nameref{th:NisTotallyOrdered}, we have \(n + 1 \le k\) or \(n + 1 > k\).
    If \(n + 1 > k\), then we immediately have \(\mbf{Q}(n + 1)\).
    If \(n + 1 \le k\), by \Cref{lem:basicLess}, \(n + 1 < k + 1\).
    By \Cref{exer:3.2.2} and \nameref{th:NisTotallyOrdered}, we must have \(n < k\).
    Hence, \(\mbf{P}(n)\) holds, and therefore \(\mbf{P}(n+1)\) holds by the assumption.
    By \nameref{th:induction}, \(\fall n \in \NN,\: n > k \lor \mbf{P}(n)\).
    In other words, \(\fall n \le k,\: \mbf{P}(n)\).
}

\exer[3.2.13]{The Double Induction Principle}{
    Let \(\mbf{P}(x, y)\) be a property.
    \begin{gather}
        \fall m, n \in \NN,\: [\fall k, \ell \in \NN,\: (k < m \lor k = m \land \ell < n \implies \mbf{P}(k, \ell)) \implies \mbf{P}(m, n)]\tag{\(\ast\)}\label{eq:doubleInduction}\\
        \implies \fall m, n \in \NN,\: \mbf{P}(m, n)\nonumber
    \end{gather}
}
\pf{Proof}{
    Let \(\mbf{Q}(x)\) be the property ``\(\fall n \in \NN,\: \mbf{P}(x, n)\).''

    Now, assume \(\fall k < m,\: \mbf{Q}(k)\) where \(m \in \NN\).
    For the sake of induction, assume again that \(\fall \ell < n,\: \mbf{P}(m, \ell)\) where \(n \in \NN\).
    Now, we have \(\mbf{P}(k, \ell)\) for all \(k, \ell \in \NN\) such that
    \(k < m\) or \(k = m\) and \(\ell < n\).
    Hence, by \eqref{eq:doubleInduction}, \(\mbf{P}(m, n)\).
    By \nameref{th:strongInduction}, we have \(\fall n \in \NN,\: \mbf{P}(m, n)\).
    In other words, \(\mbf{Q}(m)\).
    Again by \nameref{th:strongInduction}, we have \(\fall m \in \NN,\: \mbf{Q}(m)\),
    that is to say \(\fall m, n \in \NN,\: \mbf{P}(m, n)\).
}

\end{document}
