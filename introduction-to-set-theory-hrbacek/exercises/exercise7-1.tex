\documentclass[../introduction_to_set_theory_Note.tex]{subfiles}

\begin{document}

\subsection*{Selected Problems}

\exer[7.1.1]{}{
    If \(X\) is an infinite well-orderable set,
    then \(X\) has nonisomorphic well-orderings.
}
\pf{Proof}{
    Let \(R \subseteq X \times X\) be a well-ordering of \(X\).
    Then, by \nameref{th:counting}, there exists \(\alpha \in \Ord\)
    such that \((X, R) \cong \alpha\).
    Let \(f \colon X \bijto \alpha\).

    By \Cref{th:finiteOrdinalIsNat}, \(\omega \subseteq \alpha\).
    By \Cref{exer:4.3.2}, there exists \(g \colon \omega \bijto (\omega \cup \{\alpha\})\).
    Define \(f' \colon X \bijto (\alpha + 1)\) by
    \[
        f'(x) \triangleq \begin{cases}
            g(f(x)) & \text{if}~f(x) \in \omega \\
            f(x)    & \text{otherwise}.
        \end{cases}
    \]
    Then, \(R' \triangleq \{\,(x, y) \in X^2 \mid f'(x) < f'(y) \,\}\)
    is a well-ordering of \(X\) isomorphic to \(\alpha + 1\).
}

\exer[7.1.2]{}{
    Let \(\alpha, \beta \in \Ord\) where \(\alpha\) and \(\beta\) are countable.
    Then, \(\alpha + \beta\), \(\alpha \cdot \beta\), and \(\alpha^\beta\)
    are countable.
}
\mclm{Proof}{\hfill
\begin{itemize}[nolistsep, leftmargin=*, listparindent=\parindent]
    \ii
    By \Cref{th:sumOfWosetAndOrdinal},
    \(\alpha + \beta\) is isomorphic to the sum of \(\alpha\) and \(\beta\),
    which is countable by \Cref{th:unionOfCountable}. \checkmark

    \ii
    By \Cref{th:orderTypeOfProduct},
    \(\alpha \cdot \beta\) is isomorphic to the lexicographical ordering of \(\beta \times \alpha\),
    which is countable by \Cref{th:productOfCountable}. \checkmark

    \ii
    We only consider the case where both \(\alpha\) and \(\beta\) are countably infinite.
    \(S(\beta, \alpha)\) in \Cref{exer:6.5.16} is equipotent to \(\alpha^\beta\).
    We have
    \begin{align*}
        |S(\beta, \alpha)|
        &= |\{\,f \colon \beta \to \alpha \mid s(f)~\text{if finite}\,\}| \\
        &= \left|\textstyle \bigcup_{n \in \omega} \big\{\,f \colon \beta \to \alpha \:\big|\: |s(f)| = n\,\big\}\right|.
    \end{align*}
    For each \(n \in \omega\), as \(A_n \triangleq
    \big\{\,f \colon \beta \to \alpha \:\big|\: |s(f)| = n\,\big\} \subseteq S(\beta, \alpha)\),
    it is well-ordered by \(\preceq\) (in \Cref{exer:6.5.16});
    hence there exists a unique ordinal \(\gamma_n\) isomorphic to \(A_n\).
    Let \(h_n\) be the unique isomorphism between \(\gamma_n\) and \(A_n\).
    We also have \(|A_n| = \big| [\beta]^n \times \prod_{i<n} \alpha \big| = \aleph_0 \cdot \aleph_0^n = \aleph_0\)
    for \(n > 0\) by \Cref{th:productOfCountable,exer:4.3.5}.

    Hence, \(\left|\bigcup_{n \in \omega} \big\{\,f \colon \beta \to \alpha \:\big|\: |s(f)| = n\,\big\}\right|
    = \big|\bigcup_{n \in \omega} \ran h_n\big| = \aleph_0\)
    by \Cref{th:unionOfCountable}. \checkmark \qed
\end{itemize}
}

\exer[7.1.3]{}{
    \(\fall A,\: \exs f \colon \mcal P(A \times A) \surjto h(A)\)
}
\pf{Proof}{
    For each well-ordering \(R\) of \(W \subseteq A\), let \(\alpha_R\)
    be the unique ordinal isomorphic to \((W, R)\) thanks to \nameref{th:counting}.
    Note that \(|\alpha_R| = |A| < |h(A)|\) by \Cref{lem:ordinalHartogsGreater}, and thus \(\alpha_R < h(A)\).
    Define a function \(f \colon \mcal P(A \times A) \to h(A)\) by
    \[
        f(R) \triangleq \begin{cases}
            \alpha_R & \text{if}~R~\text{is a well-ordering of}~\field R \\
            0 & \text{otherwise.}
        \end{cases}
    \]
    Then, \(\ran f\) exactly equals \(H\) defined in the proof of
    \Cref{lem:hartogsNumberExists}, which in turn equals \(h(A)\).
}

\exer[7.1.4]{}{
    \(\fall A,\: |A| < |A| + |h(A)|\)
}
\pf{Proof}{
    \WLOG, \(A \cap h(A) = \OO\).
    Since we already have \(|A| \le |A| + |h(A)|\), we only need to prove
    \(|A| \neq |A| + |h(A)|\).

    Suppose \(|A| = |A| + |h(A)|\) for the sake of contradiction.
    Then, we have \(|h(A)| \le |A| + |h(A)| = |A|\),
    which contradicts \(|h(A)| \not\le |A|\).
}

\exer[7.1.5]{}{
    \(\fall A,\: |h(A)| < |\mcal P(\mcal P(A \times A))|\)
}
\pf{Proof}{
    As in the proof of \Cref{lem:hartogsNumberExists},
    \(\alpha \in h(A)\) if and only if there exists some \(R \subseteq A \times A\)
    such that \((\field R, R)\) is isomorphic to \(\alpha\).
    Define \(f \colon \mcal P(h(A)) \to \mcal P(\mcal P(A \times A))\) by
    \[
        X \mapsto \{\,R \subseteq A \times A \mid \exs \alpha \in X,\: (\field R, R) \cong \alpha\,\}.
    \]
    Then, \(f\) is injective; thus
    \(|h(A)| < |\mcal P(h(A))| \le |\mcal P(\mcal P(A \times A))|\)
    by \nameref{th:cantor}.
}

\exer[7.1.6]{}{
    Let \(h^\ast(A)\) be the least ordinal \(\alpha\)
    such that there does not exist \(f \colon A \surjto \alpha\).
    \begin{enumerate}[nolistsep, label=(\roman*), ref=\protect{(\roman*)}, listparindent=\parindent]
        \ii \(h^\ast(A)\) exists for all \(A\).
        \ii \(\fall \alpha \in \Ord,\: (\alpha \ge h^\ast(A) \implies \nexists f \colon A \surjto \alpha)\).
        \ii \(h^\ast(A)\) is an initial ordinal.
        \ii \(h(A) \le h^\ast(A)\).
        \ii If \(A\) is well-orderable, then \(h(A) = h^\ast(A)\).
    \end{enumerate}
}
\mclm{Proof}{\hfill
\begin{enumerate}[nolistsep, label=(\roman*), leftmargin=*, listparindent=\parindent]
    \ii
    Let \(\opname{Pt}(A)\) be the set of all \hyperref[dfn:partition]{partitions} of \(A\). Let
    \[
        H^\ast \triangleq \{\,\alpha \in \Ord \mid
        \exs S \in \opname{Pt}(A),\: \exs R \subseteq S \times S,\:
        (S, R) \cong \alpha\,\}.
    \]
    \(H^\ast\) exists by \nameref{th:counting} and \nameref{ax:replacement}.
    \clm[ljtFsxqK]{
        \(\fall \alpha \in \Ord,\: (\exs f \colon A \surjto \alpha \iff \alpha \in H^\ast)\)
    }{
    \begin{itemize}[nolistsep, wide=0pt, widest={(\(\Rightarrow\))}, leftmargin=*, listparindent=\parindent]
        \ii[(\(\Rightarrow\))]
        Let \(f \colon A \surjto \alpha\).
        Then, \(S \triangleq \{\,f\inv[\{\beta\}] \mid \beta < \alpha\,\}\)
        is a partition of \(A\), and
        the relation \(R\) on \(S\) defined by \(R = \{\,(f\inv[\{\beta\}], f\inv[\{\gamma\}]) \mid
        \beta < \gamma < \alpha\,\}\) is isomorphic to \(\alpha\) since \(\ran f = \alpha\).
        Hence, \(\alpha \in H^\ast\).

        \ii[(\(\Leftarrow\))]
        There exist \(S \in \opname{Pt}(A)\) and \(R \subseteq S \times S\)
        such that \((S, R) \cong \alpha\).
        Let \(g \colon S \bijto \alpha\) be the isomorphism between \((S, R)\) and \(\alpha\).
        Define \(f \colon A \to \alpha\) by
        \(a \mapsto g(C)\) where \(C \in S\) is the unique element of \(S\)
        such that \(a \in C\).
        Then, \(\ran f = \alpha\) as each \(C \in S\) is nonempty.
        \qed
    \end{itemize}
    }

    \Cref{clm:ljtFsxqK} says that \(H^\ast\) is transitive.
    Moreover, by \ref{itm:basicOrdinal.iv},
    \(H^\ast\) is well-ordered by \(\in\).
    Hence, \(H^\ast \in \Ord\).

    By \Cref{lem:noOrdinalContainsItself}, \(H^\ast \notin H^\ast\);
    thus \(\nexs f \colon A \surjto H^\ast\) by \Cref{clm:ljtFsxqK}.
    If \(\alpha \in H^\ast\), then \(\exs f \colon A \surjto H^\ast\)
    by \Cref{clm:ljtFsxqK}. Therefore, \(h^\ast(A) = H^\ast\).

    \ii
    Take any \(\alpha \in \Ord\).
    Assume \(\exs f \colon A \surjto \alpha\).
    Then, by \Cref{clm:ljtFsxqK}, \(\alpha < h^\ast(A)\).

    \ii
    Suppose \(\exs \beta < h^\ast(A),\: |\beta| = |h^\ast(A)|\) for the sake of contradiction.
    Then, there exists \(f \colon A \surjto \beta\).
    Let \(g \colon \beta \bijto h^\ast(A)\).
    Then, \(g \circ f \colon A \surjto h^\ast(A)\), which is a contradiction.

    \ii
    Take any \(\alpha < h(A)\).
    Then, \(\exs g \colon \alpha \injto A\).
    Now, define \(f \colon A \to \alpha\) by
    \[
        f(a) \triangleq \begin{cases}
            g\inv(a) & \text{if}~a \in \ran g \\
            0 & \text{otherwise.}
        \end{cases}
    \]
    Then, \(f\) is onto \(\alpha\); thus \(\alpha < h^\ast(A)\) by \Cref{clm:ljtFsxqK}.
    Therefore, \(h(A) \le h^\ast(A)\).

    \ii
    Assume \((A, \preceq)\) is a well-ordered set.
    Take any \(\alpha < h^\ast(A)\).
    Then, there exists \(f \colon A \surjto \alpha\).
    Define \(g \colon \alpha \to A\) by
    \(g(\beta) \triangleq \min_{\preceq} f\inv[\{\beta\}]\).
    It is well defined since \(f\) is onto \(\alpha\).
    Then, \(g\) is injective; thus \(|\alpha| \le |A|\),
    which implies \(\alpha < h(A)\). \qed
\end{enumerate}
}

\end{document}
