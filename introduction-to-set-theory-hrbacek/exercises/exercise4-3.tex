\documentclass[../introduction_to_set_theory.tex]{subfiles}

\begin{document}

\subsection*{Selected Problems}

\exer[4.3.1]{}{
Let \(|A_1| = |A_2|\) and \(|B_1| = |B_2|\).
\begin{enumerate}[nolistsep, label=(\roman*), ref=\protect{\Cref{exer:4.3.1} (\roman*)}, listparindent=\parindent]
    \ii
    If \(A_1 \cap B_1 = \OO\) and \(A_2 \cap B_2 = \OO\),
    then \(|A_1 \cup B_1| = |A_2 \cup B_2|\).
    \ii
    \(|A_1 \times B_1| = |A_2 \times B_2|\)
    \ii
    \(|\Seq(A_1)| = |\Seq(A_2)|\)
\end{enumerate}
}
\mclm{Proof}{\hfill
\begin{enumerate}[nolistsep, label=(\roman*), leftmargin=*, listparindent=\parindent]
    \ii
    Let \(f_A \colon A_1 \hooktwoheadrightarrow A_2\) and
    \(f_B \colon B_1 \hooktwoheadrightarrow B_2\).
    Then, \(f_A \cup f_B \colon A_1 \cup B_1 \hooktwoheadrightarrow A_2 \cup B_2\).

    \ii
    Let \(f_A \colon A_1 \hooktwoheadrightarrow A_2\) and
    \(f_B \colon B_1 \hooktwoheadrightarrow B_2\).
    We may define \(g \colon A_1 \times B_1 \hooktwoheadrightarrow A_2 \times B_2\) by
    \((a, b) \mapsto (f_A(a), f_B(b))\).

    \ii
    Let \(f \colon A_1 \hooktwoheadrightarrow A_2\).
    We may define \(g \colon \Seq(A_1) \hooktwoheadrightarrow \Seq(A_2)\) by
    \[\lang\,a_0, \cdots, a_{n-1}\,\rang \mapsto \lang\,f(a_0), \cdots, f(a_{n-1})\,\rang.\]
    \qed
\end{enumerate}
}

\exer[4.3.2]{}{
    If \(A\) is finite and \(B\) is countably infinite, then \(A \cup B\) is countably infinite.
}
\pf{Proof}{
    Let \(f_A \colon A \hookrightarrow \NN\) and \(f_B \colon B \hookrightarrow \NN\).
    Then, we may define \(g \colon A \cup B \hookrightarrow \NN \times \NN\) by
    \[
        g(x) = \begin{cases}
            (f_A(x), 0) & \text{if } x \in A \\
            (f_B(x), 1) & \text{if } x \in B \setminus A
        \end{cases}
    \]
    Hence, \(|A \cup B| \le \aleph_0\) by \Cref{th:productOfCountable}.
    Moreover, \(\aleph_0 = |B| \le |A \cup B|\) by \Cref{exer:4.1.3}.
    The result follows from \nameref{th:cantorBernstein}.
}

\setexernumber{3}

\exer[4.3.4]{}{
    If \(A\) is finite and nonempty, then \(\Seq(A)\) is countably infinite.
}
\pf{Proof}{
    Let \(B \triangleq A \cup \NN\).
    Then, by \Cref{exer:4.3.2}, \(B\) is countably infinite and \(\Seq(B)\) is countably infinite by
    \Cref{th:seqCountable}.
    Hence, as \(\Seq(A) \subseteq \Seq(B)\), \(|\Seq(A)| \le \aleph_0\).

    Fix any \(a \in A\).
    Let \(s\) be the infinite sequence with \(\fall i \in \NN,\: s_i = a\).
    Then, we have \(f \colon \NN \hookrightarrow \Seq(A)\)
    defined by \(f(n) = \restr{s}{n}\); thus \(\aleph_0 \le |\Seq(A)|\).
    The result follows from \nameref{th:cantorBernstein}.
}

\exer[4.3.5]{}{
    Let \(A\) be countably infinite.
    The set \([A]^n = \{\,S \subseteq A \mid |S| = n\,\}\) is countably infinite
    for all \(n > 0\).
}
\pf{Proof}{
    It is enough to show that \([\NN]^n\) is countably infinite for all \(n > 0\).
    Evidently, \(i \mapsto \{i\}\) is an injective mapping on \(\NN\) onto \([\NN]^{1}\).
    Hence, \(|[\NN]^1| = \aleph_0\).

    For the sake of induction, fix \(n > 0\) and assume \(|[\NN]^{n}| = \aleph_0\).
    We may define \(f \colon [\NN]^n \hookrightarrow [\NN]^{n+1}\) by
    \[
        f(x) \triangleq x \cup
        \big\{\,\max \{\,i \in \NN \mid i \in x \} + 1 \,\big\}.
    \]
    Hence, \(\aleph_0 \le |[\NN]^{n+1}|\).

    Now, since \(|[\NN]^n| = |\NN^n| = \aleph_0\) by \Cref{cor:productOfFiniteCountable},
    there exists an injection \(g \colon [\NN]^n \hookrightarrow \NN^n\).
    We define \(h \colon [\NN]^{n+1} \hookrightarrow \NN^{n+1}\) by
    \begin{align*}
        h(x) &\triangleq g(x \setminus \{i\}) \cup \{(n, i)\} \\
             &\quad\text{where } i = \max x.
    \end{align*}
    Hence, \(|[\NN]^{n+1}| \le |\NN^{n+1}| \le \aleph_0\).
    \Cref{exer:3.2.11} assures that \(\fall n > 0,\: |[\NN]^n| = \aleph_0\).
}

\exer[4.3.6]{}{
    A sequence \(\lang\,s_n\,\rang_{n=0}^\infty\) of natural numbers is
    \textit{eventually constant} if \[\exs n_0, s \in \NN,\: \fall n \ge n_0,\: s_n = s.\]
    The set of eventually constant sequences of natural numbers is countable.
}
\pf{Proof}{
    Let \(P\) be the set of eventually constant sequences of natural numbers.
    As \(P \subseteq Q\) where \(Q\) is the set of eventually periodic sequences of natural numbers
    (see \Cref{exer:4.3.7}),
    we have \(|P| \le \aleph_0\) by \Cref{exer:4.1.3,exer:4.3.7}.

    Moreover, we may define an injective infinite sequence into \(P\) recursively by
    \begin{alignat*}{2}
        && g_0 &= \lang\,0, 0, \cdots\,\rang \\
        \fall n \in \NN, &\quad& g_{n+1} &= \lang\,n + 1, a_0, a_1, \cdots\,\rang \\
                         &&&\qquad \text{where }g_n = \lang\,a_0, a_1, \cdots\,\rang.
    \end{alignat*}
    Hence, \(\aleph_0 \le |P|\).
    The result follows by \nameref{th:cantorBernstein}.
}

\exer[4.3.7]{}{
    A sequence \(\lang\,s_n\,\rang_{n=0}^\infty\) of natural numbers is
    \textit{eventually periodic} if \[\exs n_0 \in \NN,\: \exs p > 0,\: \fall n \ge n_0,\: s_{n+p} = s_n.\]
    The set of eventually periodic sequences of natural numbers is countably infinite.
}
\pf{Proof}{
    Let \(Q\) be the set of eventually periodic sequences of natural numbers.
    We may define \(f \colon Q \to \Seq(\NN) \times \NN\) by
    \begin{align*}\
        f(x) &\triangleq \left(\restr{x}{n^\ast + p^\ast}, p^\ast\right) \\
             &\qquad\begin{aligned}[t]
                 \text{where } n^\ast &=
                 \min \{\,n_0 \in \NN \mid \exs p > 0,\: \fall n \ge n_0,\: s_{n+p} = s_n\,\}\\
                 \text{and }p^\ast &=
                 \min \{\,p > 0 \mid \fall n \ge n^\ast,\: s_{n+p} = s_n\,\}.
             \end{aligned}
    \end{align*}
    Then, it can be easily shown that \(f\) is injective.
    Hence, \(|Q| \le |\Seq(\NN) \times \NN| = \aleph_0\) by
    \Cref{th:productOfCountable,th:seqCountable}.

    Moreover, as \(P \subseteq Q\) where \(P\) is
    the set of eventually constant sequences of natural numbers and \(\aleph_0 \le |P|\)
    by \Cref{exer:4.3.6}, we have \(|Q| = \aleph_0\) by
    \Cref{exer:4.1.3} and \nameref{th:cantorBernstein}.
}

\setexernumber{9}

\exer[4.3.10]{}{
    Let \((S, \preceq)\) be a linearly ordered set and let \(\lang\,A_n \mid n \in \NN\,\rang\)
    be an infinite sequence of finite subsets of \(S\).
    Then, \(\bigcup_{n = 0}^{\infty} A_n\) is countable.
}
\pf{Proof}{
    \WLOG, \(A_n \neq \OO\) for each \(n \in \NN\).
    \clm[FMsBqJPY]{
        For each finite \(A \subseteq S\),
        there uniquely exists a unique isomorphism between
        \(\left(|A|, \mathord{\le} \cap |A|^2\right)\)
        and \((A, \mathord{\preceq} \cap A^2)\).
    }{
        We have existence for each \(A\) by \Cref{th:uniqueFiniteTotalOrder}.
        Hence, we only prove the uniqueness by induction.
        If \(|A| = 0\), we have only one isomorphism \(\OO\).

        Fix some \(n \in \NN\) and assume the proposition holds for all \(A\) with cardinality \(n\).
        Take any \(A \subseteq S\) with \(|A| = n + 1\).
        Let \(f\) and \(g\) be two isomorphisms between
        \((n+1, \mathord{\le} \cap (n + 1)^2)\)
        and \((A, \mathord{\le} \cap A^2)\).
        Then, \(f(n) = g(n)\) since the greatest element is unique.
        Let \(B = A \setminus \{f(n)\}\).
        Then, \(\restr{f}{n}\) and \(\restr{g}{n}\) are isomorphisms between
        \((n, \mathord{\le} \cap n^2)\)
        and \((B, \mathord{\le} \cap B^2)\).
        Hence, \(\restr{f}{n} = \restr{g}{n}\), and thus \(f = g\).
        The result follows from \nameref{th:induction}.
    }

    \Cref{clm:FMsBqJPY} enables us to guarantee the existence of infinite sequence
    \(\lang\,a_n \mid n \in \NN\,\rang\) such that, for each \(n \in \NN\):
    \begin{enumerate}[nolistsep, label=(\roman*), ref=\protect{(\roman*)}, listparindent=\parindent]
        \ii
        \(\restr{a_n}{|A_n|}\) is the isomorphism between
        \(\left(|A_n|, \restr{\mathord{\le}}{|A_n|^2}\right)\)
        and \(\big(A_n, \restr{\mathord{\preceq}}{A_n^2}\big)\).
        \ii
        \(\fall k \ge |A_n|,\: a_n(k) = a_n(0)\).
    \end{enumerate}
    Hence, \(\ran a_n = A_n\) for each \(n \in \NN\), and thus
    \(\bigcup_{n=0}^{\infty} A_n\) is countable by \Cref{th:unionOfCountable}.
}

\exer[4.3.11]{}{
    Any partition of a countable set has a set of representatives.
}
\pf{Proof}{
    Let \(A\) be countable and \(S\) be a partition of \(A\).
    There exists \(f \colon A \hookrightarrow \NN\).
    Then,
    \[
        X \triangleq \{\,f\inv(\min f[C]) \mid C \in S\,\}
    \]
    is a set of representatives.
}

\end{document}
