\documentclass[../introduction_to_set_theory.tex]{subfiles}

\begin{document}

\subsection*{Selected Problems}

\setexernumber{1}

\exer[4.1.2]{}{
    Let \(A\), \(B\), and \(C\) be sets.
    \begin{enumerate}[nolistsep, label=(\roman*), leftmargin=*, listparindent=\parindent]
        \ii If \(|A| < |B|\) and \(|B| \le |C|\), then \(|A| < |C|\).
        \ii If \(|A| \le |B|\) and \(|B| < |C|\), then \(|A| < |C|\).
    \end{enumerate}
}
\mclm{Proof}{\hfill
\begin{enumerate}[nolistsep, label=(\roman*), leftmargin=*, listparindent=\parindent]
    \ii
    We already have \(|A| \le |C|\) by \ref{itm:basicLessCardiii}.
    Let \(g \colon B \hookrightarrow C\).
    Suppose \(f \colon A \hooktwoheadrightarrow C\) for the sake of contradiction.
    Then, \(f\inv \circ g \colon B \hookrightarrow A\), i.e., \(|B| \le |A|\).
    By \nameref{th:cantorBernstein}, we get \(|A| = |B|\), which is a contradiction.

    \ii
    We already have \(|A| \le |C|\) by \ref{itm:basicLessCardiii}.
    Let \(g \colon A \hookrightarrow B\).
    Suppose \(f \colon A \hooktwoheadrightarrow C\) for the sake of contradiction.
    Then, \(g \circ f\inv \colon C \hookrightarrow B\), i.e., \(|C| \le |B|\).
    By \nameref{th:cantorBernstein}, we get \(|B| = |C|\), which is a contradiction.

    \qed
\end{enumerate}
}

\exer[4.1.3]{}{
    If \(A \subseteq B\), then \(|A| \le |B|\).
}
\pf{Proof}{
    \(\mrm{Id}_A\) is an injective function on \(A\) into \(B\).
}

\setexernumber{6}

\exer[4.1.7]{}{
    If \(S \subseteq T\), then \(|A^S| \le |A^T|\). In particular, \(|A^m| \le |A^n|\) if \(m \le n\).
}
\pf{Proof}{
    If \(T = \OO\), then \(A^S = A^T = \{\OO\}\) and it is done.

    Assume \(T \neq \OO\).
    Fix some \(t \in T\).
    Now, define \(f \colon A^S \hookrightarrow A^T\) by
    \(g \mapsto g \cup \{\,(x, t) \mid x \in T \setminus S\,\}\).
}

\setexernumber{9}

\exer[4.1.10]{}{
    Let \(F \colon \mcal{P}(A) \to \mcal{P}(A)\) be \textit{monotone},
    i.e., if \(X \subseteq Y \subseteq A\), then \(F(X) \subseteq F(Y)\).
    Then, \(F\) has a least \textit{fixed point} \(\ol{X}\),
    that is to say \(F(\ol{X}) = \ol{X}\) and \(\fall X \subseteq A,\: (F(X) = X \implies \ol{X} \subseteq X)\).
}
\pf{Proof}{
    Let \(T \triangleq \{\,X \subseteq A \mid F(X) \subseteq X\,\}\).
    Then, as \(A \in T\), \(T \neq \OO\); we may let \(\ol{X} \triangleq \bigcap T\).

    Then, for all \(X \in T\), \(\ol{X} \subseteq X\); and thus \(F(\ol{X}) \subseteq F(X) \subseteq X\).
    We have \(F(\ol{X}) \subseteq \bigcap T = \ol{X}\), i.e., \(\ol{X} \in T\).

    On the other hand, we have \(F(F(\ol{X})) \subseteq F(\ol{X})\), or \(F(\ol{X}) \in T\),
    and thus \(\ol{X} = \bigcap T \subseteq F(\ol{X})\). Therefore, \(F(\ol{X}) = \ol{X}\).
    Moreover, if \(X\) is a fixed point, then \(X \in T\), and thus \(\ol{X} = \bigcap T \subseteq X\).
}

\setexernumber{13}

\exer[4.1.14]{}{
    A function \(F \colon \mcal P(A) \to \mcal P(A)\) is \textit{continuous}
    if, for each sequence \(\lang\,X_i \mid i \in \NN\,\rang\) of subsets of \(A\)
    such that \(\fall i, j \in \NN,\: (i \le j \implies X_i \subseteq X_j)\),
    \(F \big(\bigcup_{i \in \NN} X_i\big) = \bigcup_{i \in \NN} F(X_i)\) holds.

    If \(\ol{X}\) is the least fixed point of a monotone continuous function,
    \(F \colon \mcal P(A) \to \mcal P(A)\),
    then \(\ol{X} = \bigcup_{i \in \NN} X_i\)
    where we define recursively \(X_0 = \OO\), \(\fall i \in \NN,\: X_{i+1} = F(X_i)\).
}
\pf{Proof}{
    Let \(\tilde X \triangleq \bigcup_{i \in \NN} X_i\).
    We have \(X_0 = \OO \subseteq X_1\).

    If \(X_n \subseteq X_{n+1}\), then \(X_{n+1} \subseteq X_{n+2}\) since \(F\) is monotone.
    Hence, \(\fall n \in \NN,\: X_n \subseteq X_{n+1}\).
    Therefore, similarly to \Cref{exer:3.3.1},
    we have \(X_m \subseteq X_n\) whenever \(m \le n\).
    Hence, \(F(\tilde X) = \bigcup_{i \in \NN} F(X_i) = \bigcup_{i=1}^\infty X_{i} = \tilde X\);
    \(\tilde X\) is a fixed point of \(F\); hence \(\ol X \subseteq \tilde X\).

    We have \(X_0 \subseteq \ol X\).
    If \(X_n \subseteq \ol{X}\) for \(n \in \NN\), then
    \(X_{n+1} \subseteq F(\ol X)) = \ol X\).
    Hence, by \nameref{th:induction}, \(\tilde X \subseteq \ol X\).
}

\end{document}
