\documentclass[../introduction_to_set_theory_Note.tex]{subfiles}

\begin{document}

\subsection*{Selected Problems}

\exer[6.1.1]{}{
    Give an example of a totally ordered set \((L, \le)\)
    and an initial segment \(S\) of \(L\) which is not of the form \(\{\,x \in L \mid x < a\,\}\)
    for all \(a \in L\).
}
\pf{Proof}{
    If \((L, \le)\) is dense and \(\ell \in L\),
    then \(S \triangleq \{\,x \in L \mid x \le \ell\,\}\)
    is never equal to \(\{\,x \in L \mid x < a\,\}\) where \(a \in L\).
    For if they are equal, there exists \(x \in L\) such that \(\ell < x < a\)
    and such \(x\) is not in \(S\).
    One example would be \((\QQ, \le)\) and \(S = \{\,x \in \QQ \mid x \le 0\,\}\).
}

\exer[6.1.2]{}{
    \(\omega + 1\) is not isomorphic to \(\omega\) (in the well-ordering by \(\in\)).
}
\pf{Proof}{
    \(\omega\) is an initial segment of \(\omega + 1\),
    and thus they are not isomorphic by \ref{itm:noIsoToInitSeg.i}.
}

\exer[6.1.3]{}{
    There exist \(2^{\aleph_0}\) well-orderings of \(\NN\).
}
\pf{Proof}{
    Let \(S\) be the set of all well-orderings of \(\NN\).
    The cardinality of the set of all relations on \(\NN^2\)
    is \(|\mcal P(\NN^2)| = |\mcal P(\NN)| = 2^{\aleph_0}\)
    by \Cref{th:productOfCountable,th:cardPowSet}.
    Hence, \(|S| \le 2^{\aleph_0}\).

    Let \(T \triangleq \{\,f \in \NN[\NN] \mid f \colon \NN \hooktwoheadrightarrow \NN\,\}\).
    Define \(F \colon T \to \mcal P(\NN^2)\) by
    \(f \mapsto \{\,(f(m), f(n)) \mid m, n \in \NN \land m \le n \,\}\).
    Then, for each \(f \in T\),
    \((\NN, \le)\) and \((\NN, F(f))\) is an isomorphism,
    thus \(F\) is into \(S\).
    For each \(R \in S\), there exists one and only \(f \in T\) defined by
    \(f_n = \min_R \left(\NN \setminus \restr{f}{n}\right)\) for all \(n \in \NN\).
    Hence, \(|T| = |S|\).

    Now, define \(\sigma \colon \mcal P(\NN) \to T\) by
    \[
        \sigma_X(2 \cdot n) = \begin{cases}
            2 \cdot n & \text{if } n \notin X \\
            2 \cdot n + 1 & \text{if } n \in X
        \end{cases} \quad\text{and}\quad
        \sigma_X(2 \cdot n + 1) = \begin{cases}
            2 \cdot n + 1 & \text{if } n \notin X \\
            2 \cdot n & \text{if } n \in X
        \end{cases}
    \]
    for each \(n \in \NN\).
    It is evident that \(\sigma\) is injective; hence
    \(|T| \ge |\mcal P(\NN)| = 2^{\aleph_0}\).
    By \nameref{th:cantorBernstein}, \(|S| = |T| = 2^{\aleph_0}\).
}

\exer[6.1.4]{}{
    For every infinite subset \(A\) of \(\NN\),
    \((A, \mathord{\le} \cap A^2)\) is isomorphic to \((\NN, \le)\).
}
\pf{Proof}{
    Clearly, \((A, \mathord{\le} \cap A^2)\) is well-ordered.
    Noting that every initial segment of \(A\) and \(\NN\) is finite,
    \Cref{th:wosetComparable} leaves only one option---%
    \(A\) and \(\NN\) are isomorphic.
}

\end{document}
