\documentclass[../introduction_to_set_theory.tex]{subfiles}

\begin{document}

\subsection*{Selected Problems}

\exer[4.4.1]{}{
    Assume that \((A_1, \le_1)\) is similar to \((B_1, \preceq_1)\) and
    \((A_2, \le_2)\) is similar to \((B_2, \preceq_2)\).
    \begin{enumerate}[nolistsep, label=(\roman*), ref=\protect{(\roman*)}]
        \ii
        Assuming \(A_1 \cap A_2 = B_1 \cap B_2 = \OO\),
        the sum of \((A_1, \le_1)\) and \((A_2, \le_2)\) is similar to the sum of
        \((B_1, \preceq_1)\) and \((B_2, \preceq_2)\).
        (See \Cref{lem:orderOnUnion}.)

        \ii
        The lexicographic product of \((A_1, \le_1)\) and \((A_2, \le_2)\) is similar to
        the lexicographic product of \((B_1, \preceq_1)\) and \((B_2, \preceq_2)\).
    \end{enumerate}
}
\mclm{Proof}{
Let \(f_1\) be an isomorphism between \((A_1, \le_1)\) and \((B_1, \preceq_1)\),
and let \(f_2\) be an isomorphism between \((A_2, \le_2)\) and \((B_2, \preceq_2)\).
\begin{enumerate}[nolistsep, label=(\roman*), leftmargin=*, listparindent=\parindent]
    \ii
    \(g \triangleq f_1 \cup f_2\) is an isomorphism
    between `the sum of \((A_1, \le_1)\) and \((A_2, \le_2)\)' and
    `the sum of \((B_1, \preceq_1)\) and \((B_2, \preceq_2)\)'.

    \ii
    \(g \colon A_1 \times A_2 \hooktwoheadrightarrow B_1 \times B_2\)
    defined by \((a_1, a_2) \mapsto (f_1(a_1), f_2(a_2))\)
    is an isomorphism between
    `the lexicographic product of \((A_1, \le_1)\) and \((A_2, \le_2)\)' and
    `the lexicographic product of \((B_1, \preceq_1)\) and \((B_2, \preceq_2)\)'.
    \qed
\end{enumerate}
}

\exer[4.4.2]{}{
    Give an example of linear orderings \((A_1, \le_1)\) and \((A_2, \le_2)\) such that
    the sum of \((A_1, \le_1)\) and \((A_2, \le_2)\)
    does not have the same order type as
    the sum of \((A_2, \le_2)\) and \((A_1, \le_1)\).
    Do the same for lexicographic product.
}
\mclm{Proof}{
    Let \(A = \{\,a, b\,\}\) where \(a \neq b\) and \(a, b \notin \NN\)
    with \(\mathord{\preceq} = \{\,(a, a), (a, b), (b, b)\,\}\).

    \begin{itemize}[nolistsep, listparindent=\parindent]
        \ii
        The sum of \((\NN, \le)\) and \((A, \preceq)\)
        has a greatest element \(b\) but the sum of \((A, \preceq)\) and \((\NN, \le)\)
        does not have a greatest element.
        Hence, they are not similar.

        \ii
        The lexicographic product of \((\NN, \le)\) and \((A, \preceq)\)
        is isomorphic to \((\NN, \le)\) via the isomorphism
        \((n, x) \mapsto \begin{cases}
            2 \cdot n & \text{if } x = a \\
            2 \cdot n + 1 & \text{if } x = b.
        \end{cases}\)
        However, \(\{a\} \times \NN\), a subset of \(A \times \NN\),
        is bounded above by \((b, 0)\) but it does not have a greatest element
        in the lexicographic product of \((A, \preceq)\) and \((\NN, \le)\).
        Hence, they are not similar.
        \qed
    \end{itemize}
}

\exer[4.4.3]{}{
    The sum and the lexicographic product of two well-orderings are well-orderings.
}
\mclm{Proof}{
    Let \((A_1, \le_1)\) and \((A_2, \le_2)\) be well-orderings.
    \begin{itemize}[nolistsep, listparindent=\parindent]
        \ii
        Assume \(A \cap B = \OO\).
        Let \((B, \le)\) be the sum of \((A_1, \le_1)\) and \((A_2, \le_2)\).
        Take any \(C \subseteq B\) with \(C \neq \OO\).
        If \(C \cap A_1 = \OO\), then \(\min_{\le} C = \min_{\le_2} C\).
        Otherwise, \(\min_{\le} C = \min_{\le_1} (C \cap A_1)\).

        \ii
        Let \((A_1 \times A_2, \le)\) be the lexicographic product of \((A_1, \le_1)\) and \((A_2, \le_2)\).
        Take any \(R \subseteq A_1 \times A_2\) with \(R \neq \OO\).
        Let \(a_1 \triangleq \min_{\le_1} \dom R\)
        and \(a_2 \triangleq \min_{\le_2} R[\{a_1\}]\).
        Then, \((a_1, a_2) = \min_{\le} R\).
        \qed
    \end{itemize}
}

\exer[4.4.4]{}{
    If \(\lang\,A_i \mid i \in \NN\,\rang\) is an infinite sequence of
    totally ordered sets of natural numbers and \(|A_i| \ge 2\) for all \(i \in \NN\),
    then the lexicographic ordering of \(\prod_{i \in \NN} A_i\) is \textit{not} a well-ordering.
}
\pf{Proof}{
    For each \(i \in \NN\), justified by \(|A_i| \ge 2\), let \(a_i \triangleq \min A_i\) and
    \(b_i \triangleq \min (A_i \setminus \{a_i\})\).
    Define \(X \subseteq \prod_{i \in \NN} A_i\) by
    \[
        X \triangleq \textstyle\big\{
            \,f \in \prod_{i \in \NN} A_i \:\big|\:
                \exs i \in \NN,\: [f(i) = b_i \land
            \fall j \in \NN,\: (i \neq j \implies f(j) = a_j)]\,\big\}.
    \]
    Then, \(X\), being similar to \((\NN, \le\inv)\),
    does not have a least element.
}

\exer[4.4.5]{}{
    Let \(\lang\,(A_i, \le_i) \mid i \in I\,\rang\) be an indexed system of mutually
    disjoint totally ordered sets where \(I \subseteq \NN\).
    The relation \(\prec\) on \(\bigcup_{i \in I} A_i\) defined by
    \[
        a \prec b \iff (\exs i \in I,\: a <_i b) \lor
        (\exs i, j \in I,\: i < j \land a \in A_i \land b \in A_j)
    \]
    is a strict total ordering.
    Moreover, if all \(\le_i\) are well-orderings, so is \(\preceq\).
}
\mclm{Proof}{\hfill
\begin{itemize}[nolistsep, leftmargin=*, listparindent=\parindent]
    \ii
    Assume \(a \prec b\) and \(b \prec c\).
    There exist \(i, j, k \in I\) such that
    \(a \in A_i\), \(b \in A_j\) and \(c \in A_k\).
    Then, we have \(i \le j \le k\).
    If \(i < j\) or \(j < k\), we immediately have \(a \prec c\)
    by definition.
    If \(i = j = k\), by transitivity of \(<_i\),
    we have \(a \prec c\).
    Hence, \(\prec\) is transitive in \(\bigcup_{i \in I} A_i\). \checkmark

    \ii
    Suppose \(a \prec b\) and \(b \prec a\) for the sake of contradiction.
    There exists \(i, j \in I\) such that \(a \in A_i\) and \(b \in A_j\).
    We cannot have \(i < j\) or \(j < i\) as it contradicts
    one of \(a \prec b\) and \(b \prec a\).
    Hence, \(i = j\) by totality of \((\NN, \le)\).
    Then, we have \(a <_i b\) and \(b <_i a\), which is impossible
    by asymmetry of \(<_i\). Hence, \(\prec\) is asymmetric in \(\bigcup_{i \in I} A_i\). \checkmark

    \ii
    Take any \(a, b \in \bigcup_{i \in I} A_i\).
    There exists \(i, j \in I\) such that \(a \in A_i\) and \(b \in A_j\).
    If \(i \neq j\), we immediately have \(a \prec b\) or \(b \prec a\).
    If \(i = j\), as \(\le_i\) is total, we have \(a \le_i b\) or \(b \le_i a\).
    Thus, \(a \preceq b\) or \(b \preceq a\).
    Hence, \(\preceq\) is total. \checkmark

    \ii
    Assume \(\le_i\) is a well-ordering for each \(i \in \NN\).
    Take any \(X \subseteq \bigcup_{i \in I} A_i\) with \(X \neq \OO\).
    Let \(i_0 \triangleq \min \{\,i \in I \mid A_i \cap X \neq \OO\,\}\),
    thanks to \nameref{th:NisWellOrdered} and \(X\) being nonempty.
    Then, let \(a \triangleq \min_{\le_{i_0}} (A_{i_0} \cap X)\),
    which exists as \(\le_{i_0}\) is a well-ordering.
    Then, \(a = \min_{\preceq} X\). \checkmark
    \qed
\end{itemize}
}

\setexernumber{6}

\exer[4.4.7]{}{
    Let \(\preceq\) be the lexicographical ordering of \(\NN[\NN]\)
    (where \(\NN\) is ordered in the usual way) and let
    \(P \subseteq \NN[\NN]\) be the set of
    all eventually periodic, but not eventually constant, sequences of natural numbers.
    (See \Cref{exer:4.3.6,exer:4.3.7} for the definitions.)
    Then, \((P, \mathord{\preceq} \cap P^2)\) is a countably infinite dense totally ordered set
    without endpoints.
}
\pf{Proof}{
    We already have \(\mathord{\preceq} \cap P^2\) is a total ordering.

    \clm[DBALsefv]{
        \(P\) is countably infinite.
    }{
        As \(P\) is a subset of the set of all eventually periodic sequences,
        by \Cref{exer:4.1.3,exer:4.3.7}, \(|P| \le \aleph_0\).
        We may define an injective infinite sequence into \(P\) recursively by
        \begin{alignat*}{2}
            && g_0 &= \lang\,0, 1, 0, 1, \cdots\,\rang \\
            \fall n \in \NN,&\quad& g_{n+1} &= \lang\,n+1, a_0, a_1, \cdots\,\rang \\
                            &&&\qquad\text{where }g_n = \lang\,a_0, a_1, \cdots\,\rang.
        \end{alignat*}
        Hence, \(\aleph_0 \le |P|\); thus \(|P| = \aleph_0\) by \nameref{th:cantorBernstein}.
        \qed
    }

    \clm[NDZYpOnP]{
        \((P, \mathord{\preceq} \cap P^2)\) is dense.
    }{
        Take any \(f, g \in P\) with \(f \prec g\).
        Then, \(f_{i_0} < g_{i_0}\) where
        \(i_0 = \min \{\,i \in \NN \mid f_i \neq g_i\,\}\).
        Define \(h \in \NN[\NN]\) by
        \[
            h_i \triangleq \begin{cases}
                f_i & \text{if } i \le i_0 \\
                f_i + 1 & \text{otherwise.}
            \end{cases}
        \]
        Then, it is evidently \(f \prec h \prec g\).

        There exist \(n_0 \in \NN\) and \(p > 0\) such that
        \(\fall n \ge n_0,\: f_{n+p} = f_n\).
        Let \(n_0' \triangleq \max \{\,n_0, i_0 + 1\,\}\).
        Then, for each \(n \ge n_0'\),
        we have \(h_{n+p} = f_{n+p} + 1 = f_n + 1 = h_n\);
        thus \(h \in P\).
        Hence, \((P, \mathord{\preceq} \cap P^2)\) is dense. \qed
    }

    \clm[uEiOTovb]{
        \((P, \mathord{\preceq} \cap P^2)\) has no endpoints.
    }{
        Fix any \(f \in P\).
        Then, we may define \(g \in \NN[\NN]\) by \(\fall n \in \NN,\: g_n = f_n + 1\).
        It is easy to show that \(f \prec g\) and \(g \in P\).
        Hence, \((P, \mathord{\preceq} \cap P^2)\) has no greatest element.

        We may also define \(h \in \NN[\NN]\) by
        \[
            \fall n \in \NN,\: h_n = \begin{cases}
                0 & \text{if } n = 0 \\
                f_{n-1} & \text{otherwise.}
            \end{cases}
        \]
        As \(f\) is not eventually constant, \(f \neq h\) and we may let
        \(i_0 \triangleq \min \{\,i \in \NN \mid f_i \neq 0\,\}\).
        As \(\fall i < i_0,\: f_i = 0\), \(\min \mrm{diff}(f, h) = i_0\).
        Since \(0 = h_{i_0} < f_{i_0}\), \(h \prec f\).
        Thus, \((P, \mathord{\preceq} \cap P^2)\) has no least element. \qed
    }
    \noindent
    Combining \Cref{clm:DBALsefv,clm:NDZYpOnP,clm:uEiOTovb} gives the desired result.
}

\setexernumber{10}

\exer[4.4.11]{}{
    Let \((A, \preceq)\) be a dense totally ordered set.
    Show that for all \(a, b \in A\) such that \(a \prec b\),
    the \textit{closed interval} \([a, b] \triangleq \{\,x \in A \mid a \preceq x \preceq b\,\}\)
    is infinite.
}
\pf{Proof}{
    Assume there are some \(a, b \in A\) such that \(a \prec b\) and \(I = [a, b]\) is finite.
    Let \(|I| = n\). (Since \(a, b \in I\), \(n \ge 2\).)
    Then, by \Cref{th:uniqueFiniteTotalOrder},
    there exists an isomorphism \(h \colon n \hooktwoheadrightarrow I\)
    between \((n, \mathord{\le} \cap n^2)\) and \((I, \mathord{\preceq} \cap I^2)\).
    Then, \(\exs i < n,\: h(0) \prec h(i) \prec h(1)\) as \((A, \preceq)\) is dense.
    However, there does not exist \(i \in \NN\) with \(0 < i < 1\) by \Cref{exer:3.1.1},
    which is a contradiction.
}

\exer[4.4.12]{}{
    Let \((P, \preceq)\) and \((Q, \le)\) be countably infinite dense totally ordered sets
    with both endpoints. Then, \((P, \preceq)\) and \((Q, \le)\) are similar.
}
\pf{Proof}{
    Define the following:
    \begin{alignat*}{2}
        p &\triangleq\textstyle \min_{\preceq} P & p' &\triangleq\textstyle \max_{\preceq} P \\
        q &\triangleq\textstyle \min_{\le} Q &\qquad q' &\triangleq\textstyle \max_{\le} Q \\
        P_0 &\triangleq P \setminus \{\,p, p'\,\} &\qquad Q_0 &\triangleq Q \setminus \{\,q, q'\,\}.
    \end{alignat*}
    Then, \((P_0, \mathord{\preceq} \cap P_0^2)\) and \((Q_0, \mathord{\preceq} \cap Q_0^2)\)
    are countably infinite dense totally ordered sets without endpoints.
    Hence, by \Cref{th:countDenseTotal}, we have an isomorphism
    \(h \colon P_0 \hooktwoheadrightarrow Q_0\) between them.
    Then, \(h' \cup \{\,(p, q), (p', q')\,\}\) is an isomorphism between
    \((P, \preceq)\) and \((Q, \le)\).
}

\end{document}
