\documentclass[../introduction_to_set_theory.tex]{subfiles}

\begin{document}

\subsection*{Selected Problems}

\setexernumber{1}

\exer[3.4.2]{}{
    \(\fall k, m, n \in \NN,\: (m < n \iff m + k < n + k)\)
}
\pf{Proof}{
    Let \(\mbf{P}(x)\) be the property ``\(\fall m, n \in \NN,\: (m < n \iff m + x < n + x)\).''
    \(\mbf{P}(0)\) is evident from \eqref{eq:addition1}.

    Now, fix any \(k \in \NN\) and assume \(\mbf{P}(k)\).
    Then, for all \(m, n \in \NN\),
    \begin{alignat*}{2}
        m < n &\iff m + k < n + k &\qquad& \comment*{\(\mbf{P}(k)\)}\\
              &\iff (m + k) + 1 < (n + k) + 1 && \comment*{\Cref{exer:3.2.2}}\\
              &\iff m + (k + 1) < n + (k + 1). && \comment*{\nameref{th:addIsAssociative}}
    \end{alignat*}
    By \nameref{th:induction}, the result follows.
}

\exer[3.4.3]{}{
    \(\fall m, n \in \NN,\: (m \le n \iff \exs! k \in \NN,\: n = m + k)\)
}
\pf{Proof}{
    (\(\Rightarrow\))
    Fix any \(m \in \NN\) and let \(\mbf{P}(x)\) be the property ``\(\exs k \in \NN,\: x = m + k\).''
    \(\mbf{P}(m)\) holds since \(k = 0\) would satisfy by \eqref{eq:addition1}.

    Fix any \(n \in \NN\) such that \(m \le n\) and assume \(\mbf{P}(n)\).
    Then, there exists \(k\) such that \(n = m + k\),
    which leads to \(n + 1 = m + (k + 1)\) by \nameref{th:addIsAssociative}.
    Hence, \(\mbf{P}(n+1)\) holds.
    Therefore, \(\fall n \ge m,\: \exs k \in \NN,\: n = m + k\) by \Cref{exer:3.2.11}.

    To prove the uniqueness, assume \(m + k = m + \ell\) where \(k, \ell, m \in \NN\).
    \WLOG, \(k \le \ell\). If it were \(k < \ell\), by \Cref{exer:3.4.2} and \nameref{th:addIsCommutative},
    we must have \(m + k = k + m < \ell + m = \ell + m\).
    Hence, \(k = \ell\).

    (\(\Leftarrow\))
    Let \(\mbf{P}(x)\) be the property ``\(\fall m, n \in \NN,\: (n = m + x \implies m \le n)\).''
    We have evidently \(\mbf{P}(0)\) by \eqref{eq:addition1}.

    Fix any \(k \in \NN\) and assume \(\mbf{P}(k)\).
    Then, for each \(m, n \in \NN\) such that \(n = m + (k + 1)\), we have
    \(n = (m + 1) + k\) thanks to \nameref{th:addIsCommutative} and \nameref{th:addIsAssociative},
    and thus \(m < m + 1 \le n\) by \(\mbf{P}(k)\).
    Hence, by \nameref{th:induction}, the result follows.
}

\setexernumber{5}

\exer[3.4.6]{}{
    \(\fall k, m, n \in \NN,\: [k \neq 0 \implies (m < n \iff m \cdot k < n \cdot k)]\)
}
\pf{Proof}{
    Let \(\mbf{P}(x)\) be the property ``\(\fall m, n \in \NN,\: (m < n \iff m \cdot k < n \cdot k)\).''
    \(\mbf{P}(1)\) holds since, for all \(n \in \NN\),
    \begin{alignat*}{2}
        n \cdot 1 &= n \cdot (0 + 1) &\qquad& \comment*{\eqref{eq:addition1}, \nameref{th:addIsCommutative}}\\
                  &= n \cdot 0 + n && \comment*{\eqref{eq:multiplication2}}\\
                  &= 0 + n && \comment*{\eqref{eq:multiplication1}}\\
                  &= n. && \comment*{\eqref{eq:addition1}, \nameref{th:addIsCommutative}}
    \end{alignat*}

    Now, fix any \(k \in \NN\) and assume \(\mbf{P}(k)\).
    Then, for each \(m, n \in \NN\) with \(m < n\),
    \begin{alignat*}{2}
        m \cdot (k + 1) &= m \cdot k + m &\qquad& \comment*{\eqref{eq:multiplication2}}\\
                        &< m \cdot k + n && \comment*{\Cref{exer:3.4.2}}\\
                        &< n \cdot k + n && \comment*{\(\mbf{P}(k)\), \nameref{th:addIsCommutative}, \Cref{exer:3.4.2}}\\
                        &= n \cdot (k + 1). && \comment*{\eqref{eq:multiplication2}}
    \end{alignat*}
    Therefore, by \Cref{exer:3.2.11}, the result follows.
}

\end{document}
