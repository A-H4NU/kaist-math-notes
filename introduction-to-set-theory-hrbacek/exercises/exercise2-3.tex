\documentclass[../introduction_to_set_theory.tex]{subfiles}

\begin{document}

\subsection*{Selected Problems}

\setexernumber{3}
\exer[2.3.4]{}{
    Let \(f\) be a function.
    If there exists a function \(g\) such that \(g \circ f = \mrm{Id}_{\dom f}\),
    then \(f\) is invertible and \(f\inv = \restr{g}{\ran f}\).
}
\pf{Proof}{
    For \(x_1, x_2 \in \dom f\), suppose \(f(x_1) = f(x_2)\).
    Then, \(x_1 = (g \circ f)(x_1) = g(f(x_1)) = g(f(x_2)) = (g \circ f)(x_2) = x_2\).
    Hence, \(f\) is one-to-one and is inverible by \Cref{th:invIffInj}.

    Take any \((y, x) \in f\inv\). Then, as \(x \in \dom f\),
    we must have \((y, x) \in \mrm{Id}_{\dom f}\). Hence, \(f\inv \subseteq \restr{g}{\ran f}\).
    Now, take any \((y, x) \in \restr{g}{\ran f}\). Since \(y \in \ran f\),
    there exists \(x' \in \dom f\) such that \((x', y) \in f\).
    Since \(g \circ f = \mrm{Id}_{\dom f}\),
    we have \(x = x'\). Therefore, \((y, x) \in f\inv\); \(\restr{g}{\ran f} \subseteq f\inv\).
}

\setexernumber{5}

\exer[2.3.6]{}{
    Let \(f\) be a function.
    \begin{enumerate}[nolistsep, label=(\roman*), ref=\protect{\Cref{exer:2.3.6} (\roman*)}]
        \ii \(f\inv[A \cap B] = f\inv[A] \cap f\inv[B]\)
        \ii\label{itm:2.3.6.ii} \(f\inv[A \setminus B] = f\inv[A] \setminus f\inv[B]\)
    \end{enumerate}
}
\mclm{Proof}{
Thanks to \Cref{exer:2.2.3} (ii) and (iii), we only need to prove the other inclusions.
\begin{enumerate}[nolistsep, label=(\roman*)]
    \ii Take any \(x \in f\inv[A] \cap f\inv[B]\).
        Then, there exists \(a \in A\) and \(b \in B\)
        such that \(xfa\) and \(xfb\).
        Since \(f\) is a function, \(a = b\),
        and thus \(x \in f\inv[A \cap B]\).

    \ii Take any \(x \in f\inv[A \setminus B]\).
        Then, \(f(x) \in A \setminus B\).
        If \(x \in f\inv[B]\), we would have \(f(x) \in B\);
        thus \(x \notin f\inv[B]\). Therefore, \(x \in f\inv[A] \setminus f\inv[B]\). \qed
\end{enumerate}
}

\setexernumber{7}

\exer[2.3.8]{}{
    Every system of sets \(A\) can be indexed by a function.
}
\pf{Proof}{
    Let \(S\) be the function \(\mrm{Id}_A\)
    so \(S_i = i\) for all \(i \in A\).
    Then, \(A = \{\,S_i \mid i \in A\,\}\);
    \(A\) is indexed by \(S\).
}

\exer[2.3.9]{}{
    \begin{enumerate}[nolistsep, label=(\roman*)]
        \ii Let \(A\) and \(B\) be sets. Prove that \(B^A\) exists.
        \ii Let \(\lang S_i \mid i \in I\rang\) be an indexed system of sets.
            Prove that \(\prod_{i \in I} S_i\) exists.
    \end{enumerate}
}
\mclm{Proof}{\hfill
\begin{enumerate}[nolistsep, label=(\roman*)]
    \ii
    If \(f\) is a function from \(A\) into \(B\),
    then \(f \subseteq A \times B\), i.e., \(f \in \mcal P(A \times B)\).

    \ii
    If \(f\) is a function on \(I\) and \(f_i \in S_i\) for all \(i \in I\),
    then \(f\) is a function onto \(\bigcup_{i \in I} S_i\).
    Hence, \(f \in \left(\bigcup_{i \in I} S_i\right)^I\). \qed
\end{enumerate}
}

\exer[2.3.10]{}{
    Let \(\lang\,F_a \mid a \in \bigcup S\,\rang\) be an indexed system of sets.
    \begin{enumerate}[nolistsep, label=(\roman*), leftmargin=*, listparindent=\parindent]
        \ii
        \(\bigcup_{a \in \bigcup S} F_a = \bigcup_{C \in S} \left[ \bigcup_{a \in C} F_a \right]\)\vspace*{.3em}

        \ii
        \(\bigcap_{a \in \bigcup S} F_a = \bigcap_{C \in S} \left[ \bigcap_{a \in C} F_a \right]\)
        if \(S \neq \OO\) and \(\fall C \in S,\:C \neq \OO\).
    \end{enumerate}
}
\mclm{Proof}{\hfill
\begin{enumerate}[nolistsep, label=(\roman*), leftmargin=*, listparindent=\parindent]
    \ii
    \(\begin{aligned}[t]
        \textstyle x \in \bigcup_{a \in \bigcup S} F_a
        &\iff \textstyle\exs a \in \bigcup S,\: x \in F_a \\
        &\iff \exs C \in S,\: \exs a \in C,\: x \in F_a \\
        &\iff \textstyle\exs C \in S,\: x \in \bigcup_{a \in C} F_a
        \iff \textstyle x \in \bigcup_{C \in S} \left[ \bigcup_{a \in C} F_a \right]
    \end{aligned}\)\vspace*{.3em}

    \ii
    \(\begin{aligned}[t]
        \textstyle x \in \bigcap_{a \in \bigcup S} F_a
        &\iff \textstyle \fall a \in \bigcup S,\: x \in F_a \\
        &\iff \fall C \in S,\: \fall a \in C,\: x \in F_a \\
        &\iff \textstyle \fall C \in S,\: x \in \bigcap_{a \in C} F_a
        \iff \textstyle x \in \bigcap_{C \in S} \left[ \bigcap_{a \in C} F_a\right]
    \end{aligned}\)\vspace*{.3em}
    \qed
\end{enumerate}
}

\exer[2.3.11]{}{
    Let \(\lang\,F_a \mid a \in A\,\rang\) be an nonempty indexed system of sets.
    \begin{enumerate}[nolistsep, label=(\roman*), leftmargin=*, listparindent=\parindent]
        \ii \(B \setminus \bigcup_{a \in A} F_a = \bigcap_{a \in A} (B \setminus F_a)\) \vspace*{.3em}
        \ii \(B \setminus \bigcap_{a \in A} F_a = \bigcup_{a \in A} (B \setminus F_a)\)
    \end{enumerate}
}
\mclm{Proof}{\hfill
\begin{enumerate}[nolistsep, label=(\roman*), leftmargin=*, listparindent=\parindent]
    \ii
    \(\begin{aligned}[t]
        \textstyle x \in B \setminus \bigcup_{a \in A} F_a
        &\iff x \in B \land \lnot (\exs a \in A,\: x \in F_a) \\
        &\iff x \in B \land \fall a \in A,\: x \notin F_a \\
        &\iff \fall a \in A,\: (x \in B \land x \notin F_a)
        \iff \textstyle x \in \bigcap_{a \in A} (B \setminus F_a)
    \end{aligned}\)\vspace*{.3em}
    \ii
    \(\begin{aligned}[t]
        \textstyle x \in B \setminus \bigcap_{a \in A} F_a
        &\iff x \in B \land \lnot (\fall a \in A,\: x \in F_a) \\
        &\iff x \in B \land \exs a \in A,\: x \notin F_a \\
        &\iff \exs a \in A,\: (x \in B \land x \notin F_a)
        \iff \textstyle x \in \bigcup_{a \in A} (B \setminus F_a)
    \end{aligned}\)
    \qed
\end{enumerate}
}

\exer[2.3.12]{}{
    Let \(R\) be a relation and let \(\lang\,F_a \mid a \in A\,\rang\) be an indexed system of sets.
    \begin{enumerate}[nolistsep, label=(\roman*), leftmargin=*, listparindent=\parindent]
        \ii \(R \left[\bigcup_{a \in A} F_a\right] = \bigcup_{a\in A} R[F_a]\)\vspace*{.3em}
        \ii \(R \left[\bigcap_{a \in A} F_a\right] \subseteq \bigcap_{a\in A} R[F_a]\) if \(A \neq \OO\).\vspace*{.3em}
        \ii \(R \left[\bigcap_{a \in A} F_a\right] = \bigcap_{a\in A} R[F_a]\) if \(A \neq \OO\) and \(R\) is an injective function.\vspace*{.3em}
        \ii \(R\inv \left[\bigcap_{a \in A} F_a\right] = \bigcap_{a\in A} R\inv[F_a]\) if \(A \neq \OO\) and \(R\) is a function.
    \end{enumerate}
}
\mclm{Proof}{\hfill
\begin{enumerate}[nolistsep, label=(\roman*), leftmargin=*, listparindent=\parindent]
    \ii
    \(\begin{aligned}[t]
        \textstyle y \in R \left[\bigcup_{a \in A} F_a\right]
        &\iff \textstyle \exs x \in \bigcup_{a \in A} F_a,\: xRy \\
        &\iff \exs a \in A,\: \exs x \in F_a,\: xRy \\
        &\iff \textstyle \exs a \in A,\: y \in R[F_a]
        \iff x \in \bigcup_{a\in A} R[F_a]
    \end{aligned}\)\vspace*{.3em}

    \ii
    Take any \(y \in R \left[\bigcap_{a \in A} F_a\right]\).
    Then, there exists \(x \in \bigcap_{a \in A} F_a\) such that \(xRy\).
    Hence, for all \(a \in A\), \(y \in R[F_a]\),
    i.e., \(y \in \bigcap_{a \in A} R[F_a]\).

    \ii
    If \(R\) is an injective function, then \(R\inv\) is also a function.
    Hence, the result follows from (iv) and the fact that \(R = (R\inv)\inv\).

    \ii
    Thanks to (ii), since \(R\inv\) is a relation, we only need to prove the other inclusion.
    Take any \(x \in \bigcap_{a\in A} R\inv[F_a]\).
    Fix any \(a^\ast \in A\). Then, there exists \(y^\ast \in F_{a^\ast}\) such that \(xRy^\ast\).

    Now, take any \(a \in A\). Then, \(\exs y \in F_a\) such that \(xRy\).
    Since \(R\) is a function, \(y = y^\ast\); \(y^\ast \in F_a\),
    i.e., \(y^\ast \in \bigcap_{a \in A} F_a\).
    Therefore, \(x \in R\inv\left[\bigcap_{a \in A} F_a\right]\).
    \qed
\end{enumerate}
}

\end{document}
