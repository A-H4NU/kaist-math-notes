\documentclass[../introduction_to_set_theory.tex]{subfiles}

\begin{document}

\subsection*{Selected Problems}

\setexernumber{1}

\exer[1.4.2]{}{
    \begin{enumerate}[nolistsep, label=(\roman*), ref=\protect{\Cref{exer:1.4.2} (\roman*)}, leftmargin=*, listparindent=\parindent]
        \ii \(A \setminus B = (A \cup B) \setminus B = A \setminus (A \cap B)\)
        \ii \(A \setminus (B \setminus C) = (A \setminus B) \cup (A \cap C)\)
        \ii\label{itm:1.4.2.iii} \(A \cap B = A \setminus (A \setminus B)\)
    \end{enumerate}
}
\mclm{Proof}{\hfill
\begin{enumerate}[nolistsep, label=(\roman*), leftmargin=*, listparindent=\parindent]
    \ii\mbox{}\vspace*{-\baselineskip}
    \begin{flalign*}\SwapAboveDisplaySkip
        x \in A \land x \notin B
        &\iff x \in A \land x \notin B \lor x \in B \land x \notin B & \comment*{\(\lor\)-intro / \(\lor\)-syllogism} \\
        &\iff (x \in A \lor x \in B) \land x \notin B & \comment*{Distribution}
    \end{flalign*}
    \begin{flalign*}\SwapAboveDisplaySkip
        x \in A \land x \notin B
        &\iff x \in A \land x \notin A \lor x \in A \land x \notin B & \comment*{\(\lor\)-intro / \(\lor\)-syllogism} \\
        &\iff x \in A \land (x \notin A \lor x \notin B) & \comment*{Distribution} \\
        &\iff x \in A \land \lnot(x \in A \land x \in B) & \comment*{De Morgan's Law}
    \end{flalign*}

    \ii\mbox{}\vspace*{-\baselineskip}
    \begin{flalign*}\SwapAboveDisplaySkip
        x \in A \land \lnot (x \in B \land x \notin C)
        &\iff x \in A \land (x \notin B \lor x \in C) & \comment*{De Morgan's Law} \\
        &\iff (x \in A \land x \notin B) \lor (x \in A \land x \in C) & \comment*{Distribution}
    \end{flalign*}

    \ii
    By (ii), \(A \setminus (A \setminus B) = (A \setminus A) \cup (A \cap B) = A \cap B\).
    \qed
\end{enumerate}
}

\setexernumber{3}

\exer[1.4.4]{}{
    For any set \(A\), prove that a ``complement'' of \(A\) (\(\{\,x \mid x \notin A\,\}\)) does not exist.
}
\pf{Proof}{
    Let \(B\) be the complement of \(A\) for the sake of contradiction.
    Then, \(A \cup B\) is the set of all sets, which is impossible by \Cref{exer:1.3.3}.
}

\end{document}
