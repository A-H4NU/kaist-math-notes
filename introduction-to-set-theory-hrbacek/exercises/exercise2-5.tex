\documentclass[../introduction_to_set_theory.tex]{subfiles}

\begin{document}

\subsection*{Selected Problems}

\exer[2.5.1]{}{
    \begin{enumerate}[nolistsep, label=(\roman*)]
        \ii
        Let \(R\) be a partial ordering of \(A\) and let \(S\) be the strict ordering of \(A\) corresponding to \(R\).
        Let \(R^\ast\) be the partial ordering of \(A\) corresponding to \(S\). Show that \(R^\ast = R\).
        \ii
        Let \(S\) be a strict ordering of \(A\) and let \(R\) be the partial ordering of \(A\) corresponding to \(S\).
        Let \(S^\ast\) be the partial ordering of \(A\) corresponding to \(R\). Show that \(S^\ast = S\).
    \end{enumerate}
}
\mclm{Proof}{\hfill
\begin{enumerate}[nolistsep, label=(\roman*)]
    \ii
    \(R^\ast = S \cup \mrm{Id}_A = (R \setminus \mrm{Id}_A) \cup \mrm{Id}_A = R\)
    since \(\mrm{Id}_A \subseteq R\).

    \ii
    \(S^\ast = R \setminus \mrm{Id}_A = (S \cup \mrm{Id}_A) \setminus \mrm{Id}_A = S\)
    since \(\mrm{Id}_A \cap S = \OO\).

    \qed
\end{enumerate}
}

\setexernumber{5}

\exer[2.5.6]{}{
    Let \((A_1, <_1)\) and \((A_2, <_2)\) be strictly ordered sets and let \(A_1 \cap A_2 = \OO\).
    Define a relation \(\prec\) on \(B \triangleq A_1 \cup A_2\) as follows:
    \[
        x \prec y \iff (x <_1 y) \lor (x <_2 y) \lor (x \in A_1 \land y \in A_2).
    \]
    Show that \(\prec\) is a strict ordering of \(B\) and \(\mathord{\prec} \cap A_1^2 = \mathord{<_1}\), \(\mathord{\prec} \cap A_2^2 = \mathord{<_2}\).
}
\pf{Proof}{
    Note that \(\mathord{\prec} = \mathord{<}_1 \cup \mathord{<}_2 \cup A_1 \times A_2\).

    Suppose \(x \prec y\) and \(y \prec x\).
    By definition, \(x, y \in A_1\) or \(x, y \in A_2\).
    In both cases, we have (\(x <_1 y\) and \(y <_1 x\))
    or (\(x <_2 y\) and \(y <_2 x\)), which are impossible as \(<_1\) and \(<_2\) are asymmetric.
    Hence, \(\prec\) is asymmetric.
    Transitivity of \(\prec\) can be shown easily.

    Since \(\mathord{<_1} \cap A_2^2 = \mathord{<_2} \cap A_1^2 = (A_1 \times A_2) \cap A_1^2 = (A_1 \times A_2) \cap A_2^2 = \OO\),
    we get \(\mathord{\prec} \cap A_1^2 = \mathord{<_1}\) and \(\mathord{\prec} \cap A_2^2 = \mathord{<_2}\).
}

\exer[2.5.7]{}{
    Let \(R\) be a reflexive and transitive relation in \(A\) (\(R\) is called a \textit{preordering} of \(A\)).
    Define a relation \(E\) in \(A\) by
    \[
        aEb \iff aRb \land bRa.
    \]
    Show that \(E\) is an equivalence on \(A\). Define the relation \(R/E\) in \(A/E\) by
    \[
        [a]_E R/E [b]_E \iff aRb.
    \]
    Show that \(R/E\) is well-defined and that \(R/E\) is a partial ordering of \(A/E\).
}
\pf{Proof}{
    Since \(aEa \equiv aRa\) and \(R\) is reflexive, \(E\) is reflexive as well.
    Since \(aEb \equiv bEa\), \(E\) is symmetric.
    Since \(aEb \land bEc \iff (aRb \land bRc) \land (cRb \land bRa) \implies aRc \land cRa \iff aEc\),
    \(E\) is transitive. \checkmark

    Assume \([a]_E = [a']_E\) and \([b]_E = [b']_E\).
    Then, we have \(aEa'\) and \(bEb'\) by \Cref{lem:equivIffSameClass},
    i.e., \(aRa'\), \(a'Ra\), \(bRb'\), and \(b'Rb\).
    By transitivity of \(R\), it follows that \(aRb \iff a'Rb'\).
    Therefore, \(R/E\) is well-defined. \checkmark

    It can be shown readily that \(R/E\) is reflexive and transitive.
    To prove \(R/E\) is antisymmetric, assume \([a]_E R/E [b]_E\) and \([b]_E R/E [a]_E\).
    Then, \(aRb\) and \(bRa\), which means \(aEb\).
    Therefore, \([a]_E = [b]_E\) by \Cref{lem:equivIffSameClass}. \checkmark
}

\exer[2.5.8]{}{
    Let \(A = \mcal P(X)\) where \(X\) is a set.
    \begin{enumerate}[nolistsep, label=(\roman*)]
        \ii
        Any \(S \subseteq A\) has a supremum in the ordering \(\subseteq_A\); \(\sup S = \bigcup S\).
        \ii
        Any \(S \subseteq A\) has an infimum in the ordering \(\subseteq_A\);
        \(\inf S = \begin{cases}
            \bigcap S & \text{if } S \neq \OO \\
            X & \text{if } S = \OO
        \end{cases}\).
    \end{enumerate}
}
\mclm{Proof}{\hfill
\begin{enumerate}[nolistsep, label=(\roman*)]
    \ii
    As \(C \subseteq_A \bigcup S\) for all \(C \in S\), \(\bigcup S\) is an upper bound of \(S\).
    Let \(U\) be any upper bound of \(S\).
    Take any \(x \in \bigcup S\). Then, there exists \(C \in S\) such that \(x \in C\).
    Since \(C \subseteq_A U\), we have \(x \in U\).
    Therefore, \(\bigcup S \subseteq U\); \(\bigcup S\) is the least upper bound of \(S\).

    \ii
    If \(S = \OO\), then any \(C \in A\) is an lower bound of \(S\).
    Since \(\bigcup A = X\)---by (i), the supremum of the set of lower bounds of \(S\)---is a lower bound of \(S\),
    \(X\) is the infimum of \(S = \OO\). \checkmark

    If \(S \neq \OO\), as \(\bigcap S \subseteq C\) for all \(C \in S\),
    \(\bigcap S\) is a lower bound of \(S\).
    Let \(L\) be any lower bound of \(S\).
    Take any \(x \in L\). Then, \(\fall C \in L,\: x \in C\), i.e., \(x \in \bigcap S\).
    Therefore, \(L \subseteq_A \bigcap S\); \(\bigcap S\) is the infimum of \(S\). \checkmark
    \qed
\end{enumerate}
}

\exer[2.5.8]{}{
    Let \(\mrm{Fn}(X, Y)\) be the set of all functions mapping a subset of \(X\) into \(Y\),
    i.e., \(\mrm{Fn}(X, Y) = \bigcup_{Z \in \mcal P(X)} Y^Z\).
    Define a relation \(\le\) in \(\mrm{Fn}(X, Y)\) by
    \[
        f \le g \iff f \subseteq g.
    \]
    \begin{enumerate}[nolistsep, label=(\roman*)]
        \ii \(\le\) is a partial ordering of \(\mrm{Fn}(X, Y)\).
        \ii
        Let \(F \subseteq \mrm{Fn}(X, Y)\).
        \(\sup F\) exists if and only if \(F\) is a compatible system of functions.
        Moreover, \(\sup F = \bigcup F\) if it exists.
    \end{enumerate}
}
\mclm{Proof}{\hfill
\begin{enumerate}[nolistsep, label=(\roman*)]
    \ii
    \(\mathord{\le} = \mathord{\subseteq}_{\mrm{Fn}(X, Y)}\) by definition;
    \(\subseteq_{\mrm{Fn}(X, Y)}\) is already a partial ordering of \(\mrm{Fn}(X, Y)\).

    \ii
    (\(\Rightarrow\)) Assume \(h \in \mrm{Fn}(X, Y)\) is a supremum of \(F\).
    Then, \(\fall f \in F,\: f \subseteq s\).
    Take any \(f, g \in F\).
    Then, \(f \cup g \subseteq h\), and thus \(f \cup g\) is a function as \(h\) is a function.
    Therefore, by \Cref{lem:compatibleIff}, \(f\) and \(g\) are compatible.
    Hence, \(F\) is a compatible system of functions.

    (\(\Leftarrow\))
    Assume \(F\) is a compatible system of functions.
    Then, \(\bigcup F \in \mrm{Fn}(X, Y)\) by \Cref{th:compatibleThenUnionIsFunction},
    and \(f \le \bigcup F\) for all \(f \in F\) by definition; \(\bigcup F\) is an upper bound of \(F\).
    Let \(U\) be any upper bound of \(S\).
    Take any \((x, y) \in \bigcup F\). Then, there exists \(f \in S\) such that \((x, y) \in f\).
    Since \(f \subseteq_A U\), we have \(x \in U\).
    Therefore, \(\bigcup F \subseteq U\); \(\bigcup F\) is the least upper bound of \(S\).
    \qed
\end{enumerate}
}

\exer[2.5.10]{}{
    Let \(\mrm{Pt}(A)\) be the set of all partitions of \(A\).
    Define a relation $\preccurlyeq$ in \(\mrm{Pt}(A)\) by
    \[
        S_1 \preccurlyeq S_2 \iff \fall C \in S_1,\: \exs D \in S_2,\: C \subseteq D.
    \]
    (We say that the partition \(S_1\) is a \textit{refinement} of the partition \(S_2\) if \(S_1 \preccurlyeq S_2\).)
    \begin{enumerate}[nolistsep, label=(\roman*)]
        \ii
        \(\preccurlyeq\) is a partial ordering of \(\mrm{Pt}(A)\).
        \ii
        \(\inf T\) exists for all \(T \subseteq \mrm{Pt}(A)\).
        \ii
        \(\sup T\) exists for all \(T \subseteq \mrm{Pt}(A)\).
    \end{enumerate}
}
\mclm{Proof}{\hfill
\begin{enumerate}[nolistsep, label=(\roman*), listparindent=\parindent]
    \ii
    \(\preccurlyeq\) is reflexive since, for all \(S \in \mrm{Pt}(A)\) and \(C \in S\),
    \(C \subseteq C\), i.e., \(S \preccurlyeq S\). \checkmark

    Assume \(S_1 \preccurlyeq S_2\) and \(S_2 \preccurlyeq S_1\).
    Take any \(C \in S_1\).
    Then, there exists \(D \in S_2\) such that \(C \subseteq D\).
    In addition, there exists \(E \in S_1\) such that \(D \subseteq E\).
    We have \(C \subseteq E\) but \(C\) is nonempty as \(S_1\) is a partition,
    which implies \(C \cap E \neq \OO\).
    Therefore, as \(S_1\) is a partition, we must have \(C = E\) and thus \(C = D\).
    Hence, \(S_1 \subseteq S_2\). This shows that \(\preccurlyeq\) is antisymmetric. \checkmark

    Assume \(S_1 \preccurlyeq S_2\) and \(S_2 \preccurlyeq S_3\).
    Take any \(C \in S_1\).
    There exists \(D \in S_2\) such that \(C \subseteq D\).
    There exists \(E \in S_3\) such that \(D \subseteq E\).
    Hence, \(C \subseteq E\); \(S_1 \preceq S_3\).
    This shows that \(\preccurlyeq\) is transitive. \checkmark

    \ii
    Define a relation \(E\) in \(A\) by
    \(E \triangleq \{\,(a, b) \in A^2 \mid \fall S \in T,\: \exs C \in S,\: a \in C \land b \in C\,\}\).
    It can be easily shown that \(E\) is an equivalence
    mimicking the proof of \Cref{th:partitionDerivesEquiv}.
    Then, \(A/E \in \mrm{Pt}(A)\) by \Cref{th:equivDerivesPartition}.

    \clm[AEisLowerBound]{
        \(A/E\) is a lower bound of \(T\).
    }{
        If \(T = \OO\), there is nothing to prove; so assume \(T \neq \OO\).
        Take any \(S \in T\) and \(a \in A\).
        Then, there exists \(C \in S\) such that \(a \in S\) since \(S\) is a partition of \(A\).
        Let \(b \in [a]_E\). Then, there exists \(D \in S\) such that
        \(a, b \in D\), which implies \(C = D\). Therefore, \([a]_E \subseteq C\).
        Hence, \(A/E \preccurlyeq S\). \qed
    }

    \clm[AEisMaxLowerBound]{
        For each lower bound \(L\) of \(T\), \(L \preccurlyeq A/E\).
    }{
        If \(T = \OO\), then \(A/E = \{A^2\}\) and every partition of \(A\) is a lower bound.
        Since \(S \preccurlyeq \{A^2\}\) for all \(S \in \mrm{Pt}(A)\), the result follows.

        Now, assume \(T \neq \OO\).
        Let \(L\) be a lower bound of \(T\).
        Take any \(D \in L\). Fix some \(a \in D\).
        Then, each \(d \in D\) has the property that
        \(\fall S \in T,\: \exs C \in S,\: \{a,d\} \subseteq D \subseteq C\)
        as \(L\) is a lower bound of \(T\).
        Therefore, \(d \in [a]_E\); \(D \subseteq [a]_E\).
        Hence, \(L \preccurlyeq A/E\). \qed
    }
    \Cref{clm:AEisLowerBound,clm:AEisMaxLowerBound} say that \(\inf T = A/E\).
    Hence, \(\inf T\) exists.

    \ii
    Let \(T' \triangleq \{\,S' \in \mrm{Pt}(A) \mid \fall S \in T,\: S \preccurlyeq S'\,\}\).
    By (ii), \(S^\ast \triangleq \inf T'\) exists.
    \clm[SastIsUpperBound]{
        \(S^\ast\) is an upper bound of \(T\).
    }{
        In (ii), we showed that \(S^\ast = A/E\)
        where \(E = \{\,(a, b) \in A^2 \mid \fall S' \in T',\: \exs C' \in S',\: a \in C' \land b \in C'\,\}\).
        Take any \(S \in T\) and let \(C \in S\).
        Fix some \(c_0 \in C\).

        Now, take arbitrary \(c \in C\). Then, for all \(S' \in T'\),
        since \(S \preccurlyeq S'\), there exists \(D' \in S'\)
        such that \(c \in C \subseteq D'\).
        Hence, we have \(cEc_0\); \(C \subseteq [c_0]_E\).
        Therefore, \(S \preccurlyeq S^\ast\). \qed
    }
    \Cref{clm:SastIsUpperBound} essentially says that \(S^\ast \in T'\).
    By \Cref{th:basicInfimum} (iii), \(S^\ast = \min T'\), i.e., \(S^\ast = \sup T\).
    \qed
\end{enumerate}
}

\setexernumber{12}

\exer[2.5.13]{}{
    If \(h\) is isomorphism between \((P, \le)\) and \((Q, \preceq)\),
    then \(h\inv\) is an isomorphism between \((Q, \preceq)\) and \((P, \le)\).
}
\pf{Proof}{
    Take any \(q_1, q_2 \in Q\).
    Then, we have \(q_1 \preceq q_2 \iff h(h\inv(q_1)) \preceq h(h\inv(q_2)) \iff h\inv(q_1) \le h\inv(q_2)\).
}

\exer[2.5.14]{}{
    If \(f\) is an isomorphism between \((P_1, \le_1)\) and \((P_2, \le_2)\),
    and if \(g\) is an isomorphism between \((P_2, \le_2)\) and \(P_3, \le_3\),
    then \(g \circ f\) is an isomorphism between \((P_1, \le_1)\) and \((P_3, \le_3)\).
}
\pf{Proof}{
    \(\ran (g \circ f) = g[\ran f] = P_3\).
    Moreover, \(g \circ f\) is one-to-one.
    Hence, \(g \circ f \colon P_1 \hooktwoheadrightarrow P_3\).
    For all \(p, q \in P_1\), we have
    \(p \le_1 q \iff f(p) \le_2 f(q) \iff g(f(p)) \le_3 \iff g(f(q))\).
    Hence, \(g \circ f\) is an isomorphism between \((P_1, \le_1)\) and \((P_3, \le_3)\).
}

\end{document}
