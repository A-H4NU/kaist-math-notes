\documentclass[../introduction_to_set_theory.tex]{subfiles}

\begin{document}

\subsection*{Selected Problems}

\exer[3.3.1]{}{
    Let \(f \colon \NN \to A\) be an infinite sequence
    where \((A, \preceq)\) is an ordered set. Then,
    \[
        \fall n \in \NN,\: f_n \prec f_{n+1}
        \implies \fall m, n \in \NN,\: (n < m \implies f_n \prec f_m).
    \]
}
\pf{Proof}{
    Fix any \(n \in \NN\) and let \(\mbf{P}(x)\) be the property ``\(f_n \prec f_x\).''
    \(\mbf{P}(n+1)\) evidently holds.
    Now, suppose \(\mbf{P}(k)\) holds where \(k \in \NN\).
    Then, chaining \(f_n \prec f_k\) and \(f_k \prec f_{k+1}\) gives \(\mbf{P}(k+1)\).
    Therefore, by \Cref{exer:3.2.11}, we get \(\fall m \ge n + 1,\: f_n \prec f_m\).
}

\exer[3.3.2]{}{
    Let \((A, \preceq)\) be a nonempty linearly ordered set.
    We say that \(q \in A\) is a \textit{successor} of \(p \in A\) if there is no \(r \in A\)
    such that \(p \prec r \prec q\). Assume \((A, \preceq)\) has the following properties:
    \begin{enumerate}[nolistsep, label=(\roman*)]
        \ii Every \(p \in A\) has a successor.
        \ii Every nonempty subset of \(A\) has a \(\preceq\)-least element.
        \ii If \(p \in A\) is not the \(\preceq\)-least element of \(A\),
            then \(p\) is a successor of some \(q \in A\).
    \end{enumerate}
    Then, \((A, \preceq)\) is isomorphic to \((\NN, \le)\).
}
\pf{Proof}{
    By (i), for each \(p \in P\), \(\{\,q \in A \mid p \prec q\,\} \neq \OO\), and thus it has
    a \(\preceq\)-least element by (ii).
    Therefore, by \nameref{th:recursion}, there exists a sequence \(f \colon \NN \to A\)
    such that \(f_0 = \min A\) and \(\fall n \in \NN,\: f_{n+1} = \min \{\,q \in A \mid f_n \prec q\,\}\).

    \clm[88332b22]{
        \(\ran f = A\)
    }{
        Suppose \(X \triangleq A \setminus \ran f \neq \OO\) for the sake of contradiction.
        Then, by (ii), we may take \(p = \min X\).
        Since \(\min A = f_0 \in \ran f\), \(p\) is not the \(\preceq\)-least element of \(A\).
        Hence, by (iii), \(p\) is a successor of some \(q \in A\).
        As \(q \prec p\), we have \(q \in \ran f\) by minimality of \(q\),
        i.e., \(q = f_m\) for some \(m \in \NN\).
        Since there is no \(r \in A\) such that \(q \prec r \prec p\),
        we have \(p = f_{m+1}\) by definition,
        which contradicts \(p \notin \ran f\). \qed
    }
    Since \(f_n \prec f_{n+1}\) for all \(n \in \NN\),
    by \Cref{exer:3.3.1}, \(\fall m, n \in \NN,\: (m < n \implies f_m \prec f_n)\),
    which means \(f\) is injective.

    Therefore, together with \Cref{clm:88332b22}, \(f\) is an isomorphism between
    \((\NN, \le)\) and \((A, \preceq)\) by \Cref{lem:oneImplicationIsEnough}.
}

\setexernumber{4}

\exer[3.3.5]{The Recursion Theorem: Partial Version}{
    Let \(g\) be a function such that \(\dom g \subseteq A \times \NN\) and \(\ran g \subseteq A\).
    Let \(a \in A\).
    Then, there uniquely exists a sequence \(f\) of elements of \(A\) such that
    \begin{enumerate}[nolistsep, label=(\roman*)]
        \ii \(f_0 = a\)
        \ii \(\fall n \in \NN,\: [n+1 \in \dom f \implies f_{n+1} = g(f_n, n)]\)
        \ii \(f\) is either an infinite sequence or a finite sequence of length \(k+1\) and \((f_k, k) \notin \dom g\).
    \end{enumerate}
}
\pf{Proof}{
    Let \(\ol{A} = A \cup \{\ol{a}\}\) where \(\ol{a} \notin A\).
    (Such \(\ol{a}\) exists by \Cref{exer:1.3.3} (ii).)
    Define \(\ol{g} \colon \ol{A} \times \NN \to \ol{A}\) by
    \[
        \ol{g}(x, n) = \begin{cases}
            g(x, n) & \text{if \((x, n) \in \dom g\)} \\
            \ol{a} & \text{otherwise.}
        \end{cases}
    \]
    Then, \nameref{th:recursion} guarantees the existence of \(\ol{f} \colon \NN \to \ol{A}\)
    such that \(\ol f_0 = a\) and \(\fall n \in \NN,\: \ol f_{n+1} = \ol g(\ol f_n, n)\).
    We have two cases: ``\(\fall n \in \NN,\: \ol{f}_{n} \neq \ol{a}\)'' and ``\(\exs n \in \NN,\: \ol f_n = \ol{a}\).''
    They are resolved by \Cref{clm:7a376003,clm:db06f79c}, respectively.

    \clm[7a376003]{
        If ``\(\fall n \in \NN,\: \ol{f}_{n} \neq \ol{a}\),''
        then \(\ol{f}\) is an infinite sequence of elements of \(A\) that satisfies (i) and (ii).
    }{
        The assumption essentially says that
        \((\ol f_n, n) \in \dom g\) and \(\ol f_{n+1} = g(\ol f_n, n) \in A\) for all \(n \in \NN\),
        i.e., \(\ol f\) satisfies (i) and (ii).
        As \(\ol f_0 = a \in A\), \(\ol f\) is an infinite sequence of elements of \(A\).
        \qed
    }

    \clm[db06f79c]{
        If ``\(\exs n \in \NN,\: \ol f_n = \ol{a}\),''
        then there exists \(k \in \NN\) such that
        \(\restr{\ol{f}}{k+1}\) satisfies the conditions (i), (ii), and (iii).
    }{
        By \nameref{th:NisWellOrdered}, we have \(\ell \triangleq \min \{\,n \in \NN \mid \ol f_n = \ol{a}\,\}\).
        Since \(\ol f_0 \in A\), we have \(\ell \neq 0\), and thus \(\ell = k + 1\) for some \(k \in \NN\)
        by \Cref{exer:3.2.4}.
        It immediately follows that \(\fall n \le k,\: \ol{f}_n \in A\).
        Hence, \(f \triangleq \restr{\ol f}{k+1}\) is a finite sequence of length \(k+1\)
        of elements of \(A\).

        We check if \(f\) satisfies the conditions (i), (ii), and (iii):
        \begin{enumerate}[nolistsep, label=(\roman*), leftmargin=*]
            \ii \(f_0 = \ol{f}_0 = a\)
            \ii
            If \(n < k\), i.e., \(n + 1 \in \dom f = k + 1\),
            then \(f_{n+1} = \ol{f}_{n+1} = \ol g(\ol{f}_n, n) = g(f_n, n)\).
            \ii
            If \((f_k, k) \in \dom g\), then we would have
            \(\ol{f}_\ell = \ol g(\ol{f}_k, k) = \ol g(f_k, k) = g(f_k, k) \neq \ol{a}\).
            Hence, we must have \((f_k, k) \notin \dom g\). \qed
        \end{enumerate}
    }

    Now, we prove the uniqueness.
    Let \(f\) and \(h\) be two sequences of elements of \(A\)
    that satisfies the conditions (i), (ii), and (iii).
    \WLOG, \(\dom h \subseteq \dom f\).

    Let \(\mbf{P}(x)\) be the property ``\(x \in \dom h \land f_x = h_x\).''
    \(\mbf{P}(0)\) evidently holds.
    \clm[e3045233]{
        \(\fall n \in \NN,\: (n + 1 \in \dom f \land \mbf{P}(n) \implies \mbf{P}(n+1))\)
    }{
        Assume \(n + 1 \in \dom f\) and \(\mbf{P}(n)\).
        Then, since \((h_n, n) = (f_n, n) \in \dom g\),
        \(n + 1 \in \dom h\) and \(h_{n+1} = g(h_n, n) = g(f_n, n) = f_{n+1}\).
        Hence, \(\mbf{P}(n+1)\) holds. \qed
    }

    If \(f\) is a finite sequence, \Cref{clm:e3045233} and \nameref{exer:3.2.12}
    imply \(h = f\).
    If \(f\) is an infinite sequence, \Cref{clm:e3045233} and \nameref{th:induction}
    imply \(h = f\).
}

\exer[3.3.6]{}{
    If \(X \subseteq \NN\), then there is a one-to-one (finite or infinite) sequence
    \(f\) such that \(\ran f = X\).
}
\pf{Proof}{
    If \(X = \OO\), \(\lang\rang\) is the one we are looking for.
    Assume \(X \neq \OO\).

    Let \(g = \big\{\,((x, n), y) \in (X \times \NN) \times X \:\big|\: y = \min \{\,k \in X \mid x < k\,\}\,\big\}\).
    Then, \(g\) is a function with \(\dom g \subseteq \NN \times \NN\) and \(\ran g \subseteq \NN\).
    By \nameref{exer:3.3.5}, there exists a sequence \(f\) of elements of \(X\)
    such that
    \begin{enumerate}[nolistsep, label=(\roman*)]
        \ii \(f_0 = \min X\) \comment{\(\min X\) exists by \nameref{th:NisWellOrdered}}
        \ii \(\fall n \in \NN,\: (n + 1 \in \dom f \implies f_{n+1} = g(f_n, n))\)
        \ii \(f\) is either an infinite sequence or a finite sequence of length \(k+1\) and \((f_k, k) \notin \dom g\).
    \end{enumerate}

    Note that \(\dom g = \{\,(x, n) \in X \times \NN \mid \exs y \in X,\: x < y\,\}\).
    Moreover, for each \(n \in \NN\) such that \(n + 1 \in \dom f\),
    we have \(f_n < f_{n+1}\); hence \(\forall m, n \in \dom f,\: (m < n \implies f_m < f_n)\)
    (in the similar manner of \Cref{exer:3.3.1}),
    and thus \(f\) is injective.

    Suppose \(Y = X \setminus \ran f \neq \OO\) for the sake of contradiction.
    By \nameref{th:NisWellOrdered}, we may take \(y = \min Y\).
    Then, by \Cref{th:hasUpperBoundThenMaxExists}, we may let \(z = \max \{\,x \in X \mid x < y\,\}\).
    \(z = f_m\) for some \(m \in \dom f\). Hence, \(y = f_{m+1}\).
}

\end{document}
