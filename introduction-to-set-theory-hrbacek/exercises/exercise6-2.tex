\documentclass[../introduction_to_set_theory_Note.tex]{subfiles}

\begin{document}

\subsection*{Selected Problems}

\exer[6.2.1]{}{
    A set \(X\) is transitive if and only if \(X \subseteq \mcal P(X)\).
}
\pf{Proof}{
    \(X\) is transitive iff every \(x \in X\) is a subset of \(X\)
    iff every \(x \in X\) is an element of \(\mcal P(X)\)
    iff \(X \subseteq \mcal P(X)\).
}

\exer[6.2.2]{}{
    A set \(X\) is transitive if and only if \(\bigcup X \subseteq X\).
}
\mclm{Proof}{\hfill
\begin{itemize}[nolistsep, wide=0pt, widest={(\(\Rightarrow\))}, leftmargin=*, listparindent=\parindent]
    \ii[(\(\Rightarrow\))]
    Take any \(u \in \bigcup X\). There exists some \(x \in X\) such that \(u \in x\).
    As \(X\) is transitive, \(u \in X\). Hence, \(\bigcup X \subseteq X\).

    \ii[(\(\Leftarrow\))]
    Let \(u \in v \in X\). Then, \(u \in \bigcup X \subseteq X\).
    Hence, \(X\) is transitive.
    \qed
\end{itemize}
}

\setexernumber{3}

\exer[6.2.4]{}{
    \begin{enumerate}[nolistsep, label=(\roman*), ref=\protect{\Cref{exer:6.2.4} (\roman*)}, listparindent=\parindent]
        \ii
        If \(X\) and \(Y\) are transitive, then \(X \cup Y\) is transitive.
        \ii
        If \(X\) and \(Y\) are transitive, then \(X \cap Y\) is transitive.
        \ii\label{itm:6.2.4.iii}
        If \(Y\) is transitive and \(S \subseteq \mcal P(Y)\), then \(Y \cup S\) is transitive.
        \ii
        There exist \(X\) and \(Y\) such that \(X \in Y\), \(Y\) is transitive, but \(X\) is not transitive.
        \ii
        There exist \(X\) and \(Y\) such that \(X \subseteq Y\), \(Y\) is transitive, but \(X\) is not transitive.
    \end{enumerate}
}
\mclm{Proof}{\hfill
\begin{enumerate}[nolistsep, label=(\roman*), leftmargin=*, listparindent=\parindent]
    \ii
    \(\bigcup (X \cup Y) = \left(\bigcup X\right) \cup \left(\bigcup Y\right) \subseteq X \cup Y\)
    by \Cref{exer:6.2.2}.
    Hence, \(X \cup Y\) is transitive.

    \ii
    \(\bigcup (X \cap Y) \subseteq \left(\bigcup X\right) \cap \left(\bigcup Y\right) \subseteq X \cup Y\)
    by \Cref{exer:6.2.2}.
    Hence, \(X \cap Y\) is transitive.

    \ii
    Let \(u \in v \in Y \cup S\).
    If \(v \in Y\), then \(u \in Y \subseteq Y \cup S\) as \(Y\) is transitive.
    If \(v \in S\), then \(u \in v \subseteq Y\).
    Hence, \(Y \cup S\) is transitive.

    \ii
    Let \(Y \triangleq \{\,\OO, \{\OO\}, X\,\}\) where \(X \triangleq \{\{\OO\}\}\).
    Then, \(\bigcup Y = \{\,\OO, \{\OO\}\,\} \subseteq Y\);
    \(Y\) is transitive by \Cref{exer:6.2.2}.
    However, \(\bigcup X = \{\OO\} \not\subseteq X\);
    \(X\) is not transitive.

    \ii
    Let \(X \triangleq \{\{\OO\}\}\) and \(Y \triangleq \{\,\OO, \{\OO\}\,\}\).
    Then, \(\bigcup Y = \{\OO\} \subseteq Y\); \(Y\) is transitive by \Cref{exer:6.2.2}.
    \(X\) is not transitive as in (iii).
    \qed
\end{enumerate}
}

\exer[6.2.5]{}{
    If every \(X \in S\) is transitive, then \(\bigcup S\) is transitive.
}
\pf{Proof}{
    Let \(u \in v \in \bigcup S\).
    Then, \(v \in X\) for some \(X \in S\).
    As \(X\) is transitive, then \(u \in X \subseteq \bigcup S\);
    hence \(\bigcup S\) is transitive.
}

\exer[6.2.6]{}{
    An ordinal \(\alpha\) is a natural number if and only if
    every nonempty subset of \(\alpha\) has a greatest element.
}
\mclm{Proof}{\hfill
\begin{itemize}[nolistsep, wide=0pt, widest={(\(\Rightarrow\))}, leftmargin=*, listparindent=\parindent]
    \ii[(\(\Rightarrow\))]
    If \(\alpha \in \NN\), then \(\alpha\) is a finite set of natural numbers.
    Hence, if \(\OO \subsetneq X \subseteq \alpha\), by \Cref{th:subsetOfFiniteIsFinite},
    \(X\) is also finite. Then, by \Cref{exer:3.5.13}, \(X\) has a greatest element.

    \ii[(\(\Leftarrow\))]
    Suppose \(\alpha \notin \NN\) for the sake of contradiction.
    By \Cref{th:finiteOrdinalIsNat}, \(\omega \le \alpha\).
    However, \(\omega\) does not have a greatest element.
    \qed
\end{itemize}
}

\exer[6.2.7]{}{
    If a set of ordinals \(X\) does not have a greatest element,
    then \(\sup X\) is a limit ordinal.
}
\pf{Proof}{
    \(\sup X \notin X\); otherwise \(\sup X\) would be a greatest element of \(X\).
    Suppose \(\sup X\) is a successor ordinal for the sake of contradiction,
    i.e., \(\bigcup X = \beta + 1\) for some ordinal \(\beta\).

    Take any \(\alpha \in X\).
    Then, \(\alpha \le \beta + 1\) but \(\alpha\) cannot equal \(\beta + 1 = \sup X\).
    As \(\lnot (\beta < \alpha < \beta + 1)\) and \(<\) totally orders \(X\),
    we get \(\alpha \le \beta\), which contradicts the minimality of \(\sup X\).
}

\exer[6.2.8]{}{
    If \(X\) is a nonempty set of ordinals,
    then \(\bigcap X \in \Ord\), and \(\bigcap X\) is a least element of \(X\).
}
\pf{Proof}{
    Let \(m \triangleq \bigcap X\).
    Let \(u \in v \in m\).
    Then, \(\fall \alpha \in X,\: u \in v \in \alpha\).
    As each \(\alpha \in X\) is ordinal, \(\fall \alpha \in X,\: u \in \alpha\),
    i.e., \(u \in m\). Hence, \(m\) is transitive.
    Moreover, by \ref{itm:basicOrdinal.iv},
    \(m\) is well-ordered by \(\in\).
    Hence, \(m\) is an ordinal.

    Let \(m' \triangleq m + 1\).
    Suppose \(\fall \alpha \in X,\: m' \le \alpha\) for the sake of contradiction.
    Then, \(m' \le m\), which is a contradiction.
    Hence, there exists \(\alpha \in X\) such that \(m \le \alpha < m'\).
    Then, by \Cref{exer:3.1.1}, \(\alpha = m\); thus \(m \in X\).
    Therefore, \(m = \min X\).
}

\end{document}
