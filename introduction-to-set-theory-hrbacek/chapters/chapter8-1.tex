\documentclass[../introduction_to_set_theory_Note.tex]{subfiles}

\begin{document}

\section{The Axiom of Choice and its Equivalents}

\dfn[choiceFunction]{Choice Function}{
    A function \(g \colon S \to \bigcup S\) is called a \textit{choice function for \(S\)} if
    \[
        \fall X \in S,\: (X \neq \OO \implies g(X) \in X).
    \]
}

\thm[wellOrdering]{Well-Ordering Theorem}{
    A set \(A\) can be well-ordered if and only if the set
    \(\mcal P(A)\) has a choice function.
}
\mclm{Proof}{\hfill
\begin{itemize}[nolistsep, wide=0pt, widest={(\(\Rightarrow\))}, leftmargin=*, listparindent=\parindent]
    \ii[(\(\Rightarrow\))]
    Assume \(\preceq\) well-orders \(A\).
    Define \(g \colon \mcal P(A) \to A\) by
    \[
        g(X) \triangleq \begin{cases}
            \min_{\preceq} X & \text{if}~x \neq \OO \\
            \OO & \text{otherwise.}
        \end{cases}
    \]
    Then, \(g\) is a choice function of \(\mcal P(A)\).

    \ii[(\(\Leftarrow\))]
    Fix any choice function \(g \colon \mcal P(A) \to A\) of \(\mcal P(A)\)
    and any \(a \notin A\).
    Let \(\mbf{G}(x)\) be the operation defined by
    \[
        \mbf{G}(x) \triangleq \begin{cases}
            g(A \setminus \ran x) & \text{if}~x~\text{is a function and}~A \setminus \ran x \neq \OO \\
            a & \text{otherwise.}
        \end{cases}
    \]
    Then, by \nameref{th:transRecursion}, there exists an operation \(\mbf{F}(x)\)
    such that \(\fall \alpha \in \Ord,\: \mbf{F}(\alpha) = \mbf{G}\left(\restr{\mbf{F}}{\alpha}\right)\).
    \clm[FtsPjUKS]{
        There exists \(\lambda < h(A)\) such that
        \(\mbf{F}(\lambda) = a\).
    }{
        Assume \(\alpha < \beta\) and \(\mbf{F}(\beta) \neq a\).
        Then, \(\mbf{F}(\beta) \in A \setminus \ran \left(\restr{\mbf{F}}{\beta}\right)\)
        and \(\mbf{F}(\alpha) \in \left(\restr{\mbf{F}}{\beta}\right)\);
        thus \(\mbf{F}(\alpha) \neq \mbf{F}(\beta)\).
        Hence, \(\restr{\mbf{F}}{\beta + 1}\) is injective if \(\mbf{F}(\beta) \neq a\).

        Suppose \(\fall \alpha < h(A),\: \mbf{F}(\alpha) \neq a\) for the sake of contradiction.
        Then, \(\restr{\mbf{F}}{h(A)}\) is an injection on \(h(A)\) into \(A\)
        by the preceding discussion,
        which contradicts \Cref{dfn:hartogsNumber}. \qed
    }

    Hence, by \Cref{clm:FtsPjUKS},
    we may let \(\lambda \triangleq \min \{\,\xi < h(A) \mid \mbf{F}(\xi) = a\,\}\).
    The discussion in the proof of \Cref{clm:FtsPjUKS} says that \(\restr{\mbf{F}}{\lambda}\)
    is injective.
    Clearly, we have \(\mbf{F}[\lambda] \subseteq A\) by definition of \(\lambda\).
    If it were \(\mbf{F}[\lambda] \subsetneq A\),
    then \(\mbf{F}(\lambda) \in A \setminus \mbf{F}[\lambda]\);
    thus \(\mbf{F}(\lambda) \neq a\), which is a contradiction.
    Hence, \(\restr{\mbf{F}}{\lambda} \colon \lambda \bijto A\).
    \(R \triangleq \{\,(\mbf{F}(\alpha), \mbf{F}(\beta)) \in A^2 \mid \alpha < \beta < \lambda\,\}\)
    is a well-ordering of \(A\).
    \qed
\end{itemize}
}

\thm[]{}{
    Every finite system of sets has a choice function.
}
\pf{Proof}{
    Let \(\mbf{P}(n)\) be the property ``for every set \(S\) with \(|S| = n\),
    there exists a choice function for \(S\).''
    \(\mbf{P}(0)\) and \(\mbf{P}(1)\) are evidently true.

    Fix any \(0 < n < \omega\) and assume \(\mbf{P}(n)\).
    Take any set \(S\) with \(|S| = n + 1\) and fix \(X \in S\).
    \WLOG, \(X \neq \OO\).
    Then, \(|S \setminus \{X\}| = n\), and thus
    there exists a choice function \(g\) for \(S \setminus \{X\}\).
    Fix any \(x \in X\).
    Then, \(g' \triangleq g \cup \{(X, x)\}\) is a choice function for \(S\).
    The result follows from \nameref{th:induction}.
}

\axiom[choice]{The Axiom of Choice}{
    \noindent
    There exists a choice function for every set.
    \[\textstyle
        \fall S,\: \exs g \in \left(\bigcup S\right)^S,\:
        \fall X \in S,\: (X \neq \OO \implies g(X) \in X)
    \]
}

\thm[aocTFAE]{}{
    \TFAE.
    \begin{enumerate}[nolistsep, label=(\roman*), ref=\protect{\Cref{th:aocTFAE} (\roman*)}]
        \ii
        \nameref{ax:choice}
        \ii\label{itm:aocTFAE.ii}
        Every partition has a set of representatives.
        \ii\label{itm:aocTFAE.iii}
        If \(\lang\,X_i \mid i \in I\,\rang\) is an
        \hyperref[dfn:indexedSystemOfSets]{indexed system} of nonempty sets,
        then there exists a function \(f\) on \(I\) such that \(\fall i \in I,\: f(i) \in X_i\).
    \end{enumerate}
}
\mclm{Proof}{\hfill
\begin{itemize}[
    nolistsep, wide=0pt, widest={\(\text{(ii)} \Rightarrow \text{(iii)}\)},
    align=right, leftmargin=*, listparindent=\parindent
]
    \ii[\(\text{(i)} \Rightarrow \text{(ii)}\)]
    Let \(f\) be a choice function for a partition \(S\).
    Then, \(X = \ran f\) is a set of representatives for \(S\).

    \ii[\(\text{(ii)} \Rightarrow \text{(iii)}\)] Hello
    Let \(C_i \triangleq \{i\} \times X_i\) for each \(i \in I\).
    Then, as \(\fall i, i' \in I,\: (i \neq i' \implies C_i \cap C_{i'} = \OO)\),
    \(S \triangleq \{\,C_i \mid i \in I\,\}\) is a partition.
    Let \(f\) be a set of representatives for \(S\).
    Then, for each \(i \in I\), there uniquely exists \(x \in X_i\)
    such that \((i, x) \in f \cap C_i\).
    Hence, \(f\) is a function on \(I\) and \(\fall i \in I,\: f(i) \in X_i\).

    \ii[\(\text{(iii)} \Rightarrow \text{(i)}\)]
    Take any \(S\) and let \(I \triangleq S \setminus \{\OO\}\).
    Let \(X_C \triangleq C\) for all \(C \in I\).
    Hence, the indexed system of sets \(\{\,X_C \mid C \in I\,\}\)
    has a function \(f\) on \(I\) such that \(f(C) \in X_C = C\) for each \(C \in I\).

    If \(\OO \notin S\), then \(f\) is a choice function for \(S\).
    If \(\OO \in S\), then \(f \cup \{(\OO, \OO)\}\) is a choice function for \(S\).
    \qed
\end{itemize}
}

\nt{
    \noindent
    From now, the results that depend on \nameref{ax:choice} is marked with [\textsf{AoC}].
}

\thm[infIffDedekindInf]{}{
    Every infinite set has a countable subset. \needsChoice
}
\pf{Proof}{
    Let \(A\) be an infinite set.
    By \nameref{th:wellOrdering},
    there exists \(f \colon \Omega \bijto A\) where \(\Omega \in \Ord\) with \(\omega \le \Omega\).
    Then, \(f[\omega]\) is a countable subset of \(A\).
}

\thm[infiniteHasAlephCard]{}{
    For every infinite set \(S\),
    there uniquely exists \(\alpha \in \Ord\)
    such that \(|S| = \aleph_{\alpha}\).
    \needsChoice
}
\pf{Proof}{
    By \nameref{th:wellOrdering}, \(|S| = |\Omega|\) for some infinite ordinal \(\Omega\).
    Then, by \Cref{th:wosetEquipotentToInitial,th:omegaIffInfInitial},
    there exists \(\alpha \in \Ord\) such that \(|\Omega| = \aleph_{\alpha}\).
    The uniqueness is evident.
}

\dfn[cardinal]{Cardinals}{
    For every set \(X\), \(|X|\) is the unique initial ordinal equipotent to \(X\).
}

\nt{
\Cref{as:cardinalExists} is justified by \Cref{dfn:cardinal}, which is justified by these facts:
\begin{itemize}[nolistsep, leftmargin=*, listparindent=\parindent]
    \ii
    There exists \(f \colon X \bijto Y\) if and only if \(|X|\) is equal to \(|Y|\).

    \ii
    There exists \(f \colon X \injto Y\) if and only if \(|X|\) is less than or equal to \(|Y|\).
\end{itemize}
}

\thm[cardinalLinearOrder]{}{
    For any sets \(A\) and \(B\), we have \(|A| \le |B|\) or \(|B| \le |A|\).
    \needsChoice
}
\pf{Proof}{
    \(|A|\) and \(|B|\) are ordinal numbers,
}

\thm[strongUnionOfCountable]{}{
    The union of countably infinite collection of countable sets is countable.
    \needsChoice
}
\pf{Proof}{
    Let \(|S| = \aleph_0\) and \(\fall X \in S,\: |S| \le \aleph_0\).
    Fix any injective sequence \(\lang\,A_n \mid n \in \NN\,\rang\) onto \(S\).
    For each \(n \in \NN\), as \(|A_n| \le \aleph_0\),
    there exists \(f \colon \NN \surjto A_n\).
    Let \(P_n \triangleq \{\,f \in A_n^{\>\NN} \mid \ran f = A_n\,\}\),
    which is nonempty.

    By \ref{itm:aocTFAE.iii}, there exists a function \(g\) on \(\NN\)
    such that \(\fall n \in \NN,\: g_n \in P_n\).
    Then, we can define \(\eta \colon \NN \times \NN \surjto \bigcup S\)
    by \(\eta(n, k) = g_n(k)\);
    hence \(\bigcup S\) is countable by \Cref{th:imageOfCountable}.
}

\thm[]{}{
    \(\aleph_1 \le 2^{\aleph_0}\) \needsChoice
}
\pf{Proof}{
    This is a consequence of \Cref{th:infiniteHasAlephCard} and \nameref{th:cantor}.
}

\thm[imageIsLessThanDomain]{}{
    If \(f\) is a function on \(A\), then \(|\ran f| \le |A|\). \needsChoice
}
\pf{Proof}{
    Since \(|A| = |\alpha|\) for some ordinal \(\alpha\) by
    \Cref{th:infiniteHasAlephCard},
    the result follows from \Cref{exer:7.2.5}.
}

\thm[alephUnionOfAlephs]{}{
    Let \(\alpha \in \Ord\).
    If \(|S| \le \aleph_{\alpha}\) and \(\fall X \in S,\: |X| \le \aleph_{\alpha}\),
    then \(\big|\bigcup S\big| \le \aleph_{\alpha}\).
    \needsChoice
}
\pf{Proof}{
    \WLOG, \(S \neq \OO\) and \(\fall X \in S,\: X \neq \OO\).
    Write \(S = \{\,X_\nu \mid \nu < \omega_{\alpha}\,\}\)
    and, for each \(\nu < \omega_{\alpha}\),
    choose a transfinite sequence \(\lang\,x_\nu(k) \mid k < \omega_{\alpha}\,\rang\)
    such that \(X_{\nu} = \{\,a_\nu(k) \mid k < \omega_{\alpha}\,\}\).
    We may define \(f \colon \omega_\alpha \times \omega_\alpha \surjto \bigcup S\)
    by \(f(\nu, \kappa) = x_\nu(\kappa)\).
    Hence,
    \begin{alignat*}{2}
        \textstyle \left|\bigcup S\right|
        &\le \aleph_{\alpha} \cdot \aleph_{\alpha} &\qquad& \comment*{\Cref{th:imageIsLessThanDomain}} \\
        &= \aleph_{\alpha}. && \comment*{\Cref{th:squareOfAlephs}}
    \end{alignat*}
    Thus, the result follows.
}

\thm[zorn]{Zorn's Lemma}{
    \TFAE.
    \begin{itemize}[nolistsep, leftmargin=*]
        \ii
        \nameref{ax:choice}
        \ii
        Let \((A, \preceq)\) be a partially ordered set.
        If every \hyperref[dfn:chain]{chain} in \((A, \preceq)\) has an upper bound,
        then there exists a \hyperref[dfn:leastMinimal]{maximal} element of \(A\).
    \end{itemize}
}
\mclm{Proof}{\hfill
\begin{itemize}[nolistsep, wide=0pt, widest={(\(\Rightarrow\))}, leftmargin=*, listparindent=\parindent]
    \ii[(\(\Rightarrow\))]
    Let \((A, \preceq)\) be a partially ordered set
    such that every chain in \((A, \preceq)\) has an upper bound.
    Fix any \(b \in A\) and a choice function \(g\) for \(\mcal P(A)\).
    Let \(\mbf{G}(x)\) be an operation defined by
    \[
            \mbf{G}(x) \triangleq \begin{cases}
                g(A_{x}) & \text{if}~\begin{aligned}[t]
                    &x~\text{is a transfinite sequence of length}~\alpha \\
                    &\text{and}~A_{x} \triangleq \{\,a \in A \mid \fall \xi < \alpha,\: x(\xi) \prec a\,\}~
                    \text{is nonempty}
                \end{aligned} \\
                b & \text{otherwise.}
            \end{cases}
        \]
        Then, by \nameref{th:transRecursion},
        there exists an operation \(\mbf{F}(x)\) such that
        \(\fall \alpha \in \Ord,\: \mbf{F}(\alpha) = \mbf{G}\left(\restr{\mbf{F}}{\alpha}\right)\).
        Similarly to the discussion in \Cref{clm:FtsPjUKS} of \nameref{th:wellOrdering},
        \(\mbf{F}(\alpha) = b\) for some \(\alpha < h(A)\).
        Let \(\lambda \triangleq \min \{\,\alpha \in h(A) \mid \mbf{F}(\alpha) = b\,\}\).
        Then, the set \(C \triangleq \mbf{F}[\lambda]\) is a chain in \((A, \preceq)\);
        so it has an upper bound \(c \in A\).

        If \(c \prec a\) for some \(a \in A\),
        then \(a \in A_{\mbf{F} |_\lambda}\);
        thus \(\mbf{F}(\lambda) = g\left(A_{\mbf{F} |_\lambda}\right) \neq b\),
        which is a contradiction.
        Hence, \(c\) is a maximal element of \(A\).

        \[
        \mbf{G}(x) \triangleq \begin{cases}
            g(A_{x}) & \text{if}~\begin{aligned}[t]
                &x~\text{is a transfinite sequence of length}~\alpha \\
                &\text{and}~A_{x} \triangleq \{\,a \in A \mid \fall \xi < \alpha,\: x(\xi) \prec a\,\}~
                \text{is nonempty}
            \end{aligned} \\
            b & \text{otherwise.}
        \end{cases}
    \]
    Then, by \nameref{th:transRecursion},
    there exists an operation \(\mbf{F}(x)\) such that
    \(\fall \alpha \in \Ord,\: \mbf{F}(\alpha) = \mbf{G}\left(\restr{\mbf{F}}{\alpha}\right)\).
    Similarly to the discussion in \Cref{clm:FtsPjUKS} of \nameref{th:wellOrdering},
    \(\mbf{F}(\alpha) = b\) for some \(\alpha < h(A)\).
    Let \(\lambda \triangleq \min \{\,\alpha \in h(A) \mid \mbf{F}(\alpha) = b\,\}\).
    Then, the set \(C \triangleq \mbf{F}[\lambda]\) is a chain in \((A, \preceq)\);
    so it has an upper bound \(c \in A\).

    If \(c \prec a\) for some \(a \in A\),
    then \(a \in A_{\mbf{F} |_\lambda}\);
    thus \(\mbf{F}(\lambda) = g\left(A_{\mbf{F} |_\lambda}\right) \neq b\),
    which is a contradiction.
    Hence, \(c\) is a maximal element of \(A\).

    \ii[(\(\Leftarrow\))]
    It suffices to show that every system of nonempty sets \(S\)
    has some choice function.
    Let \(F \triangleq \{\,f \colon S \rightharpoonup \bigcup S \mid \fall X \in \dom f,\: f(X) \in X\,\}\).
    Then, \((F, \subseteq_F)\) is a partially ordered set.
    Moreover, if \(C \subseteq F\) is a chain in \((F, \subseteq_{F})\),
    then \(\bigcup C\) is an upper bound of \(C\).

    Therefore, by the assumption, there exists a maximal element \(\ol{f}\) of \(F\).
    Suppose \(\dom \ol{f} \subsetneq S\) for the sake of contradiction.
    Take any \(X \in S \setminus \dom \ol{f}\) and \(x \in X\).
    Then, \(\ol{f} \cup \{(X, x)\} \in F\) is clearly greater than \(\ol{f}\),
    which is a contradiction.
    Hence, \(\dom \ol{f} = S\), i.e., \(f\) is a choice function for \(S\).
    \qed
\end{itemize}
}

\thm[initialLessThanAlephThen]{}{
    Let \(\gamma \in \Ord\) and let \((A, \preceq)\) be a totally ordered set.
    If \(\fall x \in A,\: |\{\,y \in A \mid y \preceq x\,\}| < \aleph_\gamma\),
    then \(|A| \le \aleph_{\gamma}\).
    \needsChoice
}
\pf{Proof}{
    In the same way as in the proof of \nameref{th:zorn},
    we construct an operation \(\mbf{F}(x)\)
    and \(\lambda = \min \{\,\alpha < h(A) \mid \mbf{F}(\alpha) = b\,\}\).

    For any \(a \in A\), there exists \(\xi < \lambda\) such that \(a \preceq \mbf{F}(\xi)\).
    (Otherwise, \(\mbf{F}(\lambda) \neq b\) by definition.)
    Hence, \(A = \bigcup_{\xi < \lambda} \{\,y \in A \mid y \preceq \mbf{F}(\xi)\,\}\).
    Moreover, \(\lambda \le \omega_{\gamma}\) since,
    otherwise, \(\restr{\mbf{F}}{\omega_\gamma}\)
    is an injection on \(\omega_\gamma\) into \(\{\,y \in A \mid y \preceq \mbf{F}(\omega_\gamma)\,\}\),
    which is a contradiction.
    Hence, \(|A| \le \aleph_\gamma\) by \Cref{th:alephUnionOfAlephs}.
}

\end{document}
