\documentclass[../introduction_to_set_theory.tex]{subfiles}
\begin{document}

\section{Functions}

\dfn[function]{Function}{
    A binary relation \(F\) is called a \textit{function} (or \textit{mapping})
    if \[\fall a\: \fall b_1\: \fall b_2\: (aFb_1 \land aFb_2 \implies b_1 = b_2).\]

    For each \(a \in \dom F\), the unique \(b\) such that \(aFb\) is called the \textit{value of \(F\) at \(a\)}
    and is denoted \(F(a)\) of \(F_a\).
}

\notat{}{
    If \(F\) is a function with \(\dom F = A\) and \(\ran F \subseteq B\),
    we write \(F \colon A \to B\), \(\lang F(a) \mid a \in A\rang\),
    \(\lang F_a \mid a \in A\rang\), \(\lang F_a\rang_{a \in A}\) for the function \(F\).
    The range of the function \(F\) can then be denoted \(\{\,F(a) \mid a \in A\,\}\)
    or \(\{F_a\}_{a \in A}\).
}

\mlemma[functionEqualsIff]{}{
    Let \(F\) and \(G\) be functions.
    \(F = G \iff \dom F = \dom G \land \fall x \in \dom F,\: F(x) = G(x)\).
}
\pf{Proof}{
    (\(\Rightarrow\)) is direct.

    (\(\Leftarrow\))
    Take any \((x, F(x)) \in F\).
    Then, we have \((x, F(x)) = (x, G(x)) \in G\).
    Therefore, \(F \subseteq G\).
    Similarly, \(G \subseteq F\), and thus \(F = G\).
}

\dfn[]{}{
    Let \(F\) be a function and \(A\) and \(B\) be sets.
    \begin{itemize}[nolistsep]
        \ii \(F\) is a function \textit{on} \(A\) if \(\dom F = A\).
        \ii \(F\) is a function \textit{into} \(B\) if \(\ran F \subseteq B\).
        \ii \(F\) is a function \textit{onto} \(B\) if \(\ran F = B\).
        \ii The \textit{restriction} of the function \(F\) \textit{to} \(A\)
            is the function \(\restr{F}{A} \triangleq \{\,(a, b) \in F \mid a \in A\,\}\).
            If \(G\) is a restriction of \(F\) to some \(A\),
            we say that \(F\) is an \textit{extension} of \(G\).
    \end{itemize}
}

\thm[functionComposite]{}{
    Let \(f\) and \(g\) be functions.
    \begin{enumerate}[nolistsep, label=(\roman*)]
        \ii \(g \circ f\) is a function.
        \ii \(\dom (g \circ f) = (\dom f) \cap f\inv[\dom g]\).
        \ii \(\fall x \in \dom (g \circ f),\: (g \circ f)(x) = g(f(x))\).
    \end{enumerate}
}
\mclm{Proof}{\hfill
\begin{enumerate}[nolistsep, label=(\roman*)]
    \ii
    Suppose \(x(g \circ f)z_1\) and \(x(g \circ f)z_2\).
    There exists \(y_1\) and \(y_2\) such that \(xfy_1\), \(y_1gz_1\), \(xfy_2\), and \(y_2gz_2\).
    Since \(f\) and \(g\) are functions, we have \(y_1 = y_2\) and \(z_1 = z_2\).
    Therefore, \(g \circ f\) is a function.

    \ii
    \(\begin{aligned}[t]
        x \in \dom (g \circ f) &\iff \exs z\: x(g \circ f)z \\
                               &\iff \exs z\:\exs y\: x f y \land y g z \\
                               &\iff x \in \dom f \land f(x) \in \dom g
                               \iff x \in \dom f \land x \in f\inv[\dom g] \qed
    \end{aligned}\)
\end{enumerate}
}

\dfn[]{Invertible Function}{
    A function \(f\) is said to be \textit{invertible} if \(f\inv\) is a function.
}

\dfn[]{Injective Function}{
    A function \(f\) is said to be \textit{injective} (or \textit{one-to-one}) if
    \[
        \fall a_1, a_2 \in \dom f,\: (f(a_1) = f(a_2) \implies a_1 = a_2).
    \]
}

\notat{}{
    Let \(f\) be a function.
    \begin{itemize}[nolistsep, leftmargin=*]
        \ii If \(f\) is a function on \(A\) \textit{onto} \(B\), we may write \(f \colon A \twoheadrightarrow B\).
        \ii If \(f\) is an \textit{injective} function on \(A\) into \(B\), we may write \(f \colon A \hookrightarrow B\).
        \ii If \(f\) is an \textit{injective} function on \(A\) \text{onto} \(B\), we may write \(f \colon A \hooktwoheadrightarrow B\).
        \ii If \(f\) is a function on a \textit{subset} of \(A\) into \(B\), we may write \(f \colon A \rightharpoonup B\).
    \end{itemize}
}

\thm[invIffInj]{}{
    Let \(f\) be a function.
    \begin{enumerate}[nolistsep, label=(\roman*)]
        \ii \(f\) is invertible if and only if \(f\) is one-to-one.
        \ii If \(f\) is invertible, then \(f\inv\) is also invertible and \((f\inv)\inv = f\).
    \end{enumerate}
}
\mclm{Proof}{\hfill
\begin{enumerate}[nolistsep, label=(\roman*)]
    \ii
    (\(\Rightarrow\)) Suppose \(f\inv\) is a function. Then, \(f\inv(f(a)) = a\)
    for all \(a \in \dom f\).
    Hence, for all \(a_1, a_2 \in \dom f\) such that \(f(a_1) = f(a_2)\),
    it follows that \(a_1 = f\inv(f(a_1)) = f\inv(f(a_2)) = a_2\);
    \(f\) is one-to-one.

    (\(\Leftarrow\)) Suppose \(f\) is one-to-one.
    If \(yf\inv x_1\) and \(yf\inv x_2\), then \(x_1fy\) and \(x_2fy\),
    i.e., \(y = f(x_1) = f(x_2)\). Therefore, \(x_1 = x_2\); \(f\inv\) is a function.

    \ii
    As \(f\) is a relation, by \Cref{exer:2.2.4} (iii), \((f\inv)\inv = f\),
    and thus \(f\inv\) is invertible. \qed
\end{enumerate}
}

\dfn[compatible]{Compatible Functions}{
    \begin{itemize}[nolistsep, leftmargin=*]
        \ii Functions \(f\) and \(g\) are called \textit{compatible} if
            \(\fall x \in (\dom f) \cap (\dom g),\: f(x) = g(x)\).
        \ii A set of functions \(F\) is called a \textit{compatible system of functions}
            if any two functions \(f\) and \(g\) from \(F\) are compatible.
    \end{itemize}
}

\mlemma[compatibleIff]{}{
    Let \(f\) and \(g\) be functions.
    \begin{enumerate}[nolistsep, label=(\roman*)]
        \ii \(f\) and \(g\) are compatible if and only if \(f \cup g\) is a function.
        \ii \(f\) and \(g\) are compatible if and only if
            \(\restr{f}{(\dom f) \cap (\dom g)} = \restr{g}{(\dom f) \cap (\dom g)}\).
    \end{enumerate}
}
\mclm{Proof}{\hfill
\begin{enumerate}[nolistsep, label=(\roman*)]
    \ii
    (\(\Rightarrow\))
    Suppose \(x(f \cup g)y_1\) and \(x(f \cup g)y_2\).
    \WLOG, \((x, y_1) \in f\).
    If \((x, y_2) \in f\), since \(f\) is a function, \(y_1 = y_2\).
    If \((x, y_2) \in g\), since \(f\) and \(g\) are compatible, \(y_1 = f(x) = g(x) = y_2\).
    Therefore, \(f \cup g\) is a function.

    (\(\Leftarrow\))
    Take any \(x \in (\dom f) \cap (\dom g)\).
    \((x, f(x)) \in f \cup g\) and \((x, g(x)) \in f \cup g\).
    Since \(f \cup g\) is a function, we have \(f(x) = g(x)\).

    \ii
    Let \(A = (\dom f) \cap (\dom g)\).

    (\(\Rightarrow\))
    By definition, \(\dom \restr{f}{A} = \dom \restr{g}{A}
    = (\dom f) \cap (\dom g)\).
    Moreover, for all \(x \in (\dom f) \cap (\dom g)\),
    \(\restr{f}{A}(x) = f(x) = g(x) = \restr{g}{A}(x)\).
    Hence, the result follows by \Cref{lem:functionEqualsIff}.

    (\(\Leftarrow\))
    Take any \(x \in A\).
    Then, \(f(x) = \restr{f}{A}(x) = \restr{g}{A}(x) = g(x)\). \qed
\end{enumerate}
}

\thm[compatibleThenUnionIsFunction]{}{
    If \(F\) is a compatible system of functions, then \(\bigcup F\)
    is a function with \(\dom \bigcup F = \bigcup \{\,\dom f \mid f \in F\,\}\).
    The function \(\bigcup F\) extends all \(f \in F\).
}
\pf{Proof}{
    Note that \(\bigcup F\) is already a relation.
    If \((a, b_1), (a, b_2) \in \bigcup F\),
    then there exist \(f_1, f_2 \in F\) such that \((a, b_1) \in f_1\) and \((a, b_2) \in f_2\).
    Since \(f_1\) and \(f_2\) are compatible and \(a \in (\dom f_1) \cap (\dom f_2)\),
    we have \(b_1 = f_1(a) = f_2(a) = b_2\). Hence, \(\bigcup F\) is a function.

    \(\dom \bigcup F = \bigcup \{\,\dom f \mid f \in F\,\}\) since
    \[\begin{aligned}[b]
        x \in \dom \bigcup F &\iff \exs y,\:(x, y) \in \bigcup F \\
                             &\iff \exs y,\: \exs f \in F,\: (x, y) \in f \\
                             &\iff \exs f \in F,\: x \in \dom f \iff x \in \bigcup \{\,\dom f \mid f \in F\,\}.
    \end{aligned}\]

    Take any \(f \in F\).
    As \(f \cup \bigcup F = \bigcup F\), \(f\) and \(\bigcup F\) are compatible
    by \Cref{lem:compatibleIff} (i).
    Moreover, \(\dom f \cap \dom \bigcup F = \dom f\).
    Hence, by \Cref{lem:compatibleIff} (ii),
    \(f = \restr{f}{\dom f} = \restr{\left(\bigcup F\right)}{\dom f}\);
    \(\bigcup F\) extends each \(f \in F\).
}

\dfn[setOfFunctions]{}{
    Let \(A\) and \(B\) be sets.
    Then, we define
    \[
        B^A \triangleq \{\,f \mid f \text{ is a function on }A\text{ into }B\,\}.
    \]
}

\dfn[indexedSystemOfSets]{Indexed System of Sets}{
    \begin{itemize}[nolistsep, leftmargin=*]
        \ii
        Let \(S = \lang S_i \mid i \in I \rang\) be a function with domain \(I\).
        We call the function \(S\) an \textit{indexed system of sets}
        whenever we stress that the values of \(S\) are sets.

        \ii
        We say that a system of sets \(A\) is \textit{indexed} by \(S\)
        if \(A = \{\,S_i \mid i \in I\,\} = \ran S\).
    \end{itemize}
}

\notat{}{
    If \(A\) is indexed by \(S = \lang S_i \mid i \in I \rang\),
    we may write
    \[
        \bigcup \{\,S_i \mid i \in I\,\}\quad\text{or}\quad\bigcup_{i \in I} S_i
    \]
    instead of \(\bigcup A\).
    Similarly, we may write \(\bigcap \{\,S_i \mid i \in I\,\}\) or \(\bigcap_{i \in I} S_i\)
    instead of \(\bigcap A\).
}

\dfn[productIndexedSystemOfSets]{Product of Indexed System of Sets}{
    Let \(S = \lang S_i \mid i \in I \rang\) be an indexed system of sets.
    We call the set
    \[
        \prod S \triangleq \{\,f \mid f \text{ is a function on }I\text{ and } \fall i \in I,\: f_i \in S_i\,\}
    \]
    the \textit{product} of the indexed system \(S\).
}

\notat{}{
    Other notations for the product of the indexed system \(S = \lang S_i \mid i \in I \rang\)
    are:
    \[
        \prod \lang S(i) \mid i \in I \rang, \quad
        \prod_{i \in I} S(i), \quad
        \prod_{i \in I} S_i.
    \]
}

\nt{
    \noindent
    The existence of \(B^A\) and \(\prod_{i \in I} S_i\) is proved in \Cref{exer:2.3.9}.
}

\nt{
    \noindent
    If \(A = S_i\) for all \(i \in I\),
    \(\prod_{i \in I} S_i = A^I\).
}

\subfile{../exercises/exercise2-3.tex}

\end{document}
