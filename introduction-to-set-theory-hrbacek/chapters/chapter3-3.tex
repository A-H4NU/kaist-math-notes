\documentclass[../introduction_to_set_theory.tex]{subfiles}

\begin{document}

\section{Arithmetic of Natural Numbers}

\thm[additionExists]{}{
    There uniquely exists a function \(\mathord{+} \colon \NN \times \NN \to \NN\) such that
    \begin{enumerate}[nolistsep, label=(\roman*))]
        \ii \(\fall m \in \NN,\: \mathord{+}(m, 0) = m\)
        \ii \(\fall m, n \in \NN,\: \mathord{+}(m, n+1) = S(\mathord{+}(m, n))\).
    \end{enumerate}
}
\pf{Proof}{
    The result directly follows from exploiting
    \nameref{th:parametricRecursion} with \(A = P = \NN\), \(a(p) = p\) for all \(p \in \NN\),
    and \(g(p, x, n) = S(x)\) for all \(p, x, n \in \NN\).
}

\dfn[addition]{Addition}{
    The function \(+\) defined in \Cref{th:additionExists}
    is called the \textit{addition}.
}

\notat[addition]{}{
    For all \(m \in \NN\), we have
    \(\mathord{+}(m, 1) = \mathord{+}(m, 0+1) = \mathord{+}(m, 0) + 1 = m + 1\).
    Hence, we may write \(m + n\) instead of \(\mathord{+}(m, n)\)
    without causing any confusion regarding \Cref{not:nPlusOne}.
    We restate the defining properties of the addition for future reference:
    \begin{equation}
        \fall m \in \NN,\: m + 0 = m \label{eq:addition1}\tag{\(1\)}
    \end{equation}
    \begin{equation}
        \fall m, n \in \NN,\: m + (n + 1) = (m + n) + 1 \label{eq:addition2}\tag{\(2\)}
    \end{equation}
}

\thm[addIsCommutative]{\(+\) is Commutative}{
    Addition is commutative; that is to say
    \[
        \fall m, n \in \NN,\: m + n = n + m.
    \]
}
\pf{Proof}{
    Let \(\mbf{P}(x)\) be the property ``\(\fall m \in \NN,\: m + x = x + m\).''
    \clm[e3d019c6]{
        \(\mbf{P}(0)\) holds.
    }{
        Since \(m + 0 = m\) already, we only need to prove \(0 + m = m\) for all \(m \in \NN\).
        We shall make use of induction.
        First of all \(0 + 0 = 0\) holds by \eqref{eq:addition1}.

        Suppose \(0 + m = m\) where \(m \in \NN\). Then,
        \begin{alignat*}{2}
            0 + (m + 1) &= (0 + m) + 1 &\qquad& \comment*{\eqref{eq:addition2}} \\
                        &= m + 1. &\qquad& \comment*{\(0 + m = m\)}
        \end{alignat*}
        Hence, by \nameref{th:induction}, \(0 + m = m\) for all \(m \in \NN\). \qed
    }

    \clm[688efe85]{
        \(\fall n \in \NN,\: [\mbf{P}(n) \implies \mbf{P}(n+1)]\)
    }{
        Assume \(\mbf{P}(n)\).
        We shall show \(\mbf{P}(n+1)\) holds by induction.
        \(0 + (n+1) = (n+1) + 0\) is already shown by \Cref{clm:e3d019c6}.
        Hence, assume \(m + (n+1) = (n+1) + m\) for fixed \(m \in \NN\).
        Then,
        \begin{alignat*}{2}
            (m+1) + (n+1) &= ((m+1) + n) + 1 &\qquad& \comment*{\eqref{eq:addition2}}\\
                          &= (n + (m + 1)) + 1 &\qquad& \comment*{\(\mbf{P}(n)\)}\\
                          &= ((n + m) + 1) + 1 &\qquad& \comment*{\eqref{eq:addition2}}\\
                          &= ((m + n) + 1) + 1 &\qquad& \comment*{\(\mbf{P}(n)\)}\\
                          &= (m + (n + 1)) + 1 &\qquad& \comment*{\eqref{eq:addition2}}\\
                          &= ((n + 1) + m) + 1 &\qquad& \comment*{\(m + (n+1) = (n+1) + m\)}\\ 
                          &= (n + 1) + (m + 1). &\qquad& \comment*{\eqref{eq:addition2}}
        \end{alignat*}
        Hence, by \nameref{th:induction}, \(\mbf{P}(n+1)\) holds. \qed
    }

    \noindent
    From \Cref{clm:e3d019c6}, \Cref{clm:688efe85}, and \nameref{th:induction},
    we get \(\fall m, n \in \NN,\: m + n = n + m\).
}

\thm[addIsAssociative]{\(+\) is Associative}{
    Addition is associative; that is to say
    \[
        \fall k, m, n \in \NN,\: (k + m) + n = k + (m + n).
    \]
}
\pf{Proof}{
    Let \(\mbf{P}(x)\) be the property ``\(\fall k, m \in \NN,\: (k + m) + x = k + (m + x)\).''
    \(\mbf{P}(0)\) is direct by \eqref{eq:addition1}.

    Now, fix any \(n \in \NN\) and assume \(\mbf{P}(n)\).
    Then, for all \(k, m \in \NN\),
    \begin{alignat*}{2}
        (k + m) + (n + 1) &= ((k + m) + n) + 1 &\qquad& \comment*{\eqref{eq:addition2}}\\
                          &= (k + (m + n)) + 1 &\qquad& \comment*{\(\mbf{P}(n)\)}\\
                          &= k + ((m + n) + 1) &\qquad& \comment*{\eqref{eq:addition2}}\\
                          &= k + (m + (n + 1)). &\qquad& \comment*{\eqref{eq:addition2}}
    \end{alignat*}
    Hence, by \nameref{th:induction}, the result follows.
}

\thm[multiplicationExists]{}{
    There uniquely exists a function \(\mathord{\cdot} \colon \NN \times \NN \to \NN\) such that
    \begin{enumerate}[nolistsep, label=(\roman*)]
        \ii \(\fall m \in \NN,\: m \cdot 0 = 0\)
        \ii \(\fall m, n \in \NN,\: m \cdot (n + 1) = m \cdot n + m\).
    \end{enumerate}
}
\pf{Proof}{
    The result directly follows from exploiting
    \nameref{th:parametricRecursion} with \(A = P = \NN\), \(a(p) = 0\) for all \(p \in \NN\),
    and \(g(p, x, n) = x + p\) for all \(p, x, n \in \NN\).
}

\dfn[multiplication]{Multiplication}{
    The function \(\cdot\) defined in \Cref{th:multiplicationExists}
    is called the \textit{multiplication}.
    \begin{equation}
        \fall m \in \NN,\: m \cdot 0 = 0 \label{eq:multiplication1}\tag{\(3\)}
    \end{equation}
    \begin{equation}
        \fall m, n \in \NN,\: m \cdot (n+1) = m \cdot n + m \label{eq:multiplication2}\tag{\(4\)}
    \end{equation}
}

\thm[multIsCommutative]{\(\cdot\) is Commutative}{
    Multiplication is commutative, i.e., 
    \[\fall m, n \in \NN,\: m \cdot n = n \cdot m.\]
}
\pf{Proof}{
    Let \(\mbf{P}(x)\) be the property ``\(\fall m \in \NN,\: m \cdot x = x \cdot m\).''
    \clm[30d818bf]{
        \(\mbf{P}(0)\) holds.
    }{
        Since \(m \cdot 0 = 0\) already by \eqref{eq:multiplication1},
        we only need to prove \(0 \cdot m = 0\) for all \(m \in \NN\).
        We shall make use of induction.
        First of all \(0 \cdot 0 = 0\) holds by \eqref{eq:multiplication1}.

        Suppose \(0 \cdot m = 0\) where \(m \in \NN\). Then,
        \begin{alignat*}{2}
            0 \cdot (m + 1) &= 0 \cdot m + 0 &\qquad& \comment*{\eqref{eq:multiplication2}} \\
                        &= 0 + 0 &\qquad& \comment*{\(0 \cdot m = 0\)} \\
                        &= 0.
        \end{alignat*}
        Hence, by \nameref{th:induction}, \(0 \cdot m = 0\) for all \(m \in \NN\). \qed
    }

    \clm[11595ce5]{
        \(\fall n \in \NN,\: [\mbf{P}(n) \implies \mbf{P}(n + 1)]\)
    }{
        Fix any \(n \in \NN\) and assume \(\mbf{P}(n)\).
        We shall prove \(\mbf{P}(n + 1)\) by induction.
        We already have \(0 \cdot (n + 1) = (n + 1) \cdot 0\) by \Cref{clm:30d818bf}.

        Fix any \(m \in \NN\) and assume \(m \cdot (n + 1) = (n + 1) \cdot m\).
        Then,
        \begin{alignat*}{2}
            (m + 1) \cdot (n + 1) &= (m + 1) \cdot n + (m + 1) &\qquad& \comment*{\eqref{eq:multiplication2}}\\
                                  &= n \cdot (m + 1) + (m + 1) && \comment*{\(\mbf{P}(n)\)}\\
                                  &= (n \cdot m + n) + (m + 1) && \comment*{\eqref{eq:multiplication2}}\\
                                  &= (m \cdot n + n) + (m + 1) && \comment*{\(\mbf{P}(n)\)}\\
                                  &= (m \cdot n + m) + (n + 1) && \comment*{\nameref{th:addIsCommutative}, \nameref{th:addIsAssociative}}\\
                                  &= m \cdot (n + 1) + (n + 1) && \comment*{\eqref{eq:multiplication2}}\\
                                  &= (n + 1) \cdot m + (n + 1) && \comment*{\(m \cdot (n + 1) = (n + 1) \cdot m\)}\\
                                  &= (n + 1) \cdot (m + 1). && \comment*{\eqref{eq:multiplication2}}\\
        \end{alignat*}
        Hence, by \nameref{th:induction}, \(\mbf{P}(n + 1)\) holds.
    }

    \noindent
    From \Cref{clm:30d818bf}, \Cref{clm:11595ce5}, and \nameref{th:induction},
    we get \(\fall m, n \in \NN,\: m \cdot n = n \cdot m\).
}

\thm[multDistrOverAdd]{\(\cdot\) Distributes Over \(+\)}{
    Multiplication is distributive over addition, i.e.,
    \begin{alignat*}{1}
        & \fall k, m, n \in \NN,\: k \cdot (m + n) = k \cdot m + k \cdot n \quad\text{and}\\
        & \fall k, m, n \in \NN,\: (m + n) \cdot k = m \cdot k + n \cdot k.
    \end{alignat*}
}
\pf{Proof}{
    Let \(\mbf{P}(x)\) be the property ``\(\fall k, m \in \NN,\: k \cdot (m + x) = k \cdot m + k \cdot x\).''
    \(\mbf{P}(0)\) holds by \eqref{eq:addition1} and \eqref{eq:multiplication1}.
    
    Fix any \(n \in \NN\) and assume \(\mbf{P}(n)\).
    Then, for each \(k, m \in \NN\),
    \begin{alignat*}{2}
        k \cdot (m + (n + 1)) &= k \cdot ((m + n) + 1) &\qquad& \comment*{\nameref{th:addIsAssociative}}\\
                              &= k \cdot (m + n) + k && \comment*{\eqref{eq:multiplication2}}\\
                              &= (k \cdot m + k \cdot n) + k && \comment*{\(\mbf{P}(n)\)}\\
                              &= k \cdot m + (k \cdot n + k) && \comment*{\nameref{th:addIsAssociative}}\\ 
                              &= k \cdot m + k \cdot (n + 1). && \comment*{\eqref{eq:multiplication2}}
    \end{alignat*}
    Hence, by \nameref{th:induction}, we have \(\fall k, m, n \in \NN,\: k \cdot (m + n) = k \cdot m + k \cdot n\).

    Now, we have, for each \(k, m, n \in \NN\),
    \begingroup\setlength{\belowdisplayskip}{0pt}
    \begin{alignat*}{2}
        (m + n) \cdot k &= k \cdot (m + n) &\qquad& \comment*{\nameref{th:multIsCommutative}}\\
                        &= k \cdot m + k \cdot n \\
                        &= m \cdot k + n \cdot k. && \comment*{\nameref{th:multIsCommutative}}
    \end{alignat*}
    \endgroup
}

\thm[multIsAssociative]{\(\cdot\) is Associative}{
    Multiplication is associative, i.e.,
    \[
        \fall k, m, n \in \NN,\: (k \cdot m) \cdot n = k \cdot (m \cdot n).
    \]
}
\pf{Proof}{
    Let \(\mbf{P}(x)\) be the property ``\(\fall k, m \in \NN,\: (k \cdot m) \cdot x = k \cdot (m \cdot x)\).''
    \(\mbf{P}(0)\) is direct from \eqref{eq:multiplication1}.

    Fix any \(n \in \NN\) and assume \(\mbf{P}(n)\). Then, for each \(k, m \in \NN\),
    \begin{alignat*}{2}
        (k \cdot m) \cdot (n + 1)
          &= (k \cdot m) \cdot n + k \cdot m &\qquad& \comment*{\eqref{eq:multiplication2}} \\
          &= k \cdot (m \cdot n) + k \cdot m && \comment*{\(\mbf{P}(n)\)} \\
          &= k \cdot (m \cdot n + m) && \comment*{\nameref{th:multDistrOverAdd}} \\ 
          &= k \cdot (m \cdot (n + 1)). && \comment*{\eqref{eq:multiplication2}}
    \end{alignat*}
    Hence, the result follows by \nameref{th:induction}.
}

\mlemma[timesOne]{}{
    \(\fall m \in \NN,\: m \cdot 1 = m\)
}
\pf{Proof}{
    \begin{alignat*}{2}
        m \cdot 1 &= m \cdot (0 + 1) &\qquad& \comment*{\eqref{eq:addition1}, \nameref{th:addIsCommutative}}\\
                  &= m \cdot 0 + m && \comment*{\eqref{eq:multiplication2}}\\
                  &= 0 + m && \comment*{\eqref{eq:multiplication1}}\\
                  &= m && \comment*{\eqref{eq:addition1}, \nameref{th:addIsCommutative}}
    \end{alignat*}
}

\thm[exponentiationExists]{}{
    There uniquely exists a function \(\mathord{\uparrow} \colon \NN \times \NN \to \NN\) such that
    \begin{enumerate}[nolistsep, label=(\roman*), ref=\protect{(\roman*)}, listparindent=\parindent]
        \ii \(\fall m \in \NN,\: m \uparrow 0 = 1\)
        \ii \(\fall m, n \in \NN,\: m \uparrow (n + 1) = (m \uparrow n) \cdot m\)
    \end{enumerate}
}
\pf{Proof}{
    The result directly follows from exploiting
    \nameref{th:parametricRecursion} with \(A = P = \NN\), \(a(p) = 1\) for all \(p \in \NN\),
    and \(g(p, x, n) = x \cdot p\) for all \(p, x, n \in \NN\).
}

\dfn[exponentiation]{Exponentiation}{
    The function \(\uparrow\) defined in \Cref{th:exponentiationExists}
    is called the \textit{exponentiation}.
    We write \(m^n\) instead of \(m \uparrow n\).
    \begin{equation}
        \fall m \in \NN,\: m^0 = 1 \label{eq:exponentiation1}\tag{\(5\)}
    \end{equation}
    \begin{equation}
        \fall m, n \in \NN,\: m^{n+1} = m^n \cdot m \label{eq:exponentiation2}\tag{\(6\)}
    \end{equation}
}

\thm[lawsOfExponents]{Laws of Exponents}{
\begin{enumerate}[nolistsep, label=(\roman*), ref=\protect{\nameref{th:lawsOfExponents} (\roman*)}, listparindent=\parindent]
    \ii \(\fall m \in \NN,\: m^1 = m\)
    \ii\label{itm:lawsOfExponents.ii} \(\fall k, m, n \in \NN,\: k^{m+n} = k^m \cdot k^n\)
    \ii \(\fall k, m, n \in \NN, (m \cdot n)^k = m^k \cdot n^k\)
    \ii \(\fall k, m, n \in \NN,\: (k^m)^n = k^{m \cdot n}\)
\end{enumerate}
}
\mclm{Proof}{\hfill
\begin{enumerate}[nolistsep, label=(\roman*), leftmargin=*, listparindent=\parindent]
    \ii Take any \(m \in \NN\). Then,
    \begin{alignat*}{2}
        m^1 &= m^{0 + 1} &\qquad& \comment*{\eqref{eq:addition1}, \nameref{th:addIsCommutative}}\\
            &= m^0 \cdot m && \comment*{\eqref{eq:exponentiation2}}\\
            &= 1 \cdot m && \comment*{\eqref{eq:exponentiation1}}\\
            &= m. && \comment*{\nameref{th:multIsCommutative}, \Cref{lem:timesOne}}
    \end{alignat*}

    \ii
    Let \(\mbf{P}(x)\) be the property ``\(\fall k, m \in \NN, k^{m+x} = k^m \cdot k^x\).''
    \(\mbf{P}(0)\) holds since, for each \(k, m \in \NN\),
    \begin{alignat*}{2}
        k^{m + 0} &= k^m &\qquad& \comment*{\eqref{eq:addition1}}\\
                  &= k^m \cdot 1 && \comment*{\Cref{lem:timesOne}}\\
                  &= k^m \cdot k^0. && \comment*{\eqref{eq:exponentiation1}}
    \end{alignat*}
    Now, fix \(n \in \NN\) and assume \(\mbf{P}(n)\). Then,
    \begin{alignat*}{2}
        k^{m + (n + 1)} &= k^{(m + n) + 1} &\qquad& \comment*{\nameref{th:addIsAssociative}}\\
                        &= k^{m+n} \cdot k && \comment*{\eqref{eq:exponentiation2}}\\
                        &= (k^m \cdot k^n) \cdot k && \comment*{\(\mbf{P}(x)\)}\\
                        &= k^m \cdot (k^n \cdot k) && \comment*{\nameref{th:multIsAssociative}}\\
                        &= k^m \cdot k^{n+1}. && \comment*{\eqref{eq:exponentiation2}}
    \end{alignat*}
    Therefore, by \nameref{th:induction}, the result follows.

    \ii
    Let \(\mbf{P}(x)\) be the property ``\(\fall m, n \in \NN,\: (m \cdot n)^x = m^x \cdot n^x\).''
    \(\mbf{P}(0)\) holds since, for each \(m, n \in \NN\),
    \begin{alignat*}{2}
        (m \cdot n)^0 &= 1 &\qquad& \comment*{\eqref{eq:exponentiation1}}\\
                      &= 1 \cdot 1 && \comment*{\Cref{lem:timesOne}}\\
                      &= m^0 \cdot n^0. &&\comment*{\eqref{eq:exponentiation1}}
    \end{alignat*}
    Now, fix \(k \in \NN\) and assume \(\mbf{P}(k)\). Then,
    \begin{alignat*}{2}
        (m \cdot n)^{k+1} &= (m \cdot n)^k \cdot (m \cdot n) &\qquad& \comment*{\eqref{eq:exponentiation2}}\\
                          &= (m^k \cdot n^k) \cdot (m \cdot n) && \comment*{\(\mbf{P}(k)\)}\\
                          &= (m^k \cdot m) \cdot (n^k \cdot n) && \comment*{\nameref{th:multIsCommutative}, \nameref{th:multIsAssociative}}\\
                          &= m^{k+1} \cdot n^{k+1}. && \comment*{\eqref{eq:exponentiation2}}
    \end{alignat*}
    Therefore, by \nameref{th:induction}, the result follows.

    \ii
    Let \(\mbf{P}(x)\) be the property ``\(\fall k, m \in \NN,\: (k^m)^x = k^{m \cdot x}\).''
    \(\mbf{P}(0)\) holds since, for each \(k, m \in \NN\),
    \begin{alignat*}{2}
        (k^m)^0 &= 1 &\qquad& \comment*{\eqref{eq:exponentiation1}}\\
                &= k^0 && \comment*{\eqref{eq:exponentiation1}}\\
                &= k^{m \cdot 0}. && \comment*{\eqref{eq:multiplication1}}
    \end{alignat*}
    Now, fix \(n \in \NN\) and assume \(\mbf{P}(n)\). Then,
    \begin{alignat*}{2}
        (k^m)^{n+1} &= (k^m)^n \cdot k^m &\qquad& \comment*{\eqref{eq:exponentiation2}}\\
                    &= k^{m \cdot n} \cdot k^m && \comment*{\(\mbf{P}(n)\)}\\
                    &= k^{m \cdot n + m} && \comment*{\ref{itm:lawsOfExponents.ii}}\\
                    &= k^{m \cdot (n+1)}. && \comment*{\eqref{eq:multiplication2}}
    \end{alignat*}
    Therefore, by \nameref{th:induction}, the result follows.
    \qed
\end{enumerate}
}

\thm[summationExists]{}{
    There uniquely exists \(\Sigma \colon \Seq(\NN) \to \NN\) such that
    \begin{enumerate}[nolistsep, label=(\roman*), ref=\protect{(\roman*)}, listparindent=\parindent]
        \ii \(\Sigma (\lang\rang) = 0\).
        \ii \(\Sigma (k) = \Sigma \left(\restr{k}{n}\right) + k_n\)for all \(k \in \Seq(\NN)\)
            with length \(n + 1\).
    \end{enumerate}
}
\pf{Proof}{
    Let \(g \colon \Seq(\NN) \times \NN \times \NN\) be defined by
    \begin{equation*}
        g(k, s, n) = \begin{cases}
            s + k_n & \text{if } n \in \dom k \\
            s & \text{otherwise}.
        \end{cases}
    \end{equation*}
    Then, by \nameref{th:parametricRecursion},
    there exists a function \(f \colon \Seq(\NN) \times \NN \to \NN\)
    such that
    \begin{enumerate}[nolistsep, label=(\roman*), ref=\protect{(\roman*)}, listparindent=\parindent]
        \ii
        \(\fall k \in \Seq(\NN), f(k, 0) = 0\)
        \ii
        \(\fall n \in \NN,\: \fall k \in \Seq(\NN), f(k, n+1) = g(k, f(k, n), n) = \begin{cases}
            f(k, n) + k_n & \text{if } n \in \dom k \\
            f(k, n) & \text{otherwise.}
        \end{cases}\)
        \hfill [\(\ast\)]\label{eq:aruAgjmV}\vspace*{.3em}
    \end{enumerate}
    Now, define \(\Sigma \colon \Seq(\NN) \to \NN\) by \(\Sigma (k) = f(k, \dom k)\).
    (i) evidently holds.

    \clm[IktFGrmo]{
        Let \(k, \ell \in \Seq(\NN)\).
        If \(k \subseteq \ell\), then \(f(k, \dom k) = f(\ell, \dom k)\).
    }{
        Let \(\mbf{P}(x)\) be the property 
        \[
            \fall k, \ell \in \Seq(\NN),\:
            [\dom k = x \land k \subseteq \ell \implies f(k, x) = f(\ell, x)].
        \]
        \(\mbf{P}(0)\) is evident.
        Now, fix \(n \in \NN\) and assume \(\mbf{P}(n)\).

        Fix any \(k \in \Seq(\NN)\) with \(\dom k = n + 1\).
        Then, for any \(\ell \in \Seq(\NN)\) with \(k \subseteq \ell\),
        \begin{alignat*}{2}
            f(\ell, n + 1) &= f(\ell, n) + \ell_n &\qquad& \comment*{[\hyperref[eq:aruAgjmV]{\(\ast\)}]}\\
                       &= f\left(\restr{\ell}{n}, n\right) + \ell_n && \comment*{\(\mbf{P}(n)\)}\\
                       &= f\left(\restr{k}{n}, n\right) + k_n && \comment*{\(k \subseteq \ell\)}\\
                       &= f(k, n) + k_n && \comment*{\(\mbf{P}(n)\)}\\
                       &= f(k, n + 1). && \comment*{[\hyperref[eq:aruAgjmV]{\(\ast\)}]}
        \end{alignat*}
        Hence, by \nameref{th:induction}, the result follows. \qed
    }
    
    Let \(k \in \Seq(\NN)\) with length \(n + 1\).
    Then, \(\Sigma(k) = f(k, n+1) = f(k, n) + k_n\).
    \begin{alignat*}{2}
        \Sigma(k) &= f(k, n+1) \\
                  &= f(k, n) + k_n &\qquad& \comment*{[\hyperref[eq:aruAgjmV]{\(\ast\)}]}\\
                  &= f\left(\restr{k}{n}, n\right) + k_n && \comment*{\Cref{clm:IktFGrmo}} \\
                  &= \Sigma \left(\restr{k}{n}\right) + k_n.
    \end{alignat*}
    The uniqueness easily follows.
}

\notat{Summation}{
    For the function \(\Sigma\) defined in \Cref{th:summationExists},
    we write
    \[
        \sum_{0 \le i < n} k_i \quad\text{or}\quad \sum_{i=0}^{n-1} k_i
    \]
    instead of \(\Sigma(\lang k_0, \cdots, k_{n-1}\rang)\).
}

% \thm[lawsOfSummation]{}{
%     
% }

\subfile{../exercises/exercise3-3.tex}

\end{document}
