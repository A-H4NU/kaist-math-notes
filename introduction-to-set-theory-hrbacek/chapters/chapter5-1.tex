\documentclass[../introduction_to_set_theory_Note.tex]{subfiles}

\begin{document}

\section{Cardinal Arithmetic}

\dfn[]{Sum and Product of Two Cardinals}{
    Let \(|A| = \kappa\) and \(|B| = \lambda\).
    \begin{itemize}[nolistsep, leftmargin=*, listparindent=\parindent]
        \ii We write \(|A \cup B| = \kappa + \lambda\) if \(A \cap B\).
        \ii We write \(|A \times B| = \kappa \cdot \lambda\).
    \end{itemize}
    These are justified by \ref{itm:4.3.1.i} and \hyperref[itm:4.3.1.ii]{(ii)}.
}

\mlemma[rGWFJiTo]{}{
    If \(|A_1| = |A_2|\) and \(|B_1| = |B_2|\),
    then \(\big|A_1^{B_1}\big| = \big|A_2^{B_2}\big|\).
}
\pf{Proof}{
    Let \(f \colon A_1 \hooktwoheadrightarrow A_2\) and
    \(g \colon B_1 \hooktwoheadrightarrow B_2\).
    Define \(F \colon A_1^{B_1} \to A_2^{B_2}\) by
    \(k \mapsto f \circ k \circ g\inv\).
    Then, \(F\) is one-to-one and onto.
    \[
        \begin{tikzcd}[ampersand replacement=\&]
            B_1 \rar["g"] \dar["k"] \& B_2 \dar["F(k) = f \circ k \circ g\inv"] \\
            A_1 \rar["f"] \& A_2
        \end{tikzcd}
    \]
}

\dfn[]{Exponentiation of Two Cardinals}{
    Let \(|A| = \kappa\) and \(|B| = \lambda\).
    We write \(\kappa^\lambda =  \big|A^B\big|\).
    This is justified by \Cref{lem:rGWFJiTo}.
}

\nt{
    \noindent
    Here are some direct facts regarding sum, product, and exponentiation of cardinal numbers.
    \begin{enumerate}[nolistsep, label=(\roman*), ref=\protect{(\roman*)}, listparindent=\parindent]
        \ii \(\kappa + \lambda = \lambda + \kappa\).
        \ii \(\kappa + (\lambda + \mu) = (\lambda + \kappa) + \mu\).
        \ii \(\kappa \le \kappa + \lambda\).
        \ii If \(\kappa_1 \le \kappa_2\) and \(\lambda_1 \le \lambda_2\),
            then \(\kappa_1 + \lambda_1 \le \kappa_2 + \lambda_2\).
        \ii \(\kappa \cdot \lambda = \lambda \cdot \kappa\).
        \ii \(\kappa \cdot (\lambda \cdot \mu) = (\lambda \cdot \kappa) \cdot \mu\).
        \ii \(\kappa \cdot (\lambda + \mu) = \kappa \cdot \lambda + \kappa \cdot \mu\).
        \ii \(\kappa \le \kappa \cdot \lambda\) if \(\lambda > 0\).
        \ii If \(\kappa_1 \le \kappa_2\) and \(\lambda_1 \le \lambda_2\),
            then \(\kappa_1 \cdot \lambda_1 \le \kappa_2 \cdot \lambda_2\).
        \ii \(\kappa + \kappa = 2 \cdot \kappa\).
        \ii \(\kappa \le \kappa^\lambda\) if \(\lambda > 0\).
        \ii \(\lambda \le \kappa^\lambda\) if \(\kappa > 1\).
        \ii If \(\kappa_1 \le \kappa_2\) and \(\lambda_1 \le \lambda_2\),
            then \(\kappa_1^{\lambda_1} \le \kappa_2^{\lambda_2}\).
        \ii \(\kappa \cdot \kappa = \kappa^2\).
    \end{enumerate}
}

\thm[cardExp]{}{
    Let \(\kappa, \lambda, \mu\) be cardinal numbers.
    \begin{enumerate}[nolistsep, label=(\roman*), ref=\protect{\Cref{th:cardExp} (\roman*)}]
        \ii \(\kappa^{\lambda + \mu} = \kappa^{\lambda} \cdot \kappa^{\mu}\)
        \ii\label{itm:cardExp.ii} \((\kappa^\lambda)^\mu = \kappa^{\lambda \cdot \mu}\)
        \ii \((\kappa \cdot \lambda)^\mu = \kappa^\mu \cdot \lambda^\mu\)
    \end{enumerate}
}
\mclm{Proof}{
    Let \(\kappa = |K|\), \(\lambda = |L|\), and \(\mu = |M|\).
    \begin{enumerate}[nolistsep, label=(\roman*), listparindent=\parindent]
        \ii
        Assume \(L \cap N = \OO\).
        Then, we may define \(F \colon K^L \times K^M \hooktwoheadrightarrow K^{L \cup M}\)
        by \((f, g) \mapsto f \cup g\). \vspace*{.2em}

        \ii
        Define \(F \colon (K^L)^M \hooktwoheadrightarrow K^{L \times M}\) by
        \(f \mapsto \big\{\,\big((\ell, m), f_m(\ell)\big)
        \:\big|\: m \in M,\: \ell \in L\,\big\}\). \vspace*{.2em}

        \ii
        Define \(F \colon K^M \times L^M \hooktwoheadrightarrow (K \times L)^M\)
        by \((f_1, f_2) \mapsto \big\{\,\big(m, (f_1(m), f_2(m))\big) \:\big|\: m \in M\,\big\}\).
        \qed
    \end{enumerate}
}

\thm[cantor]{Cantor's Theorem}{
    \(\fall X,\: |X| < |\mcal P(X)|\)
}
\pf{Proof}{
    The function \(f \colon X \to \mcal P(X)\) defined by \(f(x) = \{x\}\)
    is injective, and we have \(|X| \le |\mcal P(X)|\).

    We now show that there exists no function on \(X\) onto \(\mcal P(X)\).
    Let \(f \colon X \to \mcal P(X)\) be arbitrary.
    Define \(S \triangleq \{\,x \in X \mid x \notin f(x)\,\}\)
    and suppose there exists \(z \in X\) such that \(f(z) = S\) for the sake of contradiction.
    Then, we have \(z \in S\) if and only if \(z \notin f(z) = S\).
    Therefore, \(f\) is not onto \(\mcal P(X)\).
}

\thm[cardPowSet]{}{
    \(\fall X,\: |\mcal P(X)| = 2^{|X|}\)
}
\pf{Proof}{
    For each \(S \subseteq \NN\), define the \textit{characteristic function} of \(S\),
    \(\chi_S \colon X \to \{0, 1\}\) by
    \[
        \chi_S(n) \triangleq \begin{cases}
            0 & \text{if } n \in S \\
            1 & \text{if } n \notin S.
        \end{cases}
    \]
    Then, we define \(F \colon \mcal P(X) \hooktwoheadrightarrow 2^X\)
    by \(S \mapsto \chi_S\).
}

\cor[]{}{
    \(\fall S,\:\exs Y,\:\fall X \in S,\: |X| < |Y|\)
}
\pf{Proof}{
    Take any \(S\).
    Then, let \(Y \triangleq \mcal P\left(\bigcup S\right)\).
    By \Cref{th:cantor} and \Cref{exer:4.1.3}, \(|Y| > \left|\bigcup S\right| \ge |X|\)
    for all \(X \in S\).
}

\subfile{../exercises/exercise5-1.tex}

\end{document}
