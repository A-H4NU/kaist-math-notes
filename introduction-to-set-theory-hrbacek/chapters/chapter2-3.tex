\documentclass[../introduction_to_set_theory.tex]{subfiles}

\begin{document}

\section{Equivalences and Partitions}

\dfn[equivalence]{Equivalence}{
    Let \(R\) be a binary relation in \(A\).
    \begin{itemize}[nolistsep, leftmargin=*]
        \ii \(R\) is called \textit{reflexive in} \(A\) if \(\fall a \in A,\: aRa\).
        \ii \(R\) is called \textit{symmetric in} \(A\) if \(\fall a, b \in A,\: (aRb \implies bRa)\).
        \ii \(R\) is called \textit{transitive in} \(A\) if \(\fall a, b, c \in A,\: (aRb \land bRc \implies aRc)\).
        \ii \(R\) is called an \textit{equivalence on} \(A\) if it is reflexive, symmetric, and transitive in \(A\).
    \end{itemize}
}

\dfn[equivalenceClass]{Equivalence Class}{
    Let \(E\) be an equivalence on \(A\) and let \(a \in A\).
    The \textit{equivalence class of \(a\) modulo} \(E\) is the set
    \[
        [a]_E \triangleq \{\,x \in A \mid xEa\,\}.
    \]
}

\mlemma[equivIffSameClass]{}{
    Let \(E\) be an equivalence on \(A\) and let \(a, b \in A\).
    \begin{enumerate}[nolistsep, label=(\roman*)]
        \ii \(aEb \iff [a]_E = [b]_E\)
        \ii \(\lnot (aEb) \iff [a]_E \cap [b]_E = \OO\)
    \end{enumerate}
}
\mclm{Proof}{\hfill
\begin{enumerate}[nolistsep, label=(\roman*)]
    \ii
    (\(\Rightarrow\))
    Suppose \(aEb\).
    Take any \(c \in [a]_E\). Then, \(cEa\) and \(aEb\), and thus \(cEb\)
    by transitivity. Hence, \(c \in [b]_E\); \([a]_E \subseteq [b]_E\).
    \([b]_E \subseteq [a]_E\) can be shown similarly since \(bEa\) holds as \(E\) is symmetric.

    (\(\Leftarrow\))
    Suppose \([a]_E = [b]_E\). Since \(aEa\) by reflexivity,
    we have \(a \in [a]_E = [b]_E\). Therefore, \(aEb\).

    \ii
    (\(\Rightarrow\))
    Suppose \([a]_E \cap [b]_E \neq \OO\).
    Then, there exists \(c \in [a]_E \cap [b]_E\), i.e., \(cEa\) and \(cEb\).
    Then, as \(E\) is symmetric, we have \(aEc\),
    and therefore \(aEb\) by transitivity.

    (\(\Leftarrow\))
    Suppose \(aEb\). Then, since \(aEa\) by reflexivity,
    we have \(a \in [a]_E\). We can see \(a \in [b]_E\) from (i).
    Hence, \([a]_E \cap [b]_E \neq \OO\).
    \qed
\end{enumerate}
}

\dfn[partition]{Partition}{
    A system \(S\) of nonempty sets is called a \textit{partition} of \(A\) if
    \begin{enumerate}[nolistsep, label=(\roman*)]
        \ii \(S\) is a system of mutually disjoint sets (\Cref{dfn:mutualDisjoint}) and
        \ii \(\bigcup S = A\).
    \end{enumerate}
}

\dfn[allEquivClasses]{System of All Equivalence Classes}{
    Let \(E\) be an equivalence on \(A\).
    The \textit{system of all equivalence classes} modulo \(E\)
    is the set
    \[
        A/E \triangleq \{\,[a]_E \mid a \in A\,\}.
    \]
}

\thm[equivDerivesPartition]{}{
    Let \(E\) be an equivalence on \(A\).
    Then, \(A/E\) is a partition of \(A\).
}
\pf{Proof}{
    If \([a]_E \neq [b]_E\), then by \Cref{lem:equivIffSameClass},
    we have \([a]_E \cap [b]_E = \OO\).
    Since \(E\) is reflexive, \(a \in [a]_E\); each \([a]_E\) is nonempty.
    Therefore, \(A/E\) is a system of mutually disjoint nonempty sets.

    Take any \(a \in A\).
    Since \(E\) is reflexive, \(a \in [a]_E \subseteq \bigcup A/E\).
    Therefore, \(A \subseteq \bigcup A/E\).
    Conversely, since \([a]_E \subseteq A\), we have \(\bigcup A/E \subseteq A\).
}

\dfn[equivalenceFromPartition]{}{
    Let \(S\) be a partition of \(A\). The relation \(E_S\) in \(A\) is defined by
    \[
        E_S \triangleq \{\,(a, b) \in A \times A \mid \exs C \in S,\: a \in C \land b \in C\,\}.
    \]
}

\thm[partitionDerivesEquiv]{}{
    Let \(S\) be a partition of \(A\).
    Then, \(E_S\) is a equivalence on \(A\).
}
\mclm{Proof}{\hfill
\begin{itemize}[nolistsep]
    \ii
    Take any \(a \in A\). As \(A = \bigcup S\), there exists \(C \in S\) such that \(a \in C\).
    Therefore, \(aE_Sa\). \(E_S\) is reflexive.

    \ii
    Assume \(aE_Sb\). Then, there exists \(C \in S\) such that \(a, b \in C\).
    Hence, \(b E_S a\). \(E_S\) is symmetric.

    \ii
    Assume \(aE_Sb\) and \(bE_Sc\).
    Then, there exist \(C, D \in S\) such that
    \(a, b \in C\) and \(b, c \in D\).
    Then, \(C \cap D \neq \OO\) as \(b\) belongs to both sets.
    Hence, \(C = D\), which implies \(aE_Sc\). \(E_S\) is transitive.
    \qed
\end{itemize}
}

\thm[equivAndPartAreSame]{}{
    \begin{enumerate}[nolistsep, label=(\roman*)]
        \ii If \(E\) is an equivalence on \(A\) and \(S = A/E\), then \(E_S = E\).
        \ii If \(S\) is a partition of \(A\), then \(A/E_S = S\).
    \end{enumerate}
}
\mclm{Proof}{\hfill
\begin{enumerate}[nolistsep, label=(\roman*)]
    \ii
    \(aE_Sb \underbrace{\iff}_{\text{\Cref{dfn:equivalenceFromPartition}}} \exs C \in S,\: a \in C \land b \in C
    \iff \exs c \in A,\: a \in [c]_E \land b \in [c]_E \underbrace{\iff}_{\text{\Cref{lem:equivIffSameClass}}} aEb \).

    \ii
    Take any \([a]_{E_S} \in A/E_S\).
    Since \(S\) is a partition, there (uniquely) exists \(C\) such that \(a \in C\).
    Then, for all \(b\), we have \(
        b \in C
        \iff a E_S b
        \underbrace{\iff}_{\text{\Cref{lem:equivIffSameClass}}} b \in [a]_{E_S}\);
    \(C = [a]_{E_S}\). Therefore, \(A/E_S \subseteq S\).

    For the converse, take any \(C \in S\).
    As \(C\) is nonempty, we may take some \(a \in C\).
    Similarly, we have \(C = [a]_{E_S}\). Therefore, \(C \subseteq A/E_S\).
    \qed
\end{enumerate}
}

\nt{
    \Cref{th:equivAndPartAreSame} essentially states that
    equivalence and partition describe the same ``mathematical reality.''
}

\dfn[setOfRepresentatives]{Set of Representatives}{
    A set \(X \subseteq A\) is called a \textit{set of representatives}
    for the equivalence \(E_S\) (or for the partition \(S\) of \(A\)) if
    \[
        \fall C \in S,\: \exs a \in C,\: X \cap C = \{a\}.
    \]
}

\subfile{../exercises/exercise2-4.tex}

\section{Orderings}

\dfn[ordering]{Partial Ordering and Strict Ordering}{
    Let \(R\) be a binary relation in \(A\).
    \begin{itemize}[nolistsep, leftmargin=*]
        \ii \(R\) is called \textit{antisymmetric in} \(A\) if \(\fall a, b \in A,\: (aRb \land bRa \implies a=b)\).
        \ii \(R\) is called \textit{asymmetric in} \(A\) if \(\fall a, b \in A,\: \lnot (aRb \land bRa)\).
        \ii \(R\) is called a \textit{(partial) ordering} of \(A\) if it is reflexive, antisymmetric, and transitive in \(A\).
        \ii \(R\) is called a \textit{strict ordering} of \(A\) if it is asymmetric and transitive in \(A\).
        \ii If \(R\) is a partial ordering of \(A\), then the pair \((A, R)\) is called an \textit{ordered set}.
    \end{itemize}
}

\exmp{}{
\begin{itemize}[nolistsep, leftmargin=*]
    \ii
    Define the relation \(\subseteq_A\) in \(A\) as follows:
    \(x \subseteq_A y\) if and only if \(x, y \in A \land x \subseteq y\).
    Then, \((A, \subseteq_A)\) is an ordered set.
    \ii
    The relation \(\mrm{Id}_A\) is a partial ordering of \(A\).
\end{itemize}
}

\thm[partialAndStrictAreSame]{}{
    \begin{enumerate}[nolistsep, label=(\roman*)]
        \ii
        Let \(R\) be a partial ordering of \(A\). Then the relation \(S\) in \(A\) defined by
        \[
            S \triangleq R \setminus \mrm{Id}_A
        \]
        is a strict ordering.

        \ii
        Let \(S\) be a strict ordering of \(A\). Then the relation \(R\) in \(A\) defined by
        \[
            R \triangleq S \cup \mrm{Id}_A
        \]
        is a partial ordering.
    \end{enumerate}
}
\mclm{Proof}{\hfill
\begin{enumerate}[nolistsep, label=(\roman*)]
    \ii
    Suppose \(aSb\) and \(bSa\).
    Since \(S \subseteq R\), we have \(aRb\) and \(bRa\).
    As \(R\) is antisymmetric, we have \(aRa\),
    which is impossible since \(S \cap \mrm{Id}_S = \OO\).
    Hence, \(S\) is asymmetric in \(A\).

    Now, assuming \(aSb\) and \(bSc\),
    we also have \(aRc\) since \(R\) is transitive.
    Moreover, \(a\) cannot be equal to \(c\) since \(S\) is shown to be asymmetric.
    Therefore, \(aSc\); \(S\) is transitive in \(A\).

    \ii
    Assume \(aRb\) and \(bRa\).
    If \(a \neq b\), then we have \(aSb\) and \(bSa\), which is impossible.
    Therefore, \(a = b\); \(R\) is antisymmetric.

    Assume \(aRb\) and \(bRc\).
    If \(a = b\) or \(b = c\), then we immediately have \(aRc\).
    If \(a \neq b\) and \(b \neq c\), then \(aSb\) and \(bSc\),
    and thus \(aSc\) as \(S\) is transitive in \(A\);
    \(R\) is transitive in \(A)\).

    \(R\) is reflexive in \(A\) since \(\mrm{Id}_A \subseteq R\).
    \qed
\end{enumerate}
}

\notat{}{
\begin{itemize}[nolistsep, leftmargin=*]
    \ii If \(R\) is a partial ordering, we say \(S = R \setminus \mrm{Id}_A\)
        \textit{corresponds to the partial ordering} \(R\).
    \ii If \(S\) is a strict ordering, we say \(R = S \cup \mrm{Id}_A\)
        \textit{corresponds to the strict ordering} \(S\).
\end{itemize}
}

\dfn[comparable]{Comparability}{
    Let \(a, b \in A\) and let \(\le\) be a partial ordering of \(A\).
    \begin{itemize}[nolistsep, leftmargin=*]
        \ii
        We say that \(a\) and \(b\) are \textit{comparable} in the ordering \(\le\)
        if \(a \le b\) or \(b \le a\).

        \ii
        We say that \(a\) and \(b\) are \textit{incomparable} in the ordering \(\le\)
        if neither \(a \le b\) nor \(b \le a\).
    \end{itemize}
    They can be stated equivalently in terms of the corresponding strict ordering \(<\).
    \begin{itemize}[nolistsep, leftmargin=*]
        \ii
        We say that \(a\) and \(b\) are \textit{comparable} in the ordering \(<\)
        if \(a = b\) or \(a < b\) or \(b < a\).

        \ii
        We say that \(a\) and \(b\) are \textit{incomparable} in the ordering \(<\)
        if none of \(a = b\), \(a < b\), and \(b < a\) holds.
    \end{itemize}
}

\dfn[totalOrdering]{Total Ordering}{
    An ordering \(\le\) (or \(<\)) is called \textit{linear} or \textit{total}
    if any two elements of \(A\) are comparable.
    The pair \((A, \le)\) is then called a \textit{totally ordered set}.
}

\dfn[chain]{Chain}{
    Let \((A, \le)\) be an ordered set and \(B \subseteq A\).
    \(B\) is a \textit{chain} in \(A\) if any two elements of \(B\) are comparable.
}

\dfn[leastMinimal]{Least/Minimal/Greatest/Maximal Element}{
    Let \((A, \le)\) be an ordered set and \(B \subseteq A\).
    \begin{itemize}[nolistsep, leftmargin=*]
        \ii
        \(b \in B\) is the \textit{least element} of \(B\) in the ordering \(\le\)
        if \(\fall x \in B,\: b \le x\).

        \ii
        \(b \in B\) is a \textit{minimal element} of \(B\) in the ordering \(\le\)
        if \(\fall x \in B,\: (x \le b \implies x = b)\).

        \ii
        \(b \in B\) is the \textit{greatest element} of \(B\) in the ordering \(\le\)
        if \(\fall x \in B,\: x \le b\).

        \ii
        \(b \in B\) is a \textit{maximal element} of \(B\) in the ordering \(\le\)
        if \(\fall x \in B,\: (b \le x \implies x = b)\).
    \end{itemize}
}

\notat{}{
    Let \((A, \le)\) be an ordered set and \(B \subseteq A\).
    \begin{itemize}[nolistsep, leftmargin=*]
        \ii The least element of \(B\) is denoted \(\min B\).
        \ii The greatest element of \(B\) is denoted \(\max B\).
    \end{itemize}
}

\thm[basicLeastMinimal]{}{
    Let \((A, \le)\) be an ordered set and \(B \subseteq A\).
    \begin{enumerate}[nolistsep, label=(\roman*)]
        \ii \(B\) has at most one least element.
        \ii The least element of \(B\)---it it exists---is also minimal.
        \ii If \(B\) is a chain, then every minimal element of \(B\) is also least.
    \end{enumerate}
}
\mclm{Proof}{\hfill
\begin{enumerate}[nolistsep, label=(\roman*)]
    \ii
    If \(b\) and \(b'\) are least elements of \(B\),
    then \(b \le b'\) and \(b' \le b\) by the definition.
    As \(\le\) is antisymmetric, we have \(b = b'\).

    \ii
    Let \(b\) be the least element of \(B\) (assuming its existence).
    Take any \(x \in B\) and assume \(x \le b\).
    Then, as \(b\) is the least, we have \(b \le x\).
    As \(\le\) is antisymmetric, \(x = b\); \(b\) is minimal.

    \ii
    Let \(b\) be a minimal element of \(B\).
    Take any \(x \in B\).
    Since \(b\) and \(x\) are comparable, it is \(x \le b\) or \(b \le x\).
    If \(x \le b\), then \(x = b\) as \(b\) is minimal.
    Therefore, \(b\) is the least.
    \qed
\end{enumerate}
}

\nt{
    \noindent
    \Cref{th:basicLeastMinimal} still holds when `least' and `minimal' are replaced by `greatest' and `maximal', respectively.
}

\dfn[bound]{Lower/Upper Bound and Infimum/Supremum}{
    Let \((A, \le)\) be an ordered set and \(B \subseteq A\).
    \begin{itemize}[nolistsep, leftmargin=*]
        \ii
        \(a \in A\) is a \textit{lower bound} of \(B\) in the ordered set \((A, \le)\)
        if \(\fall x \in B,\: a \le x\).
        \ii
        \(a \in A\) is called an \textit{infimum} (or \textit{greatest lower bound}) of \(B\) in the ordered set \((A, \le)\)
        if \(a = \max \{\,x \in A \mid x \text{ is a lower bound of }B\,\}\).
        \ii
        \(a \in A\) is an \textit{upper bound} of \(B\) in the ordered set \((A, \le)\)
        if \(\fall x \in B,\: x \le a\).
        \ii
        \(a \in A\) is called an \textit{supremum} (or \textit{least upper bound}) of \(B\) in the ordered set \((A, \le)\)
        if \(a = \min \{\,x \in A \mid x \text{ is an upper bound of }B\,\}\).
    \end{itemize}
}

\notat{}{
    Let \((A, \le)\) be an ordered set and \(B \subseteq A\).
    \begin{itemize}[nolistsep, leftmargin=*]
        \ii The infimum of \(B\) is denoted \(\inf B\).
        \ii The supremum of \(B\) is denoted \(\sup B\).
    \end{itemize}
}

\thm[basicInfimum]{}{
    Let \((A, \le)\) be an ordered set and \(B \subseteq A\).
    \begin{enumerate}[nolistsep, label=(\roman*)]
        \ii \(B\) has at most one infimum.
        \ii If \(b\) is the least element of \(B\), then \(b\) is the infimum of \(B\).
        \ii If \(b \in B\) is the infimum of \(B\), then \(b\) is the least element of \(B\).
    \end{enumerate}
}
\mclm{Proof}{\hfill
\begin{enumerate}[nolistsep, label=(\roman*)]
    \ii
    The result follows from the definition and \Cref{th:basicLeastMinimal} (i).

    \ii
    \(b\) is a lower bound of \(B\).
    If \(x\) is a lower bound of \(B\), since \(b \in B\), we must have \(x \le b\).
    Therefore, \(b\) is the greatest lower bound.

    \ii
    \(b \in B\) is a lower bound of \(B\), and thus \(b\) is the least element.
    \qed
\end{enumerate}
}

\nt{
    \noindent
    \Cref{th:basicInfimum} still holds when `least' and `infimum' are replaced by `greatest' and `supremum', respectively.
}

\dfn[isomorphismOrderedSet]{Isomorphism Between Ordered Sets}{
    An \textit{isomorphism} between two ordered sets \((P, \le)\) and \((Q, \preceq)\)
    is a function \(f \colon P \hooktwoheadrightarrow Q\) such that
    \[
        \fall p_1, p_2 \in P,\: (p_1 \le p_2 \iff f(p_1) \preceq f(p_2)).
    \]
    If an isomorphism exists between \((P, \le)\) and \((Q, \preceq)\),
    then we say \((P, \le)\) and \((Q, \preceq)\) are \textit{isomorphic}.
    This is justified by \Cref{exer:2.5.13}.
}

\mlemma[oneImplicationIsEnough]{}{
    Let \((P, \le)\) be a totally ordered set and let \((Q, \preceq)\) be an ordered set.
    Let \(h \colon P \hooktwoheadrightarrow Q\) be a function such that
    \[
        \fall p_1, p_2 \in P,\: (p_1 \le p_2 \implies h(p_1) \preceq h(p_2)).
    \]
    Then, \(h\) is an isomorphism between \((P, \le)\) and \((Q, \preceq)\),
    and \((Q, \le)\) is totally ordered.
}
\pf{Proof}{
    Take any \(p_1, p_2 \in P\) and assume \(h(p_1) \preceq h(p_2)\).
    Suppose \(p_2 < p_1\) for the sake of contradiction.
    Then, since \(h\) is injective, \(h(p_1) \neq h(p_2)\), and thus \(h(p_1) \prec h(p_2)\).
    Then, we have \(\lnot (p_2 \le p_1)\), which is a contradiction.
    Hence, \(\lnot(p_2 < p_1)\).
    Therefore, \(p_1 \le p_2\) since \((P, \le)\) is totally ordered.

    Take any \(q_1, q_2 \in Q\).
    Then, since \(h\) is onto \(Q\), there exist \(p_1, p_2 \in P\)
    such that \(q_1 = h(p_1)\) and \(p_2 = h(p_2)\).
    Since \(P\) is totally ordered, it is \(p_1 \le p_2\) or \(p_2 \le p_1\).
    In either case, we have \(q_1 \preceq q_2\) or \(p_2 \preceq q_1\).
    Therefore, \((Q, \preceq)\) is totally ordered.
}

\subfile{../exercises/exercise2-5.tex}

\end{document}
