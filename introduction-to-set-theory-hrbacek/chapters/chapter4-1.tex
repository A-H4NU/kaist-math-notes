\documentclass[../introduction_to_set_theory.tex]{subfiles}

\begin{document}

\section{Cardinality of Sets}

\dfn[equipotent]{Equipotent Sets}{
    Let \(A\) and \(B\) be sets.
    \(A\) is \textit{equipotent to} \(B\)
    if there is a function \(f \colon A \hooktwoheadrightarrow B\).
    We write \(|A| = |B|\).
}

\mlemma[basicEquipotent]{}{
    Let \(A\), \(B\), and \(C\) be sets.
    \begin{enumerate}[nolistsep, label=(\roman*), leftmargin=*, listparindent=\parindent]
        \ii \(|A| = |A|\).
        \ii If \(|A| = |B|\), then \(|B| = |A|\).
        \ii If \(|A| = |B|\) and \(|B| = |C|\), then \(|A| = |C|\).
    \end{enumerate}
}
\mclm{Proof}{\hfill
\begin{enumerate}[nolistsep, label=(\roman*), leftmargin=*, listparindent=\parindent]
    \ii
    \(\mrm{Id}_A\) is an injective function on \(A\) onto \(A\).
    \ii
    If \(f \colon A \hooktwoheadrightarrow B\),
    then \(f\inv \colon B \hooktwoheadrightarrow A\).
    \ii
    If \(f \colon A \hooktwoheadrightarrow B\),
    and if \(g \colon B \hooktwoheadrightarrow C\),
    then \(f \circ g \colon A \hooktwoheadrightarrow C\).
    \qed
\end{enumerate}
}

\nt{
    \noindent
    \Cref{lem:basicEquipotent} essentially says that \(|A| = |B|\) behaves like an equivalence relation.
}

\dfn[]{}{
    \begin{itemize}[nolistsep, leftmargin=*, listparindent=\parindent]
        \ii 
        We say \textit{the cardinality of \(A\) is less than or equal to the cardinality of \(B\)}
        if there is a function \(f \colon A \hookrightarrow B\). We write \(|A| \le |B|\).
        \ii
        We say \textit{the cardinality of \(A\) is less than the cardinality of \(B\)}
        if \(|A| \le |B|\) and \(\lnot (|A| = |B|)\).
        We write \(|A| < |B|\).
    \end{itemize}
}

\mlemma[basicLessCard]{}{
    Let \(A\), \(B\), and \(C\) be sets.
    \begin{enumerate}[nolistsep, label=(\roman*), ref=\protect{\Cref{lem:basicLessCard} (\roman*)}, leftmargin=*, listparindent=\parindent]
        \ii If \(|A| = |B|\), then \(|A| \le |B|\).
        \ii \(|A| \le |A|\)
        \ii \label{itm:basicLessCardiii}
        If \(|A| \le |B|\) and \(|B| \le |C|\), then \(|A| \le |C|\).
    \end{enumerate}
}
\mclm{Proof}{\hfill
\begin{enumerate}[nolistsep, label=(\roman*), leftmargin=*, listparindent=\parindent]
    \ii 
    If \(f \colon A \hooktwoheadrightarrow B\),
    then \(f\) is injective as well.

    \ii
    \(\mrm{Id}_A\) is an injective function on \(A\) into \(A\).

    \ii
    If \(f \colon A \hookrightarrow B\),
    and if \(g \colon B \hookrightarrow C\),
    then \(f \circ g \colon A \hookrightarrow C\).
    \qed
\end{enumerate}
}

\mlemma[cantorBernsteinLem]{}{
    If \(A_1 \subseteq B \subseteq A\) and \(|A_1| = |A|\), then \(|B| = |A|\).
}
\nt{
    We present two proofs for \Cref{lem:cantorBernsteinLem}.
    The second proof can be viewed as a more fundamental proof
    in the sense that it does not depend on \nameref{ax:infinity}.
}
\pf{Proof 1}{
    Let \(f \colon A \hooktwoheadrightarrow A_1\).
    Define a sequence \(\lang\,A_i \mid i \in \NN\,\rang\) and \(\lang\,B_i \mid i \in \NN\,\rang\) recursively by
    \begin{alignat}{3}
        A_0 &= A, &\qquad B_0 &= B, \nonumber\\
        \fall n \in \NN,\: A_{n+1} &= f[A_n], &\qquad \fall n \in \NN,\:B_{n+1} &= f[B_n] \label{eq:uvGTBFKn}\tag{\(\ast\)}
    \end{alignat}
    thanks to \nameref{th:recursion}.

    We clearly have \(A_1 \subseteq B_0 \subseteq A_0\).
    If \(A_{n+1} \subseteq B_n \subseteq A_n\),
    then \(A_{n+2} = f[A_{n+1}] \subseteq B_{n+1} = f[B_n] \subseteq A_{n+1} = f[A_n]\)
    by \eqref{eq:uvGTBFKn}.
    Hence, by \eqref{eq:uvGTBFKn} and \nameref{th:induction},
    we have \(A_{n+1} \subseteq B_n \subseteq A_n\) for all \(n \in \NN\).

    Let, for each \(n \in \NN\), \(C_n \triangleq A_n \setminus B_n\).
    Then, by \ref{itm:2.3.6.ii}, \(C_{n+1} = f[A_n] \setminus f[B_n] = f[A_n \setminus B_n] = f[C_n]\).
    Let
    \[
        C \triangleq \bigcup_{n=0}^\infty C_n \quad\text{and}\quad
        D \triangleq A \setminus C.
    \]
    Hence, \(f[C] = \bigcup_{n=1}^\infty C_n \subseteq C\).
    Now, define a function \(g \colon A \to A\) by
    \[
        g(x) = \begin{cases}
            f(x) & \text{if } x \in C \\
            x & \text{if } x \in D.
        \end{cases}
    \]
    We immediately notice that \(\restr{g}{C} = \restr{f}{C}\) and \(\restr{g}{D}\) are injective
    and their ranges---\(f[C]\) and \(D\)---are disjoint; \(g\) is injective.

    As, \(\fall n \ge 1,\: C_n \subseteq A_n \subseteq B_0 = B\), we have \(f[C] \subseteq B\).
    If \(x \in D\), then \(x \in A \setminus C_0 = A \setminus (A \setminus B) = B\) by \ref{itm:1.4.2iii}.

    Now, we shall show \(B \subseteq f[C] \cup D\) and thus \(B = \ran g\).
    Take any \(y \in B\). Then, \(y \in C\) or \(y \in D\).
    If \(y \in D\), then it is done; so assume \(y \in C\).
    Then, as \(y \notin A \setminus B = C_0\), \(y \in f[C]\).
    Hence, \(g \colon A \hooktwoheadrightarrow B\).
}

\pf{Proof 2}{
    Let \(f \colon A \hooktwoheadrightarrow A_1\).
    Let \(F \colon \mcal P(A) \to \mcal P(A)\) be defined by
    \(F(X) = (A \setminus B) \cup f[X]\).
    If \(X \subseteq Y \subseteq A\), then \(F(X) = (A \setminus B) \cup f[X] \subseteq (A \setminus B) \cup f[Y] = F(Y)\).
    Hence, by \Cref{exer:4.1.10}, there exists \(C \subseteq A\)
    such that \[C = (A \setminus B) \cup f[C].\]
    Let \(D \triangleq A \setminus C\).

    Now, define a function \(g \colon A \to A\) by
    \[
        g(x) = \begin{cases}
            f(x) & \text{if } x \in C \\
            x & \text{if } x \in D.
        \end{cases}
    \]
    Then, since \(f[C] \subseteq C\), \(g\) is injective.

    Moreover,
    \(f[C] \subseteq \ran f = A_1 \subseteq B\) and
    \(D = A \setminus C = A \setminus ((A \setminus B) \cup f[C]) \subseteq A \setminus (A \setminus B) = B\),
    and thus \(\ran g \subseteq B\).
    
    Now, take any \(y \in B\).
    If \(y \in C\), then, as \(y \notin A \setminus B\), \(y \in f[C]\).
    Hence, \(B \subseteq f[C] \cup D\).
    Therefore, \(g \colon A \hooktwoheadrightarrow B\).
}

\thm[cantorBernstein]{Cantor--Bernstein Theorem}{
    If \(|X| \le |Y|\) and \(|Y| \le |X|\), then \(|X| = |Y|\).
}
\pf{Proof}{
    Let \(f \colon X \hookrightarrow Y\) and \(g \colon Y \hookrightarrow X\).
    Then, \(g \colon Y \hooktwoheadrightarrow g[Y]\), i.e., \(|Y| = |g[Y]|\);
    and \(g \circ f \colon X \hooktwoheadrightarrow (g \circ f)[X]\), i.e., \(|X| = |(g \circ f)[X]|\).
    Moreover, \((g \circ f)[X] \subseteq g[Y] \subseteq X\).
    Hence, by \Cref{lem:cantorBernsteinLem}, \(|g[Y]| = |X|\).
    We conclude \(|X| = |Y|\) from \Cref{lem:basicEquipotent}.
}

\assume[cardinalExists]{}{
    There are sets called \textit{cardinal numbers} (or \textit{cardinals})
    with the property that for every set \(X\) there is a unique cardinal \(|X|\)
    (the \textit{cardinal number of} \(X\), the \textit{cardinality of} \(X\))
    and sets \(X\) and \(Y\) are equipotent if and only if \(|X|\) is equal to \(Y\).
}

\nt{
    \Cref{as:cardinalExists} essentially asserts the existence
    of a unique ``representative'' for each class of mutually equipotent sets.
    \Cref{as:cardinalExists} is \textit{harmless} in the sense that we only use it for
    convenience and we could formulate the theorems without it.
    We prove \Cref{as:cardinalExists} in \Cref{chap:aoc}: \nameref{chap:aoc}.
    % TODO: change ``The Axiom of Choice'' to reference.
    However, for certain classes of sets, cardinal numbers can be defined without the Axiom of Choice.
}

\subsection*{Selected Problems}

\setexernumber{1}

\exer[4.1.2]{}{
    Let \(A\), \(B\), and \(C\) be sets.
    \begin{enumerate}[nolistsep, label=(\roman*), leftmargin=*, listparindent=\parindent]
        \ii If \(|A| < |B|\) and \(|B| \le |C|\), then \(|A| < |C|\).
        \ii If \(|A| \le |B|\) and \(|B| < |C|\), then \(|A| < |C|\).
    \end{enumerate}
}
\mclm{Proof}{\hfill
\begin{enumerate}[nolistsep, label=(\roman*), leftmargin=*, listparindent=\parindent]
    \ii
    We already have \(|A| \le |C|\) by \ref{itm:basicLessCardiii}.
    Let \(g \colon B \hookrightarrow C\).
    Suppose \(f \colon A \hooktwoheadrightarrow C\) for the sake of contradiction.
    Then, \(f\inv \circ g \colon B \hookrightarrow A\), i.e., \(|B| \le |A|\).
    By \nameref{th:cantorBernstein}, we get \(|A| = |B|\), which is a contradiction.

    \ii
    We already have \(|A| \le |C|\) by \ref{itm:basicLessCardiii}.
    Let \(g \colon A \hookrightarrow B\).
    Suppose \(f \colon A \hooktwoheadrightarrow C\) for the sake of contradiction.
    Then, \(g \circ f\inv \colon C \hookrightarrow B\), i.e., \(|C| \le |B|\).
    By \nameref{th:cantorBernstein}, we get \(|B| = |C|\), which is a contradiction.

    \qed
\end{enumerate}
}

\exer[4.1.3]{}{
    If \(A \subseteq B\), then \(|A| \le |B|\).
}
\pf{Proof}{
    \(\mrm{Id}_A\) is an injective function on \(A\) into \(B\).
}

\setexernumber{6}

\exer[4.1.7]{}{
    If \(S \subseteq T\), then \(|A^S| \le |A^T|\). In particular, \(|A^m| \le |A^n|\) if \(m \le n\).
}
\pf{Proof}{
    If \(T = \OO\), then \(A^S = A^T = \{\OO\}\) and it is done.

    Assume \(T \neq \OO\).
    Fix some \(t \in T\).
    Now, define \(f \colon A^S \hookrightarrow A^T\) by
    \(g \mapsto g \cup \{\,(x, t) \mid x \in T \setminus S\,\}\).
}

\setexernumber{9}

\exer[4.1.10]{}{
    Let \(F \colon \mcal{P}(A) \to \mcal{P}(A)\) be \textit{monotone},
    i.e., if \(X \subseteq Y \subseteq A\), then \(F(X) \subseteq F(Y)\).
    Then, \(F\) has a least \textit{fixed point} \(\ol{X}\),
    that is to say \(F(\ol{X}) = \ol{X}\) and \(\fall X \subseteq A,\: (F(X) = X \implies \ol{X} \subseteq X)\).
}
\pf{Proof}{
    Let \(T \triangleq \{\,X \subseteq A \mid F(X) \subseteq X\,\}\).
    Then, as \(A \in T\), \(T \neq \OO\); we may let \(\ol{X} \triangleq \bigcap T\).

    Then, for all \(X \in T\), \(\ol{X} \subseteq X\); and thus \(F(\ol{X}) \subseteq F(X) \subseteq X\).
    We have \(F(\ol{X}) \subseteq \bigcap T = \ol{X}\), i.e., \(\ol{X} \in T\).

    On the other hand, we have \(F(F(\ol{X})) \subseteq F(\ol{X})\), or \(F(\ol{X}) \in T\),
    and thus \(\ol{X} = \bigcap T \subseteq F(\ol{X})\). Therefore, \(F(\ol{X}) = \ol{X}\).
    Moreover, if \(X\) is a fixed point, then \(X \in T\), and thus \(\ol{X} = \bigcap T \subseteq X\).
}

\setexernumber{13}

\exer[4.1.14]{}{
    A function \(F \colon \mcal P(A) \to \mcal P(A)\) is \textit{continuous}
    if, for each sequence \(\lang\,X_i \mid i \in \NN\,\rang\) of subsets of \(A\)
    such that \(\fall i, j \in \NN,\: (i \le j \implies X_i \subseteq X_j)\),
    \(F \big(\bigcup_{i \in \NN} X_i\big) = \bigcup_{i \in \NN} F(X_i)\) holds.

    If \(\ol{X}\) is the least fixed point of a monotone continuous function,
    \(F \colon \mcal P(A) \to \mcal P(A)\),
    then \(\ol{X} = \bigcup_{i \in \NN} X_i\)
    where we define recursively \(X_0 = \OO\), \(\fall i \in \NN,\: X_{i+1} = F(X_i)\).
}
\pf{Proof}{
    Let \(\tilde X \triangleq \bigcup_{i \in \NN} X_i\).
    We have \(X_0 = \OO \subseteq X_1\).

    If \(X_n \subseteq X_{n+1}\), then \(X_{n+1} \subseteq X_{n+2}\) since \(F\) is monotone.
    Hence, \(\fall n \in \NN,\: X_n \subseteq X_{n+1}\).
    Therefore, similarly to \Cref{exer:3.3.1},
    we have \(X_m \subseteq X_n\) whenever \(m \le n\).
    Hence, \(F(\tilde X) = \bigcup_{i \in \NN} F(X_i) = \bigcup_{i=1}^\infty X_{i} = \tilde X\);
    \(\tilde X\) is a fixed point of \(F\); hence \(\ol X \subseteq \tilde X\). 

    We have \(X_0 \subseteq \ol X\).
    If \(X_n \subseteq \ol{X}\) for \(n \in \NN\), then
    \(X_{n+1} \subseteq F(\ol X)) = \ol X\).
    Hence, by \nameref{th:induction}, \(\tilde X \subseteq \ol X\).
}

\end{document}
