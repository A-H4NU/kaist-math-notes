\documentclass[../introduction_to_set_theory_Note.tex]{subfiles}

\begin{document}

\section{Addition and Multiplication of Alephs}

\nt{
    Refer to \Cref{sec:cardinalArith}: \nameref{sec:cardinalArith}
    for definitions of addition and multiplication between two cardinals.
}

\thm[squareOfAlephs]{}{
    \(\fall \alpha \in \Ord,\: \aleph_{\alpha} \cdot \aleph_{\alpha} = \aleph_{\alpha}\)
}
\pf{Proof}{
    Take any ordinal \(\alpha\) and define \(\preceq\) on \(\omega_{\alpha} \times \omega_{\alpha}\) by
    \[
        (\alpha_1, \alpha_2) \preceq (\beta_1, \beta_2)
        \iff
        (\max \{\alpha_1, \alpha_2\}, \alpha_1, \alpha_2) \le_{\alpha} (\max \{\beta_1, \beta_2\}, \beta_1, \beta_2)
    \]
    where \(\le_{\alpha}\) is the lexicographical ordering of
    \(\omega_{\alpha} \times \omega_{\alpha} \times \omega_{\alpha}\).
    Then, \(\preceq\) is naturally a well-ordering of \(\omega_{\alpha} \times \omega_{\alpha}\)
    since \(R_{\alpha}\) is a well-ordering.

    Now, we will prove by transfinite induction on \(\alpha\).
    We already have \(\aleph_0 \cdot \aleph_0 = \aleph_0\) by
    \Cref{th:productOfCountable}.

    Take any ordinal \(\alpha\) and assume that
    \(\fall \beta < \alpha,\: \aleph_\beta \cdot \aleph_\beta \le \aleph_\beta\).

    \clm[pfPmbtHV]{
        For any \((\alpha_1, \alpha_2) \in \omega_{\alpha} \times \omega_{\alpha}\),
        we have \(|X| < \aleph_\alpha\) where
        \[
            X \triangleq \{\,(\xi_1, \xi_2) \in \omega_{\alpha} \times \omega_{\alpha} \mid (\xi_1, \xi_2) \prec (\alpha_1, \alpha_2)\,\}.
        \]
    }{
        Let \(\beta \triangleq \max \{\alpha_1, \alpha_2\} + 1\).
        We have \(\beta < \omega_{\alpha}\).
        Then, for every \((\xi_1, \xi_2) \in X\),
        we have \(\max \{\xi_1, \xi_2\} \le \max \{\alpha_1, \alpha_2\} < \beta\)
        by definition. Hence, \(\xi_1 < \beta\) and \(\xi_2 < \beta\).
        In other words, \(X \subseteq \beta \times \beta\).

        As \(\omega_{\alpha}\) is an initial ordinal,
        \(|\beta| < \aleph_{\alpha}\).
        Moreover, by \Cref{th:wosetEquipotentToInitial},
        there exists \(\gamma < \alpha\) such that \(|\beta| \le \aleph_\gamma\). Therefore,
        \begin{alignat*}{2}
            |X| &\le |\beta \times \beta| &\qquad& \comment*{\Cref{exer:4.1.3}} \\
                &\le \aleph_{\gamma} \cdot \aleph_{\gamma} \\
                &\le \aleph_{\gamma} && \comment*{Induction Hypothesis} \\
                &< \aleph_{\alpha}.
        \end{alignat*}
        \qed
    }

    If the order type of \((\omega_{\alpha} \times \omega_{\alpha}, \preceq)\) were greater than \(\omega_{\alpha}\),
    then there exists an initial segment \(X\) of \((\omega_{\alpha} \times \omega_{\alpha}, \preceq)\)
    such that \(|X| = \aleph_{\alpha}\),
    which is impossible by \Cref{clm:pfPmbtHV}.
    Hence, \(\aleph_{\alpha} \cdot \aleph_{\alpha} \le \aleph_{\alpha}\).
    The result follows from \nameref{th:firstTransInduction}.
}

\cor[productOfAlephs]{}{
    \begin{enumerate}[nolistsep, label=(\roman*), ref=\protect{\Cref{cor:productOfAlephs} (\roman*)}]
        \ii\label{itm:productOfAlephs.i}
        \(\fall \alpha, \beta \in \Ord,\: (\alpha \le \beta \implies \aleph_{\alpha} \cdot \aleph_{\beta} = \aleph_{\beta})\)
        \ii\label{itm:productOfAlephs.ii}
        \(\fall \alpha \in \Ord,\: \fall n \in \NN,\: n \cdot \aleph_{\alpha} = \aleph_{\alpha}\)
    \end{enumerate}
}
\mclm{Proof}{\hfill
\begin{enumerate}[nolistsep, label=(\roman*), leftmargin=*, listparindent=\parindent]
    \ii
    It is direct that \(\aleph_{\beta} = 1 \cdot \aleph_{\beta} \le \aleph_{\alpha} \cdot \aleph_{\beta}\).
    On the other hand, by \Cref{th:squareOfAlephs},
    \(\aleph_{\alpha} \cdot \aleph_{\beta} \le \aleph_{\beta} \cdot \aleph_{\beta} = \aleph_{\beta}\).
    Hence, by \nameref{th:cantorBernstein}, \(\aleph_{\alpha} \cdot \aleph_{\beta} = \aleph_{\beta}\).

    \ii
    It is direct that \(\aleph_{\alpha} = 1 \cdot \aleph_{\alpha} \le n \cdot \aleph_{\alpha}\).
    On the other hand, by \Cref{th:squareOfAlephs},
    \(n \cdot \aleph_{\alpha} \le \aleph_{\alpha} \cdot \aleph_{\alpha} = \aleph_{\alpha}\).
    Hence, by \nameref{th:cantorBernstein}, \(\aleph_{\alpha} \cdot \aleph_{\beta} = \aleph_{\beta}\).
    \qed
\end{enumerate}
}

\cor[sumOfAlephs]{}{
    \begin{enumerate}[nolistsep, label=(\roman*), ref=\protect{\Cref{cor:productOfAlephs} (\roman*)}]
        \ii
        \(\fall \alpha, \beta \in \Ord,\: (\alpha \le \beta \implies \aleph_{\alpha} + \aleph_{\beta} = \aleph_{\beta})\)
        \ii
        \(\fall \alpha \in \Ord,\: \fall n \in \NN,\: n + \aleph_{\alpha} = \aleph_{\alpha}\)
    \end{enumerate}
}
\mclm{Proof}{\hfill
\begin{enumerate}[nolistsep, label=(\roman*), leftmargin=*, listparindent=\parindent]
    \ii
    \(\aleph_{\beta} \le \aleph_{\alpha} + \aleph_{\beta} \le \aleph_{\beta} + \aleph_{\beta}
    = 2 \cdot \aleph_{\beta} = \aleph_{\beta}\).
    \ii
    \(\aleph_{\alpha} \le n + \aleph_{\alpha} \le \aleph_{\alpha} + \aleph_{\alpha}
    = 2 \cdot \aleph_{\alpha} = \aleph_{\alpha}\).
    \qed
\end{enumerate}
}

\subfile{../exercises/exercise7-2.tex}

\end{document}
