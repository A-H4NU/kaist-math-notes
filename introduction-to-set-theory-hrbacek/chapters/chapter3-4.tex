\documentclass[../introduction_to_set_theory.tex]{subfiles}

\begin{document}

\section{Operations and Structures}

\dfn[operation]{Operation}{
    \begin{itemize}[nolistsep]
        \ii A \textit{unary operation} on \(S\) is a function \(S \rightharpoonup S\).
        \ii A \textit{binary operation} on \(S\) is a function \(S^2 \rightharpoonup S\).
    \end{itemize}
}

\notat{Binary Operation}{
    Non-letter symbols such as \(+\), \(\times\), \(\ast\), \(\vartriangle\), etc.,
    are often used to denote operations.
    The value of the operation \(\ast\) at \((x, y)\) is then denoted
    \(x \ast y\) rather than \(\ast(x, y)\).
}

\dfn[closed]{Closedness Under Operation}{
    Let \(f\) be a binary operation on \(S\) and \(A \subseteq S\).
    \(A\) is said to be \textit{closed under the operation \(f\)}
    if \(\fall x, y \in A,\: [(x, y) \in \dom f \implies f(x, y) \in A]\).
}

\dfn[nTuple]{\(\boldsymbol{n}\)-Tuple}{
    Let \(n \in \NN\).
    An \(n\)-tuple is a finite sequence of length \(n\).
}

\nt{
    Let \(\lang\,a_0, \cdots, a_{n-1}\,\rang\) and \(\lang\,b_0, \cdots, b_{n-1}\,\rang\) be two \(n\)-tuples.
    We have, by \Cref{lem:functionEqualsIff},
    \[
        \lang\,a_0, \cdots, a_{n-1}\,\rang = \lang\,b_0, \cdots, b_{n-1}\,\rang
        \iff \fall i < n,\: a_i = b_i.
    \]
    This satisfies the usual defining property of \(n\)-tuple.
}

\nt{
\begin{itemize}[nolistsep, leftmargin=*]
    \ii
    If \(\lang\, A_i \mid 0 \le i < n\,\rang\) is a finite sequence (of sets),
    then the product of the indexed system of sets \(\prod_{0 \le i < n} A_i\) (\Cref{dfn:productIndexedSystemOfSets})
    is just the set of all \(n\)-tuples \(a = \lang\,a_0, \cdots, a_{n-1}\,\rang\) such that
    \(\fall i < n,\: a_i \in A_i\).

    \ii
    If \(\fall i < n, A_i = A\), then \(\prod_{0 \le i < n} A_i = A^n\).

    \ii \(A^0 = \{\lang\rang\}\).
\end{itemize}
}

\notat{}{
    \setlength{\parindent}{1.5em}
    The `ordered pair' (\Cref{dfn:orderedPair}), \((a_0, a_1) = \{\{a_0\}, \{a_0, a_1\}\}\),
    is different set from the `\(2\)-tuple' (\Cref{dfn:nTuple}), \(\lang a_0, a_1\rang = \{(0, a_0), (1, a_1)\}\).
    Consequently, \(A_0 \times A_1\) (\Cref{dfn:cartesianProduct}) does not generally
    equal to \(\prod_{0 \le i < 2} A_i\) (\Cref{dfn:productIndexedSystemOfSets}).

    However, since there is a natural one-to-one correspondence
    \begin{align*}
       \delta : A_0 \times A_1 &\longhooktwoheadrightarrow \textstyle\prod_{0 \le i < 2} A_i \\
       (a_0, a_1) &\longmapsto \lang a_0, a_1 \rang,
    \end{align*}
    for almost all practical purposes---when only the defining property of \(n\)-tuple is needed)---%
    it makes so difference which definition one uses.

    Therefore, we do not distinguish between ordered pairs and \(2\)-tuples now on.
    That is to say we use notations
    \[
        \lang\,a_0, \cdots, a_{n-1}\,\rang \quad\text{and}\quad
        (a_0, \cdots, a_{n-1})
    \]
    interchangeably from now on.
}

\dfn[nAryOperation]{\(\boldsymbol{n}\)-ary Relation}{
    An \textit{\(n\)-ary relation} \(R\) in \(A\) is a subset of \(A^n\).
    We write \(R(a_0, a_1, \cdots, a_{n-1})\) instead of \(\lang a_0, a_1, \cdots, a_{n-1}\rang \in R\).
}

\dfn[nAryOperation]{\(\boldsymbol{n}\)-ary Operation}{
    An \textit{\(n\)-ary operation} \(F\) on \(A\) is a function \(A^n \rightharpoonup A\).
    We write \(F(a_0, a_1, \cdots, a_{n-1})\) instead of \(F(\lang a_0, a_1, \cdots, a_{n-1}\,\rang)\).
}

\nt{
\begin{itemize}[nolistsep, leftmargin=*]
    \ii \(1\)-ary relations in \(A\) need not be distinguished from subsets of \(A\).
    \ii \(1\)-ary operations on \(A\) need not be distinguished from functions \(A \rightharpoonup A\).
    \ii Nonempty \(0\)-ary operations on \(A\) need not be distinguished from \(A\).
        (A nonempty \(0\)-ary operation is of the form \(\{(\lang\rang, a)\}\) where \(a \in A\);
        a nonempty \(0\)-ary operation is called a \textit{constant}.)
\end{itemize}
}

\dfn[structure]{Structure}{
    \begin{itemize}[nolistsep, leftmargin=*]
        \ii
        A \textit{type} \(\tau\) is an ordered pair \((\lang\,r_0, \cdots, r_{m-1}\,\rang, \lang\,f_0, \cdots, f_{n-1}\,\rang)\)
        of finite sequences of natural numbers.

        \ii
        A \textit{structure of type \(\tau\)} is a triple
        \[
            \mfr{A} = (A, \lang\,R_0, \cdots, R_{m-1}\,\rang, \lang\,F_0, \cdots, F_{n-1}\,\rang)
        \]
        where \(R_i\) is an \(r_i\)-ary relation on \(A\) for each \(i < m\)
        and \(F_j\) is an \(f_j\)-ary operation on \(A\) for each \(j < n\).
        In addition, we require \(F_j \neq \OO\) if \(f_j = 0\),
        i.e., \(F_j\) should be constant.
        \(A\) is called the \textit{universe} of the structure \(\mfr{A}\).
    \end{itemize}
}

\exmp{}{
    \noindent
    \(\mfr{N} = (\NN, \lang\le\rang, \lang 0, +, \cdot\rang)\) is a structure of type
    \((\lang 2 \rang, \lang 0, 2, 2 \rang)\).
}

\notat{}{
    We often write the structure of type \((\lang\,r_0, \cdots, r_{m-1}\,\rang, \lang\,f_0, \cdots, f_{n-1}\,\rang)\)
    as a \((1 + m + n)\)-tuple, for example, \((\NN, \le, 0, +, \cdot)\),
    when it is understood which symbol represent relations and which operations.
}

\dfn[]{Isomorphism Between Structures}{
    Let \(\mfr{A}\) and \(\mfr{A}'\) be structures of the same type \(\tau = (\lang\,r_0, \cdots, r_{m-1}\,\rang, \lang\,f_0, \cdots, f_{n-1}\,\rang)\).
    Write \(\mfr{A} = (A, \lang\,R_0, \cdots, R_{m-1}\,\rang, \lang\,F_0, \cdots, F_{n-1}\,\rang)\)
    and \(\mfr{A}' = (A', \lang\,R'_0, \cdots, R'_{m-1}\,\rang, \lang\,F'_0, \cdots, F'_{n-1}\,\rang)\).
    An \textit{isomorphism} between structures \(\mfr{A}\) and \(\mfr{A}'\)
    is a mapping \(h \colon A \hooktwoheadrightarrow A'\) such that
    \begin{enumerate}[nolistsep, label=(\roman*), leftmargin=*]
        \ii
        \(\fall i < m,\: \fall a \in A^{r_i},\: [R_i(a_0, \cdots, a_{r_i-1}) \iff R'_i(h(a_0), \cdots, h(a_{r_i}-1))]\)
        \ii
        \(
            \fall j < n,\: \fall a \in A^{f_j},
            \big[(a_0, \cdots, a_{f_j-1}) \in \dom F_j \iff (h(a_0), \cdots, h(a_{f_j-1})) \in \dom F'_j\big]
        \)
        \ii
        \(\begin{aligned}[t]
            \fall j < n,\: \fall a \in A^{f_j},
            \big[&(a_0, \cdots, a_{f_j-1}) \in \dom F_j \\
             &\implies h(F_j(a_0, \cdots, a_{f_j-1})) = F'_j(h(a_0), \cdots, h(a_{f_j-1}))\big].
        \end{aligned}\)
    \end{enumerate}
}

\dfn[]{Automorphism}{
    An isomorphism between a structure \(\mfr{A}\) and itself is called an \textit{automorphism}.
}

\dfn[]{Closed Set}{
    Fix a structure \(\mfr{A} = (A, \lang\,R_0, \cdots, R_{m-1}\,\rang, \lang\,F_0, \cdots, F_{n-1}\,\rang)\).
    A set \(B \subseteq A\) is called \textit{closed} if
    \[
        \fall j < n,\: \fall  a \in B^{f_j},\: [ (a_0, \cdots, a_{f_j-1}) \in \dom F_j \implies F_j(a_0, \cdots, a_{f_j-1}) \in B].
    \]
}

\dfn[closure]{Closure}{
    Fix a structure \(\mfr{A} = (A, \lang\,R_0, \cdots, R_{m-1}\,\rang, \lang\,F_0, \cdots, F_{n-1}\,\rang)\).
    Let \(C \subseteq A\). The \textit{closure} of \(C\),
    \[
        \cl{C} \triangleq \bigcap \{\,B \subseteq A \mid C \subseteq B \text{ and }B\text{ is closed}\,\},
    \]
    is the least closed set containing all elements of \(C\).
}

\thm[closureAsUnion]{}{
    Let \(\mfr{A} = (A, \lang\,R_0, \cdots, R_{m-1}\,\rang, \lang\,F_0, \cdots, F_{n-1}\,\rang)\)
    be a structure and let \(C \subseteq A\).
    If the sequence \(\lang\,C_i \mid i \in \NN\,\rang\) is defined recursively by
    \begin{align*}
        C_0 &= C; \\
        \fall i \in \NN,\: C_{i+1} &= C_i \cup \bigcup_{j=0}^{n-1} F_j[C_i^{f_j}],
    \end{align*}
    then \(\cl{C} = \bigcup_{i=0}^\infty C_i\).
}
\pf{Proof}{
    Note the recursive definition is justified by \nameref{th:recursion}.
    Let \(\tilde{C} \triangleq \bigcup_{i=0}^\infty C_i\).

    \clm[c7dffb1b]{
        \(\cl C \subseteq \tilde C\)
    }{
        Since we have \(C_0 \subseteq \tilde C\), it is enough to show that \(\tilde C\)
        is closed.

        Take any \(j < n\) and \(a \in \tilde{C}^{f_j}\).
        By the definition of \(\tilde C\), \(\fall r < f_j,\: \exs i_r \in \NN,\: a_r \in C_{i_{r}}\).
        We may take \(\bar \imath = \max \{\,i_r \mid r < f_j\,\}\) by \Cref{exer:3.5.13}.
        Since \(C_i \subseteq C_{i+1}\) for all \(i \in \NN\),
        we have \(a_r \in C_{i_r} \subseteq C_{\bar \imath}\) for all \(r < f_j\).
        Hence, if \((a_0, \cdots, a_{f_j-1}) \in \dom F_j\),
        we have \(F_j(a_0, \cdots, a_{f_j-1}) \in F_j[C_{\bar \imath}^{f_j}] \subseteq C_{\bar \imath + 1} \subseteq \tilde C\).
        Hence, \(\tilde C\) is closed. \qed
    }

    \clm[4bb66ea3]{
        \(\tilde C \subseteq \cl C\)
    }{
        Clearly \(C_0 = C \subseteq \cl{C}\).
        If \(C_i \subseteq \cl{C}\), then, for each \(j < n\), \(F_j[C_i^{f_j}] \subseteq \cl{C}\)
        since \(\cl{C}\) is closed.
        Hence, \(C_{i+1} \subseteq \cl{C}\).
        Therefore, by \nameref{th:induction}, \(\fall i \in \NN,\: C_i \subseteq \cl{C}\);
        hence \(\tilde C \subseteq \cl{C}\). \qed
    }
    \noindent
    Combining \Cref{clm:c7dffb1b,clm:4bb66ea3} completes the proof.
}

\thm[generalInduction]{The General Induction Principle}{
    Let \(\mfr{A} = (A, \lang\,R_0, \cdots, R_{m-1}\,\rang, \lang\,F_0, \cdots, F_{n-1}\,\rang)\)
    be a structure and let \(C \subseteq A\).
    Let \(\mbf{P}(x)\) be a property.
    If
    \begin{enumerate}[nolistsep, label=(\roman*), leftmargin=*]
        \ii \(\fall a \in C,\: \mbf{P}(a)\)
        \ii \(\fall j < n,\: \fall a \in A^{f_j},\:
            \big[ (a_0, \cdots, a_{f_j-1}) \in \dom F_j
            \land \fall i < f_j,\: \mbf{P}(a_i)
            \implies \mbf{P}(F_j(a_0, \cdots, a_{f_j-1})) \big]\)
    \end{enumerate}
    hold, then \(\fall x \in \cl{C},\: \mbf{P}(x)\).
}
\pf{Proof}{
    Let \(B = \{\,x \in A \mid \mbf{P}(x)\,\}\).
    (i) says \(C \subseteq B\) and (ii) says \(B\) is closed.
    Therefore, \(\cl{C} \subseteq B\).
}

\nt{
    \nameref{th:induction} is a special case of \nameref{th:generalInduction}
    for the structure \((\NN, S)\) where \(S\) is the successor function.
}

\subfile{../exercises/exercise3-5.tex}

\end{document}
