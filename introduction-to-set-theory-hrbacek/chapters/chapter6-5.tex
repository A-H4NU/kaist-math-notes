\documentclass[../introduction_to_set_theory_Note.tex]{subfiles}

\begin{document}

\section{Ordinal Arithmetic}

\dfn[ordinalAddition]{Addition of Ordinal Numbers}{
    For each ordinal \(\beta\):
    \begin{align}
        \label{eq:ordinalAdd1}\tag{1}
        &\beta + 0 = \beta. \\
        \label{eq:ordinalAdd2}\tag{2}
        &\beta + (\alpha + 1) = (\beta + \alpha) + 1\text{ for all ordinals }\alpha. \\
        \label{eq:ordinalAdd3}\tag{3}
        &\beta + \alpha = \sup \{\,\beta + \gamma \mid \gamma < \alpha \,\}
        \text{ for all limit ordinals }\alpha \neq 0.
    \end{align}
}

\nt{
    Let \(\mbf{G}_1\) and \(\mbf{G}_2\) be operations defined by
    \begin{align*}
        \mbf{G}_1(z, x) &= x + 1 \\
        \mbf{G}_2(z, x) &= \begin{cases}
            \bigcup \ran x & \text{if }x\text{ is a nonempty function} \\
            z & \text{otherwise.}
        \end{cases}
    \end{align*}
    Then, \Cref{th:secondTransRecursionParam} gives an operation \(\mbf{F}\) such that:
    for each ordinal \(\beta\),
    \begin{enumerate}[nolistsep, label=(\roman*), ref=\protect{(\roman*)}, listparindent=\parindent]
        \ii
        \(\mbf{F}(\beta, 0) = \mbf{G}_2(\beta, \OO) = \beta\).
        \ii
        \(\mbf{F}(\beta, \alpha + 1) = \mbf{G}_1(\beta, \mbf{F}(\beta, \alpha)) = \mbf{F}(\beta, \alpha) + 1\)
        for each ordinal \(\alpha\).
        \ii
        \(\mbf{F}(\beta, \alpha) = \mbf{G}_2\left(\beta, \restr{\mbf{F}_\beta}{\alpha}\right)
        = \bigcup \ran \left(\restr{\mbf{F}_\beta}{\alpha}\right)\)
        for each limit ordinal \(\alpha\).
    \end{enumerate}

    Let \(\mbf{P}(\alpha)\) be the property
    ``\(\mbf{F}(\beta, \alpha)\) is an ordinal for each ordinal \(\alpha\).''
    \begin{itemize}[nolistsep, listparindent=\parindent]
        \ii
        For each ordinal \(\alpha\), if \(\mbf{P}(\alpha)\) holds,
        then \(\mbf{F}(\beta, \alpha + 1) = \mbf{F}(\beta, \alpha) + 1\)
        is an ordinal for all ordinals \(\beta\)
        by \Cref{lem:succOfOrdinalIsOrdinal}.

        \ii
        For each limit ordinal \(\alpha\),
        if \(\mbf{P}(\beta)\) holds for all ordinals \(\beta < \alpha\),
        then \(\ran \left(\restr{\mbf{F}_\beta}{\alpha}\right)\)
        is a set of ordinals, and thus
        \(\mbf{F}(\beta, \alpha) = \bigcup \ran \left(\restr{\mbf{F}_\beta}{\alpha}\right)\)
        is an ordinal. (See the proof of \ref{itm:basicOrdinal.v}).
    \end{itemize}
    Therefore, by \nameref{th:secondTransInduction},
    \(\mbf{F}(\beta, \alpha)\) is an ordinal for each ordinal \(\alpha\) and \(\beta\).
    Hence, \Cref{dfn:ordinalAddition} is justified;
    \(\beta + \alpha\) is just a shortcut for \(\mbf{F}(\beta, \alpha)\).
}

\nt{
\begin{itemize}[nolistsep, leftmargin=*, listparindent=\parindent]
    \ii
    Addition of ordinals is not commutative:
    \[
        n + \omega = \sup \{\,n + m \mid n < \omega\,\} = \omega \neq \omega + n
        \text{ for all }n \in \omega \setminus \{0\}.
    \]
    \ii
    Addition of ordinals is not right-cancellative:
    \[
        1 + \omega = \omega = 2 + \omega\text{ but }1 \neq 2.
    \]
\end{itemize}
}

\thm[sumOfWosetAndOrdinal]{}{
    Let \((W_1, \le_1)\) and \((W_2, \le_2)\) be well-ordered sets,
    isomorphic to \(\alpha_1\) and \(\alpha_2\), respectively,
    and let \((W, \le)\) be the sum of \((W_1, \le_1)\) and \((W_2, \le_2)\).
    (See \Cref{lem:orderOnUnion}.)
    Then, \((W, \le)\) is isomorphic to \(\alpha_1 + \alpha_2\).
}
\mclm{Proof}{
    We may assume that \(W_1\) and \(W_2\) are disjoint, that \(W = W_1 \cup W_2\).
    We shall conduct \nameref{th:secondTransInduction} on \(\alpha_2\).
    \begin{itemize}[nolistsep, leftmargin=*, listparindent=\parindent]
        \ii
        Let \(\alpha_2 = \beta + 1\) for some ordinal \(\beta\)
        and assume the theorem holds whenever \(W_2\) is isomorphic to \(\beta\).
        \(W_2\) has a greatest element \(a\),
        which is also a greatest element of \(W\).
        Moreover, \(W_2[a]\) is isomorphic to \(\beta\),
        and thus \(W[a]\) is isomorphic to the sum of \(W_1\) and \(W_2[a]\).

        By the induction hypothesis,
        we have an isomorphism
        \(h \colon W[a] \hooktwoheadrightarrow \alpha_1 + \beta\).
        Then, \(h \cup \{(a, \alpha_1 + \beta)\}\) is an isomorphism between
        \(W\) and \((\alpha_1 + \beta) + 1 = \alpha_1 + (\beta + 1) = \alpha_1 + \alpha_2\).

        \ii
        Let \(\alpha_2\) is a limit ordinal
        and assume the theorem holds whenever \(W_2\) is isomorphic to an ordinal less than \(\alpha_2\).
        For each \(\beta < \alpha_2\), justified by \Cref{th:wosetComparable},
        let \(a_\beta\) be the unique element of \(W_2\)
        such that \(\beta\) is isomorphic to \(W_2[a_\beta]\).

        Then, by the induction hypothesis, for each \(\beta < \alpha_2\),
        noting that the sum of \(W_1\) and \(W_2[a_\beta]\) is isomorphic to \(W[a_\beta]\),
        we may let \(f_\beta\) to be the unique isomorphism between \(\alpha_1 + \beta\)
        and \(W[a_\beta]\). (Once again, uniqueness follows from \Cref{th:wosetComparable}.)
        Now, define \(f \triangleq \bigcup_{\beta < \alpha_2} f_{\beta}\).
        If \(\beta < \gamma < \alpha_2\),
        then \(\restr{f_\gamma}{\alpha_1 + \beta}\)
        is an isomorphism between \(\alpha_1 + \beta\) and \(W[a_\beta]\),
        and thus \(f_\beta \subsetneq f_\gamma\).
        Therefore, by \Cref{th:compatibleThenUnionIsFunction}, \(f\) is a function,
        and \(f\) is an isomorphism between \(\bigcup \{\,\alpha_1 + \beta \mid \beta < \alpha_2\,\}
        = \alpha_1 + \alpha_2\) and \(W\). \qed
    \end{itemize}
}
\mlemma[basicOrdinalArith]{}{
    Let \(\alpha\), \(\beta\), and \(\gamma\) be ordinals.
    \begin{enumerate}[nolistsep, label=(\roman*), ref=\protect{\Cref{lem:basicOrdinalArith} (\roman*)}]
        \ii\label{itm:basicOrdinalArith.i}
        \(\alpha < \beta \iff \gamma + \alpha < \gamma + \beta\)
        \ii\label{itm:basicOrdinalArith.ii}
        \(\alpha = \beta \iff \gamma + \alpha = \gamma + \beta\)
        \ii\label{itm:basicOrdinalArith.iii}
        \((\alpha + \beta) + \gamma = \alpha + (\beta + \gamma)\)
    \end{enumerate}
}
\mclm{Proof}{\hfill
\begin{enumerate}[nolistsep, label=(\roman*), leftmargin=*, listparindent=\parindent]
    \ii
    \begin{itemize}[nolistsep, wide=0pt, widest={[\(\Rightarrow\))}, leftmargin=*, listparindent=\parindent]
        \ii[(\(\Rightarrow\))]
        Fix any ordinal \(\beta\) and
        assume \(\alpha < \beta' \implies \gamma + \alpha < \gamma + \beta'\)
        for all ordinals \(\alpha\), \(\beta\), and \(\gamma\) with \(\beta' < \beta\)
        for the sake of induction. We shall assume \(\beta \neq 0\).
        Take any ordinals \(\alpha\) and \(\gamma\) such that \(\alpha < \beta\)
        If \(\beta = \delta + 1\), then \(\delta \ge \alpha\). (Otherwise, we have
        \(\delta < \alpha < \delta + 1\), which is impossible.) Then, we have
        \begin{alignat*}{2}
            \gamma + \alpha
            &\le \gamma + \delta &\qquad& \comment*{Induction Hypothesis} \\
            &< (\gamma + \delta) + 1 \\
            &= \gamma + (\delta + 1) && \comment*{\eqref{eq:ordinalAdd2}} \\
            &= \gamma + \beta.
        \end{alignat*}
        In the case of \(\beta\) being an limit ordinal,
        then \(\alpha + 1 < \beta\), and thus
        \begin{alignat*}{2}
            \gamma + \alpha
            &< (\gamma + \alpha) + 1 &\qquad& \\
            &= \gamma + (\alpha + 1) && \comment*{\eqref{eq:ordinalAdd2}} \\
            &\le \sup \{\,\gamma + \delta \mid \delta < \beta\,\} && \comment*{\(\alpha + 1 < \beta\)} \\
            &= \gamma + \beta. && \comment*{\eqref{eq:ordinalAdd3}}
        \end{alignat*}
        The result follows from \nameref{th:firstTransInduction}.
        \ii[(\(\Leftarrow\))]
        Assume \(\gamma + \alpha < \gamma + \beta\) where \(\alpha\), \(\beta\), and \(\gamma\)
        are ordinals.
        We clearly cannot have \(\alpha = \beta\).
        If \(\beta < \alpha\), then (\(\Rightarrow\))
        implies \(\gamma + \beta < \gamma + \alpha\),
        which is impossible.
        Hence, the only remaining option is \(\alpha < \beta\).
    \end{itemize}

    \ii
    By (i), \(\alpha < \beta\) or \(\alpha > \beta\) immediately implies
    \(\gamma + \alpha \neq \gamma + \beta\).
    Hence, the result follows.

    \ii
    We shall conduct the transfinite induction on \(\gamma\).
    Fix any ordinal \(\gamma\) and assume
    \((\alpha + \beta) + \gamma' = \alpha + (\beta + \gamma')\)
    for all ordinals \(\alpha\), \(\beta\), and \(\gamma'\) such that \(\gamma' < \gamma\).
    Note the case \(\gamma = 0\) is evident by \eqref{eq:ordinalAdd1}.

    If \(\gamma = \delta + 1\) for some ordinal \(\delta\), then
    \begin{alignat*}{2}
        (\alpha + \beta) + (\delta + 1)
        &= [ (\alpha + \beta) + \delta ] + 1 &\qquad& \comment*{\eqref{eq:ordinalAdd2}} \\
        &= [ \alpha + (\beta + \delta) ] + 1 && \comment*{Induction Hypothesis} \\
        &= \alpha + [(\beta + \delta) + 1] && \comment*{\eqref{eq:ordinalAdd2}} \\
        &= \alpha + [\beta + (\delta + 1)]. && \comment*{\eqref{eq:ordinalAdd2}}
    \end{alignat*}

    Now, assume \(\gamma\) is a limit ordinal other than \(0\).
    \clm[xcqeUAWE]{
        \(\beta + \gamma\) is a limit ordinal.
    }{
        If \(\xi < \beta + \gamma = \sup \{\,\beta + \delta \mid \delta < \gamma\,\}\),
        there exists \(\delta < \gamma\) such that \(\xi < \beta + \delta\).
        Then,
        \begin{alignat*}{2}
            \xi + 1 &< (\beta + \delta) + 1 &\qquad& \\
                    &= \beta + (\delta + 1) && \comment*{\eqref{eq:ordinalAdd2}} \\
                    &= \beta + \gamma. && \comment*{\(\delta + 1 < \gamma\), (i)}
        \end{alignat*}
        \qed
    }

    \clm[WpOrRgov]{
        \((\alpha + \beta) + \gamma \le \alpha + (\beta + \gamma)\)
    }{
        Take any \(\delta < \gamma\).
        Then, as \(\beta + \delta < \beta + \gamma\) by (i),
        \(\xi \triangleq (\beta + \delta) + 1\) satisfies \(\beta + \delta < \xi < \beta + \gamma\)
        thanks to \Cref{clm:xcqeUAWE}.

        Then,
        \begin{alignat*}{2}
            (\alpha + \beta) + \delta
            &= \alpha + (\beta + \delta) &\qquad& \comment*{Induction Hypothesis} \\
            &< \alpha + \xi. && \comment*{(i)}
        \end{alignat*}
        Hence, \(\sup \{\,(\alpha + \beta) + \delta \mid \delta < \gamma\,\}
        \le \sup \{\,\alpha + \xi \mid \xi < \beta + \gamma\,\}\). \qed
    }

    \clm[kqDYBUJN]{
        \(\alpha + (\beta + \gamma) \le (\alpha + \beta) + \gamma\)
    }{
        If \(\xi < \beta + \gamma\),
        then there exists \(\delta < \gamma\) such that \(\xi < \beta + \delta\)
        by \Cref{clm:xcqeUAWE};
        thus
        \begin{alignat*}{2}
            \alpha + \xi
            &< \alpha + (\beta + \delta) &\qquad& \comment*{(i)} \\
            &= (\alpha + \beta) + \delta. && \comment*{Induction Hypothesis}
        \end{alignat*}
        In other words, \(\sup \{\,\alpha + \xi \mid \xi < \beta + \gamma\,\}
        \le \sup \{\,(\alpha + \beta) + \delta \mid \delta < \gamma\,\}\). \qed
    }
    \noindent
    \Cref{clm:WpOrRgov,clm:kqDYBUJN} finishes the proof
    with the help of \nameref{th:firstTransInduction}. \qed
\end{enumerate}
}

\mlemma[ordinalUniqueDifference]{}{
    Let \(\alpha\) and \(\beta\) be ordinals.
    If \(\alpha \le \beta\), then there uniquely exists an ordinal \(\xi\)
    such that \(\alpha + \xi = \beta\).
}
\pf{Proof}{
    \(\beta \setminus \alpha = \{\,\nu \mid \alpha \le \nu < \beta\,\}\)
    is a well-ordered set by \ref{itm:basicOrdinal.iv}.
    There exists an ordinal \(\xi\) which is isomorphic to \(\beta \setminus \alpha\).
    Moreover, the sum of \(\alpha\) and \(\beta \setminus \alpha\)
    is \(\beta\); hence \(\alpha + \xi = \beta\) by
    \Cref{th:sumOfWosetAndOrdinal}.
    The uniqueness follows from \ref{itm:basicOrdinalArith.ii}.
}

\dfn[ordinalMultiplication]{Multiplication Of Ordinal Numbers}{
    For each ordinal \(\beta\):
    \begin{align}
        \label{eq:ordinalMult1}\tag{4}
        &\beta \cdot 0 = \beta. \\
        \label{eq:ordinalMult2}\tag{5}
        &\beta \cdot (\alpha + 1) = \beta \cdot \alpha + \beta\text{ for all ordinals }\alpha. \\
        \label{eq:ordinalMult3}\tag{6}
        &\beta \cdot \alpha = \sup \{\,\beta \cdot \gamma \mid \gamma < \alpha \,\}
        \text{ for all limit ordinals }\alpha \neq 0.
    \end{align}
}

\nt{
    Let \(\mbf{G}_1\) and \(\mbf{G}_2\) be operations defined by
    \begin{align*}
        \mbf{G}_1(z, x) &= \begin{cases}
            x + z & \text{if }x\text{ and }z\text{ are ordinals} \\
            \OO & \text{ otherwise}
        \end{cases} \\
        \mbf{G}_2(z, x) &= \begin{cases}
            \bigcup \ran x & \text{if }x\text{ is a nonempty function} \\
            0 & \text{otherwise.}
        \end{cases}
    \end{align*}
    Then, \Cref{th:secondTransRecursionParam} gives an operation \(\mbf{F}(\beta, \alpha)\)
    that justifies \Cref{dfn:ordinalMultiplication}.
}

\thm[orderTypeOfProduct]{}{
    Let \(\alpha\) and \(\beta\) be ordinal numbers.
    Then, the order type of the lexicographic ordering of \(\beta \times \alpha\) is \(\alpha \cdot \beta\).
}
\pf{Proof}{
    We shall conduct \nameref{th:secondTransInduction} on \(\beta\).
    Let \(\mbf{P}(\beta)\) be the property that
    \begin{center}
    ``for all ordinals \(\alpha\), the function \(h\) on \(\beta \times \alpha\)
    defined by \((\eta, \xi) \mapsto \alpha \cdot \eta + \xi\) \\
    is an isomorphism between \(\beta \times \alpha\) and \(\alpha \cdot \beta\).''
    \end{center}
    \(\mbf{P}(0)\) evidently holds.

    Take any ordinals \(\alpha\) and  \(\beta\) and assume that \(\mbf{P}(\beta)\) holds.
    Let \(h'\) be the function on \((\beta + 1) \times \alpha\)
    defined by \((\eta, \xi) \mapsto \alpha \cdot \eta + \xi\).
    Then, \(\restr{h'}{\beta \times \alpha}\) is the
    isomorphism between \(\beta \times \alpha\) and \(\alpha \cdot \beta\) by \(\mbf{P}(\beta)\).
    For all ordinals \(\eta < \beta\) and \(\xi_1, \xi_2 < \alpha\),
    \begin{alignat*}{2}
        \alpha \cdot \eta + \xi_1
        &< \alpha \cdot \eta + \alpha &\qquad& \comment*{\ref{itm:basicOrdinalArith.i}} \\
        &= \alpha \cdot (\eta + 1) && \comment*{\eqref{eq:ordinalMult2}} \\
        &\le \alpha \cdot \beta && \comment*{\ref{itm:6.5.8.i}} \\
        &\le \alpha \cdot \beta + \xi_2; && \comment*{\ref{itm:basicOrdinalArith.i}} \\
        \shortintertext{moreover,}
        \alpha \cdot \beta + \xi_1
        &< \alpha \cdot \beta + \xi_2 && \comment*{\ref{itm:basicOrdinalArith.i}} \\
        & \qquad\text{if }\xi_1 < \xi_2.
    \end{alignat*}
    Hence, \(h\) is an isomorphism between \((\beta + 1) \times \alpha\) and \(\ran h\).
    It is evident that \(\alpha \cdot \beta \subseteq \ran h \subseteq \alpha \cdot (\beta + 1)\).

    Now, take any \(\gamma\) such that \(\alpha \cdot \beta \le \gamma < \alpha \cdot (\beta + 1)\).
    Then, by \Cref{lem:ordinalUniqueDifference}, there exists an ordinal \(\xi\)
    such that \(\gamma = \alpha \cdot \beta + \xi\).
    Then, by \ref{itm:noIsoToInitSeg.i}, \(\xi < \beta\);
    hence \(\gamma = h'(\beta, \xi)\);
    thus \(\ran h = \alpha \cdot (\beta + 1)\).

    Now, take any ordinal \(\beta\) and assume that
    \(\mbf{P}(\beta')\) holds for all \(\beta' < \beta\).
    Take any ordinal \(\alpha\).
    By the inductive assumption, \nameref{ax:replacement}, and \nameref{th:counting},
    there exists a set
    \[
        H \triangleq \{\,h_{\beta'} \mid h_{\beta'}
        \text{ is an isomorphism between }\beta' \times \alpha\text{ and }\alpha \cdot \beta'
        \text{ where }\beta' < \beta\,\}.
    \]
    Let \(h' \triangleq \bigcup H\).
    Then, by the inductive assumption, \(H\) is a compatible system of functions,
    hence \(h'\) is a function with \(\dom h' = \bigcup_{\beta' < \beta} \beta' \times \alpha
    = \beta \times \alpha\) by \Cref{th:compatibleThenUnionIsFunction}.
    Moreover, \(\ran h' = \bigcup_{\beta' < \beta} \alpha \cdot \beta' = \alpha \cdot \beta\).
    It is evident that \(h'\) is a isomorphism.
    The result follows from \nameref{th:secondTransInduction}.
}

\dfn[ordinalExponentiation]{Exponentiation of Ordinal Numbers}{
    For each ordinal \(\beta\):
    \begin{align}
        \label{eq:ordinalExpo1}\tag{7}
        &\beta^0 = 1. \\
        \label{eq:ordinalExpo2}\tag{8}
        &\beta^{\alpha + 1} = \beta^\alpha \cdot \beta\text{ for all ordinals }\alpha. \\
        \label{eq:ordinalExpo3}\tag{9}
        &\beta^\alpha = \sup \{\,\beta^\gamma \mid 0 < \gamma < \alpha \,\}
        \text{ for all limit ordinals }\alpha \neq 0.
    \end{align}
    \let\thefootnote\relax\footnotetext{
        In the textbook, it defines \(\beta^\alpha = \sup \{\,\beta^\gamma \mid \gamma < \alpha \,\}\)
        for limit ordinals, which results in
        \(0^\omega = \sup \{\,0^0, 0^1, 0^2, \cdots\,\} = \sup \{0, 1\} = 1\).
        I considered this as a mistake or a typo, so we just
        define in \href{https://en.wikipedia.org/wiki/Ordinal_arithmetic\#Exponentiation}{this way}.
    }
}

\nt{
    Let \(\mbf{G}_1\) and \(\mbf{G}_2\) be operations defined by
    \begin{align*}
        \mbf{G}_1(z, x) &= \begin{cases}
            x \cdot z & \text{if }x\text{ and }z\text{ are ordinals} \\
            \OO & \text{ otherwise}
        \end{cases} \\
        \mbf{G}_2(z, x) &= \begin{cases}
            \bigcup \ran x & \text{if }x\text{ is a nonempty function} \\
            1 & \text{otherwise.}
        \end{cases}
    \end{align*}
    Then, \Cref{th:secondTransRecursionParam} gives an operation \(\mbf{F}(\beta, \alpha)\)
    that justifies \Cref{dfn:ordinalExponentiation}.
}

\nt{
    \noindent
    Ordinal arithmetic differs from arithmetic of cardinals.
    \begin{itemize}[nolistsep, leftmargin=*, listparindent=\parindent]
        \ii
        \(2^{\aleph_0} > \aleph_0\) but \(2^\omega = \omega\).
        \ii
        \(\aleph_0^n = \aleph_0\) but \(\omega^n > \omega\).
        \ii
        \(\omega^\omega\) is countable.
        (It is an order type of lexicographic ordering of \(\Seq(\NN)\).)
    \end{itemize}
}

\nt{
    One can use arithmetic operations to generate larger and larger ordinals:
    \begin{gather*}
        0, 1, \cdots, \omega, \omega + 1, \cdots, \omega \cdot 2, \cdots \omega \cdot 3, \cdots, \\
        \omega \cdot \omega = \omega^2, \cdots, \omega^3, \cdots, \omega^\omega, \cdots, \omega^\omega, \cdots,
        \omega^{\omega^2}, \cdots, \omega^{\omega^3}, \cdots, \omega^{\omega^\omega}, \cdots
    \end{gather*}
    We define
    \[
        \veps \triangleq \sup \{\,\omega, \omega^\omega, \omega^{\omega^\omega}, \omega^{\omega^{\omega^\omega}}, \cdots\,\}
    \]
    and then form \(\veps + 1\), \(\veps + \omega\), \(\veps^\omega\), \(\veps^\veps\),
    \(\veps^{\veps^\veps}\), etc.
}

\subfile{../exercises/exercise6-5.tex}

\end{document}
