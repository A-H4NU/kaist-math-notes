\documentclass[../introduction_to_set_theory_Note.tex]{subfiles}

\begin{document}

\section{Transfinite Induction and Recursion}

\thm[firstTransInduction]{The Transfinite Induction Principle: First Version}{
    Let \(\mbf{P}(x)\) be a property.
    Assume that, for each ordinal number \(\alpha\),
    \eqref{eq:transInduction} holds.
    Then, \(\mbf{P}(\alpha)\) holds for all ordinals \(\alpha\). \begin{align}
        \text{If }\mbf{P}(\beta)\text{ holds for each ordinal }\beta\text{ such that }\beta < \alpha
        \text{, then }\mbf{P}(\alpha)\text{ holds}.
        \label{eq:transInduction}\tag{1}
    \end{align}
}
\pf{Proof}{
    Suppose there exists an ordinal \(\gamma\) such that \(\lnot \mbf{P}(\gamma)\)
    for the sake of contradiction.
    Let \(S \triangleq \{\,\beta \mid \beta\text{ is an ordinal such that }
    \beta \le \alpha\text{ and }\lnot\mbf{P}(\beta)\,\}\).
    As \(S \neq \OO\), by \ref{itm:basicOrdinal.iv},
    \(S\) has a least element \(\alpha\).
    As every ordinal \(\beta < \alpha\) satisfies \(\mbf{P}(\beta)\),
    by assumption, we must have \(\mbf{P}(\alpha)\) by \eqref{eq:transInduction},
    which contradicts \(\alpha \in S\).
}

\thm[secondTransInduction]{The Transfinite Induction Principle: Second Version}{
    Let \(\mbf{P}(x)\) be a property. Assume that:
    \begin{enumerate}[nolistsep, label=(\roman*), listparindent=\parindent]
        \ii For each ordinal \(\alpha\), \(\mbf{P}(\alpha) \implies \mbf{P}(\alpha + 1)\).
        \ii For each limit ordinal \(\alpha\), if \(\mbf{P}(\beta)\) holds for each ordinal
            \(\beta\) such that \(\beta < \alpha\), then \(\mbf{P}(\alpha)\) holds.
    \end{enumerate}
    Then, \(\mbf{P}(\alpha)\) holds for all ordinals \(\alpha\).
}
\pf{Proof}{
    Take any ordinal \(\alpha\) and assume that
    \(\mbf{P}(\beta)\) holds for all ordinals less than \(\alpha\).
    If \(\alpha\) is a limit ordinal, then \(\mbf{P}(\alpha)\) holds by (ii).
    Otherwise, i.e., if \(\alpha = \beta + 1\) for some ordinal \(\beta\),
    then, by (i), \(\mbf{P}(\alpha)\) holds as \(\mbf{P}(\beta)\) holds.
    Therefore, \eqref{eq:transInduction} holds.
    The result follows from \nameref{th:firstTransInduction}.
}

\dfn[transfiniteSequence]{Transfinite Sequence}{
    A function whose domain is an ordinal \(\alpha\)
    is called \textit{transfinite sequence of length \(\alpha\)}.
}

\thm[transRecursion]{The Transfinite Recursion Theorem}{
    Let \(\mbf{G}\) be an operation.
    Then, the property \(\mbf{P}(x, y)\) stated in \eqref{eq:transRecursion} defines an operation \(\mbf{F}\)
    such that \(\fall \alpha \in \Ord,\: \mbf{F}(\alpha) = \mbf{G} \left(\restr{\mbf{F}}{\alpha}\right)\)
    where \(\restr{\mbf{F}}{\alpha}\) is the transfinite sequence
    of length \(\alpha\) defined by \(x \mapsto \mbf{F}(x)\).

    \begin{equation}
        \begin{aligned}
            & \mbf{P}(x, y) \text{ is the property defined by:} \\
            & \quad\left\{
            \begin{aligned}
                & \begin{aligned}
                    &x \in \Ord\text{ and }y = t(x)\text{ for some computation }t \\
                    &\qquad\text{ of length }x\text{ based on }\mbf{G},
                \end{aligned} \\
                \text{or }& x \notin \Ord \land y = \OO
            \end{aligned}
            \right.
        \end{aligned}
        \label{eq:transRecursion}\tag{2}
    \end{equation}
    where \(t\) is called a \textit{computation of length \(\alpha\) based on \(\mbf{G}\)}
    if \(t\) is a function whose domain is \(\alpha + 1\)
    and for all \(\beta \le \alpha\), \(t(\beta) = \mbf{G}\left(\restr{t}{\beta}\right)\).
}
\pf{Proof}{
    We first have to show that \(\mbf{P}(x, y)\) defined in \eqref{eq:transRecursion}
    defines an operation.
    \clm[bfvdLoYw]{
        For any ordinal \(\alpha\),
        there uniquely exists a computation of length \(\alpha\) based on \(\mbf{G}\).
    }{
        Fix any ordinal \(\alpha\) and assume for the sake of induction that:
        \begin{equation}
            \begin{gathered}
                \text{For each ordinal \(\beta\) with \(\beta < \alpha\),} \\
                \text{there uniquely exists a computation of length \(\beta\)}.
            \end{gathered}
            \label{eq:pnZedvFP}\tag{\(\ast\)}
        \end{equation}
        Then, by \nameref{ax:replacement}, a set
        \[
            T \triangleq \{\,t \mid t\text{ is a computation of length }\beta\text{ for some }\beta < \alpha\,\}
        \]
        exists.
        Note that, for each \(\beta < \alpha\),
        there uniquely exists \(t \in T\) whose length is \(\beta\) by \eqref{eq:pnZedvFP}.
        Let \(\hat t \triangleq \bigcup T\) and
        \(\tau \triangleq \hat t \cup \{(\alpha, \mbf{G}(\hat t))\}\).

        \clmwt[uRkAHbWC]{
            \(\tau\) is a function and \(\dom \tau = \alpha + 1\).
        }{
            We have to show that \(T\) is a compatible system of functions.
            First, take any \(t_1, t_2 \in T\) and let
            \(\beta_1 \triangleq \dom t_1\) and \(\beta_2 \triangleq \dom t_2\).
            \WLOG, by \ref{itm:basicOrdinal.iii}, \(\beta_1 \le \beta_2\).

            For the sake of transfinite induction, fix any ordinal \(\gamma\) with \(\gamma < \beta_1\)
            and assume that \(t_1(\delta) = t_2(\delta)\) for all ordinals \(\delta < \gamma\).
            Then, \(\restr{t_1}{\gamma} = \restr{t_2}{\gamma}\);
            thus \(t_1(\gamma) = \mbf{G}\left(\restr{t_1}{\gamma}\right)
            = \mbf{G}\left(\restr{t_2}{\gamma}\right) = t_2(\gamma)\).
            Hence, by \nameref{th:firstTransInduction},
            \(t_1(\gamma) = t_2(\gamma)\) for all ordinals \(\gamma < \beta_1\).

            Hence, \(\hat t\) is a function by \Cref{th:compatibleThenUnionIsFunction} and
            \begin{alignat*}{2}
                \dom \hat t &= \textstyle\bigcup_{t \in T} \dom t &\qquad& \comment*{\Cref{th:compatibleThenUnionIsFunction}} \\
                            &= \textstyle\bigcup_{\beta < \alpha} (\beta + 1) && \comment*{\eqref{eq:pnZedvFP}} \\
                            &= \alpha.
            \end{alignat*}
            Therefore, by \Cref{th:compatibleThenUnionIsFunction},
            \(\tau\) is a function and \(\dom \tau = \alpha \cup \{\alpha\} = \alpha + 1\). \qed
        }

        \clmwt[vEQSOsaD]{
            For each ordinal \(\beta\) with \(\beta \le \alpha\),
            \(\tau(\beta) = \mbf{G}\left(\restr{\tau}{\beta}\right)\).
        }{
            If \(\beta = \alpha\),
            we immediately have \(\tau(\alpha) = \mbf{G}(\hat t) = \mbf{G}\left(\restr{\tau}{\alpha}\right)\).
            If \(\beta < \alpha\), then take \(t \in T\) with \(\dom t = \beta + 1\).
            Then, we have \(\tau(\beta) = t(\beta) = \mbf{G}\left(\restr{t}{\beta}\right)
            = \mbf{G}\left(\restr{\tau}{\beta}\right)\) as \(t \subseteq \tau\). \qed
        }

        \noindent
        \Cref{clm:uRkAHbWC,clm:vEQSOsaD} asserts that \(\tau\) is a computation
        of length \(\alpha\) based on \(\mbf{G}\).

        To prove the uniqueness, let \(\sigma\) be another computation of length \(\alpha\).
        Note that, if \(\tau(\delta) = \sigma(\delta)\) for all \(\delta < \gamma\),
        then \(\tau(\gamma) = \mbf{G}\left(\restr{\tau}{\gamma}\right)
        = \mbf{G}\left(\restr{\sigma}{\gamma}\right) = \sigma(\gamma)\).
        Hence, by \nameref{th:firstTransInduction},
        \(\tau(\gamma) = \sigma(\gamma)\) for all \(\gamma \le \alpha\);
        thus \(\tau = \sigma\).

        Hence, we showed that \eqref{eq:pnZedvFP}
        implies, there uniquely exists a computation of length \(\alpha\).
        Hence, by \nameref{th:firstTransInduction},
        for all ordinals \(\alpha\), there uniquely exists \(y\)
        that satisfies \(\mbf{P}(\alpha, y)\). \qed
    }

    By \Cref{clm:bfvdLoYw}, \(\fall x\:\exs! y\: \mbf{P}(x, y)\);
    \(\mbf{P}\) defines an operation \(\mbf{F}\).
    (In other words, \(\mbf{F}(\alpha) = t(\alpha)\) where \(t\)
    is the unique computation of length \(\alpha\) based on \(\mbf{G}\).)

    Fix any ordinal \(\alpha\).
    Let \(t\) be the unique computation of length \(\alpha\).
    Then, for each \(\gamma < \alpha\), \(\restr{t}{\gamma + 1}\)
    is a computation of length \(\gamma\),
    and thus \(\mbf{F}(\gamma) = \restr{t}{\gamma + 1}(\gamma) = t(\gamma)\);
    in other words, \(\restr{\mbf{F}}{\alpha} = \restr{t}{\alpha}\).
    Then,
    \begin{alignat*}{2}\SwapAboveDisplaySkip
        \mbf{F}(\alpha)
        &= t(\alpha) &\qquad& \comment*{Definition of \(\mbf{F}\)} \\
        &= \mbf{G}\left(\restr{t}{\alpha}\right) && \comment*{\(t\) is a computation} \\
        &= \mbf{G}\left(\restr{\mbf{F}}{\alpha}\right). && \comment*{\(\restr{\mbf{F}}{\alpha} = \restr{t}{\alpha}\)}
    \end{alignat*}
    The theorem is now proven.
}

\notat{}{
    If \(\mbf{F}(z, x)\) is an operation in two variables,
    i.e., \(\mbf{P}(z, x, y)\) is an property such that
    \(\fall z\: \fall x\: \exs! y\: \mbf{P}(z, x, y)\)
    and \(\mbf{F}(z, x)\) denotes the unique \(y\)
    such that \(\mbf{P}(z, x, y)\).
    We write \(\mbf{F}_z(x)\) in place of \(\mbf{F}(z, x)\)
    so that we may treat \(\mbf{F}_z\) as an operation.
}

\thm[transRecursionParam]{The Transfinite Recursion Theorem: Parametric Version}{
    Let \(\mbf{G}\) be an operation.
    Then, the property \(\mbf{Q}(z, x, y)\) stated in \eqref{eq:transRecursionParam}
    defines an operation \(\mbf{F}\)
    such that \(\fall z,\: \fall \alpha \in \Ord,\: \mbf{F}(z, \alpha)
    = \mbf{G}\left(z, \restr{\mbf{F}_z}{\alpha}\right)\).

    \begin{equation}
        \begin{aligned}
            & \mbf{Q}(z, x, y)\text{ is the property defined by:} \\
            & \quad\left\{
            \begin{aligned}
                & \begin{aligned}
                    &x \in \Ord\text{ and }y = t(x)\text{ for some computation }t \\
                    &\qquad\text{ of length }x\text{ based on }\mbf{G}\text{ and }z,
                \end{aligned} \\
                \text{or }& x \notin \Ord \land y = \OO
            \end{aligned}
            \right.
        \end{aligned}
        \label{eq:transRecursionParam}\tag{3}
    \end{equation}
    where \(t\) is called a \textit{computation of length \(\alpha\) based on \(\mbf{G}\) and \(z\)}
    if \(t\) is a function whose domain is \(\alpha + 1\)
    and for all \(\beta \le \alpha\), \(t(\beta) = \mbf{G}\left(z, \restr{t}{\beta}\right)\).
}
\pf{Proof}{
    Replace every \(\mbf{G}(-)\) into \(\mbf{G}(z, -)\)
    in the proof of \nameref{th:transRecursion}.
}

\nt{
    \noindent
    We now present some variations of
    \nameref{th:transRecursion} and \nameref{th:transRecursionParam}.
}

\thm[secondTransRecursion]{}{
    Let \(\mbf{G}_1\) and \(\mbf{G}_2\) be operations.
    Let \(\mbf{G}\) be the operation defined in \eqref{eq:secondTransRecursion}.
    Then, the property \(\mbf{P}(x, y)\) stated in \eqref{eq:transRecursion} defines an operation \(\mbf{F}\)
    such that:
    \begin{enumerate}[nolistsep, label=(\roman*), ref=\protect{(\roman*)}, listparindent=\parindent]
        \ii For each ordinal \(\alpha\), \(\mbf{F}(\alpha + 1) = \mbf{G}_1(\mbf{F}(\alpha))\).\vspace*{.3em}
        \ii For each limit ordinal \(\alpha\), \(\mbf{F}(\alpha) = \mbf{G}_2\left(\restr{\mbf{F}}{\alpha}\right)\).
    \end{enumerate}\vspace*{1em}
    \begin{equation}
        \begin{aligned}
            & y = \mbf{G}(x) \text{ if and only if} \\
            & \quad\left\{
            \begin{aligned}
                          & x\text{ is a transfinite sequence of length }\alpha + 1\text{ and }y = \mbf{G}_1(x(\alpha)), \\
                \text{or }&
                \begin{aligned}[t]
                    & x\text{ is a transfinite sequence of length }\alpha
                    \text{ such that }\alpha\text{ is a limit ordinal} \\
                    &\qquad\text{and }y = \mbf{G}_2(x), \\
                \end{aligned} \\
                \text{or }& x\text{ is not a transfinite sequence and }y = \OO. \\
            \end{aligned}
            \right.
        \end{aligned}
        \label{eq:secondTransRecursion}\tag{4}
    \end{equation}
}
\mclm{Proof}{
    For each ordinal \(\alpha\), \(\mbf{F}(\alpha) = \mbf{G}\left(\restr{\mbf{F}}{\alpha}\right)\)
    by \nameref{th:transRecursion}.\vspace*{.3em}
    \begin{itemize}[nolistsep, leftmargin=*, listparindent=\parindent]
        \ii
        If \(\alpha\) is an ordinal,
        then \(\mbf{F}(\alpha + 1) = \mbf{G}\left(\restr{\mbf{F}}{\alpha + 1}\right)
        = \mbf{G}_1\left(\restr{\mbf{F}}{\alpha + 1}(\alpha)\right)
        = \mbf{G}_1\left(\mbf{F}(\alpha)\right)\).\vspace*{.3em}
        \ii
        If \(\alpha\) is a limit ordinal,
        then \(\mbf{F}(\alpha) = \mbf{G}\left(\restr{\mbf{F}}{\alpha}\right)
        = \mbf{G}_2\left(\restr{\mbf{F}}{\alpha}\right)\). \qed
    \end{itemize}
}

\thm[secondTransRecursionParam]{}{
    Let \(\mbf{G}_1\) and \(\mbf{G}_2\) be operations.
    Let \(\mbf{G}\) be the operation defined in \eqref{eq:secondTransRecursionParam}.
    Then, the property \(\mbf{Q}(z, x, y)\) stated in \eqref{eq:transRecursionParam} defines an operation \(\mbf{F}\)
    such that:
    \begin{enumerate}[nolistsep, label=(\roman*), ref=\protect{(\roman*)}, listparindent=\parindent]
        \ii
        For each ordinal \(\alpha\) and each set \(z\),
        \(\mbf{F}(z, \alpha + 1) = \mbf{G}_1(z, \mbf{F}(z, \alpha))\).\vspace*{.3em}
        \ii
        For each limit ordinal \(\alpha\) and each set \(z\),
        \(\mbf{F}(z, \alpha) = \mbf{G}_2\left(z, \restr{\mbf{F}_z}{\alpha}\right)\).
    \end{enumerate}\vspace*{1em}
    \begin{equation}
        \begin{aligned}
            & y = \mbf{G}(z, x) \text{ if and only if} \\
            & \quad\left\{
            \begin{aligned}
                          & x\text{ is a transfinite sequence of length }\alpha + 1\text{ and }y = \mbf{G}_1(z, x(\alpha)), \\
                \text{or }&
                \begin{aligned}[t]
                    & x\text{ is a transfinite sequence of length }\alpha
                    \text{ such that }\alpha\text{ is a limit ordinal} \\
                    &\qquad\text{and }y = \mbf{G}_2(z, x), \\
                \end{aligned} \\
                \text{or }& x\text{ is not a transfinite sequence and }y = \OO. \\
            \end{aligned}
            \right.
        \end{aligned}
        \label{eq:secondTransRecursionParam}\tag{5}
    \end{equation}
}
\mclm{Proof}{
    For each ordinal \(\alpha\) and each set \(z\),
    \(\mbf{F}(z, \alpha) = \mbf{G}\left(z, \restr{\mbf{F}_z}{\alpha}\right)\)
    by \nameref{th:transRecursionParam}.\vspace*{.3em}
    Then, for each set \(z\):
    \begin{itemize}[nolistsep, leftmargin=*, listparindent=\parindent]
        \ii
        If \(\alpha\) is an ordinal,
        then \(\mbf{F}(z, \alpha + 1) = \mbf{G}\left(z, \restr{\mbf{F}_z}{\alpha + 1}\right)
        = \mbf{G}_1\left(z, \restr{\mbf{F}_z}{\alpha + 1}(\alpha)\right)
        = \mbf{G}_1(z, \mbf{F}(z, \alpha))\).\vspace*{.3em}
        \ii
        If \(\alpha\) is a limit ordinal,
        then \(\mbf{F}(z, \alpha) = \mbf{G}\left(z, \restr{\mbf{F}_z}{\alpha}\right)
        = \mbf{G}_2\left(z, \restr{\mbf{F}_z}{\alpha}\right)\). \qed
    \end{itemize}
}

\cor[]{}{
    Let \(\Omega\) be an ordinal number, \(A\) be a set,
    and \(S = \bigcup_{\alpha < \Omega} A^{\alpha}\)
    be the set of all transfinite sequences of elements of \(A\) of length less than \(\Omega\).
    Let \(g \colon S \to A\).
    Then, there exists a unique function \(f \colon \Omega \to A\) such that
    \[
        f(\alpha) = g \left(\restr{f}{\alpha}\right)\text{ for all ordinals }\alpha < \Omega.
    \]
}
\pf{Proof}{
    Define an operation \(\mbf{G}\) by
    \[
        \mbf{G}(t) \triangleq \begin{cases}
            g(t) & \text{if } t \in S \\
            \OO  & \text{otherwise.}
        \end{cases}
    \]
    Then, by \nameref{th:transRecursion},
    there exists an operation \(\mbf{F}\) such that
    \(\mbf{F}(\alpha) = \mbf{G}\left(\restr{\mbf{F}}{\alpha}\right)\)
    holds for all ordinals \(\alpha\).
    Then, \(f \triangleq \restr{\mbf{F}}{\Omega}\)
    satisfies the condition.
}

\mclm{\hypertarget{yinFUHIP}{Proof} of {\rm\nameref{th:operationRecursion}}}{
    Define an operation \(\mbf{G}'\) by
    \[
        \mbf{G}'(x) \triangleq \begin{cases}
            \mbf{G}(x_n, n) & \text{if }x\text{ is a sequence of length }n + 1\text{ where }n \in \omega\\
            a  & \text{otherwise.}
        \end{cases}
    \]
    By \nameref{th:transRecursion},
    there exists an operation \(\mbf{F}\) such that
    \(\mbf{F}(\alpha) = \mbf{G}'\left(\restr{\mbf{F}}{\alpha}\right)\)
    for all ordinals \(\alpha\).
    Now, let \(\lang a_n \rang_{n \in \NN} \triangleq \restr{\mbf{F}}{\omega}\).
    Then,
    \begin{itemize}[nolistsep, leftmargin=*, listparindent=\parindent]
        \ii \(a_0 = \mbf{F}(0) = \mbf{G}'(\OO) = a\)
        \ii For each \(n \in \NN\),
        \(a_{n+1} = \mbf{F}(n+1) = \mbf{G}'\left(\restr{\mbf{F}}{n+1}\right) =
        \mbf{G}\left(\restr{\mbf{F}}{n+1}(n), n\right)
        = \mbf{G}\left(\mbf{F}(n), n\right) = \mbf{G}(a_n, n)\). \qed
    \end{itemize}
}

\end{document}
