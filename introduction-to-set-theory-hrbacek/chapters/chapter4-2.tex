\documentclass[../introduction_to_set_theory.tex]{subfiles}

\begin{document}

\section{Finite Sets}

\dfn[finite]{Finite Set and Infinite Set}{
    A set \(S\) is \textit{finite} if it is equipotent to some natural number \(n \in \NN\).
    We then define \(|S| = n\) and say \textit{\(S\) has \(n\) elements.}
    A set is \textit{infinite} if it is not finite.
}

\nt{
    According to \Cref{dfn:finite}, cardinal numbers of finite sets
    are the natural numbers.
    We evidently have \(\fall n \in \NN,\: |n| = n\).
}

\mlemma[finiteProperSubset]{}{
    If \(n \in \NN\) and \(X \subsetneq n\),
    then there is no \(f \colon n \hooktwoheadrightarrow X\).
}
\pf{Proof}{
    If \(n = 0\), there is no \(X \subsetneq n\); the assertion is true.

    Assume the assertion holds for \(n\).
    Suppose there is some \(f \colon (n + 1) \hooktwoheadrightarrow X\)
    where \(X \subsetneq n + 1\).
    There are two cases: \(n \in X\) and \(n \notin X\).

    If \(n \notin X\), then \(X \subseteq n\), and thus
    \(\restr{f}{n} \colon n \hooktwoheadrightarrow X \setminus \{f(n)\}\);
    however \(X \setminus \{f(n)\} \subsetneq X \subseteq n\),
    which is a contradiction.

    If \(n \in X\), then \(n = f(k)\) for some \(k \le n\).
    Define a function \(g\) on \(n\) by following:
    \[
        g(i) = \begin{cases}
            f(n) & \text{if } i = k < n \\
            f(i) & \text{otherwise.}
        \end{cases}
    \]
    Then, \(g \colon n \hooktwoheadrightarrow X \setminus \{n\}\)
    and \(X \setminus \{n\} \subsetneq n\), which is also a contradiction.
}

\cor[basicFinite]{}{
\begin{enumerate}[nolistsep, label=(\roman*), ref=\protect{\Cref{cor:basicFinite} (\roman*)}, listparindent=\parindent]
    \ii
    If \(m \neq n\) where \(m, n \in \NN\), then there is no \(f \colon m \hooktwoheadrightarrow n\).
    \ii
    If \(|S| = m\) and \(|S| = n\), then \(m = n\).
    \ii\label{itm:basicFinite.iii}
    \(\NN\) is infinite.
\end{enumerate}
}
\mclm{Proof}{\hfill
\begin{enumerate}[nolistsep, label=(\roman*), leftmargin=*, listparindent=\parindent]
    \ii
    If \(n \neq m\), by \nameref{th:NisLinearlyOrdered},
    we have \(n \subsetneq m\) or \(m \subsetneq n\).
    In either case, we do not have such function by \Cref{lem:finiteProperSubset}.

    \ii
    By \Cref{lem:basicEquipotent},
    we have \(|m| = |n|\). (i) asserts that \(m = n\);
    otherwise we cannot have \(|m| = |n|\).

    \ii
    By \Cref{exer:3.2.3}, there exists \(f \colon \NN \hooktwoheadrightarrow X\)
    where \(X \subsetneq \NN\).
    If there exists \(n \in \NN\) and \(g \colon n \hooktwoheadrightarrow \NN\),
    \(g\inv \circ f\inv \circ f \circ g\) is a function on \(n\) onto a proper subset of \(n\).
    This contradicts \Cref{lem:finiteProperSubset}.
    \qed
\end{enumerate}
}

\thm[subsetOfFiniteIsFinite]{}{
    If \(X\) is a finite set and \(Y \subseteq X\), then \(Y\) is finite.
}
\pf{Proof}{
    We may assume \(X = \{\,x_0, \cdots, x_{n-1}\,\}\), where \(\lang\,x_0, \cdots, x_{n-1}\,\rang\)
    is an injective sequence, and \(Y \neq \OO\).

    Let \(g\) be a function on a subset of \(n \times \NN\) into \(n\)
    defined by
    \begin{equation}
        g(a, \texttt{-}) = \begin{cases}
            \min \{\,j \in n \mid a < j \land x_j \in Y\,\} & \text{if it exists} \\
            \text{undefined} & \text{otherwise}.
        \end{cases}
        \label{eq:YjMjSKBE} \tag{\(\ast\)}
    \end{equation}
    By \nameref{exer:3.3.5},
    there exists a sequence \(k\) of elements in \(n\) such that
    \begin{enumerate}[nolistsep, label=(\roman*), listparindent=\parindent]
        \ii \(k_0 = \min \{\,j \in n \mid x_j \in Y\,\}\). \comment{\(Y \neq \OO\)}
        \ii \(\fall i \in \NN,\: [i + 1 \in \dom k \implies k_{i+1} = g(k_i, i)
            = \min \{\,j \in n \mid k_i < j \land x_j \in Y\,\}]\).
        \ii \(k\) is either an infinite sequence or a finite sequence of length \(\ell + 1\)
            and \((k_\ell, \ell) \notin \dom g\).
    \end{enumerate}

    By (ii) and \eqref{eq:YjMjSKBE}, \(\fall i \in \NN,\: (i + 1 \in \dom k \implies k_i < k_{i+1})\).
    Hence, \(k\) is injective.
    If \(k\) were an infinite sequence, i.e., \(k \colon \NN \hookrightarrow n\),
    then \(|\NN| \le |n|\). Together with \Cref{exer:4.1.3} and \nameref{th:cantorBernstein},
    we get \(|\NN| = |n|\), which contradicts \ref{itm:basicFinite.iii}.
    Hence, \(k\) is a finite sequence of length \(\ell\).

    Let \(y_i \triangleq x_{k_i}\) for each \(i < \ell\).
    By (i) and (ii), the sequence \(y\) is injective and its range is a subset of \(Y\).
    By the same argument of \Cref{clm:009f.fIsSurjective} of \Cref{th:thenIsoToN},
    we have \(\ran y = Y\). Therefore, \(y \colon \ell \hooktwoheadrightarrow Y\); \(Y\) is finite.
}

\thm[imageOfFiniteIsFinite]{}{
    If \(X\) is finite and \(f\) is a function, then \(f[X]\) is finite.
    Moreover, \(|f[X]| \le |X|\).
}
\pf{Proof}{
    We may assume \(X = \{\,x_0, \cdots, x_{n-1}\,\}\), where \(\lang\,x_0, \cdots, x_{n-1}\,\rang\)
    is an injective sequence.
    Let \(g\) be a function on a subset of \(\Seq(n)\) into \(n\)
    defined by
    \begin{equation}
        g(\lang\,k_0, \cdots, k_{\ell'-1}\,\rang) = \begin{cases}
            0 & \text{if } \ell' = 0 \\
            \min \{\,k \in n \mid k_{\ell'-1} < k \land \fall i < \ell',\: f(x_k) \neq f(x_{k_i})\,\} & \text{if it exists and }\ell' > 0 \\
            \text{undefined} & \text{otherwise}.
        \end{cases}
        \label{eq:badUlVto} \tag{\(\ast\)}
    \end{equation}

    Then, one may modify \nameref{th:generalRecursion} to its partial version
    like \nameref{exer:3.3.5} to get a sequence \(k\) of elements of \(n\) such that:
    \begin{enumerate}[nolistsep, label=(\roman*), ref=\protect{(\roman*)}, listparindent=\parindent]
        \ii \(\fall i \in \dom k,\: k_i = g\left(\restr{k}{i}\right)\).
            In particular, \(k_0 = 0\).
        \ii \(k\) is either an infinite sequence or a finite sequence of length \(\ell + 1\)
            and \(k \notin \dom g\).
    \end{enumerate}

    By (i) and \eqref{eq:badUlVto}, \(\fall i, j \in \dom k,\: (i \neq j \implies f(x_{k_i}) \neq f(x_{k_j}))\),
    i.e., the sequence \(y = \lang\,f(x_{k_i}) \mid i \in \dom k\,\rang\)
    is injective and its range is a subset of \(f[X]\).

    By the similar reason as in the proof of \Cref{th:subsetOfFiniteIsFinite},
    \(k\) is finite and \(\ran y = f[X]\).
    Finally, we get \(|f[X]| \le |X|\) from \(x \circ y\inv \colon f[X] \hookrightarrow X\).
}

\mlemma[finiteTwoUnion]{}{
    Let \(X\) and \(Y\) be finite sets.
    \begin{enumerate}[nolistsep, label=(\roman*), ref=\protect{(\roman*)}, listparindent=\parindent]
        \ii \(X \cup Y\) is finite; moreover, \(|X \cup Y| \le |X| + |Y|\).
        \ii If \(X\) and \(Y\) are disjoint, then \(|X \cup Y| = |X| + |Y|\).
    \end{enumerate}
}
\mclm{Proof}{\hfill
\begin{enumerate}[nolistsep, label=(\roman*), leftmargin=*, listparindent=\parindent]
    \ii
    Write \(X = \{\,x_0, \cdots, x_{m-1}\,\}\) and \(Y = \{\,y_0, \cdots, y_{n-1}\,\}\)
    where \(\lang\,x_0, \cdots, x_{m-1}\,\rang\) and \(\lang\,y_0, \cdots, y_{n-1}\,\rang\)
    are injective sequences.

    Now, define \(z \colon (n + m) \twoheadrightarrow X \cup Y\) by
    \[
        z_i = x_i \quad\text{for } 0 \le i < n \quad\text{and}\quad
        z_i = y_{i-n}\quad\text{for } n \le i < n + m.
    \]
    (Here, \(i - n\) is the unique \(k \in \NN\) such that \(i = n + k\). See \Cref{exer:3.4.3}.)
    Hence, by \Cref{th:imageOfFiniteIsFinite}, \(X \cup Y\) is finite and
    \(|X \cup Y| \le n + m\).

    \ii
    If \(X\) and \(Y\) are disjoint, then \(z \colon (n + m) \hooktwoheadrightarrow X \cup Y\).
    Hence, \(|X \cup Y| = n + m\).
    \qed
\end{enumerate}
}

\thm[finiteUnion]{}{
    If \(S\) is finite and if every \(X \in S\) is finite,
    then \(\bigcup S\) is finite.
}
\pf{Proof}{
    If \(|S| = 0\), then it is done.

    Assume that the statement is true for all \(S\) with \(|S| = n\).
    Let \(S = \{\,X_0, \cdots, X_n\,\}\) be a set with \(n+1\) elements
    such that each \(X_i \in S\) is finite.
    Then, we have
    \[
        \bigcup S = \left(\bigcup_{i=0}^{n-1} X_i\right) \cup X_n
    \]
    but \(\bigcup_{i=0}^{n-1} X_i\) is finite by induction hypothesis
    and thus \(\bigcup S\) is finite by \Cref{lem:finiteTwoUnion}.
    Hence, by \nameref{th:induction}, the result follows.
}

\thm[powerSetOfFiniteIsFinite]{}{
    If \(X\) is finite, then \(\mcal P(X)\) is finite.
}
\pf{Proof}{
    If \(|X| = 0\), then \(\mcal P(X) = \{\OO\}\), which is indeed finite.

    Fix any \(n \in \NN\) and assume that \(\mcal{P}(X)\) is finite for all \(X\) with \(|X| = n\).
    Take any \(Y\) with \(|Y| = n + 1\).
    Let \(Y = \{\,y_0, \cdots, y_n\,\}\) and \(X \triangleq \{\,y_0, \cdots, y_{n-1}\,\}\).
    Note that \(\mcal P(Y) = \mcal P(X) \cup U\)
    where \(U = \{\,u \subseteq Y \mid y_n \in u\,\}\).
    Moreover, \(f \colon \mcal P(X) \to U\) defined by
    \(f(x) = x \cup \{y_n\}\) is injective and onto \(U\).
    Hence, \(U\) is finite.
    By \Cref{lem:finiteTwoUnion}, \(\mcal P(Y)\) is finite.
    The result follows by \nameref{th:induction}.
}

\thm[infiniteThenBigCard]{}{
    If \(X\) is infinite, then \(|X| > n\) for all \(n \in \NN\).
}
\pf{Proof}{
    We clearly have \(0 \le |X|\).
    
    For induction, fix any \(n \in \NN\) and assume \(n \le |X|\),
    i.e., there exists \(f \colon n \hookrightarrow X\).
    By \Cref{th:imageOfFiniteIsFinite}, \(\ran f \subsetneq X\);
    we may take \(x \in X \setminus \ran f\).
    Then, \(g \triangleq f \cup \{(n, x)\}\)
    is an injective function on \(n+1\) into \(X\); hence \(n + 1 \le |X|\).
    Therefore, by \nameref{th:induction}, we have \(n \ge |X|\) for all \(n \in \NN\),
    which is suffices to induce the result.
}

\subsection*{Selected Problems}

\exer[4.2.1]{}{
    If \(S = \{\,X_0, \cdots, X_{n-1}\,\}\) is a finite set of mutually disjoint sets.
    Then, \(\left|\bigcup S\right| = \sum_{i=0}^{n-1} |X_i|\).
}
\pf{Proof}{
    If \(S = \OO\), then \(\left|\bigcup S\right| = 0 = \sum_{i=0}^{n-1} |X_i|\).

    Fix \(n \in \NN\) and assume the assertion holds for all \(S\) with \(|S| = n\).
    Then, take any set \(T\) of mutually disjoint sets with \(|T| = n+1\).
    Write \(T = \{\,X_0, \cdots, X_n\,\}\) and let \(S \triangleq = \{\,X_0, \cdots, X_{n-1}\,\}\).
    Then, since \(\bigcup T = \left(\bigcup S\right) \cup X_n\),
    and since \(\bigcup S\) and \(X_n\) are disjoint,
    \(\left|\bigcup T\right| = \left|\bigcup S\right| + |X_n|
    = \sum_{i=0}^{n-1} |X_i| + |X_n| = \sum_{i=0}^{n} |X_i|\).
    Hence, the result follows from \nameref{th:induction}.
}

\exer[4.2.1]{}{
    If \(X\) and \(Y\) are finite, then \(|X \times Y| = |X| \cdot |Y|\).
}
\pf{Proof}{
    We shall exploit the induction on \(|Y|\).
    If \(|Y| = 0\), then
    \begin{alignat*}{2}
        |X \times Y| &= 0 &\qquad& \comment*{\Cref{exer:2.2.8}}\\
                     &= |X| \cdot |Y|. && \comment*{\text{\eqref{eq:multiplication1}}}
    \end{alignat*}

    Assume the statement holds for all \(X\) and \(Y\) with \(|Y| = n\).
    Let \(Z = \{\,z_0, \cdots, z_n\,\}\) be a set with \(|Z| = n + 1\).
    Let \(Y \triangleq \{\,z_0, \cdots, z_{n-1}\,\}\).
    Then, for all \(X\), \(X \times Z = (X \times Y) \cup (X \times \{z_n\})\).
    Note that \(X \times \{z_n\}\) can be identified with \(X\) via
    \(f \colon X \hooktwoheadrightarrow X \times \{z_n\}\) defined by
    \(x \mapsto (x, z_n)\).
    Hence, if \(X\) is finite,
    \begin{alignat*}{2}
        |X \times Z| &= |X \times Y| + |X \times \{z_n\}| &\qquad& \comment*{\Cref{lem:finiteTwoUnion}}\\
                     &= |X \times Y| + |X| && \comment*{\(|X \times \{z_n\}| = |X|\)}\\
                     &= |X| \cdot |Y| + |X| && \comment*{\(\mbf{P}(n)\)}\\
                     &= |X| \cdot (|Y| + 1) && \comment*{\eqref{eq:multiplication2}}\\
                     &= |X| \cdot |Z|.
    \end{alignat*}
    Therefore, by \nameref{th:induction}, the result follows.
}

\exer[4.2.3]{}{
    If \(X\) is finite, \(|\mcal P(X)| = 2^{|X|}\).
}
\pf{Proof}{
    Let \(\mbf{P}(x)\) be the property ``\(\fall X,\: (|X| = x \implies |\mcal P(X)| = 2^{|X|})\).''
    \(\mbf{P}(0)\) holds since \(|\mcal P(\OO)| = |\{\OO\}| = 1 = 2^0\).

    Fix \(n \in \NN\) and assume \(\mbf{P}(n)\).
    Let \(Y = \{\,y_0, \cdots, y_n\,\}\) be a set with \(|Y| = n+1\).
    Let \(X \triangleq \{\,y_0, \cdots, y_{n-1}\,\}\).
    As in the proof of \Cref{th:powerSetOfFiniteIsFinite},
    \(\mcal P(Y) = \mcal P(X) \cup U\)
    where \(U = \{\,u \subseteq Y \mid y_n \in u\,\}\).
    Note that \(\mcal P(X) \cap U = \OO\) and
    \(f \colon \mcal P(X) \hooktwoheadrightarrow U\) defined by
    \(x \mapsto x \cup \{y_n\}\) asserts \(|\mcal P(X)| = |U|\).
    Therefore,
    \begin{alignat*}{2}
        |\mcal P(Y)| &= |\mcal P(X)| + |U| &\qquad& \comment*{\Cref{lem:finiteTwoUnion}}\\
                     &= 2^n + 2^n && \comment*{\(|\mcal P(X)| = |U|\), \(\mbf{P}(n)\)}\\
                     &= 2^n \cdot 1 + 2^n \cdot 1 && \comment*{\Cref{lem:timesOne}}\\
                     &= 2^n \cdot 2 && \comment*{\nameref{th:multDistrOverAdd}}\\
                     &= 2^{n+1}. && \comment*{\eqref{eq:exponentiation2}}
    \end{alignat*}
    Therefore, by \nameref{th:induction}, the result follows.
}

\exer[4.2.4]{}{
    If \(X\) and \(Y\) are finite, then \(X^Y\) is finite and \(|X^Y| = |X|^{|Y|}\).
}
\pf{Proof}{
    Let \(\mbf{P}(x)\) be the property
    ``if \(X\) is finite and \(|Y| = x\), then \(|X^Y| = |X|^x\).''
    \(\mbf{P}(0)\) holds since \(|X^{\OO}| = |\{\OO\}| = 1 = |X|^0\) for all \(X\).

    Fix \(n \in \NN\) and assume \(\mbf{P}(n)\).
    Let \(Y = \{\,y_0, \cdots, y_n\,\}\) be a set with \(|Y| = n+1\).
    Let \(Z \triangleq \{\,y_0, \cdots, y_{n-1}\,\}\).
    Take any finite set \(X\).

    We have \(|X^Y| = |X^Z \times X|\) since
    we may define \(f \colon X^Y \hooktwoheadrightarrow X^Z \times X\)
    by \(g \mapsto \left(\restr{g}{Z}, g(y_n)\right)\).
    Hence,
    \begin{alignat*}{2}
        |X^Y| &= |X^Z \times X| &\qquad& \\
              &= |X^Z| \cdot |X| && \comment*{\Cref{exer:4.2.1}}\\
              &= |X|^{n} \cdot |X| && \comment*{\(\mbf{P}(n)\)}\\
              &= |X|^{n+1}. && \comment*{\eqref{eq:exponentiation2}}
    \end{alignat*}
    The result follows by \nameref{th:induction}.
}

\setexernumber{5}

\exer[4.2.6]{}{
    \(X\) is finite if and only if every \(\OO \subsetneq U \subseteq \mcal P(X)\)
    has a \(\subseteq\)-maximal element.
}
\mclm{Proof}{\hfill
\begin{itemize}[nolistsep, wide=0pt, widest={(\(\Rightarrow\))}, leftmargin=*, listparindent=\parindent]
    \ii[(\(\Rightarrow\))]
    Let \(|X| = n\) and \(\OO \subsetneq U \subseteq \mcal P(X)\).
    Since \(|Y| \le n\) for all \(Y \in U\),
    by \Cref{th:hasUpperBoundThenMaxExists},
    we may let \(m \triangleq \max\{\,|Y| \mid Y \in U\,\}\).
    
    There exists \(Y \in U\) with \(|Y| = m\).
    Then, for each \(Y' \in U\) such that \(Y \subseteq Y'\),
    we have \(m \le |Y'|\) by \Cref{exer:4.1.3}
    and \(|Y'| \le m\) by definition of \(m\);
    thus \(|Y'| = |Y| = m\) by \nameref{th:cantorBernstein},
    which implies we may not have \(Y \subsetneq Y'\) by
    \Cref{lem:finiteProperSubset}.
    Hence, \(Y\) is a maximal element of \(U\).

    \ii[(\(\Leftarrow\))]
    Assume \(X\) is infinite.
    Let \(U = \{\,Y \subseteq X \mid Y\text{ is finite}\,\}\).
    (Note \(\OO \in U\), hence \(U \neq \OO\).)
    Take any \(Y \in U\).
    Since \(Y \subsetneq X\), we may take \(x \in X \setminus Y\).
    Then, \(Y \subsetneq Y \cup \{x\}\) and \(Y \cup \{x\} \in U\)
    by \Cref{lem:finiteTwoUnion}.
    Hence, there is no maximal element of \(U\).
    \qed
\end{itemize}
}

\end{document}
