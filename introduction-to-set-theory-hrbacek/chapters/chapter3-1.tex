\documentclass[../introduction_to_set_theory.tex]{subfiles}
\begin{document}

\section{Introduction to Natural Numbers}

\nt{
    We cannot prove an existence of an `infinite' set (in the classical sense) or discuss infinity
    only from \Crefrange{ax:existence}{ax:powerSet}.
}

\dfn[successor]{Successor}{
    The \textit{successor} of a set \(x\) is the set \(S(x) = x \cup \{x\}\).
}

\notat[nPlusOne]{\(\boldsymbol{n+1}\)}{
    We write \(n + 1\) to denote \(S(n)\).
    There is no implication regarding the classic ``addition'' in this notation.
}

\notat{Natural Numbers}{
    \begin{itemize}[nolistsep, leftmargin=*]
        \ii \(0 = \OO\)
        \ii \(1 = \{\OO\} = S(0) = 0 + 1\)
        \ii \(2 = \{\OO, \{\OO\}\} = S(1) = 1 + 1\)
        \ii \(\cdots\)
    \end{itemize}
}

\dfn[inductiveSet]{Inductive Set}{
    A set \(I\) is called \textit{inductive} if
    \[
        0 \in I \land \fall n \in I,\: (n + 1) \in I.
    \]
}

\axiom[infinity]{Axiom of Infinity}{
    \noindent
    An inductive set exists.
}

\dfn[]{Set of All Natural Numbers}{
    The \textit{set of all natural numbers} is the set
    \[
        \NN \triangleq \{\,x \mid x \in I \text{ for all inductive set }I\,\}.
    \]
}

\nt{
    \nameref{ax:infinity} guarantees the existence of \(\NN\).
    For, if \(A\) is any inductive set, then
    \(\NN = \{\,x \in A \mid x \in I \text{ for all inductive set }I\,\}\).
}

\mlemma[NisInductive]{}{
    \(\NN\) is inductive. In addition, if \(I\) is an inductive set, then \(\NN \subseteq I\).
}
\pf{Proof}{
    Since \(0 \in I\) for all inductive set, \(0 \in \NN\).
    If \(n \in \NN\), then \(n \in I\) for all inductive set,
    and thus \((n + 1) \in I\) for all inductive set.
    Therefore, \((n + 1) \in \NN\). Hence, \(\NN\) is inductive.

    \(\NN \subseteq I\) directly follows from the definition of \(\NN\).
}

\dfn[]{}{
    The relation \(<\) on \(\NN\) is defined by: \(m < n\) if and only if \(m \in n\).
}

\notat{}{
    Although we did not prove \(<\) is a strict ordering of \(\NN\),
    we shall use \(\le\) to denote the relation on \(\NN\):
    \[
        \mathord{\le} \triangleq \mathord{<} \cup \mrm{Id}_{\NN}
    \]
}

\subfile{../exercises/exercise3-1.tex}

\section{Properties of Natural Numbers}

\thm[induction]{The Induction Principle}{
    Let \(\mbf{P}(x)\) be a property (possibly with parameters).
    \[
        \mbf{P}(0) \land \fall n \in \NN,\: (\mbf{P}(n) \implies \mbf{P}(n+1))
        \implies \fall n \in \NN,\: \mbf{P}(n)
    \]
}
\pf{Proof}{
    The premise simply says that \(A = \{\,n \in \NN \mid \mbf{P}(n)\,\}\) is inductive.
    Therefore, \(\NN \subseteq A\) follows.
}

\mlemma[basicLess]{}{
    \begin{enumerate}[nolistsep, label=(\roman*), ref=\protect{\Cref{lem:basicLess} (\roman*)}]
        \ii \label{itm:basicLess.i}\(\fall n \in \NN,\: 0 \le n\)
        \ii \label{itm:basicLess.ii}\(\fall k, n \in \NN,\: (k < n + 1 \iff k < n \lor k = n)\)
    \end{enumerate}
}
\mclm{Proof}{\hfill
\begin{enumerate}[nolistsep, label=(\roman*), listparindent=\parindent]
    \ii
    Let \(\mbf{P}(x)\) be the property ``\(0 \le x\).''
    \(\mbf{P}(0)\), i.e., \(0 \le 0\), holds since \(0 = 0\).

    Now, assume \(n \in \NN\) and \(\mbf{P}(n)\).
    If \(n = 0\), then we have \(0 \in S(0) = n + 1\) by definition (\Cref{dfn:successor}).
    If \(0 < n\), then \(0 \in n\), and thus \(0 \in n \cup \{n\} = S(n)\).
    Therefore, by \nameref{th:induction}, the result follows.

    \ii
    Note that \(k \in n \cup \{n\}\) if and only if \(k \in n\) or \(k = n\).
    \qed
\end{enumerate}
}

\thm[NisTotallyOrdered]{\((\NN, \le)\) is Totally Ordered}{
    \((\NN, \le)\) is a totally ordered set.
}
\pf{Proof}{
    We first need to prove that \((\NN, \le)\) is an ordered set.
    \clm[lessIsTransitive]{
        \(<\) is transitive in \(\NN\).
    }{
        Let \(\mbf{P}(x)\) be the property ``\(\fall k, m \in \NN,\: (k < m \land m < x \implies k < x)\).''
        \(\mbf{P}(0)\) is true because there is no \(m \in \NN\) such that \(m \in 0 = \OO\).

        Now assume \(n \in \NN\) and \(\mbf{P}(n)\).
        Now, let \(k, m \in \NN\) and \(k < m\) and \(m < n + 1\).
        By \ref{itm:basicLess.ii}, \(m < n\) or \(m = n\).
        \begin{itemize}[nolistsep]
            \ii
            If \(m < n\), then we have \(k < n\) as \(\mbf{P}(n)\) holds,
            \ii
            If \(m = n\), then we immediately have \(k < n\).
        \end{itemize}
        In both cases, we have \(k < n\); thus \(k < n + 1\) by \ref{itm:basicLess.ii}.
        Therefore, the result follows from \nameref{th:induction}.
        \qed
    }

    \clm[lessIsAsymmetric]{
        \(<\) is asymmetric in \(\NN\).
    }{
        Let \(\mbf{P}(x)\) be the property ``\(\lnot(x < x)\).''
        \(\mbf{P}(0)\) evidently holds since \(\OO \notin \OO\).

        Now, assume \(n \in \NN\) and \(\mbf{P}(n)\).
        Suppose \((n + 1) < (n + 1)\) for the sake of contradiction.
        By \ref{itm:basicLess.ii}, we have \((n + 1) = n\) or \((n + 1) < n\).
        In both cases, we have \(n < n\) by \(n < (n + 1)\) (from \ref{itm:basicLess.ii}) and
        \Cref{clm:lessIsTransitive}, which contradicts \(\mbf{P}(n)\).
        Therefore, \(\mbf{P}(n+1)\) holds.
        The result follows from \nameref{th:induction}. \qed
    }
    Hence, \((\NN, \le)\) is an ordered set by \Cref{clm:lessIsTransitive,clm:lessIsAsymmetric,th:partialAndStrictAreSame}.
    We are left to prove that \(\le\) is a total ordering of \(\NN\).

    \clm[nPlusOne]{
        \(\fall n, m \in \NN,\: n < m \implies (n + 1) \le m\)
    }{
        Let \(\mbf{P}(x)\) be the property ``\(\fall n \in \NN, (n < x \implies n + 1 \le x)\).''
        \(\mbf{P}(0)\) holds since there is no \(n \in \NN\) such that \(n < 0\).

        Now, assume \(m \in \NN\) and \(\mbf{P}(m)\).
        Take any \(n \in \NN\) such that \(n < (m + 1)\).
        Then, by \Cref{lem:basicLess}, we have \(n = m\) or \(n < m\).
        If \(n =  m\), then we have \((n + 1) = (m + 1)\), which implies \((n + 1) \le (m + 1)\).
        If \(n < m\), then \((n + 1) \le m < (m + 1)\).
        Therefore, the result follows from \nameref{th:induction}. \qed
    }

    \clm[lessIsLinear]{
        \(<\) is a total ordering of \(\NN\).
    }{
        Let \(\mbf{P}(x)\) be the property ``\(\fall m \in \NN,\: m = x \lor m < x \lor x < m\).''
        \(\mbf{P}(0)\) is essentially \ref{itm:basicLess.i}.

        Assume \(n \in \NN\) and \(\mbf{P}(n)\).
        Take any \(m \in \NN\).
        If \(m < n\) or \(m = n\), we have \(m < (n+1)\) by \ref{itm:basicLess.ii}.
        If \(n < m\), by \Cref{clm:nPlusOne}, we have \((n + 1) \le m\).
        Hence, \(\mbf{P}(n+1)\) holds.
        Therefore, the result follows from \nameref{th:induction}. \qed
    }
}

\notat{}{
    We may write ``\(\fall k < n, \mbf{P}(k)\)'' instead of ``\(\fall k \in \NN,\: (k < n \implies \mbf{P}(k))\)''
    or ``\(\exs k < n, \mbf{P}(k)\)'' instead of ``\(\exs k \in \NN,\: k < n \land \mbf{P}(k)\)''
    when no confusion may arise.
    We may similarly write \((\fall/\exs) k (\mathord{\le}/\mathord{>}/\mathord{\ge}) n\), \(\mbf{P}(k)\).
}

\thm[strongInduction]{The Strong Induction Principle}{
    Let \(\mbf{P}(x)\) be a property (possibly with parameters).
    If, for all \(n \in \NN\), \(\mbf{P}(k)\) holds for all \(k < n\),
    then \(\mbf{P}(n)\) holds for all \(n \in \NN\).
    \[
        \fall n \in \NN,\: [\fall k < n,\: \mbf{P}(k) \implies \mbf{P}(n)] \implies \fall n \in \NN,\: \mbf{P}(n)
    \]
}
\pf{Proof}{
    Assume the premise (\(\fall n \in \NN,\: [\fall k < n,\: \mbf{P}(k) \implies \mbf{P}(n)]\)).
    Let \(\mbf{Q}(n)\) be the property ``\(\fall k < n, \mbf{P}(k)\).''
    \(\mbf{Q}(0)\) holds since there is no \(k < 0\).

    Now, assume \(n \in \NN\) and \(\mbf{Q}(n)\).
    Then, by the premise, we have \(\mbf{P}(n)\).
    \ref{itm:basicLess.ii} enables us to say that \(\fall k \in \NN,\: (k < n + 1 \implies P(k))\).
    Therefore, \(\fall n \in \NN,\: \mbf{Q}(n)\) holds by \nameref{th:induction}.

    Take any \(k \in \NN\). Then, we have \(k < k + 1\) and thus \(\mbf{P}(k)\) holds
    by \(\mbf{Q}(k+1)\).
}

\dfn[wellOrdering]{Well-Ordering}{
    A total ordering \(\preceq\) of a set \(A\) is a \textit{well-ordering}
    if every nonempty subset of \(A\) has a least element.
    Then, the ordered set \((A, \preceq)\) is called a \textit{well-ordered set}.
}

\thm[NisWellOrdered]{\((\NN, \le)\) is Well-Ordered}{
    \((\NN, \le)\) is a well-ordered set.
}
\pf{Proof}{
    Let \(X \subseteq \NN\) has no least element.
    For each \(n \in \NN\), if \(\fall k < n,\: k \in \NN \setminus X\), we must have \(n \in \NN \setminus X\)
    since otherwise \(n = \min X\).
    Then, by \nameref{th:strongInduction}, \(\fall n \in \NN,\: n \in \NN \setminus X\),
    i.e., \(X = \OO\).
}

\thm[hasUpperBoundThenMaxExists]{}{
    Let \(\OO \subsetneq X \subseteq \NN\).
    If \(X\) has an upper bound in the ordering \(\le\), then \(X\) has a greatest element.
}
\pf{Proof}{
    Let \(Y \triangleq \{\,k \in \NN \mid k \text{ is an upper bound of }X\,\}\).
    The assumption says that \(Y \neq \OO\).
    By \nameref{th:NisWellOrdered}, \(n \triangleq \min Y = \sup X\) exists.

    Suppose \(n \notin X\) for the sake of contradiction.
    Then, \(\fall m \in X,\: m < n\), which implies \(n \neq 0\) as \(X \neq \OO\).
    Therefore, \(n = k + 1\) for some \(k \in \NN\) by \Cref{exer:3.2.4};
    and thus \(\fall m \in X,\: m \le k\) by \ref{itm:basicLess.ii}.
    Then, \(k\) is an upper bound of \(A\) and \(k < n\), which is a contradiction to \(n = \sup X\).
    Therefore, \(n \in X\), and hence \(n = \max X\) by \Cref{th:basicInfimum}.
}

\subfile{../exercises/exercise3-2.tex}

\end{document}
