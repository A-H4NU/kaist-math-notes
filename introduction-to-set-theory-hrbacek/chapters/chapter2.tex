\documentclass[../introduction_to_set_theory.tex]{subfiles}
\begin{document}

\section{Ordered Pairs}

\dfn[orderedPair]{Ordered Pair}{
    \((a, b) \triangleq \{\{a\}, \{a, b\}\}\)
}

\thm[orderedPair]{}{
    \((a, b) = (a', b') \iff a = a' \land b = b'\)
}
\pf{Proof}{
    (\(\Leftarrow\)) is direct.

    (\(\Rightarrow\))
    If \(a = b\), we have \(\{\{a\}\} = \{\{a'\}, \{a', b'\}\}\),
    and thus \(\{a\} = \{a'\} = \{a', b'\}\),
    leaving the only option \(a = a' = b'\).

    If \(a \neq b\), we must have \(a' \neq b'\) by the argument above.
    Hence, we have \(\{\{a\}, \{a, b\}\} = \{\{a'\}, \{a', b'\}\}\),
    which implies \(\{a\} = \{a'\}\) and \(\{a, b\} = \{a', b'\}\).
}

\dfn[]{Ordered Triples and Quadruples}{
    \begin{itemize}[nolistsep]
        \ii \((a, b, c) = ((a, b), c)\)
        \ii \((a, b, c, d) = ((a, b, c), d)\)
    \end{itemize}
}

\subsection*{Selected Problems}

\exer[2.1.1]{}{
    If \(a, b \in A\), then \((a, b) \in \mcal P(\mcal P(A))\).
}
\pf{Proof}{
    If \(a, b \in A\), then \(\{a\}, \{a, b\} \in \mcal P(A)\),
    and thus \((a, b) = \{\{a\}, \{a, b\}\} \subseteq \mcal P(A)\).
}

\section{Relations}

\dfn[relation]{Binary Relation}{
    A set \(R\) is a \textit{binary relation}
    if all elements of \(R\) are ordered pairs.

    \[
        R \text{ is a binary relation} \iff
        (a \in R \implies \exs x,\: \exs y,\: a = (x, y))
    \]
}

\notat{}{
    If \((x, y) \in R\), we write \(xRy\) and say \textit{\(x\) is in relation \(R\) with \(y\)}.
}

\dfn{Domain, Range, and Field of Binary Relation}{
    Let \(R\) be a binary relation.
    \begin{itemize}[nolistsep]
        \ii \(\dom R \triangleq \{\,x \mid \exs y\: xRy\,\}\) is called the \textit{domain} of \(R\).
        \ii \(\ran R \triangleq \{\,y \mid \exs x\: xRy\,\}\) is called the \textit{range} of \(R\).
        \ii \(\field R \triangleq \dom R \cup \ran R\) is called the \textit{field} of \(R\).
        \ii If \(\field R \subseteq X\), we say that \(R\) is a \textit{relation in \(X\)}
            or that \(R\) is a relation \textit{between} elements of \(X\).
    \end{itemize}
}

\mlemma[]{}{
    Let \(R\) be a binary relation.
    Then, \(\dom R\) and \(\ran R\) exist.
}
\pf{Proof}{
    By \Cref{exer:2.2.1}, if \(xRy\), then \(x, y \in A \triangleq \bigcup \left(\bigcup R\right)\).
    Hence, \(\dom R\) and \(\ran R\) exist.
}

\dfn[]{Image and Inverse Image}{
    Let \(R\) be a binary relation and \(A\) be a set.
    \begin{itemize}[nolistsep]
        \ii \(R[A] \triangleq \{\,y \in \ran R \mid \exs x \in A,\: xRy\,\}\)
            is called the \textit{image} of \(A\) under \(R\).
        \ii \(R\inv[A] \triangleq \{\,x \in \dom R \mid \exs y \in A,\: xRy\,\}\)
            is called the \textit{inverse image} of \(A\) under \(R\).
    \end{itemize}
}

\notat{}{
    We write \(\{\,(x, y) \mid \mbf{P}(x, y)\,\}\) instead of
    \(\{\,w \mid \exs x,\: \exs y,\: w = (x, y) \land \mbf{P}(x, y)\,\}\).
}

\dfn[]{Inverse Relation}{
    Let \(R\) be a binary relation.
    The \textit{inverse} of \(R\) is the set
    \[
        R\inv \triangleq \{\,(x, y) \mid yRx\,\}.
    \]
}

\dfn[]{Composition}{
    Let \(R\) and \(S\) be binary relations.
    The relation
    \[
        S \circ R \triangleq \{\,(x, z) \mid \exs y,\: xRy \land ySz\}
    \]
    is called the \textit{composition} of \(R\) and \(S\).
}

\dfn[]{Membership Relation and Identity Relation}{
    Let \(A\) be a set.
    \begin{itemize}[nolistsep]
        \ii The \textit{membership relation on \(A\)} is defined by
            \[
                \in_A\: \triangleq \{\,(a, b) \mid a, b \in A \land a \in b\,\}.
            \]
        \ii The \textit{identity relation on} \(A\) is defined by
            \[
                \mrm{Id}_A \triangleq \{\,(a, a) \mid a \in A\,\}.
            \]
    \end{itemize}
}

\dfn[cartesianProduct]{Cartesian Product}{
    Let \(A\) and \(B\) be sets.
    The set
    \(
        A \times B \triangleq \{\,(a, b) \mid a \in A \land b \in B\,\}
    \)
    is called the \textit{Cartesian product} product of \(A\) and \(B\).
}

\mlemma[cartesianProductExists]{}{
    Let \(A\) and \(B\) be sets.
    \(A \times B\) exists.
}
\pf{Proof}{
    If \(a \in A\) and \(b \in B\), by \Cref{exer:2.1.1}, we have \((a, b) \in \mcal{P}(\mcal{P}(A \cup B))\).
}

\cor[]{}{
    Let \(R\) and \(S\) be binary relations and \(A\) be a set.
    Then, \(R\inv\), \(S \circ R\), \(\in_A\), and \(\mrm{Id}_A\) exist.
}
\mclm{Proof}{\hfill
    \begin{itemize}[nolistsep]
        \ii If \(yRx\), then \((x, y) \in (\ran R) \times (\dom R)\).
        \ii If \((x, z) \in S \circ R\), then \((x, z) \in (\dom R) \times (\ran S)\).
        \ii If \(a, b \in A\), then \((a, b) \in A \times A\).
        \ii If \(a \in A\), then \((a, a) \in A \times A\).\qed
    \end{itemize}
}

\mlemma[invImgRisImgRInv]{}{
    Let \(R\) be a binary relation.
    The inverse image of \(A\) under \(R\) is equal to the image of \(A\) under \(R\inv\).
}
\pf{Proof}{
    Note that \(\dom R = \{\,x \mid \exs y\: xRy\,\} = \{\,x \mid \exs y \: yR\inv x\,\} = \ran R\inv\).
    Therefore,
    \begin{align*}
        & x \in (\text{the inverse image of }A\text{ under }R) \\
        \iff& x \in \dom R \land \exs y \in A,\: xRy \\
        \iff& x \in \ran R\inv \land \exs y \in A,\: yR\inv x \\
        \iff& x \in (\text{the image of }A\text{ under }R\inv).
    \end{align*}
}

\nt{
    \noindent
    \Cref{lem:invImgRisImgRInv} resolves the possible ambiguity on the expression \(R\inv[A]\).
}

\notat{}{
    We write \(A^2\) instead of \(A \times A\).
}

\subsection*{Selected Problems}

\exer[2.2.1]{}{
    Let \(R\) be a binary relation. Let \(A = \bigcup \big(\bigcup R\big)\).
    Prove that \((x, y) \in R\) implies \(x \in A\) and \(y \in A\).
}
\pf{Proof}{
    If \((x, y) = \{\{x\}, \{x, y\}\} \in R\),
    Then \(\{x, y\} \in \bigcup R\), and thus \(x, y \in A\).
}

\setexernumber{2}

\exer[2.2.3]{}{
    Let \(R\) be a binary relation and \(A\) and \(B\) be sets. Prove:
    \begin{enumerate}[nolistsep, label=(\roman*)]
        \ii \(R[A \cup B] = R[A] \cup R[B]\).
        \ii \(R[A \cap B] \subseteq R[A] \cap R[B]\).
        \ii \(R[A \setminus B] \supseteq R[A] \setminus R[B]\).
        \ii Show by an example that \(\subseteq\) and \(\supseteq\) in parts (ii) and (iii) cannot be
            replaced by \(=\).
        \ii \(R\inv[R[A]] \supseteq A \cap \dom R\) and \(R[R\inv[B]] \supseteq B \cap \ran R\).
            Give examples where equality does not hold.
    \end{enumerate}
}
\mclm{Proof}{
    \hfill
    \begin{enumerate}[nolistsep, label=(\roman*)]
        \ii
        \(\begin{aligned}[t]
            y \in R[A \cup B] &\iff \exs x,\: x \in A \cup B \land xRy \\
                              &\iff \exs x,\: (x \in A \land xRy) \lor (x \in B \land xRy) \\
                              &\iff y \in R[A] \lor y \in R[B] \iff y \in R[A] \cup R[B]
        \end{aligned}\)

        \ii
        Take any \(y \in R[A \cap B]\).
        Then, there exists \(x \in A \cap B\) such that \(xRy\).
        Hence, \(y \in R[A]\) and \(y \in R[B]\).

        \ii
        Take any \(y \in R[A] \setminus R[B]\).
        Then, there exists \(x \in A\) such that \(xRy\).
        If \(x \in B\), it implies that \(y \in R[B]\), which is a contradiction.
        Hence, \(x \in A \setminus B\).
        Therefore, \(y \in R[A \setminus B]\).

        \ii
        Let \(a\), \(b\), and \(c\) be mutually different sets.
        Let \(R = \{(a, a), (b, a), (c, c)\}\).
        Let \(A = \{a, c\}\) and \(B = \{b, c\}\).
        Then, \(R[A \cap B] = \{c\} \subsetneq R[A] \cap R[B] = \{a, c\}\),
        and \(R[A] \setminus R[B] = \OO \subsetneq R[A \setminus B] = \{a\}\).

        \ii
        Take any \(a \in A \cap \dom R\).
        Then, there exists \(b\) such that \(aRb\).
        Moreover, \(b \in R[A]\).
        Since \(b R\inv a\), we conclude that \(a \in R\inv[R[A]]\).

        Take any \(b \in B \cap \ran R\).
        Then, there exists \(a\) such that \(aRb\).
        Moreover, \(a \in R\inv[B]\).
        Hence, \(b \in R[R\inv[B]]\).
    \end{enumerate}
}

\exer[2.2.4]{}{
    Let \(R \subseteq X \times Y\). Prove:
    \begin{enumerate}[nolistsep, label=(\roman*)]
        \ii \(R[X] = \ran R\) and \(R\inv[Y] = \dom R\).
        \ii \(\dom R = \ran R\inv\) and \(\ran R = \dom R\inv\).
        \ii \((R\inv)\inv = R\).
        \ii \(R\inv \circ R \supseteq \mrm{Id}_{\dom R}\) and \(R \circ R\inv \supseteq \mrm{Id}_{\ran R}\)
    \end{enumerate}
}
\mclm{Proof}{\hfill
\begin{enumerate}[nolistsep, label=(\roman*)]
    \ii
    We already have \(R[X] \subseteq \ran R\) by definition.
    Take any \(y \in \ran R\).
    There exists \(x\) such that \((x, y) \in R\).
    Since \(R \subseteq X \times Y\), \(x \in X\).
    Therefore, \(y \in R[X]\); \(\ran R \subseteq R[X]\).
    A similar argument goes for \(R\inv[Y]\).

    \ii
    See the proof of \Cref{lem:invImgRisImgRInv}.

    \ii
    For any relation \(R\) and for all \(x\) and \(y\), we have \(xRy \iff yR\inv x\).
    Since \(R\inv\) is also a relation, we have
    \(xRy \iff yR\inv x \iff x (R\inv)\inv y\).

    \ii
    Take any \(x \in \dom R\).
    Then, there exists \(y\) such that \(xRy\).
    Then, \(y R\inv x\), and thus \(x(R \inv \circ R)x\).
    A similar argument goes for \(R \circ R\inv\). \qed

\end{enumerate}
}

\setexernumber{7}
\exer[2.2.8]{}{
    \(A \times B = \OO\) if and only if \(A = \OO\) or \(B = \OO\).
}
\pf{Proof}{
    (\(\Rightarrow\))
    If \(A \neq \OO\) and \(B \neq \OO\),
    we have \((a, b) \in A \times B\) where \(a \in A\) and \(b \in B\),
    and thus \(A \times B \neq \OO\).

    (\(\Leftarrow\))
    If \(A \times B \neq \OO\),
    then \(a \in A\) and \(b \in B\) where \((a, b) \in A \times B\).
}

\section{Functions}

\dfn[function]{Function}{
    A binary relation \(F\) is called a \textit{function} (or \textit{mapping})
    if \[\fall a\: \fall b_1\: \fall b_2\: (aFb_1 \land aFb_2 \implies b_1 = b_2).\]

    For each \(a \in \dom F\), the unique \(b\) such that \(aFb\) is called the \textit{value of \(F\) at \(a\)}
    and is denoted \(F(a)\) of \(F_a\).
}

\notat{}{
    If \(F\) is a function with \(\dom F = A\) and \(\ran F \subseteq B\),
    we write \(F \colon A \to B\), \(\lang F(a) \mid a \in A\rang\),
    \(\lang F_a \mid a \in A\rang\), \(\lang F_a\rang_{a \in A}\) for the function \(F\).
    The range of the function \(F\) can then be denoted \(\{\,F(a) \mid a \in A\,\}\)
    or \(\{F_a\}_{a \in A}\).
}

\mlemma[functionEqualsIff]{}{
    Let \(F\) and \(G\) be functions.
    \(F = G \iff \dom F = \dom G \land \fall x \in \dom F,\: F(x) = G(x)\).
}
\pf{Proof}{
    (\(\Rightarrow\)) is direct.

    (\(\Leftarrow\))
    Take any \((x, F(x)) \in F\).
    Then, we have \((x, F(x)) = (x, G(x)) \in G\).
    Therefore, \(F \subseteq G\).
    Similarly, \(G \subseteq F\), and thus \(F = G\).
}

\dfn[]{}{
    Let \(F\) be a function and \(A\) and \(B\) be sets.
    \begin{itemize}[nolistsep]
        \ii \(F\) is a function \textit{on} \(A\) if \(\dom F = A\).
        \ii \(F\) is a function \textit{into} \(B\) if \(\ran F \subseteq B\).
        \ii \(F\) is a function \textit{onto} \(B\) if \(\ran F = B\).
        \ii The \textit{restriction} of the function \(F\) \textit{to} \(A\)
            is the function \(\restr{F}{A} \triangleq \{\,(a, b) \in F \mid a \in A\,\}\).
            If \(G\) is a restriction of \(F\) to some \(A\),
            we say that \(F\) is an \textit{extension} of \(G\).
    \end{itemize}
}

\thm[functionComposite]{}{
    Let \(f\) and \(g\) be functions.
    \begin{enumerate}[nolistsep, label=(\roman*)]
        \ii \(g \circ f\) is a function.
        \ii \(\dom (g \circ f) = (\dom f) \cap f\inv[\dom g]\).
        \ii \(\fall x \in \dom (g \circ f),\: (g \circ f)(x) = g(f(x))\).
    \end{enumerate}
}
\mclm{Proof}{\hfill
\begin{enumerate}[nolistsep, label=(\roman*)]
    \ii
    Suppose \(x(g \circ f)z_1\) and \(x(g \circ f)z_2\).
    There exists \(y_1\) and \(y_2\) such that \(xfy_1\), \(y_1gz_1\), \(xfy_2\), and \(y_2gz_2\).
    Since \(f\) and \(g\) are functions, we have \(y_1 = y_2\) and \(z_1 = z_2\).
    Therefore, \(g \circ f\) is a function.

    \ii
    \(\begin{aligned}[t]
        x \in \dom (g \circ f) &\iff \exs z\: x(g \circ f)z \\
                               &\iff \exs z\:\exs y\: x f y \land y g z \\
                               &\iff x \in \dom f \land f(x) \in \dom g
                               \iff x \in \dom f \land x \in f\inv[\dom g] \qed
    \end{aligned}\)
\end{enumerate}
}

\dfn[]{Invertible Function}{
    A function \(f\) is said to be \textit{invertible} if \(f\inv\) is a function.
}

\dfn[]{Injective Function}{
    A function \(f\) is said to be \textit{injective} (or \textit{one-to-one}) if
    \[
        \fall a_1, a_2 \in \dom f,\: (f(a_1) = f(a_2) \implies a_1 = a_2).
    \]
}

\notat{}{
    Let \(f \colon A \to B\) be a function.
    \begin{itemize}[nolistsep]
        \ii If \(f\) is a function \textit{onto} \(B\), we may write \(f \colon A \twoheadrightarrow B\).
        \ii If \(f\) is one-to-one, we may write \(f \colon A \hookrightarrow B\).
        \ii If \(f\) is one-to-one and onto \(B\), we may write \(f \colon A \hooktwoheadrightarrow B\).
    \end{itemize}
}

\thm[invIffInj]{}{
    Let \(f\) be a function.
    \begin{enumerate}[nolistsep, label=(\roman*)]
        \ii \(f\) is invertible if and only if \(f\) is one-to-one.
        \ii If \(f\) is invertible, then \(f\inv\) is also invertible and \((f\inv)\inv = f\).
    \end{enumerate}
}
\mclm{Proof}{\hfill
\begin{enumerate}[nolistsep, label=(\roman*)]
    \ii
    (\(\Rightarrow\)) Suppose \(f\inv\) is a function. Then, \(f\inv(f(a)) = a\)
    for all \(a \in \dom f\).
    Hence, for all \(a_1, a_2 \in \dom f\) such that \(f(a_1) = f(a_2)\),
    it follows that \(a_1 = f\inv(f(a_1)) = f\inv(f(a_2)) = a_2\);
    \(f\) is one-to-one.

    (\(\Leftarrow\)) Suppose \(f\) is one-to-one.
    If \(yf\inv x_1\) and \(yf\inv x_2\), then \(x_1fy\) and \(x_2fy\),
    i.e., \(y = f(x_1) = f(x_2)\). Therefore, \(x_1 = x_2\); \(f\inv\) is a function.

    \ii
    As \(f\) is a relation, by \Cref{exer:2.2.4} (iii), \((f\inv)\inv = f\),
    and thus \(f\inv\) is invertible. \qed
\end{enumerate}
}

\dfn[compatible]{Compatible Functions}{
    \begin{itemize}[nolistsep]
        \ii Functions \(f\) and \(g\) are called \textit{compatible} if
            \(\fall x \in (\dom f) \cap (\dom g),\: f(x) = g(x)\).
        \ii A set of functions \(F\) is called a \textit{compatible system of functions}
            if any two functions \(f\) and \(g\) from \(F\) are compatible.
    \end{itemize}
}

\mlemma[compatibleIff]{}{
    Let \(f\) and \(g\) be functions.
    \begin{enumerate}[nolistsep, label=(\roman*)]
        \ii \(f\) and \(g\) are compatible if and only if \(f \cup g\) is a function.
        \ii \(f\) and \(g\) are compatible if and only if
            \(\restr{f}{(\dom f) \cap (\dom g)} = \restr{g}{(\dom f) \cap (\dom g)}\).
    \end{enumerate}
}
\mclm{Proof}{\hfill
\begin{enumerate}[nolistsep, label=(\roman*)]
    \ii
    (\(\Rightarrow\))
    Suppose \(x(f \cup g)y_1\) and \(x(f \cup g)y_2\).
    \WLOG, \((x, y_1) \in f\).
    If \((x, y_2) \in f\), since \(f\) is a function, \(y_1 = y_2\).
    If \((x, y_2) \in g\), since \(f\) and \(g\) are compatible, \(y_1 = f(x) = g(x) = y_2\).
    Therefore, \(f \cup g\) is a function.

    (\(\Leftarrow\))
    Take any \(x \in (\dom f) \cap (\dom g)\).
    \((x, f(x)) \in f \cup g\) and \((x, g(x)) \in f \cup g\).
    Since \(f \cup g\) is a function, we have \(f(x) = g(x)\).

    \ii
    Let \(A = (\dom f) \cap (\dom g)\).

    (\(\Rightarrow\))
    By definition, \(\dom \restr{f}{A} = \dom \restr{g}{A}
    = (\dom f) \cap (\dom g)\).
    Moreover, for all \(x \in (\dom f) \cap (\dom g)\),
    \(\restr{f}{A}(x) = f(x) = g(x) = \restr{g}{A}(x)\).
    Hence, the result follows by \Cref{lem:functionEqualsIff}.

    (\(\Leftarrow\))
    Take any \(x \in A\).
    Then, \(f(x) = \restr{f}{A}(x) = \restr{g}{A}(x) = g(x)\). \qed
\end{enumerate}
}

\thm[compatibleThenUnionIsFunction]{}{
    If \(F\) is a compatible system of functions, then \(\bigcup F\)
    is a function with \(\dom \bigcup F = \bigcup \{\,\dom f \mid f \in F\,\}\).
    The function \(\bigcup F\) extends all \(f \in F\).
}
\pf{Proof}{
    Note that \(\bigcup F\) is already a relation.
    If \((a, b_1), (a, b_2) \in \bigcup F\),
    then there exist \(f_1, f_2 \in F\) such that \((a, b_1) \in f_1\) and \((a, b_2) \in f_2\).
    Since \(f_1\) and \(f_2\) are compatible and \(a \in (\dom f_1) \cap (\dom f_2)\),
    we have \(b_1 = f_1(a) = f_2(a) = b_2\). Hence, \(\bigcup F\) is a function.

    \(\dom \bigcup F = \bigcup \{\,\dom f \mid f \in F\,\}\) since
    \[\begin{aligned}[b]
        x \in \dom \bigcup F &\iff \exs y,\:(x, y) \in \bigcup F \\
                             &\iff \exs y,\: \exs f \in F,\: (x, y) \in f \\
                             &\iff \exs f \in F,\: x \in \dom f \iff x \in \bigcup \{\,\dom f \mid f \in F\,\}.
    \end{aligned}\]

    Take any \(f \in F\).
    As \(f \cup \bigcup F = \bigcup F\), \(f\) and \(\bigcup F\) are compatible
    by \Cref{lem:compatibleIff} (i).
    Moreover, \(\dom f \cap \dom \bigcup F = \dom f\).
    Hence, by \Cref{lem:compatibleIff} (ii),
    \(f = \restr{f}{\dom f} = \restr{\left(\bigcup F\right)}{\dom f}\);
    \(\bigcup F\) extends each \(f \in F\).
}

\dfn[setOfFunctions]{}{
    Let \(A\) and \(B\) be sets.
    Then, we define
    \[
        B^A \triangleq \{\,f \mid f \text{ is a function on }A\text{ into }B\,\}.
    \]
}

\dfn[indexedSystemOfSets]{Indexed System of Sets}{
    \begin{itemize}[nolistsep]
        \ii 
        Let \(S = \lang S_i \mid i \in I \rang\) be a function with domain \(I\).
        We call the function \(S\) an \textit{indexed system of sets}
        whenever we stress that the values of \(S\) are sets.

        \ii
        We say that a system of sets \(A\) is \textit{indexed} by \(S\)
        if \(A = \{\,S_i \mid i \in I\,\} = \ran S\).
    \end{itemize}
}

\notat{}{
    If \(A\) is indexed by \(S = \lang S_i \mid i \in I \rang\),
    we may write
    \[
        \bigcup \{\,S_i \mid i \in I\,\}\quad\text{or}\quad\bigcup_{i \in I} S_i
    \]
    instead of \(\bigcup A\).
    Similarly, we may write \(\bigcap \{\,S_i \mid i \in I\,\}\) or \(\bigcap_{i \in I} S_i\)
    instead of \(\bigcap A\).
}

\dfn[productIndexedSystemOfSets]{Product of Indexed System of Sets}{
    Let \(S = \lang S_i \mid i \in I \rang\) be an indexed system of sets.
    We call the set
    \[
        \prod S \triangleq \{\,f \mid f \text{ is a function on }I\text{ and } \fall i \in I,\: f_i \in S_i\,\}
    \]
    the \textit{product} of the indexed system \(S\).
}

\notat{}{
    Other notations for the product of the indexed system \(S = \lang S_i \mid i \in I \rang\)
    are:
    \[
        \prod \lang S(i) \mid i \in I \rang, \quad
        \prod_{i \in I} S(i), \quad
        \prod_{i \in I} S_i.
    \]
}

\nt{
    \noindent
    The existence of \(B^A\) and \(\prod_{i \in I} S_i\) is proved in \Cref{exer:2.3.9}.
}

\subsection*{Selected Problems}

\setexernumber{3}
\exer[2.3.4]{}{
    Let \(f\) be a function.
    If there exists a function \(g\) such that \(g \circ f = \mrm{Id}_{\dom f}\),
    then \(f\) is invertible and \(f\inv = \restr{g}{\ran f}\).
}
\pf{Proof}{
    For \(x_1, x_2 \in \dom f\), suppose \(f(x_1) = f(x_2)\).
    Then, \(x_1 = (g \circ f)(x_1) = g(f(x_1)) = g(f(x_2)) = (g \circ f)(x_2) = x_2\).
    Hence, \(f\) is one-to-one and is inverible by \Cref{th:invIffInj}.

    Take any \((y, x) \in f\inv\). Then, as \(x \in \dom f\),
    we must have \((y, x) \in \mrm{Id}_{\dom f}\). Hence, \(f\inv \subseteq \restr{g}{\ran f}\).
    Now, take any \((y, x) \in \restr{g}{\ran f}\). Since \(y \in \ran f\),
    there exists \(x' \in \dom f\) such that \((x', y) \in f\).
    Since \(g \circ f = \mrm{Id}_{\dom f}\),
    we have \(x = x'\). Therefore, \((y, x) \in f\inv\); \(\restr{g}{\ran f} \subseteq f\inv\).
}

\setexernumber{5}

\exer[2.3.6]{}{
    Let \(f\) be a function.
    \begin{enumerate}[nolistsep, label=(\roman*)]
        \ii \(f\inv[A \cap B] = f\inv[A] \cap f\inv[B]\)
        \ii \(f\inv[A \setminus B] = f\inv[A] \setminus f\inv[B]\)
    \end{enumerate}
}
\mclm{Proof}{
Thanks to \Cref{exer:2.2.3} (ii) and (iii), we only need to prove the other inclusions.
\begin{enumerate}[nolistsep, label=(\roman*)]
    \ii Take any \(x \in f\inv[A] \cap f\inv[B]\).
        Then, there exists \(a \in A\) and \(b \in B\)
        such that \(xfa\) and \(xfb\).
        Since \(f\) is a function, \(a = b\),
        and thus \(x \in f\inv[A \cap B]\).

    \ii Take any \(x \in f\inv[A \setminus B]\).
        Then, \(f(x) \in A \setminus B\).
        If \(x \in f\inv[B]\), we would have \(f(x) \in B\);
        thus \(x \notin f\inv[B]\). Therefore, \(x \in f\inv[A] \setminus f\inv[B]\). \qed
\end{enumerate}
}

\setexernumber{7}

\exer[2.3.8]{}{
    Every system of sets \(A\) can be indexed by a function.
}
\pf{Proof}{
    Let \(S\) be the function \(\mrm{Id}_A\)
    so \(S_i = i\) for all \(i \in A\).
    Then, \(A = \{\,S_i \mid i \in A\,\}\);
    \(A\) is indexed by \(S\).
}

\exer[2.3.9]{}{
    \begin{enumerate}[nolistsep, label=(\roman*)]
        \ii Let \(A\) and \(B\) be sets. Prove that \(B^A\) exists.
        \ii Let \(\lang S_i \mid i \in I\rang\) be an indexed system of sets.
            Prove that \(\prod_{i \in I} S_i\) exists.
    \end{enumerate}
}
\mclm{Proof}{\hfill
\begin{enumerate}[nolistsep, label=(\roman*)]
    \ii
    If \(f\) is a function from \(A\) into \(B\),
    then \(f \subseteq A \times B\), i.e., \(f \in \mcal P(A \times B)\).

    \ii
    If \(f\) is a function on \(I\) and \(f_i \in S_i\) for all \(i \in I\),
    then \(f\) is a function onto \(\bigcup_{i \in I} S_i\).
    Hence, \(f \in \left(\bigcup_{i \in I} S_i\right)^I\). \qed
\end{enumerate}
}

\section{Equivalences and Partitions}

\dfn[equivalence]{Equivalence}{
    Let \(R\) be a binary relation in \(A\).
    \begin{itemize}[nolistsep]
        \ii \(R\) is called \textit{reflexive in} \(A\) if \(\fall a \in A,\: aRa\).
        \ii \(R\) is called \textit{symmetric in} \(A\) if \(\fall a, b \in A,\: (aRb \implies bRa)\).
        \ii \(R\) is called \textit{transitive in} \(A\) if \(\fall a, b, c \in A,\: (aRb \land bRc \implies aRc)\).
        \ii \(R\) is called an \textit{equivalence on} \(A\) if it is reflexive, symmetric, and transitive in \(A\).
    \end{itemize}
}

\dfn[equivalenceClass]{Equivalence Class}{
    Let \(E\) be an equivalence on \(A\) and let \(a \in A\).
    The \textit{equivalence class of \(a\) modulo} \(E\) is the set
    \[
        [a]_E \triangleq \{\,x \in A \mid xEa\,\}.
    \]
}

\mlemma[equivIffSameClass]{}{
    Let \(E\) be an equivalence on \(A\) and let \(a, b \in A\).
    \begin{enumerate}[nolistsep, label=(\roman*)]
        \ii \(aEb \iff [a]_E = [b]_E\)
        \ii \(\lnot (aEb) \iff [a]_E \cap [b]_E = \OO\)
    \end{enumerate}
}
\mclm{Proof}{\hfill
\begin{enumerate}[nolistsep, label=(\roman*)]
    \ii 
    (\(\Rightarrow\))
    Suppose \(aEb\).
    Take any \(c \in [a]_E\). Then, \(cEa\) and \(aEb\), and thus \(cEb\)
    by transitivity. Hence, \(c \in [b]_E\); \([a]_E \subseteq [b]_E\).
    \([b]_E \subseteq [a]_E\) can be shown similarly since \(bEa\) holds as \(E\) is symmetric.

    (\(\Leftarrow\))
    Suppose \([a]_E = [b]_E\). Since \(aEa\) by reflexivity,
    we have \(a \in [a]_E = [b]_E\). Therefore, \(aEb\).

    \ii
    (\(\Rightarrow\))
    Suppose \([a]_E \cap [b]_E \neq \OO\).
    Then, there exists \(c \in [a]_E \cap [b]_E\), i.e., \(cEa\) and \(cEb\).
    Then, as \(E\) is symmetric, we have \(aEc\),
    and therefore \(aEb\) by transitivity.

    (\(\Leftarrow\))
    Suppose \(aEb\). Then, since \(aEa\) by reflexivity,
    we have \(a \in [a]_E\). We can see \(a \in [b]_E\) from (i).
    Hence, \([a]_E \cap [b]_E \neq \OO\).
    \qed
\end{enumerate}
}

\dfn[partition]{Partition}{
    A system \(S\) of nonempty sets is called a \textit{partition} of \(A\) if
    \begin{enumerate}[nolistsep, label=(\roman*)]
        \ii \(S\) is a system of mutually disjoint sets (\Cref{dfn:mutualDisjoint}) and
        \ii \(\bigcup S = A\).
    \end{enumerate}
}

\dfn[allEquivClasses]{System of All Equivalence Classes}{
    Let \(E\) be an equivalence on \(A\).
    The \textit{system of all equivalence classes} modulo \(E\)
    is the set
    \[
        A/E \triangleq \{\,[a]_E \mid a \in A\,\}.
    \]
}

\thm[equivDerivesPartition]{}{
    Let \(E\) be an equivalence on \(A\).
    Then, \(A/E\) is a partition of \(A\).
}
\pf{Proof}{
    If \([a]_E \neq [b]_E\), then by \Cref{lem:equivIffSameClass},
    we have \([a]_E \cap [b]_E = \OO\).
    Since \(E\) is reflexive, \(a \in [a]_E\); each \([a]_E\) is nonempty.
    Therefore, \(A/E\) is a system of mutually disjoint nonempty sets.

    Take any \(a \in A\).
    Since \(E\) is reflexive, \(a \in [a]_E \subseteq \bigcup A/E\).
    Therefore, \(A \subseteq \bigcup A/E\).
    Conversely, since \([a]_E \subseteq A\), we have \(\bigcup A/E \subseteq A\).
}

\dfn[equivalenceFromPartition]{}{
    Let \(S\) be a partition of \(A\). The relation \(E_S\) in \(A\) is defined by
    \[
        E_S \triangleq \{\,(a, b) \in A \times A \mid \exs C \in S,\: a \in C \land b \in C\,\}.
    \]
}

\thm[partitionDerivesEquiv]{}{
    Let \(S\) be a partition of \(A\).
    Then, \(E_S\) is a equivalence on \(A\).
}
\mclm{Proof}{\hfill
\begin{itemize}[nolistsep]
    \ii 
    Take any \(a \in A\). As \(A = \bigcup S\), there exists \(C \in S\) such that \(a \in C\).
    Therefore, \(aE_Sa\). \(E_S\) is reflexive.

    \ii
    Assume \(aE_Sb\). Then, there exists \(C \in S\) such that \(a, b \in C\).
    Hence, \(b E_S a\). \(E_S\) is symmetric.

    \ii
    Assume \(aE_Sb\) and \(bE_Sc\).
    Then, there exist \(C, D \in S\) such that
    \(a, b \in C\) and \(b, c \in D\).
    Then, \(C \cap D \neq \OO\) as \(b\) belongs to both sets.
    Hence, \(C = D\), which implies \(aE_Sc\). \(E_S\) is transitive.
    \qed
\end{itemize}
}

\thm[equivAndPartAreSame]{}{
    \begin{enumerate}[nolistsep, label=(\roman*)]
        \ii If \(E\) is an equivalence on \(A\) and \(S = A/E\), then \(E_S = E\).
        \ii If \(S\) is a partition of \(A\), then \(A/E_S = S\).
    \end{enumerate}
}
\mclm{Proof}{\hfill
\begin{enumerate}[nolistsep, label=(\roman*)]
    \ii 
    \(aE_Sb \underbrace{\iff}_{\text{\Cref{dfn:equivalenceFromPartition}}} \exs C \in S,\: a \in C \land b \in C
    \iff \exs c \in A,\: a \in [c]_E \land b \in [c]_E \underbrace{\iff}_{\text{\Cref{lem:equivIffSameClass}}} aEb \).

    \ii
    Take any \([a]_{E_S} \in A/E_S\).
    Since \(S\) is a partition, there (uniquely) exists \(C\) such that \(a \in C\).
    Then, for all \(b\), we have \(
        b \in C
        \iff a E_S b
        \underbrace{\iff}_{\text{\Cref{lem:equivIffSameClass}}} b \in [a]_{E_S}\);
    \(C = [a]_{E_S}\). Therefore, \(A/E_S \subseteq S\).

    For the converse, take any \(C \in S\).
    As \(C\) is nonempty, we may take some \(a \in C\).
    Similarly, we have \(C = [a]_{E_S}\). Therefore, \(C \subseteq A/E_S\).
    \qed
\end{enumerate}
}

\nt{
    \Cref{th:equivAndPartAreSame} essentially states that
    equivalence and partition describe the same ``mathematical reality.''
}

\dfn[setOfRepresentatives]{Set of Representatives}{
    A set \(X \subseteq A\) is called a \textit{set of representatives}
    for the equivalence \(E_S\) (or for the partition \(S\) of \(A\)) if
    \[
        \fall C \in S,\: \exs a \in C,\: X \cap C = \{a\}.
    \]
}

\subsection*{Selected Problems}
\setexernumber{1}

\exer[2.4.2]{}{
    Let \(f\) be a function on \(A\) onto \(B\).
    Define a relation \(E\) in \(A\) by: \(aEb\) if and only if \(f(a) = f(b)\).
    \begin{enumerate}[nolistsep, label=(\roman*)]
        \ii
        Show that \(E\) is an equivalence on \(A\).
        \ii
        Show that \([a]_E = [a']_E\) implies that \(f(a) = f(a')\)
        so that the function \(\varphi\) on \(A/E\) into \(B\)
        defined by \(\varphi([a]_E) = f(a)\) is well-defined.
        Show also that \(\varphi\) is \textit{onto} \(B\).
        \ii
        Let \(j\) be the function on \(A\) onto \(A/E\) given by \(j(a) = [a]_E\).
        Show that \(\varphi \circ j = f\).
    \end{enumerate}
}
\mclm{Proof}{\hfill
\begin{enumerate}[nolistsep, label=(\roman*)]
    \ii
    \(E\) can readily be shown to be reflexive, symmetric, and transitive.

    \ii
    Assume \([a]_E = [a']_E\). Then, \(f(a) = f(a')\) by definition of \(E\). Hence, \(\varphi\) is well-defined.
    Take any \(b \in B\). Since \(f\) is onto, there exists \(a \in A\) such that \(f(a) = b\).
    Hence, \(\varphi([a]_E) = f(a) = b\); \(\varphi\) is onto \(B\).

    \ii
    \(\dom (\varphi \circ j) = (\dom j) \cap j\inv[\dom \varphi] = A = \dom f\)
    since \(j\) is onto.
    For all \(a \in A\), \((\varphi \circ j)(a) = \varphi([a]_E) = f(a)\).
    Hence, by \Cref{lem:functionEqualsIff}, \(\varphi \circ j = f\).
    \qed
\end{enumerate}
}

\section{Orderings}

\dfn[ordering]{Partial Ordering and Strict Ordering}{
    Let \(R\) be a binary relation in \(A\).
    \begin{itemize}[nolistsep]
        \ii \(R\) is called \textit{antisymmetric in} \(A\) if \(\fall a, b \in A,\: (aRb \land bRa \implies a=b)\).
        \ii \(R\) is called \textit{asymmetric in} \(A\) if \(\fall a, b \in A,\: \lnot (aRb \land bRa)\).
        \ii \(R\) is called a \textit{(partial) ordering} of \(A\) if it is reflexive, antisymmetric, and transitive in \(A\).
        \ii \(R\) is called a \textit{strict ordering} of \(A\) if it is asymmetric and transitive in \(A\).
        \ii If \(R\) is a partial ordering of \(A\), then the pair \((A, R)\) is called an \textit{ordered set}.
    \end{itemize}
}

\exmp{}{
\begin{itemize}[nolistsep]
    \ii 
    Define the relation \(\subseteq_A\) in \(A\) as follows:
    \(x \subseteq_A y\) if and only if \(x, y \in A \land x \subseteq y\).
    Then, \((A, \subseteq_A)\) is an ordered set.
    \ii
    The relation \(\mrm{Id}_A\) is a partial ordering of \(A\).
\end{itemize}
}

\thm[partialAndStrictAreSame]{}{
    \begin{enumerate}[nolistsep, label=(\roman*)]
        \ii
        Let \(R\) be a partial ordering of \(A\). Then the relation \(S\) in \(A\) defined by
        \[
            S \triangleq R \setminus \mrm{Id}_A
        \]
        is a strict ordering.

        \ii
        Let \(S\) be a strict ordering of \(A\). Then the relation \(R\) in \(A\) defined by
        \[
            R \triangleq S \cup \mrm{Id}_A
        \]
        is a partial ordering.
    \end{enumerate}
}
\mclm{Proof}{\hfill
\begin{enumerate}[nolistsep, label=(\roman*)]
    \ii
    Suppose \(aSb\) and \(bSa\). 
    Since \(S \subseteq R\), we have \(aRb\) and \(bRa\).
    As \(R\) is antisymmetric, we have \(aRa\),
    which is impossible since \(S \cap \mrm{Id}_S = \OO\).
    Hence, \(S\) is asymmetric in \(A\).

    Now, assuming \(aSb\) and \(bSc\),
    we also have \(aRc\) since \(R\) is transitive.
    Moreover, \(a\) cannot be equal to \(c\) since \(S\) is shown to be asymmetric.
    Therefore, \(aSc\); \(S\) is transitive in \(A\).

    \ii
    Assume \(aRb\) and \(bRa\).
    If \(a \neq b\), then we have \(aSb\) and \(bSa\), which is impossible.
    Therefore, \(a = b\); \(R\) is antisymmetric.

    Assume \(aRb\) and \(bRc\).
    If \(a = b\) or \(b = c\), then we immediately have \(aRc\).
    If \(a \neq b\) and \(b \neq c\), then \(aSb\) and \(bSc\),
    and thus \(aSc\) as \(S\) is transitive in \(A\);
    \(R\) is transitive in \(A)\).

    \(R\) is reflexive in \(A\) since \(\mrm{Id}_A \subseteq R\).
    \qed
\end{enumerate}
}

\notat{}{
\begin{itemize}[nolistsep]
    \ii If \(R\) is a partial ordering, we say \(S = R \setminus \mrm{Id}_A\)
        \textit{corresponds to the partial ordering} \(R\).
    \ii If \(S\) is a strict ordering, we say \(R = S \cup \mrm{Id}_A\)
        \textit{corresponds to the strict ordering} \(S\).
\end{itemize}
}

\dfn[comparable]{Comparability}{
    Let \(a, b \in A\) and let \(\le\) be a partial ordering of \(A\).
    \begin{itemize}[nolistsep]
        \ii 
        We say that \(a\) and \(b\) are \textit{comparable} in the ordering \(\le\)
        if \(a \le b\) or \(b \le a\).

        \ii
        We say that \(a\) and \(b\) are \textit{incomparable} in the ordering \(\le\)
        if neither \(a \le b\) nor \(b \le a\).
    \end{itemize}
    They can be stated equivalently in terms of the corresponding strict ordering \(<\).
    \begin{itemize}[nolistsep]
        \ii 
        We say that \(a\) and \(b\) are \textit{comparable} in the ordering \(<\)
        if \(a = b\) or \(a < b\) or \(b < a\).

        \ii
        We say that \(a\) and \(b\) are \textit{incomparable} in the ordering \(<\)
        if none of \(a = b\), \(a < b\), and \(b < a\) holds.
    \end{itemize}
}

\dfn[totalOrdering]{Total Ordering}{
    An ordering \(\le\) (or \(<\)) is called \textit{linear} or \textit{total}
    if any two elements of \(A\) are comparable.
    The pair \((A, \le)\) is then called a \textit{linearly ordered set}.
}

\dfn[chain]{Chain}{
    Let \((A, \le)\) be an ordered set and \(B \subseteq A\).
    \(B\) is a \textit{chain} in \(A\) if any two elements of \(B\) are comparable.
}

\dfn[leastMinimal]{Least/Minimal/Greatest/Maximal Element}{
    Let \((A, \le)\) be an ordered set and \(B \subseteq A\).
    \begin{itemize}[nolistsep]
        \ii
        \(b \in B\) is the \textit{least element} of \(B\) in the ordering \(\le\)
        if \(\fall x \in B,\: b \le x\).

        \ii
        \(b \in B\) is a \textit{minimal element} of \(B\) in the ordering \(\le\)
        if \(\fall x \in B,\: (x \le b \implies x = b)\).

        \ii
        \(b \in B\) is the \textit{greatest element} of \(B\) in the ordering \(\le\)
        if \(\fall x \in B,\: x \le b\).

        \ii
        \(b \in B\) is a \textit{maximal element} of \(B\) in the ordering \(\le\)
        if \(\fall x \in B,\: (b \le x \implies x = b)\).
    \end{itemize}
}

\notat{}{
    Let \((A, \le)\) be an ordered set and \(B \subseteq A\).
    \begin{itemize}[nolistsep]
        \ii The least element of \(B\) is denoted \(\min B\).
        \ii The greatest element of \(B\) is denoted \(\max B\).
    \end{itemize}
}

\thm[basicLeastMinimal]{}{
    Let \((A, \le)\) be an ordered set and \(B \subseteq A\).
    \begin{enumerate}[nolistsep, label=(\roman*)]
        \ii \(B\) has at most one least element.
        \ii The least element of \(B\)---it it exists---is also minimal.
        \ii If \(B\) is a chain, then every minimal element of \(B\) is also least.
    \end{enumerate}
}
\mclm{Proof}{\hfill
\begin{enumerate}[nolistsep, label=(\roman*)]
    \ii
    If \(b\) and \(b'\) are least elements of \(B\),
    then \(b \le b'\) and \(b' \le b\) by the definition.
    As \(\le\) is antisymmetric, we have \(b = b'\).

    \ii
    Let \(b\) be the least element of \(B\) (assuming its existence).
    Take any \(x \in B\) and assume \(x \le b\).
    Then, as \(b\) is the least, we have \(b \le x\).
    As \(\le\) is antisymmetric, \(x = b\); \(b\) is minimal.

    \ii
    Let \(b\) be a minimal element of \(B\).
    Take any \(x \in B\).
    Since \(b\) and \(x\) are comparable, it is \(x \le b\) or \(b \le x\).
    If \(x \le b\), then \(x = b\) as \(b\) is minimal.
    Therefore, \(b\) is the least.
    \qed
\end{enumerate}
}

\nt{
    \noindent
    \Cref{th:basicLeastMinimal} still holds when `least' and `minimal' are replaced by `greatest' and `maximal', respectively.
}

\dfn[bound]{Lower/Upper Bound and Infimum/Supremum}{
    Let \((A, \le)\) be an ordered set and \(B \subseteq A\).
    \begin{itemize}[nolistsep]
        \ii
        \(a \in A\) is a \textit{lower bound} of \(B\) in the ordered set \((A, \le)\)
        if \(\fall x \in B,\: a \le x\).
        \ii
        \(a \in A\) is called an \textit{infimum} (or \textit{greatest lower bound}) of \(B\) in the ordered set \((A, \le)\)
        if \(a = \max \{\,x \in A \mid x \text{ is a lower bound of }B\,\}\).
        \ii
        \(a \in A\) is an \textit{upper bound} of \(B\) in the ordered set \((A, \le)\)
        if \(\fall x \in B,\: x \le a\).
        \ii
        \(a \in A\) is called an \textit{supremum} (or \textit{least upper bound}) of \(B\) in the ordered set \((A, \le)\)
        if \(a = \min \{\,x \in A \mid x \text{ is an upper bound of }B\,\}\).
    \end{itemize}
}

\notat{}{
    Let \((A, \le)\) be an ordered set and \(B \subseteq A\).
    \begin{itemize}[nolistsep]
        \ii The infimum of \(B\) is denoted \(\inf B\).
        \ii The supremum of \(B\) is denoted \(\sup B\).
    \end{itemize}
}

\thm[basicInfimum]{}{
    Let \((A, \le)\) be an ordered set and \(B \subseteq A\).
    \begin{enumerate}[nolistsep, label=(\roman*)]
        \ii \(B\) has at most one infimum.
        \ii If \(b\) is the least element of \(B\), then \(b\) is the infimum of \(B\).
        \ii If \(b \in B\) is the infimum of \(B\), then \(b\) is the least element of \(B\).
    \end{enumerate}
}
\mclm{Proof}{\hfill
\begin{enumerate}[nolistsep, label=(\roman*)]
    \ii
    The result follows from the definition and \Cref{th:basicLeastMinimal} (i).

    \ii
    \(b\) is a lower bound of \(B\).
    If \(x\) is a lower bound of \(B\), since \(b \in B\), we must have \(x \le b\).
    Therefore, \(b\) is the greatest lower bound.

    \ii
    \(b \in B\) is a lower bound of \(B\), and thus \(b\) is the least element.
    \qed
\end{enumerate}
}

\nt{
    \noindent
    \Cref{th:basicInfimum} still holds when `least' and `infimum' are replaced by `greatest' and `supremum', respectively.
}

\dfn[]{Isomorphism Between Ordered Sets}{
    An \textit{isomorphism} between two ordered sets \((P, \le)\) and \((Q, \preceq)\)
    is a function \(f \colon P \hooktwoheadrightarrow Q\) such that
    \[
        \fall p_1, p_2 \in P,\: (p_1 \le p_2 \iff f(p_1) \preceq f(p_2)).
    \]
    If an isomorphism exists between \((P, \le)\) and \((Q, \preceq)\),
    then we say \((P, \le)\) and \((Q, \preceq)\) are \textit{isomorphic}.
    This is justified by \Cref{exer:2.5.13}.
}

\mlemma[oneImplicationIsEnough]{}{
    Let \((P, \le)\) be a linearly ordered set and let \((Q, \preceq)\) be an ordered set.
    Let \(h \colon P \hooktwoheadrightarrow Q\) be a function such that
    \[
        \fall p_1, p_2 \in P,\: (p_1 \le p_2 \implies h(p_1) \preceq h(p_2)).
    \]
    Then, \(h\) is an isomorphism between \((P, \le)\) and \((Q, \preceq)\),
    and \((Q, \le)\) is linearly ordered.
}
\pf{Proof}{
    Take any \(p_1, p_2 \in P\) and assume \(h(p_1) \preceq h(p_2)\).
    Suppose \(p_2 < p_1\) for the sake of contradiction.
    Then, since \(h\) is injective, \(h(p_1) \neq h(p_2)\), and thus \(h(p_1) \prec h(p_2)\).
    Then, we have \(\lnot (p_2 \le p_1)\), which is a contradiction.
    Hence, \(\lnot(p_2 < p_1)\).
    Therefore, \(p_1 \le p_2\) since \((P, \le)\) is linearly ordered.

    Take any \(q_1, q_2 \in Q\).
    Then, since \(h\) is onto \(Q\), there exist \(p_1, p_2 \in P\)
    such that \(q_1 = h(p_1)\) and \(p_2 = h(p_2)\).
    Since \(P\) is linearly ordered, it is \(p_1 \le p_2\) or \(p_2 \le p_1\).
    In either case, we have \(q_1 \preceq q_2\) or \(p_2 \preceq q_1\).
    Therefore, \((Q, \preceq)\) is linearly ordered.
}

\subsection*{Selected Problems}

\exer[2.5.1]{}{
    \begin{enumerate}[nolistsep, label=(\roman*)]
        \ii
        Let \(R\) be a partial ordering of \(A\) and let \(S\) be the strict ordering of \(A\) corresponding to \(R\).
        Let \(R^\ast\) be the partial ordering of \(A\) corresponding to \(S\). Show that \(R^\ast = R\).
        \ii
        Let \(S\) be a strict ordering of \(A\) and let \(R\) be the partial ordering of \(A\) corresponding to \(S\).
        Let \(S^\ast\) be the partial ordering of \(A\) corresponding to \(R\). Show that \(S^\ast = S\).
    \end{enumerate}
}
\mclm{Proof}{\hfill
\begin{enumerate}[nolistsep, label=(\roman*)]
    \ii
    \(R^\ast = S \cup \mrm{Id}_A = (R \setminus \mrm{Id}_A) \cup \mrm{Id}_A = R\)
    since \(\mrm{Id}_A \subseteq R\).

    \ii
    \(S^\ast = R \setminus \mrm{Id}_A = (S \cup \mrm{Id}_A) \setminus \mrm{Id}_A = S\)
    since \(\mrm{Id}_A \cap S = \OO\).

    \qed
\end{enumerate}
}

\setexernumber{5}

\exer[2.5.6]{}{
    Let \((A_1, <_1)\) and \((A_2, <_2)\) be strictly ordered sets and let \(A_1 \cap A_2 = \OO\).
    Define a relation \(\prec\) on \(B \triangleq A_1 \cup A_2\) as follows:
    \[
        x \prec y \iff (x <_1 y) \lor (x <_2 y) \lor (x \in A_1 \land y \in A_2).
    \]
    Show that \(\prec\) is a strict ordering of \(B\) and \(\mathord{\prec} \cap A_1^2 = \mathord{<_1}\), \(\mathord{\prec} \cap A_2^2 = \mathord{<_2}\).
}
\pf{Proof}{
    Note that \(\mathord{\prec} = \mathord{<}_1 \cup \mathord{<}_2 \cup A_1 \times A_2\).

    Suppose \(x \prec y\) and \(y \prec x\).
    By definition, \(x, y \in A_1\) or \(x, y \in A_2\).
    In both cases, we have (\(x <_1 y\) and \(y <_1 x\))
    or (\(x <_2 y\) and \(y <_2 x\)), which are impossible as \(<_1\) and \(<_2\) are asymmetric.
    Hence, \(\prec\) is asymmetric.
    Transitivity of \(\prec\) can be shown easily.

    Since \(\mathord{<_1} \cap A_2^2 = \mathord{<_2} \cap A_1^2 = (A_1 \times A_2) \cap A_1^2 = (A_1 \times A_2) \cap A_2^2 = \OO\),
    we get \(\mathord{\prec} \cap A_1^2 = \mathord{<_1}\) and \(\mathord{\prec} \cap A_2^2 = \mathord{<_2}\).
}

\exer[2.5.7]{}{
    Let \(R\) be a reflexive and transitive relation in \(A\) (\(R\) is called a \textit{preordering} of \(A\)).
    Define a relation \(E\) in \(A\) by
    \[
        aEb \iff aRb \land bRa.
    \]
    Show that \(E\) is an equivalence on \(A\). Define the relation \(R/E\) in \(A/E\) by
    \[
        [a]_E R/E [b]_E \iff aRb.
    \]
    Show that \(R/E\) is well-defined and that \(R/E\) is a partial ordering of \(A/E\).
}
\pf{Proof}{
    Since \(aEa \equiv aRa\) and \(R\) is reflexive, \(E\) is reflexive as well.
    Since \(aEb \equiv bEa\), \(E\) is symmetric.
    Since \(aEb \land bEc \iff (aRb \land bRc) \land (cRb \land bRa) \implies aRc \land cRa \iff aEc\),
    \(E\) is transitive. \checkmark

    Assume \([a]_E = [a']_E\) and \([b]_E = [b']_E\).
    Then, we have \(aEa'\) and \(bEb'\) by \Cref{lem:equivIffSameClass},
    i.e., \(aRa'\), \(a'Ra\), \(bRb'\), and \(b'Rb\).
    By transitivity of \(R\), it follows that \(aRb \iff a'Rb'\).
    Therefore, \(R/E\) is well-defined. \checkmark

    It can be shown readily that \(R/E\) is reflexive and transitive.
    To prove \(R/E\) is antisymmetric, assume \([a]_E R/E [b]_E\) and \([b]_E R/E [a]_E\).
    Then, \(aRb\) and \(bRa\), which means \(aEb\).
    Therefore, \([a]_E = [b]_E\) by \Cref{lem:equivIffSameClass}. \checkmark
}

\exer[2.5.8]{}{
    Let \(A = \mcal P(X)\) where \(X\) is a set.
    \begin{enumerate}[nolistsep, label=(\roman*)]
        \ii
        Any \(S \subseteq A\) has a supremum in the ordering \(\subseteq_A\); \(\sup S = \bigcup S\).
        \ii
        Any \(S \subseteq A\) has an infimum in the ordering \(\subseteq_A\);
        \(\inf S = \begin{cases}
            \bigcap S & \text{if } S \neq \OO \\
            X & \text{if } S = \OO
        \end{cases}\).
    \end{enumerate}
}
\mclm{Proof}{\hfill
\begin{enumerate}[nolistsep, label=(\roman*)]
    \ii
    As \(C \subseteq_A \bigcup S\) for all \(C \in S\), \(\bigcup S\) is an upper bound of \(S\).
    Let \(U\) be any upper bound of \(S\).
    Take any \(x \in \bigcup S\). Then, there exists \(C \in S\) such that \(x \in C\).
    Since \(C \subseteq_A U\), we have \(x \in U\).
    Therefore, \(\bigcup S \subseteq U\); \(\bigcup S\) is the least upper bound of \(S\).

    \ii
    If \(S = \OO\), then any \(C \in A\) is an lower bound of \(S\).
    Since \(\bigcup A = X\)---by (i), the supremum of the set of lower bounds of \(S\)---is a lower bound of \(S\), 
    \(X\) is the infimum of \(S = \OO\). \checkmark

    If \(S \neq \OO\), as \(\bigcap S \subseteq C\) for all \(C \in S\),
    \(\bigcap S\) is a lower bound of \(S\).
    Let \(L\) be any lower bound of \(S\).
    Take any \(x \in L\). Then, \(\fall C \in L,\: x \in C\), i.e., \(x \in \bigcap S\).
    Therefore, \(L \subseteq_A \bigcap S\); \(\bigcap S\) is the infimum of \(S\). \checkmark
    \qed
\end{enumerate}
}

\exer[2.5.8]{}{
    Let \(\mrm{Fn}(X, Y)\) be the set of all functions mapping a subset of \(X\) into \(Y\),
    i.e., \(\mrm{Fn}(X, Y) = \bigcup_{Z \in \mcal P(X)} Y^Z\).
    Define a relation \(\le\) in \(\mrm{Fn}(X, Y)\) by
    \[
        f \le g \iff f \subseteq g.
    \]
    \begin{enumerate}[nolistsep, label=(\roman*)]
        \ii \(\le\) is a partial ordering of \(\mrm{Fn}(X, Y)\).
        \ii
        Let \(F \subseteq \mrm{Fn}(X, Y)\).
        \(\sup F\) exists if and only if \(F\) is a compatible system of functions.
        Moreover, \(\sup F = \bigcup F\) if it exists.
    \end{enumerate}
}
\mclm{Proof}{\hfill
\begin{enumerate}[nolistsep, label=(\roman*)]
    \ii
    \(\mathord{\le} = \mathord{\subseteq}_{\mrm{Fn}(X, Y)}\) by definition;
    \(\subseteq_{\mrm{Fn}(X, Y)}\) is already a partial ordering of \(\mrm{Fn}(X, Y)\).

    \ii
    (\(\Rightarrow\)) Assume \(h \in \mrm{Fn}(X, Y)\) is a supremum of \(F\).
    Then, \(\fall f \in F,\: f \subseteq s\).
    Take any \(f, g \in F\).
    Then, \(f \cup g \subseteq h\), and thus \(f \cup g\) is a function as \(h\) is a function.
    Therefore, by \Cref{lem:compatibleIff}, \(f\) and \(g\) are compatible.
    Hence, \(F\) is a compatible system of functions.

    (\(\Leftarrow\))
    Assume \(F\) is a compatible system of functions.
    Then, \(\bigcup F \in \mrm{Fn}(X, Y)\) by \Cref{th:compatibleThenUnionIsFunction},
    and \(f \le \bigcup F\) for all \(f \in F\) by definition; \(\bigcup F\) is an upper bound of \(F\).
    Let \(U\) be any upper bound of \(S\).
    Take any \((x, y) \in \bigcup F\). Then, there exists \(f \in S\) such that \((x, y) \in f\).
    Since \(f \subseteq_A U\), we have \(x \in U\).
    Therefore, \(\bigcup F \subseteq U\); \(\bigcup F\) is the least upper bound of \(S\).
    \qed
\end{enumerate}
}

\exer[2.5.10]{}{
    Let \(\mrm{Pt}(A)\) be the set of all partitions of \(A\).
    Define a relation $\preccurlyeq$ in \(\mrm{Pt}(A)\) by
    \[
        S_1 \preccurlyeq S_2 \iff \fall C \in S_1,\: \exs D \in S_2,\: C \subseteq D.
    \]
    (We say that the partition \(S_1\) is a \textit{refinement} of the partition \(S_2\) if \(S_1 \preccurlyeq S_2\).)
    \begin{enumerate}[nolistsep, label=(\roman*)]
        \ii
        \(\preccurlyeq\) is a partial ordering of \(\mrm{Pt}(A)\).
        \ii
        \(\inf T\) exists for all \(T \subseteq \mrm{Pt}(A)\).
        \ii
        \(\sup T\) exists for all \(T \subseteq \mrm{Pt}(A)\).
    \end{enumerate}
}
\mclm{Proof}{\hfill
\begin{enumerate}[nolistsep, label=(\roman*), listparindent=\parindent]
    \ii
    \(\preccurlyeq\) is reflexive since, for all \(S \in \mrm{Pt}(A)\) and \(C \in S\),
    \(C \subseteq C\), i.e., \(S \preccurlyeq S\). \checkmark

    Assume \(S_1 \preccurlyeq S_2\) and \(S_2 \preccurlyeq S_1\).
    Take any \(C \in S_1\).
    Then, there exists \(D \in S_2\) such that \(C \subseteq D\).
    In addition, there exists \(E \in S_1\) such that \(D \subseteq E\).
    We have \(C \subseteq E\) but \(C\) is nonempty as \(S_1\) is a partition,
    which implies \(C \cap E \neq \OO\).
    Therefore, as \(S_1\) is a partition, we must have \(C = E\) and thus \(C = D\).
    Hence, \(S_1 \subseteq S_2\). This shows that \(\preccurlyeq\) is antisymmetric. \checkmark

    Assume \(S_1 \preccurlyeq S_2\) and \(S_2 \preccurlyeq S_3\).
    Take any \(C \in S_1\).
    There exists \(D \in S_2\) such that \(C \subseteq D\).
    There exists \(E \in S_3\) such that \(D \subseteq E\).
    Hence, \(C \subseteq E\); \(S_1 \preceq S_3\).
    This shows that \(\preccurlyeq\) is transitive. \checkmark

    \ii
    Define a relation \(E\) in \(A\) by
    \(E \triangleq \{\,(a, b) \in A^2 \mid \fall S \in T,\: \exs C \in S,\: a \in C \land b \in C\,\}\).
    It can be easily shown that \(E\) is an equivalence
    mimicking the proof of \Cref{th:partitionDerivesEquiv}.
    Then, \(A/E \in \mrm{Pt}(A)\) by \Cref{th:equivDerivesPartition}.

    \clm[AEisLowerBound]{
        \(A/E\) is a lower bound of \(T\).
    }{
        If \(T = \OO\), there is nothing to prove; so assume \(T \neq \OO\).
        Take any \(S \in T\) and \(a \in A\).
        Then, there exists \(C \in S\) such that \(a \in S\) since \(S\) is a partition of \(A\).
        Let \(b \in [a]_E\). Then, there exists \(D \in S\) such that
        \(a, b \in D\), which implies \(C = D\). Therefore, \([a]_E \subseteq C\).
        Hence, \(A/E \preccurlyeq S\). \qed
    }

    \clm[AEisMaxLowerBound]{
        For each lower bound \(L\) of \(T\), \(L \preccurlyeq A/E\).
    }{
        If \(T = \OO\), then \(A/E = \{A^2\}\) and every partition of \(A\) is a lower bound.
        Since \(S \preccurlyeq \{A^2\}\) for all \(S \in \mrm{Pt}(A)\), the result follows.

        Now, assume \(T \neq \OO\).
        Let \(L\) be a lower bound of \(T\).
        Take any \(D \in L\). Fix some \(a \in D\).
        Then, each \(d \in D\) has the property that
        \(\fall S \in T,\: \exs C \in S,\: \{a,d\} \subseteq D \subseteq C\)
        as \(L\) is a lower bound of \(T\).
        Therefore, \(d \in [a]_E\); \(D \subseteq [a]_E\).
        Hence, \(L \preccurlyeq A/E\). \qed
    }
    \Cref{clm:AEisLowerBound,clm:AEisMaxLowerBound} say that \(\inf T = A/E\).
    Hence, \(\inf T\) exists.

    \ii
    Let \(T' \triangleq \{\,S' \in \mrm{Pt}(A) \mid \fall S \in T,\: S \preccurlyeq S'\,\}\).
    By (ii), \(S^\ast \triangleq \inf T'\) exists.
    \clm[SastIsUpperBound]{
        \(S^\ast\) is an upper bound of \(T\).
    }{
        In (ii), we showed that \(S^\ast = A/E\)
        where \(E = \{\,(a, b) \in A^2 \mid \fall S' \in T',\: \exs C' \in S',\: a \in C' \land b \in C'\,\}\).
        Take any \(S \in T\) and let \(C \in S\).
        Fix some \(c_0 \in C\).

        Now, take arbitrary \(c \in C\). Then, for all \(S' \in T'\),
        since \(S \preccurlyeq S'\), there exists \(D' \in S'\)
        such that \(c \in C \subseteq D'\).
        Hence, we have \(cEc_0\); \(C \subseteq [c_0]_E\).
        Therefore, \(S \preccurlyeq S^\ast\). \qed
    }
    \Cref{clm:SastIsUpperBound} essentially says that \(S^\ast \in T'\).
    By \Cref{th:basicInfimum} (iii), \(S^\ast = \min T'\), i.e., \(S^\ast = \sup T\).
    \qed
\end{enumerate}
}

\setexernumber{12}

\exer[2.5.13]{}{
    If \(h\) is isomorphism between \((P, \le)\) and \((Q, \preceq)\),
    then \(h\inv\) is an isomorphism between \((Q, \preceq)\) and \((P, \le)\).
}
\pf{Proof}{
    Take any \(q_1, q_2 \in Q\).
    Then, we have \(q_1 \preceq q_2 \iff h(h\inv(q_1)) \preceq h(h\inv(q_2)) \iff h\inv(q_1) \le h\inv(q_2)\).
}

\exer[2.5.14]{}{
    If \(f\) is an isomorphism between \((P_1, \le_1)\) and \((P_2, \le_2)\),
    and if \(g\) is an isomorphism between \((P_2, \le_2)\) and \(P_3, \le_3\),
    then \(g \circ f\) is an isomorphism between \((P_1, \le_1)\) and \((P_3, \le_3)\).
}
\pf{Proof}{
    \(\ran (g \circ f) = g[\ran f] = P_3\).
    Moreover, \(g \circ f\) is one-to-one.
    Hence, \(g \circ f \colon P_1 \hooktwoheadrightarrow P_3\).
    For all \(p, q \in P_1\), we have
    \(p \le_1 q \iff f(p) \le_2 f(q) \iff g(f(p)) \le_3 \iff g(f(q))\).
    Hence, \(g \circ f\) is an isomorphism between \((P_1, \le_1)\) and \((P_3, \le_3)\).
}

\end{document}
