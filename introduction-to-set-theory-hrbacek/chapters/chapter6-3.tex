\documentclass[../introduction_to_set_theory_Note.tex]{subfiles}

\begin{document}

\section{The Axiom Schema of Replacement}

We lack some tools to guarantee the existence of some sets
that look harmless. For example, you might want to construct a sequence
\[
    \lang\,\OO, \{\OO\}, \{\{\OO\}\}, \cdots\,\rang
\]
with the help of \nameref{th:recursion},
but it requires an existence of set such to which
every element of the sequence belongs.
However, we do not have enough axioms to prove the existence.
To get through this difficulty, \nameref{ax:replacement} comes to rescue.


\axiom[replacement]{The Axiom Schema of Replacement}{
    Let \(\mbf{P}(x, y)\) be a property such that \(\fall x\: \exs! y\: \mbf{P}(x, y)\).
    For every set \(A\), there exists a set \(B\) such that,
    for every \(x \in A\), there exists \(y \in B\) for which \(\mbf{P}(x, y)\) holds.
    \[
        \fall x \exs! y \mbf{P}(x, y) \implies
        \fall A \exs B \fall x \bigl[ x \in A \implies \exs y [y \in B \land \mbf{P}(x, y)] \bigr]
    \]
}

\nt{
    Let \(\mbf{P}(x, y)\) be a property such that \(\fall x\: \exs! y\: \mbf{P}(x, y)\)
    and let \(A\) be a set.
    \nameref{ax:replacement} guarantees the existence of the set
    \[
        \{\,y \mid \exs x \in A,\:\mbf{P}(x, y)\,\}.
    \]
}

\thm[counting]{The Counting Theorem}{
    For every well-ordered set \((W, \preceq)\),
    there uniquely exists an ordinal number which is isomorphic to \(W\).
}
\pf{Proof}{
    Let \(\alpha\) and \(\beta\) be two distinct ordinal numbers with \(\alpha < \beta\).
    Then, \(\alpha = \{\,\gamma \in \beta \mid \gamma < \alpha\,\} = \beta[\alpha]\)
    (\Cref{cor:ordinalIsSetOfSmallerOrdinals,lem:elementOfOrdinalIsOrdinal})
    is an initial segment of \(\beta\);
    thus \(\alpha\) and \(\beta\) are not isomorphic by \ref{itm:noIsoToInitSeg.i}.
    Hence, the uniqueness follows.

    Let \(A \triangleq \{\,a \in W \mid W[a] \text{ is isomorphic to some ordinal number}\,\}\).
    By the same reason as above,
    for each \(a \in A\), there uniquely exists an ordinal number \(\alpha_a\) which
    is isomorphic to \(W[a]\).
    Let \(\mbf{P}(x, y)\) be the property
    \[
        (x \in A \text{ and } y \text{ is an ordinal isomorphic to } W[x])
        \text{ or } (x \notin A \text{ and } y = \OO).
    \]
    Hence, by \nameref{ax:replacement},
    there exists the set \(S \triangleq \{\,z \mid \exs a \in A,\: z = \alpha_a\,\}\).

    \clm[UwimBbPO]{
        \(S\) is an ordinal number.
    }{
        \(S\) is well-ordered by \(\in\) by \ref{itm:basicOrdinal.iv}.
        Let \(\gamma \in \alpha_a \in S\).
        Let \(\varphi \colon \alpha_a \hooktwoheadrightarrow W[a]\)
        be the isomorphism between \(\alpha_a\) and \(W[a]\) and let \(c \triangleq \varphi(\gamma)\).
        Then, \(\restr{\varphi}{\gamma}\) is an isomorphism between \(\gamma\) and \(W[c]\);
        thus \(\gamma \in S\). Hence, \(S\) is an ordinal.
    }

    \clm[fKcQIApV]{
        Let \(a \in A\), \(b \in W\), and \(b \prec a\).
        Then, \(b \in A\).
        Moreover, \(\alpha_b < \alpha_a\).
    }{
        Let \(\varphi \colon W[a] \hooktwoheadrightarrow \alpha_a\)
        be the isomorphism between \(W[a]\) and \(\alpha_a\).
        Then, \(\restr{\varphi}{W[b]}\) is an isomorphism between
        \(W[b]\) and \(\varphi[W[b]]\). As \(W[b]\) is an initial segment of \(W[a]\),
        \(\varphi[W[b]]\) is an initial segment of \(\alpha_a\).
        Hence, \(\varphi[W[b]]\) is an ordinal.
        By \Cref{lem:psbsOfOrdinalIsElement}, \(\varphi[W[b]] = \beta\) for some \(\beta < \alpha_a\).
        Hence, \(b \in A\) and \(\alpha_b = \beta\). \qed
    }

    Now, define \(f \colon A \to S\) by \(a \mapsto \alpha_a\).
    By \Cref{clm:UwimBbPO,clm:fKcQIApV}, \(f\) is an isomorphism between \(A\) and \(S\).
    If \(\exs c \in W,\: A = W[c]\), then \(c \in A\) by definition.
    Hence, \(A\) is not an initial segment of \(W\) in spite of \Cref{clm:fKcQIApV};
    \(A = W\). (See \Cref{dfn:initialSegment}.)
    Hence, \(f\) is an isomorphism between \(A\) and an ordinal \(S\).
}

\dfn[orderType]{Order Type}{
    If \(W\) is a well-ordered set, then the \textit{order type} of \(W\)
    is the unique ordinal number isomorphic to \(W\).
    This is justified by \nameref{th:counting}.
}

\thm[operationRecursion]{The Recursion Theorem}{
    Let \(\mbf{P}(x, y)\) be a property such that \(\fall x\: \exs! y\: \mbf{P}(x, y)\).
    For any \(a\), there exists a unique infinite sequence \(\lang\,a_n \mid n \in \NN\,\rang\) such that
    \begin{enumerate}[nolistsep, label=(\roman*), ref=\protect{(\roman*)}, listparindent=\parindent]
        \ii \(a_0 = a\) and
        \ii \(\fall n \in \NN,\: \mbf{P}((a_n, n), a_{n+1})\).
    \end{enumerate}
}

\noindent
The proof of \nameref{th:operationRecursion} is given in the next section.

\subfile{../exercises/exercise6-3.tex}

\end{document}
