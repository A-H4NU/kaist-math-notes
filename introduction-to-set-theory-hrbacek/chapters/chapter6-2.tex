\documentclass[../introduction_to_set_theory_Note.tex]{subfiles}

\begin{document}

\section{Ordinal Numbers}

\dfn[transitiveSet]{Transitive Set}{
    A set \(T\) is \textit{transitive} if every element of \(T\) is a subset of \(T\),
    i.e,
    \[
        \fall u\:\fall v\: (u \in v \land v \in T \implies u \in T).
    \]
}

\dfn[ordinal]{Ordinal Number}{
    A set \(\alpha\) is an \textit{ordinal number} (or \textit{ordinal}) if
    \begin{enumerate}[nolistsep, label=(\roman*), ref=\protect{(\roman*)}, listparindent=\parindent]
        \ii \(\alpha\) is transitive
        \ii \(\alpha\) is well-ordered by \(\mathord{\in}_\alpha = \{\,(x, y) \in \alpha^2 \mid x \in y\,\}\).
    \end{enumerate}
}

\nt{
    \noindent
    Every natural number is an ordinal. (\nameref{th:NisWellOrdered})
}

\dfn[]{}{
    \(\omega \triangleq \NN\). (\Cref{not:Qtmidocs})
}

\mlemma[succOfOrdinalIsOrdinal]{}{
    If \(\alpha\) is an ordinal number, then \(S(\alpha)\) is also an ordinal number.
}
\pf{Proof}{
    As \(\alpha\) is transitive, \(S(\alpha) = \alpha \cup \{\alpha\}\) is transitive.
    Moreover, \(S(\alpha)\) is well-ordered by \(\in_{S(\alpha)}\)
    as \((\alpha, \in_{\alpha})\) is well-ordered.
}

\notat[]{}{
    We denote the successor of \(\alpha\) by \(\alpha + 1\).
}

\dfn[]{Successor Ordinal and Limit Ordinal}{
    Let \(\alpha\) be an ordinal number.
    \begin{itemize}[nolistsep, leftmargin=*, listparindent=\parindent]
        \ii
        \(\alpha\) is called a \textit{successor ordinal}
        if \(\alpha = \beta + 1\) for some ordinal \(\beta\).
        \ii
        Otherwise, \(\alpha\) is called a \textit{limit ordinal}.
    \end{itemize}
}

\dfn[]{}{
    For all ordinals \(\alpha\) and \(\beta\),
    we define \(\alpha < \beta\) if and only if \(\alpha \in \beta\).
}

\mlemma[noOrdinalContainsItself]{}{
    If \(\alpha\) is an ordinal number, then \(\alpha \notin \alpha\).
}
\pf{Proof}{
    If \(\alpha \in \alpha\), then it will
    contradict the asymmetry of \(<_{\alpha}\).
}

\mlemma[elementOfOrdinalIsOrdinal]{}{
    Let \(\alpha\) be an ordinal number.
    Then, every \(x \in \alpha\) is an ordinal number.
}
\mclm{Proof}{
    Take any \(x \in \alpha\).
    \begin{itemize}[nolistsep, leftmargin=*, listparindent=\parindent]
        \ii
        To show \(x\) is transitive, let \(u \in v \in x\).
        As \(\alpha\) is transitive, we have \(v \in \alpha\),
        and therefore \(u \in \alpha\) by the same reason.
        Then, as \(\in_{\alpha}\) is transitive in \(\alpha\),
        \(u \in x\). \checkmark

        \ii
        As \(\alpha\) is transitive, \(x \subseteq \alpha\).
        Therefore, \(\mathord{\in}_x = \mathord{\in}_\alpha \cap x^2\).
        As \(\in_{\alpha}\) is a well-ordering, so is \(\in_x\). \checkmark \qed
    \end{itemize}
}

\cor[ordinalIsSetOfSmallerOrdinals]{}{
    Let \(\alpha\) be an ordinal number.
    Then, \(\alpha = \{\,\beta \mid \beta \text{ is an ordinal number with } \beta < \alpha\,\}\).
}

\mlemma[psbsOfOrdinalIsElement]{}{
    Let \(\alpha\) and \(\beta\) be ordinal numbers.
    Then \(\alpha \subsetneq \beta \iff \alpha \in \beta\).
}
\mclm{Proof}{\hfill
\begin{itemize}[nolistsep, wide=0pt, widest={(\(\Rightarrow\))}, leftmargin=*, listparindent=\parindent]
    \ii[(\(\Rightarrow\))]
    \(\beta \setminus \alpha\) is nonempty,
    and thus there exists \(\gamma \triangleq \min_{\in_{\beta}} (\beta \setminus \alpha)\).
    Note that, any \(\delta \in \gamma \setminus \alpha\)
    would be an element of \(\beta \setminus \alpha\) less than \(\gamma\),
    which contradicts the minimality of \(\gamma\).
    Hence, \(\gamma \setminus \alpha = \OO\), i.e., \(\gamma \subseteq \alpha\).

    Now we claim that \(\alpha \subseteq \gamma\) and thus \(\alpha = \gamma \in \beta\).
    Take any \(\delta \in \alpha\) and suppose \(\delta \notin \gamma\) for the sake of contradiction.
    As \(\in_{\beta}\) is a total ordering, it is either \(\gamma \in \delta\) or \(\gamma = \delta\).
    In both cases, as \(\in_{\beta}\) is transitive, \(\gamma \in \alpha\),
    which contradicts \(\gamma \in \beta \setminus \alpha\).

    \ii[(\(\Leftarrow\))]
    Take any \(\gamma \in \alpha\).
    Then, as \(\beta\) is transitive, we have \(\gamma \in \beta\) as well.
    Hence, \(\alpha \subseteq \beta\).
    Since \(\alpha \neq \beta\) by \Cref{lem:noOrdinalContainsItself},
    we conclude \(\alpha \subsetneq \beta\).
    \qed
\end{itemize}
}

\thm[basicOrdinal]{}{
Let \(\alpha\), \(\beta\), and \(\gamma\) be ordinal numbers.
\begin{enumerate}[nolistsep, label=(\roman*), ref=\protect{\Cref{th:basicOrdinal} (\roman*)}, listparindent=\parindent]
    \ii
    \(\alpha < \beta \land \beta < \gamma \implies \alpha < \gamma\)
    \ii
    \(\lnot(\alpha < \beta \land \beta < \alpha)\)
    \ii\label{itm:basicOrdinal.iii}
    \((\alpha < \beta) \lor (\alpha = \beta) \lor (\beta < \alpha)\)
    \ii\label{itm:basicOrdinal.iv}
    Every nonempty set of ordinal numbers has a \(\le\)-least element.
    \ii\label{itm:basicOrdinal.v}
    For every set \(X\) of ordinal numbers,
    there exists an ordinal number \(\alpha'\) such that \(\alpha' \notin X\).
\end{enumerate}
}
\mclm{Proof}{\hfill
\begin{enumerate}[nolistsep, label=(\roman*), leftmargin=*, listparindent=\parindent]
    \ii
    The result follows as \(\gamma\) is transitive.

    \ii
    If \(\alpha \in \beta \in \alpha\),
    as \(\alpha\) is transitive, we have \(\alpha \in \alpha\),
    which contradicts \Cref{lem:noOrdinalContainsItself}.

    \ii
    Let \(\gamma \triangleq \alpha \cap \beta\).
    If \(\alpha = \beta\), then \(\gamma = \alpha\) is an ordinal.
    If \(\alpha \neq \beta\), then we have either \(\gamma \subsetneq \alpha\)
    or \(\gamma \subseteq \beta\); hence
    \(\gamma\) is an ordinal by \Cref{lem:psbsOfOrdinalIsElement,lem:elementOfOrdinalIsOrdinal}.

    We have three cases: \(\gamma = \alpha\), \(\gamma = \beta\), or \(\gamma \subsetneq \alpha \land \gamma \subsetneq \beta\).
    The last case is not possible since we have
    \(\gamma \in \alpha\) and \(\gamma \in \beta\) by \Cref{lem:psbsOfOrdinalIsElement},
    which implies \(\gamma \in \alpha \cap \beta = \gamma\) that contradicts
    \Cref{lem:noOrdinalContainsItself}.
    The first two cases lead to one of \(\alpha = \beta\), \(\alpha \in \beta\), and \(\beta \in \alpha\)
    by \Cref{lem:psbsOfOrdinalIsElement}.

    \ii
    Let \(A\) be a nonempty set of ordinals.
    Fix any \(\alpha \in A\).
    If \(\alpha \cap A = \OO\), then by (iii), \(\alpha = \min_{\le} A\).
    If \(\alpha \cap A \neq \OO\),
    then \(\alpha \cap A \subseteq \alpha\) has \(\beta \triangleq \min_{\in_{\alpha}} A\),
    which is a \(\le\)-minimal element of \(A\).

    \ii
    Let \(X\) be a set of ordinal numbers.
    Then, \(\bigcup X\) is transitive by \Cref{exer:6.2.5}.
    Moreover, (iv) implies that \(\in\) well-orders \(\bigcup X\).
    Hence, \(\bigcup X\) is an ordinal.

    Let \(\alpha \triangleq S\left(\bigcup X\right)\).
    \(\alpha\) is an ordinal by \Cref{lem:succOfOrdinalIsOrdinal}.
    Suppose \(\alpha \in X\) for the sake of contradiction.
    Then, \(\alpha \subseteq \bigcup X\);
    which implies \(\alpha = \bigcup X\) or \(\alpha \in \bigcup X\)
    by \Cref{lem:psbsOfOrdinalIsElement}.
    In both cases, \(\alpha \in S \left(\bigcup X\right) = \alpha\),
    which contradicts \Cref{lem:noOrdinalContainsItself}.
    \qed
\end{enumerate}
}

\nt{
    \noindent
    \ref{itm:basicOrdinal.v} insists that there is no set of all ordinals.
}

\dfn[supOrdinal]{Supremum of a Set of Ordinals}{
    If \(X\) is a set of ordinal numbers,
    \(\bigcup X\) is the \textit{supremum} of \(X\) and is denoted \(\sup X\).
}

\nt{
    \noindent
    \Cref{dfn:supOrdinal} is justified the fact that:
    \begin{enumerate}[nolistsep, label=(\roman*), ref=\protect{(\roman*)}, listparindent=\parindent]
        \ii
        If \(\alpha \in X\), then \(\alpha \in \bigcup X\),
        and thus \(\alpha \subseteq \bigcup X\), i.e., \(\alpha \le \bigcup X\), as \(\bigcup X\) is an ordinal.

        \ii
        If \(\fall \alpha \in X,\: \alpha \le \gamma\), i.e.,
        \(\fall \alpha \in X,\: \alpha \subseteq \gamma\),
        then \(\bigcup X \subseteq \gamma\),
        i.e, \(\bigcup X \le \gamma\).
    \end{enumerate}
}

\nt{
    Let \(\alpha\) be an ordinal.
    If \(x \in \bigcup \alpha\), then \(x \in y \in \alpha\) for some \(y\),
    and thus \(x \in \alpha\) as \(\alpha\) is transitive.
    Hence, \(\bigcup \alpha \subseteq \alpha\).
}

\thm[finiteOrdinalIsNat]{}{
    An ordinal \(\alpha\) is finite if and only if \(\alpha \in \NN\).
}
\mclm{Proof}{\hfill
\begin{itemize}[nolistsep, wide=0pt, widest={(\(\Rightarrow\))}, leftmargin=*, listparindent=\parindent]
    \ii[(\(\Rightarrow\))]
    Let \(\alpha\) be an ordinal such that \(\alpha \notin \NN\).
    Then, by \ref{itm:basicOrdinal.iii}, \(\omega \subseteq \alpha\);
    thus \(\alpha\) is not finite.
    \qed
\end{itemize}
}

\subfile{../exercises/exercise6-2.tex}

\end{document}
