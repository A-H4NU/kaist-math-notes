\documentclass[../introduction_to_set_theory.tex]{subfiles}

\begin{document}

\section{Introduction to Sets}

\dfn[set]{Set}{
    Every object in the universe of discourse is called a \textit{set}.
}

\section{Properties}

\dfn[property]{Property}{
    Any mathematical sentence\footnote{Refer to mathematical logic textbook for detailed discussion.}
    is called a \textit{property}.
    If \(X, Y, \cdots, Z\) are free variables of a property \(\mbf{Q}\),
    we write \(\mbf{Q}(X, Y, \cdots, Z)\) and say
    \(\mbf{Q}(X, Y, \cdots, Z)\) is a property of \(X, Y, \cdots, Z\).
}

\section{Axioms}

\axiom[existence]{The Axiom of Existence}{
    \noindent
    There exists a set which has no elements.
    \[
        \exs A\:\fall x\: \lnot (x \in A)
    \]
}

\nt{
    \noindent
    \nameref{ax:existence} guarantees that the universe of discourse is not void.
}

\axiom[extensionality]{The Axiom of Extensionality}{
    \noindent
    If every element of \(X\) is an element of \(Y\) and every element of \(Y\)
    is an element of \(X\), then \(X = Y\).
    \[
        \fall X\: \fall Y\: [\fall x\: (x \in X \iff x \in Y) \implies X = Y]
    \]
}

\nt{
    \noindent
    \nameref{ax:extensionality} defines the equality relation with the containment relation(\(\in\)).
}

\mlemma[uniqueEmptySet]{}{
    There exists only one set with no elements.
}
\pf{Proof}{
    Let \(A\) and \(B\) are sets such that \(\fall x\: \lnot(x \in A)\) and \(\fall x\: \lnot(x \in B)\).
    Then, we have \(\fall x\: (x \in A \iff x \in B)\).
    Therefore, by \nameref{ax:extensionality}, \(A = B\) is guaranteed.
}

\dfn[emptySet]{Empty Set}{
    The unique set with no elements is called the \textit{empty set} and is denoted \(\OO\).
}

\nt{
    \noindent
    \Cref{dfn:emptySet} is justified by \Cref{lem:uniqueEmptySet}.
}

\axiom[comprehension]{The Axiom Schema of Comprehension}{
    Let \(\mbf{P}(x)\) be a property of \(x\).
    For any set \(A\), there exists a set \(B\) such that \(x \in B\) if and only if \(x \in A\) and \(\mbf{P}(x)\).
    \[
        \fall A\: \exs B\: (x \in B \iff x \in A \land \mbf{P}(x))
    \]
}
\nt{
    \noindent
    \Cref{ax:comprehension} is a axiom \textit{schema} since
    it provides unlimited amount of axioms for varying \(\mbf{P}\).
}

\mlemma[uniqueComprehension]{}{
    Let \(\mbf{P}(x)\) be a property of \(x\).
    For any set \(A\), there uniquely exists a set \(B\) such that \(x \in B\) if and only if \(x \in A\) and \(\mbf{P}(x)\).
}
\pf{Proof}{
    Let \(B'\) be another set such that \(x \in B'\) if and only if \(x \in A\) and \(\mbf{P}(x)\).
    Then, for any \(x\), we have \(x \in B' \iff x \in A \land \mbf{P}(x) \iff x \in B\).
    Hence, by \nameref{ax:extensionality}, we have \(B = B'\).
}

\notat[setBuilder]{Set-Builder Notation}{
    Let \(\mbf{P}(x)\) be a property of \(x\). Let \(A\) be a set.
    The unique set \(B\) such that \(x \in B\) if and only if \(x \in A\) and \(\mbf{P}(x)\)
    is denoted \(\{\,x \in A \mid \mbf{P}(x)\,\}\).
}
\nt{
    \noindent
    \Cref{not:setBuilder} is justified by \Cref{lem:uniqueComprehension}.
}

\axiom[pair]{The Axiom of Pair}{
    \noindent
    For any \(A\) and \(B\), there exists \(C\) such that
    \(x \in C\) if and only if \(x = A\) or \(x = B\).
    \[
        \fall A\: \fall B\: \exs C\: (x \in C \iff x = A \lor x = B)
    \]
}
\nt{
    \noindent
    Similarly, the set \(C\) such that \(x \in C \iff x = A \lor x = B\) is unique
    by \nameref{ax:extensionality}.
}

\notat[pair]{}{
    Let \(A\) and \(B\) be sets.
    The unique set \(C\) such that \(x \in C\) if and only if \(x = A\) or \(x = B\)
    is denoted \(\{A, B\}\).
    In particular, if \(A = B\), we write \(\{A\}\) instead of \(\{A, A\}\).
}

\axiom[union]{The Axiom of Union}{
    \noindent
    For any \(S\), there exists \(U\) such that \(x \in U\)
    if and only if \(x \in A\) for some \(A \in S\).
    \[
        \fall S\: \exs U\: (x \in U \iff \exs A\: x \in A \land A \in S)
    \]
}
\dfn[union]{The Union of System of Sets}{
    Let \(S\) be a set.
    The unique set \(U\) such that \(x \in U\) if and only if
    \(x \in A\) for some \(A \in S\)
    is denoted \(\bigcup S\).
}
\dfn[unionTwoSets]{The Union of Two Sets}{
    Let \(A\) and \(B\) be sets.
    Then, \(A \cup B\) denotes the unique set \(\bigcup \{A, B\}\).
}

\dfn[subset]{Subset}{
    Let \(A\) and \(B\) sets.
    \(B\) is said to be a \textit{subset} of \(A\)
    if \(\fall x\: (x \in B \implies x \in A)\).
    If \(B\) is a subset of \(A\), then we write \(B \subseteq A\).
}

\axiom[powerSet]{The Axiom of Power Set}{
    \noindent
    For any \(S\), there exists \(P\) such that
    \(X \in P\) if and only if \(X \subseteq S\).
}

\nt{
    \noindent
    Similarly, the set \(P\) is unique by \nameref{ax:extensionality}.
}

\dfn[powerSet]{Power Set}{
    Let \(S\) be a set.
    The unique set \(P\) such that \(X \in P\) if and only if \(X \subseteq S\)
    is called the \textit{power set} of \(S\) and is denoted \(\mcal P(S)\).
}

\mlemma[setDefinedByProperty]{}{
    Let \(\mbf{P}(x)\) be a property of \(x\).
    Let \(A\) and \(A'\) be sets such that
    \(\mbf{P}(x) \implies x \in A \land x \in A'\).
    Then, \(\{\,x \in A \mid \mbf{P}(x)\,\} = \{\,x \in A' \mid \mbf{P}(x)\,\}\).
}
\pf{Proof}{
    For all \(x\), we have \(x \in A \land \mbf{P}(x) \iff \mbf{P}(x) \iff x \in A' \land \mbf{P}(x)\).
    Therefore, by \nameref{ax:extensionality}, the result follows.
}

\notat[setDefinedByProperty]{}{
    Let \(\mbf{P}(x)\) be a property of \(x\).
    If there exists a set \(A\) such that \(\mbf{P}(x)\) implies \(x \in A\),
    we write \(\{\,x \mid \mbf{P}(x)\,\} \triangleq \{\,x \in A \mid \mbf{P}(x)\,\}\),
    and it is called \textit{the set of all \(x\) with the property \(\mbf{P}(x)\)}.
}
\nt{
    \noindent
    \Cref{not:setDefinedByProperty} is justified by \Cref{lem:setDefinedByProperty}.
}

\subsection*{Selected Problems}

\exer[1.3.1]{}{
    The set of all \(x\) such that \(x \in A\) and \(x \notin B\) exists.
}
\pf{Proof}{
    We have \(x \in A \land x \notin B \implies x \in A\).
    Hence, the set exists and is equal to \(\{\,x \in A \mid x \in A \land x \notin B\,\}\).
}

\exer[1.3.2]{}{
    Prove \nameref{ax:existence} only from \nameref{ax:comprehension}
    and The Weak Axiom of Existence. \\[.3em]
    \textsf{Weak Axiom of Existence}\; Some set exists.
}
\pf{Proof}{
    Let \(A\) be a set known to exist.
    Then, there exists \(B = \{\,x \in A \mid x \neq x\,\}\) by \nameref{ax:comprehension}.
    Since \(\fall x\: (x = x)\), \(\fall x\: (x \notin B)\).
}

\exer[1.3.3]{}{
    \begin{enumerate}[nolistsep, label=(\alph*)]
        \ii Prove that a set of all sets(\(\{\,x \mid \top\,\}\)) does not exist.
        \ii Prove that \(\fall A\: \exs x\: (x \notin A)\).
    \end{enumerate}
}
\mclm{Proof}{
\hfill
\begin{enumerate}[nolistsep, label=(\alph*)]
    \ii
    Suppose \(V = \{\,x \mid \top\,\}\) exists.
    Then, by \nameref{ax:comprehension}, \(R = \{\,x \in V \mid x \notin x\,\}\) exists.
    However, we have \(R \in R \iff R \notin R\) by definition of \(R\).
    Hence, \(V\) does not exist.

    \ii
    Suppose \(\exs A\: \fall x\: (x \in A)\) for the sake of contradiction.
    Then, \(A\) is the set of all sets, which is impossible by (a). \qed
\end{enumerate}
}

\setexernumber{5}

\exer[1.3.6]{}{
    Prove \(\fall X\: \lnot (\mcal P(X) \subseteq X)\).
}
\pf{Proof}{
    Let \(Y = \{\,u \in X \mid u \notin u\,\}\).
    Then, by definition, \(Y \subseteq X\), and thus \(Y \in \mcal P(X)\).
    Now, suppose \(Y \in X\) for the sake of contradiction.
    Then, \(Y \in Y \iff Y \in X \land Y \notin Y \iff Y \notin Y\),
    which is a contradiction. Hence, \(Y \notin X\).
}

\section{Elementary Operations on Sets}

\dfn[properSubset]{Proper Subset}{
    Let \(A\) and \(B\) sets.
    \(B\) is said to be a \textit{proper subset} of \(A\)
    if \(B \subseteq A\) and \(B \neq A\).
    If \(B\) is a proper subset of \(A\), we write \(B \subsetneq A\).
}

\dfn[]{Elementary Operations on Sets}{
    \begin{enumerate}[nolistsep, label=(\roman*)]
        \ii Intersection
        \begin{itemize}[nolistsep]
            \ii The \textit{intersection} of \(A\) and \(B\), \(A \cap B\),
                is the set \(\{\,x \mid x \in A \land x \in B\,\}\).
        \end{itemize}
        \ii Union
        \begin{itemize}[nolistsep]
            \ii The \textit{union} of \(A\) and \(B\), \(A \cup B\),
                is the set \(\{\,x \mid x \in A \lor x \in B\,\}\).
        \end{itemize}
        \ii Difference
        \begin{itemize}[nolistsep]
            \ii The \textit{difference} of \(A\) and \(B\), \(A \setminus B\),
                is the set \(\{\,x \mid x \in A \land x \notin B\,\}\).
        \end{itemize}
        \ii Symmetric Difference
        \begin{itemize}[nolistsep]
            \ii The \textit{symmetric difference} of \(A\) and \(B\), \(A \symdif B\),
                is the set \((A \setminus B) \cup (B \setminus A)\).
        \end{itemize}
    \end{enumerate}
}

\mlemma{Simple Properties of Elementary Operations}{
    \begin{enumerate}[nolistsep, label=(\roman*)]
        \ii Commutativity
        \begin{itemize}[nolistsep]
            \ii \(A \cap B = B \cap A\)
            \ii \(A \cup B = B \cup A\)
            \ii \(A \symdif B = B \symdif A\)
        \end{itemize}
        \ii Associativity
        \begin{itemize}[nolistsep]
            \ii \((A \cap B) \cap C = A \cap (B \cap C)\)
            \ii \((A \cup B) \cup C = A \cup (B \cup C)\)
            \ii \((A \symdif B) \symdif C = A \symdif (B \symdif C)\)
        \end{itemize}
        \ii Distributivity
        \begin{itemize}[nolistsep]
            \ii \(A \cap (B \cup C) = (A \cap B) \cup (A \cap C)\)
            \ii \(A \cup (B \cap C) = (A \cup B) \cap (A \cup C)\)
        \end{itemize}
        \ii De Morgan's Laws
        \begin{itemize}[nolistsep]
            \ii \(C \setminus (A \cap B) = (C \setminus A) \cup (C \setminus B)\)
            \ii \(C \setminus (A \cup B) = (C \setminus A) \cap (C \setminus B)\)
        \end{itemize}
        \ii Miscellaneous
        \begin{itemize}[nolistsep]
            \ii \(A \cap (B \setminus C) = (A \cap B) \setminus C\)
            \ii \(A \setminus B = \OO \iff A \subseteq B\)
            \ii \(A \symdif B = \OO \iff A = B\)
        \end{itemize}
    \end{enumerate}
}

\dfn[intersection]{Intersection of System of Sets}{
    Let \(S\) be a nonempty set. Then, the \textit{intersection} \(\bigcap S\)
    is the set \(\{\,x \mid \fall A \in S\: (x \in A)\,\}\).
}
\nt{
    Note that \(\bigcap S\) exists for all nonempty \(S\)
    since \(\fall A \in S\: (x \in A) \implies x \in A_1\)
    where \(A_1\) is any set such that \(A_1 \in S\).
}

\dfn[mutualDisjoint]{System of Mutually Disjoint Sets}{
    We say the sets \(A\) and \(B\) are \textit{disjoint} if \(A \cap B = \OO\).
    A set \(S\) is a \textit{system of mutually disjoint sets} if
    \(\fall A, B \in S,\: (A \neq B \implies A \cap B = \OO)\).
}

\subsection*{Selected Problems}

\setexernumber{1}

\exer[1.4.2]{}{
    \begin{enumerate}[nolistsep, label=(\roman*), leftmargin=*, listparindent=\parindent]
        \ii \(A \setminus B = (A \cup B) \setminus B = A \setminus (A \cap B)\)
        \ii \(A \setminus (B \setminus C) = (A \setminus B) \cup (A \cap C)\)
        \ii \(A \cap B = A \setminus (A \setminus B)\)
    \end{enumerate}
}
\mclm{Proof}{\hfill
\begin{enumerate}[nolistsep, label=(\roman*), ref=\protect{\Cref{exer:1.4.2} (\roman*)} , leftmargin=*, listparindent=\parindent]
    \ii\mbox{}\vspace*{-\baselineskip}
    \begin{flalign*}\SwapAboveDisplaySkip
        x \in A \land x \notin B
        &\iff x \in A \land x \notin B \lor x \in B \land x \notin B & \comment*{\(\lor\)-intro / \(\lor\)-syllogism} \\
        &\iff (x \in A \lor x \in B) \land x \notin B & \comment*{Distribution}
    \end{flalign*}
    \begin{flalign*}\SwapAboveDisplaySkip
        x \in A \land x \notin B
        &\iff x \in A \land x \notin A \lor x \in A \land x \notin B & \comment*{\(\lor\)-intro / \(\lor\)-syllogism} \\
        &\iff x \in A \land (x \notin A \lor x \notin B) & \comment*{Distribution} \\
        &\iff x \in A \land \lnot(x \in A \land x \in B) & \comment*{De Morgan's Law}
    \end{flalign*}

    \ii\mbox{}\vspace*{-\baselineskip}
    \begin{flalign*}\SwapAboveDisplaySkip
        x \in A \land \lnot (x \in B \land x \notin C)
        &\iff x \in A \land (x \notin B \lor x \in C) & \comment*{De Morgan's Law} \\
        &\iff (x \in A \land x \notin B) \lor (x \in A \land x \in C) & \comment*{Distribution}
    \end{flalign*}

    \ii\label{itm:1.4.2iii}
    By (ii), \(A \setminus (A \setminus B) = (A \setminus A) \cup (A \cap B) = A \cap B\).
    \qed
\end{enumerate}
}

\setexernumber{3}

\exer[1.4.4]{}{
    For any set \(A\), prove that a ``complement'' of \(A\) (\(\{\,x \mid x \notin A\,\}\)) does not exist.
}
\pf{Proof}{
    Let \(B\) be the complement of \(A\) for the sake of contradiction.
    Then, \(A \cup B\) is the set of all sets, which is impossible by \Cref{exer:1.3.3}.
}

\end{document}
