\documentclass[../introduction_to_set_theory_Note.tex]{subfiles}

\begin{document}

\section{Initial Ordinals}

\dfn[initialOrdinal]{Initial Ordinal}{
    An ordinal number \(\alpha\) is called an \textit{initial ordinal}
    if it is not equipotent to any \(\beta < \alpha\).
}

\exmp{}{
    \begin{itemize}[nolistsep, leftmargin=*, listparindent=\parindent]
        \ii Every natural number is an initial ordinal.
        \ii \(\omega\) is an initial ordinal. (\ref{itm:basicFinite.iii})
        \ii None of \(\omega + 1, \omega + 2, \cdots, \omega \cdot \omega, \cdots, \omega^{\omega}, \cdots\) is initial.
    \end{itemize}
}

\thm[wosetEquipotentToInitial]{}{
    Each well-orderable set \(X\) is equipotent to a unique initial ordinal number.
}
\pf{Proof}{
    By \nameref{th:counting}, there exists an ordinal number \(\alpha\) such that
    \(|X| = |\alpha|\).
    Hence, \(\alpha_0 \triangleq \min \{\,\alpha \in \Ord \mid |X| = |\alpha|\,\}\)
    exists by \nameref{ax:comprehension} and \ref{itm:basicOrdinal.iv}.
    Then, \(\alpha_0\) is a limit ordinal
    because \(\exs \beta < \alpha_0,\: |\alpha_0| = |\beta|\) would imply \(|\beta| = |X|\),
    which contradicts the minimality of \(\alpha_0\).

    If \(\alpha_1\) and \(\alpha_2\) are different initial ordinals,
    then we cannot have \(|\alpha_1| = |\alpha_2|\)
    since one of them is less than the other.
    Hence, the uniqueness is shown.
}

\dfn[]{Cardinality of Well-Orderable Sets}{
    If \(X\) is a well-orderable set,
    we define \(|X|\) to be the unique initial ordinal which is equipotent to \(\alpha\).
    This is justified by \Cref{th:wosetEquipotentToInitial}.
}

\mlemma[hartogsNumberExists]{}{
    Let \(A\) be any set.
    Then, there exists the least ordinal number \(\alpha\)
    such that \(|\alpha| \not\le |A|\).
}
\pf{Proof}{
    By \nameref{th:counting},
    for each well-ordered set \((W, R)\) where \(W \subseteq A\),
    there exists a unique ordinal \(\alpha\) such that \(\alpha\) is isomorphic to \((W, R)\).
    Hence, by \nameref{ax:replacement}, the set
    \[
        H \triangleq \{\,\alpha \in \Ord \mid
        \exs R \subseteq A \times A,\: (\field R, R) \cong \alpha\,\}
    \]
    exists.

    \clm[eYkMWAIP]{
        \(\fall \alpha \in \Ord,\: (|\alpha| \le |A| \iff \alpha \in H)\)
    }{
        \begin{itemize}[nolistsep, wide=0pt, widest={(\(\Rightarrow\))}, leftmargin=*, listparindent=\parindent]
            \ii[(\(\Rightarrow\))]
            It is direct from the definition.
            \ii[(\(\Leftarrow\))]
            Let \(f \colon \alpha \injto A\) and \(W \triangleq \ran f\).
            Define \(R \triangleq \{\,(f(\beta), f(\gamma)) \mid \beta < \gamma < \alpha\,\}\).
            Then, \((W, R)\) is a well-ordered set
            and \(f\) is an isomorphism between \(\alpha\) and \((W, R)\).
            \qed
        \end{itemize}
    }

    As \(H\) is a set of ordinal numbers, it is well-ordered by \(\in\).
    Moreover, if \(\alpha \in \beta \in H\),
    then \(\alpha \subsetneq \beta\) by \Cref{lem:psbsOfOrdinalIsElement};
    thus \(|\alpha| \le |\beta| \le |A|\) by \Cref{exer:4.1.3},
    which implies \(\alpha \in H\) by \Cref{clm:eYkMWAIP}.
    Hence, \(H \in \Ord\).

    We have \(H \notin H\) by \Cref{lem:noOrdinalContainsItself};
    hence \(|H| \not\le |A|\) by \Cref{clm:eYkMWAIP}.
    Moreover, every \(\alpha < H\) satisfies \(|\alpha| \le |A|\) by \Cref{clm:eYkMWAIP};
    \(H\) is the least ordinal we are looking for.
}

\dfn[hartogsNumber]{Hartogs Number}{
    For any \(A\), let \(h(A)\) denote the least ordinal \(\alpha\)
    such that \(|\alpha| \not\le |A|\).
    \(h(A)\) is called the \textit{Hartogs number} of \(A\).
    This is justified by \Cref{lem:hartogsNumberExists}.
}

\mlemma[ordinalHartogsGreater]{}{
    \(\fall \alpha \in \Ord, |\alpha| < |h(\alpha)|\)
}
\pf{Proof}{\hfill
    \begin{itemize}[nolistsep, leftmargin=*, listparindent=\parindent]
        \ii
        If \(\alpha = h(\alpha)\), it contradicts \(|h(\alpha)| \not\le |\alpha|\).
        \ii
        If \(\alpha > h(\alpha)\), then we have \(|h(\alpha)| \le |\alpha|\) by
        \Cref{exer:4.1.3,lem:psbsOfOrdinalIsElement}, which is a contradiction.
    \end{itemize}
    Hence, we must have \(\alpha < h(\alpha)\).
    Then, by \Cref{exer:4.1.3,lem:psbsOfOrdinalIsElement}, \(|\alpha| \le |h(\alpha)|\);
    but \(|\alpha| \neq |h(\alpha)|\). Therefore, the result follows.
}

\mlemma[hartogsNumberIsInitial]{}{
    For any \(A\), \(h(A)\) is an initial ordinal.
}
\pf{Proof}{
    Suppose that \(\exs \beta < h(A),\: |\beta| = |h(A)|\).
    Then, \(|\beta| \le |A|\) by definition of \(h(A)\)
    while we also have \(|\beta| = |h(A)|\),
    which contradicts \(|h(A)| \not\le |A|\).
}

\dfn[ordinalOmega]{}{
    \begin{alignat*}{2}\SwapAboveDisplaySkip
        \omega_0 &= \omega \\
        \omega_{\alpha + 1} &= h(\omega_\alpha) &\qquad& \text{for all ordinals}~\alpha \\
        \omega_\alpha &= \sup \{\,\omega_\beta \mid \beta < \alpha\,\} && \text{for all nonzero limit ordinals}~\alpha
    \end{alignat*}
}

\nt{
    Since \(|\omega_{\alpha+1}| \not\le |\omega_\alpha|\),
    we have \(|\omega_\alpha| < |\omega_{\alpha+1}|\)
    as one of two ordinals must have the other as its subset.
    Therefore, with \nameref{th:secondTransInduction},
    one might be able to prove \(\alpha < \beta \implies |\omega_\alpha| < |\omega_\beta|\).
}

\thm[omegaIffInfInitial]{}{
    \begin{enumerate}[nolistsep, label=(\roman*), ref=\protect{\Cref{th:omegaIffInfInitial} (\roman*)}]
        \ii
        For each \(\alpha \in \Ord\), \(\omega_{\alpha}\) is an infinite initial ordinal number.
        \ii
        If \(\Omega\) is an infinite initial ordinal number,
        then \(\exs \alpha \in \Ord,\: \Omega = \omega_{\alpha}\).
    \end{enumerate}
}
\mclm{Proof}{\hfill
\begin{enumerate}[nolistsep, label=(\roman*), leftmargin=*, listparindent=\parindent]
    \ii
    \(\omega_{\alpha}\) is infinite for all \(\alpha \in \Ord\) by the discussion above.
    If \(\alpha = 0\) or \(\alpha\) is a successor ordinal,
    then \(\omega_\alpha\) is initial ordinal by \Cref{lem:hartogsNumberIsInitial}.

    Take any nonzero limit ordinal \(\alpha\)
    and suppose \(\exs \gamma < \omega_{\alpha},\: |\gamma| = |\omega_{\alpha}|\)
    for the sake of contradiction.
    Then, there exists \(\beta < \alpha\) such that \(\gamma < \omega_{\beta}\),
    which implies \(|\omega_\alpha| = |\gamma| \le |\omega_\beta| < |\omega_\alpha|\).

    \ii
    We first prove the following claim.
    \clm[BRQUcvXT]{
        For each ordinal \(\alpha\) and infinite initial ordinal \(\Omega < \omega_\alpha\),
        \(\exs \gamma < \alpha,\: \Omega = \omega_\gamma\).
    }{
        We will conduct the transfinite induction on \(\alpha\).
        Let \(\mbf{P}(\alpha)\) be the property
        ``If \(\Omega < \omega_\alpha\) is an infinite initial ordinal,
        then \(\exs \gamma < \alpha,\: \Omega = \omega_\gamma\).''
        \(\mbf{P}(0)\) holds since \(\omega_0\) is the least infinite initial ordinal.

        Take any ordinal \(\alpha\) and assume \(\mbf{P}(\alpha)\).
        Take any infinite initial ordinal \(\Omega < \omega_{\alpha + 1} = h(\omega_\alpha)\).
        Then, by \Cref{dfn:hartogsNumber}, \(|\Omega| \le |\omega_{\alpha}|\).
        \(\omega_{\alpha} < \Omega\) is not an option since
        we have \(|\omega_{\alpha}| = |\Omega|\)
        by \Cref{exer:4.1.3}, \Cref{lem:psbsOfOrdinalIsElement}, and \Cref{th:cantor},
        which contradicts the fact that \(\Omega\) is an initial ordinal.
        If \(\Omega = \omega_{\alpha}\), then it is done.
        If \(\Omega < \omega_{\alpha}\), then by \(\mbf{P}(\alpha)\),
        there exists \(\gamma < \alpha < \alpha + 1\)
        such that \(\Omega = \omega_\gamma\).

        Now, take any nonzero limit ordinal \(\alpha\) and assume
        \(\mbf{P}(\alpha')\) holds for all \(\alpha' < \alpha\).
        Take any infinite initial ordinal \(\Omega < \omega_\alpha\).
        Then, by \Cref{dfn:ordinalOmega}, there exists \(\beta < \alpha\)
        such that \(\Omega < \omega_\beta\).
        Hence, by the induction hypothesis, \(\Omega = \omega_\gamma\)
        for some \(\gamma < \beta < \alpha\). \qed
    }

    One may readily show that \(\alpha \le \omega_\alpha\) for all \(\alpha \in \Ord\)
    with \nameref{th:secondTransInduction}.
    Take any infinite limit ordinal \(\Omega\).
    Then, \(\Omega \le \omega_{\Omega} < \omega_{\Omega + 1}\).
    By \Cref{clm:BRQUcvXT}, there exists an ordinal \(\gamma \le \Omega\)
    such that \(\Omega = \omega_{\gamma}\).
    \qed
\end{enumerate}
}

\nt{
\begin{itemize}[nolistsep, leftmargin=*, listparindent=\parindent]
    \ii
    Every infinite well-orderable set is equipotent to a unique infinite initial ordinal number.
    (\Cref{th:wosetEquipotentToInitial})

    \ii
    \(\alpha\) is an infinite initial ordinal if and only if \(\alpha = \omega_{\gamma}\) for some \(\gamma \in \Ord\).
    (\Cref{th:omegaIffInfInitial})
\end{itemize}
}

\dfn[alephs]{Alephs}{
    \(\aleph_{\alpha} = \omega_{\alpha}\) for each \(\alpha \in \Ord\).
    They represent cardinalities of well-orderable infinite sets.
}

\subfile{../exercises/exercise7-1.tex}

\end{document}
