\documentclass[../introduction_to_set_theory_Note.tex]{subfiles}

\begin{document}

\section{Complete Linear Ordering}

\dfn[cut]{Cut, Dedekind Cut, and Gap}{
    Let \((P, \le)\) be a totally ordered set.
    \begin{itemize}[nolistsep, leftmargin=*, listparindent=\parindent]
        \ii
        A \textit{cut} is a pair \((A, B)\) of sets such that
        \begin{enumerate}[nolistsep, label=(\roman*), ref=\protect{(\roman*)}, listparindent=\parindent]
            \ii \(\{A, B\}\) is a \hyperref[dfn:partition]{partition} of \(P\).
            \ii \(\fall a \in A,\: \fall b \in B,\: a < b\)
        \end{enumerate}

        \ii
        A \textit{Dedekind cut} is a cut \((A, B)\) such that \(\max A\) does not exist.

        \ii
        A \textit{gap} is a cut \((A, B)\) such that \(\max A\) and \(\min B\) do not exist.
    \end{itemize}
}

\dfn[bounded]{}{
    Let \((P, \le)\) be a totally ordered set and let \(\OO \subsetneq A \subseteq P\).
    \begin{itemize}[nolistsep, leftmargin=*, listparindent=\parindent]
        \ii \(A\) is \textit{bounded} if \(A\) has both lower and upper bounds.
        \ii \(A\) is \textit{bounded from below} if it has a lower bound.
        \ii \(A\) is \textit{bounded from above} if it has an upper bound.
    \end{itemize}
}

\mlemma[completeIff]{}{
    Let \((P, \le)\) be a totally ordered set.
    Every nonempty \(S \subseteq P\) bounded from above has a supremum
    if and only if \((P, \le)\) has no gap.
}
\mclm{Proof}{\hfill
\begin{itemize}[nolistsep, wide=0pt, widest={(\(\Rightarrow\))}, leftmargin=*, listparindent=\parindent]
    \ii[(\(\Rightarrow\))]
    Suppose \((A, B)\) is a gap of \((P, \le)\).
    Then, as any \(b \in B\) is a supremum of \(A\),
    \(A\) is bounded from above.
    Hence, by assumption, there exists \(\mu = \sup A\).
    There are two cases: \(\mu \in A\) and \(\mu \in B\).

    If \(\mu \in A\), then by \ref{itm:basicInfimum.iii},
    \(\mu = \max A\), which contradicts the fact that \((A, B)\) is a gap.
    If \(\mu \in B\), as any \(b \in B\) is an upper bound of \(A\),
    \(\fall b \in B,\: \mu \le b\), i.e., \(\mu = \min B\),
    which is a contradiction as well.

    \ii[(\(\Leftarrow\))]
    Suppose there is nonempty \(S \subseteq P\) such that \(S\) is bounded above and \(S\) has no supremum.
    Let
    \begin{align*}\SwapAboveDisplaySkip
        A &\triangleq \{\,x \in P \mid \exs s \in S,\: x \le s\,\}, \\
        B &\triangleq \{\,x \in P \mid \fall s \in S,\: x > s\,\}.
    \end{align*}

    \clm[FlmPHkut]{
        \(\{A, B\}\) is a partition of \(P\).
    }{
        We have \(A \cap B = \OO\) and \(A \cup B = P\).
        As \(A = P\) implies the existence of \(\max S\),
        which contradicts the nonexistence of \(\sup S\) by \Cref{itm:basicInfimum.ii},
        we have \(B \neq \OO\); moreover, as
        \(B = P\) is impossible by \(S \cap P = \OO\),
        we have \(A \neq \OO\). Hence, \(\{A, B\}\) is a partition of \(P\).
        \qed
    }

    \clm[lnpuetJv]{
        \(\fall a \in A,\: \fall b \in B,\: a < b\).
    }{
        Take any \(a \in A\) and \(b \in B\).
        Then, there exists \(s \in S\) such that \(a \le s\).
        As \(b \in B\), we have \(a \le s < b\).
        Hence, \(a < b\). \qed
    }

    \clm[DhzmLRJS]{
        \(\max A\) and \(\min B\) do not exist.
    }{
        Suppose \(m = \min B\) exists for the sake of contradiction.
        Let \(m'\) be an upper bound of \(S\).
        If \(m' \in S\), then \(m' = \sup S\), which is a contradiction.
        Hence, we have \(\fall s \in S,\: m' > s\), i.e., \(m' \in B\).
        This implies \(m \le m'\), that is to say \(m = \sup S\).
        Hence, \(\min B\) does not exist.

        Suppose \(M = \max A\) exists for the sake of contradiction.
        Then, as \(S \subseteq A\), \(M\) is an upper bound of \(S\).
        Let \(M'\) be another upper bound of \(S\).
        Then, as \(\fall s \in S,\: M \le s \le M'\), \(M = \sup S\),
        which is a contradiction. Hence, \(\max A\) does not exist.
        \qed
    }

    \noindent
    Combining \Cref{clm:FlmPHkut,clm:lnpuetJv,clm:DhzmLRJS} gives the result.
    \qed
\end{itemize}
}

\dfn[complete]{Complete Dense Totally Ordered Set}{
    Let \((P, \le)\) be a dense totally ordered set.
    \((P, \le)\) is said to be \textit{complete}
    if every nonempty \(S \subseteq P\) bounded from above has a supremum,
    i.e., if \((P, \le)\) has no gap. (See \Cref{lem:completeIff}.)
}

\thm[completion]{Completion}{
    Let \((P, \le)\) be a dense totally ordered set without endpoints.
    Then, there exists a complete totally ordered set \((C, \preceq)\) such that
    \begin{enumerate}[nolistsep, label=(\roman*), ref=\protect{(\roman*)}, listparindent=\parindent]
        \ii \(P \subseteq C\).
        \ii \(\fall p, q \in P,\: (p < q \iff p \prec q)\).
        \ii \(\fall c, d \in C,\: (c \prec d \implies \exs p \in P,\: c \prec p \prec d)\), i.e., \(P\) is dense in \(C\).
        \ii \(C\) does not have endpoints.
    \end{enumerate}
    Moreover, such \((C, \preceq)\)
    is unique up to isomorphism \(h\) with \(\mrm{Id}_P \subseteq h\).%
    \footnote[2]{
        In other words, if \((C, \preceq)\) and \((C^\ast, \preceq^\ast)\)
        satisfy all the requirements,
        then there exists an isomorphism \(h\) between
        \((C, \preceq)\) and \((C^\ast, \preceq^\ast)\)
        such that \(\fall p \in P,\: h(p) = p\).
    }
    The complete totally ordered set \((C, \preceq)\) is called the
    \textit{completion} of \((P, \le)\).
}
\pf{Proof}{
    For each \(B \subseteq P\), one may easily verify that \(\min B\) exists
    if and only if there exists \(p \in P\) such that
    \(B = \{\,x \in P \mid x \ge p\,\}\).
    Hence, there are two types of Dedekind cuts---a Dedekind cut \((A, B)\) such that:
    \begin{enumerate}[nolistsep, label=(\roman*), listparindent=\parindent]
        \ii There (uniquely) exists \(p \in P\) such that \(B = \{\,x \in P \mid x \ge p\,\}\);
            in this case we write \((A, B) = [p]\).
        \ii \((A, B)\) is a gap.
    \end{enumerate}
    Note that \([p]\) is a Dedekind cut of \((P, \le)\) for all \(p \in P\).

    Now, define \(C\) and a relation \(\preceq\) on \(C\) by:
    \begin{gather*}
        C \triangleq \{\,(A, B) \mid (A, B) \text{ is a Dedekind cut of } (P, \le)\,\} \\
        \shortintertext{and}
        (A, B) \preceq (A', B') \iff A \subseteq A'\text{.}
    \end{gather*}

    \clm[VAPQliWa]{
        \((C, \preceq)\) is a totally ordered set.
    }{
        It is evident that \((C, \preceq)\) is an ordered set.
        Take any \((A, B), (A', B') \in C\).
        Suppose they are incomparable for the sake of contradiction, i.e.,
        \(A \setminus A' = A \cap B' \neq \OO\) and \(A' \setminus A = A' \cap B \neq \OO\).
        Let \(a \in A \cap B'\) and \(a' \in A' \cap B\).
        Then, \(a < a'\) and \(a' < a\), which is impossible
        by asymmetry of \(<\). \qed
    }

    If \(p < q\) where \(p, q \in P\),
    then we have \([p] \prec [q]\).
    Hence, \((P', \mathord{\preceq} \cap P'^2)\) is isomorphic to \((P, \le)\)
    where \(P' \triangleq \{\,[x] \mid x \in P\,\}\).
    Thus, we simply prove that \((C, \preceq)\) is a completion of \((P', \mathord{\preceq} \cap P'^2)\).

    \clm[dQPTDxRu]{
        \(\fall c, d \in C,\: (c \prec d \implies \exs p \in P,\: c \prec [p] \prec d)\).
    }{
        Take any \(c, d \in C\) with \(c \prec d\).
        In other words, \(c = (A, B)\) and \(d = (A', B')\) with \(A \subsetneq A'\).
        Take \(p \in A' \setminus A\).
        As \((P, \le)\) has no greatest element,
        \WLOG, we may assume \(p\) is not a least element of \(B\).
        Let \([p] = (A'', B'')\).
        \begin{itemize}[nolistsep, leftmargin=*, listparindent=\parindent]
            \ii
            As \((P, \le)\) is totally ordered,
            the fact that \(p\) is not a least element of \(B\)
            asserts the existence of \(b \in B\) such that \(b < p\).
            Hence, \(\fall x \in B'',\:\fall a \in A,\: a < b < p \le x\).
            Thus, \(B'' \subsetneq B\); \((A, B) \prec [p]\). \checkmark

            \ii
            As \(A'\) has no greatest element and \((P, \le)\) is totally ordered,
            there exists \(a' \in A'\) such that \(p < a'\).
            Then, we have \(\fall x \in A'',\: \fall b' \in B',\: x < p < a' < b'\),
            i.e., \(A'' \subsetneq A'\).
            Hence, \([p] \prec (A', B')\). \checkmark
            \qed
        \end{itemize}
    }

    \noindent
    \Cref{clm:dQPTDxRu} also shows that \((C, \preceq)\) is a densely ordered set.

    \clm[RxJeNpUA]{
        \((C, \preceq)\) has no endpoints.
    }{
        Take any \((A, B) \in C\).
        \begin{itemize}[nolistsep, leftmargin=*, listparindent=\parindent]
            \ii
            In the same way as in the proof of \Cref{clm:dQPTDxRu},
            any \(p \in B\) which is not a least element of \(B\) (which exists)
            satisfies \((A, B) \prec [p]\).
            Hence, \(C\) has no greatest element. \checkmark

            \ii
            Take any \(p \in A\).
            As \(A\) has no greatest element, there exists \(a \in A\) such that \(p < a\).
            Hence, \([p] \prec (A, B)\). \checkmark
            \qed
        \end{itemize}
    }

    \clm[IuBZDfdA]{
        \((C, \preceq)\) is complete.
    }{
        Let \(\OO \subsetneq S \subseteq C\) be bounded from above.
        Let \((A_0, B_0)\) be an upper bound of \(S.\)
        Define
        \[\textstyle
            A_S \triangleq \bigcup \{\,A \mid (A, B) \in S\,\} \quad\text{and}\quad
            B_S \triangleq P \setminus A_S = \bigcap \{\,B \mid (A, B) \in S\,\}.
        \]
        As \(B_0 \subseteq B_S\), \((A_S, B_S)\) is a cut.
        Moreover, if \(x\) is a greatest element of \(A_S\),
        then \(x\) is a greatest element of some \(A\) where \((A, B) \in S\).
        Hence, \(A_S\) has no greatest element; \((A_S, B_S) \in C\).

        As \(A \subseteq A_S\) for all \((A, B) \in S\),
        \((A_S, B_S)\) is an upper bound of \(S\).
        If \((A', B')\) is another upper bound of \(S\),
        then \(\fall (A, B) \in S,\: A \subseteq A'\),
        i.e., \(A_S \subseteq A'\).
        Hence, \((A_S, B_S) \preceq (A', B')\).
        Thus, \((A_S, B_S) = \sup_{\preceq} S\). \qed
    }

    \Crefrange{clm:VAPQliWa}{clm:IuBZDfdA} shows that \((C, \preceq)\) satisfies
    the requirements of the theorem; hence the existence part is done.
    We now prove the uniqueness.

    \clm[]{
        Let \((C, \preceq)\) and \((C^\ast, \preceq^\ast)\)
        be two complete totally ordered sets that satisfies
        (i)--(iv).
        Then, there exists an isomorphism \(h\) between
        \((C, \preceq)\) and \((C^\ast, \preceq^\ast)\)
        such that \(\fall p \in P,\: h(p) = p\).
    }{
        For each \(c \in C\) and \(c^\ast \in C^\ast\) define
        \[
            S_c \triangleq \{\,p \in P \mid p \prec c\,\} \quad\text{and}\quad
            S_{c^\ast} \triangleq \{\,p \in P \mid p \prec^\ast c^\ast\,\}.
        \]
        Note that:
        \begin{enumerate}[nolistsep, label=(\Roman*), listparindent=\parindent]
            \ii
            \(\sup_{\preceq} S_c = c\) and \(\sup_{\preceq^\ast} S_{c^\ast} = c^\ast\) by (iii).

            \ii
            Take any \(c \in C\).
            By (iii) and (iv), there exists \(q \in P\) such that \(c \prec q\).
            Hence, \(\fall p \in S_c,\: p \prec c \prec q\),
            and thus, \(\fall p \in S_c,\: p \prec^\ast q\) by (ii).
            Therefore, every \(S_c\) is bounded from above in \((C^\ast, \preceq^\ast)\).
            By symmetry, every \(S_{c^\ast}\) is bounded from above in \((C, \preceq)\).

            \ii
            For \(p \in P\) and \(\OO \subsetneq X \subseteq P\) such that both
            \(\sup_{\preceq} X\) and \(\sup_{\preceq^\ast} X\) exist,
            we have
            \[
                \textstyle p \prec \sup_{\preceq} X \iff \exs x \in X,\: p \prec x
                                         \iff \exs x \in X,\: p \prec^\ast x
                                         \iff\textstyle p \prec^\ast \sup_{\preceq^\ast} X.
            \]
        \end{enumerate}

        \noindent
        Now, define \(h \colon C \to C^\ast\) by \(c \mapsto \sup_{\preceq^\ast} S_c\).
        \(h\) is well-defined by (II) and \((C^\ast, \preceq^\ast)\) being complete.
        We now show that \(h\) is a desired isomorphism.

        \begin{itemize}[nolistsep, leftmargin=*, listparindent=\parindent]
            \ii
            Take any \(c^\ast \in C^\ast\).
            Let \(c \triangleq \sup_{\preceq} S_{c^\ast}\).
            Then,
            \begin{alignat*}{2}
                h(c) &= \textstyle\sup_{\preceq^\ast} S_c &\qquad& \\
                     &= \textstyle\sup_{\preceq^\ast} \{\,p \in P \mid p \prec \sup_{\preceq} S_{c^\ast}\,\} \\
                     &= \textstyle\sup_{\preceq^\ast} \{\,p \in P \mid p \prec^\ast \sup_{\preceq^\ast} S_{c^\ast}\,\} && \comment*{(III)} \\
                     &= \textstyle\sup_{\preceq^\ast} \{\,p \in P \mid p \prec^\ast c^\ast\,\} && \comment*{(I)} \\
                     &= \textstyle\sup_{\preceq^\ast} S_{c^\ast} = c^\ast. && \comment*{(I)}
            \end{alignat*}
            Hence, \(h\) is onto \(C^\ast\). \checkmark

            \ii
            Take any \(c, d \in C\) with \(c \prec d\).
            Then, by (iii), there exist \(p_1, p_2 \in P\) such that \(c \prec p_1 \prec p_2 \prec d\).
            We then have \(\sup_{\preceq^\ast} S_c \preceq^\ast p_1 \prec^\ast p_2 \preceq^\ast \sup_{\preceq^\ast} S_d \prec\).
            Hence, \(h(c) \prec^\ast h(d)\); \(h\) is an isomorphism. \checkmark

            \ii
            For each \(p \in P \subseteq C \cap C^\ast\), by (I),
            \(h(p) = \sup_{\preceq^\ast} S_p = p\). \checkmark
        \end{itemize}
    }
    \noindent
    Thus the theorem is now proven.
}

\subfile{../exercises/exercise4-5.tex}

\end{document}
