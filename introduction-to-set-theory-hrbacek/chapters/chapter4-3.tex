\documentclass[../introduction_to_set_theory.tex]{subfiles}

\begin{document}

\section{Countably Infinite sets}

\dfn[countably infinite]{Countably Infinite Set}{
    \begin{itemize}[nolistsep, leftmargin=*, listparindent=\parindent]
        \ii A set \(S\) is \textit{countably infinite} if \(|S| = |\NN|\).
        \ii A set \(S\) is \textit{countable} if \(|S| \le |\NN|\).
        \ii \(|\NN| = \NN\), i.e., the cardinality of countably infinite sets is \(\NN\).
    \end{itemize}
}

\nt{
    In the book, the author uses the term `countable' and `at most countable'
    for \(|S| = |\NN|\) and \(|S| \le |\NN|\), respectively.
}

\notat[]{Cardinality of Countably Infinite Sets}{
    We use the symbol \(\aleph_0\) (read \textit{aleph-naught})
    to denote the cardinality of countably infinite sets,
    i.e., \(\aleph_0 = \NN\).
}

\thm[subsetOfCountable]{}{
    A subset of a countably infinite set is countable.
}
\pf{Proof}{
    Assume \(A\) is countably infinite and \(B \subseteq A\) is infinite.
    Let \(\lang a_i \rang_{i \in \NN}\) be an injective sequence whose range is \(A\).

    Let \(g \colon \Seq(\NN) \to \NN\) be defined by
    \[
        g(k) \triangleq \min \big\{\,i \in \NN \:\big|\: a_i \in B \setminus \{\,a_{k_j} \mid j \in \dom k\,\}\,\big\}.
    \]
    Note that \(g\) is well-defined since \(B\) is infinite.
    Then, by \nameref{th:generalRecursion}, there exists a sequence \(\lang k_i \rang_{i \in \NN}\) of natural numbers
    such that \(\fall n \in \NN,\: k_n = g\big(\restr{k}{n}\big)\).
    By construction, \(\lang k_i \rang_{i \in \NN}\) is injective,
    and thus \(\lang a_{k_i} \rang_{i \in \NN}\) is an injective sequence whose range is \(B\)
    by the same argument of \Cref{clm:009f.fIsSurjective} of \Cref{th:thenIsoToN}.
}

\cor[]{}{
    A set is countable if and only if it is either finite or countably infinite.
}
\mclm{Proof}{\hfill
\begin{itemize}[nolistsep, wide=0pt, widest={(\(\Rightarrow\))}, leftmargin=*, listparindent=\parindent]
    \ii[(\(\Rightarrow\))]
    Let \(S\) be countable.
    Let \(f \colon S \hookrightarrow \NN\).
    Then, \(|S| = |\ran f|\) and \(\ran f\) is a subset of \(\NN\).
    Hence, by \Cref{th:subsetOfCountable}, \(S\) is countably infinite if it is not finite.

    \ii[(\(\Leftarrow\))]
    \Cref{th:infiniteThenBigCard}
    \qed
\end{itemize}
}

\thm[imageOfCountable]{}{
    If \(X\) is countably infinite and \(f\) is a function, then \(f[X]\) is countable.
}
\pf{Proof}{
    If \(f[X] = \OO\), then it is done; assume \(f[X] \neq \OO\). \WLOG, \(X \subseteq \dom f\).
    Let \(\lang x_i \rang_{i \in \NN}\) be an injective sequence whose range is \(X\).
    Let \(g \colon f[X] \to \NN\) be defined by
    \[g(y) \triangleq \min \{\,i \in \NN \mid y = f(x_i)\,\}.\]
    \(g\) is injective, and thus \(|f[X]| \le \aleph_0\).
}

\thm[productOfCountable]{}{
\begin{enumerate}[nolistsep, label=(\roman*), ref=\protect{(\roman*)}, listparindent=\parindent]
    \ii
    If \(A\) and \(B\) are countably infinite, then \(A \times B\) is countably infinite.

    \ii
    If \(A\) is countably infinite and \(B \neq \OO\) is finite,
    then \(A \times B\) is countably infinite.

    \ii
    If \(A\) and \(B\) are countable, then \(A \times B\) is countable.
\end{enumerate}
}
\mclm{Proof}{\hfill
\begin{enumerate}[nolistsep, label=(\roman*), leftmargin=*, listparindent=\parindent]
    \ii
    The function \(f \colon \NN \times \NN \to \NN\) defined by
    \(f(x, y) = 2^x \cdot 3^y\) is injective by elementary number theory.
    Also, we have an injection \(g \colon \NN \to \NN \times \NN\)
    defined by \(g(x) = (x, 0)\).
    Hence, by \nameref{th:cantorBernstein}, we have \(|\NN \times \NN| = \aleph_0\).

    \ii
    Let \(|B| = n\).
    Then, we have
    \begin{alignat*}{2}
        |A \times B| &= |\NN \times n| &\qquad& \\
                     &\le |\NN \times \NN| && \comment*{\Cref{exer:4.1.3}} \\
                     &= \aleph_0. && \comment*{\Cref{th:productOfCountable}}
    \end{alignat*}
    Let \(b \in B\). Then, we have
    \begin{alignat*}{2}
        \aleph_0 &= |A| &\qquad& \\
              &= |A \times \{b\}| && \comment*{\(a \mapsto (a, b)\)} \\
              &\le |A \times B|. && \comment*{\Cref{exer:4.1.3}}
    \end{alignat*}
    Hence, by \nameref{th:cantorBernstein}, \(|A \times B| = \aleph_0\).

    \ii
    If one of them is empty, then \(A \times B = \OO\).
    If \(A\) and \(B\) are finite, then \(A \times B\) is finite by \Cref{exer:4.2.2}.
    If any of them is countably infinite, and if both are nonempty, then \(A \times B\)
    is countably infinite by (i) and (ii).
    \qed
\end{enumerate}
}

\cor[productOfFiniteCountable]{}{
    Let \(\lang\,A_i \mid i \in n\,\rang\) be a system of countably infinite sets where \(n > 0\).
    Then, \(\prod_{i=0}^{n-1} A_i\) is countably infinite.
}
\pf{Proof}{
    Let \(\mbf{P}(x)\) be the property ``\(\prod_{i=0}^{x-1} A_i\) is countably infinite
    for each system \(\lang\,A_i \mid i \in x\,\rang\) of countably infinite sets.
    \(\mbf{P}(1)\) evidently holds.

    Fix \(n > 0\) and assume \(\mbf{P}(n)\).
    Now, take any system \(\lang\,A_i \mid i \in n + 1\,\rang\) of countably infinite sets.
    Then, since we have a natural mapping \(f \colon \prod_{i = 0}^{n} A_i \hooktwoheadrightarrow
    \left(\prod_{i=0}^{n-1} A_i\right) \times A_n\) defined by
    \(\lang\,a_0, \cdots, a_{n}\,\rang \mapsto (\lang\,a_0, \cdots, a_{n-1}\,\rang, a_n)\),
    we get
    \begin{alignat*}{2}
        \textstyle \left|\prod_{i = 0}^{n} A_i\right|
        &= \textstyle \left|\left(\prod_{i=0}^{n-1} A_i\right) \times A_n\right| &\qquad& \\
        &= |\NN \times \NN| && \comment*{\(\mbf{P}(n)\)} \\
        &= \aleph_0. && \comment*{\Cref{th:productOfCountable}}
    \end{alignat*}
    Hence, we have \(\mbf{P}(n + 1)\).

    Therefore, by \Cref{exer:3.2.11}, the result follows.
}

\thm[unionOfCountable]{}{
    Let \(\lang\,a_n \mid n \in \NN\,\rang\) countably infinite system of infinite sequences.
    Then, \(\bigcup_{n \in \NN} \ran a_n\) is countable.
}
\pf{Proof}{
    Define \(f \colon \NN \times \NN \twoheadrightarrow \bigcup_{n \in \NN} \ran a_n\)
    by \(f(n, k) = a_n(k)\). The result follows
    by \Cref{th:imageOfCountable} and \Cref{th:productOfCountable}.
}

\nt{
    Note that we cannot yet prove the proposition ``the union of
    countably infinite system of countable sets is countable''
    since, if \(\lang\,A_n \mid n \in \NN\,\rang\) is the system,
    we do not have enough tools to show the existence of
    \(\lang\,a_n \mid n \in \NN\,\rang\) such that \(\ran a_n = A_n\) for each \(n \in \NN\).
}

\thm[seqCountable]{}{
    If \(A\) is countably infinite, then \(\Seq(A)\) is countably infinite.
}
\pf{Proof}{
    It is enough to show \(\Seq(\NN) = \bigcup_{n \in \NN} \NN^n\) is countably infinite.
    Fix any \(f \colon \NN \twoheadrightarrow \NN \times \NN\).
    Define \(\lang\,a_n \mid n \in \NN\,\rang\) recursively by
    \begin{alignat*}{2}
        \fall i \in \NN, && a_0(i) &= \lang\rang \\
        \fall n, i \in \NN, &\:& a_{n+1}(i) &= \lang\,b_0, \cdots, b_{n-1}, i_2\,\rang \\
               &&& \hphantom{=1}\text{where \(g(i) = (i_1, i_2)\) and
                   \(a_n(i_1) = \lang\,b_0, \cdots, b_{n-1}\,\rang\).}
    \end{alignat*}
    The existence is justified by \nameref{th:recursion}.
    Then, with \nameref{th:induction}, it is easy to prove that
    \(\ran a_n = \NN^n\) for each \(n \in \NN\).
    Hence, by \Cref{th:unionOfCountable}, \(\bigcup_{n \in \NN} \NN^n\) is countably infinite.
}

\cor[]{}{
    The set of all finite subsets of a countably infinite set is countably infinite.
}
\pf{Proof}{
    Let \(A\) be countably infinite.
    Let \(f \colon \Seq(A) \to \mcal P(A)\)
    by \(f(\lang\,a_0, \cdots, a_{n-1}\,\rang) = \{\,a_0, \cdots, a_{n-1}\,\}\).
    Then, \(\ran f\) is countable by \Cref{th:imageOfCountable} and \Cref{th:seqCountable}.
    \(\ran f\) is countably infinite since we have an injection \(a \mapsto \{a\}\).
}

\thm[equivClassOfContable]{}{
    An equivalence on a countably infinite set has at most countably many equivalence classes.
}
\pf{Proof}{
    Let \(E\) be an equivalence on a countably infinite set \(A\).
    Let \(f \colon A \twoheadrightarrow A/E\) be defined by \(a \mapsto [a]_E\).
    Hence, by \Cref{th:imageOfCountable}, \(A/E\) is countable.
}

\thm[countableClosure]{}{
    Let \(\mfr{A} = (A, \lang\,R_0, \cdots, R_{m-1}\,\rang, \lang\,F_0, \cdots, F_{n-1}\,\rang)\)
    be a structure.
    If \(C \subseteq A\) is countable, then \(\cl{C}\) is also countable.
}
\pf{Proof}{
    \Cref{th:closureAsUnion} says that \(\cl C = \bigcup_{i \in \NN} C_i\)
    where \(C_0 = C\) and \(C_{i+1} = C_i \cup \bigcup_{j=0}^{n-1} F_j[C_i^{f_j}]\).

    Let \(c \colon \NN \twoheadrightarrow C\).
    Let \(g \colon \NN \twoheadrightarrow
    (n + 1) \times \NN \times \NN^{f_0} \times \cdots \times \NN^{f_{n-1}}\).
    Now, define \(\lang\,a_i \mid i \in \NN\,\rang\) recursively by
    \begin{alignat*}{2}
        \fall k \in \NN &\quad& a_0(k) &\triangleq c(k) \\
        \fall i, k \in \NN, &\:& a_{i+1}(k) &\triangleq \begin{cases}
            F_p(a_i(r_p^0), \cdots, a_i(r_p^{f_{p-1}})) & \text{if } 0 \le p < n \\
            a_i(q) & \text{if } p = n
        \end{cases} \\ &&&
        \hphantom{=.}\text{where } g(k) = \lang\,p, q, \lang\,r_0^0, \cdots, r_0^{f_0-1}\,\rang,
        \cdots, \lang\,r_{n-1}^{0}, \cdots, r_{n-1}^{f^{n-1}-1}\,\rang\,\rang.
    \end{alignat*}
    It is apparent by \nameref{th:induction} that \(\ran a_i = C_i\) for each \(i \in \NN\).
    Hence, by \Cref{th:unionOfCountable}, \(\cl{C}\) is countable.
}

\subsection*{Selected Problems}

\exer[4.3.1]{}{
Let \(|A_1| = |A_2|\) and \(|B_1| = |B_2|\).
\begin{enumerate}[nolistsep, label=(\roman*), ref=\protect{\Cref{exer:4.3.1} (\roman*)}, listparindent=\parindent]
    \ii
    If \(A_1 \cap B_1 = \OO\) and \(A_2 \cap B_2 = \OO\),
    then \(|A_1 \cup B_1| = |A_2 \cup B_2|\).
    \ii
    \(|A_1 \times B_1| = |A_2 \times B_2|\)
    \ii
    \(|\Seq(A_1)| = |\Seq(A_2)|\)
\end{enumerate}
}
\mclm{Proof}{\hfill
\begin{enumerate}[nolistsep, label=(\roman*), leftmargin=*, listparindent=\parindent]
    \ii
    Let \(f_A \colon A_1 \hooktwoheadrightarrow A_2\) and
    \(f_B \colon B_1 \hooktwoheadrightarrow B_2\).
    Then, \(f_A \cup f_B \colon A_1 \cup B_1 \hooktwoheadrightarrow A_2 \cup B_2\).

    \ii
    Let \(f_A \colon A_1 \hooktwoheadrightarrow A_2\) and
    \(f_B \colon B_1 \hooktwoheadrightarrow B_2\).
    We may define \(g \colon A_1 \times B_1 \hooktwoheadrightarrow A_2 \times B_2\) by
    \((a, b) \mapsto (f_A(a), f_B(b))\).

    \ii
    Let \(f \colon A_1 \hooktwoheadrightarrow A_2\).
    We may define \(g \colon \Seq(A_1) \hooktwoheadrightarrow \Seq(A_2)\) by
    \[\lang\,a_0, \cdots, a_{n-1}\,\rang \mapsto \lang\,f(a_0), \cdots, f(a_{n-1})\,\rang.\]

    \qed
\end{enumerate}
}

\exer[4.3.2]{}{
    If \(A\) is finite and \(B\) is countably infinite, then \(A \cup B\) is countably infinite.
}
\pf{Proof}{
    Let \(f_A \colon A \hookrightarrow \NN\) and \(f_B \colon B \hookrightarrow \NN\).
    Then, we may define \(g \colon A \cup B \hookrightarrow \NN \times \NN\) by
    \[
        g(x) = \begin{cases}
            (f_A(x), 0) & \text{if } x \in A \\
            (f_B(x), 1) & \text{if } x \in B \setminus A
        \end{cases}
    \]
    Hence, \(|A \cup B| \le \aleph_0\) by \Cref{th:productOfCountable}.
    Moreover, \(\aleph_0 = |B| \le |A \cup B|\) by \Cref{exer:4.1.3}.
    The result follows from \nameref{th:cantorBernstein}.
}

\setexernumber{3}

\exer[4.3.4]{}{
    If \(A\) is finite and nonempty, then \(\Seq(A)\) is countably infinite.
}
\pf{Proof}{
    Let \(B \triangleq A \cup \NN\).
    Then, by \Cref{exer:4.3.2}, \(B\) is countably infinite and \(\Seq(B)\) is countably infinite by
    \Cref{th:seqCountable}.
    Hence, as \(\Seq(A) \subseteq \Seq(B)\), \(|\Seq(A)| \le \aleph_0\).

    Fix any \(a \in A\).
    Let \(s\) be the infinite sequence with \(\fall i \in \NN,\: s_i = a\).
    Then, we have \(f \colon \NN \hookrightarrow \Seq(A)\)
    defined by \(f(n) = \restr{s}{n}\); thus \(\aleph_0 \le |\Seq(A)|\).
    The result follows from \nameref{th:cantorBernstein}.
}

\exer[4.3.5]{}{
    Let \(A\) be countably infinite.
    The set \([A]^n = \{\,S \subseteq A \mid |S| = n\,\}\) is countably infinite
    for all \(n > 0\).
}
\pf{Proof}{
    It is enough to show that \([\NN]^n\) is countably infinite for all \(n > 0\).
    Evidently, \(i \mapsto \{i\}\) is an injective mapping on \(\NN\) onto \([\NN]^{1}\).
    Hence, \(|[\NN]^1| = \aleph_0\).

    For the sake of induction, fix \(n > 0\) and assume \(|[\NN]^{n}| = \aleph_0\).
    We may define \(f \colon [\NN]^n \hookrightarrow [\NN]^{n+1}\) by
    \[
        f(x) \triangleq x \cup
        \big\{\,\max \{\,i \in \NN \mid i \in x \} + 1 \,\big\}.
    \]
    Hence, \(\aleph_0 \le |[\NN]^{n+1}|\).

    Now, since \(|[\NN]^n| = |\NN^n| = \aleph_0\) by \Cref{cor:productOfFiniteCountable},
    there exists an injection \(g \colon [\NN]^n \hookrightarrow \NN^n\).
    We define \(h \colon [\NN]^{n+1} \hookrightarrow \NN^{n+1}\) by
    \begin{align*}
        h(x) &\triangleq g(x \setminus \{i\}) \cup \{(n, i)\} \\
             &\quad\text{where } i = \max x.
    \end{align*}
    Hence, \(|[\NN]^{n+1}| \le |\NN^{n+1}| \le \aleph_0\).
    \Cref{exer:3.2.11} assures that \(\fall n > 0,\: |[\NN]^n| = \aleph_0\).
}

\setexernumber{9}

\exer[4.3.10]{}{
    Let \((S, \preceq)\) be a linearly ordered set and let \(\lang\,A_n \mid n \in \NN\,\rang\)
    be an infinite sequence of finite subsets of \(S\).
    Then, \(\bigcup_{n = 0}^{\infty} A_n\) is countable.
}
\pf{Proof}{
    \WLOG, \(A_n \neq \OO\) for each \(n \in \NN\).
    \clm[FMsBqJPY]{
        For each finite \(A \subseteq S\),
        there uniquely exists a unique isomorphism between
        \(\left(|A|, \mathord{\le} \cap |A|^2\right)\)
        and \((A, \mathord{\preceq} \cap A^2)\).
    }{
        We have existence for each \(A\) by \Cref{th:uniqueFiniteTotalOrder}.
        Hence, we only prove the uniqueness by induction.
        If \(|A| = 0\), we have only one isomorphism \(\OO\).

        Fix some \(n \in \NN\) and assume the proposition holds for all \(A\) with cardinality \(n\).
        Take any \(A \subseteq S\) with \(|A| = n + 1\).
        Let \(f\) and \(g\) be two isomorphisms between
        \((n+1, \mathord{\le} \cap (n + 1)^2)\)
        and \((A, \mathord{\le} \cap A^2)\).
        Then, \(f(n) = g(n)\) since the greatest element is unique.
        Let \(B = A \setminus \{f(n)\}\).
        Then, \(\restr{f}{n}\) and \(\restr{g}{n}\) are isomorphisms between
        \((n, \mathord{\le} \cap n^2)\)
        and \((B, \mathord{\le} \cap B^2)\).
        Hence, \(\restr{f}{n} = \restr{g}{n}\), and thus \(f = g\).
        The result follows from \nameref{th:induction}.
    }

    \Cref{clm:FMsBqJPY} enables us to guarantee the existence of infinite sequence
    \(\lang\,a_n \mid n \in \NN\,\rang\) such that, for each \(n \in \NN\):
    \begin{enumerate}[nolistsep, label=(\roman*), ref=\protect{(\roman*)}, listparindent=\parindent]
        \ii
        \(\restr{a_n}{|A_n|}\) is the isomorphism between
        \(\left(|A_n|, \restr{\mathord{\le}}{|A_n|^2}\right)\)
        and \(\big(A_n, \restr{\mathord{\preceq}}{A_n^2}\big)\).
        \ii
        \(\fall k \ge |A_n|,\: a_n(k) = a_n(0)\).
    \end{enumerate}
    Hence, \(\ran a_n = A_n\) for each \(n \in \NN\), and thus
    \(\bigcup_{n=0}^{\infty} A_n\) is countable by \Cref{th:unionOfCountable}.
}

\exer[4.3.11]{}{
    Any partition of a countable set has a set of representatives.
}
\pf{Proof}{
    Let \(A\) be countable and \(S\) be a partition of \(A\).
    There exists \(f \colon A \hookrightarrow \NN\).
    Then,
    \[
        X \triangleq \{\,f\inv(\min f[C]) \mid C \in S \,\}
    \]
    is a set of representatives.
}

\end{document}
