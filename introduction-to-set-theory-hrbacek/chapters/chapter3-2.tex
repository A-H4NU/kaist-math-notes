\documentclass[../introduction_to_set_theory.tex]{subfiles}

\begin{document}
    
\section{The Recursion Theorem}

\dfn[sequence]{Sequence}{
    \begin{itemize}[nolistsep, leftmargin=*]
        \ii 
        A \textit{sequence} is a function whose domain is a natural number or \(\NN\).

        \ii
        A sequence whose domain is a natural number \(n\) is called a
        \textit{finite sequence of length \(n\)} and is denoted
        \[
            \lang\: a_i \mid i < n \:\rang \quad\text{or}\quad
            \lang\: a_i \mid i = 0, 1, \cdots, n - 1 \:\rang \quad\text{or}\quad
            \lang\: a_0, a_1, \cdots, a_{n-1} \:\rang.
        \]
        In particular, \(\lang\rang = \OO\)---the \textit{empty sequence}---is the unique sequence of length \(0\).
        \[
            \textstyle\Seq(A) \triangleq \bigcup_{n \in \NN} A^n
        \]
        denote the set of all finite sequence of elements of \(A\).

        \ii
        A sequence whose domain is \(\NN\) is called a \textit{infinite sequence}
        and is denoted
        \[
            \lang\: a_i \mid i \in \NN \:\rang \quad\text{or}\quad
            \lang\: a_i \mid i = 0, 1, 2, \cdots \:\rang \quad\text{or}\quad
            \lang a_i \rang_{i=0}^{\infty}.
        \]
        Infinite sequences of elements of \(A\) are members of \(A^{\NN}\).
        We also use the notation \(\{\,a_i \mid i \in \NN\,\}\) or \(\{a_i\}_{i=0}^\infty\), etc.,
        for the range of the sequence \(\lang a_i \mid i \in \NN \rang\).
    \end{itemize}
}

\nt{
\begin{itemize}[nolistsep, leftmargin=*]
    \ii
    A natural number \(n \in \NN\) is the set of all natural numbers less than \(n\).
    See \Cref{exer:3.2.6}.
    \ii 
    Since \(A^n \in \mcal P(\NN \times A)\) for each \(n \in \NN\),
    \(\mcal A = \{\,w \mid \exs n \in \NN,\:w = A^n\,\}\) exists,
    and thus \(\Seq(A) = \bigcup \mcal A\) exists.
\end{itemize}
}

\thm[recursion]{The Recursion Theorem}{
    Let \(A\) be a set, \(a \in A\), and \(g \colon A \times \NN \to A\).
    Then, there uniquely exists an infinite sequence \(f \colon \NN \to A\) such that
    \begin{enumerate}[nolistsep, label=(\roman*)]
        \ii \(f_0 = a\) and
        \ii \(\fall n \in \NN,\: f_{n+1} = g(f_n, n)\).
    \end{enumerate}
}
\pf{Proof}{
    We say \(t \colon (m + 1) \to A\) is an \textit{\(m\)-step computation based on \(a\) and \(g\)}
    if \(t_0 = a\) and \(\fall k < m,\: t_{k+1} = g(t_k, k)\).
    Let \(F \triangleq \{\,t \in \Seq(A) \mid t \text{ is an \(m\) step computation for some \(m \in \NN\)}\,\}\).
    Let \(f \triangleq \bigcup F\).

    \clm[3.3.2.fIsFunction]{
        \(f\) is a function.
    }{
        We shall show that \(F\) is a compatible system of functions
        so we may conclude \(f\) is a function thanks to \Cref{th:compatibleThenUnionIsFunction}.
        Take any \(t, u \in F\).
        Let \(n = \dom t \in \NN\) and \(m = \dom u \in \NN\).
        \WLOG, \(n \le m\) (thanks to \nameref{th:NisLinearlyOrdered}), i.e., \(n \subseteq m\).
        Hence, \((\dom t) \cap (\dom u) = n\).
        If \(n = 0\), then it is done; assume \(n > 0\).
        Then, there exists \(n' \in \NN\) such that \(n' + 1 = n\) by \Cref{exer:3.2.4}.

        Surely, \(t_0 = a = u_0\).
        Moreover, if \(t_k = u_k\) where \(k < n'\),
        then \(k+1 < n'+1 = n\) (\Cref{exer:3.2.2}) and \(t_{k+1} = g(t_k, k) = g(u_k, k) = u_{k+1}\).
        Therefore, by \nameref{exer:3.2.12}, we have \(\fall k \le n',\: t_k = u_k\);
        \(t\) and \(u\) are compatible.
        \qed
    }

    \clm[3.3.2.domfisN]{
        \(\dom f = \NN\) and \(\ran f \subseteq A\).
    }{
        We already have \(\dom f \subseteq \NN\) and \(\ran f \subseteq A\) by \Cref{th:compatibleThenUnionIsFunction}.
        To show \(\dom f = \NN\), it suffices to show that, for any \(n \in \NN\), there is an
        \(n\)-step computation based on \(a\) and \(g\). Clearly, \(t = \{(0, a)\}\)
        is a \(0\)-step computation.

        Assume there exists an \(n\)-step computation \(t \colon (n+1) \to A\) where \(n \in \NN\).
        Then, define \(u \colon ((n+1)+1) \to A\) by
        \(u \triangleq t \cup \{(n + 1, g(t_n, n))\}\).
        Then, one may easily verify that \(u\) is an \((n+1)\)-step computation.
        Therefore, by \nameref{th:induction}, the result follows.
        \qed
    }

    \noindent
    We now check if \(f\) satisfies the conditions (i) and (ii).
    \begin{enumerate}[nolistsep, label=(\roman*)]
        \ii 
        Clearly, \(f_0 = a\).
        \ii
        Take any \(n \in \NN\).
        Let \(t\) be an \((n+1)\)-step computation.
        Then, \(\fall k \le n,\: f_k = t_k\),
        and \(f_{n+1} = t_{n+1} = g(t_n, n) = g(f_n, n)\).
    \end{enumerate}
    Now, we are left to show the uniqueness of such \(f\).

    Let \(h \colon \NN \to A\) be a sequence that satisfies the conditions (i) and (ii).
    Clearly, \(f_0 = a = h_0\). And, if \(f_n = h_n\),
    then \(f_{n+1} = g(f_n, n) = g(h_n, n) = h_{n+1}\).
    Therefore, by \nameref{th:induction}, \(\fall k \in \NN,\: f_k = h_k\),
    i.e., \(f = k\) by \Cref{lem:functionEqualsIff}.
}

\thm[]{}{
    Let \((A, \preceq)\) be a nonempty linearly ordered set with the properties:
    \begin{enumerate}[nolistsep, label=(\roman*), leftmargin=*]
        \ii For every \(p \in A\), there exists \(q \in A\) such that \(p \prec q\).
        \ii Every nonempty subset of \(A\) that has a \(\preceq\)-least element.
        \ii Every nonempty subset of \(A\) that has an upper bound has a \(\preceq\)-greatest element.
    \end{enumerate}
    Then, \((A, \preceq)\) is isomorphic to \((\NN, \le)\).
}
\pf{Proof}{
    By (i), \(\{\,a \in A \mid x \prec a\,\} \neq \OO\) for each \(x \in A\)
    and it has a \(\preceq\)-least element.
    Hence, we may define \(g \colon A \times \NN \to A\) by
    \(g(x, n) \triangleq \min \{\,a \in A \mid x \prec a\,\}\).
    Then, \nameref{th:recursion} guarantees the existence of a function \(f \colon \NN \to A\)
    such that:
    \begin{itemize}[nolistsep]
        \ii \(f_0 = \min A\) \comment{(i) and \(A \neq \OO\)}
        \ii \(\fall n \in \NN,\: f_{n+1} = g(f_n, n) = \min \{\,a \in A \mid f_n \prec a\,\}\).
    \end{itemize}
    By \Cref{exer:3.3.1}, we have \(f_m \prec f_n\) whenever \(m < n\).
    This also implies that \(f\) is injective.

    \clm[009f.fIsSurjective]{
        \(\ran f = A\)
    }{
        Suppose \(\ran f \subsetneq A\) for the sake of contradiction.
        Then, \(A \setminus \ran f \neq \OO\), and thus we may take \(p = \min (A \setminus \ran f)\),
        which gives \(p \neq f_0\) immediately.
        Hence, \(B = \{\,a \in A \mid a \prec p\,\} \neq \OO\) and \(p\) is an upper bound of \(B\).
        By (iii), \(q = \max B\) exists. Since \(q \prec p\), we have \(q \in \ran f\),
        i.e., \(q = f_m\) for some \(m \in \NN\).

        Suppose there is some \(r \in A\) such that \(q \prec r \prec p\).
        Then, \(r \in B\), which contradicts the maximality of \(q\).
        Hence, \(p = \min \{\,a \in A \mid f_m \prec a\,\} = f_{m+1}\),
        which contradicts \(p \notin \ran f\).
        \qed
    }

    We have \(f \colon \NN \hooktwoheadrightarrow A\) by \Cref{clm:009f.fIsSurjective}.
    Hence, by \nameref{th:NisLinearlyOrdered} and \Cref{lem:oneImplicationIsEnough}, \(f\) is an isomorphism 
    between \((\NN, \le)\) and (\(A, \preceq\)).
}

\thm[generalRecursion]{The Recursion Theorem: General Version}{
    Let \(S\) be a set and let \(g \colon \Seq(S) \to S\).
    Then, there exists a unique sequence \(f \colon \NN \to S\) such that
    \[
        \fall n \in \NN,\: f_n = g \left(\restr{f}{n}\right) = g(\lang\,f_0, f_1, \cdots, f_{n-1}\,\rang).
    \]
}
\pf{Proof}{
    Define \(G \colon \Seq(S) \times \NN \to \Seq(S)\) by
    \[
        G(t, n) = \begin{cases}
            t \cup \{(n, g(t))\} & \text{if \(t\) is a sequence of length \(n\)} \\
            \lang\rang & \text{otherwise}.
        \end{cases}
    \]
    Then, by \nameref{th:recursion}, there exists a sequence \(F \colon \NN \to \Seq(S)\)
    such that:
    \begin{itemize}[nolistsep]
        \ii \(F_0 = \lang\rang\)
        \ii \(\fall n \in \NN,\: F_{n+1} = G(F_n, n)\).
    \end{itemize}

    If \(F_k \in S^k\), then \(F_{k+1} = F_k \cup \{k, g(F_k)\} \in S^{k+1}\).
    Hence, by \nameref{th:induction}, \(\fall n \in \NN,\: F_n \in S^n\).
    Moreover, since \(F_k \subsetneq_{\Seq(S)} F_{k+1}\),
    by \Cref{exer:3.3.1}, \(\fall m, n \in \NN,\: (m < n \implies F_m \subsetneq F_n)\);
    hence \(\{\,F_n \mid n \in \NN\,\}\) is a compatible system of functions.

    Let \(f \triangleq \bigcup_{n \in \NN} F_n\).
    Then, we have \(\restr{f}{n} = F_n\) for all \(n \in \NN\).
    Therefore, for each \(n \in \NN\), \(f_n = F_{n+1}(n) = g(F_n) = g\left(\restr{f}{n}\right)\).

    Let \(h \colon \NN \to S\) be another sequence such that \(\fall n \in \NN,\: h_n = g\left(\restr{h}{n}\right)\).
    Suppose \(\fall k < n, f_k = h_k\).
    Then, we have \(f_{n} = g\left(\restr{f}{n}\right) = g\left(\restr{h}{n}\right) = h_{n}\).
    Therefore, by \nameref{th:strongInduction}, \(f = h\).
}

\thm[parametricRecursion]{The Recursion Theorem: Parametric Version}{
    Let \(a \colon P \to A\) and \(g \colon P \times A \times \NN \to A\) be functions.
    Then, there uniquely exists a function \(f \colon P \times \NN \to A\) such that
    \begin{enumerate}[nolistsep, label=(\roman*)]
        \ii \(\fall p \in P,\: f(p, 0) = a(p)\)
        \ii \(\fall n \in \NN,\: \fall p \in P,\: f(p, n + 1) = g(p, f(p, n), n)\).
    \end{enumerate}
}
\pf{Proof}{
    Let \(G \colon A^P \times \NN \to A^P\) be defined by
    \[
        G(x, n)(p) = g(p, x(p), n)
    \]
    for each \(x \in A^P\), \(p \in P\), and \(n \in \NN\).
    Then, by \nameref{th:recursion}, there exists \(F \colon \NN \to A^P\)
    such that
    \[
        F_0 = a \quad\text{and}\quad \fall n \in \NN,\:F_{n+1} = G(F_n, n).
    \]
    
    Now, let \(f \colon P \times \NN \to A\) be defined by
    \(f(p, n) = F_n(p)\).
    We now check if \(f\) satisfies the conditions:
    \begin{enumerate}[nolistsep, label=(\roman*), leftmargin=*]
        \ii
        For all \(p \in P\), we have \(f(p, 0) = F_0(p) = a(p)\).

        \ii
        For each \(n \in \NN\) and \(p \in P\),
        \(f(p, n+1) = F_{n+1}(p) = G(F_n, n)(p) = g(p, F_n(p), n) = g(p, f(p, n), n)\).
        
    \end{enumerate}
    
    Let \(h \colon P \times \NN \to A\) be another function that satisfies (i) and (ii).
    Clear, we have \(\fall p \in P,\: f(p, 0) = a(p) = h(p, 0)\).
    Assuming \(\fall p \in P,\: f(p, n) = h(p, n)\) gives,
    for all \(p \in P\), \(f(p, n+1) = g(p, f(p, n), n) = g(p, h(p, n), n) = h(p, n+1)\).
    Hence, by \nameref{th:induction}, we get \(f = h\).
}

\subsection*{Selected Problems}

\exer[3.3.1]{}{
    Let \(f \colon \NN \to A\) be an infinite sequence
    where \((A, \preceq)\) is an ordered set. Then,
    \[
        \fall n \in \NN,\: f_n \prec f_{n+1}
        \implies \fall m, n \in \NN,\: (n < m \implies f_n \prec f_m).
    \]
}
\pf{Proof}{
    Fix any \(n \in \NN\) and let \(\mbf{P}(x)\) be the property ``\(f_n \prec f_x\).''
    \(\mbf{P}(n+1)\) evidently holds.
    Now, suppose \(\mbf{P}(k)\) holds where \(k \in \NN\).
    Then, chaining \(f_n \prec f_k\) and \(f_k \prec f_{k+1}\) gives \(\mbf{P}(k+1)\).
    Therefore, by \Cref{exer:3.2.11}, we get \(\fall m \ge n + 1,\: f_n \prec f_m\).
}

\exer[3.3.2]{}{
    Let \((A, \preceq)\) be a nonempty linearly ordered set.
    We say that \(q \in A\) is a \textit{successor} of \(p \in A\) if there is no \(r \in A\)
    such that \(p \prec r \prec q\). Assume \((A, \preceq)\) has the following properties:
    \begin{enumerate}[nolistsep, label=(\roman*)]
        \ii Every \(p \in A\) has a successor.
        \ii Every nonempty subset of \(A\) has a \(\preceq\)-least element.
        \ii If \(p \in A\) is not the \(\preceq\)-least element of \(A\),
            then \(p\) is a successor of some \(q \in A\).
    \end{enumerate}
    Then, \((A, \preceq)\) is isomorphic to \((\NN, \le)\).
}
\pf{Proof}{
    By (i), for each \(p \in P\), \(\{\,q \in A \mid p \prec q\,\} \neq \OO\), and thus it has
    a \(\preceq\)-least element by (ii).
    Therefore, by \nameref{th:recursion}, there exists a sequence \(f \colon \NN \to A\)
    such that \(f_0 = \min A\) and \(\fall n \in \NN,\: f_{n+1} = \min \{\,q \in A \mid f_n \prec q\,\}\).

    \clm[88332b22]{
        \(\ran f = A\)
    }{
        Suppose \(X \triangleq A \setminus \ran f \neq \OO\) for the sake of contradiction.
        Then, by (ii), we may take \(p = \min X\).
        Since \(\min A = f_0 \in \ran f\), \(p\) is not the \(\preceq\)-least element of \(A\).
        Hence, by (iii), \(p\) is a successor of some \(q \in A\).
        As \(q \prec p\), we have \(q \in \ran f\) by minimality of \(q\),
        i.e., \(q = f_m\) for some \(m \in \NN\).
        Since there is no \(r \in A\) such that \(q \prec r \prec p\),
        we have \(p = f_{m+1}\) by definition,
        which contradicts \(p \notin \ran f\). \qed
    }
    Since \(f_n \prec f_{n+1}\) for all \(n \in \NN\),
    by \Cref{exer:3.3.1}, \(\fall m, n \in \NN,\: (m < n \implies f_m \prec f_n)\),
    which means \(f\) is injective.

    Therefore, together with \Cref{clm:88332b22}, \(f\) is an isomorphism between
    \((\NN, \le)\) and \((A, \preceq)\) by \Cref{lem:oneImplicationIsEnough}.
}

\setexernumber{4}

\exer[3.3.5]{The Recursion Theorem: Finite Version}{
    Let \(g\) be a function such that \(\dom g \subseteq A \times \NN\) and \(\ran g \subseteq A\).
    Let \(a \in A\).
    Then, there uniquely exists a sequence \(f\) of elements of \(A\) such that
    \begin{enumerate}[nolistsep, label=(\roman*)]
        \ii \(f_0 = a\)
        \ii \(\fall n \in \NN,\: [n+1 \in \dom f \implies f_{n+1} = g(f_n, n)]\)
        \ii \(f\) is either an infinite sequence or a finite sequence of length \(k+1\) and \((f_k, k) \notin \dom g\).
    \end{enumerate}
}
\pf{Proof}{
    Let \(\ol{A} = A \cup \{\ol{a}\}\) where \(\ol{a} \notin A\).
    (Such \(\ol{a}\) exists by \Cref{exer:1.3.3} (ii).)
    Define \(\ol{g} \colon \ol{A} \times \NN \to \ol{A}\) by
    \[
        \ol{g}(x, n) = \begin{cases}
            g(x, n) & \text{if \((x, n) \in \dom g\)} \\
            \ol{a} & \text{otherwise.}
        \end{cases}
    \]
    Then, \nameref{th:recursion} guarantees the existence of \(\ol{f} \colon \NN \to \ol{A}\)
    such that \(\ol f_0 = a\) and \(\fall n \in \NN,\: \ol f_{n+1} = \ol g(\ol f_n, n)\).
    We have two cases: ``\(\fall n \in \NN,\: \ol{f}_{n} \neq \ol{a}\)'' and ``\(\exs n \in \NN,\: \ol f_n = \ol{a}\).''
    They are resolved by \Cref{clm:7a376003,clm:db06f79c}, respectively.

    \clm[7a376003]{
        If ``\(\fall n \in \NN,\: \ol{f}_{n} \neq \ol{a}\),''
        then \(\ol{f}\) is an infinite sequence of elements of \(A\) that satisfies (i) and (ii).
    }{
        The assumption essentially says that
        \((\ol f_n, n) \in \dom g\) and \(\ol f_{n+1} = g(\ol f_n, n) \in A\) for all \(n \in \NN\),
        i.e., \(\ol f\) satisfies (i) and (ii).
        As \(\ol f_0 = a \in A\), \(\ol f\) is an infinite sequence of elements of \(A\).
        \qed
    }

    \clm[db06f79c]{
        If ``\(\exs n \in \NN,\: \ol f_n = \ol{a}\),''
        then there exists \(k \in \NN\) such that
        \(\restr{\ol{f}}{k+1}\) satisfies the conditions (i), (ii), and (iii).
    }{
        By \nameref{th:NisWellOrdered}, we have \(\ell \triangleq \min \{\,n \in \NN \mid \ol f_n = \ol{a}\,\}\).
        Since \(\ol f_0 \in A\), we have \(\ell \neq 0\), and thus \(\ell = k + 1\) for some \(k \in \NN\)
        by \Cref{exer:3.2.4}.
        It immediately follows that \(\fall n \le k,\: \ol{f}_n \in A\).
        Hence, \(f \triangleq \restr{\ol f}{k+1}\) is a finite sequence of length \(k+1\)
        of elements of \(A\).

        We check if \(f\) satisfies the conditions (i), (ii), and (iii):
        \begin{enumerate}[nolistsep, label=(\roman*), leftmargin=*]
            \ii \(f_0 = \ol{f}_0 = a\)
            \ii 
            If \(n < k\), i.e., \(n + 1 \in \dom f = k + 1\),
            then \(f_{n+1} = \ol{f}_{n+1} = \ol g(\ol{f}_n, n) = g(f_n, n)\).
            \ii
            If \((f_k, k) \in \dom g\), then we would have
            \(\ol{f}_\ell = \ol g(\ol{f}_k, k) = \ol g(f_k, k) = g(f_k, k) \neq \ol{a}\).
            Hence, we must have \((f_k, k) \notin \dom g\). \qed
        \end{enumerate}
    }

    Now, we prove the uniqueness.
    Let \(f\) and \(h\) be two sequences of elements of \(A\)
    that satisfies the conditions (i), (ii), and (iii).
    \textsf{WLOG}, \(\dom h \subseteq \dom f\).

    Let \(\mbf{P}(x)\) be the property ``\(x \in \dom h \land f_x = h_x\).''
    \(\mbf{P}(0)\) evidently holds.
    \clm[e3045233]{
        \(\fall n \in \NN,\: (n + 1 \in \dom f \land \mbf{P}(n) \implies \mbf{P}(n+1))\)
    }{
        Assume \(n + 1 \in \dom f\) and \(\mbf{P}(n)\).
        Then, since \((h_n, n) = (f_n, n) \in \dom g\),
        \(n + 1 \in \dom h\) and \(h_{n+1} = g(h_n, n) = g(f_n, n) = f_{n+1}\).
        Hence, \(\mbf{P}(n+1)\) holds. \qed
    }

    If \(f\) is a finite sequence, \Cref{clm:e3045233} and \nameref{exer:3.2.12}
    imply \(h = f\).
    If \(f\) is an infinite sequence, \Cref{clm:e3045233} and \nameref{th:induction}
    imply \(h = f\).
}

\exer[3.3.6]{}{
    If \(X \subseteq \NN\), then there is a one-to-one (finite or infinite) sequence
    \(f\) such that \(\ran f = X\).
}
\pf{Proof}{
    If \(X = \OO\), \(\lang\rang\) is the one we are looking for.
    Assume \(X \neq \OO\).

    Let \(g = \big\{\,((x, n), y) \in (X \times \NN) \times X \:\big|\: y = \min \{\,k \in X \mid x < k\,\}\,\big\}\).
    Then, \(g\) is a function with \(\dom g \subseteq \NN \times \NN\) and \(\ran g \subseteq \NN\).
    By \nameref{exer:3.3.5}, there exists a sequence \(f\) of elements of \(X\)
    such that
    \begin{enumerate}[nolistsep, label=(\roman*)]
        \ii \(f_0 = \min X\) \comment{\(\min X\) exists by \nameref{th:NisWellOrdered}}
        \ii \(\fall n \in \NN,\: (n + 1 \in \dom f \implies f_{n+1} = g(f_n, n))\)
        \ii \(f\) is either an infinite sequence or a finite sequence of length \(k+1\) and \((f_k, k) \notin \dom g\).
    \end{enumerate}
    
    Note that \(\dom g = \{\,(x, n) \in X \times \NN \mid \exs y \in X,\: x < y\,\}\).
    Moreover, for each \(n \in \NN\) such that \(n + 1 \in \dom f\),
    we have \(f_n < f_{n+1}\); hence \(\forall m, n \in \dom f,\: (m < n \implies f_m < f_n)\)
    (in the similar manner of \Cref{exer:3.3.1}),
    and thus \(f\) is injective.

    Suppose \(Y = X \setminus \ran f \neq \OO\) for the sake of contradiction.
    By \nameref{th:NisWellOrdered}, we may take \(y = \min Y\).
    Then, by \Cref{th:hasUpperBoundThenMaxExists}, we may let \(z = \max \{\,x \in X \mid x < y\,\}\).
    \(z = f_m\) for some \(m \in \dom f\). Hence, \(y = f_{m+1}\).
}

\end{document}
