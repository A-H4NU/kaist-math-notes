\documentclass[../introduction_to_set_theory_Note.tex]{subfiles}

\begin{document}

\section{Well-Ordered Sets}

\notat[Qtmidocs]{}{
    Let \(\omega \triangleq \NN\) be the least transfinite number.
    We let \(\omega + 0 = \omega\) and \(\omega + (n + 1) = S(\omega + n)\) for each \(n \in \NN\).
    In this fashion, one may let
    \[
        \omega \cdot 2 = \omega + \omega = \{\,0, 1, 2, \cdots, \omega, \omega+1, \omega + 2, \cdots\,\},
    \]
    \[
        \omega \cdot \omega = \{\,0, 1, \cdots, \omega, \omega + 1, \cdots,
        \omega \cdot 2, \cdots, \omega \cdot 3, \cdots\,\}.
    \]
    These sets are totally ordered by \(\in\),
    and the ordering is a \hyperref[dfn:wellOrdering]{well-ordering}.
}

\dfn[]{Initial Segment}{
    \begin{itemize}[nolistsep, leftmargin=*, listparindent=\parindent]
        \ii
        Let \((L, \le)\) be a \hyperref[dfn:totalOrdering]{totally ordered} set.
        A set \(S \subsetneq L\) is called an \textit{initial segment} of \(L\) if
        \[
            \fall a \in S,\: \fall x \in L,\: (x < a \implies x \in S).
        \]

        \ii
        Let \((W, \le)\) be a well-ordered set.
        If \(a \in W\), we call the set
        \[
            W[a] \triangleq \{\,x \in W \mid x < a\,\}
        \]
        the \textit{initial segment of \(W\) given by \(a\)}.
    \end{itemize}
}

\mlemma[initSegIsGiven]{}{
    If \((W, \le)\) is a well-ordered set,
    and if \(S\) is an initial segment of \((W, \le)\),
    then there exists \(a \in W\) such that \(S = W[a]\).
}
\pf{Proof}{
    Let \(X \triangleq W \setminus S\).
    As \(X \neq \OO\), there exists \(a \triangleq \min X\).
    If \(x < a\), then \(x \notin X\), and thus \(x \in S\).
    Conversely, if \(x \ge a\), then \(x \in S\) would imply \(a \in S\),
    and thus \(x \in X\).
    Hence, \(x \in S \iff x < a\).
}

\dfn[]{Increasing Function}{
    A function \(f\) on a totally ordered set \((L, \le)\) into \(L\)
    is \textit{strictly increasing} if
    \[
        \fall x_1, x_2 \in L,\: (x_1 < x_2 \implies f(x_1) < f(x_2)).
    \]
}

\nt{
    \noindent
    Every automorphism of a totally ordered set
    is strictly increasing.
}

\mlemma[wosetIncreasing]{}{
    If \((W, \le)\) is a well-ordered set and if \(f \colon W \to W\) is a strictly increasing function,
    then \(\fall x \in W,\: x \le f(x)\).
}
\pf{Proof}{
    Suppose \(X \triangleq \{\,x \in W \mid f(x) > x\,\}\) is nonempty for the sake of contradiction.
    Let \(a = \min X\). Then, \(f(a) < a\), and \(f(f(a)) < f(a)\)
    as \(f\) is strictly increasing.
    Hence, \(f(a) \in X\) even though \(f(a) < a\), which is a contradiction.
}

\cor[noIsoToInitSeg]{}{
\begin{enumerate}[nolistsep, label=(\roman*), ref=\protect{\Cref{cor:noIsoToInitSeg} (\roman*)}, listparindent=\parindent]
    \ii\label{itm:noIsoToInitSeg.i}
    No well-ordered set is isomorphic to an initial segment of itself.
    \ii
    \(\mrm{Id}_W\) is the unique automorphism of a well-ordered set \((W, \le)\).
    \ii\label{itm:noIsoToInitSeg.iii}
    If \(W_1\) and \(W_2\) are isomorphic well-ordered sets,
    then the isomorphism between \(W_1\) and \(W_2\) is unique.
\end{enumerate}
}
\mclm{Proof}{\hfill
\begin{enumerate}[nolistsep, label=(\roman*), leftmargin=*, listparindent=\parindent]
    \ii
    Suppose \(f \colon W \to W[a]\) is an isomorphism for some \(a \in W\) for the sake of contradiction.
    Then, \(f(a) \in W[a]\) and therefore \(f(a) < a\),
    which contradicts \Cref{lem:wosetIncreasing}.

    \ii
    Let \(f\) be an automorphism of \(W\).
    Then, both \(f\) and \(f\inv\) are strictly increasing functions,
    and thus by \Cref{lem:wosetIncreasing},
    for all \(x \in W\), \(f(x) \ge w\) and \(f\inv(x) \ge x\).
    Hence, \(x \ge f(x)\).
    Therefore, \(f(x) = x\) for all \(x \in W\) due to antisymmetry.

    \ii
    If \(f\) and \(g\) are isomorphisms between \(W_1\) and \(W_2\),
    then \(f \circ g\inv\) is an automorphism of \(W_2\).
    Then, by (ii), \(f \circ g\inv = \mrm{Id}_{W_2}\);
    thus \(f = g\).
    \qed
\end{enumerate}
}

\thm[wosetComparable]{}{
    If \((W_1, \le_1)\) and \((W_2, \le_2)\) are
    well-ordered sets, then exactly one of the following holds:
    \begin{enumerate}[nolistsep, label=(\roman*), ref=\protect{(\roman*)}, listparindent=\parindent]
        \ii \((W_1, \le_1)\) and \((W_2, \le_2)\) are isomorphic.
        \ii \((W_1, \le_1)\) is isomorphic to an initial segment of \((W_2, \le_2)\).
        \ii \((W_2, \le_2)\) is isomorphic to an initial segment of \((W_1, \le_1)\).
    \end{enumerate}
    Moreover, in each case, the isomorphism is unique.
}
\pf{Proof}{
    (i), (ii), and (iii) are mutually exclusive by \ref{itm:noIsoToInitSeg.i}.
    Also, the uniqueness follows from \ref{itm:noIsoToInitSeg.iii}.
    Hence, we only need to show at least one of them holds.
    Define \(f \triangleq \{\,(x, y) \in W_1 \times W_2 \mid W_1[x] \text{ is isomorphic to } W_2[y]\,\}\).

    \clm[fLFMCNPo]{
        \(f\) is an injective function.
    }{
        First of all, \(f\) is a function since,
        if \(W_1[x]\) is isomorphic to \(W_2[y]\) and \(W_2[y']\),
        then \(y = y'\) (otherwise, one would be an initial segment of the other).
        By symmetry, \(f\) is injective. \qed
    }

    \clm[vdqpAzmT]{
        \(f\) is strictly increasing.
    }{
        Take \(x, x' \in W_1\) with \(x <_1 x'\).
        If \(h\) is the isomorphism between
        \(W_1[x']\) and \(W_2[f(x')]\),
        then \(\restr{h}{W_1[x]}\) is an isomorphism between
        \(W_1[x]\) and \(W_2[h(x)]\).
        Hence, by \Cref{clm:fLFMCNPo}, \(f(x) = h(x) <_2 f(x')\). \qed
    }

    \noindent
    By \Cref{clm:vdqpAzmT,lem:oneImplicationIsEnough},
    \(f\) is an isomorphism between \(\dom f\) and \(\ran f\).
    \Cref{clm:ZpcauFEl} completes the proof.

    \clm[ZpcauFEl]{
        If \(\dom f \neq W_1\), then \(\ran f = W_2\).
    }{
        Let \(S \triangleq \dom f\).
        If \(z < x\) and \(x \in S\),
        then by the same argument in the proof of \Cref{clm:vdqpAzmT},
        \(z \in S\). Hence, \(S\) is an initial segment of \(W_1\).

        Suppose \(T \triangleq \ran f \neq W_2\) for the sake of contradiction.
        Then, by the same argument, \(T\) is an initial segment of \(W_2\).
        \(S = W_1[a]\) and \(T = W_2[b]\) for some \(a \in W_1\) and \(b \in W_2\)
        by \Cref{lem:initSegIsGiven};
        in other words, \((a, b) \in f\), or \(a \in \dom f = W_1[a]\),
        which is impossible.
    }

    \noindent
    By symmetry, if \(\ran f \neq W_2\), we must have \(\dom f = W_1\).
}

\dfn[smallerOrderType]{}{
    Let \((W_1, \le_1)\) and \((W_2, \le_2)\) be well-ordered sets.
    We say \(W_1\) has \textit{smaller order type} than \(W_2\)
    if \(W_1\) is isomorphic to an initial segment of \(W_2\).
}

\nt{
    \noindent
    By \Cref{th:wosetComparable}, for two well-ordered sets,
    if they are not isomorphic, then one of them has smaller order type than the other.
}

\subfile{../exercises/exercise6-1.tex}

\end{document}
