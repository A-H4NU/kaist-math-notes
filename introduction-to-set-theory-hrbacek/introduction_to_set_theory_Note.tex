\documentclass[a4paper,12pt]{report}
\usepackage{subfiles}
\usepackage{calc}

\usepackage{preamble}
\usepackage{macros}
\usepackage{letterfonts}

\usepackage{pgfplots}

\pgfplotsset{compat=1.18}
\iftrue
\pgfplotsset{
    axis lines=middle,
    xlabel=$x$,
    ylabel=$y$,
    no markers,
    samples=200,
    trig format plots=rad,
    every axis plot/.append style={
        line width=.7pt,
        % smooth,
    },
}
\fi
\usepgfplotslibrary{fillbetween}
\pgfmathsetmacro{\PI}{3.141592654}

\hypersetup{
    colorlinks=true,
    linkcolor=black,
    filecolor=black,
    urlcolor=black,
    pdftitle={Introduction to Set Theory},
    pdfpagemode=FullScreen,
}

\usetikzlibrary{arrows.meta, positioning, decorations, decorations.markings}

\let\dom\relax
\DeclareMathOperator{\dom}{dom}
\let\ran\relax
\DeclareMathOperator{\ran}{ran}
\DeclareMathOperator{\field}{field}
\DeclareMathOperator{\Seq}{Seq}

\begin{document}

% NOTE: This should not be commented out for the final product.
% The usage of memoize may corrupt some tcb boxes;
% this is only for fast compilation.
\mmzset{disable}

\begin{titlepage}
	\raggedleft

	\rule{1pt}{\textheight}
	\hspace{0.05\textwidth}
	\parbox[b]{0.75\textwidth}{

		{\Huge\bfseries Summary for\\[0.5\baselineskip] Introduction to Set Theory}\\[2\baselineskip]
		\\[4\baselineskip]
		{\Large\textsc{seungwoo han}}

		\vspace{0.5\textheight}

        {\noindent
        Hrbacek, Karel, and Thomas J. Jech.
        \textit{Introduction to Set Theory, Revised and Expanded}. 3rd ed.,
        CRC Press, 1999.}\\[\baselineskip]
	}
\end{titlepage}
\pdfbookmark[section]{\contentsname}{toc}
\tableofcontents
\hypersetup{
    colorlinks=true,
    linkcolor=red!50!black,
    filecolor=black,
    urlcolor=red!50!black,
}
\pagebreak

\chapter{Sets}\label{chap:sets}
\subfile{./chapters/chapter1-1.tex}
\chapter{Relations, Function, and Ordering}\label{chap:relFuncOrder}
\subfile{./chapters/chapter2-1.tex}
\subfile{./chapters/chapter2-2.tex}
\subfile{./chapters/chapter2-3.tex}
\chapter{Natural Numbers}\label{chap:nat}
\subfile{./chapters/chapter3-1.tex}
\subfile{./chapters/chapter3-2.tex}
\subfile{./chapters/chapter3-3.tex}
\subfile{./chapters/chapter3-4.tex}
\chapter{Finite, Countable, and Uncountable Sets}\label{chap:finite}
\subfile{./chapters/chapter4-1.tex}
\subfile{./chapters/chapter4-2.tex}
\subfile{./chapters/chapter4-3.tex}
\subfile{./chapters/chapter4-5.tex}
\subfile{./chapters/chapter4-4.tex}
\chapter{Cardinal Numbers}\label{chap:cardinal}
\subfile{./chapters/chapter5-1.tex}
\chapter{Ordinal Numbers}\label{chap:ordinal}
\subfile{./chapters/chapter6-1.tex}
\chapter{Alephs}\label{chap:aleph}
\chapter{Axiom of Choice}\label{chap:aoc}

\vfill
\begin{center}
    \textbf{\textit{End.}}
\end{center}

\end{document}
