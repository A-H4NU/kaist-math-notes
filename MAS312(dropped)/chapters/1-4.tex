\documentclass[../modern_algebra_2.tex]{subfiles}

\begin{document}

\section{Unique Factorization Domains}

\begin{Definition}{Unique Factorization Domain}[]
    A \emph{unique factorization domain} is an integral domain \(R\)
    such that:
    \begin{enumerate}[label=(\roman*)]
        \ii
        Every nonzero nonunit element
        is a product of irreducible elements of \(R\).
        \ii
        If \(u = p_1 p_2 \cdots p_r = q_1 q_2 \cdots q_s\)
        are two products of irreducible elements of \(R\),
        then \(r = s\), and (possibly after reordering)
        \(p_i\) is an associate of \(q_i\) for all \(i \in [r]\).
    \end{enumerate}
\end{Definition}

\begin{Example}{}[]
\begin{enumerate}[label=(\roman*), ref=\protect{(\roman*)}, listparindent=\parindent]
    \ii
    \(\ZZ\) is a unique factorization domain by Fundamental Theorem of Arithmetic.
    \ii
    \(\mcal{O}_{\QQ(\sqrt{-5})}\) is not a unique factorization domain.
    \[
        6 = (1+\sqrt{-5})(1-\sqrt{-5}) = 2 \cdot 3
    \]
    is a two different factorizations of \(6\) into irreducible elements.
\end{enumerate}
\end{Example}

\begin{Theorem}{}[ufdIrredThenPrime]
    Let \(R\) be a UFD.
    Then, every irreducible element in \(R\) is prime.
\end{Theorem}
\begin{myproof}[Proof]
    Let \(p \in R\) be irreducible.
    If \(p \mid ab\), then \(ab = pc\) for some \(c \in R\).
    Since \(R\) is a UFD, \(a\) or \(b\) has a factor which is
    an associate of \(p\), i.e., \(p \mid a\) or \(p \mid b\).
\end{myproof}

\begin{Definition}{Ascending Chain Condition on Principal Ideals}[acc]
    Let \(R\) be an integral domain.
    \(R\) is said to satisfy \emph{ascending chain condition on principal ideals} if,
    for all infinite chains of principal ideals
    \[
        (a_1) \subseteq (a_2) \subseteq (a_3) \subseteq \cdots\text{,}
    \]
    then there exists \(n \in \NN\) such that \((a_n) = (a_{n+1}) = (a_{n+1}) = \cdots\).
\end{Definition}

\begin{Theorem}{}[ufdIff]
    Let \(R\) be an integral domain.
    \(R\) is a unique factorization domain if and only if
    \begin{enumerate}[label=(\roman*)]
        \ii \(R\) satisfies ascending chain condition on principal ideals and
        \ii if \(p\) is irreducible in \(R\), then \(p\) is prime in \(R\).
    \end{enumerate}
\end{Theorem}
\begin{myclaim}[Proof]\hfill
\begin{pftfae}[labelwidth=\widthof{(\(\Rightarrow\))}]
    \ii[(\(\Rightarrow\))]
    Thanks to \Cref{th:ufdIrredThenPrime}, we only need to check (i).

    Let \((a_1) \subseteq (a_2) \subseteq \cdots\) be an ascending chain of principal ideals.
    Let \(a_1 = u p_1^{e_1} \cdots p_n^{e_n}\) be an irreducible factorization.
    There are at most \(e_1 + \cdots + e_n\) strict inclusions.
    \ii[(\(\Leftarrow\))]
    Take any nonunit \(r \in R \setminus \{0\}\).
    We want to find an irreducible factorization of \(r\).
    If \(r\) is already irreducible, then we are done.

    Assume \(r = r_1 r_1'\) for some nonunit \(r_1, r_1' \in R \setminus \{0\}\)
    so that \((r) \subsetneq (r_1)\).
    Continue this to get an ascending chain \((r) \subsetneq (r_1) \subsetneq (r_2) \subseteq \cdots\).
    Hence, we get an irreducible factor \(r_k\) at some point.
    % TODO: fix this
    \qed
\end{pftfae}
\end{myclaim}

\begin{Corollary}{}[pidThenUfd]
    Every principal ideal domain is a unique factorization domain.
\end{Corollary}
\begin{myproof}[Proof]
    By \Cref{th:pidIrredThenPrime}, every irreducible element in \(R\) is prime.

    Let \(I_1 \subseteq I_2 \subseteq \cdots\) be an ascending chain of principal ideals in \(R\).
    Let \(I \triangleq \bigcup_{i \ge 1} I_i\) so that \(I\) is an ideal in \(R\).
    Then, as \(R\) is a principal ideal domain, \(I = (c)\) for some \(c \in R\).
    By definition, \(c \in I_n\) for some \(n \in \ZZ_{>0}\).
    Hence, \(I = (c) \subseteq I_n \subseteq I_{n+1} \subseteq \cdots \subseteq I = (c)\).
\end{myproof}

% TODO: Fermat's theorem

\begin{Theorem}{}[]
    Let \(d \in \ZZ\) be a square-free integer.
    Every nonzero nonunit element in \(\rqi{\sqrt{d}}\) is a product of irreducible elements.
\end{Theorem}
\begin{myproof}[Proof]
    Let
    \[
        S \triangleq \{\,\text{nonzero nonunit elements in }\rqi{\sqrt{d}}\text{ that are not a
        product of irreducible elements}\,\}\text{.}
    \]
    Suppose \(S \neq \OO\) for the sake of contradiction and choose \(a \in S\) such that \(|N(a)|\)
    is minimized.
    As \(a\) is not irreducible, then \(a = bc\) for some nonunit \(b, c \in R \setminus \{0\}\).
    If \(b, c \notin S\), then \(b\) and \(c\) are products of irreducible elements;
    hence \(b \in S\) or \(c \in S\). \WLOG, \(b \in S\). Then, \(|N(b)| < |N(a)|\),
    which contradicts the choice of \(a\).
\end{myproof}

\end{document}
