\documentclass[../modern_algebra_2.tex]{subfiles}

\begin{document}

\section{Basics of Integral Domains}

\begin{Definition}{Integral Domain}[integralDomain]
    A ring \(R\) is an \emph{integral domain} if
    \(R\) is a commutative ring with identity which has no zero divisor.
\end{Definition}

\begin{note}
    Here are some basic facts regarding an integral domain \(R\).
    \begin{enumerate}[label=(\arabic*)]
        \ii If \(ac = bc\) and \(c \neq 0\), then \(a = b\).
        \ii
        Let \(c_1, \cdots, c_n \in R\).
        \[
            (c_1, \cdots, c_n)
            \triangleq \{\,r_1c_1 + \cdots + r_nc_n \mid r_i \in R\,\} \subseteq R
        \]
        is called the \emph{ideal generated by \(c_1, \cdots, c_n\)}.
        If \(n = 1\), then it is called a \emph{principal ideal}.
        \ii
        For \(a, b \in R\) with \(a \neq 0\),
        we write \(a \mid b\) if \(b = ad\) for some \(d \in R\).
        \ii
        For \(a, b \in R \setminus \{0\}\),
        \(d \in R\) is a \emph{greatest common divisor} if
        \begin{enumerate}[label=(\roman*)]
            \ii
            \(d \mid a\) and \(d \mid b\); and
            \ii
            if \(d' \mid a\) and \(d' \mid b\), then \(d' \mid d\).
        \end{enumerate}
        \ii
        \(u \in R\) is a \emph{unit} in \(R\)
        if \(uv = 1\) for some \(v \in R\).
        \(v\) is called the \emph{inverse} of \(u\) and is denoted \(u\inv\).

        \ii
        For \(a, b \in R\), \(a\) is an \emph{associate} of \(b\) if \(a = bu\) for some \(u \in
        R\), or equivalently, if \((a) = (b)\).

        \ii
        For a non-unit \(p \in R \setminus \{0\}\), \(p\) is \emph{irreducible} if \(p = ab\)
        implies \(a\) or \(b\) is a unit, or equivalently, only divisors of \(p\) are associates of
        \(p\) and units.

        \ii
        For a non-unit \(p \in R \setminus \{0\}\), \(p\) is \emph{prime} in \(R\) if \(p \mid ab\)
        implies \(p \mid a\) or \(p \mid b\), or equivalently, \(p\) is prime if \((p)\) is a prime
        ideal of \(R\).

        \ii
        \(R^\ast \triangleq \{\,u \in R \mid u~\text{is a unit in}~R\,\}\)
        is a group under ``\(\cdot\)''.
    \end{enumerate}
\end{note}

\begin{Theorem}{}[primeThenIrred]
    Let \(R\) be an integral domain.
    If \(p \in R\) is prime, then it is irreducible.
\end{Theorem}
\begin{myproof}[Proof]
    Suppose \(p = ab\).
    \WLOG, \(p \mid a\).
    Then, \(a = pr\) for some \(r \in R\).
    Hence, \(p = prb\), which implies \(rb = 1\);
    \(b\) is a unit.
\end{myproof}

\begin{Example}{}[]
    \begin{enumerate}[label=(\roman*), ref=\protect{(\roman*)}]
        \ii
        \(\ZZ\) is an integral domain. \(\ZZ^\ast = \{\pm 1\}\).
        For nonzero \(n \in \ZZ\), \(n\) and \(-n\) are associate.
        \(p \in \ZZ\) is a prime number if and only if \(\pm p\) is prime in \(\ZZ\).

        \ii
        \(\ZZ{}[\sqrt{2}] \coloneqq \{\,a + b\sqrt{2} \mid a, b \in \ZZ\,\}\).
        Then, \(\pm 1 + \sqrt{2}\) are units in \(\ZZ{}[\sqrt{2}]\).
        \(\sqrt{2}\) and \(2 - \sqrt{2}\) are associate.
        There is no \(a, b \in \ZZ\) such that \((a + b\sqrt{2})\sqrt{2} = 2b + a\sqrt{2} = 1\).
        Hence, \(\sqrt{2}\) is not a unit in \(\ZZ{}[\sqrt{2}]\).

        Now, we prove that \(\sqrt{2}\) is irreducible in \(\ZZ{}[\sqrt{2}]\).
        Suppose \((a + b\sqrt{2})(c + d\sqrt{2}) = \sqrt{2}\)
        for some \(a, b, c, d \in \ZZ\).
        Then, we get \(ac + 2bd = 0\) and \(ad + bd = 1\).
        Hence,
        \begin{align*}
            -2
            &= (ac + 2bd)^2 - 2(ad + bc)^2 \\
            &= (a^2 - 2b^2)(c^2 - 2d^2).
        \end{align*}
        \WLOG, \((a + b\sqrt{2})(a - b\sqrt{2}) = a^2 - 2b^2 = \pm 1\);
        thus \(a + b\sqrt{2}\) is a unit in \(\ZZ{}[\sqrt{2}]\).
    \end{enumerate}
\end{Example}

\begin{Definition}{}[]
    \(d \in \ZZ \setminus \{0, 1\}\) is \emph{square-free} if \(c^2 \nmid d\) for all
    \(c \in \ZZ_{\ge 2}\).
    \[
        \QQ(\sqrt{d}) \triangleq \{\,a + b\sqrt{d} \mid a + b \in \QQ\,\}
    \]
    is a field. Now, we introduce a function called \emph{norm}:
    \begin{align*}
        N \colon \QQ(\sqrt{d}) &\longrightarrow \QQ \\
        a + b\sqrt{d} &\longmapsto (a + b\sqrt{d})(a - b\sqrt{d}) = a^2 - b^2d.
    \end{align*}
    Note that for \(d < 0\), \(N(\alpha) \ge 0\) for all \(\alpha \in \QQ(\sqrt{d})\).
\end{Definition}

\begin{Theorem}{}[basicNorm]
    Let \(d\) be a square-free integer. Let \(\alpha, \beta \in \QQ(\sqrt{d})\).
    \begin{enumerate}[label=(\roman*), ref=\protect{\Cref{th:basicNorm} (\roman*)}]
        \ii\label{itm:basicNorm.i}
        \(N(\alpha) = 0 \iff \alpha = 0\)
        \ii\label{itm:basicNorm.ii}
        \(N(\alpha \beta) = N(\alpha)N(\beta)\)
    \end{enumerate}
\end{Theorem}

\begin{Definition}{Ring of Quadratic Integer}[quadInt]
    Let \(d\) be a square-free integer. Then,
    \[
        \rqi{\sqrt{d}} \triangleq \begin{cases}
            \ZZ{}[\sqrt{d}] = \{\,a+b\sqrt{d} \mid a, b \in \ZZ\,\} & \text{if}~d \equiv 2, 3 \pmod{4} \\
            \ZZ{} \left[\frac{1+\sqrt{d}}{2}\right] = \left\{\,\left. a+\frac{1+\sqrt{d}}{2}b\:\right\vert\: a, b \in \ZZ\,\right\} & \text{if}~d \equiv 1 \pmod{4}
        \end{cases}
    \]
    is an integral domain.
    As \(\rqi{\sqrt{d}}\) is a subring of \(\QQ(\sqrt{d})\),
    we may apply the norm function \(N\) for \(\rqi{\sqrt{d}}\).
\end{Definition}

\begin{note}
    The weird definition follows from the fact that
    \(\ZZ{}[\sqrt{d}]\) when \(d \equiv 1 \pmod{4}\) is not integrally closed.
\end{note}

\begin{Theorem}{}[basicQuadInt]
    Let \(d\) be a square-free integer.
    \begin{enumerate}[label=(\roman*), ref=\protect{\Cref{th:basicQuadInt} (\roman*)}, listparindent=\parindent]
        \ii\label{itm:basicQuadInt.i}
        \(\fall \alpha \in \rqi{\sqrt{d}},\: N(\alpha) \in \ZZ\)
        \ii\label{itm:basicQuadInt.ii}
        \(\fall u \in \rqi{\sqrt{d}},\: (u~\text{is a unit} \iff N(u) = \pm 1)\)
        \ii\label{itm:basicQuadInt.iii}
        \(\fall \alpha \in \rqi{\sqrt{d}},\:
        (N(\alpha)~\text{is prime in}~\ZZ \implies \alpha~\text{is irreducible in}~\rqi{\sqrt{d}})\)
        \ii\label{itm:basicQuadInt.iv}
        If \(\pi \in \rqi{\sqrt{d}}\) is prime,
        then \(N(\pi) \in \{\,\pm p^2, \pm p\,\}\) for some prime \(p \in \ZZ\).
        Either \(p\) is irreducible in \(\rqi{\sqrt{d}}\) (in which \(N(\pi) = \pm p^2\))
        or \(p = \pi \pi'\) for some irreducible \(\pi'\) (in which \(N(\pi) = \pm p\)).
    \end{enumerate}
\end{Theorem}
\begin{myclaim}[Proof]
    For simplicity, let
    \[
        \omega \triangleq \begin{cases}
            \sqrt{d} & \text{if}~d \equiv 2, 3 \pmod{4} \\
            \frac{1+\sqrt{d}}{2} & \text{if}~d \equiv 1 \pmod{4}
        \end{cases} \quad\text{and}\quad
        \ol\omega \triangleq \begin{cases}
            -\sqrt{d} & \text{if}~d \equiv 2, 3 \pmod{4} \\
            \frac{1-\sqrt{d}}{2} & \text{if}~d \equiv 1 \pmod{4}
        \end{cases}
    \]
    so that \(\rqi{\sqrt{d}} = \ZZ{}[\omega]\).
\begin{enumerate}[label=(\roman*), listparindent=\parindent]
    \ii
    \[
        N(a + b\omega) = \begin{cases}
            a^2 - db^2 & \text{if}~d \equiv 2, 3 \pmod{4} \\
            a^2 + ab + \frac{1 - d}{4} b^2 d & \text{if}~d \equiv 1 \pmod{4}
        \end{cases}
    \]
    is an integer.
    \ii
    If \(u \in \ZZ{}[\omega]\) is a unit, then \(1 = N(1) = N(u u\inv) = N(u) N(u\inv)\).
    Hence, by (i), \(N(u) = \pm 1\).
    If \(N(a + b\omega) = \pm 1\), then \((a + b \omega)(a - b \omega) = \pm 1\).
    Hence, \(a + b \omega\) is a unit.
    \ii
    Suppose \(\alpha = \beta \gamma\) where \(\alpha, \beta, \gamma \in \ZZ{}[\omega]\)
    and let \(N(\alpha) = p\) is prime in \(\ZZ\).
    Then, \(p = N(\alpha) = N(\beta) N(\gamma)\) and \(N(\beta), N(\gamma) \in \ZZ\) by (i).
    Hence, \(N(\beta) = \pm 1\) or \(N(\gamma) = \pm 1\),
    which implies \(\beta\) or \(\gamma\) is a unit in \(\ZZ{}[\omega]\) by (ii).
    \ii
    Let \((\pi) \subseteq \ZZ{}[\omega]\) be a prime ideal.
    \(\pi\) is irreducible by \Cref{th:primeThenIrred}.
    Let
    \begin{align*}
        \iota \colon \ZZ &\longrightarrow \ZZ{}[\omega] \\
        a &\longmapsto a + 0 \omega
    \end{align*}
    be an injective ring homomorphism.
    Then, \(\iota\inv \bigl((\pi)\bigr) = (\pi) \cap \ZZ \subseteq \ZZ\)
    is a prime ideal in \(\ZZ\).\footnotemark
    \footnotetext{%
        Given a ring homomorphism between commutative rings with identity, the inverse image of
        prime ideal is a prime ideal.
    }
    Hence, \((\pi) \cap \ZZ = (p)\) for some prime \(p \in \ZZ\),
    and thus \(p = \pi \pi'\) for some \(\pi' \in \ZZ{}[\omega]\).
    Therefore, we get \(N(\pi) N(\pi') = N(p) = p^2\) in \(\ZZ\).
    As \(N(\pi) \in (\pi) \cap \ZZ\), we have \(p \mid N(\pi)\).
    Thus, \(N(\pi) \in \{\,\pm p^2, \pm p\,\}\).

    If \(N(\pi) = \pm p^2\), then \(\pi'\) is a unit by (ii), i.e., \(p\) is an associate of \(\pi\)
    and hence \(p\) is irreducible.
    If \(N(\pi) = \pm p\), then \(N(\pi') = \pm p\); hence \(\pi'\) is irreducible by (iii).
    \qed
\end{enumerate}
\end{myclaim}

\begin{Example}{}[]
    \begin{enumerate}[label=(\roman*), ref=\protect{(\roman*)}, listparindent=\parindent]
        \ii 
        \(\rqi{i} = \ZZ{}[i]\) is the \emph{ring of Gaussian integers}.
        \(\ZZ{}[i]^\ast = \{\,\pm 1, \pm i\,\}\).
        \(N(1 \pm i) = 2\); \(1 \pm i\) is irreducible in \(\ZZ{}[i]\).

        \ii
        Consider \(\rqi{\sqrt{-5}} = \ZZ{}[\sqrt{-5}]\).
        \(N(1 + \sqrt{-5}) = 6\); hence \(1 + \sqrt{-5}\) is not prime in \(\ZZ{}[\sqrt{-5}]\)
        by \ref{itm:basicQuadInt.iv}.

        Suppose \(1 + \sqrt{-5} = \alpha \beta\) for some \(\alpha, \beta \in \ZZ{}[\sqrt{-5}]\).
        Then, \(6 = N(1 + \sqrt{-5}) = N(\alpha \beta) = N(\alpha) N(\beta)\).
        Write \(\alpha = a + b\sqrt{-5}\) so that
        \(N(\alpha) = a^2 + 5b^2 \in \{\,1,2,3,6\,\}\).
        As \(a, b \in \ZZ\), \(N(\alpha) \in \{1, 6\}\).
        If \(N(\alpha) = 6\), then \(N(\beta) = 1\).
        Then, we may conclude that \(\alpha\) or \(\beta\) is a unit in \(\ZZ{}[\sqrt{-5}]\)
        by \ref{itm:basicNorm.ii}.
        Hence, \(1+\sqrt{-5}\) is irreducible but not prime, which is a counterexample
        of the converse of \ref{itm:basicQuadInt.iii}.

        Moreover there is no gcd of \(6\) and \(2 + 2\sqrt{-5}\). Note that
        \(6 = (1 + \sqrt{-5})(1 - \sqrt{-5}) = 2 \cdot 3\). Hence, \(1+\sqrt{-5}\) and \(2\) are
        common divisors of \(6\) and \(2 + 2\sqrt{-5}\). Suppose \(d = a + b\sqrt{-5}\) is a gcd of
        \(6\) and \(2+2\sqrt{-5}\) for the sake of contradiction. Then, by \ref{itm:basicNorm.ii},
        \(N(1+\sqrt{-5}) = 6\) and \(N(2) = 4\) both divide \(N(d) = a^2 + 5b^2\).
        Hence, \(12 \mid N(d) = a^2 + 5b^2\).
        On the other hand, as \(d\) divides both \(6\) and \(2+2\sqrt{-5}\),
        \(N(d) = a^2 + 5b^2\) divides \(N(6) = 36\) and \(N(2+2\sqrt{-5}) = 24\).
        Hence, \(N(d) = a^2 + 5b^2 = 12\); but there is no such \(a, b \in \ZZ\).
    \end{enumerate}
\end{Example}

\end{document}
