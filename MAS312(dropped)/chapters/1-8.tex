\documentclass[../modern_algebra_2.tex]{subfiles}

\begin{document}

\section{Integrally Closed Domains}

\begin{Definition}{\(\bm{R}\)-module}[Rmodule]
    Let \(R\) be a commutative ring with identity.
    An \emph{\(R\)-module} \(M\) is an abelian group \((M, +)\)
    with \(R \times M \to M\) (\((r, m) \mapsto rm\); scalar multiplication)
    such that following hold for all \(a, b \in R\) and \(m, n \in M\).
    \begin{enumerate}[label=(\arabic*)]
        \ii \((a+b)m = am+bm\).
        \ii \(a(m+n)=am+an\).
        \ii \((ab)m=a(bm)\).
        \ii \(1\cdot m = m\).
    \end{enumerate}
\end{Definition}

\begin{Example}{}[basicModule]
\begin{enumerate}[label=(\arabic*), ref=\protect{(\arabic*)}]
    \ii
    If \(G\) is an abelian group, then it is a \(\ZZ\)-module.
    \ii
    If \(F\) is a field, then \(M\) is a \(F\)-module if and only if \(M\)
    is a vector space over \(F\).
    \ii
    Let \(R\) be a commutative ring with identity and \(I\) be a subring of \(R\).
    Then, \(I\) is an \(R\)-module if and only if \(I\) is an ideal of \(R\).
\end{enumerate}
\end{Example}

\begin{Definition}{\(\bm{R}\)-submodule}[Rsubmodule]
    Let \(R\) be a commutative ring with identity
    and \(M\) be an \(R\)-module.
    Then, \(N \subseteq M\) is an \emph{\(R\)-submodule} if
    \begin{enumerate}[label=(\arabic*)]
        \ii \((N, +)\) is a subgroup of \((M, +)\) and
        \ii \(\fall a \in R,\: \fall n \in N,\: an \in N\).
    \end{enumerate}

    Let \(S \coloneqq \{\,s_1, s_2, \cdots, s_n\,\} \subseteq M\).
    The \emph{submodule generated by \(S\)} is
    \[\textstyle
        \sum_{i=1}^n Rs_i = \{\,r_1s_1 + \cdots + r_ns_n \mid r_1, \cdots, r_n \in R\,\}\text{.}
    \]
    \(M\) is \emph{finitely generated} if it is generated by some finite subset of \(M\).
\end{Definition}

\begin{Definition}{Integral}[]
    Let \(R\) and \(S\) be integral domains with \(R \subseteq S\).
    Then, \(u \in S\) is \emph{integral} over \(R\)
    if \(u\) is a root of monic polynomial \(f(x) \in R[x]\).
    Moreover, \(S\) is \emph{integral} over \(R\)
    if all elements of \(S\) are integral over \(R\).
    If \(S\) is integral over \(R\), then \(S\) is an \(R\)-module.

    Let \(u_1, \cdots, u_n \in S\).
    We define
    \[
        R[u_1, \cdots, u_n] \triangleq \{\,f(u_1, \cdots, u_n) \mid f(x_1, \cdots, x_n) \in R[x_1, \cdots, x_n]\,\}\text{.}
    \]
    Then, \(R[u_1, \cdots, u_n]\) is the smallest subring of \(S\) containing \(u_1, \cdots, u_n\).
    Furthermore, it is an \(R\)-submodule of \(S\).
\end{Definition}

\begin{note}
    In general, \(R[u_1, \cdots, u_n]\) is \emph{not} finitely generated \(R\)-module.
\end{note}

\begin{Theorem}{}[integralTFAE]
    Let \(R\) be an integral domain and \(L\) be a field.
    Let \(L\) be a subring of \(S\). For each \(u \in L\), \TFAE.
    \begin{enumerate}[label=(\arabic*)]
        \ii \(u\) is integral over \(R\).
        \ii \(R[u]\) is a finitely generated \(R\)-module.
        \ii
        There is a finitely generated nonzero \(R\)-submodule \(M\) of \(L\)
        such that \(uM \coloneqq \{\,um \mid m \in M\,\} \subseteq M\).
    \end{enumerate}
\end{Theorem}
\begin{myclaim}[Proof]\hfill
\begin{pftfae}[labelwidth=\widthof{\(\text{(ii)} \Rightarrow \text{(iii)}\)}]
    \ii[\(\text{(i)} \Rightarrow \text{(ii)}\)]
    \(u^n + a_{n-1}a^{n-1} + \cdots + a_1u + a_0 = 0\)
    for some \(a_{n-1}, \cdots, a_1, a_0 \in R\).

    Take any \(i \in \ZZ_{\ge n}\).
    Then,
    \[
        u^i = u^n u^{i-n}
        = -a_0 u^{i-n} - a_1 u^{i-n+1} - \cdots - a_{n-1}u^{i-1}\text{.}
    \]
    Hence, by induction every, \(u^i\) is in \(\sum_{j=0}^{n-1} Rs^j\).
    Therefore, \(R[u] = \sum_{j=0}^{n-1} Rs^j\) is finitely generated.
    \ii[\(\text{(ii)} \Rightarrow \text{(iii)}\)]
    Set \(M \coloneqq R[u]\).
    \ii[\(\text{(iii)} \Rightarrow \text{(i)}\)]
    Write \(M = \sum_{i=1}^n R\ell_i\) for some \(\ell_1, \cdots, \ell_n \in L\).
    As \(u\ell_i \in M\),
    write \(u\ell_i = \sum_{j=1}^n b_{ij}\ell_j\) for some \(b_{ij} \in R\).
    % TODO: fill here
    \qed
\end{pftfae}
\end{myclaim}

\begin{Corollary}{}[integralClosure]
    Let \(R\) be an integral domain and \(L\) be a field.
    Let \(R\) be a subring of \(L\).
    Then, \(S \triangleq \{\,u \in L \mid u\text{ is integral over }R\,\}\)
    is a subring of \(L\).
    In particular, if \(u \in L\) is integral over \(R\),
    then \(R[u]\) is an integral extension of \(R\).
\end{Corollary}
\begin{myproof}[Proof]
    If suffices to check if \(u, v \in S\), then \(u \pm v, uv \in S\).
    By \Cref{th:integralTFAE}, \(R[u]\) and \(R[v]\) are finitely generated \(R\)-modules.
    Write \(R[u] = \sum_{i=1}^{n} Rf_i\) and \(R[v] = \sum_{j=1}^{m} Rg_j\).
    Then, \(R[u,v] = \sum_{1 \le n\\ 1 \le m} R f_ig_j\).
    As \((u \pm v)R[u, v], uvR[uv] \subseteq R[u,v]\), by \Cref{th:integralTFAE},
    they are integral over \(R\).
\end{myproof}

\begin{Definition}{Integral Closure}[]
    Let \(R\) be an integral domain and \(L\) be a field.
    Let \(R\) be a subring of \(L\).
    The set \(S\) defined in \Cref{cor:integralClosure}
    is called the \textit{integral closure} of \(R\) in \(L\).
\end{Definition}

\begin{Definition}{Integrally Closed Domain}[]
    We say \(R\) is an \emph{integrally closed domain}
    if \(R\) is the integral closure of \(R\) in the fraction field of \(R\).
\end{Definition}

\begin{Theorem}{}[ufdThenIc]
    Every unique factorization domain is integrally closed.
\end{Theorem}
\begin{myproof}[Proof]
    Let \(R\) be a unique factorization domain and \(F\) be its fraction field.
    Take any \(u \in F\) that is integral over \(R\).
    Then, there are some \(a_0, \cdots, a_{n-1} \in R\) with \(u^n + a_{n-1}u^{n-1} + \cdots + a_1u + a_0
    = 0\). Write \(u = b/c\) where \(b,c \in R\) with \(c \neq 0\) and \((1) = (b, c)\). Then,
    \[
        c(a_{n-1}b^{n-1} + \cdots + a_1bc^{n-2} + a_0c^{n-1}) = -b^n\text{.}
    \]
    Hence, \(c\) must be a unit; thus \(u = b/c \in R\).
\end{myproof}

\begin{Example}{}[]
    We showed that \(\CC{}[x,y,z,w]/(xy-zw)\) is not a unique factorization domain
    but is an integrally closed domain.
    % TODO: fill this!
\end{Example}

\begin{Lemma}{}[]
    Let \(R\) be an integrally closed domain and let \(F\) be its fraction field.
    Let \(K\) be an extension field of \(F\).
    Let \(u \in K\) is algebraic over \(F\).
    Then, \(u\) is integral over \(R\) if and only if \(\min_{u,F}(x) \in R[x]\).
\end{Lemma}
\begin{myclaim}[Proof]\hfill
\begin{pftfae}[labelwidth=\widthof{(\(\Rightarrow\))}]
    \ii[(\(\Rightarrow\))]
    There is a monic polynomial \(f(x) \in R[x]\) such that \(f(u) = 0\).
    There is some extension field \(L\) of \(F\)
    containing all roots of \(p(x)\).
    Let \(u_1 = u, u_2, \cdots, u_n\) be all roots of \(p(x)\) in \(L\).
    We have \(p(x) = (x-u_1)(x-u_2)\cdots(x-u_n) \mid f(x)\) by the definition of minimal polynomial.
    Hence, \(u_1, u_2, \cdots, u_n\) are integral over \(R\).
    % TODO: fill this
    \ii[(\(\Leftarrow\))]
    As the minimal polynomial is monic, it is trivial.
    \qed
\end{pftfae}
\end{myclaim}

\begin{Theorem}{}[integralTrans]
    Let \(A\), \(B\), and \(C\) be integral domains with \(A \subseteq B \subseteq C\).
    Then, \(C\) is integral over \(A\) if and only if
    \(C\) is integral over \(B\) and \(B\) is integral over \(A\).
\end{Theorem}
\begin{myclaim}[Proof]\hfill
\begin{pftfae}[labelwidth=\widthof{(\(\Rightarrow\))}]
    \ii[(\(\Rightarrow\))]
    As \(B\) is a subring of \(C\), \(B\) is integral over \(A\).
    As a monic polynomial over \(A\) is a monic polynomial over \(B\), \(C\) is integral over \(B\).
    \ii[(\(\Leftarrow\))]
    Take any \(u \in C\).
    There is a polynomial \(g(x) = x^n + b_{n-1}x^{n-1} + \cdots b_1x + b_0 \in B[x]\) with \(g(u) = 0\).
    Then \(B' \triangleq A[b_0, \cdots, b_{n-1}, u] \subseteq B\) is a finitely generated
    \(A\)-module by \Cref{th:integralTFAE}.
    Then, \(uB' \subseteq B'\); hence \(u\) is integral over \(A\) by \Cref{th:integralTFAE}.
    \qed
\end{pftfae}
\end{myclaim}

\begin{Theorem}{}[]
    Let \(R\) be an integral domain and let \(F\) be its fraction field.
    Let \(K\) be an extension field of \(F\) with \(\dim_F K < \infty\).
    Let \(S\) be the integral closure of \(R\) in \(K\).
    \begin{enumerate}[label=(\arabic*)]
        \ii \(\fall u \in K,\: \exs (s, d) \in S \times D,\: u = s/d\).
        In particular, \(K\) is a fraction field of \(S\).
        \ii \(S\) is an integrally closed domain.
        \ii If \(R\) is an integrally closed domain, then \(S \cap F = R\).
    \end{enumerate}
    \emph{We will show later \(u \in K\) is algebraic over \(F\).}
\end{Theorem}
\begin{myproof}[Proof]\hfill
\begin{enumerate}[label=(\arabic*)]
    \ii
    Let \(p(x) \triangleq \min_{u,F}(x) \in F[x]\).
    Write
    \[
        p(x) = x^{n} + \frac{c_{n-1}}{d_{n-1}}x^{n-1} + \cdots \frac{c_1}{d_1}x + \frac{c_0}{d_0}
    \]
    where \(c_i, d_i \in R\) and \(d_i \neq 0\).
    Let \(d \triangleq d_0 \cdots d_{n-1}\). Then,
    \[
        0 = d^np(u) = (du)^n + \frac{c_{n-1}}{d_{n-1}} d(du)^{n-1} + \cdots
        \frac{c_1}{d_1}d^{n-1}(du) + \frac{c_0}{d_0}d^n\text{,}
    \]
    i.e., \(du \in S\).

    \ii
    Let \(S'\) be the integral closure of \(S\) in \(K\).
    Then, \(S'\) is integral over \(R\) by \Cref{th:integralTrans} so that \(S' \subseteq S\).
    Hence, \(S = S'\).

    \ii
    Trivial.\qed
\end{enumerate}

\begin{Corollary}{}[]
    Let \(R\) be an integral domain and let \(F\) be its fraction field.
    Let \(K\) be a finite extension field of \(F\).
    Let \(S\) be the integral closure of \(R\) in \(K\).
    Then, there are \(d_1, d_2, \cdots, d_n S\) such that \(d_1, \cdots, d_n\) is a basis of \(K\)
    over \(F\).
\end{Corollary}

\begin{Example}{}[]
    Let \(d \in \ZZ\) be square-free.
    Then, we claim that \(\rqi{\sqrt{d}}\) is the integral closure of \(\ZZ\) in \(\QQ(\sqrt{d})\).
    Take any \(u = a + b\sqrt{d} \in \QQ(\sqrt{d})\).
    \begin{itemize}
        \ii
        Suppose \(b = 0\).
        Then, \(\min_{u,\QQ}(x) = x - a\). Thus, \(u\) is integral over \(\ZZ\) if and
        only if \(a \in \ZZ\).
        \ii
        Suppose \(b \neq 0\).
        Then, \(\min_{u,\QQ}(x) = x^2 - 2ax + (a^2-b^2d)\).
        \(u\) is integral over \(\ZZ\) if and only if \(2a, a^2-b^2d \in \ZZ\).
        By some elementary arguments, this is equivalent to \(u \in \rqi{\sqrt{d}}\).
    \end{itemize}
\end{Example}

\end{myproof}

\end{document}
