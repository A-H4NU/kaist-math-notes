\documentclass[../probability.tex]{subfiles}

\begin{document}

\section{Expectation}

\begin{Definition}{Expectation of Discrete Random Variable}[expDRV]
    Let \(X\) be a random element taking its values in \(E\),
    and let \(f \colon E \to \RR\) be a function such that
    \begin{equation}\label{eq:expAssum}
        \sum_{x \in E} |f(x)| p(x) < \infty\text.
    \end{equation}
    One then defines the \emph{expectation} of \(f(X)\), denoted
    \(\mbb{E}[f(X)]\), by
    \[
        \mbb{E}[f(X)] \coloneqq \sum_{x \in E} f(x) p(x)\text.
    \]
\end{Definition}

\nt{%
    If \eqref{eq:expAssum} is satisfied, \(\mbb{E}[f(X)]\) is well-defined
    and finite.
    If \eqref{eq:expAssum} is not satisfied and \(f\) is nonnegative,
    then \(\mbb{E}[f(X)]\) is well-defined but can be infinite.
    Otherwise, \(\mbb{E}[f(X)]\) may not be well-defined.
}

\begin{Exercise}{}[poissonE]
    Let \(X\) be a \hyperlink{exmp:poisson}{Poisson random variable} with parameter \(\lambda\).
    We have
    \[
        \mbb{E}[X] = \lambda\text
        \qquad\text{and}\qquad
        \mbb{E}[X^2] = \lambda^2 + \lambda\text.
    \]
\end{Exercise}
\begin{solution}
    \begin{align*}
        \mbb{E}[X]
        &= \sum_{k=0}^\infty k \frac{\lambda^k}{k!} e^{-\lambda}
        = \lambda e^{-\lambda}\sum_{k=1}^\infty \frac{\lambda^{k-1}}{(k-1)!} = \lambda \\
        \mbb{E}[X^2]
        &= \sum_{k=0}^\infty k^2 \frac{\lambda^k}{k!} e^{-\lambda} \\
        &= \lambda\sum_{k=1}^\infty (k-1) \frac{\lambda^{k-1}}{(k-1)!} e^{-\lambda}
        + \lambda\sum_{k=1}^\infty \frac{\lambda^{k-1}}{(k-1)!} e^{-\lambda} \\
        &= \lambda\mbb{E}[X] + \lambda = \lambda^2 + \lambda\text.
        \qedhere
    \end{align*}
\end{solution}

\nt{%
    \Cref{dfn:expDRV} easily extends to \(f \colon E \to \CC\)
    with the same condition. Writing \(f = g + ih\),
    \eqref{eq:expAssum} is equivalent to
    \[
        \sum_{x \in E} |g(x)|p(x) < \infty\quad\text{and}\quad
        \sum_{x \in E} |h(x)|p(x) < \infty\text.
    \]
}

\nt{%
    Some properties of expectation:
    \begin{itemize}
        \ii
        \emph{Linearity}.
        \(\mbb{E}[\lambda_1 f_1(X) + \lambda_2 f_2(X)] = \lambda_1 \mbb{E}[f_1(X)] + \lambda_2
        \mbb{E}[f_2(X)]\).
        \ii
        \emph{Monotonicity}.
        If \(\fall x \in E,\:f_1(x) \le f_2(x)\), then
        \(\mbb{E}[f_1(X)] \le \mbb{E}[f_2(X)]\).
        \ii
        \(|\mbb{E}[f(X)]| \le \mbb{E}[|f(X)|]\).
        \ii
        Let \(C \subseteq E\) and let \(I_C\) be the \emph{indicator function}
        of \(C\) defined by
        \[
            I_C(x) \coloneqq \begin{cases}
                1 & \text{if }x \in C \\
                0 & \text{otherwise.}
            \end{cases}
        \]
        Then, \(\mbb{E}[I_C(X)] = \sum_{x \in E} I_C(x)p(x) = \sum_{x \in C}p(x)
        = \sum_{x \in C}P(X = x) = P \bigl( \bigcup_{x \in C} \{\,X=x\,\} \bigr)\).

        \ii
        Let \((\Omega, \mcal{F}, P)\) be a probability space and let \(A \in \mcal{F}\).
        Defining the indicator function \(I_A \colon \Omega \to \{0,1\}\) for \(A\),
        \(I_A\) is clearly a discrete random variable taking values on \(\{0,1\}\).
        We have \(\mbb{E}[I_A] = P(A)\).
    \end{itemize}
}

\begin{Theorem}{Markov's Inequality}[markov]
    Let \(f \colon E \to \RR\) satisfy \eqref{eq:expAssum}.
    Then, for \(a > 0\), we have
    \[
        P(|f(X)| \ge a) \le \frac{\mbb{E}[|f(X)|]}{a}\text.
    \]
\end{Theorem}
\begin{myproof}[Proof]
    Let \(C \coloneqq \{\,x \in E \mid |f(x)| \ge a\,\} \subseteq E\).
    Then, \(|f(x)| \ge |f(x)| I_C(x)\) and thus
    \begin{align*}
        \mbb{E}[|f(X)|]
        &\ge \mbb{E}[|f(X)| I_C(X)] \\
        &\ge \mbb{E}[aI_C(X)] \\
        &= a\mbb{E}[I_C(X)] = aP(|f(X)| \ge a)\text. \qedhere
    \end{align*}
\end{myproof}

\end{document}
