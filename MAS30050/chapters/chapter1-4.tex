\documentclass[../probability.tex]{subfiles}

\begin{document}

\section{Counting and Probability}

If \(\Omega\) is finite and we let \(p(\omega) \coloneqq P(\{\omega\})\)
with equal probabilities, then we must have
\(P(A) = (\card A)/(\card \Omega)\) for all \(A \subseteq \Omega\).
Hence, we should \emph{count}.

\begin{Example}{}[]
\begin{itemize}
    \ii
    The number of injections from \(E\) to \(F\)
    with \(p = \card(E)\) and \(n = \card(F)\) when \(p \le n\) is
    \(A_p^n = \frac{n!}{(n-p)!}\).
    \ii
    In particular, if \(p = n\), we have \(A_n^n\),
    the number of permutations of \(n\) elements, which is \(n!\).
    \ii
    The number of subsets of \(F\) with \(p\) elements is
    \(\binom{n}{p} = \frac{n!}{p!(n-p)!}\).
    \ii (Binomial formula)
    \((x+y)^n = \sum_{p=0}^n x^p y^{n-p}\).
    \(2^n = \sum_{p=0}^n \binom{n}{p}\).
    \ii \(\binom{n}{p} = \binom{n}{n-p}\).
    \ii (Pascal's formula) \(\binom{n}{p} = \binom{n-1}{p-1} + \binom{n-1}{p}\).
\end{itemize}
\end{Example}

\end{document}


