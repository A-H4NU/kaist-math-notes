\documentclass[../probability.tex]{subfiles}

\begin{document}

\section{Convergence in Law}

\begin{Definition}{Convergence in Law}[convLaw]
    Let \(\lang X_n \rang_{n \in \ZZ_{>0}}\)
    and \(X\) be real random variables with respective cumulative distribution functions
    \(\lang F_{X_n} \rang_{n \in \ZZ_{>0}}\) and \(F_X\).
    One says that \(\lang X_n \rang_{n \in \ZZ_{>0}}\) \emph{converges in law to \(X\)}
    if
    \begin{equation}\label{eq:convLaw}
        \fall x \in \RR,\: \left( \lim_{a \to x^-} F(a) = F(x) \implies
        \lim_{n \to \infty} F_{X_n}(x) = F_X(x) \right)\text.
    \end{equation}
    This is denoted by \(X_n \lawconv X\).
\end{Definition}

\begin{note}
    In the definition of convergence in law,
    the discontinuity points of the cumulative distribution function
    do not play a spacial part. If \eqref{eq:convLaw} were
    required to hold without the premise \(\lim\limits_{a \to x^-} F(a) = F(x)\),
    then defining \(X_n \equiv a + \frac{1}{n}\) and \(X \equiv a\),
    we could not say that \(X_n \lawconv X\) because \(P(X_n \le a) = 0\)
    does not converge toward \(P(X \le a) = 1\).
\end{note}

\begin{Exercise}{}[]
    Let \(\lang X_n \rang_{n \in \ZZ_{>0}}\) be a sequence of independent random variables such that
    \(Z_n \sim U([0,1])\). Define
    \[
        Z_n \coloneqq \min \{X_1, \ldots, X_n\}\text.
    \]
    Show that \(n Z_n \lawconv X\) where \(X \sim \mcal{E}(1)\).
\end{Exercise}
\begin{solution}
    For \(x \in \RR\), we have
    \[
        P( n Z_n \le x ) =
        \begin{cases}
            0 & \text{if } x < 0 \\
            1-\left(1-\dfrac{x}{n}\right)^{n} & \text{if } 0 \le x \le n \\
            1 & \text{otherwise.}
        \end{cases}
    \]
    Therefore, for every \(x \in \RR_{\ge 0}\),
    \[
        \lim_{n \to \infty} P(n Z_n \le x)
        = \lim_{n \to \infty} \left(1-\left(1-\frac{x}{n}\right)^{n}\right) = 1 - e^{-x}\text,
    \]
    which is the cumulative distribution function of \(\mcal{E}(1)\).
    \qed
\end{solution}

\begin{Theorem}{Characteristic Function Criterion}[charFunCrit]
    Let \(\lang X_n \rang_{n \in \ZZ_{>0}}\)
    be real random variables with respective characteristic distribution functions
    \(\lang \phi_{X_n} \rang_{n \in \ZZ_{>0}}\).
    If the sequence \(\lang \phi_{X_n} \rang_{n \in \ZZ_{>0}}\) converges
    pointwise to some function \(\phi \colon \RR \to \CC\)
    that is continuous at \(0\), then \(\phi\) is a characteristic function
    of some real random variable \(X\), and moreover, \(X_n \lawconv X\).
\end{Theorem}

\end{document}
