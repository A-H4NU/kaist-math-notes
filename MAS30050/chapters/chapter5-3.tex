\documentclass[../probability.tex]{subfiles}

\begin{document}

\section{Classification of States}

\begin{Notation}{}[]
    For any event \(A\) and state \(x\), we introduce the following notation:
    \[
        P_x(A) \triangleq P(A \mid X_0 = x)
    \]
    Expectations under this probability measure are denoted by \(\mbb{E}_x\).
    These simply mean that, when computing \(P_x\) or \(\mbb{E}_x\), we assume that
    the associated Markov chain starts from \(x\).
\end{Notation}

\begin{Definition}{Time of the First Jump}[tfj]
    Let \(T_y\) be the random variable of the \emph{time of the first jump} to state \(y\):
    \[
        T_y \triangleq \min \{\,n \ge 1 \mid X_n = y\,\}
    \]
    Note that, if the chain starts from \(y\), the time zero does \emph{not} count as a visit.
\end{Definition}

\begin{Definition}{Stopping Time}[]
    Assume \(\lang X_n \rang_{n \in \ZZ_{\ge 0}}\) is a Markov chain.
    Given a (extended) random variable \(T\), with values in the set of time indices \(\{0,1,\cdots,\infty\}\),
    \(T\) is called a \emph{stopping time} (with respect to \(\lang X_n \rang_{n \in \ZZ_{\ge 0}}\))
    if the occurrence or non-occurrence of the event \(\{T = n\}\) can be determined
    only by looking at the values \(X_0, \cdots, X_n\).
\end{Definition}

\begin{Definition}{Ever Jumping}[everJump]
    Denote the \emph{probability of ever jumping to \(y\), starting from \(x\)}, by \(\rho_{xy}\):
    \[
        \rho_{xy} \triangleq P_x(T_y < \infty)\text.
    \]
\end{Definition}

\begin{Theorem}{Strong Markov Property of a Markov Chain}[strongMarkov]
    Assume \(\lang X_n \rang_{n \in \ZZ_{\ge 0}}\) is a Markov chain and \(T\)
    is a stopping time. Conditional on \(T < \infty\) and \(X_T = y\),
    any other information about \(X_0, \cdots, X_{T-1}\) is irrelevant for the future
    distribution of the Markov chain.
    Namely, the new process \(\lang \tilde{X}_n \triangleq X_{T+n} \rang_{n \in \ZZ_{\ge 0}}\)
    is also a Markov chain, with the same transition matrix and with initial state \(y\),
    and it is independent of \(T\) and the past values \((X_0, \cdots, X_{T-1})\).
\end{Theorem}
\begin{myproof}[Proof]
    Skipped.
\end{myproof}

\begin{note}
    If we restrict \Cref{th:strongMarkov}
    by forcing \(T\) to be \emph{deterministic}, then the new property
    becomes a \emph{regular (or, standard) Markov property}.
\end{note}

\begin{Definition}{\(\bm{k}\)-th Jump}[]
    For \(k \ge 1\), we introduce the \emph{time of the \(k\)-th jump} to state \(y\):
    \begin{align*}
        T_y^1 &\triangleq T_y\text, \\
        T_y^{k} &\triangleq \min \bigl\{\,n > T_{y}^{k-1} \bigm| X_n = y\,\bigr\}\text.
    \end{align*}
    Note that they are stopping times.
\end{Definition}

\begin{Lemma}{}[kthProb]
    \(P_x(T_y^k < \infty) = \rho_{xy}\rho_{yy}^{k-1}\).
\end{Lemma}
\begin{myproof}[Proof]
    For each \(k \ge 2\), we have
    \begin{alignat*}{2}
        P_y(T_y^k < \infty)
        &= P_y(T_y^k < \infty \mid T_y^{k-1} < \infty) P_y(T_y^{k-1} < \infty) &\qquad \\
        &= \rho_{yy} \cdot P_y(T_y^{k-1} < \infty)\text. && \comment{\Cref{th:strongMarkov}}
    \end{alignat*}
    Hence, the result follows from mathematical induction.
\end{myproof}

\begin{Definition}{Transient and Recurrent States}[transientRecurrent]
    \begin{itemize}
        \ii
        If \(\rho_{yy} < 1\), then the state \(y\) is called \emph{transient}.
        \ii
        If \(\rho_{yy} = 1\), then the state \(y\) is called \emph{recurrent}.
    \end{itemize}
\end{Definition}

\begin{note}
    By \Cref{lem:kthProb}, we have the following:
    \begin{itemize}
        \ii
        If a state \(y\) is transient, then \(\lim_{k \to \infty} P_y(T_y^k < \infty) = 0\).
        \ii
        If a state \(y\) is recurrent, then \(\lim_{k \to \infty} P_y(T_y^k < \infty) = 1\).
    \end{itemize}
\end{note}

\begin{Definition}{The Number of Visits}[]
    Let \(N_y\) be the random variable of the number of visits to state \(y\), i.e.,
    \[
        N_y \triangleq \sum_{k=1}^\infty \mbf{1}_{\{T_y^k < \infty\}}\text,
    \]
    where \(\mbf{1}_A\) is an indicator of event \(A\).
\end{Definition}

\begin{Lemma}{}[nvisitsTransient]
    If \(y\) is a transient state,
    \[
        \mbb{E}_x[N_y] = \frac{\rho_{xy}}{1-\rho_{yy}}\text.
    \]
\end{Lemma}
\begin{myproof}[Proof]
    \begin{alignat*}{2}
        \mbb{E}_x[N_y]
        &= \mbb{E}_x \left[ \sum_{k=1}^\infty \mbf{1}_{\{T_y^k < \infty\}}\right] &\qquad \\
        &= \sum_{k=1}^\infty \mbb{E}_x \left[ \mbf{1}_{\{T_y^k < \infty\}}\right]
        && \comment{Tonelli's Theorem} \\
        &= \sum_{k=1}^\infty P(T_y^k < \infty) \\
        &= \sum_{k=1}^\infty \rho_{xy}\rho_{yy}^{k-1}
        && \comment{\Cref{lem:kthProb}} \\
        &= \frac{\rho_{xy}}{1-\rho_{yy}}\tag*{\qedhere}
    \end{alignat*}
\end{myproof}

\begin{Lemma}{}[nvisitsRecurrent]
    If \(y\) is a recurrent state, then
    \[
        P_y(N_y = \infty) = 1\text.
    \]
\end{Lemma}
\begin{myproof}[Proof]
    Note that
    \[
        \{N_y = \infty\} = \bigcap_{k=1}^\infty \{T_y^k < \infty\}\text.
    \]
    Hence,
    \begin{alignat*}{2}
        P_y(N_y = \infty)
        &= \lim_{N \to \infty} P_y\left( \bigcap_{k=1}^N \{T_y^k < \infty\} \right)
        &\qquad& \comment{\nameref{th:seqCont}} \\
        &= \lim_{N \to \infty} P_y\bigl(T_y^N < \infty\bigr) \\
        &= \lim_{N \to \infty} \rho_{yy}^N
        && \comment{\Cref{lem:kthProb}} \\
        &= 1\text. \tag*{\qedhere}
    \end{alignat*}
\end{myproof}

\begin{Lemma}{}[recurrIff]
    A state \(y\) is recurrent if and only if \(\mbb{E}_y[N_y] = \infty\).
\end{Lemma}
\begin{myproof}[Proof]
    Combine \Cref{lem:nvisitsTransient,lem:nvisitsRecurrent}.
\end{myproof}

\begin{Definition}{Communicating States}[communStates]
    We say that \(x\) \emph{communicates with} \(y\), and denote it by \(x \to y\),
    if
    \[
        \rho_{xy} = P_x(T_y < \infty) > 0\text.
    \]
\end{Definition}

\begin{Lemma}{}[communIff]
    \(x\) communicates with \(y\) if and only if
    there is some \(m \in \ZZ_{>0}\) such that \(p^m(x, y) > 0\).
\end{Lemma}
\begin{myproof}[Proof]\hfill
\begin{pftfae}[labelwidth=\widthof{(\(\Rightarrow\))}]
    \ii[(\(\Rightarrow\))]
    As \(P_x(T_x < \infty) = \sum_{k=1}^\infty P_x(T_y = k) > 0\),
    there is some \(m \in \ZZ_{>0}\) such that \(P_x(T_y = k) > 0\).
    Such \(m\) satisfies \(p^m(x,y) \ge P_x(T_y = k) > 0\).
    \ii[(\(\Leftarrow\))]
    Trivial.
    \qedhere
\end{pftfae}
\end{myproof}

\begin{Lemma}{}[communTransitive]
    If \(x \to y\) and \(y \to z\), then \(x \to z\).
\end{Lemma}
\begin{myproof}[Proof]
    By \Cref{lem:communIff}, there are \(m_1, m_2 \in \ZZ_{>0}\)
    such that \(p^{m_1}(x, y) > 0\) and \(p^{m_2}(y, z) > 0\).
    Hence, we have \(p^{m_1+m_2}(x,z) \ge p^{m_1}(x, y) \cdot p^{m_2}(y, z) > 0\).
    The result follows from \Cref{lem:communIff}.
\end{myproof}

\begin{Lemma}{}[betrayTransient]
    If \(x \to y\) and \(\rho_{yx} < 1\), then \(x\) is a transient state.
\end{Lemma}
\begin{myproof}[Proof]
    Let \(K \triangleq \{\,k \in \ZZ_{>0} \mid p^k(x, y) > 0\,\}\).
    There is a sequence \(y_1, \cdots, y_{K-1}\) of states so that
    \[
        p(x, y_1) p(y_1, y_2) \cdots p(y_{K-2}, y_{K-1}) p(y_{K-1}, y) > 0
    \]
    Then, we have
    \[
        P_x(T_x = \infty)
        \ge p(x, y_1) p(y_1, y_2) \cdots p(y_{K-2}, y_{K-1}) p(y_{K-1}, y) \cdot (1 - \rho_{yx}) > 0
    \]
    so that \(x\) is transient.
\end{myproof}

\begin{Lemma}{}[recurrTransThenReach]
    If \(x\) is recurrent and \(x \to y\), then \(\rho_{yx} = 1\).
\end{Lemma}
\begin{myproof}[Proof]
    Direct consequence of \Cref{lem:betrayTransient}.
\end{myproof}

\begin{Lemma}{}[recurrIff2]
    \(\mbb{E}_x[N_y] = \sum_{n=1}^\infty p^n(x, y)\).
    Moreover, \(y\) is recurrent if and only if \(\sum_{n=1}^\infty p^n(y,y) = \infty\).
\end{Lemma}
\begin{myproof}[Proof]
    Note that \(N_y = \sum_{n=1}^\infty \mbf{1}_{\{X_n = y\}}\).
    Hence, by the same argument as in the proof of \Cref{lem:nvisitsRecurrent},
    \(\mbb{E}_x[N_y] = \sum_{n=1}^\infty p^n(x, y)\).
    The result follows from \Cref{lem:recurrIff}.
\end{myproof}

\begin{Lemma}{}[recurrSpread]
    If \(x\) is recurrent and \(x \to y\), then \(y\) is recurrent.
\end{Lemma}
\begin{myproof}[Proof]
    By \Cref{lem:recurrTransThenReach}, \(\rho_{yx} = 1\).
    By \Cref{lem:communIff}, there are some \(m_1, m_2 \in \ZZ_{>0}\)
    such that \(p^{m_1}(x, y) > 0\) and \(p^{m_2}(y, x) > 0\).
    Then, for \(n \ge m_1 + m_2\), we have
    \[
        p^{n}(y, y) \ge p^{m_2}(y, x) \cdot p^{n-m_1-m_2}(x, x) \cdot p^{m_1}(x, y)\text.
    \]
    Hence, by \Cref{lem:recurrIff2}, \(y\) is recurrent.
\end{myproof}

\begin{Definition}{Closed Set}[closedSet]
    A nonempty set \(A\) of states is \emph{closed} if
    \[
        \fall i \in A,\: \fall j \in \mbb{S} \setminus A,\:
        p(i, j) = 0\text.
    \]
\end{Definition}

\begin{Lemma}{}[finClosedHasRecurr]
    If \(A\) is a finite closed set,
    then \(A\) has at least one recurrent state.
\end{Lemma}
\begin{myproof}[Proof]
    Suppose all states in \(A\) are transient for the sake of contradiction.
    Fix any state \(x \in A\).
    Then,
    \begin{alignat*}{2}
        \infty
        &> \sum_{y \in A} \mbb{E}_x[N_y]
        &\qquad& \comment{\Cref{lem:nvisitsTransient}} \\
        &= \sum_{y \in A} \sum_{n = 1}^\infty p^n(x, y)
        && \comment{\Cref{lem:recurrIff2}} \\
        &= \sum_{n=1}^\infty \sum_{y \in A} p^n(x, y) \\
        &= \sum_{n=1}^\infty 1
        && \comment{\(A\) is closed} \\
        &= \infty\text,
    \end{alignat*}
    which is a contradiction.
\end{myproof}

\begin{Definition}{Irreducible Set}[irreducibleSet]
    A nonempty set \(A\) of states is \emph{irreducible} if
    \[
        \fall x \in A,\: \fall y \in A,\: x \to y\text.
    \]
\end{Definition}

\begin{Theorem}{}[finClosedIrrAllRecurr]
    All states in a finite closed irreducible set is recurrent.
\end{Theorem}
\begin{myproof}[Proof]
    Combine \Cref{lem:recurrSpread,lem:finClosedHasRecurr}\text.
\end{myproof}

\begin{Theorem}{Decomposition Theorem}[decomposition]
    If the state space \(\mbb{S}\) of a Markov chain is finite, then
    \[
        \mbb{S} = T \uplus R_1 \uplus \cdots \uplus R_k
    \]
    where \(T\) is the set of transient states,
    and each \(R_i\)'s are closed irreducible sets of recurrent states.
\end{Theorem}
\begin{myproof}[Proof]
    The relation \(x \sim y\) defined by \(x \to y\) and \(y \to x\) is an equivalence relation
    on \(\mbb{S} \setminus T\) by \Cref{lem:communTransitive}.
    Let \(R\) be an equivalence class of \(\mbb{S} \setminus T\) under \(\sim\).
    Then, it is closed and irreducible.
\end{myproof}

\end{document}
