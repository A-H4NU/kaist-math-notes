\documentclass[../probability.tex]{subfiles}

\begin{document}

\section{Univariate Probability Densities}

Recall \Cref{dfn:pdf}.

% TODO: Some examples of probability densities.

\begin{Example}{Uniform Density}[uniform]
    A random variable \(X\) with the probability density
    \[
        f(x) = \begin{cases}
            \dfrac{1}{b-a} & \text{if } a \le x \le b \\
            0 & \text{otherwise}
        \end{cases}
    \]
    is said to be \emph{uniformly distributed} on \([a, b]\).
    This is denoted by \(X \sim U([a, b])\).
\end{Example}

\begin{Example}{Exponential Density}[exponential]
    For \(\lambda \in \RR_{>0}\), the random variable \(X\) with the probability density
    \[
        f(x) = \begin{cases}
            \lambda e^{-\lambda x} & \text{if } x \ge 0 \\
            0 & \text{otherwise}
        \end{cases}
    \]
    is called an \emph{exponential random variable}.
    This is denoted by \(X \sim \mcal{E}(\lambda)\).
\end{Example}

\begin{Example}{Gaussian Density}[gaussian]
    For \(\mu \in \RR\) and \(\sigma \in \RR_{>0}\), the random variable \(X\) with the probability
    density
    \[
        f(x) = \frac{1}{\sqrt{2\pi}\sigma} e^{-\frac{1}{2} \left(\frac{x-m}{\sigma}\right)^2}
    \]
    is called a \emph{Gaussian random variable}.
    This is denoted by \(X \sim \mcal{N}(m, \sigma^2)\).
    When \(X \sim \mcal{N}(0, 1)\), we say that \(X\) is a \emph{standard Gaussian random variable}.
\end{Example}

\begin{Example}{Gamma Density}[gamma]
    Let \(\alpha, \beta \in \RR_{>0}\).
    The random variable \(X\) with the probability density
    \[
        f(x) = \begin{cases}
            \dfrac{\beta^\alpha x^{\alpha-1}e^{-\beta x}}{\Gamma(\alpha)}
            &\text{if }x > 0 \\
            0 & \text{otherwise}
        \end{cases}
    \]
    is called a \emph{gamma distributed random variable}
    where
    \[
        \Gamma(\alpha) = \int_0^\infty x^{\alpha-1} e^{-x} \d x\text.
    \]
    This is denoted by \(X \sim \Gamma(\alpha, \beta)\).
\end{Example}

\begin{note}
    When \(\alpha = 1\), the gamma distribution is simply the exponential distribution:
    \[
        \Gamma(1, \beta) = \mcal{E}(\beta)\text.
    \]
    When \(\alpha = n/2\) and \(\beta = 1/2\), the corresponding distribution
    is called the \emph{chi-squared distribution} with \(n\) degrees of freedom.
    When \(X\) admits this density, we denote this by
    \[
        X \sim \chi_n^2\text.
    \]
\end{note}

\end{document}
