\documentclass[../probability.tex]{subfiles}

\begin{document}

\section{Discrete Random Elements}

\begin{Definition}{Discrete Random Element}[dre2]
    Let \(E\) be a denumerable set and let \((\Omega, \mcal{F}, P)\) be a
    probability space. Any function \(X \colon \Omega \to E\) such that
    \[
        \fall x \in E,\: \{\,\omega \mid X(\omega) = x\,\} \in \mcal{F}
    \]
    is called a \emph{discrete random element} of \(E\).
    When \(E \subseteq \RR\), we refer to \(X\) as a \emph{discrete random variable}.
    This allows us to define
    \[
        p(x) \coloneqq P(X = x)
    \]
    for \(x \in E\).
    The collection \(\{p(x)\}_{x \in E}\)
    is the \emph{distribution} of \(X\). It satisfies
    \[
        0 \le p(x) \le 1\quad\text{and}\quad
        \sum_{x \in E} p(x) = 1\text.
    \]
\end{Definition}

\nt{%
    \(E\) being denumerable enables us to define in such way.
    Note the difference from \Cref{dfn:rv}.
}

\begin{Example}{Bernoulli Distribution}[bernoulli]
    The coin tossing experiment of a single coin with bias \(p\) (\(0 \le p \le 1\))
    is described by a discrete random variable \(X\)
    taking its values in \(E = \{0,1\}\) with the distribution
    \[
        P(X = 1) = p,\qquad P(X = 0) = 1-p\text.
    \]
    This is called the \emph{Bernoulli distribution} of parameter \(p\).
\end{Example}

\begin{Example}{Binomial Distribution}[binomial]
    Let \(X_1, \cdots, X_n\) be \(n\) indepenent random variables
    with the Bernoulli distribution of parameter \(p\).
    The distribution of a discrete random variable \(S_n = \sum_{i=1}^n X_i\)
    satisfies
    \[
        P(S_n = k) = \frac{n!}{k!(n-k)!} p^k (1-p)^{n-k}
    \]
    for \(0 \le k \le n\).
    This is called the \emph{binomial distribution} of size \(n\) and parameter \(p\).
\end{Example}

\begin{Example}{Geometric Distribution}[geometric]
    Let \(\lang X_n \rang_{n \in \ZZ_{>0}}\)
    be a sequence of independent random variables with the Bernoulli distribution of parameter
    \(p\).
    Let \(T\) be a random element such that
    \[
        T = \begin{cases}
            \min \{\,n \mid X_n = 1\,\} & \text{if }\{\,n \mid X_n = 1\,\} \neq \OO \\
            +\infty & \text{otherwise.}
        \end{cases}
    \]
    Then, we have
    \[
        P(T = k) = p(1-p)^{k-1}
    \]
    for \(k \ge 1\) and \(P(T = \infty) = 0 \text{ or } 1\) according to whether \(p > 0\)
    or \(p = 0\).
    We call \(T\) a \emph{geometry random variable} of paramter \(p\).
    This is symbolized by \(T \sim \mcal{G}(p)\).
\end{Example}

\begin{Example}{Multinomial Distribution}[multinomial]
    Suppose you have \(k\) boxes in which you place \(n\) balls at random in the following manner.
    The balls are thrown into the boxes independently of one another, and the probability
    that a given ball falls in a box \(i\) is \(p_i\). Of course, \(0 \le p_i \le k\) and
    \(\sum_{i=1}^k p_i = 1\).
    Let \(N_i\) (\(1 \le i \le k\)) denote the number of balls that fall into box \(i\).
    The random vector \(N = (N_1, \cdots, N_k)\) takes its values in the \(k\)-tuples
    of integers \((n_1, \cdots, n_k)\) satisfying
    \[
        n_1 + \cdots + n_k = n\text.
    \]
    The probability that \(N_i = n_i\) for all \(i\) is given by
    \[
        P(N_1 = n_1, \cdots, N_k = n_k)
        = \frac{n!}{n_1! \cdots n_k!} p_1^{n_1} \cdots p_k^{n_k}\text,
    \]
    where \(n_1 + \cdots + n_k = n\).
    This type of distribution is called the \emph{multinomial distribution} of size \((n, k)\)
    and of parameters \((p_1, \cdots, p_k)\). Notation \((N_1, \cdots, N_k) \sim \mcal{M}(n, k, p_i)\)
    expresses that \((N_1, \cdots, N_k)\) is a multinomial random variable.
\end{Example}

\begin{Example}{Poisson Distribution}[poisson]
    A random variable \(X\) that takes its values in \(E = \ZZ_{\ge 0}\) and
    admits the distribution
    \[
        P(X=k) = e^{-\lambda} \frac{\lambda^k}{k!}
    \]
    for \(k \ge 0\), where \(\lambda\) is a nonnegative real number,
    is called a \emph{Poisson random variable} with parameter \(\lambda\).
    This is denoted by \(X \sim \Poisson(\lambda)\).
\end{Example}

\end{document}
