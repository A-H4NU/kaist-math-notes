\documentclass[../probability.tex]{subfiles}

\begin{document}

\section{Characteristic Function of a Random Variable}

\begin{Definition}{Characteristic Function}[charFun]
    Let \(X\) be a real random variable with the probability density function \(f_X\).
    The \emph{characteristic function} \(\phi_X \colon \RR \to \CC\) of \(X\) is defined as
    \[
        \phi_X(u) \triangleq \mbb{E}[e^{iuX}] = \int_{-\infty}^{\infty} e^{iux} f(x) \,dx\text.
    \]
\end{Definition}

\begin{note}
\begin{itemize}
    \ii
    \Cref{dfn:charFun} is well-defined
    as \(\cos\) and \(\sin\) are bounded.
    \ii
    \(\phi_{aX+b}(u) = \mbb{E}[e^{iuaX}e^{iub}]=e^{iub}\phi_X(au)\)
    for any real numbers \(a\) and \(b\).
    \ii
    If two real random variables \(X\) and \(Y\) satisfy
    \(\mbb{E}[e^{iuX}] = \mbb{E}[e^{iuY}]\) for all \(u \in \RR\),
    then \(P(X \le x) = P(Y \le x)\) for all \(x \in \RR\).
    Hence, the characteristic function uniquely determines the distribution of a random variable.

    \ii
    It should be emphasized that two random variables with the same distribution function
    are not necessarily identical random variables.
    For instance, take \(X \sim \mcal{N}(0,1)\) and \(Y = -X\).
\end{itemize}
\end{note}

% TODO: Exercises from the lecture note.

\end{document}
