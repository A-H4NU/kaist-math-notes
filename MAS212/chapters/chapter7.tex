\documentclass[MAS212_Note.tex]{subfiles}
\begin{document}

\section{Cyclic Subspaces and Annihilators}

\dfn{\(T\)-cyclically Generated Subspace}{
    Let \(V\) be a finite-dimensional vector space over \(F\) and let \(T \in L(V)\).
    For \(\alpha \in V\), the subspace
    \[
        Z(\alpha; T) = \{\,g(T)\alpha \mid g \in F[x]\,\}
    \]
    of \(V\) is called the \textit{\(T\)-cyclic subspace generated by \(\alpha\)}.
    If \(Z(\alpha; T) = V\), then we say \(V\) is \textit{cyclically generated by} \(\alpha\),
    and \(\alpha\) is a \textit{cyclic vector for} \(T\).
}

\nt{
    \noindent Some immediate facts:
    \begin{itemize}[nolistsep]
        \ii \(Z(\alpha; T)\) is \(T\)-invariant.
        \ii \(Z(0; T) = \{0\}\).
        \ii If \(\alpha \neq 0\) is an eigenvector,
            then \(Z(\alpha; T) = \spn \{\alpha\}\).
        \ii If \(\dim Z(\alpha; T) = 1\), then \(\alpha \neq 0\) and \(Z(\alpha; T) = \spn \{\alpha\}\);
            thus \(\alpha\) is an eigenvector.
    \end{itemize}
    So, we need \(\alpha\) be neither too bad nor too good to utilize \(Z(\alpha; T)\).
}

\dfn{\(T\)-annihilator}{
    Let \(V\) be a finite-dimensional vector space over \(F\) and let \(T \in L(V)\).
    For \(\alpha \in V\), the \textit{\(T\)-annihilator of} \(\alpha\) is the subspace
    \[
        M(\alpha; T) \triangleq \{\,g \in F[x] \mid g(T)\alpha = 0\,\}.
    \]
    In other words, \(M(\alpha; T) = S_T(\alpha; \{0\})\).
}
\nt{
    \(T\)-annihilator of \(\alpha\) is the \(T\)-conductor of \(\alpha\) to \(\{0\}\),
    \(M(\alpha; T)\) is a nonzero ideal and thus has a unique monic generator \(p_{\alpha}\).
    \(p_{\alpha}\) is also called the \(T\)-annihilator of \(\alpha\)
    Hence, as the minimal polynomial \(m\) of \(T\) resides in \(M(\alpha; T)\),
    we have \(p_{\alpha} \mid m\).
}

\thm[cycleAndAnnihilator]{}{
    Let \(V\) be a finite-dimensional vector space over \(F\) and let \(T \in L(V)\).
    Let \(\alpha \in V \setminus \{0\}\) be fixed.
    Let \(p_{\alpha}\) be the \(T\)-annihilator of \(\alpha\).
    \begin{enumerate}[nolistsep, label=(\roman*)]
        \ii If \(k = \deg p_{\alpha}\),
            \(\{\alpha, T \alpha, \cdots, T^{k-1}\alpha\}\)
            is a basis for \(Z(\alpha; T)\), hence \(\deg p_{\alpha} = \dim Z(\alpha; T)\).
        \ii Let \(U \triangleq T \big|_{Z(\alpha; T)} \in L(Z(\alpha; T))\). 
            Then, the minimal polynomial of \(U\) is \(p_\alpha\).
    \end{enumerate}
}
\pf{Proof}{
\hfill
\begin{enumerate}[nolistsep, label=(\roman*), listparindent=\parindent]
    \ii 
    Let \(g \in F[x]\) be arbitrary.
    By \Cref{th:euclid}, we have \(g = p_{\alpha}q + r\) where \(q, r \in F[x]\)
    in which either \(r = 0\) or \(r \neq 0\) and \(\deg r < \deg p_{\alpha}\).
    As \((p_{\alpha}) = M(\alpha; T)\), we also have \(p_{\alpha} q \in M(\alpha; T)\),
    and thus
    \[
        g(T) \alpha = q(T) p_{\alpha}(T) \alpha + r(T) \alpha = r(T) \alpha.
    \]
    Hence, \(Z(\alpha; T) = \spn \{\alpha, T \alpha, \cdots, T^{k-1} \alpha\}\).
    We are left with proving that they are linearly independent.

    Suppose they are not linearly independent for the sake of contradiction.
    Then there exist \(c_0, \cdots, c_{k-1} \in F\) not all zero such that
    \(\big(\sum_{i=0}^{k-1} c_i T^i\big) \alpha = 0\),
    which means \(g_0(x) = \sum_{i=0}^{k-1} c_i x^i \in M(\alpha; T)\)
    with \(\deg g_0 < \deg p_{\alpha}\), violating the minimality of \(p_{\alpha}\).
    Hence, they are linearly independent.

    \ii
    Take any \(v \in Z(\alpha; T)\). Then, there exists \(g \in F[x]\)
    so \(v = g(T) \alpha\).
    Then, \(p_{\alpha}(U) v = g(T) p_{\alpha}(T) \alpha = 0\).
    Hence, \(p_{\alpha}(U) = 0\).

    Moreover, there does not exist \(q \in F[x]\) with \(q(U) = 0\)
    by the definition of \(p_{\alpha}\).
    Hence, the result follows.
\end{enumerate}
}

\nt{
    With respect to the ordered basis \(\mcal B = \{\alpha; T \alpha; \cdots; T^{k-1}\alpha\}\)
    for \(Z(\alpha; T)\). Then,
    \[
        [U]_{\mcal B} = \begin{bmatrix}
            0 & 0 & \cdots & 0 & -c_0 \\
            1 & 0 & \cdots & 0 & -c_1 \\
            0 & 1 & \cdots & 0 & -c_2 \\
            \vdots & \vdots & \ddots & \vdots & \vdots \\
            0 & 0 & \cdots & 1 & -c_{k-1} \\
        \end{bmatrix} 
    \]
    where \(p_{\alpha}(x) = \sum_{i=0}^{k-1} c_ix^i + x^k\).
}

\dfn{Companion Matrix}{
    The matrix \([U]_{\mcal B}\) above is called
    the \textit{companion matrix} of \(p_{\alpha}\).
}

\section{Cyclic Decompositions and the Rational Form}

\dfn{Complementary \(T\)-invariant Subspace}{
    Let \(V\) be a finite-dimensional vector space over \(F\) and let \(T \in L(V)\).
    Let \(W\) be a \(T\)-invariant subspace of \(V\).
    If \(W'\) is a \(T\)-invariant subspace of \(V\) such that
    \(V = W \oplus W'\), we call it a \textit{complementary \(T\)-invariant
    subspace} of \(W\).
}

\dfn{\(T\)-admissible Subspace}{
    Let \(V\) be a finite-dimensional vector space over \(F\) and let \(T \in L(V)\).
    We say a subspace \(W\) of \(V\) is \textit{\(T\)-admissible} if
    \begin{enumerate}[nolistsep, label=(\roman*)]
        \ii \(W\) is \(T\)-invariant and
        \ii \(\fall f \in F[x],\: \fall \beta \in V,\: \big( f(T) \beta \in W
            \implies \exs \gamma \in W,\: f(T)\beta = f(T) \gamma\big)\).
    \end{enumerate}
}

\mlemma[compTinvarThenAdmiss]{}{
    Let \(V\) be a finite-dimensional vector space over \(F\) and let \(T \in L(V)\).
    Suppose \(W\) and \(W'\) are \(T\)-invariant subspaces such that
    \(V = W \oplus W'\).
    Then, \(W\) and \(W'\) are \(T\)-admissible.
}
\pf{Proof}{
    The condition (i) is already true.
    Suppose \(f(T) \beta \in W\) where \(f \in F[x]\) and \(\beta \in V\).
    We can write \(\beta = \gamma + \gamma'\) where \(\gamma \in W\) and \(\gamma' \in W'\).
    Then, \(f(T) \beta = f(T) \gamma + f(T) \gamma'\).
    As \(W\) and \(W'\) are \(T\)-invariant, we have
    \(f(T)\beta - f(T) \gamma = f(T) \gamma' \in W \cap W'\).
    Hence, \(f(T) \beta = f(T) \gamma\).
}

\notat{}{
    Let \(V\) be a finite-dimensional vector space over \(F\) and let \(T \in L(V)\).
    If \(T\) is the only subjective linear transform, we may write
    \begin{itemize}[nolistsep]
        \ii \(f \alpha\) instead of \(f(T)(\alpha)\) for each \(f \in F[x]\) and \(\alpha \in V\).
        \ii \(f W\) instead of \(f(T)(W)\) for each \(f \in F[x]\) and \(W \subseteq V\).
    \end{itemize}
}

\mlemma[cycDecompLem]{}{
    Let \(V\) be a finite-dimensional vector space over \(F\) and let \(T \in L(V)\).
    Then, the following hold.
    \begin{enumerate}[nolistsep, label=(\roman*)]
        \ii \(fZ(\alpha; T) = Z(f \alpha; T)\) for each \(\alpha \in V\) and \(f \in F[x]\).
        \ii If \(V = \oplus_{i=1}^k W_i\) where each \(W_i\) is \(T\)-invariant,
            then \(fV = \oplus_{i=1}^k fW_i\).
        \ii For \(\alpha, \gamma \in V\), if \(\alpha\) and \(\gamma\) have the same \(T\)-annihilator,
            then \(f \alpha = f \gamma\) has the same \(T\)-annihilator.
            Therefore, \(\dim Z(f \alpha; T) = \dim Z(f \gamma; T)\).
    \end{enumerate}
}
\mclm{Proof}{
    \hfill
    \begin{enumerate}[nolistsep, label=(\roman*)]
        \ii \(fZ(\alpha; T) = \{\,f g \alpha \mid g \in F[x]\,\} = \{\,g f \alpha \mid g \in F[x]\,\} = Z(f \alpha; T)\)
        \ii It is evident that \(fV = \sum_{i=1}^k fW_i\).
            Suppose \(\sum_{i=1}^k f \alpha_i = 0\) for some \(\alpha_i \in W_i\).
            As \(f \alpha_i \in V_i\), we have \(f \alpha_i = 0\) for all \(i \in [k]\).
            Hence, \(W_1, \cdots, W_k\) are independent.
        \ii We have \(M(\alpha; T) = M(\gamma; T)\), i.e., \(\fall g \in F[x],\: (g \alpha = 0 \iff g \alpha = 0)\).
            Hence, \(M(f \alpha; T) = \{\,g \in F[x] \mid g f \alpha = 0\,\}
            = \{\,g \in F[x] \mid g f \gamma = 0\,\} = M(f \alpha; T)\). \qed
    \end{enumerate}
}

\thm[cycDecomp]{Cyclic Decomposition Theorem}{
    Let \(V\) be a finite-dimensional vector space over \(F\) and let \(T \in L(V)\).
    Let \(W_0\) be a proper \(T\)-admissible subspace of \(V\).
    Then, there exist \(\alpha_1, \cdots, \alpha_r \in V \setminus \{0\}\) 
    such that
    \begin{enumerate}[nolistsep, label=(\roman*)]
        \ii \(V = W_0 \oplus \big(\oplus_{i=1}^r Z(\alpha_i; T)\big)\) and
        \ii \(p_{i+1} \mid p_i\) for each \(i \in [r - 1]\)
    \end{enumerate}
    where \(p_i\) is the \(T\)-annihilator of \(\alpha_i\).
    Furthermore, \(r\) and \(p_{1}, \cdots, p_{r}\) are uniquely decided.
}
\pf{Proof}{
    In this proof we denote the monic generator of \(S_T(\alpha; W)\) as \(s_T(\alpha; W)\) for conciseness.

    \mclm{Claim 0}{
        For \(\alpha, \beta \in V\) and a subspace \(W\) of \(V\),
        if \(\alpha - \beta \in W\), then \(S_T(\alpha; W) = S_T(\beta; W)\).
        Moreover, if \(W\) is \(T\)-invariant,
        then \(W + Z(\alpha; T) = W + Z(\beta; T)\).
    }

    Let \(\gamma \triangleq \alpha - \beta \in W\).
    Then, \(g \in S_T(\alpha; W) \iff g \alpha \in W
    \iff g (\beta + \gamma) \in W \iff g \beta \in W \iff g \in S_T(\beta; W)\).

    Assuming \(W\) is \(T\)-invariant,
    we have, for each \(g \alpha \in Z(\alpha; T)\),
    \(g \alpha = g (\beta + \gamma) \in Z(\beta; T) + W\);
    hence \(Z(\alpha; T) + W \subseteq Z(\beta; T) + W\). \checkmark

    \mclm{Claim 1}{
        For a proper \(T\)-admissible subspace \(W\) of \(V\),
        there exists \(\alpha \in V \setminus W\)
        such that \(s_T(\alpha; W) \alpha = 0\).
    }

    Take any \(\beta \in V \setminus W\).
    Let \(f \triangleq s_T(\beta; W)\) so \(f \beta \in W\).
    By \(T\)-admissibility, \(\exs \gamma \in W,\: f \beta = f \gamma\).
    Let \(\alpha \triangleq \beta - \gamma\) so that \(f \alpha = 0\).
    Moreover, \(S_T(\alpha; W) = S_T(\beta; W) = (f)\) as \(W\) is \(T\)-invariant.
    Hence, \(f = s_T(\beta; W) = s_T(\alpha; W)\).
    and \(f \in M(\alpha; T)\),
    which implies \((f) = S_T(\alpha; W) \subseteq M(\alpha; T)\).
    Conversely, if \(g \in M(\alpha; T)\), then \(g \alpha = 0 \in W\)
    and thus \(M(\alpha; T) \subseteq S_T(\alpha; W)\);
    \(f\) is the \(T\)-annihilator of \(\alpha\) as well.

    \mclm{Claim 2}{
        Let \(W\) be a subspace of \(V\).
        If \(s_T(\alpha; W) \alpha = 0\), then
        \(S_T(\alpha; W) = M(\alpha; T)\) and \(W \cap Z(\alpha; T) = \{0\}\).
    }

    It is easily shown that \(S_T(\alpha; T) = M(\alpha; T)\).
    Suppose \(g \alpha \in W \cap Z(\alpha; T)\).
    Then, \(g \in S_T(\alpha; W) = M(\alpha; T)\),
    and thus \(g \alpha = 0\). \checkmark
    

    \mclm{Claim 3}{
        For a proper \(T\)-invariant subspace \(W\) of \(V\),
        \(\beta \in \argmax_{\alpha \in V} \deg s_T(\alpha; W)\) exists,
        moreover, \(W \cap \argmax_{\alpha \in V} \deg s_T(\alpha; W) = \OO\).
        As a corollary, \(W + Z(\beta; T)\) is a \(T\)-invariant subspace of \(V\)
        which has \(W\) as its proper subspace.
    }

    If \(p\) is the characteristic polynomial of \(T\),
    then \(p \alpha = 0 \in W\) for all \(\alpha \in V\) by \Cref{th:cayley},
    i.e., \(p \in S_T(\alpha; T)\).
    Therefore, \(\deg s_T(\alpha; W)\) is bounded above by \(\deg p = \dim V\).
    Hence, \(A = \argmax_{\alpha \in V} \deg s_T(\alpha; W) \neq \OO\), thus we may
    take \(\beta \in A\).

    If \(\beta \in W\), we will have \(s_T(\alpha; W) = 1\) for all \(\alpha \in V\)
    and thus \(\alpha = s_T(\beta; W) \alpha \in W\),
    contradicting \(W \subsetneq V\). \checkmark\par
    {
        \setlength{\interspacetitleruled}{0pt}%
        \setlength{\algotitleheightrule}{0pt}%
        \LinesNumberedHidden%
        \begin{algorithm}[H]
            \TitleOfAlgo{Construct \(\beta_1, \cdots, \beta_r\) and \(W_1, \cdots, W_r\)}
            \(i \gets 0\)\;
            \While{\(W_i \neq V\)}{
                Take any \(\beta_{i+1} \in \argmax_{\alpha \in V} \deg s_T(\alpha; W_i)\)
                \Comment*[r]{well-defined by \textbf{\textit{Claim 3}}}
                \(W_{i+1} \gets W_i + Z(\beta_{i+1}, W_i)\)\;
                \(i \gets i + 1\)\;
            }
        \end{algorithm}
    }

    This algorithm above eventually ends in at most \(\dim V\) loops
    until we have \(V = W_0 + \sum_{i=1}^{r} Z(\beta_i, W_{i-1})\) by \textbf{\textit{Claim 3}}.
    Also, by the construction, \(W_k = W_{k-1} + Z(\beta_k, W_{k-1})\)
    for each \(k \in [r]\),
    and each \(W_k\) is \(T\)-invariant.
    \[
        \textstyle W_k = W_0 + \sum_{i=1}^{k-1} Z(\beta_i; W_{i-1})
    \]

    \mclm{Claim 4}{
        For each \(k \in [r]\) and \(\beta \in V\),
        write \(f \beta = \beta_0 + \sum_{i=1}^{k-1} g_i \beta_i\)
        where \(f = s_T(\beta; W_{k-1})\), \(g_i \in F[x]\),
        and \(\beta_i \in W_{i}\) for each \(i \in [k-1]\).
        Then, \(f \mid g_i\) for each \(i \in [k-1]\),
        and \(\beta_0 = f \gamma_0\) for some \(\gamma_0 \in W_0\).
    }

    Fix \(k \in [r]\) for now.
    By \Cref{th:euclid}, \(g_i = fq_i + r_i\)
    for some \(q_i, r_i \in F[x]\) such that it is either \(r_i = 0\)
    or \(\deg r_i < \deg f\).
    Let \(\gamma \triangleq \beta - \sum_{i=1}^{k-1} h_i \beta_i\).
    Then, we have:
    \[\begin{aligned}[t]
        f\gamma &= f\beta - \textstyle\sum_{i=1}^{k-1} fh_i \beta_i \\
                  &= \textstyle\big(\beta_0 + \sum_{i=1}^{k-1} g_i \beta_i\big)
                      - \sum_{i=1}^{k-1} (g_i-r_i) \beta_i \\
                  &= \textstyle \beta_0 + \sum_{i=1}^{k-1} r_i \beta_i.
    \end{aligned}\]
    Note that, by \textbf{\textit{Claim 0}}, \(S_T(\gamma; W_{k-1}) = S_T(\beta; W_{k-1}) = (f)\).

    For the sake of contradiction, suppose \(r_i \neq 0\) for some \(i \in [k-1]\) and
    let \(j\) be the maximum among such \(i\) so
    \(f\gamma = \beta_0 + \sum_{i=1}^{j} r_i \beta_i\).
    Let \(p \triangleq s_T(\gamma; W_{j-1})\).
    As \(W_j \subseteq W_{k-1}\), we have \(p \in S_T(\gamma; W_{k-1}) = (f)\), i.e.,
    \(p = fg\) for some \(g \in F[x]\). Then,
    \[\textstyle
        p \gamma = gf \gamma = g \beta_0 + \sum_{i=1}^{j-1} gr_i \beta_i + g r_j \beta_j.
    \]
    Then, \(p \gamma \in W_{j-1}\) by the definition of \(p\)
    and \(g \big(\beta_0 + \sum_{i=1}^{j-1} r_i \beta_i\big) \in W_{j-1}\)
    as \(W_{j-1}\) is \(T\)-invariant. Hence, we have \(g r_j \beta_j \in W_{j-1}\),
    i.e., \(g r_j \in S_T(\beta_j; W_{j-1})\).
    Hence, by the construction of \(\beta_j\), \[
        \deg (g r_j) 
        \underbrace{\ge}_{\text{by definition}}
        \deg s_T(\beta_j; W_{j-1})
        \underbrace{\ge}_{\text{by construction of } \beta_j}
        \deg s_T(\gamma; W_{j-1}) = \deg p = \deg (fg).
    \]
    Therefore, \(\deg r_j \ge \deg f\), which is a contradiction.
    Hence, \(r_i = 0\) for all \(i \in [k-1]\); \(f \mid g_i\).

    Now, we are left with \(\beta_0 = f \gamma\).
    By \(T\)-admissibility of \(W_0\), there exists \(\gamma_0 \in W_0\)
    such that \(f \gamma_0 = f \gamma = \beta_0\). \checkmark
    \vspace*{.7em}

    Fix any \(k \in [r]\).
    Let \(p_k \triangleq s_T(\beta_k; W_{k-1})\).
    Then, by \textbf{\textit{Claim 4}},
    \(p_k \beta_k = p_k \gamma_0 + \sum_{i=1}^{k-1} p_k h_i \beta_i\)
    for some \(\gamma_0 \in W_0\) and \(h_i \in F[x]\).
    Let \(\alpha_k \triangleq \beta_k - \gamma_0 - \sum_{i=1}^{k-1} h_i \beta_i\)
    so that \(p_k \alpha_k = 0\) and \(\alpha_k - \beta_k \in W_{k-1}\).
    Then, by \textbf{\textit{Claim 0}} and  \textbf{\textit{Claim 2}},
    we have:
    \begin{itemize}[nolistsep]
        \ii \((p_k) = S_T(\beta; W_{k-1}) =  S_T(\alpha_k; W_{k-1}) = M(\alpha_k; T)\)
        \ii \(W_{k-1} \cap Z(\alpha_k; T) = \{0\}\)
        \ii \(W_{k-1} + Z(\alpha_k; T) = W_{k-1} + Z(\beta_k; T)\)
    \end{itemize}
    As \(k\) is arbitrary, we have
    \[
        W_k = W_0 \oplus \big(\oplus_{i=1}^k Z(\alpha_i; T)\big).
    \]

    Moreover, note that \(\alpha_1, \cdots, \alpha_r\) retains the defining property
    of \(\beta_1, \cdots, \beta_r\),
    i.e., \(\alpha_k \in \argmax_{\alpha \in V} \deg s_T(\alpha; W_{k-1})\).
    \textbf{\textit{Claim 4}} holds when \(\beta_1, \cdots, \beta_r\)
    are replaced with \(\alpha_1, \cdots, \alpha_r\).
    Hence, applying the alternative version of \textbf{\textit{Claim 4}}
    to the trivial equation
    \[
        \textstyle p_k \alpha_k = 0 \cdot 0 + \sum_{i=1}^{k-1} p_i \alpha_i,
    \]
    we have \(p_k \mid p_i\) for each \(i \in [k - 1]\).
    The existence part of the theorem is now proven. \checkmark

    Now, we shall show the uniqueness of such decomposition.
    Suppose \(V = W_0 \oplus \big(\oplus_{i=1}^s Z(\gamma_i; T)\big)\)
    is another cyclic decomposition where \(q_{k+1} \mid q_k\) for each \(k \in [s - 1]\).
    (\(q_i\) is the \(T\)-annihilator of \(\alpha_i\).)
    Let \(S_T(V; W_0) \triangleq \{\,f \in F[x] \mid f V \subseteq W_0\,\}\)
    so \(S_T(V; W_0)\) is an ideal as \(W_0\) is \(T\)-invariant. 

    \mclm{Claim 5}{
        \(p_1\) and \(q_1\) are the monic generator of \(S_T(V; W_0)\).
        Thus, \(p_1 = q_1\).
    }
    
    As it is already \(S_T(V; W_0) \subseteq S_T(\alpha_1; W_0)\),
    it is enough to show that \(p_1 \in S_T(V; W_0)\).
    Take any \(\beta \in V\) and write \(\beta = \beta_0 + \sum_{i=1}^{r} f_i \alpha_i\)
    where \(\beta_0 \in W_0\) and \(f_i \in F[x]\).
    Then, \(p_1 \beta = p_1 \beta_0 + \sum_{i=1}^{r} p_1 f_i \alpha_i\).
    As \(p_1 | p_i\) for each \(i \in [r]\), we have \(p_1 \alpha_i = 0\).
    Hence, \(p_1 \beta \in W_0\); \(p_1 \in S_T(V; W_0)\).
    Similarly, \(q_1 \in S_T(V; W_0)\). \checkmark

    Now, we will prove the main question by induction.
    \mclm{Claim 6}{
        Fix any \(1 \le k < r\). If \(s \ge k\) and \(p_i = q_i\) for each \(i \in [k]\),
        then \(s > k\) and \(p_{k+1} = q_{k+1}\).
    }

    By (i) of \Cref{th:cycleAndAnnihilator} and \(k < r\),
    we have \(\dim W_0 + \sum_{i=1}^k \dim Z(\alpha_i; T) < \dim V\).
    And, thus \(\dim W_0 + \sum_{i=1}^k \dim Z(\gamma_i; T) < \dim V\) as \(p_i = q_i\).
    Hence, \(s > k\). Now, we may discuss facts about \(q_{k+1}\).

    The two decompositions give
    \[\begin{aligned}[t]
        p_{k+1}V &= p_{k+1}W_0 \oplus \big(\oplus_{i=1}^k Z(p_{k+1} \alpha_i; T)\big) \qquad\text{and}\\
        p_{k+1}V &= p_{k+1}W_0 \oplus \big(\oplus_{i=1}^s Z(p_{k+1} \gamma_i; T)\big)
    \end{aligned}\]
    together with (i) and (ii) of \Cref{lem:cycDecompLem}.
    (For the first representation, we only add up to \(k\) since \(p_{k+1} \alpha_i\)
    for all \(i \in \{\,k+1,\cdots,r\,\}\).)
    Now, we have
    \[\begin{aligned}[t]
        \dim (p_{k+1}V) &= \textstyle \dim (p_{k+1}W_0) + \sum_{i=1}^k \dim Z(p_{k+1} \alpha_i; T) \\
                        &= \textstyle \dim (p_{k+1}W_0) + \sum_{i=1}^s \dim Z(p_{k+1} \gamma_i; T) \\
                        &= \textstyle \dim (p_{k+1}W_0) + \sum_{i=1}^k \dim Z(p_{k+1} \alpha_i; T) + \sum_{i=k+1}^{s} \dim Z(p_{k+1} \gamma_i; T),
    \end{aligned}\]
    by (iii) of \Cref{lem:cycDecompLem}, and thus
    \(\dim Z(p_{k+1} \gamma_{k+1}; T) = 0\), i.e., \(p_{k+1} \gamma_{k+1} = 0\).
    Thus, \(q_{k+1} \mid p_{k+1}\).
    Similarly, we also have \(p_{k+1} \mid q_{k+1}\), and thus \(p_{k+1} = q_{k+1}\). \checkmark

    Using \textbf{\textit{Claim 6}}, we reach \(r \le s\) and \(p_i = q_i\) for each \(i \in [r]\)
    by mathematical induction.
    By symmetry, we also have \(s \le r\) and thus the theorem is proven.
}

\cor{}{
    Let \(V\) be a finite-dimensional vector space over \(F\) and let \(T \in L(V)\).
    Let \(W\) be a \(T\)-invariant subspace of \(V\).
    Then, \(W\) is \(T\)-admissible
    if and only if there exists another \(T\)-invariant subspace \(W'\) of \(V\)
    such that \(V = W \oplus W'\).
}
\pf{Proof}{
    (\(\Rightarrow\)) \Cref{th:cycDecomp}
    (\(\Leftarrow\)) \Cref{lem:compTinvarThenAdmiss}
}

\nt{
\begin{itemize}[nolistsep]
    \ii Every \(T \in L(V)\) has \(\alpha \in V\)
        such that \(T\)-annihilator of \(\alpha\) equals the minimal polynomial of \(T\).
        (\(\alpha_1\) when \(W_0 = \{0\}\). \(T\)-conductor of \(\alpha_1\) to \(W_0\)
        is \(T\)-annihilator of \(\alpha_1\) and it is the minimal polynomial.)
    \ii If \(T \in L(V)\) has a cyclic vector, then characteristic polynomial of \(T\)
        equals the minimal polynomial of \(T\).
\end{itemize}
}

\cor[cycDecompAndPoly]{}{
    Let \(V\) be a finite-dimensional vector space over \(F\) and let \(T \in L(V)\).
    Let \(V = \oplus_{i=1}^r Z(\alpha_i; T)\) be the cyclic decomposition
    obtained by applying \Cref{th:cycDecomp} to \(W_0 = \{0\}\).
    \begin{enumerate}[nolistsep, label=(\roman*)]
        \ii \(p_1\), the \(T\)-annihilator of \(\alpha_1\), is the minimal polynomial of \(T\).
        \ii The characteristic polynomial of \(T\) equals \(\prod_{i=1}^r p_i\).
        \ii \(T\) has a cyclic vector if and only if the characteristic polynomial
            and the minimal polynomial are identical.
    \end{enumerate}
}
\mclm{Proof}{
    \hfill
    \begin{enumerate}[nolistsep, label=(\roman*)]
        \ii \(p_1\) is the monic generator of \(S_T(V; \{0\}) = \{\,g \in F[x] \mid \fall \alpha \in V,\: g \alpha = 0\,\}\)
            by \textbf{\textit{Claim 6}} in the proof of \Cref{th:cycDecomp}.
            In other words, \(p_1\) is the minimal polynomial of \(T\).

        \ii Let \(T_i \triangleq T\big|_{Z(\alpha_i; T)}\) be a linear operator on \(Z(\alpha_i ;T)\).
            By \Cref{th:cycleAndAnnihilator}, \(p_i\) is the characteristic polynomial of \(T_i\).
            Hence, \(\prod_{i=1}^r p_i\) is the characteristic polynomial of \(T\).

        \ii \(\exs \alpha \in V,\: T = Z(\alpha; T)\) \(\underbrace{\iff}_\text{uniqueness}\)
            \(r = 1\) \(\underbrace{\iff}_\text{(ii)}\) \(p_1 = (\text{the characteristic polynomial of }T)\)
            \qed
    \end{enumerate}
}

\thm[genCayley]{Generalized Cayley-Hamilton Theorem}{
    Let \(V\) be a finite-dimensional vector space over \(F\) and let \(T \in L(V)\).
    Let \(m\) and \(f\) be the minimal and the characteristic polynomial polynomial, respectively.
    Then,
    \begin{enumerate}[nolistsep, label=(\roman*)]
        \ii \(m \mid f\).
        \ii \(m\) and \(f\) have the same prime factors except possibly for multiplicities.
        \ii Suppose \(m = \prod_{i=1}^k f_i^{r_i}\) and \(f = \prod_{i=1}^k f_i^{d_i}\)
            are the prime factorizations.
            Then, \(d_i = (\mrm{nullity}\,f_i(T)^{r_i})/(\deg f_i)\).
    \end{enumerate}
}
\pf{Proof}{
\hfill
\begin{enumerate}[nolistsep, label=(\roman*)]
    \ii \Cref{th:cayley}
    \ii Apply \Cref{th:cycDecomp} to \(W_0 = \{0\}\).
        Then, we have \(V = \oplus_{i=1}^r Z(\alpha_i; T)\).
        and \(p_{i+1} \mid p_i\) for each \(i \in [r-1]\) where
        \(p_i\) is the \(T\)-annihilator of \(\alpha_i\).
        (ii) of \Cref{cor:cycDecompAndPoly} says \(f = \prod_{i=1}^r p_i\).
        If a prime factor divides \(f\), then it divides one of \(p_i\), which divides \(p_1 = m\).
        Hence, \(f\) and \(m\) has the same prime factors.
    \ii Apply \Cref{th:pdt} with respect to \(V\) and \(T\).
        Then, we have \(V = \oplus_{i=1}^k W_i\) where \(W_i = \ker f_i(T)^{r_i}\).
        Take \(T_i = T\big|_{W_i}\) so \(f_i^{r_i}\) is the minimal polynomial of \(T_i\).
        By (ii) and the supposition, the characteristic polynomial \(g_i\) of \(T_i\) is \(f_i^{d_i}\)
        where \(r_i \le d_i\).
        Hence, \(\dim W_i = \deg g_i = d_i \cdot \deg f_i\).
        Therefore, \(d_i = (\dim W_i) / (\deg f_i)
        = (\mrm{nullity}\,f_i(T)^{r_i}) / (\deg f_i)\).
\end{enumerate}
}

\cor[nilpotCharPoly]{}{
    Let \(V\) be an \(n\)-dimensional vector space over \(F\) and let \(N \in L(V)\)
    be nilpotent. Then, the characteristic polynomial of \(N\) is \(x^n\).
}
\pf{Proof}{
    As \(N^k = 0\) for some \(k \in \NN\),
    the minimal polynomial is \(x^r\) for some \(r\).
    Hence, by (ii) of \Cref{th:genCayley}, we have \(x^n\) as the characteristic polynomial of \(N\).
}

\dfn{Rational Form}{
    Let \(F\) be a field.
    A matrix \(A \in F^{n \times n}\) is said to be in a \textit{rational form} if 
    \[
        A = \begin{bmatrix}
            A_1 & \vec{0}_{k_1 \times k_2} & \cdots & \vec{0}_{k_1 \times k_r} \\
            \vec{0}_{k_2 \times k_1} & A_2 & \cdots & \vec{0}_{k_2 \times k_r} \\
            \vdots & \vdots & \ddots & \vdots \\
            \vec{0}_{k_r \times k_1} & \vec{0}_{k_r \times k_2} & \cdots & A_r
        \end{bmatrix}
    \]
    where
    \begin{itemize}[nolistsep]
        \ii \(A_i\) is a companion matrix of \(p_i \in F[x]\) with \(\deg p_i = k_i\) for each \(i \in [r]\) and
        \ii \(p_{i+1} \mid p_i\) for each \(i \in [r-1]\).
    \end{itemize}
    % Let \(V\) be a finite-dimensional vector space over \(F\) and let \(T \in L(V)\).
    % Let \(V = \oplus_{i=1}^r Z(\alpha_i; T)\) be a cyclic decomposition
    % obtained by applying \Cref{th:cycDecomp} to \(W_0 = \{0\}\).
    % By (i) of \Cref{th:cycleAndAnnihilator}, \(\mcal B_i = \{\,\alpha_1, \cdots, T^{k_i-1} \alpha_i\}\)
    % is a basis for \(Z(\alpha_i; T)\) where \(k_i = \dim Z(\alpha_i; T)\).
    % Let \(\mcal B = (\mcal B_1, \cdots, \mcal B_r)\) be a basis for \(V\).
    % Then,
    % \[
    %     [T]_{\mcal B} = \begin{bmatrix}
    %         [T_1]_{\mcal B_1} & 0 & \cdots & 0 \\
    %         0 & [T_2]_{\mcal B_2} & \cdots & 0 \\
    %         \vdots & \vdots & \ddots & \vdots \\
    %         0 & 0 & \cdots & [T_r]_{\mcal B_r}
    %     \end{bmatrix}
    % \]
    % is called a \textit{rational form} of \(T\).
    % Note that each submatrix \([T_i]_{\mcal B_i}\) is the companion matrix of \(p_i\),
    % the \(T\)-annihilator of \(\alpha_i\).
}

\thm[uniqueRationalForm]{}{
    Let \(F\) be a field.
    For any \(B \in F^{n \times n}\), there uniquely exists a matrix in a rational form
    which is similar to \(B\).
}
\pf{Proof}{
    Let \(F^n = \oplus_{i=1}^r Z(\alpha_i; B)\) be a cyclic decomposition
    obtained by applying \Cref{th:cycDecomp} to \(W_0 = \{0\}\).
    By (i) of \Cref{th:cycleAndAnnihilator}, \(\mcal B_i = \{\,\alpha_1, \cdots, T^{k_i-1} \alpha_i\}\)
    is a basis for \(B_i = Z(\alpha_i; T)\) where \(k_i = \dim Z(\alpha_i; B)\).
    Let \(\mcal B = (\mcal B_1, \cdots, \mcal B_r)\) be a basis for \(F^n\).
    Then,
    \[
        [B]_{\mcal B} = \begin{bmatrix}
            [B_1]_{\mcal B_1} & 0 & \cdots & 0 \\
            0 & [B_2]_{\mcal B_2} & \cdots & 0 \\
            \vdots & \vdots & \ddots & \vdots \\
            0 & 0 & \cdots & [B_r]_{\mcal B_r}
        \end{bmatrix}
    \]
    is in a rational form since \([B_i]_{\mcal B_i}\) is the companion matrix of \(p_i\),
    the \(T\)-annihilator of \(\alpha_i\).

    Now, suppose that \(A\) is a \(n \times n\) matrix over \(F\)
    which is similar to \(B\) and is in a rational form.
    Let \(A\) be composed of \(r\) companion matrices of
    \(p_1, \cdots, p_r\) where \(p_{i+1} \mid p_i\) for each \(i \in [r-1]\).
    Then, we have \(\beta_1, \cdots, \beta_r \in F^n\)
    where \(p_i\) is the \(A\)-annihilator of \(\beta_i\),
    which is the \(B\)-annihilator of \(\beta_i\) as well.
    Hence, \(F^n = \oplus_{i=1}^r Z(\beta_i; B)\).
    By \Cref{th:cycDecomp}, \(r\) and \(p_1, \cdots, p_r\) are uniquely determined.
}

\section{The Jordan Form}

\nt{
    Let \(V\) be an \(n\)-dimensional vector space over \(F\) and let \(N \in L(V)\) be nilpotent.
    Applying \Cref{th:cycDecomp} to \(N\), we have \(V = \oplus_{i=1}^r Z(\alpha_i; N)\)
    and \(p_{i+1} \mid p_i\) for each \(i \in [r-1]\) where \(p_i\) is the \(T\)-annihilator of \(\alpha_i\).
    \(p_1\), the minimal polynomial of \(N\), is of the form \(x^k\) where
    \(1 \le k \le n\).
    Therefore, the divisibility relation simply asserts
    \(k = k_1 \ge k_2 \ge \cdots \ge k_r\) where \(p_i(x) = x^{k_i}\).
    So the rational form of \(N\) is
    \[
        \begin{bmatrix}
            A_1 & 0 & \cdots & 0 \\
            0 & A_2 & \cdots & 0 \\
            \vdots & \vdots & \ddots & \vdots \\
            0 & 0 & \cdots & A_r
        \end{bmatrix}
    \]
    where \(A_i\) is the companion matrix of \(x^{k_i}\), which is the \(k_i \times k_i\) matrix
    \[
        A_i = \begin{bmatrix}
            0 & 0 & \cdots & 0 & 0 \\
            1 & 0 & \cdots & 0 & 0 \\
            0 & 1 & \cdots & 0 & 0 \\
            \vdots & \vdots & \ddots & \vdots & \vdots \\
            0 & 0 & \cdots & 1 & 0
        \end{bmatrix}.
    \]
    Hence, there are total \(p(n)\) nilpotent \(n \times n\) linear operators on \(V\)
    that are not similar to each other where \(p(n)\) denotes the number of partitions of \(n\).
}

\mlemma{}{
    Let \(V\) be an \(n\)-dimensional vector space over \(F\) and let \(N \in L(V)\)
    be nilpotent.
    Let \(V = \oplus_{i=1}^r Z(\alpha_i; T)\) be the cyclic decomposition
    obtained by applying \Cref{th:cycDecomp} to \(W_0 = \{0\}\).
    Let \(k_i \triangleq \deg p_i\) where \(p_i\) is the \(T\)-annihilator of \(\alpha_i\).
    Then, \(\mcal B = \{N^{k_1-1} \alpha_1, \cdots, N^{k_r-1} \alpha_r\}\)
    is a basis for \(\ker N\).
}
\pf{Proof}{
    The fact that \(\mcal B \subseteq \ker N\) is evident as \(p_i = x^{k_i}\).
    Also, \(\mcal B\) is linearly independent as each vector is in independent subspaces (\(Z(\alpha_i; N)\)).

    Take any \(\alpha \in \ker N\).
    Then, \(\alpha = \sum_{i=1}^{r} f_i \alpha_i\) for some \(f_i \in F[x]\).
    \textsf{WLOG}, \(\deg f_i < \deg p_i = k_i\).
    Then, \(N \alpha = \sum_{i=1}^{r} N f_i(T) \alpha_i = 0\),
    and thus \(N f_i(T) \alpha_i = 0\) for each \(i \in [r]\).
    Therefore, \(p_i \mid x f_i\).
    Moreover, as \(p_i = x^{k_i}\), and as \(\deg f_i < k_i\), we have \(f_i = c_i x^{k_i-1}\)
    for some \(c_i \in F\).
    Then, we have \(\alpha = \sum_{i=1}^{r} c_i N^{k_i-1} \alpha_i\);
    \(\ker N \subseteq \spn \mcal B\).
}

\nt{
    Assume \(T\) is triangulable.

    \textit{Step 1}
    Let \(f(x) = \prod_{i=1}^k (x-c_i)^{d_i}\) be the characteristic polynomial.
    Let \(m(x) = \prod_{i=1}^k (x-c_i)^{r_i}\) be the minimal polynomial. (\(1 \le r_i \le d_i\))
    Take \(W_i = \ker (T-c_iI)^{r_i}\). By \Cref{th:pdt},
    we have \(V = \oplus_{i=1}^k W_i\). Let \(T_i = T \big|_{W_i}\).
    Each \(W_i\) is \(T\)-invariant and \(T_i\)'s minimal polynomial is \((x-c_i)^{r_i}\).

    \textit{Step 2}
    For each \(W_i\), let \(N_i = T_i - c_iI \in L(W_i)\) so \(N_i\) is nilpotent.
    So, \(T_i = N_i + c_iI\).
    We consider, for each \(W_i\), the cyclic decomposition of \(W_i\)
    with respect to \(N_i\).

    For each \(W\), we have
    \(W = Z(\alpha_1; N) \oplus \cdots \oplus Z(\alpha_{s_i}; N)\).
    Take \(\mcal B_j = \{\alpha_j, N \alpha_j, \cdots, N^{k_j-1} \alpha_j\}\).
    So \(\left[N\big|_{Z(\alpha_j; N)}\right]_{\mcal B_j}\) is \(\delta_{i-1,j}\) .
    So \(\left[T\big|_{Z(\alpha_j; N)}\right]_{\mcal B_j}\) is \(\delta_{i-1,j} + c_i \delta_{ij}\). (\ttt{< fix this})
    (This is called a \textit{elementary Jordan block}.)
    \(\mcal B^i = \bigcup_j \mcal B_j\)
}



\end{document}
