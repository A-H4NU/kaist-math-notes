\documentclass[MAS212_Note.tex]{subfiles}
\begin{document}

\section{Eigenvalues}

\dfn{Eigenvalue}{
    Let \(V\) be a vector space over \(F\).
    Let \(T \colon V \to V\) be a linear operator.
    \begin{itemize}[nolistsep]
        \ii \(c \in F\) is said to be an \textit{eigenvalue} (or a \textit{characteristic value}) of \(T\)
            if there exists \(v \in V \setminus \{0\}\) such that \(T(v) = cv\).
            Such \(v\) is called an \textit{eigenvector} (or a \textit{characteristic vector})
            of \(T\) associated to \(c\).

        \ii For each \(c \in F\), \(E_c \triangleq \{\,v \in V \mid T(v) = cv\,\}\)
            is called an \textit{eigenspace} (or a \textit{characteristic space})
            associated to \(c\).
    \end{itemize}
}

\thm[eigenTFAE]{}{
    Let \(V\) be a vector space over \(F\).
    Let \(T \colon V \to V\) be a linear operator.
    Then, \textsf{TFAE}.
    \begin{enumerate}[nolistsep, label=(\roman*)]
        \ii \(c \in F\) is an eigenvalue of \(T\).
        \ii \(T - cI\) is singular.
        \ii \(\det (T - cI) = 0\).
    \end{enumerate}
}
\pf{Proof}{
    The equivalence of (i) and (ii) is trivial.
    The equivalence of (ii) and (iii) is evident from \Cref{cor:AadjA}.
}

\dfn{Characteristic Polynomial}{
    Let \(A\) be an \(n \times n\) matrix over \(F\).
    Define \(f(x) \triangleq \det (xI - A) \in F[x]\).
    Then, \(f\) is a monic polynomial in \(x\) of degree \(n = \dim V\).

    If there exists a basis \(\mcal B\) for \(V\) and \(A = [T]_{\mcal B}\),
    then we call \(f(x) = \det (xI - A)\) the \textit{characteristic polynomial} of \(T\).
}
\nt{
    \noindent
    The choice of basis does not affect the characteristic polynomial.
    See \Cref{th:diffBasisSimilar}.
}

\nt{
    \noindent
    If \(f\) is a characteristic polynomial of \(T\),
    then \(f(c) = 0\) if and only if \(c\) is an eigenvalue of \(T\).
}

\cor{}{
    If \(T\) is a linear operator on \(V\),
    then there are at most \(n\) eigenvalues of \(T\).
}
\pf{Proof}{
    Every polynomial of degree \(n\) has at most \(n\) solutions.
}

\dfn{Diagonalizability}{
    Let \(V\) be a finite-dimensional vector space over \(F\).
    Let \(T \in L(V)\).
    We say \(T\) is \textit{diagonalizable}
    if there exists a basis \(\mcal B\) such that it consists of eigenvectors of \(T\).
}

\nt{
    \begin{itemize}[nolistsep]
        \ii If \(\mcal B = \{\,v_1, \cdots, v_n\,\}\) and
            \(Tv_i = c_iv_i\) for each \(i \in [n]\), then
            \([T]_{\mcal B} = \mrm{diag}(c_1, c_2, \cdots, c_n)\).
        \ii If \(T \in L(V)\) is diagonalizable, then the characteristic polynomial
            can be completely decomposed into a product of linear factors.
    \end{itemize}
}

\end{document}
