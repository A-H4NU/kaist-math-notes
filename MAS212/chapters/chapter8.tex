\documentclass[MAS212_Note.tex]{subfiles}
\begin{document}

\section{Inner Products}

\dfn{Inner Product}{
    Fix the field \(F\) to \(F = \RR\) or \(F = \CC\).
    An inner product \((-,-)\) on \(V\) is a function
    \((-,-) \colon V \times V \to F\) satisfying
    \begin{enumerate}[nolistsep, label=(\roman*)]
        \ii \((-,\ )\) is linear over \(F\).
        \ii \((\beta, \alpha) = \ol{(\alpha, \beta)}\)
        \ii If \(\alpha \neq 0\), \((\alpha, \alpha) > 0\).
    \end{enumerate}
}

\nt{
    \begin{itemize}[nolistsep]
        \ii If \(F = \RR\), (i) and (ii) say that \((\ ,-)\) is also linear over \(F\).
            Thus, an inner product is symmetric and bilinear.
        \ii If \(F = \CC\), \((a, c \gamma) = \ol{c} (\alpha, \gamma)\),
            i.e., \((-,-)\) is sesqui-linear.
        \ii If \(F = \CC\), \((\alpha, \alpha) = \ol{(\alpha, \alpha)}\), i.e., \((\alpha, \alpha) \in \RR\).
    \end{itemize}
}

\exmp{}{
    \begin{itemize}[nolistsep]
        \ii For \([x_i], [y_i] \in \CC[n]\),
            the inner product defined by \(([x_i], [y_i]) = \sum_{i=1}^{n} x_i \ol{y_i}\)
            is called the \textit{standard inner product}.
        \ii \(F = \RR\), let \(A \in \RR[n \times n]\) such that
            \(x^T A x > 0\) for all \(x \in \RR[n] \setminus \{0\}\).
            We say \(A\) is positive definite.
            The function \((x, y) = x^T A y\) is an inner product.
    \end{itemize}
}

\dfn[]{Conjugate Transpose}{
    For an \(n \times n\) matrix \(A\) over \(F = \RR\) or \(F = \CC\),
    the \(n \times n\) matrix \(A^\ast\) over \(F\) defined by
    \[
        (A^\ast)_{ij} = \ol{(A)_{ji}}
    \]
    for each \(i, j \in [n]\) is called the \textit{conjugate transpose} of \(A\).
}

\dfn{Self-adjoint Matrix}{
    A matrix \(A \in \CC[n \times n]\) is said to be \textit{self-adjoint}
    if \(A^\ast = A\).
}

\thm{}{
    Let \(V = \CC[n]\).
    Let \((-, -) \colon V \times V \to \CC\) be an inner product.
    Then, there exists a self-adjoint positive definite matrix \(A \in \CC[n \times n]\)
    such that \((x, y) = x^\ast Ay\).
}
\pf{Proof}{
    Choose a basis, e.g., the standard basis \(\{e_1, \cdots, e_n\}\).
    Let \((A)_{ij} = (e_i, e_j)\).
    Let \(x, y \in \CC[n]\) and write \(x = \sum_{i=1}^{n} x_i e_i\) and \(y = \sum_{j=1}^{n} x_j e_j\).
    Then,
    \[\begin{aligned}[t]
        (x, y) &= \textstyle \big(\sum_{i=1}^{n} x_ie_i, \sum_{j=1}^{n} y_je_j\big) \\
               &= \textstyle \sum_{i=1}^{n} \sum_{j=1}^{n} \ol{y_j} (A)_{ij} x_i = y^\ast A x.
    \end{aligned}\]
}

\cor[]{}{
    Let \(V = \RR[n]\).
    Let \((-, -) \colon V \times V \to \RR\) be an inner product.
    Then, there exists a symmetric positive definite matrix \(A \in \RR[n \times n]\)
    such that \((x, y) = x^\ast Ay\).
}

\exmp{}{
    Let \(V = \mcal C([a, b], \CC)\).
    Define, for \(f, g \in V\), \((f, g) = \int_{a}^{b} f(t) \ol{g(t)} \d t\).
    That is an inner product on \(V\).
}

\section{Inner Product Space}

\dfn{Inner Product Space}{
    A vector space \(V\) over \(F = \RR\) or \(F = \CC\)
    equipped with a specified inner product
    is called an \textit{inner product space}.
}

\notat{Norm}{
    We write
    \[
        \|v\| = \sqrt{(v, v)}.
    \]
    This is called a \textit{norm} of \(v\).
}

\thm[]{}{
    Let \(V\) be an inner product space.
    \begin{enumerate}[nolistsep, label=(\roman*)]
        \ii \(\|c \alpha\| = |c| \cdot \|\alpha\|\) for all \(c \in F\) and \(\alpha \in V\).
        \ii \(\|\alpha\| > 0\) for all \(\alpha \in V \setminus \{0\}\).
        \ii \(|(\alpha, \beta)| \le \|\alpha\| \cdot \|\beta\|\) for all \(\alpha, \beta \in V\). (\textit{Cauchy–Schwarz})
        \ii \(\|\alpha+\beta\| \le \|\alpha\| + \|\beta\|\) for all \(\alpha, \beta \in V\). (\textit{Triangle Inequality})
    \end{enumerate}
}
\pf{Proof}{ \hfill
\begin{enumerate}[nolistsep, label=(\roman*)]
    \ii \checkmark
    \ii \checkmark
    \ii If \(\alpha = 0\), we have nothing to prove, so suppose \(\alpha \neq 0\).
        Let
        \[
            \beta^{\parallel} \triangleq \frac{(\beta, \alpha)}{(\alpha, \alpha)} \alpha
            \quad\text{and}\quad
            \beta^\perp \triangleq \beta - \frac{(\beta, \alpha)}{(\alpha, \alpha)} \alpha.
        \]
        Then, \((\beta^\perp, \alpha) = 0\) and \(\beta^\parallel + \beta^\perp = \beta\).
        Let \(c = (\beta, \alpha)/(\alpha, \alpha)\).
        We have \[0 \le \|\beta^\perp\|^2 = (\beta - c \alpha, \beta^\perp)
        = (\beta, \beta) - |c|^2 (\alpha, \alpha)
        = \|\beta\|^2 - \frac{|(\alpha, \beta)|^2}{\|\alpha\|^2}.\]
        Rearranging the inequality gives the result.
    \ii \(\|\alpha+\beta\|^2 = (\alpha, \alpha) + (\alpha, \beta) + (\beta, \alpha) + (\beta, \beta)
        \le \|\alpha\|^2 + 2\|\alpha\|\cdot\|\beta\| + \|\beta\|^2 = (\|\alpha\| + \|\beta\|)^2\)
\end{enumerate}
}

\nt{
    For \(\alpha, \beta \in V \setminus \{0\}\),
    we define the \textit{angle} between \(\alpha\) and \(\beta\) be \(\theta \in \RR\)
    such that
    \[
        \cos \theta = \frac{(\alpha, \beta)}{\|\alpha\| \cdot \|\beta\|}.
    \]
}

\dfn{Orthogonality}{
    Let \(V\) be an inner product space.
    \begin{itemize}[nolistsep]
        \ii For \(\alpha, \beta \in V\), we say \(\alpha\) and \(\beta\) are \textit{orthogonal}
            if \((\alpha, \beta) = 0\).
        \ii For \(S \subseteq V\), we say \(S\) is \textit{orthogonal}
            if \(\alpha, \beta \in S\) and \(\alpha \neq \beta\),
            then \((\alpha, \beta) = 0\).
        \ii \(\{\alpha_1, \cdots, \alpha_n\} \subseteq V\) is said to be
            \textit{orthonormal} if it is orthogonal and if \(\|\alpha_i\| = 1\).
    \end{itemize}
}

\thm{}{
    Let \(V\) be an inner product space.
    Then, every orthogonal subset \(S\) of \(V \setminus \{0\}\) is linearly independent.
}
\pf{Proof}{
    Take any distinct \(\alpha_1, \cdots, \alpha_k \in S\).
    Suppose \(\sum_{i=1}^{k} c_i \alpha_i = 0\) for some \(c_i\).
    Then, for each \(k \in [n]\),
    \[ \displaystyle
        0 = \big(\sum_{i=1}^{k} c_i \alpha_i, \alpha_k\big)
        = \sum_{i=1}^{k} c_i (\alpha_i, \alpha_k) = c_k (\alpha_k, \alpha_k).
    \]
    Since \((\alpha_k, \alpha_k) > 0\), it must be \(c_k = 0\).
}

\nt{
    If \(S = \{\alpha_1, \cdots, \alpha_m\} \subseteq V\) is orthogonal,
    we may explicitly write every \(\beta \in \spn S\) in the form of
    \[
        \beta = c_1 \alpha_1 + c_2 \alpha_2 + \cdots + c_m \alpha_m
    \]
    by setting \(c_i = \dfrac{(\beta, \alpha_i)}{\|\alpha_i\|^2}\).
}

\thm[gramSchmidt]{Gram–Schmidt}{
    Let \(V\) be an inner product space.
    Suppose \(\{\beta_1, \cdots, \beta_n\}\) is a linearly independent subset of \(V\).
    Then, there exists an orthogonal set \(\{\alpha_1, \cdots, \alpha_n\}\) of vectors
    such that, for each \(k \in [n]\),
    \(\{\alpha_1, \cdots, \alpha_k\}\) is a basis of \(\spn \{\beta_1, \cdots, \beta_k\}\).
}
\pf{Proof}{
    Take \(\alpha_1 = \beta_1\).
    Then, for each \(k \in \{2, \cdots, n\}\), set
    \[
        \alpha_k \triangleq \beta_k - \sum_{i=1}^{k-1} \frac{(\beta_k, \alpha_i)}{\|\alpha_i\|^2} \alpha_i.
    \]
    Then, \(\{\alpha_1, \cdots, \alpha_n\}\) satisfies the condition.
}

\cor{}{
    Every finite dimensional inner product space has an orthonormal basis.
}
\pf{Proof}{
    Normalize vectors from \Cref{th:gramSchmidt}.
}

\dfn{Orthogonal Complement}{
    Let \(V\) be an inner product space.
    Let \(W\) be a subset of \(V\) and \(\beta \in V\).
    The subspace of \(V\)
    \[
        W^\perp = \{\,\beta \in V \mid \fall \alpha \in W,\: (\alpha, \beta) = 0\,\}
    \]
    is called the \textit{orthogonal complement of} \(W\).
}

\dfn{Best Approximation}{
    Let \(V\) be an inner product space.
    Let \(W\) be a subspace of \(V\) and \(\beta \in V\).
    A \textit{best approximation of \(\beta\) to \(W\)} is a vector \(\alpha \in W\)
    such that
    \[
        \forall \gamma \in W,\: \|\beta - \alpha\| \le \|\beta - \gamma\|.
    \]
}

\thm[bestApproxIffProj]{}{
    Let \(V\) be an inner product space.
    Let \(W\) be a subspace of \(V\) and \(\beta \in V\).
    \begin{enumerate}[nolistsep, label=(\roman*)]
        \ii \(\alpha \in W\) is a best approximation of \(\beta\) to \(W\)
            if and only if \(\beta - \alpha \in W^\perp\).
        \ii If a best approximation of \(\beta\) to \(W\) exists, then it is unique.
        \ii If \(W\) is finite-dimensional and \(\{ \alpha_1, \cdots, \alpha_n\}\)
            is an orthonormal basis for \(W\), then
            \(\alpha = \sum_{k=1}^{n} (\beta, \alpha_k) \alpha_k\)
            is the best approximation of \(\beta\) to \(W\).
    \end{enumerate}
}
\mclm{Proof}{
    For any \(\gamma \in V\), we have \(\beta - \gamma = (\beta - \alpha) + (\alpha - \gamma)\)
    and thus
    \[
        \|\beta - \gamma\|^2 = \|\beta - \alpha\|^2 + 2 \Re (\beta-\alpha, \alpha-\gamma) + \|\alpha - \gamma\|^2.
    \]

    \begin{enumerate}[nolistsep, label=(\roman*)]
        \ii
        (\(\Leftarrow\))
        Suppose \(\beta - \alpha \in W^\perp\).
        Then, for any \(\gamma \in W\), as \(\alpha - \gamma \in W\),
        \[
            \|\beta - \gamma\|^2 = \|\beta-\alpha\|^2 + \|\alpha - \gamma\|^2
            \ge \|\beta - \alpha\|^2.
        \]

        (\(\Rightarrow\))
        Suppose \(\alpha\) is a best approximation of \(\beta\) to \(W\).
        For any \(\gamma \in W\), we have
        \[
            2 \Re (\beta - \alpha, \alpha - \gamma) + \|\alpha - \gamma\|^2 \ge 0
        \] by the inequality above and \(\|\beta - \gamma\|^2 \ge \|\beta - \alpha\|^2\).
        In other words, for any \(\tau \in W\),
        \[
            2 \Re (\beta - \alpha, \tau) + \|\tau\|^2 \ge 0.
        \]
        Putting \[\tau = -\frac{(\beta - \alpha, \alpha - \gamma)}{\|\alpha - \gamma\|^2} (\alpha - \gamma),\]
        we have
        \[
            -2 \frac{|(\beta - \alpha, \alpha - \gamma)|^2}{\|\alpha - \gamma\|^2}
            + \frac{|(\beta - \alpha, \alpha - \gamma)|^2}{\|\alpha - \gamma\|^2} \ge 0,
        \]
        which holds if and only if \((\beta - \alpha, \alpha - \gamma) = 0\).

        \ii
        Directly follows from the inequality in the proof of (i)-(\(\Leftarrow\)).

        \ii
        A calculation shows that \(\beta - \alpha\) is orthogonal to
        every vector in \(W\). Hence, by (i), \(\alpha\) is the best approximation of \(\beta\) to \(W\).
        \qed
    \end{enumerate}
}

\dfn{Orthogonal Projection}{
    Let \(V\) be an inner product space.
    Let \(W\) be a subspace of \(V\).
    \begin{itemize}[nolistsep]
        \ii
        Let \(\beta \in V\) and \(\alpha \in W\). If \(\beta - \alpha \in W^\perp\),
        then \(\alpha\) is called the \textit{orthogonal projection of \(\beta\) on \(W\)}.

        \ii 
        If every vector in \(V\) has an orthogonal projection on \(W\),
        the mapping that assigns to each vector in \(V\) its orthogonal projection on \(W\)
        is called the \textit{orthogonal projection of \(V\)} on \(W\).
    \end{itemize}
}

\nt{
    If a subspace \(W\) of \(V\) is finite-dimensional,
    then there always exists the orthogonal projection of \(V\) on \(W\).
}

\cor[]{}{
    Let \(V\) be an inner product space.
    Let \(W\) be a finite-dimensional subspace of \(V\).
    Let \(E \colon V \to W\) be the orthogonal projection of \(V\) on \(W\).
    Then, \(\beta \mapsto \beta - E \beta\)
    is the orthogonal projection of \(V\) on \(W^\perp\).
}
\pf{Proof}{
    Take any \(\beta \in V\). Then, \(\beta - E \beta \in W^\perp\)
    by \Cref{th:bestApproxIffProj} (i).
    Also, for any \(\gamma \in W^\perp\),
    \(\beta - \gamma = E \beta + (\beta - E \beta - \gamma)\),
    and thus
    \[
        \|\beta - \gamma\|^2 = \|E \beta\|^2 + \|\beta - E \beta - \gamma\|^2
        \ge \|\beta - (\beta - E \beta)\|^2
    \]
    as \(E \beta \in W\) and \(\beta - E \beta - \gamma \in W^\perp\).
    Hence, \(\beta - E \beta\) is the best approximation of \(\beta\) to \(W^\perp\).
}

\thm[]{}{
    Let \(V\) be an inner product space.
    Let \(W\) be a finite-dimensional subspace of \(V\).
    Let \(E\) be the orthogonal projection of \(V\) on \(W\).
    Then \(E\) is an idempotent linear transformation of \(V\) to \(W\),
    \(\ker E = W^\perp\), and \(V = W \oplus W^\perp\).
}
\pf{Proof}{
    By the definition, we have
    \[
        \fall \beta \in W,\: E \beta = \beta,
    \]
    which means \(E(E \beta) = E \beta\) for all \(\beta \in V\). \checkmark

    Take any \(\alpha, \beta \in V\).
    Then, \((c \alpha + \beta) - (c E \alpha + E \beta) = c(\alpha - E \alpha) + (\beta - E \beta) \in W^\perp\).
    Also, \(c E \alpha + E \beta \in W\). By (ii) of \Cref{th:bestApproxIffProj},
    \(E (c \alpha + \beta) = c E \alpha + E \beta\). \checkmark

    The equation \(\beta = E \beta + (\beta - E \beta)\)
    shows that \(V = W + W^\perp\).
    For independence, for any \(\alpha \in W \cap W^\perp\),
    it must be \((\alpha, \alpha) = 0\); thus \(\alpha = 0\). \checkmark
}

\notat{Projection}{
    Let \(V\) be an inner product space.
    Let \(W\) be a finite-dimensional subspace of \(V\).
    Then, as \(V = W \oplus W^\perp\), we have a unique expression \(v = w + w'\)
    where \(w \in W\) and \(w' \in W^\perp\).
    Then, we write
    \[
        w = \mrm{proj}_W v \quad\text{and}\quad
        w' = \mrm{proj}_{W^\perp} v.
    \]
}

\nt{
    Let \(\{\alpha_1, \cdots, \alpha_k\}\) and \(\{\alpha_{k+1}, \cdots, \alpha_n\}\)
    be orthogonal bases for \(W\) and \(W^\perp\), repectively.
    Then,
    \[
        \mrm{proj}_{W} \beta \triangleq \sum_{i=1}^{k} \frac{(\beta, \alpha_i)}{\|\alpha_i\|^2} \alpha_i
        \quad\text{and}\quad
        \mrm{proj}_{W^\perp} \beta \triangleq \sum_{i=k+1}^{n} \frac{(\beta, \alpha_i)}{\|\alpha_i\|^2} \alpha_i.
    \]
}

\section{Linear Functionals and Adjoints}

\thm[VandDualNatIso]{}{
    Let \(V\) be an \(n\)-dimensional inner product space over \(F = \RR\) or \(F = \CC\).
    Then,
    \[
        \fall f \in V^\ast,\: \exs! \beta \in V,\: f(-) = (-, \beta).
    \]
    In other words, the linear transform \(V \to V^\ast\) defined by \(\beta \mapsto (-, \beta)\)
    is an isomorphism.
}
\pf{Proof}{
    Choose an orthogonal basis \(\{\alpha_1, \cdots, \alpha_n\}\) of \(V\).
    Let \(\beta = \sum_{i=1}^{n} c_i \alpha_i\).
    Then,
    \[
        f(\alpha_j) = \textstyle \big(\alpha_j, \sum_{i=1}^{n} c_i \alpha_i\big)
        = \sum_{i=1}^{n} \ol{c_i} (\alpha_j, \alpha_i)
        = \ol{c_j} \|\alpha_j\|^2.
    \]
    Hence, if such \(\beta\) exists, it must be
    \[
        \beta = \sum_{i=1}^{n} \frac{\ol{f(\alpha_i)}}{\|\alpha_i\|^2} \alpha_i.
    \]
    So, it is only left to check if this \(\beta\) actually works.
    It can be easily checked by reversing the computation above.
}

\nt{
    \Cref{th:VandDualNatIso} essentially says that \(V\) and \(V^\ast\) are naturally related
    if \(V\) is equipped with an inner product.
}

\thm[]{}{
    Let \(V\) be an \(n\)-dimensional inner product space over \(F = \RR\) or \(F = \CC\).
    \[
        \fall T \in L(V),\: \exs! T^\ast \in L(V),\: \fall \alpha, \beta \in V,\:(T \alpha, \beta) = (\alpha, T^\ast \beta).
    \]
}
\pf{Proof}{
    Fix \(\beta \in V\).
    For the linear functional \(\alpha \mapsto (T \alpha, \beta)\),
    \Cref{th:VandDualNatIso} guarantees the unique existence of \(\beta' \in V\)
    such that \((T(-), \beta) = (-, \beta')\).
    Hence, we may define \(T^\ast \in L(V)\) by \(\beta \mapsto \beta'\)
    so we have \((T \alpha, \beta) = (\alpha, T^\ast \beta)\) for all \(\alpha \in V\).
    (It is a linear operator.)
}

\nt{
    Such \(T^\ast\) is called a \textit{(Hermitian) transpose}.
}

\thm[matrixRepAndInnerProd]{}{
    Let \(V\) be an \(n\)-dimensional inner product space over \(F = \RR\) or \(F = \CC\).
    Let \(\mcal B = \{\alpha_1, \cdots, \alpha_n\}\) be an orthonormal basis for \(V\).
    Let \(T \in L(V)\) and \(A = [T]_{\mcal B}\).
    Then, \((A)_{ij} = (T \alpha_j, \alpha_i)\) for each \(i, j \in [n]\).
}
\pf{Proof}{
    Take any \(\alpha \in V\) and write \(\alpha = \sum_{i=1}^{n} (\alpha, \alpha_i) \alpha_i\).
    Hence,
    \[
        T(\alpha_j) = \textstyle \sum_{i=1}^{n} (T \alpha_j, \alpha_i) \alpha_i.
    \]
    By definition, we have \(T(\alpha_j) = \sum_{i=1}^{n} A_{ij} \alpha_i\).
    Therefore, \(A_{ij} = (T \alpha_j, \alpha_i)\).
}

\cor[transposeAllSame]{}{
    Let \(V\) be an \(n\)-dimensional inner product space over \(F = \RR\) or \(F = \CC\).
    Let \(T \in L(V)\). Let \(\mcal B\) be an orthonormal basis for \(V\).
    Then, \([T^\ast]_{\mcal B} = \big([T]_{\mcal B}\big)^\ast\), i.e.,
    the matrix representation of \(T^\ast\) equals the Hermitian transpose of the matrix representation of \(T\).
}
\pf{Proof}{
    Let \(A = [T]_{\mcal B}\) and \(B = [T^\ast]_{\mcal B}\).
    By the \Cref{th:matrixRepAndInnerProd}, we have
    \[
        (A)_{ij} = (T \alpha_j, \alpha_i) \quad\text{and}\quad
        (B)_{ij} = (T^\ast \alpha_j, \alpha_i).
    \]
    Then, \(\ol{(B)_{ij}} = (\alpha_i, T^\ast \alpha_j)\), and thus
    \(\ol{(B)_{ji}} = (\alpha_j, T^\ast \alpha_i)\).
    However, we have \((T \alpha_j, \alpha_i) = (\alpha_j, T^\ast\alpha_i)\) by the definition of \(T^\ast\), and thus
    \((A)_{ij} = \ol{(B)_{ji}}\).
}

\nt{
    \Cref{cor:transposeAllSame} essentially says that all transposes we discussed
    represents the same object when orthonormal basis is given.
}

\dfn[]{Hermitian Operator}{
    Let \(V\) be an \(n\)-dimensional inner product space over \(F = \RR\) or \(F = \CC\).
    Let \(T \in L(V)\).
    We say \(T\) is \textit{Hermitian} (or, \textit{self-adjoint}) if \(T = T^\ast\).
    If \(F = \RR\), we sometimes call it \textit{symmetric}.
}

\dfn[]{Isometry}{
    Let \(V\) and \(W\) be inner product spaces over \(F = \RR\) or \(F = \CC\).
    \(T \in L(V, W)\) is said to \textit{preserve the inner products}, or is called an \textit{isometry}, if
    \[
        \fall \alpha, \beta \in V,\:
        (\alpha, \beta) = (T \alpha, T \beta).
    \]
}

\dfn[]{Isomorphism of Inner Product Spaces}{
    Let \(V\) and \(W\) be inner product spaces over \(F = \RR\) or \(F = \CC\).
    \(T \in L(V, W)\) is called an \textit{isomorphism of inner product spaces} if it is an
    isomorphism of vector spaces and it preserves the inner products.
}

\thm[isomorphTFAE]{}{
    Let \(V\) and \(W\) be finite-dimensional inner product spaces over \(F = \RR\) or \(F = \CC\).
    Let \(T \in L(V, W)\). \textsf{TFAE}.
    \begin{enumerate}[nolistsep, label=(\roman*)]
        \ii \(T\) preserves the inner products.
        \ii \(T\) is an isomorphism of inner product spaces.
        \ii For any arbitrary orthonormal basis \(\mcal B\) of \(V\),
            \(T(\mcal B)\) is an orthonormal basis for \(W\).
        \ii For some orthonormal basis \(\mcal B\) for \(V\),
            \(T(\mcal B)\) is an orthonormal basis for \(W\).
    \end{enumerate}
}
\pf{Proof}{
    (\(\text{(i)} \Rightarrow \text{(ii)}\)) We need to show that \(T\) is an isomorphism
    of vector spaces. Take any \(\alpha \in \ker T\).
    Then, \(0 = (T \alpha, T \alpha) = (\alpha, \alpha) = \|\alpha\|^2\);
    hence \(\alpha = 0\). Therefore, \(T\) is injective, and thus is an isomorphism.

    (\(\text{(ii)} \Rightarrow \text{(iii)}\))
    Let \(\mcal B - \{\alpha_1, \cdots, \alpha_n\}\) be an orthonormal orthonormal basis for \(V\).
    Hence, \((T \alpha_i, T \alpha_j) = (\alpha_i, \alpha_j) = \delta_{ij}\);
    and thus \(T(\mcal B) = \{T \alpha_1, \cdots, T \alpha_n\}\) is an orthonormal basis for \(W\).


    \(\text{(ii)} \Rightarrow \text{(i)}\) and \(\text{(iii)} \Rightarrow \text{(iv)}\) are obvious.

    
    (\(\text{(iv)} \Rightarrow \text{(i)}\))
    Let \(\mcal B = \{\alpha_1, \cdots, \alpha_n\}\) be an orthonormal basis for \(V\)
    such that \(T(\mcal B)\) is also an orthonormal basis for \(W\).
    Then, \((T \alpha_i, T \alpha_j) = \delta_{ij} = (\alpha_i, \alpha_j)\).
}

\thm[]{}{
    Let \(V\) and \(W\) be inner product spaces over \(F = \RR\) or \(F = \CC\).
    For \(T \in L(V, W)\),
    \[
        T\text{ preserves the inner products} \iff \fall \alpha \in V,\: \|T \alpha\| = \|\alpha\|.
    \]
}
\pf{Proof}{
    (\(\Rightarrow\)) \(\|T \alpha\|^2 = (T \alpha, T \alpha) = (\alpha, \alpha) = \|\alpha\|^2\).

    (\(\Leftarrow\)) (Polarization identity) \((\alpha, \beta)\) can be expressed in terms of \(\|\alpha \pm \beta\|\)
    and \(\|\alpha \pm i \beta\|\). (Exercise)
}

\dfn[]{Unitary Operator}{
    Let \(V\) be an inner product space over \(F = \RR\) or \(F = \CC\).
    \(T \in L(V)\) is called a \textit{unitary operator}
    if it is an isomorphism of inner product spaces.
}

\thm[unitaryIff]{}{
    Let \(V\) be an inner product space over \(F = \RR\) or \(F = \CC\).
    For \(U \in L(V)\),
    \[
        U \text{ is unitary} \iff
        U^\ast \text{ exists and } UU^\ast = U^\ast U = \mrm{Id}.
    \]
}
\pf{Proof}{
    (\(\Rightarrow\))
    For \(U\) being an isomorphism, there exists the inverse \(U\inv \in L(V)\) of \(U\).
    Then, for all \(\alpha, \beta \in V\),
    \((U \alpha, \beta) = (U \alpha, U U\inv \beta) = (\alpha, U\inv \beta)\).
    Hence, \(U\inv = U^\ast\).

    (\(\Leftarrow\))
    \(U\) is invertible and \(U\inv = U^\ast\).
    Then, for all \(\alpha, \beta \in V\), \((U \alpha, U \beta) = (\alpha, U^\ast U \beta) = (\alpha, \beta)\).
    Hence, \(U\) preserves the inner product.
    Moreover, \(U\) is an isomorphism of vector spaces as well.
    Hence, \(U\) is unitary.
}

\nt{
    For finite-dimensional inner product spaces, for any \(T \colon V \to V\),
    we have \(T^\ast\).
    If the dimension is infinite, the existence of \(T^\ast\) is not immediate.
}


\dfn[]{Unitary Matrix}{
    Let \(A\) be an \(n \times n\) matrix over \(F = \RR\) or \(F = \CC\).
    We say \(A\) is a \textit{unitary matrix} if \[AA^\ast = A^\ast A = I_n.\]
}

\thm[]{}{
    Let \(V\) be a finite-dimensional inner product space over \(F = \RR\) or \(F = \CC\).
    For \(U \in L(V)\),
    \[
        U \text{ is unitary} \iff
        \exs \text{ orthonormal basis }\mcal B\text{ for }V,\: [U]_{\mcal B}\text{ is a unitary matrix}.
    \]
}

\thm[]{}{
    Let \(V\) be a finite-dimensional inner product space over \(F = \RR\) or \(F = \CC\).
    Let \(U_1, U_2 \in L(V)\) be unitary operators.
    Then, \(U_1 \circ U_2\) is also unitary.
}
\pf{Proof}{
    For all \(\alpha, \beta \in V\),
    \((U_1 U_2 \alpha, U_1 U_2 \beta) = (U_2 \alpha, U_2 \beta) = (\alpha, \beta)\).
}

\thm[]{}{
    Let \(V\) be a finite-dimensional inner product space over \(F = \RR\) or \(F = \CC\).
    If \(U \in L(V)\) is unitary, then \(U\inv\) is unitary as well.
}
\pf{Proof}{
    \(U\inv = U^\ast\), and \(UU^\ast = U^\ast U = \mrm{Id}\).
}

\dfn[]{}{
    Let \(V\) be an \(n\)--dimensional inner product space over \(F = \RR\) or \(F = \CC\).
    Let
    \[
        U(V) \triangleq \{\,T \in L(V) \mid T \text{ is unitary}\,\}.
    \]
    \[
        O(n) \triangleq\{\,A \in F^{n\times n} \mid A A^t = A^t A = I\,\}. 
    \]
    \begin{itemize}[nolistsep]
        \ii When \(F = \CC\), the group \((U(V), \circ) \simeq (U(\CC^n, \circ))\) is called the \textit{\(n\)-th unitary group}.
        \ii When \(F = \RR\), the group \((O(n), \cdot)\) is called the \textit{real orthogonal group}.
        \ii When \(F = \CC\), the group \((O(n, \CC), \cdot)\) is called the \textit{complex orthogonal group}.
        \ii \(SU(n) = \{\,A \in U(n) \mid \det A = 1\,\}\) is called the special unitary group.
        \ii \(SO(n) = \{\,A \in O(n) \mid \det A = 1\,\}\) is called the special orthogonal group.
    \end{itemize}
}

\section{Normal Operators}


\dfn[]{Normal Operator}{
    Let \(V\) be a finite-dimensional inner product space over \(F = \RR\) or \(F = \CC\).
    \(T \in L(V)\) is \textit{normal} if \(T T^\ast = T^\ast T\).
}

\nt{
    If we can find an orthonormal basis \(\mcal B\) for \(V\)
    such that the vectors in \(\mcal B\) are also characteristic vectors of \(T\),
    then \([T]_{\mcal B}\) is diagonal, and thus \(T T^\ast = T^\ast T\).
    In other words, \(T\) having an orthonormal basis consisting of characteristic vectors
    implies \(T\) is normal.
    In this section, we prove the converse.
}

\thm[eigvalAdj]{}{
    Let \(V\) be an \(n\)-dimensional inner product space over \(F = \RR\) or \(F = \CC\).
    Let \(T \in L(V)\) be normal.
    Then, for a characteristic vector \(\alpha \in V\),
    \(c \in F\) is a eigenvalue of \(T\) associated to \(\alpha\) 
    if and only if \(\ol c\) is an eigenvalue of \(T^\ast\) associated to \(\alpha\).
}
\pf{Proof}{
    For any \(\beta \in V\) and any normal \(U \in L(V)\), we have
    \[
        \|U \beta\|^2 = (U \beta, U \beta)
        = (\beta, U^\ast U \beta)
        = (\beta, U U^\ast \beta) = (U^\ast \beta, U^\ast \beta) = \|U^\ast \beta\|^2.
    \]

    Also, for any \(c \in F\), \(T - cI\) is normal.
    Hence, if \(c \in F\) is an eigenvalue and \(\alpha \in V\) is an eigenvector
    associated to \(c\), we have \(\|(T^\ast - \ol{c}I) \alpha\| = \|(T-cI) \alpha\| = 0\),
    and thus \(\ol{c}\) is an eigenvalue and \(\alpha\) is an eigenvector associated to \(\ol{c}\).
}

\thm[]{}{
    Let \(V\) be an \(n\)-dimensional inner product space over \(F = \RR\) or \(F = \CC\).
    Let \(T \in L(V)\).
    Suppose \(\mcal B\) is an orthonormal basis for \(V\) such that \([T]_{\mcal B}\)
    is upper triangular. Then,
    \[
        T \text{ is normal} \iff [T]_{\mcal B} \text{ is diagonal.}
    \]
}
\pf{Proof}{
    Let \(A \triangleq [T]_{\mcal B}\). Then \(A^\ast = [T^\ast]_{\mcal B}\) by \Cref{cor:transposeAllSame}.

    (\(\Leftarrow\))
    We apparently have \(A A^\ast = A^\ast A\), i.e., \(T T^\ast = T^\ast T\).
    \(T\) is normal.

    (\(\Rightarrow\))
    \(A\) is upper triangular. Let \(a_{ij} = (A)_{ij}\) so \(a_{ij} = 0\) if \(i > j\).
    Let \(\mcal B = \{\alpha_1, \cdots, \alpha_n\}\).
    We immediately have that \(T \alpha_1 = a_{11} \alpha_1\),
    and thus \(T^\ast \alpha_1 = \cl{a_{11}} \alpha_1\) by \Cref{th:eigvalAdj}.
    On the other hand, we have \(T^\ast \alpha_1 = \sum_{j=1}^{n} \cl{a_{1j}} \alpha_j\).
    Hence, \(\alpha_{1j} = 0\) for \(1 < j \le n\).

    Repeating the argument above, we get \(a_{ij} = 0\) if \(i < j\).
    Hence, \(A\) is diagonal.
}

\mlemma[orthoCompAdjInvar]{}{
    Let \(V\) be an \(n\)-dimensional inner product space over \(F = \RR\) or \(F = \CC\).
    Let \(T \in L(V)\). Let \(W\) be a \(T\)-invariant subspace of \(V\).
    Then, \(W^\perp\) is \(T^\ast\)-invariant.
}
\pf{Proof}{
    Take any \(\beta \in W^\ast\).
    For all \(\alpha \in W\), we have
    \((\alpha, T^\ast \beta) = (T \alpha, \beta) = 0\) as \(W\) is \(T\)-invariant.
    Hence, \(T^\ast \beta \in W^\perp\).
}

\thm[onCOrthoTriangular]{}{
    Let \(V\) be an \(n\)-dimensional inner product space over \(\CC\).
    Let \(T \in L(V)\).
    Then, there exists an orthonormal basis \(\mcal B\) for \(V\) such that
    \([T]_{\mcal B}\) is upper triangular.
}
\pf{Proof}{
    We prove by induction on \(\dim V\).
    Suppose that \(n = \dim V > 1\) and that the theorem holds for any inner product space
    with dimension less than \(n\).

    Since \(\CC\) is algebraically closed,
    there exists an eigenvalue \(c \in \CC\) and an eigenvector \(\alpha \in V\)
    such that \(T^\ast \alpha = c \alpha\). \textsf{WLOG}, \(\|\alpha\|=1\).
    Then, as \(\spn \{\alpha\}\) is \(T^\ast\)-invariant, we have
    \(W \triangleq \big(\spn \{\alpha\}\big)^\perp\) is \(T\)-invariant by \Cref{lem:orthoCompAdjInvar}.
    By the induction hypothesis on \(W\) and \(T\big|_W\),
    there exists an orthonormal basis \(\mcal B' = \{\alpha_1, \cdots, \alpha_{n-1}\}\) for \(W\)
    such that \(\big[T\big|_W\big]_{\mcal B'}\) is upper triangular.
    Then, \(\mcal B = \mcal B' \cup \{\alpha\}\) is a basis for \(V\) and
    \[
        [T]_{\mcal B} = \begin{bmatrix}
            \big[T\big|_W\big]_{\mcal B'} & \ast \\
            0 & \ast
        \end{bmatrix}
    \]
    is upper triangular.
}

\cor[]{}{
    Let \(V\) be an \(n\)-dimensional inner product space over \(\CC\).
    Let \(T \in L(V)\) be normal.
    Then, there exists an orthonormal basis for \(V\)
    consisting of eigenvectors of \(T\).
}

\cor[]{}{
    Let \(A \in \CC^{n \times n}\).
    \begin{enumerate}[nolistsep, label=(\roman*)]
        \ii There exists a unitary matrix \(P \in U(n)\) such that \(P\inv AP\) is upper triangular.
        \ii If \(A\) is normal, then \(P\inv AP\) is diagonal.
            In other words, \(A\) is \textit{unitarily diagonalizable}.
    \end{enumerate}
}

\exmp{}{
    \noindent
    Let \(V\) be an \(n\)-dimensional inner product space over \(\CC\).
    Let \(T \in L(V)\).
    \begin{enumerate}[nolistsep, label=(\roman*)]
        \ii If \(T\) is Hermitian, then \(T\) is normal.
        \ii If \(T\) is unitary, then \(T\) is normal.
    \end{enumerate}
}

\end{document}
