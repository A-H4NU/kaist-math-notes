\documentclass[MAS212_Note.tex]{subfiles}


\begin{document}
\section{Determinant Functions}

\dfn{\(n\)-linear and Iterating}{
    Let \(K\) be a ring.
    Let \(\mcal D \to K^{n \times n} \to K\) be a function.
    This is considered as a function on \(n\) row vectors.
    \begin{enumerate}[nolistsep, label=(\roman*)]
        \ii We say \(\mcal D\) is \(n\)-linear if \(\mcal D\) is a linear function
            on the \(i^{\text{th}}\) row while fixing all other rows.
            \[ \footnotesize
                \mcal D \begin{bmatrix}
                    \cdots & a_1 + a_1' & \cdots \\ \cdots & a_2 & \cdots \\  & \vdots &  \\ \cdots & a_n & \cdots
                \end{bmatrix} = \mcal D \begin{bmatrix}
                    \cdots & a_1 & \cdots \\ \cdots & a_2 & \cdots \\  & \vdots &  \\ \cdots & a_n & \cdots
                \end{bmatrix} + \mcal D \begin{bmatrix}
                    \cdots & a_1' & \cdots \\ \cdots & a_2 & \cdots \\  & \vdots &  \\ \cdots & a_n & \cdots
                \end{bmatrix}
            \]
        \ii An \(n\)-linear function \(\mcal D\) is called \textit{iterating}
            if \(\mcal D(A) = 0\) when two rows are equal.
    \end{enumerate}
}

\nt{
    \noindent
    If \(\mcal D\) is iterating, and if \(A'\) is obtained by switching \(i^{\text{th}}\) and
    \(j^{\text{th}}\) rows of \(A\), then \(\mcal D(A') = -\mcal D(A)\).
}

\dfn{Determinant}{
    Let \(K\) be a commutative ring with unity.
    Let \(\mcal D \colon K^{n \times n} \to K\) be a function.
    We say \(\mcal D\) is a determinant function if
    \begin{enumerate}[nolistsep, label=(\roman*)]
        \ii \(\mcal D\) is \(n\)-linear,
        \ii \(\mcal D\) is alternating, and
        \ii \(\mcal D(I_n) = 1\).
    \end{enumerate}
}

\dfn{Minor Matrix}{
    Let \(K\) be a commutative ring with unity.
    Let \(A \in K^{n \times n}\) where \(n > 1\).
    For each \(i, j \in [n]\), define \(A(i\mid j)\)
    be the \((n-1)\times(n-1)\) matrix with the \(i^{\text{th}}\) row and the \(j^{\text{th}}\)
    column are removed. \(A(i \mid j)\) is called \((i, j)\)-\textit{minor} of \(A\).
}

\thm[detExs]{}{
    There exists a determinant function \(\mcal D \colon K^{n \times n} \to K\).
}
\pf{Proof}{
    We shall prove by exploiting mathematical induction.
    If \(n=1\), the identity function is a determinant function.

    Suppose we have found a function \(\mcal D \colon K^{(n-1)\times(n-1)}\)
    which is \((n-1)\)-linear and alternating.
    We shall denote \(\mcal D(A(i \mid j)) = D_{ij}(A)\).
    Define \(E_{i}(A) \triangleq \sum_{i=1}^{n} (-1)^{i+j}A_{ij}D_{ij}(A)\)
    for each \(j \in [n]\).
    
    \mclm{Claim}{
        \(E_j\) is an \(n\)-linear function on \(K^{n \times n}\).
    } \par
    \(D_{ij}(A)\) is independent from the entries of the \(i\)-th row and the \(j\)-th column.
    Hence, \(D_{ij}\) is \(n\)-linear as \(\mcal D\) is \((n-1)\)-linear.
    Furthermore, \(A \mapsto A_{ij}D_{ij}(A)\) is also \(n\)-linear;
    thus \(E_j\) is linear combination of \(n\)-linear functions.

    \mclm{Claim}{
        \(E_j\) is an alternating function on \(K^{n \times n}\).
    } \par
    For the sake of simplicity, suppose \(A\) has two equal rows at \(\alpha_k\) and \(\alpha_{k+1}\).
    Hence, when \(i \neq k\) and \(i \neq k+1\),
    \(A(i \mid j)\) has two identical rows; thus \(D_{ij}(A) = D(A(i \mid j)) = 0\).
    Thus, \(E_j(A) = (-1)^{k+j}A_{kj}D_{kj}(A) + (-1)^{k+j+1}A_{(k+1),j}D_{(k+1),j}(A)\).
    \[\begin{aligned}[t]
        E_j(A) &= (-1)^{k+j}A_{kj}D_{kj}(A) + (-1)^{k+j+1}A_{(k+1),j}D_{(k+1),j}(A) \\
               &= (-1)^{k+j} \big(A_{kj}D_{kj}(A) - A_{kj}D_{kj}(A)\big) = 0
    \end{aligned}\]

    \mclm{Claim}{
        \(E_j(I_n) = 1\).
    } \par
    \(I_n(i \mid j) = I_{n-1}\).
}

\cor{}{
    The function defined recursively in the proof of \Cref{th:detExs} is a determinant function.
}

\dfn{Permutation}{
    Let \(S\) be a set.
    A permutation \(\sigma\) of \(S\)
    is a bijective function \(\sigma \colon S \to S\).

    \(S_n\) is the set of bijective functions from \([n]\) onto \([n]\).
}

\dfn{Transposition}{
    \(\tau \in S_n\) is called a \textit{transposition}
    if it interchanges just the values of two members.

    A transposition that interchanges \(i\) and \(j\) is usually written as \((i, j)\).
}

\dfn{Cycle}{
    A cycle is like:
    \[
        i_1 \mapsto i_2 \mapsto i_3 \mapsto \cdots \mapsto i_n \mapsto i_1.
    \]
    This is written as \((i_1, i_2, \cdots, i_n)\).
}

\nt{
    \begin{itemize}[nolistsep]
        \ii Every permutation can be written as a product of disjoint cycles.
        \ii Every cycle can be written as a product of transpositions.
        \ii Every permutation can be written as a product of transpositions.
    \end{itemize}
}

\thm{}{
    For any permutation \(\sigma \in S_n\), the number of transpositions
    needed to express \(\sigma\) modular \(2\) is an invariant of \(\sigma\).
}

\dfn{Sign of Permutation}{
    \[
        \sign(\sigma) \triangleq \begin{cases}
            1 & \text{if } \sigma \text{ is even} \\
            -1 & \text{if } \sigma \text{ is odd}
        \end{cases}
    \]
}

\cor{}{
    For \(\sigma_1, \sigma_2 \in S_n\), \(\sign(\sigma_1 \sigma_2) = \sign(\sigma_1) \sign(\sigma_2)\).
}

\thm[detIsUnique]{}{
    There exists a unique determinant function \(\mcal D \colon K^{n \times n} \to K\),
    which is equal to \[
        \mcal D(A) = \sum_{\sigma \in S_n} \sign(\sigma)\prod_{j \in [n]} A_{j,\sigma(j)}.
    \]
}
\pf{Proof}{
    Let \(e_1, \cdots, e_n\) be the rows of \(I_n\).
    For \(A \in K^{n \times n}\), let \(\alpha_i\) be the \(i\)-th rows of \(A\).
    Then, \(\alpha_i = \sum_{j=1}^{n} A_{ij}e_j\).

    Note that, if \(j_i = j_{i'}\), then \(\mcal D(e_{j_1}, \cdots, e_{j_n}) = 0\).
    Also, if \(\sigma \in S_n\), \(\mcal D(e_{\sigma(1)}, \cdots, e_{\sigma(n)})
    = \sign(\sigma) \mcal D(I_n) = \sign(\sigma)\).

    \[\begin{aligned}[t]
        \mcal D(A) &= \mcal D(\alpha_1, \alpha_2, \cdots, \alpha_n) \\
        &= \textstyle \mcal D \big( \sum_{j=1}^{n} A_{1j} e_j, \alpha_2, \cdots, \alpha_n \big) \\
        &= \textstyle \sum_{j=1}^{n} A_{1j} \mcal D(e_j, \alpha_2, \cdots, \alpha_n) \\
        &= \cdots \textstyle =\sum_{j_1=1}^{n} \sum_{j_2=1}^{n} \cdots \sum_{j_n=1}^{n} A_{1j_1}A_{2j_2}\cdots A_{n,j_n} \mcal D(e_{j_1}, \cdots, e_{j_n}) \\
        &= \textstyle \sum_{\sigma \in S_n} \sign(\sigma) \prod_{j \in [n]} A_{j,\sigma(j)}
    \end{aligned}\]

    Note that, if \(\mcal D\) is a \(n\)-linear and alternating, then \(\mcal D(A) = \det A \cdot \mcal D(I_n)\).
}

\cor[]{}{
    \(\det (AB) = \det A \cdot \det B\)
}

\cor[]{}{
    Any cofactor expansion gives the same value.
}

\cor[]{}{
    \(\det A^t = \det A\)
}
\pf{Proof}{
    \Cref{th:classicTrans} and \Cref{th:detIsUnique}.
}

\exer{}{
    Let \(A\) be \(r \times r\) matrix and \(C\) be an \(s \times s\) matrix.
    Then, \[
        \det \begin{bmatrix}
            A & B \\ 0 & C
        \end{bmatrix} = \det A \cdot \det C.
    \]
    Hint) Fixing \(A, B\), define \(\mcal D(A, B, C)\).
}

\dfn{Adjoint Matrix}{
    Let \(A\) be an \(n \times n\) matrix.
    \(C_{ij} \triangleq (-1)^{i+j} \det (A(i \mid j))\) for each \(i, j \in [n]\)
    is called the \textit{\((i, j)\)-cofactor}.
    Then, \(\adj A \triangleq C^t\) where \((C)_{ij} = C_{ij}\)
    is called the \textit{adjoint} of \(A\).
}

\cor[AadjA]{}{
    \(A \cdot \adj A = (\det A) I_n\).
    If \(\det A \in K\) is invertible, then \(A\inv = (\det A)\inv \adj A\).
}
\end{document}
