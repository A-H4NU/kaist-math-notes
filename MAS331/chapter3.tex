\documentclass[MAS331_Note.tex]{subfiles}

\begin{document}

\section{Connected Space}
\dfn{Separation and Connectedness}{
    Let $X$ be a topological space.
    A \textit{separation} of $X$ is a pair $U$ and $V$
    of subsets of $X$ which satisfy the following.
    \begin{enumerate}[nolistsep, label=(\roman*)]
        \ii $U$ and $V$ are open in $X$.
        \ii $U \cap V = \OO$.
        \ii $U \cup V = X$.
    \end{enumerate}
    The space $X$ is said to be \textit{connected}
    if there does not exist a separation of $X$.
}
\nt{
    Connectedness is a topological property.
}
\nt{
    A space $X$ is connected if and only if the only subsets of $X$
    that are both open and closed in $X$ are the empty sets and $X$ itself.
}

\mlemma[sepIffNotContainLimPts]{}{
    If $Y$ is a subspace of $X$,
    $A, B \subseteq Y$ is a separation of $Y$
    if and only if $A \cap B = \OO$, $A \cup B = Y$,
    and neither $A$ nor $B$ contains a limit point of the other.
}
\pf{Proof}{
    Suppose $A$ and $B$ form a separation of $Y$.
    Then, $A$ is both open and closed in $Y$; thus
    the closure of $A$ in $Y$ is $\cl A \cap Y = A$
    by \Cref{th:closureSubspace}.
    In other words, $\cl A \cap B = \OO$.
    Similarly, $A \cap \cl B = \OO$. \checkmark

    Suppose $A$ and $B$ are disjoint subsets of $Y$ whose union is $Y$
    and $A \cap B' = A' \cap B = \OO$.
    Thus, $A \cap \cl B = \cl A \cap B = \OO$.
    This implies $\cl A \cap Y = A$ and $\cl B \cap Y = B$; $A$ and $B$ are closed in $Y$,
    and thus they are open in $Y$ as well.
}

\mlemma[connInDisconn]{}{
    If the sets $C$ and $D$ form a separation of a space $X$,
    and if $Y$ is a connected subspace of $X$,
    then $Y$ lies entirely within $C$ or $D$.
}
\pf{Proof}{
    $C \cap Y$ and $D \cap Y$ are open in $Y$.
    Also, $(C \cap Y) \cup (D \cap Y) = (C \cup D) \cap Y = Y$.
    If they were both nonempty, they would form a separation of $Y$.
    Thus, one of them is empty; $Y$ is entirely in the other.
}

\thm[UconnIfHaveCommonPt]{}{
    Let $X$ be a topological space.
    Let $\{A_\alpha\}_{\alpha \in J}$ be a family of connected subspaces of $X$.
    If $\bigcap_{\alpha \in J} A_\alpha \neq \OO$,
    then $\bigcup_{\alpha \in J} A_\alpha$ is connected.
}
\pf{Proof}{
    Take any $p \in \bigcap_{\alpha \in J} A_\alpha$.
    Suppose $C$ and $D$ form a separation of $Y = \bigcup_{\alpha \in J} A_\alpha$.
    \textsf{WLOG}, $p \in C$.
    For each $\alpha \in J$, since $p \in C \cap A_\alpha$,
    by \Cref{lem:connInDisconn}, $A_\alpha \subseteq C$.
    Thus, $\bigcup_{\alpha \in J} A_\alpha \subseteq C$,
    contradicting that $D \cap Y \neq \OO$.
}

\thm[connIfBtwAandClA]{}{
    Let $A$ be a connected subspace of $X$.
    If $A \subseteq B \subseteq \cl A$,
    then $B$ is also connected.
}
\pf{Proof}{
    Suppose $B = C \cup D$ is a separation of $B$
    for the sake of contradiction.
    By \Cref{lem:connInDisconn}, \textsf{WLOG}, $A \subseteq C$.
    Then, $B \subseteq \cl A \subseteq \cl C$.
    Since $\cl C \cap D = \OO$ by \Cref{lem:sepIffNotContainLimPts},
    $B \cap D = \OO$, which makes $C$ and $D$ not form a separation, \#.
}

\thm[contiSendsConnToConn]{Connected Space and Continuous Map}{
    Let $f \colon X \to Y$ be a continuous map.
    If $X$ is connected, then $\Img f$ is connected.
}
\pf{Proof}{
    Note that the surjective map $g \colon X \to \Img f$ obtained by
    restricting the codomain of $f$ is also continuous by \Cref{th:constructConti}.
    Suppose $\Img f = A \cup B$ is a separation of $\Img f$.
    Then, $g\inv(A)$ and $g\inv(B)$ are open and disjoint sets in $X$
    whose union is $X$, which is a contradiction to the connectedness of $X$.
}

\thm[finProdConnIsConn]{Connected Space and Finite Product}{
    Let $\{X_i\}_{i=1}^n$ be a finite family of connected spaces.
    Then,
    \[
        X = \prod_{i=1}^n X_i
    \] is connected.
}
\pf{Proof}{
    It is enough to prove for two connected spaces $X$ and $Y$;
    extension to finite products can be done inductively.
    We may assume $X$ and $Y$ are nonempty.
    Take any $a \times b \in X \times Y$.
    Let $x \in X$.
    $X \times \{b\}$ and $\{x\} \times Y$ as subspaces of $X \times Y$
    are connected since they are homeomorphic with $X$ and $Y$, respectively.
    Thus,
    \[
        T_x = (X \times \{b\}) \cup (\{x\} \times Y)
    \] is connected by \Cref{th:UconnIfHaveCommonPt}, having
    $x \times b$ as a common point of two spaces.
    Thus, \[
        X \times Y = \bigcup_{x \in X} T_x
    \] is connected as they have a point $a \times b$ in common.
}

\thm[arbProdConnIsConn]{Connected Space and Product Topology}{
    Let $\{X_\alpha\}_{\alpha \in J}$ be a family of connected spaces.
    Then,
    \[
        X = \prod_{\alpha \in J} X_{\alpha}
    \] is connected in the product topology.
}
\pf{Proof}{
    We may assume that $X_\alpha \neq \OO$ for each $\alpha \in J$.
    Let $\vec a = (a_\alpha)_{\alpha \in J}$ be a fixed point of $X$.

    We first note that, given any finite subset $K$ of $J$,
    $X_K \triangleq \{\,(x_\alpha)_{\alpha \in J} \mid
    \fall \alpha \in J \setminus K,\: x_\alpha = a_\alpha\,\}$
    is a connected subspace of $X$ as $X_K$ is homeomorphic with $\prod_{\alpha \in K} X_\alpha$,
    which is connected by \Cref{th:finProdConnIsConn}.
    Note that $Y \triangleq \bigcup \{\,X_K \mid K \subseteq J \text{ and } K \text{ is finite}\,\}$
    as a subspace of $X$ is connected since $\vec a \in X_K$ for every finite $K \subseteq J$.

    Let $\vec x \in X$ and $B = \bigcap_{i=1}^n \pi_{\alpha_i}\inv(U_{\alpha_i})$ be
    any basis that contains $\vec x$ where $\alpha_i \in J$ for each $i \in [n]$.
    Define $\vec x' \in X$ be
    \[
        (\vec x')_{\alpha} \triangleq \begin{cases}
            x_\alpha & \text{if } \alpha = \alpha_i \text{ for some } i \in [n] \\
            a_\alpha & \text{otherwise.}
        \end{cases}
    \]
    Then, $\vec x' \in B \cap Y$. Thus, by \Cref{th:inClosureIffNeighCapANonempty},
    $\cl Y = X$. By \Cref{th:connIfBtwAandClA}, $X$ is connected.
}

\exmp{$\RR[\omega]$ in the Box Topology is Disconnected}{
    Let
    \[
        \begin{aligned}[t]
            A &= \{\,\vec x \in \RR[\omega] \mid \vec x \text{ is bounded}\,\}\text{ and} \\
            B &= \{\,\vec x \in \RR[\omega] \mid \vec x \text{ is unbounded}\,\}\text{.}  \\
        \end{aligned}
    \]
    If $\vec a$ is in either $A$ or $B$, $\prod_{i \in \ZZ_+} (a_i - 1, a_i + 1)$
    is an open set that is contained in either $A$ or $B$.
    Thus, each $A$ and $B$ are disjoint open sets in $\RR[\omega]$
    whose union is $\RR[\omega]$.
}

\section{Connected Subspaces of the Real Line}

\dfn[defLinConti]{Linear Continuum}{
    A simply ordered set $L$ having more than one element
    is called \textit{linear continuum} if the following hold:
    \begin{enumerate}[nolistsep, label=(\roman*)]
        \ii $L$ has the least upper bound property.
        \ii $\fall x, y \in L,\: (x < y \implies \exs z \in L,\: x < z < y)$.
    \end{enumerate}
}

\nt{
    $\RR$ is a linear continuum.
}

\exmp[orderSqIsLinConti]{The Ordered Square is a Linear Continuum}{
    Let $I = [0, 1]$ and $I_0^2 = I \times I$ be the ordered square
    with the dictionary ordering.
    \begin{enumerate}[nolistsep, label=(\roman*)]
        \ii Let $\OO \neq A \subseteq I_0^2$ and $\pi_1 \colon I_0^2 \to I$
            be the projection onto its first factor.
            Then, $\pi_1(A)$ is bounded above by $1$.
            Let $b \triangleq \sup \pi_1(A)$. ($[0, 1]$ has l.u.b. property.)
            
            If $b \in A$, it implies that $A \cap (\{b\} \times I) \neq \OO$
            and is bounded above by $1$.
            Thus, we may let $c \triangleq \sup \big(A \cap (\{b\} \times I)\big)$.
            One may readily check that $\sup A_0 = b \times c$.

            If $b \neq A_0$, then $b \times 0$ is
            the trivial least upper bound of $A_0$. \checkmark

        \ii Suppose $x_1 \times y_1 < x_2 \times y_2$.
            If $x_1 < x_2$, then
            $x_1 \times y_1 < (x_1+x_2)/2 \times 0 < x_2 \times y_2$.
            If $x_1 = x_2$, then, 
            $x_1 \times y_1 < x_1 \times (y_1 + y_2) / 2 < x_2 \times y_2$. \checkmark
    \end{enumerate}
}

\thm[linContiIsConn]{}{
    If $L$ is a linear continuum in the order topology,
    any convex subspace of $L$ is connected.
}
\pf{Proof}{
    Let $Y$ be a convex subspace of $L$.
    Suppose $Y = A \cup B$ is a separation of $Y$ for the sake of contradiction.
    Take any $a \in A$ and $b \in B$. \textsf{WLOG}, $a < b$.
    $[a, b] \subseteq Y$ as $Y$ is convex,
    and $[a, b]$ as a subspace of $Y$ is exactly $[a, b]$ in the order topology
    by \Cref{th:convexThenSubspaceIsOrder}.
    Hence,
    \[
        A_0 \triangleq A \cap [a, b] \quad\text{and}\quad
        B_0 \triangleq B \cap [a, b]
    \]
    form a separation of $[a, b]$.

    Let $c \triangleq \sup A_0$. Then, $c \ge a$ as $a \in A_0$, and
    $c \le b$ as, if $c$ were larger than $b$,
    there would be $z \in L$ such that $b < z < c$,
    which is an upper bound of $A_0$ smaller than $c$.
    However, we claim that $c \notin A_0 \cup B_0 = [a, b]$,
    which leads to a contradiction.

    ($c \notin A_0$) Suppose $c \in A_0$ for the sake of contradiction.
    Since $A_0$ is open in $[a, b]$,
    there must exist $e \in (c, b]$ such that $[c, e) \subseteq A_0$.
    ($e$ cannot be larger than $b$ as $b \notin A_0$.)
    As the existence of $e' \in (c, e) \cap L$ is guaranteed
    and such $e'$ is in $A_0$, $c$ is no longer an upper bound of $A_0$, \#.

    ($c \notin B_0$) Suppose $c \in B_0$ for the sake of contradiction.
    Since $B_0$ is open in $[a, b]$,
    there exists $e \in [a, c)$ such that $(e, c] \subseteq B_0$.
    ($e$ cannot be smaller than $a$ as $a \notin B_0$.)
    Since, $(c, \infty) \cap A_0 = \OO$ as $c$ is the supremum of $A_0$,
    $e$ is an upper bound of $A_0$ that is smaller than $c$, \#.
}

\cor[RisConn]{}{
    $\RR$ and intervals and rays in $\RR$ are connected.
}

\thm[ivt]{Intermediate Value Theorem}{
    Let $X$ be a connected space and $Y$ has an order topology.
    Let $f \colon X \to Y$ be a continuous map.
    Then, if $a, b \in X$ and $r \in Y$ satisfy $f(a) \le r \le f(b)$,
    there exists $c \in X$ such that $f(c) = r$.
}
\pf{Proof}{
    If $f(a) = r$ or $f(b) = r$, then done.
    So suppose $f(a) < r < f(b)$.
    $\Img f$ is connected by \Cref{th:contiSendsConnToConn}.
    Let
    \[
        A \triangleq \Img f \cap (-\infty, r) \quad\text{ and }\quad
        B \triangleq \Img f \cap (r, \infty)\text{.}
    \]   
    Then, $A$ and $B$ are open in $\Img f$ and $f(a) \in A$ and $f(b) \in B$.
    Thus, it cannot happen that $\Img f \setminus \{r\} = A \cup B = \Img f$ since $\Img f$ is connected.
    Therefore, $r \in \Img f$.
}

\dfn{Path and Path Connectedness}{
    Let $X$ be a space.
    Given $x, y \in X$, a \textit{path} in $X$
    from $x$ to $y$ is a continuous map
    $f \colon [a, b] \to X$ where $[a, b]$ is a subspace of $\RR$,
    $f(a) = x$, and $f(b) = y$.
    The space $X$ is \textit{path connected}
    if there exists a path in $X$ from $x$ to $y$
    for every $x, y \in X$.
}

\exmp[puncturedEuclIsPathConn]{Punctured Euclidean Space}{
    Define \textit{punctured Euclidean space} to be the space $\RR[n] \setminus \{\vec 0\}$,
    where $\vec 0$ is the origin in $\RR[n]$.
    If $n > 1$, the space is path connected.
    We can join $\vec x$ and $\vec y$ by the line segment that has $x$ and $y$ as endpoints
    if the segment does not go through $\vec 0$.
    Otherwise, we may choose a point $\vec x'$
    by flipping the sign of a coordinate of $\vec x$.
    We have a line that connects $\vec x$ and $\vec x'$
    and other line that connects $\vec x'$ and $\vec y$.
}

\thm[pathConnIsConn]{}{
    Every path connected space is connected.
}
\pf{Proof}{
    Let $X$ be a path connected space.
    If $X = \OO$, it is done; let $X \neq \OO$.
    Take $x \in X$.
    For each $y \in X$, let $f_y \colon [0, 1] \to X$
    be a path from $x$ to $y$.
    Since $[0, 1]$ is connected (\Cref{cor:RisConn}),
    $\Img f_y$ is connected by \Cref{th:contiSendsConnToConn}.
    As $x \in \bigcap_{y \in X} \Img f_y$,
    $X = \bigcup_{y \in X} \Img f_y$ is connected by \Cref{th:UconnIfHaveCommonPt}.
}

\exmp{Connectedness Does Not Imply Path Connectedness}{
    By \Cref{exmp:orderSqIsLinConti}, $I_0^2$ is connected.
    Suppose $I_0^2$ is path connected for the sake of contradiction.
    Then, there is a path $f \colon [0, 1] \to I_0^2$ from $0 \times 0$ to $1 \times 1$.
    \Cref{th:ivt} says that $\Img f = I_0^2$.
    For each $x \in I$, let $U_x = f\inv(\{x\} \times I)$.
    Note that $U_x \neq \OO$.
    Since each $U_x$ is open as $f$ is continuous,
    by the denseness of $\QQ$ in $\RR$, there exists
    $q_x \in U_x \cap \QQ$ for each $x \in X$.
    This implies the existence of a injection $g \colon I \to \QQ$
    defined by $x \mapsto q_x$,
    which is a contradiction as $I$ is uncountable. (\Cref{th:countSetTFAE})
}

\thm[contiSendsPathConnToPathConn]{Path Connected Space and Continuous Map}{
    Let $f \colon X \to Y$ be a continuous map.
    If $X$ is path connected, then $\Img f$ is path connected.
}
\pf{Proof}{
    Take $y_1, y_2 \in \Img f$.
    There exist $x_1, x_2 \in X$ such that $f(x_1) = y_1$ and $f(x_2) = y_2$.
    Since $X$ is connected, there exists a continuous map
    $g \colon [0, 1] \to X$ such that $g(0) = x_1$ and $g(1) = x_2$.
    Then, $f \circ g \colon [0, 1] \to \Img f$ is
    a continuous map such that $(f \circ g)(0) = y_1$ and $(f \circ g)(1) = y_2$
    by \Cref{th:constructConti}.
}
\exmp{Unit Sphere}{
    Define the \textit{unit sphere} $S^{n-1}$ in $\RR[n]$ by the equation
    \[
        S^{n-1} \triangleq \{\,\vec x \in \RR[n] \mid \|\vec x\| = 1\,\}\text{.}
    \]
    Then, the map $g \colon \RR[n] \setminus \{\vec 0\} \to S^{n-1}$ defined by
    $\vec x \mapsto \vec x / \|\vec x\|$ is a continuous surjective map.
    Moreover, if $n > 1$, since $\RR[n] \setminus \{0\}$
    is path connected (\Cref{exmp:puncturedEuclIsPathConn}),
    $S^{n-1} = \Img g$ is also path connected by \Cref{th:contiSendsPathConnToPathConn}.
}

\exmp[topSineCurve]{Topologist's Sine Curve}{
    Let
    \[
        S \triangleq \left\{\, x \times \sin \frac{1}{x} \in \RR[2] \:\bigg|\: x \in (0, 1] \right\}\text{.}
    \]
    Since $S$ is a image of $(0, 1]$ under a continuous map $x \mapsto x \times \sin (1/x)$,
    $S$ is (path) connected.
    Thus, $\cl S$ is connected by \Cref{th:connIfBtwAandClA}.
    Note that $S_0 \triangleq \cl S \setminus S = \{0\} \times [-1, 1]$.
    ($S_0$ is also closed.)

    Suppose $\cl S$ is path connected for the sake of contradiction.
    Then, there is a path $f \colon [0, 1] \to \cl S$ from $0 \times 0$ to $f(1) \in S$.
    $f\inv(S_0)$ is closed in $[0, 1]$ by \Cref{th:contiTFAE}.
    Hence $b \triangleq \sup f\inv(S_0) \in f\inv(S_0)$ and $b \neq 1$.
    $f(b) \in S_0$ and $f((b, 1]) \subseteq S$.

    Reparametrize $f \colon [0, 1] \to \cl S$ so that
    $t \mapsto x(t) \times y(t)$; $f(0) \in S_0$ and $f((0, 1]) \subseteq S$.
    ($y(t) = \sin (1/x(t))$)
    Since $x(t) > 0$ for $t \in (0, 1]$, $x$ is continuous, and $x(0) = 0$,
    we may construct a sequence $\{t_n\}_{n \in \ZZ_+}$ such that
    \[
        \lim_{n \to \infty} t_n = 0\text{,} \quad
        x(t_n) = \frac{1}{(n+1/2)\pi}\text{,} \quad\text{and thus}
    \]\[
        y(t_n) = \sin (1/x(t_n)) = \sin \big((n+1/2)\pi\big) = (-1)^n\text{.}
    \]
    However, $\{y(t_n)\}_{n \in \ZZ_+}$ diverges
    although $y$ is continuous and $t_n \to 0$.
    Thus, $\cl S$ is not path connected.
}

\section{Components and Local Connectedness}

\dfn{Component}{
    Given a space $X$, let $\sim$ be a equivalent relation defined by
    \[
        x \sim y \text{ if there is a connected subspace of } X
        \text{ containing } x \text{ and } y\text{.}
    \]
    The equivalence classes of $\sim$ is called \textit{(connected) components} of $X$.
}
\nt{
    Reflexivity follows from the fact that $\{x\}$ is a connected subspace of $X$
    that contains $x$.
    Symmetry is direct.

    Let $x, y, z \in X$ and suppose $x \sim y$ and $y \sim z$.
    There are connected subspaces $U$ and $V$ such that $x, y \in U$ and $y, z \in V$.
    Then, $U \cup V$ is a connected subspace of $X$ that contains both $x$ and $z$
    by \Cref{th:UconnIfHaveCommonPt}.
}
\nt{
    Let $\{C_{\alpha}\}_{\alpha \in J}$ be the set of components of $X$.
    Then, it is a partition of $X$ (indeed).
}

\thm[connSubsIsInSomeComp]{}{
    Let $\{C_{\alpha}\}_{\alpha \in J}$ be the set of components of $X$.
    If $A \subseteq X$ is a connected subspace of $X$,
    then $A \subseteq C_\alpha$ for some $\alpha \in J$.
}
\pf{Proof}{
    If $A = \OO$, it is done; suppose $A \neq \OO$.

    Let $C_\alpha$ and $C_\beta$ be connected components.
    If $A \cap C_\alpha \neq \OO$ and $A \cap C_\beta \neq \OO$,
    we may take $x \in A \cap C_\alpha$ and $y \in A \cap C_\beta$,
    which makes $x \sim y$.
    This implies $x_\alpha \sim x_\beta$
    for all $x_\alpha \in C_\alpha$ and $x_\beta \in C_\beta$;
    thus $C_\alpha = C_\beta$.

    Now, take any $\alpha \in A$.
    Since $\{C_\alpha\}_{\alpha \in J}$ is a partition of $X$,
    there exists some $\alpha \in J$ such that $a \in C_\alpha$.
    By the previous result, $A \cap C_\beta = \OO$
    for all $\beta \in J \setminus \{\alpha\}$.
    Hence, $A \subseteq C_\alpha$
}

\thm[componentIsConn]{}{
    Let $\{C_{\alpha}\}_{\alpha \in J}$ be the set of components of $X$.
    Then, for each $\alpha \in J$, $C_\alpha$ is connected.
}
\pf{Proof}{
    Take any $x_0 \in C_\alpha$.
    Then, for each $x \in C_\alpha$,
    there exists a connected subspace $A_x$
    that contains both $x_0$ and $x$.
    By \Cref{th:connSubsIsInSomeComp}, $A_x \subseteq C_\alpha$.
    Thus, $C_\alpha = \bigcup_{x \in C_\alpha} A_x$,
    which is connected by \Cref{th:UconnIfHaveCommonPt}.
}

\dfn{Path Component}{
    Given a space $X$, let $\sim$ be a equivalent relation defined by
    \[
        x \sim y \text{ if there is a path in } X
        \text{ from } x \text{ to } y\text{.}
    \]
    The equivalence classes of $\sim$ is called \textit{path components} of $X$.
}

\nt{
    The relation is reflexive since
    $f \colon [0, 1] \to X$ defined by $f(t) = x$ is a path from $x$ to $x$.

    The relation is symmetric since,
    if $f \colon [a, b] \to X$ is a path from $x$ to $y$,
    then $g \colon [a, b] \to X$ defined by $g(t) = f(a + b - t)$
    is a path from $y$ to $x$.

    The relation is transitive since,
    if $f \colon [a, b] \to X$ and $g \colon [c, d] \to X$ are
    paths from $x$ to $y$ and from $y$ to $z$, respectively,
    then $h \colon [a, b + d - c]$ defined by
    \[
        h(t) = \begin{cases}
            f(t)         & \text{if } a \le t \le b \\
            g(t - b + c) & \text{otherwise.}
        \end{cases}
    \]
    is a path from $x$ to $z$.
    $h$ is continuous by \Cref{th:pasting}.
}

\thm[pathConnSubsIsInSomePathComp]{}{
    Let $\{P_{\alpha}\}_{\alpha \in J}$ be the set of path components of $X$.
    If $A \subseteq X$ is a path connected subspace of $X$,
    then $A \subseteq P_\alpha$ for some $\alpha \in J$.
}
\pf{Proof}{
    Analogous to the proof of \Cref{th:connSubsIsInSomeComp}.
}

\thm[pathComponentIsPathConn]{}{
    Let $\{P_{\alpha}\}_{\alpha \in J}$ be the set of path components of $X$.
    Then, for each $\alpha \in J$, $P_\alpha$ is path connected.
}
\pf{Proof}{
    Analogous to the proof of \Cref{th:componentIsConn}.
}

\cor[pathCompIsInSomeComp]{}{
    Every path component is entirely contained in a connected component.
}
\pf{Proof}{
    Every path component is path connected by \Cref{th:pathComponentIsPathConn},
    and thus connected by \Cref{th:pathConnIsConn}.
    By \Cref{th:connSubsIsInSomeComp}, it is contained in some connected component.
}

\cor[compIsClosed]{}{
    Every component is closed.
}
\pf{Proof}{
    Let $C_\alpha$ be a connected component of $X$.
    Since $\cl{C_\alpha}$ is connected by \Cref{th:connIfBtwAandClA},
    and since $\cl{C_\alpha} \cap C_\alpha \neq \OO$,
    $\cl{C_\alpha} \subseteq C_\alpha$ by \Cref{th:connSubsIsInSomeComp}.
}

\cor[compIsOpenIfFiniteNumComps]{}{
    If there are a finite number of components,
    then each component is open.
}
\pf{Proof}{
    Let $X = \bigcup_{i=1}^n C_i$ where each $C_i$ is a component.
    Then, for each $i \in [n]$, $C_i = X \setminus \bigcup_{j \in [n] \setminus \{i\}} C_j$.
    $C_i$ is open as $\bigcup_{j \in [n] \setminus \{i\}} C_j$ is closed by \Cref{cor:compIsClosed}.
}

\exmp{Path Component Is Not Necessarily Open or Closed}{
    Let $\cl S$ be the topologist's sine curve discussed in \Cref{exmp:topSineCurve}.
    Then, $S$ and $S_0$ are the two path components of $\cl S$.
    $S$ is not closed and $S_0$ is not open.
}

\exmp{}{
    Let $A \triangleq S \cup (S_0 \setminus \{0\} \times \QQ)$.
    Since $S \subseteq A \subseteq \cl S$, $A$ is connected by \Cref{th:connIfBtwAandClA}.
    However, $\{0 \times r\}$ for every  $r \in [0, 1] \setminus \QQ$
    is a path component. Thus, $A$ has uncountably many path components.
}

\dfn{Locally Connected Space}{
    Let $X$ be a topological space.
    $X$ is \textit{locally connected at} $x$
    if, for any neighborhood $U$ of $x$,
    there exists a connected neighborhood $V$ of $x$
    such that $x \in V \subseteq U$.
    $X$ is \textit{locally connected}
    if $X$ is locally connected at every point of $X$.
}

\dfn{Locally Path Connected Space}{
    Let $X$ be a topological space.
    $X$ is \textit{locally path connected at} $x$
    if, for any neighborhood $U$ of $x$,
    there exists a path connected neighborhood $V$ of $x$
    such that $x \in V \subseteq U$.
    $X$ is \textit{locally path connected}
    if $X$ is locally path connected at every point of $X$.
}

\nt{
    If a topological space $X$ is locally path connected,
    then it is locally connected as well.
}

\thm[localConnIff]{}{
    A topological space $X$ is locally connected
    if and only if, for every open set $U$ in $X$,
    each connected component of $U$ is open.
}
\pf{Proof}{
    ($\Rightarrow$)
    Let $U$ be open in $X$ and let $\{C_\alpha\}_{\alpha \in J}$
    be the set of components of $U$.
    Take any $C_\alpha$ and let $x \in C_\alpha$.
    Since $X$ is locally connected at $x$,
    there exists a connected neighborhood $V$ of $x$
    such that $x \in V \subseteq U$.
    By \Cref{th:connSubsIsInSomeComp},
    $x \in V \subseteq C_\alpha$.
    This proves that $C_\alpha$ is open.

    ($\Leftarrow$)
    Let $x \in X$ and $U$ be a neighborhood of $x$.
    Let $\{C_\alpha\}_{\alpha \in J}$ be the components of $U$.
    There exists some $\alpha_0 \in J$ such that $x \in C_{\alpha_0}$.
    Since $C_{\alpha_0}$ is open by assumption,
    $C_{\alpha_0}$ is a connected neighborhood of $x$
    and satisfies $x \in C_{\alpha_0} \subseteq U$.
}

\thm[localPathConnIff]{}{
    A topological space $X$ is locally path connected
    if and only if, for every open set $U$ in $X$,
    each path component of $U$ is open.
}
\pf{Proof}{
    Analogous to \Cref{th:localConnIff}.
}

\thm[localPathConnThenPathCompIsComp]{}{
    Let $X$ be a locally path connected space.
    Then, the connected components and the path components are the same.
}
\pf{Proof}{
    Let $C$ be a connected component of $X$.
    $C$ is open by \Cref{th:localConnIff} as $X$ is locally connected.
    Let $x \in C$ and let $P$ be the path component which $x$ is contained in.
    Then, $P \subseteq C$ by \Cref{cor:pathCompIsInSomeComp}.
    
    Suppose $P \subsetneq C$ for the sake of contradiction.
    Let
    \[
        Q \triangleq \bigcup \{\, \hat P \subseteq C \mid \hat P \text{ is a path component of }X
        \text{ and } \hat P \neq P\,\}\text{.}
    \]
    Since path component of an open set, especially, $C$, is open by \Cref{th:localPathConnIff},
    $P$ and $Q$ are open.
    Moreover, since $C = P \cup Q$, they form a separation of $C$,
    which is a contradiction, \#.
}

\section{Compact Spaces}

\dfn{Open Cover}{
    A collection $\mcal A$ of subsets of a space $X$ is said to \textit{cover} $X$,
    or to be a \textit{covering} of $X$, if $\bigcup \mcal A = X$.
    It is called an \textit{open covering} if $A$ is open in $X$ for each $A \in \mcal A$.
}

\dfn{Compactness}{
    A space $X$ is said to be \textit{compact} if every open covering $\mcal A$ of $X$
    contains a finite subcollection that also covers $X$.
}

\exmp{$\RR$ Is Not Compact}{
    The open cover $\mcal A \triangleq \{\,(n, n+2) \mid n \in \ZZ\,\}$
    does not have a finite subcollection that covers $\RR$.
    Thus, $\RR$ is not compact.
}

\mlemma[subsCompIff]{}{
    Let $Y$ be a subspace of $X$.
    Then $Y$ is compact if and only if
    every covering of $Y$ by sets open in $X$
    contains a finite subcollection covering $Y$.
}
\pf{Proof}{
    ($\Rightarrow$)
    Let $\mcal A = \{A_\alpha\}_{\alpha \in J}$
    is a covering of $Y$ by open sets in $X$.
    Then, the collection
    \[
        \{\,A_\alpha \cap Y \mid \alpha \in J\,\}
    \]
    is an open covering of $Y$.
    Thus, there exists a finite subcollection
    \[
        \{\,A_{\alpha_1} \cap Y, \cdots, A_{\alpha_n} \cap Y\,\}
    \]
    that covers $Y$.
    Then, $\{\,A_{\alpha_1}, \cdots, A_{\alpha_n}\,\}$
    is a finite subcollection of $\mcal A$ that covers $Y$.

    ($\Leftarrow$)
    Let $\mcal A = \{A_{\alpha}\}_{\alpha \in J}$
    be an open covering of $Y$.
    For each $\alpha \in J$, there is an open set $\hat A_\alpha$ in $X$
    such that $A_\alpha = \hat A_\alpha \cap Y$.
    Then, the collection $\{\hat A_\alpha\}_{\alpha \in J}$
    composed of open sets in $X$ that covers $Y$;
    by the assumption, there exists a finite subcollection
    \[
        \{\,\hat A_{\alpha_1}, \cdots, \hat A_{\alpha_n}\,\}
    \]
    that covers $Y$.
    Then, $\{\,A_{\alpha_1}, \cdots, A_{\alpha_n}\,\}$
    is a finite subcollection of $\mcal A$ that covers $Y$.
}

\thm[cldInCptIsCpt]{}{
    Let $X$ be a compact space.
    If $Y$ is a closed subset of $X$,
    then $Y$ as a subspace of $X$ is compact.
}
\pf{Proof}{
    If $Y = \OO$, then it is done. So, suppose $Y \neq \OO$.
    Let $\mcal A$ be a covering of $Y$ composed of sets open in $X$.
    \[
        \mcal B \triangleq \mcal A \cup \{X \setminus Y\}
    \]
    is an open covering of $X$.
    Thus, it has a finite subcollection
    \[
        \{\,A_1, A_2, \cdots, A_n, X \setminus Y\,\}
    \]
    that covers $X$ where $A_i \in \mcal A$ for each $i \in [n]$.
    (\textsf{WLOG}, $X \setminus Y$ is in the subcollection
    since we may just add $X \setminus Y$ and does not affect its finiteness.)
    Then, $\{A_i\}_{i \in [n]}$ is a finite subcollection of $\mcal A$
    that covers $Y$.
}

\thm[cptInHausIsCld]{}{
    Let $X$ be a Hausdorff space.
    If $Y \subseteq X$ is a compact subspace of $X$,
    then $Y$ is closed in $X$.
}
\pf{Proof}{
    If $Y = \OO$ or $Y = X$, then it is done; suppose $\OO \neq Y \subsetneq X$.
    Let $x_0 \in X \setminus Y$.
    For each $y \in Y$, there are disjoint neighborhoods $U_y$ and $V_y$
    of $x_0$ and $y$ in $X$.
    Then, $\{V_y\}_{y \in Y}$ is an open covering of $Y$.
    Thus there exists a finite subcollection of it
    \[
        \{\,V_{y_1}, V_{y_2}, \cdots, V_{y_n}\,\}
    \]
    that covers $Y$.

    Let
    \[
        V \triangleq \bigcup_{i=1}^n V_{y_i} \quad\text{and}\quad
        U \triangleq \bigcap_{i=1}^n U_{y_i}\text{.}
    \]
    Then, $U$ is a neighborhood of $x_0$ and does not intersect $V$,
    which covers $Y$. Thus, $U \subseteq X \setminus Y$.
    Hence, $X \setminus Y$ is open; $Y$ is closed.
}

\exmp{Being Hausdorff Is Needed}{
    Let $X = \RR$ be endowed with the finite complement topology.
    Then, every subset of $X$ is compact.
    To see this, suppose $\mcal A$ is a collection of open sets
    in $X$ that covers $Y \subseteq X$.
    Then, take any $A \in \mcal A$ and
    it will cover all but finitely many points in $Y$.
    For each remaining point, choose an open set in $\mcal A$
    that contains the point.
    Thus, we get a finite collection of $\mcal A$
    that covers $Y$.
    However, only closed sets are finite subsets of $X$ and $\RR$.
}

\cor[cptAndPntOutside]{}{
    Let $X$ be a Hausdorff space.
    If $Y \subseteq X$ is a compact subspace of $X$,
    then, given any $x_0 \in X \setminus Y$,
    there are disjoint open sets $U$ and $V$ in $X$
    containing $x_0$ and $Y$, respectively.
}
\pf{Proof}{
    $U$ and $V$ defined in the proof of \Cref{th:cptInHausIsCld}
    are those.
}

\thm[contiSendsCptToCpt]{}{
    Let $X$ be a compact space.
    Let $f \colon X \to Y$ be a continuous map.
    Then, $\Img f$ as a subspace of $Y$ is compact.
}
\pf{Proof}{
    Let $\mcal A$ be a covering of the set $\Img f$
    by sets open in $Y$. Then, the collection
    \[
        \{\,f\inv(A) \mid A \in \mcal A\,\}
    \]
    is an open covering of $X$ as $f$ is continuous.
    Hence, there are a finite subcollection $\{\,A_1,\cdots,A_n\,\}$ of $\mcal A$
    such that $\{\,f\inv(A_i)\,\}_{i \in [n]}$ covers $X$.
    The sets $\{\,A_1, \cdots, A_n\,\}$ covers $\Img f$.
}

\thm[bijContiFromCompToHausIsHomeo]{}{
    Let $f \colon X \to Y$ be a bijective continuous function.
    If $X$ is compact and $Y$ is Hausdorff, then $f$ is a homeomorphism.
}
\pf{Proof}{
    We only need to prove $f\inv$ is continuous.
    Let $A \subseteq X$ is closed in $X$.
    Then, $A$ is compact by \Cref{th:cldInCptIsCpt}.
    Thus, since $f \big\vert_A \colon A \to Y$ is continuous (\Cref{th:constructConti}),
    $f(A)$ is compact by \Cref{th:contiSendsCptToCpt}.
    By \Cref{th:cptInHausIsCld}, $f(A)$ is closed.
    Hence, we proved that $f(A)$ is closed for each closed subset $A$ of $X$;
    $f\inv$ is continuous by \Cref{th:contiTFAE}.
}

\mlemma[tube]{The Tube Lemma}{
    Let $X$ and $Y$ be topological spaces and $Y$ is compact.
    Given any $x_0 \in X$ and an open set $N$ in $X \times Y$
    that contains $\{x_0\} \times Y$,
    there exists a neighborhood $W$ of $x_0$ in $X$
    such that $W \times Y \subseteq N$.
}
\pf{Proof}{
    For each $y \in Y$,
    there exists a basis element $U_y \times V_y$
    in the product topology such that
    $x_0 \times y \in U_y \times V_y \subseteq N$.
    Then, $\mcal A \triangleq \{\,U_y \times V_y \mid y \in Y\,\}$
    is a covering of $\{x_0\} \times Y$ by
    open sets in $X \times Y$.
    Since $\{x_0\} \times Y$,
    being homeomorphic with $Y$, is compact,
    there is a finite subcollection
    \[
        \mcal A' = \{\,U_{y_1} \times V_{y_1}, \cdots, U_{y_n} \times V_{y_n}\,\}
    \]
    of $\mcal A$ that covers $\{x_0\} \times Y$.
    Note that $\{x_0\} \times Y \subseteq \bigcup_{i=1}^n (U_{y_i} \times V_{y_i}) \subseteq N$.
    Let
    \[
        W \triangleq \bigcap_{i=1}^n U_{y_i}\text{.}
    \]
    Then, $W$ is a neighborhood of $x_0$ in $X$.

    Now, take $x \times y \in W \times Y$.
    There exists some $i \in [n]$ such that $y \in V_{y_i}$;
    $x \times y \in U_{y_i} \times V_{y_i} \subseteq N$.
    This shows $W \times Y \subseteq N$.
}

\nt{
    The set $W \times Y$ is often called a \textit{tube} about $x_0 \times Y$.
}

\nt{
    \Cref{lem:tube} may not hold if $Y$ is not compact.
    If $X = Y = \RR$, the open set
    \[
        N \triangleq \left\{\,x \times y \in \RR[2] \:\left|\: |x| < \frac{1}{y^2+1} \right.\,\right\}
    \]
    does contain $\{0\} \times Y$ but there is no open neighborhood $W$ of $0$ in $X$
    such that $W \times Y \subseteq N$.
}

\thm[prodCptIffEachCpt]{}{
    Let $X_1, X_2, \cdots, X_n$ be topological spaces.
    Then, $\prod_{i=1}^n X_i$ is compact if and only if $X_i$ is compact for each $i \in [n]$.
}
\pf{Proof}{
    It is enough to prove for two topological spaces $X$ and $Y$.

    ($\Rightarrow$)
    It is enough to prove $X$ is compact.
    Let $\mcal A$ be an open covering of $X$.
    Then, $\{\,A \times Y \mid A \in \mcal A\,\}$
    is an open covering of $X \times Y$;
    there exists a finite subcollection
    \[
        \{\,A_1 \times Y, A_2 \times Y, \cdots, A_n \times Y\,\}
    \]
    that covers $X \times Y$.
    Thus, $\{\,A_i \mid i \in [n]\,\}$ is a finite subcollection of $\mcal A$
    that covers $X$.

    ($\Leftarrow$)
    Let $\mcal A$ be an open covering of $X \times Y$.
    For each $x \in X$, since $\{x\} \times Y$ is compact,
    there are finite subcollection $\{\,A_1, A_2, \cdots, A_{n_x}\,\} \subseteq \mcal A$
    that covers $\{x\} \times Y$.
    Then, $N_x \triangleq \bigcup_{i=1}^{n_x} A_i$ is an open set in $X \times Y$
    that contains $\{x\} \times Y$.
    Thus, by \Cref{lem:tube}, there exists a tube $W_x \times Y$
    such that $\{x\} \times Y \subseteq W_x \times Y \subseteq N_x$.
    
    Noting that $\{\,W_x \mid x \in X\,\}$ is an open covering of $X$,
    there are finite subcover $\{\,W_{x_1}, W_{x_2}, \cdots, W_{x_k}\,\}$
    that covers $X$.
    Hence, $\{\,W_{x_i} \times Y \mid i \in [k]\,\}$ covers $X \times Y$
    and each element of it is covered by finite elements in $\mcal A$.
}

\nt{
    \Cref{th:prodCptIffEachCpt} holds for an arbitrary product.
    See \Cref{th:tychonoff}.
}

\dfn{Finite Intersection Property}{
    A collection $\mcal C$ of subsets of $X$
    is said to have \textit{finite intersection property}
    if, for any finite subcollection
    \[
        \{\,C_1, C_2, \cdots, C_n\,\} \subseteq \mcal C
    \]
    of $\mcal C$, we have
    \[
        \bigcap_{i=1}^n C_i \neq \OO\text{.}
    \]

    In other words,
    \[
        \fall n \in \ZZ_+,\: \fall \mcal C' \in \binom{\mcal C}{n},\:
        \bigcap \mcal C' \neq \OO\text{.}
    \]
}

\thm[cptIff]{}{
    Let $X$ be a topological space.
    Then $X$ is compact if and only if,
    for every collection $\mcal C$ of closed sets in $X$
    having the finite intersection property,
    the intersection $\bigcap \mcal C$ is nonempty.
}
\pf{Proof}{
    Given a collection $\mcal A$ of subsets of $X$, let
    \[
        \mcal C \triangleq \{\,X \setminus A \mid A \in \mcal A\,\}\text{.}
    \]
    Then the following hold.
    \begin{itemize}[nolistsep]
        \ii $\mcal A$ is a collection of open sets if and only if $\mcal C$ is a collection of closed sets.
        \ii $\bigcup \mcal A = X$ if and only if $\bigcap \mcal C = \OO$.
        \ii The finite subcollection $\{\,A_1, \cdots, A_n\,\}$ covers $X$
            if and only if $\bigcap_{i=1}^n (X \setminus A_i) = \OO$.
    \end{itemize}
    Therefore, these are equivalent.
    \begin{enumerate}[nolistsep, label=(\roman*)]
        \ii Every open covering of $X$ allows a finite subcover.
        \ii A collection of open sets in $X$ that does not allow a finite subcover does not cover $X$.
            - contrapositive of (i)
        \ii A collection of closed sets in $X$ that does not allow a nonempty intersection of finite subcollection
            does not have a nonempty intersection.
    \end{enumerate}
}

\dfn{Nested Sequence}{
    A sequence of sets $\{C_n\}_{n \in \ZZ_+}$ is called a \textit{nested sequence} if
    $C_n \supseteq C_{n+1}$ for each $n \in \ZZ_+$.
}

\cor[cldNestInCpt]{}{
    Let $X$ be a compact space.
    Let $\{C_n\}_{n \in \ZZ_+}$ be a nested sequence of nonempty closed sets in $X$.
    Then, \[
        \bigcap_{n \in \ZZ_+} C_n \neq \OO\text{.}
    \]
}
\pf{Proof}{
    Let $\mcal C \triangleq \{\,C_n \mid n \in \ZZ_+\,\}$.
    Then, $\mcal C$ satisfies the finite intersection property as
    \[
        C_{n_1} \cap C_{n_2} \cap \cdots \cap C_{n_k}
        = C_{\max_{i=1}^k n_i} \neq \OO\text{.}
    \]
    The result follows from \Cref{th:cptIff}.
}

\thm[cptMetIsBdd]{}{
    Let $(X, d)$ be a compact metric space.
    Then, $X$ is bounded with respect to $d$.
}
\pf{Proof 1}{
    Let $\{B(x_1, 1), B(x_2, 1), \cdots, B(x_n, 1)\}$ be a finite open covering
    of $X$ by $1$-balls. Let
    \[
        M \triangleq \max_{i, j \in [n]} d(x_i, x_j)\text{.}
    \]
    Take any $x, y \in X$.
    Then, there are $i, j \in [n]$ such that $x \in B(x_i, 1)$ and
    $y \in B(x_j, 1)$.
    Then,
    \[
        d(x, y) \le d(x, x_i) + d(x_i, x_j) + d(x_j, y)
        < M + 2\text{.}
    \]
    Hence, $X$ is bounded with respect to $d$.
}
\pf{Proof 2}{
    Suppose $X$ is unbounded under $d$.
    Choose any $x_0 \in X$.
    Then, $\{B(x_0, n)\}_{n \in \ZZ_+}$ is an open covering of $X$
    but it does not allow a finite subcover.
}

\section{Compact Subspaces of the Real Line}

\thm[cldIntvIsCpt]{}{
    Let $X$ be a simply ordered set having the least upper bound property.
    In the order topology, every closed interval $[a, b]$ in $X$ is compact.
}
\pf{Proof}{
    Let $\mcal A$ be an open covering of $[a, b]$.

    We claim that, given any $x \in [a, b)$,
    there exists $y \in (x, b]$ such that $[x, y]$ can be covered by
    at most two elements of $\mcal A$.
    \begin{enumerate}[nolistsep, label=(\roman*)]
        \ii If there exists an immediate successor $y \in (x, b]$ of $x$,
            then $[x, y] = \{x, y\}$.
            Pick two open sets in $\mcal A$ that contain $x$ and $y$, respectively.
        \ii Otherwise, let $A \in \mcal A$ with $x \in A$.
            Then, $[x, c) \subseteq A$ for some $c \in (x, b]$ and $|[x, c)| = \infty$.
            Take any $y \in (x, c) \subseteq (x, b]$, then $[x, y] \subseteq [x, c) \subseteq A$. \checkmark
    \end{enumerate}
    Let
    \[
        C \triangleq \big\{\, y \in (a, b] \:\big|\: [a, y] \text{ can be
        covered by finitely many elements of } \mcal A\,\big\}\text{.}
    \]
    By the previous claim, $C \neq \OO$, and $C$ is bounded above by $b$.
    Thus, we may let $c \triangleq \sup C$. ($a \le c \le b$, indeed.) \checkmark

    Suppose $c \notin C$ for the sake of contradiction.
    Choose $A \in \mcal A$ that contains $c$.
    Then, there exists $d \in [a, c)$ such that $(d, c] \subseteq A$.
    Hence, there exists $z \in C \cap (d, c]$.
    Since $z \in C$, the interval $[a, z]$ can be covered by finitely many,
    say $n$, elements of $\mcal A$,
    then, since $[a, c] = [a, z] \cup [z, c]$ and $[z, c] \subseteq (d, c] \subseteq A$,
    $[a, c]$ can be covered by at most $n + 1$ elements of $\mcal A$,
    which is contradicting to $c \notin C$, \#. \checkmark

    Suppose $c < b$ for the sake of contradiction.
    Then, there exists $y \in (c, b]$ such that
    $[c, y]$ can be covered by finitely many elements of $\mcal A$
    by the previous claim.
    Hence, $[a, y] = [a, c] \cup [c, y]$ can be covered by
    finitely many elements of $\mcal A$ since $c \in C$.
    This implies $y \in C$, contradicting that
    $c$ is an upper bound of $C$, \#. \checkmark
}

\exmp[orderedSqIsCpt]{}{
    The ordered square $I_o^2 = [0 \times 0, 1 \times 1]$ is compact.
}

\cor[cldIntvInRIsCpt]{}{
    Every closed interval in $\RR$ is compact.
}

\thm[cptInRnIffCldAndBdd]{}{
    A subspace $A$ of $\RR[n]$ is compact if and only if
    it is closed and it is bounded in the Euclidean metric $d$
    or the square metric $\rho$.
}
\pf{Proof}{
    It suffices to prove only for $\rho$
    as $A$ is bounded in $d$ if and only if $A$ is bounded in $\rho$.
    (See the proof of \Cref{th:eucliAndSqareMetricAreProduct}.)

    ($\Rightarrow$)
    By \Cref{th:cptInHausIsCld}, $A$ is closed. \checkmark

    The collection
    \[
        \{\,B_{\rho}(\vec 0, m) \mid m \in \ZZ_+\,\}
    \]
    is an open covering of $A$.
    Thus, $A \subseteq B_{\rho}(\vec 0, M)$ for some $M$.
    Therefore, $\rho(\vec x, \vec y) \le 2M$ for each $\vec x, \vec y \in A$.
    Thus, $A$ is bounded. \checkmark

    ($\Leftarrow$)
    There exists $M \in \RR_+$ such that
    $\rho(\vec x, \vec y) \le M$ for each $\vec x, \vec y \in A$.
    Choose a point $\vec x_0 \in A$, and let $b \triangleq \rho(\vec x_0, \vec 0)$.
    Then, $\rho(\vec x, \vec 0) \le P \triangleq N + b$ for every $\vec x \in A$.
    Thus, $A \subseteq [-P, P]^n$.
    $[-P, P]^n$ is compact by
    \Cref{cor:cldIntvInRIsCpt}, \Cref{th:prodCptIffEachCpt}, and \Cref{th:subOfProdIsProdOfSub}.
    Since $A$ is closed in $[-P, P]^n$ and $[-P, P]^n$ is compact,
    $A$ is compact by \Cref{th:cldInCptIsCpt}.
}

\thm[evt]{Extreme Value Theorem}{
    Let $X$ be a compact set and
    $Y$ be an ordered set endowed by the order topology.
    Let $f \colon X \to Y$ be a continuous map.
    Then, there exist $c, d \in X$ such that
    $f(c) \le f(x) \le f(d)$ for all $x \in X$.
}
\pf{Proof}{
    Suppose $\Img f$ does not have a maximum.
    Then,
    \[
        \{\, (-\infty, a) \subseteq \RR \mid a \in \Img f\,\}
    \]
    is an open covering of $\Img f$.
    Since $\Img f$ is compact by \Cref{th:contiSendsCptToCpt},
    $\Img f \subseteq (-\infty, a)$ for some $a \in \Img f$, \#.
}

\dfn{Distance From a Point to a Set}{
    Let $(X, d)$ be a metric space and let $\OO \neq A \subseteq X$.
    For each $x \in X$, we define the
    \textit{distance from $x$ to $A$} by the equation
    \[
        d(x, A) \triangleq \inf \{\,d(x, a) \mid a \in A\,\}\text{.}
    \]
}

\dfn{Uniform Continuity}{
    A function $f \colon X \to Y$ from the metric space $(X, d_X)$
    to the metric space $(Y, d_Y)$ is said to be \textit{uniformly continuous} if
    \[
        \fall \veps \in \RR_+,\: \exs \delta \in \RR_+,\:
        \fall x_1, x_2 \in X,\:
        \big( d_X(x_1, x_2) < \delta \implies d_Y(f(x_1), f(x_2)) \big)\text{.}
    \]
}

\thm[distFtnIsConti]{}{
    Let $(X, d)$ be a metric space and let $\OO \neq A \subseteq X$.
    Then, $f \colon X \to \RR$ defined by
    \[
        f(x) \triangleq d(x, A)
    \]
    is uniformly continuous.
}
\pf{Proof}{
    Take any $\veps \in \RR_+$ and let $\delta \triangleq \veps$.
    For any $x, y \in X$ and $a \in A$ with $d(x, y) < \veps$,
    we have $d(x, A) \le (x, a) \le d(x, y) + d(y, a)$.
    Thus,
    \[
        d(x, A) - d(x, y) \le \inf_{a \in A} d(y, a) = d(y, A)\text{,}
    \]
    which implies $|d(x, A) - d(y, A)| \le d(x, y) < \delta = \veps$.
}

\mlemma[lebesgueN]{The Lebesgue Number Lemma}{
    Let $(X, d)$ be a compact metric space.
    Then, for each open covering $\mcal A$ of $X$,
    \[
        \exs \delta \in \RR_+,\: \fall B \in \mcal P(X) \setminus \{\OO\},\:
        \big( \diam B < \delta \implies
        \exs A \in \mcal A,\: B \subseteq A\big)\text{.}
    \]
    The number $\delta$ is called a \textit{Lebesgue number} for the covering $\mcal A$.
}
\pf{Proof}{
    If $X \in \mcal A$, then every $\delta \in \RR_+$ satisfies the condition.
    Therefore, we may suppose $X \notin \mcal A$.

    Choose a finite subcollection $\{\,A_1, A_2, \cdots, A_n\,\}$ of $\mcal A$
    that covers $X$.
    For each $i \in [n]$, let $C_i \triangleq X \setminus A_i$.
    We define $f \colon X \to \RR$ by
    \[
        f(x) = \frac{1}{n} \sum_{i=1}^{n} d(x, C_i)\text{.}
    \]
    
    Take any $x \in X$. Then, there exists some $i \in [n]$
    such that $x \in A_i$.
    Since $A_i$ is open, there exists some $\veps \in \RR_+$
    such that $B(x, \veps) \subseteq A_i$; $d(x, C_i) \ge \veps$.
    Hence, $f(x) \ge \veps/n$.
    We just showed that $f(x) > 0$ for all $x \in X$.

    Since $f$ is continuous,
    there exists a minimum of $\Img f$, say $\delta$, by
    \Cref{th:evt}.
    We claim that $\delta$ is a Lebesgue number for $\mcal A$.
    Let $\OO \neq B \subseteq X$ with $\diam B < \delta$.
    Take $x_0 \in B$. Then $B \subseteq B(x_0, \delta)$.
    Then,
    \[
        \delta \le f(x_0) \le \max_{i \in [n]} d(x_0, C_i) = d(x_0, C_m)\text{.}
    \]
    where $m \in [n]$.
    Then, $B \subseteq B(x_0, \delta) \subseteq A_m$.
}

\thm[unifContiThm]{Uniform Continuity Theorem}{
    Let $(X, d_X)$ be a compact metric space; let $(Y, d_Y)$ be a metric space.
    If $f \colon X \to Y$ is a continuous map,
    then $f$ is uniformly continuous.
}
\pf{Proof}{
    Take any $\veps \in \RR_+$.
    Let
    \[
        \mcal A \triangleq \big\{\,f\inv\big(B(y, \veps/2)\big) \:\big|\: y \in Y\,\big\}
    \]
    be an open covering of $X$.
    Let $\delta$ be a Lebesgue number for $\mcal A$.
    Then, for each $x_1, x_2 \in X$ such that $d_X(x_1, x_2) < \delta$,
    since $\diam \{x_1, x_2\} = d_X(x_1, x_2) < \delta$,
    there exists $y \in Y$ such that $\{f(x_1),f(x_2)\} \subseteq B(y, \veps/2)$.
    Then, $d_Y(f(x_1), f(x_2)) < \veps$.
}

\dfn{Isolated Point}{
    If $X$ is a topological space, a point $x \in X$ is said to be an
    \textit{isolated point} of $X$ if $\{x\}$ is open in $X$.
}

\mlemma[cptHausNoIsolatedIsUncountLem]{}{
    Let $X$ be a nonempty Hausdorff space
    which has no isolated points.
    Then, for any nonempty open set $U$ of $X$
    and $x \in X$, there exists a nonempty open set $V$ contained in $U$
    such that $x \notin \cl V$.
}
\pf{Proof}{
    Take any $y \in U \setminus \{x\}$.
    (This is possible since $U \neq \{x\}$.)
    Choose disjoint neighborhoods $W_1$ and $W_2$ of $x$ and $y$, respectively.
    Then, $V = W_2 \cap U$ is the open set we are looking for.
    $V$ is empty, nonempty as $y \in V$, and its closure does not contain $x$
    by \Cref{th:inClosureIffNeighCapANonempty}.
}

\thm[cptHausNoIsolatedIsUncount]{}{
    Let $X$ be nonempty compact Hausdorff space.
    If $X$ has no isolated points, then $X$ is uncountable.
}
\pf{Proof}{
    Now let $f \colon \ZZ_+ \to X$ be any function.
    Let $V_0 = X$. Construct $V_1, V_2, \cdots$ as following.
    \begin{itemize}[nolistsep]
        \ii For each $n \in \ZZ_+$, choose $V_n$ to be a nonempty set such that $V_n \subseteq V_{n-1}$
            and $f(n) \notin \cl{V_n}$.
    \end{itemize}
    This is possible thanks to \Cref{lem:cptHausNoIsolatedIsUncountLem}.

    Now, we have a nested sequence $\{\cl{V_n}\}_{n \in \ZZ_+}$ of closed sets in $X$.
    By \Cref{cor:cldNestInCpt}, there exists $x \in \bigcap_{n \in \ZZ_+} \cl{V_n}$.
    Then, $x \neq f(n)$ for all $n \in \ZZ_+$
    as $x \in \cl{V_n}$ and $f(n) \notin \cl{V_n}$.
}

\cor{}{
    Every closed interval in $\RR$ is uncountable.
}

\section{Limit Point Compactness}

\dfn{Limit Point Compactness}{
    A topological space $X$ is said to be \textit{limit point compact}
    if every infinite subset $A$ of $X$ has $A' \neq \OO$.
}

\thm[cptThenLimPtCpt]{}{
    If a topological space $X$ is compact, then it is limit point compact.
}
\pf{Proof}{
    Suppose $A$ has no limit point.
    Then, by \Cref{cor:closedIffContainsLimPts}, $A$ is closed in $X$.
    Moreover, for each $a \in A$, there exists a neighborhood $U_a$ of $a$
    such that $U_a \cap A = \{a\}$ by the definition of limit point.
    Then,
    \[
        \mcal A \triangleq \{\,X \setminus A\,\} \cup \{\,U_a \mid a \in A\,\}
    \]
    is an open covering of $X$.
    Since $X$ is compact, there is a finite subcollection of $\mcal A$
    that covers $X$.
    As $X \setminus A$ does not intersect $A$,
    only finite number of open sets of the form $U_a$ covers $A$,
    which means $A$ is finite.
}

\exmp{Limit Point Compactness Does Not Imply Compactness}{
    Let $Y = \{a, b\}$ and give $Y$ the trivial topology.
    Any nonempty subset $A$ of $X$ has a limit point, for if
    $(n, a) \in A$ or $(n, b) \in A$, $(n, b)$ or $(n, a)$ is a limit point of $A$.
    Thus, the space $X \triangleq \ZZ_+ \times Y$ is limit point compact.
    However, it is not compact as the open covering
    \[
        \mcal A \triangleq \{\,\{n\} \times Y \mid n \in \ZZ_+\,\}
    \]
    does not have a finite subcover.
}

\dfn{Sequentially Compact}{
    A topological space $X$ is said to be \textit{sequentially compact}
    if any sequence $\{x_n\}_{n \in \ZZ_+}$ in $X$
    has a convergent subsequence.
}

\mlemma[lebesgueNForSeqCpt]{The Lebesgue Number Lemma For Sequentially Compact Spaces}{
    Let $(X, d)$ be a sequentially compact metric space.
    Then, for each open covering $\mcal A$ of $X$,
    \[
        \exs \delta \in \RR_+,\: \fall B \in \mcal P(X) \setminus \{\OO\},\:
        \big( \diam B < \delta \implies
        \exs A \in \mcal A,\: B \subseteq A\big)\text{.}
    \]
}
\pf{Proof}{
    Suppose to the contrary that there does not exist such $\delta$.
    Let $\mcal A$ be an open covering of $X$.
    Therefore, for each $n \in \ZZ_+$,
    there exists a nonempty subset $C_n$ of $X$ such that
    $\diam C_n < 1/n$ and there is no $A \in \mcal A$
    such that $C_n \subseteq A$.

    Choose a point $x_n \in C_n$.
    Since $X$ is sequentially compact,
    $\{x_n\}$ has a convergent subsequence $\{x_{n_i}\}_{i \in \ZZ_+}$.
    Let $x$ be a point to which the subsequence converges.
    (Such point is unique by \Cref{th:seqConvUniqueInHaus}.)

    Then, $x \in A$ for some $A \in \mcal A$.
    Thus, there exists $\veps \in \RR_+$ such that
    $x \in B(x, \veps) \subseteq A$.
    By the convergence, there exists $i \in \ZZ_+$
    such that $x_{n_{i}} \in B(x, \veps/2)$ and $1/n_i < \veps/2$.
    Then, $\diam C_{n_i} < 1/n_i < \veps/2$; hence
    $C_{n_i} \subseteq B(x_{n_i}, \veps/2) \subseteq B(x, \veps) \subseteq A$, \#.
}

\mlemma[seqCptCoverByFinEpsBalls]{}{
    Let $(X, d)$ be a sequentially compact metric space.
    Then, for each $\veps \in \RR_+$,
    there exists a finite subset $A$ of $X$
    such that $\{\,B(a, \veps) \mid a \in A\,\}$ covers $X$.
}
\pf{Proof}{
    Suppose there does not exist such finite cover for the sake of contradiction.
    Construct a sequence $\{x_n\}_{n \in \ZZ_+}$ in $X$ as following.
    \begin{itemize}[nolistsep]
        \ii $x_1 \in X$
        \ii For each $n \in \ZZ_+$, $x_{n+1} \in X \setminus \bigcup_{i=1}^n B(x_i, \veps)$.
    \end{itemize}
    This is possible as $\bigcup_{i=1}^n B(x_i, \veps) \subsetneq X$ for every $n \in \ZZ_+$.
    However, since $d(x_m, x_n) \ge \veps$ for every distinct $n, m \in \ZZ_+$ by construction,
    every $\veps/2$-ball may contain at most one $x_n$.
    Hence, $\{x_n\}$ has no convergent subsequence.
}

\thm[cptTFAE]{}{
    Let $X$ be a metrizable space. \textsf{TFAE}
    \begin{enumerate}[nolistsep, label=(\roman*)]
        \ii $X$ is compact.
        \ii $X$ is limit point compact.
        \ii $X$ is sequentially compact.
    \end{enumerate}
}
\pf{Proof}{
    (i) $\Rightarrow$ (ii) is already proved by \Cref{th:cptThenLimPtCpt}. \checkmark

    ((ii) $\Rightarrow$ (iii))
    Let $\{x_n\}_{n \in \ZZ_+}$ be a sequence in $X$.
    Consider the set $A \triangleq \{\,x_n \mid n \in \ZZ_+\,\}$.

    If $A$ is finite, there exists $x \in X$ such that
    $x_n = x$ for infinitely many $n \in \ZZ_+$.
    In this case, $\{x_n\}_{n \in \ZZ_+}$ has a constant, thus convergent, subsequence.
    
    If $A$ is infinite, by the limit point compactness,
    $A$ has a limit point $x$.
    Let $n_0 = 1$. Construct $\{n_i\}_{i \in \ZZ_+}$ as following.
    \begin{itemize}[nolistsep]
        \ii For each $i \in \ZZ_+$, choose $n_i \in \ZZ_+$ so that
            $x_{n_i} \in B(x, 1/i)$ and $n_i > n_{i-1}$.
    \end{itemize}
    This is possible since $B(x, \veps) \cap A$ is infinite
    for every $\veps \in \RR_+$ by \Cref{th:t1LimPtIffDelNeiIntersectsInf}.
    The sequence $\{x_{n_i}\}_{i \in \ZZ_+}$ converges to $x$. \checkmark

    ((iii) $\Rightarrow$ (i))
    Let $\mcal A$ be an open covering of $X$.
    By \Cref{lem:lebesgueNForSeqCpt}, there exists a Lebesgue number $\delta$ for $\mcal A$.
    Let $\veps \triangleq \delta/3$.
    By \Cref{lem:seqCptCoverByFinEpsBalls},
    there exists a finite open covering $\mcal A'$ by $\veps$-balls.
    Since every $A' \in \mcal A'$ has $\diam A \le 2\delta/3 < \veps$,
    $A' \subseteq A$ for some $A \in \mcal A$.
    The collection of such $A$ consists of a finite subcollection of $\mcal A$
    that covers $X$. \checkmark
}

\section{Local Compactness}

\dfn{Local Compactness}{
    Let $X$ be a topological space.
    $X$ is said to be \textit{locally compact at} $x$ if there exist a compact
    subspace $C$ and an open set $U$ of $X$ such that $x \in U \subseteq C$.
    $X$ is said to be \textit{locally compact} if it is locally compact at
    every point.
}

\nt{
    If $X$ is a compact space, $X$ is locally compact.
}

\thm[compactify]{}{
    Let $X$ be a topological space.
    Then $X$ is locally compact Hausdorff if and only if
    there exists a space $Y$ which satisfies the following.
    \begin{enumerate}[nolistsep, label=(\roman*)]
        \ii $X$ is a subspace of $Y$.
        \ii $|Y \setminus X| = 1$.
        \ii $Y$ is a compact Hausdorff space.
    \end{enumerate}
    Moreover, such $Y$ is unique up to homeomorphism.
    (In other words, if $Y$ and $Y'$ satisfy the three conditions,
    then they are homeomorphic with each other.)
}
\pf{Proof}{
    ($\Rightarrow$)
    Let $Y = X \cup \{\infty\}$ where $\infty \notin X$ is a new point.
    Give $Y$ the topology $\mcal T_Y \triangleq \mcal T_X \cup \mcal T'$
    where
    \[
        \mcal T' \triangleq \{\,Y \setminus C \mid
        C \subseteq X \text{ is compact subspace of } X\,\}
    \]
    $\mcal T_Y$ is actually a topology:
    \begin{enumerate}[nolistsep, label=(\roman*)]
        \ii For intersections,
            \[
                \begin{aligned}[t]
                    U_1 \cap U_2 &\in \mcal T_X \\
                    (Y \setminus C_1) \cap (Y \setminus C_2)
                                 &= Y \setminus (C_1 \cup C_2) \in \mcal T' \\
                    U_1 \cap (Y \setminus C_1)
                                 &= U_1 \cap (X \setminus C_1) \in \mcal T_X
                                 \text{. \checkmark}
                \end{aligned}
            \]
        \ii For unions,
            \[
                \begin{aligned}[t]
                    \textstyle \bigcup_{\alpha \in J} U_\alpha
                    &\in \mcal T_X \\
                    \textstyle \bigcup_{\beta \in I} (Y \setminus C_\beta)
                    &= Y \setminus \textstyle \bigcap_{\beta \in I} C_\beta
                    \in \mcal T' \\
                    \textstyle \left(\bigcup_{\alpha \in J} U_\alpha\right)
                    \cup \left(\bigcup_{\beta \in I} (Y \setminus C_\beta)\right)
                    &= U \cup (Y \setminus C) = Y \setminus (C \setminus U)
                    \in \mcal T'\text{.}
                \end{aligned}
            \]
            $C \setminus U$ is compact since it is a closed subspace
            of a compact space $C$. \checkmark
    \end{enumerate}
    $X$ is a subspace of $Y$ since $X \cap (Y \setminus C) = X \setminus C \in
    \mcal T_X$.

    Now, we claim that $Y$ endowed with $\mcal T_Y$ is \textbf{compact}.
    Let $\mcal A$ be an open covering of $Y$.
    It must be $\mcal A \cap \mcal T' \neq \OO$
    since $\infty \notin X = \bigcup \mcal T_X$.
    Take any $Y \setminus C \in \mcal A \cap \mcal T'$.
    Then, consider
    \[
        \mcal A' \triangleq \big\{\,A \cap X \:\big|\:
        A \in \mcal A \setminus \{\,Y \setminus C\,\}\,\big\}\text{,}
    \]
    which is a covering of $C$ by open sets in $X$.
    Hence, there is a finite subcollection
    \[
        \{\,A_1 \cap X, A_2 \cap X, \cdots, A_n \cap X\,\}
    \]
    of $\mcal A'$ that covers $C$.
    Then,
    \[
        \{\,Y \setminus C, A_1, A_2, \cdots, A_n\,\}
    \]
    is a finite subcover of $\mcal A$.

    To show $Y$ is \textbf{Hausdorff}, let $x$ and $y$ be
    two different points of $Y$.
    If $x, y \in X$, then it is done by
    $X$ being a Hausdorff space.
    If $x \in X$ and $y = \infty$,
    then since $X$ is locally compact,
    there exists a compact subspace $C$ of $X$
    that contains a neighborhood $U$ of $x$ in $X$.
    Then, $U$ and $Y \setminus C$
    are disjoint neighborhoods of $x$ and $y$, respectively.

    We now prove the \textbf{uniqueness}.
    Let $Y$ and $Y'$ be the spaces that satisfy the conditions.
    Let $\{p\} = Y \setminus X$ and $\{q\} = Y' \setminus X$.
    Note that $X$ is open in both $Y$ and $Y'$
    as $\{p\}$ and $\{q\}$ are closed in $Y$ and $Y'$, respectively,
    by \Cref{th:finiteSetIsClosedInHaus}.
    Define a map $f \colon Y \to Y'$ by
    \[
        x \mapsto \begin{cases}
            x & \text{if } x \in X \\
            q & \text{if } x = p\text{.}
        \end{cases}
    \]
    $f$ is naturally a bijection, and we only need to prove
    $f$ is an open map thanks to the symmetry.

    If $U$ is an open set in $Y$ that does not contain $p$,
    $U$ is open in $X$ and thus $f(U) = U$ is open in $Y'$.
    If $U$ is an open set in $Y$ that contains $p$,
    then $C = Y \setminus U$ is closed in a compact space $Y$;
    $C$ is compact by \Cref{th:cldInCptIsCpt}.
    $C$, being a compact subspace of $X$,
    is also a compact subspace of $Y'$, which is Hausdorff.
    Hence $C$ is closed in $Y'$ as well by \Cref{th:cptInHausIsCld}.
    Hence, $h(U) = Y' \setminus C$ is open in $Y'$.
    
    ($\Leftarrow$)
    $X$ is Hausdorff because it is a subspace of a Hausdorff space.
    To show $X$ is locally compact, take any $x \in X$.
    Choose disjoint neighborhoods $U$ and $V$ of $x$ and the single point in
    $Y \setminus X$, respectively, in $Y$. Then, the set $C = Y \setminus V$ is
    closed in $Y$, so it is a compact subspace of $Y$.
    Since $C \subseteq X$, it is a compact subspace of $X$ that contains the
    neighborhood $U$ of $x$.
}

\nt{
    In \Cref{th:compactify}, if $X$ is already compact, then $\mcal T'$
    contains the singleton $\{\infty\}$, which makes $\infty$ an isolated
    point. Therefore, $\cl X = X$; $X$ is closed in $Y$.

    However, if $X$ is not compact, then every neighborhood of $\infty$
    intersects $X$, which means $\infty$ is a limit point of $X$.
    Hence, $\cl X = Y$.
}

\nt{
    In either case, every open set in $X$ is still open in $Y$.
    Moreover, every open set in $Y$ that does not contain $\infty$
    is open in $X$.
}

\dfn{One-point Compactification}{
    If $Y$ is a compact Hausdorff space and $X \subsetneq Y$ is a proper
    subspace of $Y$ such that $\cl X = Y$, then $Y$ is said to be a
    \textit{compactification} of $X$. If $|Y \setminus X| = 1$, then $Y$ is
    called the \textit{one-point compactification} of $X$.
}

\cor{}{
    A topological space $X$ has a one-point compactification if and only if $X$
    is a locally compact Hausdorff space that is not itself compact.
}

\thm[HausLocalCptIff]{}{
    Let $X$ be a Hausdorff space. Then $X$ is locally compact if and only if,
    for any given $x \in X$ and neighborhood $U$ of $x$, there exists a
    neighborhood $V$ of $x$ such that $\cl V$ is compact and $\cl V \subseteq U$.
}
\pf{Proof}{
    ($\Rightarrow$)
    Take the one-point compactification $Y$ of $X$, and let $C \triangleq Y
    \setminus U$. Since $C$ is closed in $Y$, it is compact by
    \Cref{th:cldInCptIsCpt}. By \Cref{cor:cptAndPntOutside}, there are disjoint
    open sets $V$ and $W$ in $Y$ such that $x \in V$ and $C \subseteq W$.
    Then, $\cl V$, being closed in a compact set $Y$, is compact by
    \Cref{th:cldInCptIsCpt}. Furthermore, $\cl V \cap W = \OO$;
    otherwise, \Cref{th:inClosureIffNeighCapANonempty} ensures the nonempty
    intersection of $V$ and $W$. This implies $\cl V \subseteq Y \setminus W
    \subseteq Y \setminus C = U$, as desired. Hence, $V$ is an open set in $X$.

    ($\Leftarrow$)
    $C = \cl V$ is a compact subspace of $X$ that contains a neighborhood $V$
    of $x$ in $X$.
}

\cor[subspLocalCptIf]{}{
    Let $X$ be a topological space and $A$ be a subspace of $X$.
    \begin{itemize}[nolistsep]
        \ii If $X$ is locally compact and $A$ is closed in $X$, then $A$ is
            locally compact.
        \ii If $X$ is locally compact \textit{Hausdorff} and $A$ is open in
            $X$, then $A$ is locally compact.
    \end{itemize}
}
\pf{Proof}{
    \hfill
    \begin{itemize}[nolistsep]
        \ii Suppose $A$ is closed in $X$. Given $x \in A$, let $U$ be an open
            set and $C$ be a compact subspace such that $x \in U \subseteq C$.
            Then, $C \cap A$ is closed in $C$ and thus compact by
            \Cref{th:cldInCptIsCpt}, and it contains a neighborhood $U \cap A$
            of $x$ in $A$. \checkmark
        \ii Suppose $A$ is open in $X$. Given $x \in A$, by
            \Cref{th:HausLocalCptIff}, there exists a neighborhood $V$ of $x$
            in $X$ such that $\cl V$ is compact and $\cl V \subseteq A$.
            Then, $\cl V$ is a compact subspace of $A$ containing the
            neighborhood $V$ of $x$ in $A$. \checkmark
    \end{itemize}
}

\cor[]{}{
    A topological space $X$ is homeomorphic to an open subspace of a compact
    Hausdorff space if and only if $X$ is locally compact Hausdorff.
}
\pf{Proof}{
    ($\Rightarrow$) \Cref{th:compactify}. \quad
    ($\Leftarrow$) \Cref{cor:subspLocalCptIf}.
}

\end{document}
