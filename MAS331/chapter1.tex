\documentclass[MAS331_Note.tex]{subfiles}

\begin{document}
\chapter{Set Theory and Logic}
\section{Basic Notation}
\nt{
	\begin{itemize}[nolistsep]
		\item Sets: $A, B, C, \cdots, \RR, \QQ, \ZZ$
		\item Elements: $a, b, c, \cdots, 3, 3/4, \pi$
		\item $a \in A$, $3 \in \ZZ$, $3/4 \notin \ZZ$
		\item $A \subseteq B$, $A \subsetneq B$         , $A \not\subseteq B$
		\item $\varnothing$: empty set
		\item $A \times B \coloneqq \{(a, b) \mid a \in A \text{ and } b \in B\}$ (Cartesian product)
	\end{itemize}
}
\dfn{Function, Restriction, and Composition}{
A \textit{function} $f$ from a set $A$ to a set $B$ is an assignment of an element of $B$
to each element of $A$.
\begin{itemize}[noitemsep]
	\item $A$: Domain
	\item $B$: Range or Codomain
	\item $\Img f \coloneqq \{ f(a) \mid a \in A \}$: Image; $\Img f \subseteq B$
\end{itemize}
If $A_0 \subseteq A$ and $f \colon A \to B$ is a function,
then the \textit{restriction} of $f$ to $A_0$ is denoted by $f \big|_{A_0}$ and is defined as \[
	f \big|_{A_0} (a_0) \coloneqq f(a_0)
\]
for each $a_0 \in A_0$.
If $f \colon A \to B$ and $g \colon B \to C$, then the \textit{composite} $g \circ f$ is defined as \[
	(g \circ f)(a) \coloneqq g \big(f(a)\big)
\]
for each $a \in A$.
}
\dfn{Injectivity, Surjectivity and Bijectivity}{
	A function $f \colon A \to B$ is
	\begin{enumerate}[noitemsep,label=\roman*)]
		\item \textit{injective} (or \textit{one-to-one}, 1-1) if $\forall a, a' \in A,\: f(a) = f(a') \implies a = a'$,
		\item \textit{surjective} (or \textit{onto}) if $\forall b \in B,\: \exists a \in A,\: b = f(a)$, and
		\item \textit{bijective} if $f$ is both injective and surjective.
	\end{enumerate}
}
\dfn{Inverse Function}{
	If $f \colon A \to B$ is bijective, then the inverse of $f$ is denoted by \[
		f^{-1} \colon B \to A
	\]
	and is defined as \[
		f^{-1}(b) = a
	\] for each $b \in B$ where $f(a) = b$.
}
\exmp{}{
	\begin{enumerate}[nolistsep, label=\alph*)]
		\item $f$ is bijective $\iff$ $f^{-1}$ is bijective.
		\item The inverse is unique.
	\end{enumerate}
}
\solve{
	Suppose $f$ is bijective. Then, \[
		f^{-1}(b_1) = f^{-1}(b_2) \implies b_1 = (f \circ f^{-1})(b_1) = (f \circ f^{-1})(b_2) = b_2\text{.}
	\]
	Therefore, $f^{-1}$ is injective.

	Take any $a \in A$. Then, $b \coloneqq f(a) \in B$ satisfies $f^{-1}(b) = a$.
	Therefore, $f^{-1}$ is surjective.

	Now, suppose $f^{-1}$ is bijective. Then, \[
		f(a_1) = f(a_2) \implies a_1 = (f^{-1} \circ f)(a_1) = (f^{-1} \circ f)(a_2) = a_2\text{.}
	\]
	Therefore, $f$ is injective.

	Take any $b \in B$. Then, $a \coloneqq f^{-1}(a) \in B$ satisfies $f(a) = b$.
	Therefore, $f$ is surjective; a) is now proven.

	Let $g$ and $h$ are inverses of $f$. Take any $b \in B$.
	Since $f$ is bijective, $\exists! a \in A,\; f(a) = b$.
	Therefore, $g(b) = a = h(b)$, which implies $g = h$; b) is now proven.
}

\dfn{Image and Preimage of a Set}{
	Let $f \colon A \to B$ and $A_0 \subseteq A$, $B_0 \subseteq B$.
	\begin{itemize}
		\item $f(A_0) \coloneqq \{b \mid b = f(a_0) \text{ and } a_0 \in A \}$
		\item $f^{-1}(B_0) \coloneqq \{a \mid f(a) \in B_0 \}$
	\end{itemize}
}
\exmp[compFandFinvAndSurjInj]{}{
	\begin{enumerate}[nolistsep, label=\alph*)]
		\item $A_0 \subseteq f^{-1} \big(f(A_0)\big)$
		\item $f$ is injective if and only if $\forall A_0 \subseteq A,\: A_0 = f^{-1}\big(f(A_0)\big)$.
		\item $f\big(f^{-1}(B_0)\big) \subseteq B_0$
		\item $f$ is surjective if and only if $\forall B_0 \subseteq B,\: B_0 = f \big(f^{-1}(B_0)\big)$.
	\end{enumerate}
}
\solve{
	\paragraph*{a)}
	For every $a_0 \in A_0$, $f(a_0) \in f(A_0)$, which implies $a_0 \in f^{-1}\big(f(A_0)\big)$.
	Therefore, $A_0 \subseteq f^{-1} \big(f(A_0)\big)$ holds.
	\paragraph*{b)}
	Suppose $f$ is injective. Take any $A_0 \subseteq A$ and $a_0 \in f^{-1}\big(f(A_0)\big)$.
	Then, $f(a_0) \in f(A_0)$. We may take $a_1 \in A_0$ such that $f(a_0) = f(a_1) \in f(A_0)$.
	Since $f$ is injective, $a_0 = a_1 \in A_0$.

	Suppose `$\forall A_0 \subseteq A,\: A_0 = f^{-1}\big(f(A_0)\big)$' holds. Suppose $f(a_1) = f(a_2) = b_0$.
	Let $A_0 \coloneqq \{ a_1 \}$. Then, $A_0 = f^{-1}\big(f(A_0)\big) = f^{-1}\big(\{b_0\}\big) \ni a_2$.
	This means $a_2 \in \{ a_1\}$, which implies $a_1 = a_2$.
	\paragraph*{c)}
	Take any $b_0 \in f \big(f^{-1}(B_0)\big)$. Then, there is some $a_0 \in f^{-1}(B_0)$ such that $f(a_0) = b_0$.
	Such $a_0$ satisfies $f(a_0) \in B_0$, which implies $b_0 = f(a_0) \in B_0$.
	Therefore, $f\big(f^{-1}(B_0)\big) \subseteq B_0$ holds.
	\paragraph*{d)}
	Suppose $f$ is surjective. Take any $B_0 subs B$ and $b_0 \in B_0$. Then, there is some $a_0 \in A$ such that $f(a_0) = b_0$,
	which implies $a_0 \in f\inv(B_0)$. Therefore, $b_0 \in f \big(f\inv(B_0)\big)$; $B_0 \subseteq \big(f\inv(B_0)\big)$.

	Suppose `$\forall B_0 \subseteq B,\: B_0 = f \big(f^{-1}(B_0)\big)$' holds. Take any $b_0 \in B$ and let $B_0 \coloneqq \{b_0\}$.
	Since $b_0 \in f \big(f\inv(B_0)\big)$, There is some $a_0 \in f\inv(B_0)$ such that $f(a_0) = b_0$.
	Therefore, $f$ is surjective.
}

\section{Relations}
\dfn{Relation}{
	A relation $\sim$ on a set $A$ is a subset of $A \times A$. \[
		x \sim y \coloneqq (x, y) \in \:\sim
	\]
}
\dfn{Equivalence Relation and Equivalence Class}{
	A relation $\sim$ on a set $A$ is an \textit{equivalence relation} if
	\begin{enumerate}[noitemsep, label=(\arabic*)]
		\item $x \sim x$ for each $x \in A$ (reflexive)
		\item $x \sim y \implies y \sim x$ (symmetric)
		\item $x \sim y \land y \sim z \implies x \sim z$. (transitive)
	\end{enumerate}
	Moreover, the \textit{equivalence class} of $x$ is defined as \[
		\big\{ y \in A \:\big|\: y \sim x \big\}\text{.}
	\]
}
\exmp{Partition}{
	If there are equivalence classes $E$ and $E'$, then they are either $E = E'$ or $E \cap E' = \varnothing$.
	This implies, if we let $\mcal{E} \coloneqq \{ E \mid E \text{ is an equivalence class of } x \text{ where } x \in A \}$,
	$A = \bigcup_{E \in \mcal{E}} E$.
}
\solve{
	Since if $E \cap E' = \varnothing$ it is done, suppose $E \cap E' \neq \varnothing$. There are $a$ and $a'$ such that
	$E$ and $E'$ are equivalence classes of $a$ and $a'$ respectively.
	We may take $a_0 \in E \cap E'$. By definition and transitivity, $a \sim a_0 \sim a'$.
	Therefore, for all $x \in E$, $x \in E'$ since $x \sim a \sim a'$, which implies $E \subseteq E'$.
	In the same way, $E' \subseteq E$.
}

\dfn{Order Relation}{
	A relation $<$ on a set $A$ is an \textit{order relation} if
	\begin{enumerate}[noitemsep, label=(\arabic*)]
		\item $x < y$ or $y < x$ for each $x \neq y \in A$
		\item $x \not< x$ for each $x \in A$
		\item $x < y \land y < z \implies x < z$.
	\end{enumerate}
	Also, we define \[
		(a, b) \coloneqq \big\{ x \in X \:\big|\: a < x < b \big\}\text{.}
	\]
}
\dfn{Order Type}{
	Let $A$ and $B$ be sets with order relations $<_A$ and $<_B$, respectively.
	Then, $A$ and $B$ have the same \textit{order type} if there is a bijection
	$f \colon A \to B$ such that $a_1 <_A a_2 \iff f(a_1) <_B f(a_2)$.
}
\dfn{Dictionary Order Relation}{
	Let $A$, $B$ be sets with order relations $<_A$, $<_B$ respectively.
	Then, there is an order relation $<_{A \times B}$ on $A \times B$
	defined as $(a_1, b_1) <_{A \times B} (a_2, b_2)$ if \[
		a_1 <_A a_2 \text{ or } a_1 = a_2 \text{ and } b_2 <_B b_2\text{.}
	\]
	This is often called \textit{dictionary order relation} on $A \times B$.
}
\dfn{Boundedness} {
	Let $A_0 \subseteq A$ with an order relation $<_A$.
	\begin{itemize}[noitemsep]
		\item The \textit{largest element} of $A_0$ is $b \in A_0$ if $x \in A_0 \implies x \le b$.
		\item The \textit{smallest element} of $A_0$ is $b \in A_0$ if $x \in A_0 \implies x \ge b$.
	\end{itemize}
	\begin{itemize}[noitemsep]
		\item $A_0$ is \textit{bounded above} by $b \in A$ if $x \in A_0 \implies x \le b$.
		      \begin{itemize}[nolistsep]
			      \item The smallest such $b$ is called the \textit{least uppder bound} or the \textit{supremum} of $A_0$.
		      \end{itemize}
		\item $A_0$ is \textit{bounded below} by $b \in A$ if $x \in A_0 \implies x \ge b$.
		      \begin{itemize}[nolistsep]
			      \item The largest such $b$ is called the \textit{greatest lower bound} or the \textit{infimum} of $A_0$.
		      \end{itemize}
	\end{itemize}
	\begin{itemize}[noitemsep]
		\item $A$ has \textit{least upper bound property} if every bounded above nonempty set $A_0 \subseteq A$
		      has a least upper bound.
		\item $A$ has \textit{greatest lower bound property} if every bounded below nonempty set $A_0 \subseteq A$
		      has a greatest lower bound.
	\end{itemize}
}
\thm{}{
	A set $A$ with an order relation $<_A$ has l.u.b. property if and only if $A$ has g.l.b. property.
}
\pf{Proof}{
	Suppose $A$ has l.u.b. property. Let $A_0$ be any bounded below nonempty subset of $A$.
	Let $L \coloneqq \{ a \in A \mid a \text{ is a lower bound of } A_0 \}$.
	Take a $a_0 \in A_0$. Then, since $\ell \le_A a_0$ for all $\ell \in L$, $L$ is bounded above by $a_0$.
	By l.u.b. property of $A$, there is $\ell_0 \coloneqq \sup L \in A$.

	Take any $a_0$ in $A_0$. Since $a_0$ is an upper bound of $L$ and $\ell_0$ is the least upper bound, $\ell_0 \le_A a_0$.
	Therefore, $\ell_0$ is a lower bound of $A_0$.

	Suppose $\ell_0 <_A \ell_1$ and $\ell_1$ is a lower bound of $A_0$.
	This implies $\ell_1 \in L$, which contradicts to $\ell_1 \le_A \sup L = \ell_0$.
	Therefore, $\ell_0$ is the greatest lower bound, and $A$ has g.l.b. property.

	The inverse can be proven by the similar reasoning.
}
\thm[completeR]{Completeness of $\RR$}{
	The set of real numbers $\RR$ has least upper bound property.
}

\section{The Integers and the Real Numbers}

\thm[wop]{Well-Ordering Property}{
	Every nonempty subset of $\ZZ_+$ has a smallest element.
}
\pf{Proof}{
	We first prove that, for each $n \in \ZZ_+$, every nonempty subset of $[n] \coloneqq \{1, 2, \cdots, n\}$ has a smallest element,
	using induction. For the base case, it is known the the only nonempty subset of $[1]$, $\{1\}$, has $1$ as its smallest element.

	Suppose the statement holds for $n = k$. Now take any nonempty subset $S$ of $[k + 1]$.
	If $S = \{k+1\}$, $k+1$, the only element of $S$, is a smallest element of $S$.
	Otherwise, $S \setminus \{k+1\}$ is nonempty and is a subset of $[k]$; we may let $\mu \coloneqq \min S$ by the induction hypothesis.
	Then, $\mu$ is also a smallest element of $S$, regardless of whether it is $k + 1 \in S$ or $k + 1 \notin S$.

	Now, take any $\varnothing \neq T \subseteq \ZZ_+$ and $m \in T$.
	Then, by our previous result, since $T \cap [m]$ is a nonempty subset of $[m]$, it has a smallest element, which is also a smallest element of $T$.
}

\section{Cartesian Products}

\dfn{Indexing Function and Indexed Family of Sets}{
Let $\mcal{A}$ be a nonempty collection of sets.
An \textit{indexing function} for $\mcal{A}$ is a surjective function
$f \colon J \to \mcal{A}$ where $A_{\alpha} \coloneqq f(\alpha)$.
An \textit{indexed family} of sets is defined as $\big\{A_{\alpha}\big\}_{\alpha \in J}$.

Now, we define \[
	\begin{aligned}[t]
		\bigcup_{\alpha \in J} A_{\alpha} & \coloneqq \big\{x \:\big|\: \exists \alpha \in J,\: x \in A_{\alpha} \big\}                                                                           \\
		\bigcap_{\alpha \in J} A_{\alpha} & \coloneqq \big\{x \:\big|\: \forall \alpha \in J,\: x \in A_{\alpha} \big\}                                                                           \\
		\prod_{\alpha \in J} A_{\alpha}   & \coloneqq \big\{ f \colon J \to \textstyle\bigcup_{\alpha \in J} A_{\alpha} \:\big|\: \forall \alpha \in J,\: f(\alpha) \in A_{\alpha} \big\}\text{.}
	\end{aligned}
\]
}

\section{Finite Sets}
\dfn{Finite Set and Cardinality}{
	A set $A$ is \textit{finite} if there is a bijective $f \colon A \to [n]$ for some $n \in \ZZ_+$ or $A = \varnothing$.
	\begin{itemize}[nolistsep]
		\item In the former case, we say \textit{cardinality} $n$ or $|A| = n$.
		\item In the latter case, we say \textit{cardinality} $0$ or $|A| = 0$.
	\end{itemize}
}
\nt{
	Let $A$ and $B$ be finite sets. Then, $|A| = |B| = n$ if and only if $\exists$ bijective $f \colon A \to B$.
}

\mlemma[cardOneMinus]{}{
	Let $a_0 \in A$. Then, \[
		|A| = n \iff |\,A \setminus \{a_0\} | = n - 1\text{.}
	\]
}
\pf{Proof}{
For $n = 1$, it is trivial. So suppose $n \ge 2$.
\par ($\Rightarrow$) There is a bijection $f \colon A \to [n]$.
If $f(a_0) = n$, then $f \big|_{A \setminus \{a_0\}}$ is a bijection from $A \setminus \{a_0\}$ to $[n-1]$, and it's done.
Otherwise, let $a_1 \coloneqq f^{-1}(n)$. Define $g \colon A \to A$ by \[
	g(a) \coloneqq \begin{cases}
		a_0 & \text{if } a = a_1 \\
		a_1 & \text{if } a = a_0 \\
		a   & \text{otherwise.}
	\end{cases}
\] $g$ is bijective. Then, $f \circ g$ is a bijection from $A$ to $[n]$ such that $(f \circ g)(a_0) = n$.
\par ($\Leftarrow$) Trivial.
}

\thm[cardPropSubset]{}{
	Let $A$ be a set with $|A| = n$ and $B \subsetneq A$.
	Then, there is no bijection between $B$ and $[n]$, but (provided $B \neq \varnothing$)
	there is a bijection between $B$ and $[m]$ for some $m < n$.
}
\pf{Proof by Induction}{
	(Base case) It is trivial for $n = 1$.
	\par (Induction) Suppose it is true for $n \ge 1$.
	WTS for the case $|A| = n+1$. Suppose $B \neq \varnothing$ because we have nothing to talk about then.
	Let $a_0 \in B$.
	By \Cref{lem:cardOneMinus}, there is a bijection $g \colon A \setminus \{a_0\} \to [n]$.
	Since $B \setminus \{a_0\} \subsetneq A \setminus \{a_0\}$, by induction hypothesis,
	we have two things.
	\begin{itemize}[nolistsep]
		\item There is no bijection between $B \setminus \{a_0\}$ and $[n]$.
		\item As long as $B \neq \{a_0\}$, there is a bijection from $B \setminus \{a_0\}$ to $[m]$ for some $m < n$.
	\end{itemize}
	We conclude that there is no bijection from $B$ and $[n+1]$ since, if there were, there would be a
	trivial bijection from $B \setminus \{a_0\}$ to $[n]$.
	Moreover, we can construct a bijection between $B$ and $[m+1]$, and $m+1 < n+1$.
}

\cor[cardUnique]{Uniqueness of Cardinality}{
	The cardinality of a finite set is uniquely determined.
}
\pf{Proof}{
	Let $m < n$ and suppose $m$ and $n$ are cardinalities of a finite set $A$.
	Then there are bijections $f \colon A \to [m]$ and $g \colon A \to [n]$.
	Then, $f \circ g^{-1}$ is a bijection from $[m]$ to $[n]$ but it is impossible
	since $[m] \subsetneq [n]$ and because of \Cref{th:cardPropSubset}.
}

\cor[ZplusIsInf]{}{
	$\ZZ_+$ is not finite.
}
\pf{Proof by Contradiction}{
	Suppose $\ZZ_+$ is finite and $|\ZZ_+| = n$.
	$f \colon \ZZ_+ \to \ZZ_+ \setminus \{1\}$ with $x \mapsto x + 1$ is bijective.
	Then, by \Cref{lem:cardOneMinus}, $n - 1 = |\,\ZZ_+ \setminus \{1\}| = |\ZZ_+| = n$, \#.
}

\thm[finSetTFAE]{}{
	Let $A$ be a set. {\sf TFAE}
	\begin{enumerate}[noitemsep, label=(\roman*)]
		\item $|A| = n$
		\item $\exists$ surjective $[m] \twoheadrightarrow A$ for some $m \in \ZZ_+$.
		\item $\exists$ injective $A \hookrightarrow [m]$ for some $m \in \ZZ_+$.
	\end{enumerate}
}
\pf{Proof}{
	((i) $\to$ (ii)) There is a bijective function from $A$ to $[n]$, and it is also surjective.
	\par ((ii) $\to$ (iii)) Let $f$ be a surjective function from $[m]$ to $A$.
	Since $f$ is surjective, $f^{-1}(\{a\}) \neq \varnothing$ for every $a \in A$.
	Let $M \coloneqq \max \{\, \min f^{-1}(\{a\}) \mid a \in A \,\}$.
	$M$ is well defined thanks to \Cref{th:wop} and the fact that $\varnothing \neq f^{-1}(\{a\}) \subseteq [m]$.
	Then the function $g \colon A \to [M]$ defined by $a \mapsto \min f^{-1}(\{a\})$ is injective.
	\par ((iii) $\to$ (i)) Let $f$ be an injective function from $A$ to $[m]$.
	Then, $g \colon A \to \Img f$ defined by $a \mapsto f(a)$ is bijective.
	$A$ is finite because $\Img f$ is finite by \Cref{th:cardPropSubset}.
}

\exer[finUandCOfFinSetIsFin]{}{
	\begin{enumerate}[label=(\roman*), nolistsep]
		\item Finite unions of finite sets are finite.
		\item Finite Cartesian products of finite sets are finite.
	\end{enumerate}
}
\solve{
(i) Suppose there are $n$ finite sets $A_1$, $A_2$, $\cdots$, $A_n$ to union.
WLOG, $A_i \neq \varnothing$ for each $i \in [n]$.
Let $M \coloneqq \max_{i \in [n]} |\,A_i\,|$ and $g_i \colon [|\,A_i\,|] \to A_i$ be a bijective function for each $i \in [n]$.
Extend each $g_i$ to $g_i' \colon [M] \to A_i$ by \[
	g_i'(k) = \begin{cases}
		g_i(k) & \text{if } k \le |A_i| \\
		g_i(1) & \text{otherwise.}
	\end{cases}
\] for $k \in [M]$. Now, we define $f \colon [nM] \to \bigcup_{i \in [n]} A_i$ by \[
	f\big(n(i-1)+k\big) \coloneqq g_i'(k)
\] for each $i \in [n]$ and $k \in [M]$. Then, $f$ is surjective.
Therefore, $\bigcup_{i \in [n]} A_i$ is finite by \Cref{th:finSetTFAE}.

(ii) Suppose there are $n$ finite sets $A_1$, $A_2$, $\cdots$, $A_n$ to construct a Cartesian product with.
WLOG, $A_i \neq \varnothing$ for each $i \in [n]$.
Let $M \coloneqq \max_{i \in [n]} |\,A_i\,|$ and $h_i \colon A_i \to [|\,A_i\,|]$ be a bijective function for each $i \in [n]$.
Let $p_i$ be the $i^{\text{th}}$ prime. (i.e., $p_1 = 2$, $p_2 = 3$, $p_3 = 5$.)
Define a function $f \colon \prod_{i \in [n]} A_i \to \left[ \left( \prod_{i=1}^n p_i \right)^M \right]$ by \[
	f(a_1, a_2, \cdots, a_n) \coloneqq \prod_{i=1}^n p_i^{\> h_i(a_i)}\text{.}
\]
$f$ is injective since prime factorization of a natural number is unique.
Therefore, $\prod_{i \in [n]} A_i$ is finite by \Cref{th:finSetTFAE}.
}

\section{Countable and Uncountable Sets}
\dfn{Infinite and Countably Infinite}{
	A set $A$ is said to be \textit{infinite} if it is not finite.
	It is said to be \textit{countably infinite} if there is a bijective correspondence \[
		f \colon A \to \ZZ_+\text{.}
	\]
}
\exmp{}{
	$\ZZ_+$, $\ZZ$, and $\ZZ_+ \times \ZZ_+$ are countably infinite.
}
\dfn{Countability}{
	A set is said to be \textit{countable} if it is either finite or countably infinite.
	A set that is not countable is said to be \textit{uncountable}.
}

\mlemma[ZsubsCount]{}{
	Any subset of $\ZZ_+$ is countable.
}
\pf{Proof}{
	Let $C \subseteq \ZZ_+$. If $C$ is finite, then it's done; we now assume $C$ is infinite.
	Now we want to show that $C$ is countably infinite.

	Define $h \colon \ZZ_+ \to C$ by the following.
	\begin{enumerate}[nolistsep, label=(\alph*)]
		\item $h(1) \coloneqq \min C$
		\item $h(n+1) \coloneqq \min \big(C \setminus h([n])\big)$ for each $n \in \ZZ_+$
	\end{enumerate}
	$h$ is well defined because $C \setminus h([n])$ is always nonempty.
	Morever, $h$ is injective since it is $h(m) < h(n)$ whenever $m < n$.

	Now, we are going to show $h$ is surjective. To do this, first take any $c \in C$.
	Since $C$ is infinite and $h$ is injective, $\Img h \not\subseteq [c]$, which means $\exists n \in \ZZ_+,\: h(n) > c$.
	From this, we get $m \coloneqq \min \{\, n \in \ZZ_+ \mid h(n) \ge c \,\}$ is well-defined.
	From the definition of $m$, we also get, for any $1 \le i < m$, we have $h(i) < c \le h(m)$.
	Therefore, $c \notin h([m-1])$.
	Together with $h(m) = \min \big(C \setminus h([m-1])\big)$, we get $h(m) \le c \le h(m)$, which implies $c = h(m)$.
}

\thm[countSetFTAE]{}{
	Let $A \neq \varnothing$. {\sf TFAE}
	\begin{enumerate}[noitemsep, label=(\roman*)]
		\item $A$ is countable.
		\item $\exists$ surjective $\ZZ_+ \twoheadrightarrow A$.
		\item $\exists$ injective $A \hookrightarrow \ZZ_+$.
	\end{enumerate}
}
\pf{Proof}{
	((i) $\to$ (ii)) Trivial.
	\par ((ii) $\to$ (iii)) Let $f \colon \ZZ_+ \twoheadrightarrow A$.
	Define $g \colon A \to \ZZ_+$ by $a \mapsto \min f^{-1}(\{a\})$.
	$g$ is well-defined because $f^{-1}(\{a\}) \neq \varnothing$ for every $a \in A$ and \Cref{th:wop} holds.
	$g$ is also injective since $f^{-1}(\{a_1\}) \cap f^{-1}(\{a_2\}) = \varnothing$ if $a_1 \neq a_2 \in A$.
	\par ((iii) $\to$ (i)) Let $f$ be an injection from $A$ to $\ZZ_+$.
	If we define $g \colon A \to \Img f$ by $a \mapsto f(a)$, $g$ is a bijection.
	Since $\Img f \subseteq \ZZ_+$, $A$ is countable by \Cref{lem:ZsubsCount}.
}

\cor[subsOfCountIsCount]{}{
	If $A \subseteq B$ and $B$ is countable, then $A$ is countable.
}
\pf{Proof}{
	$A \xhookrightarrow{\text{trivial injection}} B \xhookrightarrow{\text{injection}} \ZZ_+$ and \Cref{th:countSetFTAE}.
}

\cor[ZsqIsCount]{}{
	$\ZZ_+ \times \ZZ_+$ is countably infinite.
}
\pf{Proof}{
	$f \colon \ZZ_+ \times \ZZ_+ \to \ZZ_+$ with $(x, y) \mapsto 2^x 3^y$ is an injection.

	Or, $g \colon \ZZ_+ \times \ZZ_+ \to \ZZ_+$ with $(x, y) \mapsto \dfrac{(x+y-1)(x+y-2)}{2} + y$ is a bijection.
}
\cor[QIsCount]{}{
	$\QQ$ is countably infinite.
}
\pf{Proof}{
	$f \colon \ZZ \times \ZZ_+ \to \QQ$ with $(x, y) \mapsto x/y$ is surjective.
}

\exer[UnionOfCountSets]{}{
	The union of a countable number of countable sets is countable.
}

\solve{
    Let $\{A_i\}_{i \in J}$ be an indexed family of sets where $J$ and $A_i$'s are countable.
    WLOG, $A_i \neq \varnothing$ for each $i \in J$.
    For each $i \in J$, since $A_i$ is countable, by \Cref{th:countSetFTAE},
    there is a surjection $g_i \colon \ZZ_+ \twoheadrightarrow A_i$.
    Similarly, since $J$ is countable, there is a surjection $h \colon \ZZ_+ \twoheadrightarrow J$.

    Now, construct a function $f \colon \ZZ_+ \times \ZZ_+ \to \bigcup_{i \in J} A_i$ by \[
        f(i, j) \coloneqq g_{h(i)}(j)\text{.}
    \] $f$ is naturally surjective by the contruction. Therefore, $\bigcup_{i \in J} A_i$ is countable.
}

\exer[ProductOfCountSets]{}{
	The Cartesian product of a finite number of countable sets is countable.
}
\solve{
    Suppose there are $n \in \ZZ_+$ sets $A_1$, $A_2$, $\cdots$, $A_n$ to make Cartesian product with
    and each $A_i$ is countable. WLOG, $A_i \neq \varnothing$ for each $i \in [n]$.
    For each $i \in [n]$, there is a injection $g_i \colon A_i \to \ZZ_+$ by \Cref{th:countSetFTAE}.

    Now, construct a fuction $f \colon \prod_{i=1}^n A_i \to \ZZ_+$ by \[
        f(a_1, a_2, \cdots, a_n) \coloneqq \prod_{i=1}^n p_i^{g_i(a_i)}\text{,}
    \] where $p_i$ is the $i^{\text{th}}$ prime.
    Since prime factorization of a natural number is unique, $f$ is injective; therefore $\prod_{i=1}^{n} A_i$ is countable.
}

\thm[2toZUncount]{}{
	Let $X_i \coloneqq \{0, 1\}$ for each $i \in \ZZ_+$. Then, $\prod_{i \in \ZZ_+} X_i$ is uncountable.
}
\pf{Proof}{
	Let $f \colon \ZZ_+ \to \prod_{i \in \ZZ_+} X_i$ is any function.
	Denote $f(n) = (x_{n,1}, x_{n,2}, \cdots) \in \prod_{i \in \ZZ_+} X_i$
	and construct $y = (y_1, y_2, \cdots) \in \prod_{i \in \ZZ_+} X_i$ by \[
		y_i \coloneqq 1 - x_{i,i}
	\] for each $i \in \ZZ_+$.
	Then, $y \notin \Img f$; therefore, one cannot construct a surjection from $\ZZ_+$ to $\prod_{i \in \ZZ_+} X_i$.
}

\cor[PSetOfZUncount]{}{
	$\mcal{P}(\ZZ_+)$ is uncountable.
}
\pf{Proof}{
	$f \colon \mcal{P}(\ZZ_+) \to \prod_{i \in \ZZ_+} X_i$ defined by \[
		S \mapsto (y_1, y_2, \cdots) \text{ where } y_i \coloneqq \begin{cases}
			0 & \text{if } i \in S    \\
			1 & \text{if } i \notin S
		\end{cases}
	\] is a bijection, and $\prod_{i \in \ZZ_+} X_i$ is uncountable by \Cref{th:2toZUncount}.
}

\thm[noInjSurjWithPSet]{}{
	Let $A$ be a set. Then, there is no injection $\mcal{P}(A) \hookrightarrow A$,
	and there is no surjection $A \twoheadrightarrow \mcal{P}(A)$.
}
\pf{Proof}{
	Since a surjective map can be naturally deducted from $f \colon B \hookrightarrow C$
	(by constructing $g \colon C \to B$ by $g(c) \in f^{-1}(\{c\})$ for $c \in \Img f$ and
	map $c$ to an arbitrary element in $B$ for $c \notin \Img f$),
	it suffices to show $A \twoheadrightarrow \mcal{P}(A)$ does not exist.

	Let $f \colon A \to \mcal{P}(A)$ be any function, and let $B \coloneqq \{\, a \in A \mid a \notin f(a) \,\} \in \mcal{P}(A)$.
	Suppose $B = f(a_0)$ for some $a_0 \in A$. Then, by the definition of $B$, \[
		a_0 \in B \iff a_0 \notin f(a_0) = B\text{,}
	\] which is a contradiction.
	Therefore, any such $f$ cannot be surjective.
}

\section{Infinite Sets and the Axiom of Choice}

\thm[InfSetTFAE]{}{
	Let $A$ be a set. {\sf TFAE}
	\begin{enumerate}[noitemsep, label=(\roman*)]
		\item $A$ is infinite.
		\item $\exists$ injection $f \colon \ZZ_+ \hookrightarrow A$.
		\item $\exists$ bijection $g \colon A \to B$ where $B \subsetneq A$.
	\end{enumerate}
}
\pf{Proof}{
	((i) $\to$ (ii)) Construct $f \colon \ZZ_+ \to A$ recursively as following.
    Let $c \colon \mcal{P}(A) \setminus \{\varnothing\} \to A$ be a function
    such that $c(A') \in A'$ for every $\varnothing \neq A' \subseteq A$.
    Its existence is guaranteed by \Cref{lem:choiceFtnExs}.
	\begin{enumerate}[nolistsep, label=(\arabic*)]
		\item $f(1) \coloneqq c(A)$
        \item $f(n+1) \coloneqq c(A \setminus f([n]))$ for each $n \in \ZZ_+$.
	\end{enumerate}
	\par Suppose $A \setminus f([n]) = \varnothing$ for some $n \in \ZZ_+$.
	Then, $A \subseteq f([n])$, and $f([n])$ is finite by \Cref{th:finSetTFAE};
	therefore $A$ is finite by \Cref{th:cardPropSubset}.
	Thus, $f$ is well-defined and it is injective by definition.

	((ii) $\to$ (iii))
	Let $f \colon \ZZ_+ \hookrightarrow A$ be an injectiion.
	Define $g \colon A \to A \setminus \{\,f(1)\,\}$ by \[
		g(a) \coloneqq \begin{cases}
			f(n+1) & \text{if } a = f(n) \text{ for some } n \in \NN_+ \\
			a      & \text{if } a \notin \Img f\text{.}
		\end{cases}
	\] $g$ is well-defined because $f$ is injective, and it is bijective by definition.

	((iii) $\to$ (i)) This is just a contrapositive of \Cref{th:cardPropSubset}.
}

\thm[AoC]{Axiom of Choice}{
    Given a collection $\mscr{A}$ of disjoint nonempty sets,
    there exists a set $C$ such that $C \subseteq \bigcup \mscr{A}$ and $\forall A \in \mscr{A},\: |\,C \cap A\,| = 1$.
}

\mlemma[choiceFtnExs]{Existence of a Choice Function}{
    Given a collection $\mscr{B}$ of nonempty sets, there exists a function \[
        c \colon \mscr{B} \to \textstyle\bigcup\mscr{B}
    \] such that $c(B) \in B$ for each $B \in \mscr{B}$.
}
\pf{Proof}{
    Let $\mscr{A} \coloneqq \big\{\, \{\, (B, x) \mid x \in B \,\} \:\big|\: B \in \mscr{B} \,\big\}$.
    Then, by \Cref{th:AoC}, there exists $c \subseteq \mscr{A}$ such that $c \subseteq \bigcup\mscr{A}$ and
    each $B \in \mscr{B}$ appears only once in the first coordinate in $c$.
    Therefore, $c$ is a function such that $c(B) \in C$ for each $B \in \mscr{B}$.
}

\section{Well-Ordered Sets}
\dfn{Well-Ordered}{
	A set $A$ with an order relation is an \textit{well-ordered} set if
	every nonempty subset of $A$ has a smallest element.
}
\exmp{}{
	\begin{itemize}[nolistsep]
		\item $\ZZ_+$ is well-ordered.
		\item $\{1, 2\} \times \ZZ_+$ is well ordered with respect to the dictionary ordering.
	\end{itemize}
}

\thm{}{
	Every nonempty finite set has the order type of $[n]$, and thus it is well-ordered.
}
\pf{Proof}{
	We shall first claim that, if $A$ is a nonempty finite set, then it has a largest element.
	It can be prove by induction on $|A|$. If $|A| = 1$, then it is trivial.
	Suppose the claim holds for $|A| = n$, and suppose $|A| = n + 1$ and $a_0 \in A$.
	Then, $A \setminus \{\,a_0\,\}$ has a largest element $a_1$.
	This implies $A$ has a largest element $\max \{\,a_0, a_1\,\}$.

	Now, we prove there is an order-preserving bijection $f \colon A \to [n]$.
	This will also be proven with induction.
	It is true when $|A| = 1$, so suppose it is true for $|A| = n \in \ZZ_+$ and let $|A| = n + 1$.
	By above, we may let $a_0 \coloneqq \max A$.
	By induction hypothesis, there is an order-preserving bijection $f' \colon A \setminus \{\,a_0\,\} \to [n]$.
	Define $f \colon A \to [n+1]$ by \[
		f(a) \coloneqq \begin{cases}
			f'(a) & \text{if } a \neq a_0      \\
			n+1   & \text{if } a = a_0\text{.}
		\end{cases}
	\] Then, $f$ is an order-preserving bijection from $A$ to $[n+1]$.
}

\thm[CofFinManyWOisWO]{}{
	The Cartesian product of finitely many well-ordered sets is well-ordered with respect to the dictionary ordering.
}
\pf{Proof by Induction}{
	We will prove this by induction on the number of sets. If there is one set, then it is trivial.

	Assume the theorem holds for $n$ sets. Suppose we have $n+1$ sets $A_1, A_2, \cdots, A_{n+1}$.
	Then, $\prod_{i=2}^{n+1} A_i$ is well-ordered with respect to a dictionary ordering $<_1$.

	Let $<_2$ and $<_3$ be the dictionary order of $A_1 \times \prod_{i=2}^{n+1} A_i$ and $\prod_{i=1}^{n+1} A_i$, respectively.
	Since $\left(A_1 \times \prod_{i=2}^{n+1} A_i, <_2\right)$ and $\left(\prod_{i=1}^{n+1} A_i, <_3\right)$ has the same order type,
	we only need to prove that $\left(A_1 \times \prod_{i=2}^{n+1} A_i, <_2\right)$ is well-ordered.

	Let $\varnothing \neq S \subseteq A_1 \times \prod_{i=2}^{n+1} A_i$.
	If we define $S' \coloneqq \{\, a_1 \mid (a_1, b) \in S \,\} \subseteq A_1$,
	$S'$ is a nonempty subset of $A_1$, and therefore has $a_1' \coloneqq \min S'$.
	Similarly, if we define $S'' \coloneqq \{\, b_1 \mid (a_1', b_1) \in S \,\} \subseteq \prod_{i=2}^{n+1} A_i$,
	$S''$ is nonempty and has a smallest element $b_1'$. Then, $(a_1', b_1')$ is a smallest element of $A_1 \times \prod_{i=2}^{n+1} A_i$
	with respect to $<_2$.
}

\exer{}{
	$\prod_{i \in \ZZ_+} \ZZ_+$ is not well-ordered with respect to the dictionary ordering.
}
\sol
Let $x_{ij} \coloneqq \begin{cases}
		2 & \text{if }i = j    \\
		1 & \text{if }i \neq j
	\end{cases}$ for each $i \in \ZZ_+$ and $j \in \ZZ_+$.
The set $A \coloneqq \{\, (x_{i1}, x_{i2}, \cdots) \mid i \in \ZZ_+ \,\} \subseteq \prod_{i \in \ZZ_+} \ZZ_+$ has no smallest element.

\thm[wot]{Well-Ordering Theorem}{
	If $A$ is a set, then there exists an order relation on $A$ that is well-ordering.
}
The proof of \Cref{th:wot} involves the Axiom of Choice.

\cor[woUncountSet]{}{
	There exists an uncountable well-ordered set.
}

\dfn[section]{Section}{
	Let $X$ be a well-ordered set. Given $\alpha \in X$, let \[
		S_\alpha \coloneqq \{\, x \in X \mid x < \alpha \}\text{.}
	\] $S_\alpha$ is called the \textit{section} of $X$ by $\alpha$.
}

\mlemma[onlySOmegaUncount]{}{
	There exists a well-ordered set $A$ with the largest element $\Omega$, such that
	\begin{itemize}[nolistsep]
		\item section $S_{\Omega}$ of $A$ is uncountable, and,
		\item for every $\alpha \in A \setminus \{\,\Omega\,\}$, section $S_{\alpha}$ of $A$ is countable.
	\end{itemize}
}
\pf{Proof}{
	By \Cref{cor:woUncountSet}, there exists an uncountable well-ordered set $B$.
	Let $C \coloneqq \{\,1,2\,\} \times B$ be a set with a dictionary ordering.
	$C$ is well-ordered by \Cref{th:CofFinManyWOisWO}.

	Let $S \coloneqq \{\, \alpha \in C \mid \text{section } S_{\alpha} \text{ of } C \text{ is uncountable} \,\} \subseteq C$.
	We may let $\Omega \coloneqq \min S$. Then, the set $\ol{S_{\Omega}} = S_{\Omega} \cup \{\,\Omega\,\}$
	satisfies the two conditions.
}

\thm[AHasUBinSOmega]{}{
	If $A$ is a countable subset of $S_{\Omega}$ (in \Cref{lem:onlySOmegaUncount}),
	then $A$ has an upper bound in $S_{\Omega}$.
}
\pf{Proof}{
	For each $a \in A$, the section $S_{a}$ is countable;
	therefore, the union $B \coloneqq \bigcup_{a \in A} S_a$ is also countable by \Cref{exer:UnionOfCountSets}.

	Since $S_\Omega$ is uncountable, we may take an $x \in S_\Omega \setminus B$.
	If it were $x < a$ for some $a \in A$, then $x$ would be contained in $S_a$, which is a subset of $B$, \#.
	Therefore, $x \in S_\Omega$ is an upper bound of $A$.
}
\end{document}
