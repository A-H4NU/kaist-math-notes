\documentclass[MAS331_Note.tex]{subfiles}

\begin{document}

\section{Complete Metric Spaces}

\dfn{Cauchy Sequence}{
    Let $(X, d)$ be a metric space. A sequence $\{x_n\}_{n \in \ZZ_+}$
    of points of $X$ is said to be \textit{Cauchy} in $(X, d)$ if
    \[
        \fall \veps \in \RR_+,\: \exs N \in \ZZ_+,\:
        \fall n, m \in \ZZ_+,\: \big(n,m\ge N \implies d(x_n, x_m) < \veps\big)\text{.}
    \]
}

\dfn{Complete Metric Space}{
    Let $(X, d)$ be a metric space. $(X, d)$ is said to be \textit{complete}
    if every Cauchy sequence in converges.
}

\nt{
    \noindent Here are some immediate facts.
    \begin{itemize}[nolistsep]
        \ii Every convergent sequence in $(X, d)$ is a Cauchy sequence.
        \ii $\{x_n\}_{n \in \ZZ_+}$ is Cauchy in $(X, d)$ if and only if
            it is Cauchy in $(X, \ol d)$.
        \ii $(X, d)$ is complete if and only if $(X, \ol d)$ is Cauchy.
        \ii If $A$ is a closed subset of a complete metric space $(X, d)$,
            then it is complete in the restricted metric, i.e.,
            $\big(A, d\big|_{A^2}\big)$ is complete.
    \end{itemize}
}

\mlemma[complIfHasConvSubseq]{}{
    A metric space $(X, d)$ is complete if every Cauchy sequence in $X$ has a
    convergent subsequence.
}
\pf{Proof}{
    Let $\{x_n\}_{n \in \ZZ_+}$ be a Cauchy sequence in $(X, d)$. Let
    $\{x_{n_i}\}_{i \in \ZZ_+}$ be a subsequence that converges to $x \in X$.

    Given $\veps \in \RR_+$, there exists $N \in \ZZ_+$ such that
    $d(x_n, x_m) < \veps/2$ for all $n, m \ge N$. Then choose $i \in \ZZ_+$
    such that $n_i \ge N$ and $d(x_{n_i}, x) < \veps/2$. Then,
    for every $n \in \ZZ_+$ not smaller than $N$,
    \[
        d(x_n, x) \le d(x_n, d_{n_i}) + d(x_{n_i}, x) < \veps\text{.}
    \]
}

\thm[RkisCompl]{}{
    Euclidean space $\RR[k]$ is complete in either of its usual metrics,
    the euclidean metric $d$ of the square metric $\rho$.
}
\pf{Proof}{
    To show the metric space $(\RR[k], \rho)$ is complete, let $\{\vec x_n\}_{n \in \ZZ_+}$
    be a Cauchy
    sequence in $(\RR[k], \rho)$. There exists $N \in \ZZ_+$ such that
    \[
        \rho(\vec x_n, \vec x_m) \le 1
    \]
    for all $n, m \ge N$. Then
    \[
        M \triangleq \max \{\,\rho(\vec x_1, \vec 0), \cdots, \rho(\vec x_N, \vec 0),
        \rho(\vec x_N, \vec 0) + 1\,\}
    \]
    is an upper bound of $\{\rho(\vec x_n, \vec 0)\}_{n \in \ZZ_+}$.

    Thus, $\{x_n\}_{n \in \ZZ_+}$ is a sequence in $[-M, M]^k$ (product),
    which is compact by \Cref{th:cldIntvIsCpt} and \Cref{th:prodCptIffEachCpt}.
    ($[-M, M]^k$ as a subspace of $(\RR[k], \rho)$ is also a metric space
    and is compact by \Cref{th:subOfProdIsProdOfSub}.)
    As $[-M, M]^k$ is sequentially compact by \Cref{th:cptTFAE}, $\{x_n\}_{n \in \ZZ_+}$
    has a convergent subsequence in $[-M, M]^k$ (product). The subsequence converges
    in $(\RR[k], \rho)$ as well. By \Cref{lem:complIfHasConvSubseq}, $(\RR[k], \rho)$
    is complete.

    For $(\RR[k], d)$, note the following.
    \begin{itemize}[nolistsep]
        \ii $\{\vec x_n\}_{n \in \ZZ_+}$ is Cauchy in $(\RR[k], d)$ if and
            only if it is Cauchy in $(\RR[k], \rho)$.
        \ii $\{\vec x_n\}_{n \in \ZZ_+}$ converges in $(\RR[k], d)$ if and
            only if it converges in $(\RR[k], \rho)$.
    \end{itemize}
}

\mlemma[convInProdIff]{}{
    Let $X$ be the product space $X \triangleq \prod_{\alpha \in J} X_\alpha$;
    let $\{\vec x_n\}$ be a sequence of points of $X$. Then
    $\vec x_n \to \vec x$ if and only if $\pi_\alpha(\vec x_n) \to \pi_\alpha(\vec x)$
    for each $\alpha \in J$.
}
\pf{Proof}{
    ($\Rightarrow$) For each $\alpha \in J$, $\pi_\alpha$ is continuous.
    Hence, by \Cref{lem:seqConvIffImgConv}, $\pi_\alpha(\vec x_n) \to
    \pi_\alpha(\vec x)$.

    ($\Leftarrow$) Let $\bigcap_{i=1}^k\pi_{\alpha_i}\inv(U_{\alpha_i})$ be a
    basis element that contains $\vec x$.
    For each $i \in [k]$, there exists $N_i \in \ZZ_+$ such that
    $\pi_{\alpha_i}(\vec x_n) \in U_{\alpha_i}$ for all $n \ge N_i$.
    Then, for every $n \ge \max_{i=1}^k N_i$, $\vec x_n \in
    \bigcap_{i=1}^k\pi_{\alpha_i}\inv(U_{\alpha_i})$.
}

\thm{}{
    Suppose $(X_i, d_i)$ is a complete metric space for each $i \in \ZZ_+$.
    Let $X \triangleq \prod_{i \in \ZZ_+} X_i$ be a product space.
    Then, let $D \colon X \times X \to \RR$ be defined by
    \[
        D(\vec x, \vec y) \triangleq \sup \left\{\,\frac{\ol d_i(x_i, y_i)}{i} \:\bigg|\: i \in \ZZ_+\,\right\}\text{.}
    \]
    Then, the metric space $(X, D)$ is complete.
}
\pf{Proof}{
    $D$ induces the product topology by \Cref{th:ctMetIsMet}.
    Let $\{\vec x_n\}_{n \in \ZZ_+}$ be a Cauchy sequence in $(X, D)$.
    Since
    \[
        \ol d_i(\pi_i(\vec x), \pi_i(\vec y)) \le i \cdot D(\vec x, \vec y)
    \]
    for each $i \in \ZZ_+$ and $x, y \in X$,
    $\{\pi_i(\vec x_n)\}_{n \in \ZZ_+}$ is a Cauchy sequence in $(X_i, d_i)$;
    it converges. Hence, $\{\vec x_n\}$ converges in $X$ by \Cref{lem:convInProdIff}.
}

\thm[complThenFtnSpaceCompl]{}{
    If $(Y, d)$ is a complete metric space, then the metric space
    $(Y^J, \ol \rho)$ is complete. ($\ol\rho$ is the uniform metric.)
}
\pf{Proof}{
    Suppose $\{f_n\}_{n \in \ZZ_+}$ is Cauchy
    in $\ol \rho$. Take any $\alpha \in J$. Since
    \[
        \ol d(f_n(\alpha), f_m(\alpha)) \le \ol \rho(f_n, f_m)
    \]
    for each $n, m \in \ZZ_+$, the sequence $\{f_n(\alpha)\}_{n \in \ZZ_+}$
    is Cauchy in $(Y, \ol d)$, and thus converges.
    Let $y_\alpha$ be the point to which
    it converges. Let $f \colon J \to Y$ be defined by
    \[
        \alpha \mapsto y_\alpha\text{.}
    \]
    
    We now claim that $f_n \to f$. Let $\veps \in \RR_+$ be given.
    There exists $N \in \ZZ_+$ such that $\ol\rho(f_n, f_m) < \veps/2$
    whenever $n, m \ge N$.

    Take any $\alpha \in J$ and $\veps' \in \RR_+$.
    There exists $M \in \ZZ_+$ such that
    $\ol d(f_m(\alpha), f(\alpha)) < \veps$ for each $m \ge M$.
    Then, for $n \ge N$ and $m \ge \max \{N, M\}$,
    \[
        \ol d(f_n(\alpha), f(\alpha)) \le \ol d(f_n(\alpha), f_m(\alpha))
        + \ol d(f_m(\alpha), f(\alpha)) < \veps/2 + \veps'\text{.}
    \]
    Since $\veps'$ was arbitrary, $\ol d(f_n(\alpha), f(\alpha)) \le \veps/2$
    for each $n \ge N$. Since $\alpha$ was arbitrary, $\ol \rho(f_n, f) \le \veps/2 < \veps$
    for all $n \ge N$.
}

\dfn{Space of Continuous/Bounded Function}{
    Let $X$ be a topological space and let $(Y, d)$ be a metric space.
    Then, define $\mcal C(X, Y), \mcal B(X, Y) \subseteq Y^X$ by
    \[
        \mcal C(X, Y) \triangleq \{\,f \in Y^X \mid f \text{ is continuous}\,\}
    \]
    and
    \[
        \mcal B(X, Y) \triangleq \{\,f \in Y^X \mid f \text{ is bounded
        with respect to }d\,\}\text{.}
    \]
}

\thm[CandBCld]{}{
    Let $X$ be a topological space and let $(Y, d)$ be a metric space.
    The set $\mcal C(X, Y)$ and $\mcal B(X, Y)$ are closed in $(Y^X, \ol\rho)$.
}
\pf{Proof}{
    ($\mcal C(X, Y)$ is closed.)
    Let $f \in Y^X$ be a limit point of $\mcal C(X, Y)$. Then, there exists
    a sequence $\{f_n\}_{n \in \ZZ_+}$ in $\mcal C(X, Y)$ converging to $f$
    in the metric $\ol\rho$ by \Cref{lem:seqLemma}.

    Now, we claim that $\{f_n\}_{n \in \ZZ_+}$ converges to $f$ uniformly.
    Take any $\veps \in \RR_+$. Then, there exists some $N \in \ZZ_+$ such that
    $\ol\rho(f_n, f) < \veps$ for each $n \ge N$.
    Then, for every $x \in X$ and $n \ge N$,
    \[
        \ol d(f_n(x), f(x)) \le \ol\rho(f_n, f) < \veps\text{,}
    \]
    which implies $\{f_n\}_{n \in \ZZ_+}$ uniformly converges to $f$.
    Then, by \Cref{th:unifLimThm}, $f \in \mcal C(X, Y)$.
    Hence, $\mcal C(X, Y)$ is closed by \Cref{cor:closedIffContainsLimPts}.

    ($\mcal B(X, Y)$ is closed.)
    Let $f \in Y^X$ be a limit point of $\mcal B(X, Y)$.
    Then, there exists
    a sequence $\{f_n\}_{n \in \ZZ_+}$ in $\mcal B(X, Y)$ converging to $f$
    in the metric $\ol\rho$ by \Cref{lem:seqLemma}.

    There exists $N \in \ZZ_+$ such that $\ol\rho(f_N, f) < 1$.
    Let $x, y \in X$. Then,
    \[
        d(f(x), f(y)) \le d(f(x), f_N(x)) + d(f_N(x), f_N(y))
        + d(f_N(y), f(y)) < \diam \Img f_N + 2\text{.}
    \]
    Hence, $f \in \mcal B(X, Y)$. Therefore, $\mcal B(X, Y)$ is closed by
    \Cref{cor:closedIffContainsLimPts}.
}

\cor[CandBCompl]{}{
    Let $X$ be a topological space and let $(Y, d)$ be a complete metric space.
    The set $\mcal C(X, Y)$ and $\mcal B(X, Y)$ are complete in $(Y^X, \ol\rho)$.
}
\pf{Proof}{
    $(Y^X, \ol\rho)$ is complete by \Cref{th:complThenFtnSpaceCompl}.
    Since $\mcal C(X, Y)$ and $\mcal B(X, Y)$ are closed
    by \Cref{th:CandBCld}, they are also complete.
}

\mlemma{}{
    Let $X$ be a compact space and let $(Y, d)$ be a metric space.
    Then, $\mcal C(X, Y) \subseteq \mcal B(X, Y)$, i.e., every continuous
    function from $X$ to $Y$ is bounded.
}
\pf{Proof}{
    Let $f \in \mcal C(X, Y)$. Then, $\Img f$ is compact by
    \Cref{th:contiSendsCptToCpt}. Thus, it is bounded by \Cref{th:cptMetIsBdd}.
}

\dfn{Sup Metric}{
    Let $(Y, d)$ be a metric space. We may define another metric $\rho$ on
    the set $\mcal B(X, Y)$ by the equation
    \[
        \rho(f, g) \triangleq \sup \{\,d(f(x), g(x)) \mid x \in X\,\}\text{.}
    \]
    The metric $\rho$ is called the \textit{sup metric}.
}

\nt{
    Let $\rho$ and $\ol \rho$ be the sup metric and the uniform metric,
    respectively, on $\mcal B(X, Y)$.
    Then, the following holds.
    \[
        \ol \rho(f, g) = \min \{\rho(f, g), 1\}
    \]
    This means that $\ol \rho$ is just the standard bonded metric
    derived from $\rho$.
}

\nt{
    Let $X$ be a topological space and let $(Y, d)$ be a complete metric space.
    $\mcal B(X, Y)$ is complete in $(Y^X, \rho)$.
    If $X$ is compact, $\mcal C(X, Y)$ is complete in $(Y^X, \rho)$.
}

\dfn{Isometric Imbedding}{
    Let $(X, d_X)$ and $(Y, d_Y)$ be metric spaces.
    If $f \colon X \to Y$ has the property such that
    \[
        \fall x_1, x_2 \in X\,\:
        d_Y(f(x_1), f(x_2)) = d_X(x_1, x_2)\text{,}
    \]
    $f$ is called an \textit{isometric imbedding} of $X$ in $Y$.
}

\nt{
    Let $(X, d_X)$ and $(Y, d_Y)$ be metric spaces.
    If $f \colon X \to Y$ is an isometric imbedding of $X$ in $Y$,
    then it is an imbedding of $X$ in $Y$.
    \pf{Proof}{
        If $f(x_1) = f(x_2)$, then $d_X(x_1, x_2) = d_Y(f(x_1), f(x_2)) = 0$,
        and thus $x_1 = x_2$. This shows that $f$ is an injection.
        Moreover, for each $x \in X$ and $\veps \in \RR_+$,
        \begin{itemize}[nolistsep]
            \ii $f(B_{d_X}(x, \veps)) = B_{d_Y}(f(x), \veps) \cap \Img f$ and
            \ii $f\inv(B_{d_Y}(f(x), \veps)) = B_{d_X}(x, \veps)$.
        \end{itemize}
        Hence, $f$ is an imbedding of $X$ in $Y$.
    }
}

\thm[isoImbed]{}{
    Let $(X, d)$ be a metric space. Then, there is an isometric imbedding of
    $X$ in a complete metric space.
}
\pf{Proof}{
    Consider $\mcal B(X, \RR)$. Let $x_0 \in X$ be fixed.
    For each $a \in X$, define $\phi_a \colon X \to \RR$ by the equation
    \[
        \phi_a(x) \triangleq d(x, a) - d(x, x_0)\text{.}
    \]
    From the triangle inequality, we get
    \[
        |\phi_a(x)| = |d(x, a) - d(x, x_0)|
        \le d(a, x_0)
    \]
    for each $x \in X$. Hence, $\phi_a \in \mcal B(X, \RR)$.
    Define $\Phi \colon X \to \mcal B(X, \RR)$ by letting $a \mapsto \phi_a$.

    We now claim that $\Phi$ is an isometric imbedding of $X$ in the
    complete metric space $(\mcal B(X, \RR), \rho)$.
    By definition, for each $a, b \in X$,
    \[\begin{aligned}[t]
        \rho(\phi_a, \phi_b)
        &= \sup \big\{\, |\phi_a(x) - \phi_b(x)| \:\big|\: x \in X\,\big\} \\
        &= \sup \big\{\, |d(x, a) - d(x, b)| \:\big|\: x \in X\,\big\}
        \le d(a, b)\text{.}
    \end{aligned}\]
    Moreover,
    \[
        d(a, b) = |d(a, a) - d(a, b)|
        \le \sup \big\{\, |d(x, a) - d(x, b)| \:\big|\: x \in X\,\big\}
        = \rho(\phi_a, \phi_b)\text{.}
    \]
    Hence, $d(a, b) = \rho(\phi_a, \phi_b)$; $\Phi$ is an isometric imbedding.
}

\dfn{Completion}{
    Let $X$ be a metric space. If $h \colon X \to Y$ is an isometric imbedding
    of $X$ into a complete metric space $Y$, then the space
    $\cl{h(X)}$ of $Y$ is a complete metric space. It is called the
    \textit{completion} of $X$.
}

\nt{
    \noindent The completion of $X$ is uniquely determined up to an isometry.
}

\section{A Space-Filling Curve}

\thm[spaceFill]{}{
    Let $I = [0, 1]$. There exists a surjective continuous map $f \colon I \to I^2$
    where $I^2$ is a subspace of $\RR^2$.
}
\pf{Proof}{
    \tikzset{test/.style={
    postaction={
        decorate,
        decoration={
            markings,
            % mark=at position \pgfdecoratedpathlength-0.5pt with {\arrow[blue,line width=#1] {>}; },
            mark=between positions 0 and \pgfdecoratedpathlength-8pt step .5pt with {
                \pgfmathsetmacro\myval{multiply(divide(
                    \pgfkeysvalueof{/pgf/decoration/mark info/distance from start}, \pgfdecoratedpathlength),100)};
                \pgfsetfillcolor{blue!\myval!red};
                \pgfpathcircle{\pgfpointorigin}{#1};
                \pgfusepath{fill};}
    }}}}
    \pgfplotsset{
        clip=false,
        axis x line*=box,
        axis y line*=box,
        xlabel=,
        ylabel=,
        xmin=0, xmax=1,
        ymin=0, ymax=1,
        x=4cm, y=4cm,
        xtick distance=.0625,
        ytick distance=.0625,
        xticklabels={,0,,,,,,,,,,,,,,,,1},
        yticklabels={,0,,,,,,,,,,,,,,,,1},
    }
    We will define a sequence $\{f_n\}_{n \in \ZZ_+}$ in $\mcal C(I, I^2)$
    where $I^2$ and $\RR^2$ has the square metric
    \[
        d(\vec x, \vec y) = \max \{ |x_1-y_1|, |x_2-y_2| \}
    \]
    so that the sup metric is
    \[
        \rho(f, g) = \sup \{\, d(f(t), g(t)) \mid t \in I \,\}\text{.}
    \]
    (In the figures below, $0$ is mapped to the red point and $1$ is mapped
    to the blue point.)
    \begin{center}
    \footnotesize
    \begin{tikzpicture}[]
        \pgfmathsetmacro{\n}{1}
        \pgfmathsetmacro{\twoton}{2^\n}
        \pgfmathsetmacro{\twotonminusone}{2^\n-1}
        \begin{axis}[]
            \addplot[gray, semithick] coordinates { (0.5, 0) (0.5, 1) };
            \addplot[gray, semithick] coordinates { (0, 0.5) (1, 0.5) };
            \addplot [mesh, point meta=\thisrow{i}, ultra thick, colormap={}{
                    color(0cm)=(red);
                    color(1cm)=(blue);
                }] table {./f0.dat};
            \node at (0.5, -0.15) {$f_0$};
        \end{axis}
    \end{tikzpicture}
    \begin{tikzpicture}[]
        \pgfmathsetmacro{\n}{2}
        \pgfmathsetmacro{\twoton}{2^\n}
        \pgfmathsetmacro{\twotonminusone}{2^\n-1}
        \begin{axis}[]
            \foreach \k in {1, ..., \twotonminusone} {
                \addplot[gray, dotted, thin] coordinates {
                    ({\k/\twoton}, 0) ({\k/\twoton}, 1)
                };
                \addplot[gray, dotted, thin] coordinates {
                    (0, {\k/\twoton}) (1, {\k/\twoton})
                };
            }
            \addplot[gray, semithick] coordinates { (0.5, 0) (0.5, 1) };
            \addplot[gray, semithick] coordinates { (0, 0.5) (1, 0.5) };
            \addplot [mesh, point meta=\thisrow{i}, ultra thick, colormap={}{
                    color(0cm)=(red);
                    color(1cm)=(blue);
                }] table {./f1.dat};
            \node at (0.5, -0.15) {$f_1$};
        \end{axis}
    \end{tikzpicture}
    \begin{tikzpicture}[]
        \pgfmathsetmacro{\n}{3}
        \pgfmathsetmacro{\twoton}{2^\n}
        \pgfmathsetmacro{\twotonminusone}{2^\n-1}
        \begin{axis}[]
            \foreach \k in {1, ..., \twotonminusone} {
                \addplot[gray, dotted, thin] coordinates {
                    ({\k/\twoton}, 0) ({\k/\twoton}, 1)
                };
                \addplot[gray, dotted, thin] coordinates {
                    (0, {\k/\twoton}) (1, {\k/\twoton})
                };
            }
            \addplot[gray, semithick] coordinates { (0.5, 0) (0.5, 1) };
            \addplot[gray, semithick] coordinates { (0, 0.5) (1, 0.5) };
            \addplot [mesh, point meta=\thisrow{i}, ultra thick, colormap={}{
                    color(0cm)=(red);
                    color(1cm)=(blue);
                }] table {./f2.dat};
            \node at (0.5, -0.15) {$f_2$};
        \end{axis}
    \end{tikzpicture}
    \end{center}
    
    \begin{center}
    \footnotesize
    \begin{tikzpicture}[]
        \pgfmathsetmacro{\n}{4}
        \pgfmathsetmacro{\twoton}{2^\n}
        \pgfmathsetmacro{\twotonminusone}{2^\n-1}
        \begin{axis}[]
            \foreach \k in {1, ..., \twotonminusone} {
                \addplot[gray, dotted, thin] coordinates {
                    ({\k/\twoton}, 0) ({\k/\twoton}, 1)
                };
                \addplot[gray, dotted, thin] coordinates {
                    (0, {\k/\twoton}) (1, {\k/\twoton})
                };
            }
            \addplot[gray, semithick] coordinates { (0.5, 0) (0.5, 1) };
            \addplot[gray, semithick] coordinates { (0, 0.5) (1, 0.5) };
            \addplot [mesh, point meta=\thisrow{i}, thick, colormap={}{
                    color(0cm)=(red);
                    color(1cm)=(blue);
                }] table {./f3.dat};
            \draw[black, semithick] (.25, .125) rectangle (.375, .25);
            \node at (0.5, -0.15) {$f_3$};
        \end{axis}
    \end{tikzpicture}
    \begin{tikzpicture}[]
        \pgfmathsetmacro{\n}{4}
        \pgfmathsetmacro{\twoton}{2^\n}
        \pgfmathsetmacro{\twotonminusone}{2^\n-1}
        \begin{axis}[]
            \foreach \k in {1, ..., \twotonminusone} {
                \addplot[gray, dotted, thin] coordinates {
                    ({\k/\twoton}, 0) ({\k/\twoton}, 1)
                };
                \addplot[gray, dotted, thin] coordinates {
                    (0, {\k/\twoton}) (1, {\k/\twoton})
                };
            }
            \addplot[gray, semithick] coordinates { (0.5, 0) (0.5, 1) };
            \addplot[gray, semithick] coordinates { (0, 0.5) (1, 0.5) };
            \addplot [mesh, point meta=\thisrow{i}, thick, colormap={}{
                    color(0cm)=(red);
                    color(1cm)=(blue);
                }] table {./f4.dat};
            \draw[black, semithick] (.25, .125) rectangle (.375, .25);
            \node at (0.5, -0.15) {$f_4$};
        \end{axis}
    \end{tikzpicture}
    \end{center}

    \noindent
    In other words, $\{f_n\}_{n \in \ZZ_+}$ can be described by
    the following recursive relation.
    \begin{center}
    \begin{tikzpicture}
        \begin{axis}[]
            \node at (0.5, 0.5) {$f_n$};
            \node at (0.5, -0.15) {$f_{n}$};
        \end{axis}
    \end{tikzpicture}
    \begin{tikzpicture}
        \begin{axis}[]
            \addplot[gray, semithick] coordinates { (0.5, 0) (0.5, 1) };
            \addplot[gray, semithick] coordinates { (0, 0.5) (1, 0.5) };
            \node[rotate=-90] at (0.25, 0.25) {$f_n$};
            \node at (0.25, 0.75) {$f_n$};
            \node at (0.75, 0.75) {$f_n$};
            \node[rotate=90] at (0.75, 0.25) {$f_n$};
            \node at (0.5, -0.15) {$f_{n+1}$};
        \end{axis}
    \end{tikzpicture}
    \end{center}

    \noindent
    Or, we may describe it with mathematically as following.
    \[
        f_0(t) \triangleq \begin{cases}
            t \times t & \text{if } 0 \le t < 1/2 \\
            t \times (1-t) & \text{if } 1/2 \le t \le 1
        \end{cases}
    \]
    and
    \[
        f_{n+1}(t) \triangleq \begin{cases}
            \frac{\pi_2(f_n(1-4t))}{2} \times \big(\frac{1}{2}-\frac{\pi_1(f_n(1-4t))}{2}\big)
            & \text{if } 0 \le t < 1/4 \\
            \frac{f_n(4t-1)}{2}+\big(0\times\frac{1}{2}\big)
            & \text{if } 1/4 \le t < 1/2 \\
            \frac{f_n(4t-2)}{2}+\big(\frac{1}{2}\times\frac{1}{2}\big)
            & \text{if } 1/2 \le t < 3/4 \\
            \big(1-\frac{\pi_2(f_n(4-4t))}{2}\big) \times \frac{\pi_1(f_n(4-4t))}{2}
            & \text{if } 3/4 \le t \le 1\text{.} \\
        \end{cases}
    \]

    Note that each of the small triangular paths (that is similar to $f_0$)
    that make up $f_n$ lies in
    a square of edge length $1/2^n$, and the triangular path is replaced by
    four smaller triangular paths (that are similar to $f_1$ when combined)
    that lie in the same square. (See the black squares in the figure above.)
    Hence, for each $n \in \ZZ_+$, we have
    \[
        \rho(f_n, f_{n+1}) \le \frac{1}{2^n}\text{.}
    \]
    Hence, for each $n, m \in \ZZ_+$ with $n < m$,
    \[
        \rho(f_n, f_m) \le \sum_{j=n}^{m-1} \rho(f_j, f_{j+1})
        < \sum_{j=n}^{m-1} 2^{-j} = 2^{1-n}(1-2^{m-n}) < 2^{1-n}\text{.}
    \]
    Hence, $\{f_n\}_{n \in \ZZ_+}$ is a Cauchy sequence in $\mcal C(I, I^2)$
    under $\rho$.
    
    Since $I^2$ is closed in $\RR^2$, which is complete under the usual metric,
    $\mcal C(I, I^2)$ is complete by \Cref{cor:CandBCompl}.
    Hence, $\{f_n\}_{n \in \ZZ_+}$ converges
    to some function $f \in \mcal C(I, I^2)$.

    We now claim that $f$ is surjective.
    Since $I$ is compact (\Cref{cor:cldIntvInRIsCpt}), $f(I)$ is compact by
    \Cref{th:contiSendsCptToCpt}. Since $I^2$ is $T_2$, $f(I)$ is closed
    by \Cref{th:cptInHausIsCld}. 
    Let $\vec x$ be a point of $I^2$. Take any $\veps \in \RR_+$.
    Since $f_n \to f$, there exists $N \in \ZZ_+$ such that
    \[
        \rho(f_N, f) < \veps/2 \quad\text{and}\quad 1/2^N < \veps/2\text{.}
    \]
    Moreover, by the construction, there exists $t_0 \in I$ such that
    \[
        d(\vec x, f_N(t_0)) \le 1/2^N\text{.}
    \]
    Hence, we have
    \[
        d(\vec x, f(t_0)) \le d(\vec x, f_N(t_0)) + d(f_N(t_0), f(t_0))
        < \veps\text{.}
    \]
    Therefore, $\veps$-neighborhood of $\vec x$ intersects $f(I)$;
    this implies $\vec x \in \cl{f(I)}$ by \Cref{th:inClosureIffNeighCapANonempty}.
    Since $f(I)$ is closed, $\vec x \in f(I)$.
}

\cor[spaceFillIn]{}{
    Let $I = [0, 1]$. There exists a surjective continuous map $f \colon I \to I^n$
    where $I^n$ is a subspace of $\RR[n]$.
}
\pf{Proof}{
    \hfill
    \begin{center}
    \fbox{
    \begin{minipage}{.9\linewidth}
        \textbf{Claim.} 
        For each $k \in \ZZ_+$, there exists a surjective continuous map
        $f \colon I \to I^{2^k}$.
        \pf{Proof}{
            (Induction on $k$) The base case $f \colon I \to I^2$ is proven
            in \Cref{th:spaceFill}.

            Suppose we have surjective continuous map
            $g \colon I \to I^{2^k}$.
            We already have surjective and continuous $h \colon I \to I^2$.
            Then, $p \colon I \times I \to I^{2^k} \times I^{2^k}$
            defined by
            \[
                p(s, t) \triangleq g(s) \times g(t)
            \]
            is continuous since $p(x) = g(\pi_1(x)) \times g(\pi_2(x))$.
            (See \Cref{th:constructConti}.)
            Then, $f \colon I \to I^{2^k} \times I^{2^k}$
            defined by $f \triangleq p \circ h$ is continuous.
            $f$ is surjective, indeed.
            Since $I^{2^k} \times I^{2^k}$ is homeomorphic with $I^{2^{k+1}}$,
            there exists a surjective continuous map from $I$ to $I^{2^{k+1}}$.
        }
    \end{minipage}
    }
    \end{center}

    Let $k \in \ZZ_+$ be an integer such that $n < 2^k$.
    Since $I^{2^k}$ is homeomorphic with $I^{n} \times I^{2^k-n}$,
    by the claim, there exists a surjective continuous map
    $g \colon I \to I^n \times I^{2^k-n}$.
    Then, $f \colon I \to I^n$ defined by $f = \pi_1 \circ g$ is
    surjective and continuous.
}

\cor{}{
    There exists a surjective continuous map from $\RR$ to $\RR[n]$.
}
\pf{Proof}{
    Let $g \colon \ZZ \to \ZZ \times \ZZ$ be a surjection.
    For each $n \in \ZZ$, we have a surjective continuous map $f_n$
    from $[2n, 2n+1]$ to $[(\pi_1 \circ g)(n), (\pi_1 \circ g)(n) + 1]
    \times [(\pi_2 \circ g)(n), (\pi_2 \circ g)(n) + 1]$ by \Cref{th:spaceFill}.
    If we define $f \colon \RR \to \RR[2]$ be the function that equals
    $f_n$ on $[2n, 2n+1]$ and connects $f_n(2n+1)$ and $f_{n+1}(2n+2)$
    on $[2n+1, 2n+2]$ by a line segment for each $n \in \ZZ_+$ is a surjective
    and continuous map. The map $f \colon \RR \to \RR[2]$ is
    \[
        f(t) \triangleq \begin{cases}
            f_n(t) & \text{if } 2n \le t < 2n+1 \\
            (f_{n+1}(2n+2)-f_n(2n+1))(t-2n-1) + f_n(2n+1)
                   & \text{if } 2n+1 \le t < 2n+2\text{.}
        \end{cases}
    \]

    Similarly (\Cref{cor:spaceFillIn}), there exists a surjective continuous
    map from $\RR$ to $\RR[n]$.
}

\section{Compactness in Metric Spaces}

\nt{
    We know that a subspace $A$ of $\RR[n]$ is compact
    if and only if $A$ is closed and bounded under either the square metric
    or the Euclidean metric.
    In this chapter, we want to generalize this into $\mcal C(X, \RR[n])$
    where $X$ is a topological space.
    Which subspace $\mcal F \subseteq (\mcal C(X, \RR[n]), \ol\rho)$
    would be compact?
    Note that $(\RR[n], d) \cong (\mcal C(X, \RR[n]), \ol\rho)$ when $X$ is
    a singleton.
}

\dfn{Totally Bounded Metric Space}{
    A metric space $(X, d)$ is said to be \textit{totally bounded} if,
    for every $\veps \in \RR_+$, there exists a finite covering of $X$
    by $\veps$-balls.
}

\nt{
    \noindent
    Total boundedness implies boundedness.
    Copy the proof of \Cref{th:cptMetIsBdd}.
}

\nt{
    \noindent
    $(X, d)$ is totally bounded if and only if $(X, \ol d)$ is totally bounded
    where $\ol d$ is the standard bounded metric corresponding to $d$.
}

\nt{
    Total boundedness and completeness are independent.
    Under the metric $d(a, b) = |a, b|$, $\RR$ is complete but
    not totally bounded.
    The subspace $(-1, 1)$ is totally bounded but not complete.
}

\thm[metCptIffComplAndTotBdd]{}{
    A metric space $(X, d)$ is compact if and only if
    it is complete and totally bounded.
}
\pf{Proof}{
    ($\Rightarrow$)
    $X$ is sequentially compact by \Cref{th:cptTFAE}, and thus
    every sequence has a convergent subsequence. Therefore,
    $(X, d)$ is complete by \Cref{lem:complIfHasConvSubseq}.

    Moreover, $(X, d)$ is totally bounded since the open covering
    \[
        \{\,B(x, \veps) \mid x \in X\,\}
    \]
    will give a finite subcover of $X$ for any $\veps \in \RR_+$.

    ($\Leftarrow$)
    We claim that $X$ is sequentially compact, and thus compact by \Cref{th:cptTFAE}.
    Let $\{x_n\}_{n \in \ZZ_+}$ be a sequence of points of $X$.
    
    Let $J_0 \triangleq \ZZ_+$.
    Construct nested infinite subsets $J_1, J_2, \cdots$ of $\ZZ_+$
    as following.
    \begin{itemize}[nolistsep]
        \ii Given $J_{k-1}$, cover $X$ by finitely many $1/k$-balls.
            Then, there exists a $1/k$-ball $B_k$ among them such that
            there are infinitely many $n \in J_{k-1}$ such that
            $x_n \in B_k$. Let $J_k \triangleq
            \{\,n \in J_{k-1} \mid x_n \in B_k\,\}$.
    \end{itemize}
    Now, construct $\{n_i\}_{i \in \ZZ_+}$ as following.
    \begin{itemize}[nolistsep]
        \ii Take any $n_1 \in J_1$.
        \ii Given $n_i$, choose $n_{i+1} \in J_{i+1}$ such that $n_{i+1} > n_i$.
    \end{itemize}
    For all $i, j \ge k$, $x_{n_i}, x_{n_j} \in B_k$ so that
    $d(x_{n_i}, x_{n_j}) < 2/k$. Hence, the subsequence $\{x_{n_i}\}_{i \in \ZZ_+}$
    of $\{x_n\}_{n \in \ZZ_+}$ is a Cauchy sequence, and thus converges
    since $(X, d)$ is complete. This shows that $X$ is sequentially compact.
}

\dfn{Equicontinuity}{
    Let $X$ be a topological space and let $(Y, d)$ be a metric space.
    Let $\mcal F \subseteq \mcal C(X, Y)$.
    Given $x_0 \in X$, $\mcal F$ is said to be \textit{equicontinuous at} $x_0$
    if, for each $\veps \in \RR_+$, there exists a neighborhood $U$ of $x_0$
    such that
    \[
        d(f(x), f(x_0)) < \veps
    \]
    for all $x \in U$ and $f \in \mcal F$.
    If $\mcal F$ is equicontinuous at every $x_0 \in X$, $\mcal F$ is simply
    said to be \textit{equicontinuous}.
}

\nt{
    \noindent
    Let $X$ be a topological space and $(Y, d)$ be a metric space,
    If $\mcal F \subseteq \mcal G \subseteq \mcal C(X, Y)$ and
    $\mcal G$ is equicontinuous under $d$, then so is $\mcal F$.
}

\nt{
    \noindent
    Equicontinuity depends on the specific metric $d$ rather than
    merely on the topology of $Y$.
}

\mlemma[totBddThenEquiconti]{}{
    Let $X$ be a topological space and let $(Y, d)$ be a metric space.
    If $\mcal F \subseteq \mcal C(X, Y)$ is totally bounded under the
    uniform metric $\ol\rho$ corresponding to $d$, then $\mcal F$
    is equicontinuous under $d$.
}
\pf{Proof}{
    Take any $\veps \in \RR_+$ and $x_0 \in X$.
    Let $\delta \triangleq \veps/3$; cover $\mcal F$ by finitely many
    $\delta$-balls
    \[
        B(f_1, \delta), \cdots, B(f_n, \delta)
    \]
    in $(\mcal C(X, Y), \ol\rho)$.
    For each $i \in [n]$, since $f_i$ is continuous,
    $U_i \triangleq f_i\inv\big(B_d(f_i(x_0), \delta)\big)$
    is a neighborhood of $x_0$.
    Let $U \triangleq \bigcap_{i=1}^n U_i$.
    Then,
    \[
        \fall x \in U,\: \fall i \in [n],\:
        d(f_i(x), f_i(x_0)) < \delta\text{.}
    \]

    Take any $f \in \mcal F$. Then $f \in B(f_i, \delta)$ for some $i \in [n]$.
    Then, for any $x \in U$, we have
    \[\begin{aligned}[t]
        \ol d(f(x), f_i(x)) &< \delta\text{,} \\
        d(f_i(x), f_i(x_0)) &< \delta\text{, and} \\
        \ol d(f_i(x_0), f(x_0)) &< \delta\text{.}
    \end{aligned}\]
    Since $\delta = \veps/3 < 1$, we may replace $\ol d$ with $d$ in the
    inequalities above. Hence, the triangle inequality gives
    \[
        \fall x \in U,\: d(f(x_0), f(x)) < 3\delta = \veps\text{.}
    \]
}

\mlemma[cptEquicontiThenTotBdd]{}{
    Let $X$ be a compact space; let $(Y, d)$ be a compact metric space.
    If $\mcal F \subseteq \mcal C(X, Y)$ is equicontinuous under $d$,
    then $\mcal F$ is totally bounded under $\ol\rho$ and $\rho$.
}
\pf{Proof}{
    Take any $\veps \in \RR_+$. Let $\delta \triangleq \veps/3$.
    Given any $a \in X$, there exists a neighborhood $U_a$ of $a$
    such that
    \[
        \fall x \in U_a,\: \fall f \in \mcal F,\:
        d(f(x), f(a)) < \delta
    \]
    as $\mcal F$ is equicontinuous under $d$.
    Since $X$ is compact, we may take
    $a_1, a_2, \cdots, a_k \in X$ such that $\{U_{a_i}\}_{i \in [k]}$ covers $X$.
    Since $(Y, d)$ is totally bounded, we may cover $Y$
    with finitely many $\delta$-balls $V_1, V_2, \cdots, V_m$.

    \centerbox{
        \mclm{Claim}{
            For each $f \in \mcal F$, there exists $\alpha \colon [k] \to [m]$
            such that, for each $i \in [k]$, $f(a_i) \in V_{\alpha(i)}$.
        }
        \pf{Proof}{
            For each $i \in [k]$, $f(a_i) \in V_j$ for some $j \in [m]$.
            Let $\alpha(i) \coloneqq j$. The function $\alpha \colon [k] \to [m]$
            satisfies the desired condition.
        }
    }

    Let
    \[
        J \triangleq \big\{\,\alpha \colon [k] \to [m] \:\big|\:
        \exs f \in \mcal F,\:\forall i \in [k],\: f(a_i) \in V_{\alpha(i)}\,\big\}\text{.}
    \]
    Note that $J$ is finite, and $J \neq \OO$ by \textbf{\textit{Claim}}.

    For each $\alpha \in J$, choose $f_\alpha \in \mcal F$ such that
    $\fall i \in [k],\: f(a_i) \in V_{\alpha(i)}$.
    We now claim that $\big\{\,B_\rho(f_\alpha, \veps)\:\big|\:\alpha \in J\,\big\}$
    covers $\mcal F$.

    Take any $f \in \mcal F$.
    By \textbf{\textit{Claim}}, there exists some $\alpha \in J$ such that
    $f(a_i) \in V_{\alpha(i)}$ for all $i \in [k]$.
    Let $x \in X$. Choose $i \in [k]$ such that $x \in U_{a_i}$.
    Then, since $x \in U_{a_i}$,
    \[\begin{aligned}[t]
        d(f(x), f(a_i)) &< \delta\text{ and} \\
        d(f_\alpha(x), f_\alpha(a_i)) &< \delta
    \end{aligned}\]
    hold.
    Moreover, because $f(a_i), f_{\alpha}(a_i) \in V_{\alpha(i)}$, we have
    \[
        d(f(a_i), f_\alpha(a_i)) < \delta\text{.}
    \]
    Combining all the inequalities, we get
    \[
        d(f(x), f_{\alpha}(x)) < 3\delta = \veps\text{.}
    \]
    As $x$ is arbitrary, we have
    \[
        \rho(f, f_\alpha) = \max_{x \in X} d(f(x), f_\alpha(x)) < \veps\text{,}
    \]
    which implies $f \in B_{\rho}(f_\alpha, \veps)$.
    (We may take $\max$ since $X$ is compact. See \Cref{th:evt}.
    Is $(x_1 \times x_2) \mapsto d(x_1, x_2)$ continuous?
    One may begin the proof with $\delta = \veps/4$ to avoid this
    additional argument.)

    $\mcal F$ is totally bounded under $\ol\rho$ as $\ol\rho$ is just a
    standard bounded metric of $\rho$.
}

\dfn{Pointwise Boundedness}{
    Let $X$ be a topological space and let $(Y, d)$ be a metric space.
    Let $\mcal F \subseteq \mcal C(X, Y)$.
    $\mcal F$ is said to be \textit{pointwise bounded} under $d$ if,
    for each $a \in X$, the subset
    \[
        \mcal F_a \triangleq \{\,f(a) \mid f \in \mcal F\,\}
    \]
    of $Y$ is bounded under $d$.
}

\thm[ascoliClassic]{Ascoli's Theorem (Classical Version)}{
    \begin{itemize}[nolistsep]
        \ii Let $X$ be a compact space and let $(\RR[n], d)$ be the Euclidean space
            in either the square metric or the Euclidean metric.
        \ii Let $\mcal F$ be a subspace of $(\mcal C(X, \RR[n]), \ol\rho)$.
    \end{itemize}
    Then, $\cl{\mcal F}$ is compact if and only if $\mcal F$
    is equicontinuous and pointwise bounded under $d$.
}
\pf{Proof}{
    Let $\mcal G \triangleq \cl{\mcal F}$.

    ($\Rightarrow$)
    Since $\mcal G$ is compact, $\mcal G$ is totally bounded
    under $\rho$ and $\ol \rho$ by \Cref{th:metCptIffComplAndTotBdd}.
    $\mcal G$ is equicontinuous under $d$ by (i) and
    \Cref{lem:totBddThenEquiconti}.
    Since $\mcal F \subseteq \mcal G$, $\mcal F$ is also equicontinuous
    under $d$. \checkmark

    Since $\mcal G$ is compact, $\mcal G$ is bounded under $d$.
    Hence, for all $a \in X$ and $f, g \in \mcal G$,
    $d(f(a), g(a)) \le \rho(f, g) \le \diam G$; $\mcal G$ is pointwise bounded
    under $d$.
    Since $\mcal F \subseteq \mcal G$, $\mcal F$ is also pointwise
    bounded under $d$. \checkmark

    ($\Leftarrow$)
    \centerbox{
    \mclm{Claim 1}{
        $\mcal G$ is equicontinuous.
    }
    \pf{Proof}{
        Take any $x_0 \in X$ and $\veps \in \RR_+$.
        By equicontinuity of $\mcal F$, there exists a neighborhood $U$
        of $x_0$ such that
        \[
            \fall x \in U,\: \fall f \in \mcal F,\: d(f(x), f(x_0)) < \veps/3\text{.}
        \]
        Now, take any $g \in \mcal G$.
        There exists $f \in \mcal F$ such that $\rho(f, g) < \veps/3$
        by \Cref{th:inClosureIffNeighCapANonempty}.
        Hence, for every $x \in U$,
        \[
            d(g(x), g(x_0)) \le d(g(x), f(x)) + d(f(x), f(x_0)) + d(f(x_0), g(x_0))
            < \veps\text{.}
        \]
        Since $g$ was arbitrary, $\mcal G$ is equicontinuous.
    }
    }

    \centerbox{
    \mclm{Claim 2}{
        $\mcal G$ is pointwise bounded under $d$.
    }
    \pf{Proof}{
        Take any $a \in X$. By pointwise boundedness of $\mcal F$,
        $\diam \mcal F_a$ is finite.
        Take any $g, g' \in \mcal G$. There are $f, f' \in \mcal F$
        such that $\rho(f, g) < 1$ and $\rho(g, g') < 1$ by
        \Cref{th:inClosureIffNeighCapANonempty}.
        \[
            d(g(a), g'(a)) \le d(g(a), f(a)) + d(f(a), f'(a)) + d(f'(a), g'(a))
            < \diam \mcal F_a + 2\text{,}
        \]
        which implies $\mcal G$ is pointwise bounded under $d$.
    }
    }

    \centerbox{
    \mclm{Claim 3}{
        There exists a compact subspace $Y$ of $\RR[n]$ such that
        \[
            \bigcup_{g \in \mcal G} g(X) \subseteq Y\text{.}
        \]
    }
    \pf{Proof}{
        For each $a \in X$, there exists a neighborhood $U_a$ of $a$
        such that $\fall x \in U_a,\: \fall g \in \mcal G,\:
        d(g(x), g(a)) < 1$ by \textbf{\textit{Claim 1}}.
        Since $X$ is compact, we may cover $X$ with finitely many $U_a$'s.
        Let us say
        \[
            U_{a_1}, U_{a_2}, \cdots, U_{a_k}
        \]
        cover $X$.
        As $\mcal G_{a_i}$ are bounded by \textbf{\textit{Claim 2}}, their union is also bounded.
        Hence, there exists $N \in \RR_+$ such that $\bigcup_{i=1}^k
        \mcal G_{a_i} \subseteq B_d(\vec 0, N)$.
        Now, take any $g \in \mcal G$ and $x \in X$. Then, $x \in U_{a_i}$
        for some $i \in [k]$. Hence,
        \[
            d(\vec 0, g(x)) \le d(\vec 0, g(a_i)) + d(g(a_i), g(x))
            < N + 1\text{;}
        \]
        $\bigcup_{g \in \mcal G} g(X) \subseteq \cl{B_d(\vec 0, N+1)}$.
    }
    }

    $\mcal G$, being a closed subspace of the complete metric space
    $(\mcal C(X, \RR[n]), \rho)$, is complete. \checkmark

    Moreover, from \textbf{\textit{Claim 3}}, we have a compact subspace
    $Y$ of $\RR[n]$ such that $\mcal G \subseteq \mcal C(X, Y)$.
    Therefore, since $\mcal G$ is equicontinuous by \textbf{\textit{Claim 1}},
    by \Cref{lem:cptEquicontiThenTotBdd}, $\mcal G$ is totally bounded
    under $\rho$. \checkmark

    $\mcal G$, being complete and totally bounded under $\rho$, is
    in turn compact by \Cref{th:metCptIffComplAndTotBdd}.
}

\cor{}{
    \begin{itemize}[nolistsep]
        \ii Let $X$ be a compact space and let $(\RR[n], d)$ be the Euclidean space
            in either the square metric or the Euclidean metric.
        \ii Let $\mcal F$ be a subspace of $(\mcal C(X, \RR[n]), \ol\rho)$.
    \end{itemize}
    Then, $\mcal F$ is compact
    if and only if it is closed, bounded under $\rho$, and equicontinuous
    under $d$.
}
\pf{Proof}{
    ($\Rightarrow$)
    $\mcal F$ is closed (\Cref{th:cptInHausIsCld}) and bounded under $\rho$ (\Cref{th:cptMetIsBdd}).
    Closedness of $\mcal F$ implies $\mcal F = \cl{\mcal F}$,
    and thus $\mcal F$ is equicontinuous by \Cref{th:ascoliClassic}.

    ($\Leftarrow$)
    Boundedness under $\rho$ implies pointwise boundedness under $d$.
    Since $\mcal F$ is equicontinuous and pointwise bounded under $d$,
    $\cl{\mcal F}$ is compact by \Cref{th:ascoliClassic}.
    $\mcal F = \cl{\mcal F}$ as $\mcal F$ is closed.
}

\section{Pointwise and Compact Convergence}

\dfn{Point-Open Topology}{
    Let $X$ and $(Y, \mcal T_Y)$ be topological spaces.
    Given a point $x \in X$ and an open set $U$ of $Y$, let
    \[
        S(x, U) \triangleq \{\,f \colon X \to Y \mid f(x) \in U\,\}\text{.}
    \]
    Then,
    \[
        \mcal S \triangleq \{\,S(x, U) \mid x \in X \text{ and }
        U\text{ is open in }Y\,\}
    \]
    is a subbasis for $Y^X$. The topology generated by the subbasis $\mcal S$
    is called the \textit{point-open topology} (or the \textit{topology of
    pointwise convergence}).
}

\nt{
    \noindent
    The point-open topology on $Y^X$ is just the product topology
    on $\prod_{x \in X} Y$; $S(x, U) = \pi_x\inv(U)$.
}

\thm{}{
    A sequence $\{f_n\}_{n \in \ZZ_+}$ in $Y^X$ converges to $f \in Y^X$
    in the point-open topology if and only if, for each $x \in X$,
    $f_n(x) \to f(x)$.
}
\pf{Proof}{
    This is just a corollary of \Cref{lem:convInProdIff}.
}

\exmp{$\mcal C(X, Y)$ Is Not Closed in the Point-Open Topology}{
    Consider the space $\RR[I]$ where $I = [0, 1]$.
    The sequence $\{f_n\}_{n \in \ZZ_+}$ in $\mcal C(I, \RR)$ defined by
    $f_n(x) = x^n$ converges to $f \in Y^X$ where
    \[
        f(x) = \begin{cases}
            0 &\text{if } 0 \le x < 1 \\
            1 &\text{if } x = 1\text{.}
        \end{cases}
    \]
    However, $f \notin \mcal C(I, \RR)$; $\mcal C(I, \RR)$ is not closed
    in the point-open topology.
}

\nt{
    \noindent
    If $(Y, d)$ is a metric space, the point-open topology on $Y^X$
    is coarser than the uniform topology.
}

\nt{
    The point-open topology is too coarse to guarantee
    $\mcal C(X, Y)$ is closed.
    We want to investigate some topology coarser than the uniform topology,
    which is metrizable, that still has $\mcal C(X, Y)$ closed.
}

\dfn{Topology of Compact Convergence}{
    Let $X$ be a topological space and let $(Y, d)$ be a metric space.
    Given $f \in Y^X$, a nonempty compact subspace $C$ of $X$, and $\veps \in \RR_+$,
    let
    \[
        B_C(f, \veps) \triangleq \left\{\,g \colon X \to Y \:\left|\:
        \sup_{x \in C} d\big(f(x), g(x)\big) < \veps\,\right.\right\}\text{.}
    \]
    Then,
    \[
        \mcal B \triangleq \{\,B_C(f, \veps) \mid
        f \in Y^X\text{, } \veps \in \RR_+\text{, and } C \text{ is a compact
        subspace of }X\,\}
    \]
    is a basis for $Y^X$. The topology generated by $\mcal B$ is called
    the \textit{topology of compact convergence}.
}

\nt{
    $\mcal B$ is actually a basis for $Y^X$.
    If $g \in B_C(f, \veps)$, then
    \[
        \delta \triangleq \veps - \sup_{x \in C} d(f(x), g(x))
    \]
    grants us $B_C(g, \delta) \subseteq B_C(f, \veps)$.
    Hence, if we have $g \in B_{C_1}(f_1, \veps_1) \cap B_{C_2}(f_2, \veps_2)$,
    there exists $\delta \in \RR_+$ such that
    $g \in B_{C_1 \cup C_2}(g, \delta) \subseteq
    B_{C_1}(f_1, \veps_1) \cap B_{C_2}(f_2, \veps_2)$.
}

\nt{
    \noindent
    Let $X$ be a topological space and let $(Y, d)$ be a metric space.
    Here are some direct facts regarding the three topologies
    on $Y^X$.
    \begin{itemize}[nolistsep]
        \ii If $X$ is endowed with the discrete topology,
            the point-open topology coincides with the topology of compact
            convergence.
        \ii If $X$ is a compact space, the uniform topology coincides with
            the topology of compact convergence.
    \end{itemize}
}

\thm{}{
    Let $X$ be a topological space and let $(Y, d)$ be a metric space.
    A sequence $\{f_n\}_{n \in \ZZ_+}$ in $Y^X$ converges to $f$
    in the topology of compact convergence if and only
    if, for each compact subspace $C$ of $X$, $f_n \big|_C \to f\big|_C$
    uniformly.
}
\pf{Proof}{
    ($\Rightarrow$)
    Take any $\veps \in \RR_+$.
    Since $B_C(f, \veps)$ is a neighborhood of $f$, there exists $N \in \ZZ_+$
    such that $\sup_{x \in C} d(f_n(x), f(x)) < \veps$ for all $n \ge N$.
    Hence, $d \big(f_n\big|_C(x), f\big|_C(x)\big) < \veps$ for all
    $n \ge N$ and $x \in C$, which implies $f_n\big|_C$ converges to
    $f\big|_C$ uniformly.

    ($\Leftarrow$)
    Take any basis element $B_C(f, \veps)$ that contains $f$.
    Since $f_n\big|_C$ uniformly converges to $f\big|_C$,
    there exists $N \in \ZZ_+$ such that $d \big(f_n\big|_C(x), f\big|_C(x)\big)
    < \veps/2$ for all $n \ge N$ and $x \in C$.
    Hence, $\sup_{x \in C} d(f_n(x), f(x)) \le \veps/2 < \veps$ for every
    $n \ge N$.
}

\dfn{Compactly Generated Space}{
    A topological space $X$ is said to be \textit{compactly generated}
    if, for every $A \subseteq X$, $A$ is open in $X$
    if $A \cap C$ is open in $C$ for each compact subspace $C$.
}

\nt{
    Equivalently, $X$ is compactly generated if and only if,
    for every $B \subseteq X$, $B$ is closed in $X$
    if $B \cap C$ is closed in $C$ for each compact subspace $C$.
}

\mlemma[locCptOrFirstCtThenCptGen]{}{
    If a topological space $X$ is locally compact or first-countable,
    then $X$ is compactly generated.
}
\pf{Proof}{
    Suppose $X$ is locally compact. Let $A \cap C$ be open in $C$
    for every compact subspace $C$ of $X$.
    Take any $x \in A$. There are a compact subspace $C_0$ and a neighborhood
    $U$ of $x$ such that $U \subseteq C_0$. Since $C_0 \cap A$ is open in
    $C_0$, there exists an open set $V$ of $X$ such that
    $C_0 \cap A = C_0 \cap V$.
    Then, $A \cap C_0 \cap U = A \cap U = U \cap V$ is open in $X$,
    hence so is $A$. \checkmark

    Suppose $X$ is first-countable.
    Let $B \cap C$ be closed in $C$ for every compact subspace $C$ of $X$.
    Take any $x \in \cl B$. Then there exists a sequence
    $\{x_n\}_{n \in \ZZ_+}$ in $B$ that converges to $x$ by \Cref{lem:seqLemma}.
    Let $C_0 \triangleq \{x\} \cup \{\,x_n \mid n \in \ZZ_+\,\}$.
    Then, $x \in (\text{closure of }B \cap C_0\text{ in }C_0)$ by \Cref{lem:seqLemma}.
    Moreover, $C_0$ is compact, which implies $B \cap C_0$ is closed in $C_0$;
    $x \in B \cap C_0 \subseteq B$. Hence, $B$ is closed. \checkmark
}

\mlemma[contiIfRestrConti]{}{
    Let $X$ and $Y$ be topological spaces.
    If $X$ is compactly generated, then a function $f \colon X \to Y$
    is continuous if, for each compact subspace $C$ of $X$,
    $f\big|_C$ is continuous.
}
\pf{Proof}{
    Let $V$ be open in $Y$. Take any compact subspace $C$ of $X$.
    Then,
    \[
        f\inv(V) \cap C = \big(f\big|_C\big)\inv(V)
    \]
    is open. Hence, $f\inv(V)$ is open since $X$ is compactly generated.
}

\thm[]{}{
    Let $X$ be a compactly generated space and let
    $(Y, d)$ be a metric space. Then, $\mcal C(X, Y)$ is closed
    in the topology of compact convergence.
}
\pf{Proof}{
    Let $f \colon X \to Y$ be a limit point of $\mcal C(X, Y)$.
    Take any compact subspace $C$ of $X$.
    Then, for each $n \in \ZZ_+$, by \Cref{th:inClosureIffNeighCapANonempty},
    we may choose $f_n \in \mcal C(X, Y) \cap B_C(f, 1/n)$.
    Then, it is immediate that $f_n\big|_C \to f\big|_C$ uniformly,
    hence $f\big|_C$ is continuous by \Cref{th:unifLimThm}.
    Since $C$ was arbitrary, $f$ is continuous by \Cref{lem:contiIfRestrConti}.
}

\cor{}{
    Let $X$ be a compactly generated space and let
    $(Y, d)$ be a metric space.
    If a sequence $\{f_n\}_{n \in \ZZ_+}$ in $\mcal C(X, Y)$
    converges to $f$ in the topology of compact convergence,
    then $f$ is continuous.
}

\thm{}{
    Let $X$ be a topological space and let $(Y, d)$ be a metric space.
    On $Y^X$, we have the following inclusions:
    \[
        (\text{point-open}) \subseteq (\text{compact convergence}) \subseteq (\text{uniform})\text{.}
    \]
}
\pf{Proof}{
    ($(\text{point-open}) \subseteq (\text{compact convergence})$)
    Let $S(x, U)$ be any subbasis element of the point-open topology and let $f \in S(x, U)$.
    Since $f(x) \in U$, there exists $\veps \in \RR_+$ such that
    $B_d(f(x), \veps) \subseteq U$.
    Then, we have
    \[
        f \in B_{\{x\}}(f, \veps) \subseteq S(x, U)\text{.}
    \]

    ($(\text{compact convergence}) \subseteq (\text{uniform})$)
    Let $B_C(g, \veps)$ be any basis element of the topology of compact convergence
    and let $f \in B_C(g, \veps)$. Then, there exists $\delta \in \RR_+$
    such that $f \in B_C(f, \delta) \subseteq B_C(g, \veps)$.
    Then, we have
    \[
        f \in B_{\ol\rho}(f, \delta) \subseteq B_C(g, \veps)\text{.}
    \]
}

\nt{
    The topology of compact convergence is pretty satisfying---in a sense that
    it is coarser than the uniform topology while having $\mcal C(X, Y)$ closed.
    However, it still require $Y$ to be a metric space, and it depends on
    the specific metric, which is what topologists do not like.
    We will investigate a topology that generalize the topology of
    compact convergence to non-metrizable topologies.
    (Actually, we could not find a satisfying topology on $Y^X$;
    but we have one on $\mcal C(X, Y)$.)
}

\dfn{Compact-Open Topology}{
    Let $X$ and $Y$ be topological spaces. If $C$ is a compact subspace of $X$
    and $U$ is open in $Y$, let
    \[
        S(C, U) \triangleq \{\,f \in \mcal C(X, Y) \mid f(C) \subseteq U\,\}\text{.}
    \]
    Then,
    \[
        \mcal S \triangleq \{\,S(C, U) \mid C \text{ is a compact subspace of }
        X\text{ and }U\text{ is open in }Y\,\}
    \]
    is a subbasis for $\mcal C(X, Y)$. The topology generated by the
    subbasis $\mcal S$ is called the \textit{compact-open topology}.
}

\thm[cptOpenIsCptConvWhenMetric]{}{
    Let $X$ be a topological space and let $(Y, d)$ be a metric space.
    Then, on the set $\mcal C(X, Y)$, the compact-open topology and the
    topology of compact convergence coincide.
}
\pf{Proof}{
    ($\text{(compact-open)} \subseteq \text{(compact convergence)}$)
    Let $S(C, U)$ be a subbasis element for the compact-open topology
    and let $f \in S(C, U)$. Since $f$ is continuous, $f(C)$ is compact
    subspace of $U$. Therefore, a (uniformly) continuous function $f(C) \to \RR$
    defined by $x \mapsto d(x, X \setminus U)$ has a minimum value
    $\veps \in \RR_{\ge 0}$ by \Cref{th:evt}. (See \Cref{th:distFtnIsConti}.)
    $\veps \neq 0$ since every $x \in f(C)$ is mapped to a positive number.
    Then, we have
    \[
        B_C(f, \veps) \cap \mcal C(X, Y) \subseteq S(C, U)\text{.}
    \]
    Hence, the topology of compact convergence is finer by \Cref{lem:finerIff}.
    \checkmark

    ($\text{(compact convergence)} \subseteq \text{(compact-open)}$)
    Let $B$ be any basis element of the topology of compact convergence
    and let $f \in B$. Then, $B$ contains a basis element of the form
    $B_C(f, \veps) \cap \mcal C(X, Y)$.
    For each $x \in C$, let $V_x \triangleq f\inv\big(B_d(f(x), \veps/4)\big)$
    so that $f(V_x) \subseteq B_d(f(x), \veps/4)$.
    Then, $f(\cl{V_x}) \subseteq \cl{f(V_x)} \subseteq B_d(f(x), \veps/3)$
    by \Cref{th:contiTFAE}.
    Then, for some finitely many values of $x$, say $x_1, x_2, \cdots, x_n$,
    $V_{x_i}$'s cover $C$. Let $C_i \triangleq \cl{V_{x_i}} \cap C$
    for each $i \in [n]$.
    Then, since $C_{x_i}$ is closed in $C$ by \Cref{th:closedIffYCapB},
    it is compact by \Cref{th:cldInCptIsCpt}.

    Now, consider a basis element
    \[
        B_0 \triangleq \bigcap_{i=1}^n S\big(C_i, B_d(f(x_i), \veps/3)\big)
    \]
    of the compact-open topology.

    Take any $i \in [n]$ and $x \in C_i$. Then,
    \[
        f(x) \in B_d(f(x_i), \veps/3)
    \]
    by the definition of $C_i$. Hence, as $x$ was arbitrary,
    $f \in S\big(C_i, B_d(f(x_i), \veps/3)\big)$.
    Since $i$ was arbitrary as well, $f \in B_0$.

    Take any $g \in B_0$ and $x \in C$.
    Then, $x \in C_i$ for some $i \in [n]$.
    As $f \in B_0$ as well, we have
    \[
        d(f(x), g(x)) \le d(f(x), f(x_i)) + d(f(x_i), g(x)) < 2\veps/3\text{.}
    \]
    As $x$ was arbitrary, $\sup_{x \in C} d(f(x), g(x)) \le 2\veps/3 < \veps$;
    which implies $g \in B_C(f, \veps) \cap \mcal C(X, Y)$.
    
    Hence, the compact-open topology is finer by \Cref{lem:finerIff}.
    \checkmark
}

\cor{}{
Let $X$ be a topological space and let $Y$ be a metric space.
    \begin{enumerate}[nolistsep, label=(\roman*)]
        \ii The topology of compact convergence on $\mcal C(X, Y)$
            does not depend on the metric of $Y$.
        \ii If $X$ is compact, the uniform topology on $\mcal C(X, Y)$
            does not depend on the metric of $Y$.
    \end{enumerate}
}
\pf{Proof}{
    \hfill
    \begin{enumerate}[nolistsep, label=(\roman*)]
        \ii The compact-open topology, which equals the topology of compact
            convergence by \Cref{th:cptOpenIsCptConvWhenMetric},
            does not depend on the metric.
        \ii The uniform topology coincides the topology of compact convergence,
            which does not depend on the metric by (i).
    \end{enumerate}
}

\thm[evalIsConti]{}{
    Let $X$ and $Y$ be topological spaces.
    If $X$ is locally compact Hausdorff and $\mcal C(X, Y)$ is endowed with
    compact-open topology, then the map
    \[
        e \colon X \times \mcal C(X, Y) \to Y
    \]
    defined by
    \[
        x \times f \mapsto f(x)
    \]
    is continuous.
    The map $e$ is called the \textit{evaluation map}.
}
\pf{Proof}{
    Let $V$ be open in $Y$ and let $e(x, f) \in V$, i.e., $f(x) \in V$.
    By \Cref{th:HausLocalCptIff}, there exists a neighborhood $U$ of $x$
    in $X$ such that $\cl U$ is compact and $\cl U \subseteq f\inv(V)$.

    Now, take any $x' \times f' \in U \times S(\cl U, V)$.
    $U \times S(\cl U, V)$ is an open set in $X \times \mcal C(X, Y)$.
    Then, $e(x', f') = f'(x') \in V$. Therefore,
    \[
        x \times f \in U \times S(\cl U, V) \subseteq e\inv(V)\text{,}
    \]
    which implies $e$ is continuous.
}

\dfn{Natural Bijection Between $Y^{X \times Z}$ and $(Y^X)^Z$}{
    Given a function $f \colon X \times Z \to Y$, there is a corresponding
    function $F \colon Z \to Y^X$, defined by the equation
    \[
        (F(z))(x) = f(x, z)\text{.}
    \]
    Conversely, given $F \colon Z \to Y^X$, this equation defines
    a corresponding function $f \colon Z \times Z \to Y$.
    We say that $F$ is the map of $Z$ into $Y^X$ that is
    \textit{induced} by $f$.
}

\nt{
    \mclm{Claim}{
        If $f \colon X \times Z \to Y$ is continuous, then the induced function
        $F \colon Z \to Y^X$ has $\Img F \subseteq \mcal C(X, Y)$.
    }
    \pf{Proof}{
        Let $g = F(z) \in Y^X$. Then, $g\inv(V) \times \{z\} = f\inv(V) \cap (X \times \{z\})$
        is open in $X \times \{z\}$, while $X \cong X \times \{z\}$.
        Hence, $g$ is continuous.
    }
}

\thm{}{
    Let $X$, $Y$, and $Z$ be topological spaces.
    Give $\mcal C(X, Y)$ the compact-open topology.
    If $f \colon X \times Z \to Y$ is continuous, then the
    induced function $F \colon Z \to \mcal C(X, Y)$ is continuous as well.
    The converse holds if $X$ is locally compact Hausdorff.
}
\pf{Proof}{
    ($\Rightarrow$)
    Let $S(C, U)$ be any subbasis element of $\mcal C(X, Y)$ and let
    $z_0 \in F\inv(S(C, U))$.
    $F(z_0) \in S(C, U)$ simply means that,
    $f(C \times \{z_0\}) = (F(z_0))(C) \subseteq U$.

    From the continuity of $f$, $f\inv(U)$ is open in $X \times Z$
    and \[
        C \times \{z_0\} \subseteq f\inv(U) \cap (C \times Z)\text{.}
    \]
    $f\inv(U) \cap (C \times Z)$ is an open set in $C \times Z$.
    Therefore, by \Cref{lem:tube}, there exists a neighborhood $W$ of $z_0$
    in $Z$ such that 
    \[
        C \times W \subseteq f\inv(U) \cap (C \times Z) \subseteq f\inv(U)\text{.}
    \]
    In other words, for each $z \in W$,
    we have $\fall x \in C,\: (F(z))(x) = f(x, z) \in U$;
    \[
        F(z) \in S(C, U)\text{.}
    \]
    Since $z$ was arbitrary, $F(W) \subseteq S(C, U)$, i.e,
    $z_0 \in W \subseteq F\inv(S(C, U))$. Hence, $F$ is continuous.

    ($\Leftarrow$)
    By \Cref{th:evalIsConti}, the evaluation map $e \colon X \times \mcal C(X, Y) \to Y$
    is continuous.
    Since $f = e \circ (i_X \times F)$ where $i_X$ is the identity map,
    $f$ is continuous by \Cref{th:constructConti}.
    \[
        X \times Z
        \xrightarrow{\makebox[1.5cm]{\footnotesize$i_X \times F$}} X \times \mcal C(X, Y)
        \xrightarrow{\makebox[1.5cm]{\footnotesize$e$}} Y
    \]
}

\end{document}
