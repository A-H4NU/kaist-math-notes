\documentclass[MAS331_Note.tex]{subfiles}

\begin{document}

\section{Complete Metric Spaces}

\dfn{Cauchy Sequence}{
    Let $(X, d)$ be a metric space. A sequence $\{x_n\}_{n \in \ZZ_+}$
    of points of $X$ is said to be \textit{Cauchy} in $(X, d)$ if
    \[
        \fall \veps \in \RR_+,\: \exs N \in \ZZ_+,\:
        \fall n, m \in \ZZ_+,\: \big(n,m\ge N \implies d(x_n, x_m) < \veps\big)\text{.}
    \]
}

\dfn{Complete Metric Space}{
    Let $(X, d)$ be a metric space. $(X, d)$ is said to be \textit{complete}
    if every Cauchy sequence in converges.
}

\nt{
    \noindent Here are some immediate facts.
    \begin{itemize}[nolistsep]
        \ii Every convergent sequence in $(X, d)$ is a Cauchy sequence.
        \ii $\{x_n\}_{n \in \ZZ_+}$ is Cauchy in $(X, d)$ if and only if
            it is Cauchy in $(X, \ol d)$.
        \ii $(X, d)$ is complete if and only if $(X, \ol d)$ is Cauchy.
    \end{itemize}
}

\mlemma[complIfHasConvSubseq]{}{
    A metric space $(X, d)$ is complete if every Cauchy sequence in $X$ has a
    convergent subsequence.
}
\pf{Proof}{
    Let $\{x_n\}_{n \in \ZZ_+}$ be a Cauchy sequence in $(X, d)$. Let
    $\{x_{n_i}\}_{i \in \ZZ_+}$ be a subsequence that converges to $x \in X$.

    Given $\veps \in \RR_+$, there exists $N \in \ZZ_+$ such that
    $d(x_n, x_m) < \veps/2$ for all $n, m \ge N$. Then choose $i \in \ZZ_+$
    such that $n_i \ge N$ and $d(x_{n_i}, x) < \veps/2$. Then,
    for every $n \in \ZZ_+$ not smaller than $N$,
    \[
        d(x_n, x) \le d(x_n, d_{n_i}) + d(x_{n_i}, x) < \veps\text{.}
    \]
}

\thm[RkisCompl]{}{
    Euclidean space $\RR[k]$ is complete in either of its usual metrics,
    the euclidean metric $d$ of the square metric $\rho$.
}
\pf{Proof}{
    To show the metric space $(\RR[k], \rho)$ is complete, let $\{\vec x_n\}_{n \in \ZZ_+}$
    be a Cauchy
    sequence in $(\RR[k], \rho)$. There exists $N \in \ZZ_+$ such that
    \[
        \rho(\vec x_n, \vec x_m) \le 1
    \]
    for all $n, m \ge N$. Then
    \[
        M \triangleq \max \{\,\rho(\vec x_1, \vec 0), \cdots, \rho(\vec x_N, \vec 0),
        \rho(\vec x_N, \vec 0) + 1\,\}
    \]
    is an upper bound of $\{\rho(\vec x_n, \vec 0)\}_{n \in \ZZ_+}$.

    Thus, $\{x_n\}_{n \in \ZZ_+}$ is a sequence in $[-M, M]^k$ (product),
    which is compact by \Cref{th:cldIntvIsCpt} and \Cref{th:prodCptIffEachCpt}.
    ($[-M, M]^k$ as a subspace of $(\RR[k], \rho)$ is also a metric space
    and is compact by \Cref{th:subOfProdIsProdOfSub}.)
    As $[-M, M]^k$ is sequentially compact by \Cref{th:cptTFAE}, $\{x_n\}_{n \in \ZZ_+}$
    has a convergent subsequence in $[-M, M]^k$ (product). The subsequence converges
    in $(\RR[k], \rho)$ as well. By \Cref{lem:complIfHasConvSubseq}, $(\RR[k], \rho)$
    is complete.

    For $(\RR[k], d)$, note the following.
    \begin{itemize}[nolistsep]
        \ii $\{\vec x_n\}_{n \in \ZZ_+}$ is Cauchy in $(\RR[k], d)$ if and
            only if it is Cauchy in $(\RR[k], \rho)$.
        \ii $\{\vec x_n\}_{n \in \ZZ_+}$ converges in $(\RR[k], d)$ if and
            only if it converges in $(\RR[k], \rho)$.
    \end{itemize}
}

\mlemma[convInProdIff]{}{
    Let $X$ be the product space $X \triangleq \prod_{\alpha \in J} X_\alpha$;
    let $\{\vec x_n\}$ be a sequence of points of $X$. Then
    $\vec x_n \to \vec x$ if and only if $\pi_\alpha(\vec x_n) \to \pi_\alpha(\vec x)$
    for each $\alpha \in J$.
}
\pf{Proof}{
    ($\Rightarrow$) For each $\alpha \in J$, $\pi_\alpha$ is continuous.
    Hence, by \Cref{lem:seqConvIffImgConv}, $\pi_\alpha(\vec x_n) \to
    \pi_\alpha(\vec x)$.

    ($\Leftarrow$) Let $\bigcap_{i=1}^k\pi_{\alpha_i}\inv(U_{\alpha_i})$ be a
    basis element that contains $\vec x$.
    For each $i \in [k]$, there exists $N_i \in \ZZ_+$ such that
    $\pi_{\alpha_i}(\vec x_n) \in U_{\alpha_i}$ for all $n \ge N_i$.
    Then, for every $n \ge \max_{i=1}^k N_i$, $\vec x \in
    \bigcap_{i=1}^k\pi_{\alpha_i}\inv(U_{\alpha_i})$.
}

\thm{}{
    Suppose $(X_i, d_i)$ is a complete metric space for each $i \in \ZZ_+$.
    Let $X \triangleq \prod_{i \in \ZZ_+} X_i$ be a product space.
    Then, let $D \colon X \times X \to \RR$ be defined by
    \[
        D(\vec x, \vec y) \triangleq \sup \left\{\,\frac{\ol d_i(x_i, y_i)}{i} \:\bigg|\: i \in \ZZ_+\,\right\}\text{.}
    \]
    Then, the metric space $(X, D)$ is complete.
}
\pf{Proof}{
    $D$ induces the product topology by \Cref{th:ctMetIsMet}.
    Let $\{\vec x_n\}_{n \in \ZZ_+}$ be a Cauchy sequence in $(X, D)$.
    Since
    \[
        \ol d_i(\pi_i(\vec x), \pi_i(\vec y)) \le i \cdot D(\vec x, \vec y)
    \]
    for each $i \in \ZZ_+$ and $x, y \in X$,
    $\{\pi_i(\vec x_n)\}_{n \in \ZZ_+}$ is a Cauchy sequence in $(X_i, d_i)$;
    it converges. Hence, $\{\vec x_n\}$ converges in $X$ by \Cref{lem:convInProdIff}.
}

\thm[complThenFtnSpaceCompl]{}{
    If $(Y, d)$ is a complete metric space, then the metric space
    $(Y^J, \ol \rho)$ is complete. ($\ol\rho$ is the uniform metric.)
}
\pf{Proof}{
    Suppose $\{f_n\}_{n \in \ZZ_+}$ is Cauchy
    in $\ol \rho$. Take any $\alpha \in J$. Since
    \[
        \ol d(f_n(\alpha), f_m(\alpha)) \le \ol \rho(f_n, f_m)
    \]
    for each $n, m \in \ZZ_+$, the sequence $\{f_n(\alpha)\}_{n \in \ZZ_+}$
    is Cauchy in $(Y, \ol d)$, and thus converges.
    Let $y_\alpha$ be the point to which
    it converges. Let $f \colon J \to Y$ be defined by
    \[
        \alpha \mapsto y_\alpha\text{.}
    \]
    
    We now claim that $f_n \to f$. Let $\veps \in \RR_+$ be given.
    There exists $N \in \ZZ_+$ such that $\ol\rho(f_n, f_m) < \veps/2$
    whenever $n, m \ge N$.

    Take any $\alpha \in J$ and $\veps' \in \RR_+$.
    There exists $M \in \ZZ_+$ such that
    $\ol d(f_m(\alpha), f(\alpha)) < \veps$ for each $m \ge M$.
    Then, for $n \ge N$ and $m \ge \max \{N, M\}$,
    \[
        \ol d(f_n(\alpha), f(\alpha)) \le \ol d(f_n(\alpha), f_m(\alpha))
        + \ol d(f_m(\alpha), f(\alpha)) < \veps/2 + \veps'\text{.}
    \]
    Since $\veps'$ was arbitrary, $\ol d(f_n(\alpha), f(\alpha)) \le \veps/2$
    for each $n \ge N$. Since $\alpha$ was arbitrary, $\ol \rho(f_n, f) \le \veps/2 < \veps$
    for all $n \ge N$.
}

\dfn{Space of Countinuous/Bounded Function}{
    Let $X$ be a topological space and let $(Y, d)$ be a metric space.
    Then, define $\mcal C(X, Y), \mcal B(X, Y) \subseteq Y^X$ by
    \[
        \mcal C(X, Y) \triangleq \{\,f \in Y^X \mid f \text{ is continuous}\,\}
    \]
    and
    \[
        \mcal B(X, Y) \triangleq \{\,f \in Y^X \mid f \text{ is bounded
        with respect to }d\,\}\text{.}
    \]
}

\thm[CandBCld]{}{
    Let $X$ be a topological space and let $(Y, d)$ be a metric space.
    The set $\mcal C(X, Y)$ and $\mcal B(X, Y)$ are closed in $(Y^X, \ol\rho)$.
}
\pf{Proof}{
    ($\mcal C(X, Y)$ is closed.)
    Let $f \in Y^X$ be a limit point of $\mcal C(X, Y)$. Then, there exists
    a sequence $\{f_n\}_{n \in \ZZ_+}$ in $\mcal C(X, Y)$ converging to $f$
    in the metric $\ol\rho$ by \Cref{lem:seqLemma}.

    Now, we claim that $\{f_n\}_{n \in \ZZ_+}$ converges to $f$ uniformly.
    Take any $\veps \in \RR_+$. Then, there exists some $N \in \ZZ_+$ such that
    $\ol\rho(f_n, f) < \veps$ for each $n \ge N$.
    Then, for every $x \in X$ and $n \ge N$,
    \[
        \ol d(f_n(x), f(x)) \le \ol\rho(f_n, f) < \veps\text{,}
    \]
    which implies $\{f_n\}_{n \in \ZZ_+}$ uniformly converges to $f$.
    Then, by \Cref{th:unifLimThm}, $f \in \mcal C(X, Y)$.
    Hence, $\mcal C(X, Y)$ is closed by \Cref{cor:closedIffContainsLimPts}.

    ($\mcal B(X, Y)$ is closed.)
    Let $f \in Y^X$ be a limit point of $\mcal B(X, Y)$.
    Then, there exists
    a sequence $\{f_n\}_{n \in \ZZ_+}$ in $\mcal B(X, Y)$ converging to $f$
    in the metric $\ol\rho$ by \Cref{lem:seqLemma}.

    There exists $N \in \ZZ_+$ such that $\ol\rho(f_N, f) < 1$.
    Let $x, y \in X$. Then,
    \[
        d(f(x), f(y)) \le d(f(x), f_N(x)) + d(f_N(x), f_N(y))
        + d(f_N(y), f(y)) < \diam \Img f_N + 2\text{.}
    \]
    Hence, $f \in \mcal B(X, Y)$. Therefore, $\mcal B(X, Y)$ is closed by
    \Cref{cor:closedIffContainsLimPts}.
}

\cor[CandBCompl]{}{
    Let $X$ be a topological space and let $(Y, d)$ be a complete metric space.
    The set $\mcal C(X, Y)$ and $\mcal B(X, Y)$ are complete in $(Y^X, \ol\rho)$.
}
\pf{Proof}{
    $(Y^X, \ol\rho)$ is complete by \Cref{th:complThenFtnSpaceCompl}.
    Any Cauchy sequence in $\mcal C(X, Y)$ is Cauchy in $Y^X$, and thus
    converges, say, to $f$. Then, since $\mcal C(X, Y)$ is closed by
    \Cref{th:CandBCld},
    $f \in \mcal C(X, Y)$. Hence, $\mcal C(X, Y)$ is complete.
    $\mcal B(X, Y)$ is, similarly, complete.
}

\mlemma{}{
    Let $X$ be a compact space and let $(Y, d)$ be a metric space.
    Then, $\mcal C(X, Y) \subseteq \mcal B(X, Y)$, i.e., every continuous
    function from $X$ to $Y$ is bounded.
}
\pf{Proof}{
    Let $f \in \mcal C(X, Y)$. Then, $\Img f$ is compact by
    \Cref{th:contiSendsCptToCpt}. Thus, it is bounded by \Cref{th:cptMetIsBdd}.
}

\dfn{Sup Metric}{
    Let $(Y, d)$ be a metric space. We may define another metric $\rho$ on
    the set $\mcal B(X, Y)$ by the equation
    \[
        \rho(f, g) \triangleq \sup \{\,d(f(x), g(x)) \mid x \in X\,\}\text{.}
    \]
    The metric $\rho$ is called the \textit{sup metric}.
}

\nt{
    Let $\rho$ and $\ol \rho$ be the sup metric and the uniform metric,
    respectively, on $\mcal B(X, Y)$.
    Then, the following holds.
    \[
        \ol \rho(f, g) = \min \{\rho(f, g), 1\}
    \]
    This means that $\ol \rho$ is just the standard bonded metric
    derived from $\rho$.
}

\nt{
    Let $X$ be a topological space and let $(Y, d)$ be a complete metric space.
    $\mcal B(X, Y)$ is complete in $(Y^X, \rho)$.
    If $X$ is compact, $\mcal C(X, Y)$ is complete in $(Y^X, \rho)$.
}

\dfn{Isometric Imbedding}{
    Let $(X, d_X)$ and $(Y, d_Y)$ be metric spaces.
    If $f \colon X \to Y$ has the property such that
    \[
        \fall x_1, x_2 \in X\,\:
        d_Y(f(x_1), f(x_2)) = d_X(x_1, x_2)\text{,}
    \]
    $f$ is called an \textit{isometric imbedding} of $X$ in $Y$.
}

\nt{
    Let $(X, d_X)$ and $(Y, d_Y)$ be metric spaces.
    If $f \colon X \to Y$ is an isometric imbedding of $X$ in $Y$,
    then it is an imbedding of $X$ in $Y$.
    \pf{Proof}{
        If $f(x_1) = f(x_2)$, then $d_X(x_1, x_2) = d_Y(f(x_1), f(x_2)) = 0$,
        and thus $x_1 = x_2$. This shows that $f$ is an injection.
        Moreover, for each $x \in X$ and $\veps \in \RR_+$,
        \begin{itemize}[nolistsep]
            \ii $f(B_{d_X}(x, \veps)) = B_{d_Y}(f(x), \veps) \cap \Img f$ and
            \ii $f\inv(B_{d_Y}(f(x), \veps)) = B_{d_X}(x, \veps)$.
        \end{itemize}
        Hence, $f$ is an imbedding of $X$ in $Y$.
    }
}

\thm[isoImbed]{}{
    Let $(X, d)$ be a metric space. Then, there is an isometric imbedding of
    $X$ in a complete metric space.
}
\pf{Proof}{
    Consider $\mcal B(X, \RR)$. Let $x_0 \in X$ be fixed.
    For each $a \in X$, define $\phi_a \colon X \to \RR$ by the equation
    \[
        \phi_a(x) \triangleq d(x, a) - d(x, x_0)\text{.}
    \]
    From the triangle inequality, we get
    \[
        |\phi_a(x)| = |d(x, a) - d(x, x_0)|
        \le d(a, x_0)
    \]
    for each $x \in X$. Hence, $\phi_a \in \mcal B(X, \RR)$.
    Define $\Phi \colon X \to \mcal B(X, \RR)$ by letting $a \mapsto \phi_a$.

    We now claim that $\Phi$ is an isometric imbedding of $X$ in the
    complete metric space $(\mcal B(X, \RR), \rho)$.
    By definition, for each $a, b \in X$,
    \[\begin{aligned}[t]
        \rho(\phi_a, \phi_b)
        &= \sup \big\{\, |\phi_a(x) - \phi_b(x)| \:\big|\: x \in X\,\big\} \\
        &= \sup \big\{\, |d(x, a) - d(x, b)| \:\big|\: x \in X\,\big\}
        \le d(a, b)\text{.}
    \end{aligned}\]
    Moreover,
    \[
        d(a, b) = |d(a, a) - d(a, b)|
        \le \sup \big\{\, |d(x, a) - d(x, b)| \:\big|\: x \in X\,\big\}
        = \rho(\phi_a, \phi_b)\text{.}
    \]
    Hence, $d(a, b) = \rho(\phi_a, \phi_b)$; $\Phi$ is an isometric imbedding.
}

\dfn{Completion}{
    Let $X$ be a metric space. If $h \colon X \to Y$ is an isometric imbedding
    of $X$ into a complete metric space $Y$, then the space
    $\cl{h(X)}$ of $Y$ is a complete metric space. It is called the
    \textit{completion} of $X$.
}

\nt{
    \noindent The completion of $X$ is uniquely determined up to an isometry.
}

\end{document}
