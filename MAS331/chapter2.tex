\documentclass[MAS331_Note.tex]{subfiles}

\begin{document}
\chapter{Topological Spaces and Continuous Functions}
\section{Topological Spaces}
\dfn[defTop]{Topology and Topological Space}{
	A \textit{topology} on a set $X$ is a collection $\mcal{T}$ of subsets of $X$ such that
	\begin{enumerate}[label=(\roman*)]
		\ii $\varnothing, X \in \mcal{T}$
		\ii $\{\, U_i \mid i \in J \,\} \subseteq \mcal T \implies \bigcup_{i \in J} U_i \in \mcal T$
		\ii $\{\, U_1, U_2, \cdots, U_n \,\} \subseteq \mcal T \implies \bigcap_{i=1}^n U_i \in \mcal T$
	\end{enumerate}
	We say $(X, \mcal T)$ is a \textit{topological space}, and each element $U \in \mcal T$ is called an \textit{open set}.
}
\exmp{Discrete Topology and Trivial Topology}{
	\begin{itemize}[nolistsep]
		\ii If $X$ is any set, the collection of all subsets of $X$, $\mcal P(X)$, is a topology on $X$;
            it is called the \textit{discrete topology}.
		\ii $\{\,\varnothing, X\,\}$ is also an topology on $X$; we shall call it the \textit{trivial topology}.
	\end{itemize}
}
\exmp{Finite Complement Topology}{
	Let $X$ be any set.
	Then, $\mcal T \coloneqq \{\, U \subseteq X \mid X \setminus U \text{ is finite}\,\} \cup \{\varnothing\}$ is a topology.
	\begin{enumerate}[label=(\roman*)]
		\ii $\varnothing, X \in \mcal{T}$ \checkmark
		\ii If $\{U_\alpha\}_{\alpha \in J} \subseteq \mcal T$,
            then $X \setminus \bigcup_{\alpha \in J} U_\alpha = \bigcap_{\alpha \in J} (X - U_\alpha)$ is finite. \checkmark
		\ii If $\{\,U_1, U_2, \cdots, U_n\,\} \subseteq \mcal T$,
            $X \setminus \bigcap_{i=1}^n U_\alpha = \bigcup_{i=1}^n (X \setminus U_\alpha)$ is finite by \Cref{exer:finUandCOfFinSetIsFin}. \checkmark
	\end{enumerate}
	The topology is called the \textit{finite complement topology}.
}
\exmp{}{
	If $X = \{\,a,b,c\,\}$, then $\mcal T = \{\,\varnothing, X, \{a\}, \{a, b\}\,\}$ is a topology on $X$.
}

\dfn{Finer and Coarser Topology}{
	Let $\mcal T$ and $\mcal T'$ be topologies of a set $X$. If $\mcal T \subseteq \mcal T'$, then we say
	\begin{itemize}
		\ii $\mcal T'$ is \textit{finer} than $\mcal T$ and
		\ii $\mcal T$ is \textit{coarser} than $\mcal T'$.
	\end{itemize}
	Also, $\mcal T$ is \textit{comparable} to $\mcal T'$ if either $\mcal T \supseteq \mcal T'$ or $\mcal T \subseteq \mcal T'$.
}

\section{Basis for a Topology}
\dfn[defBasis]{Basis and Topology Generated by a Basis}{
	A \textit{basis} for $X$ is a collection $\mcal B$ of subsets of $X$ such that:
	\begin{enumerate}[label=(\roman*)]
		\ii $\forall x \in X,\: \exs B \in \mcal B,\: x \in B$ (i.e., $X = \bigcup \mcal B$) and
		\ii $\forall B_1, B_2 \in \mcal B,\: \big(x \in B_1 \cap B_2 \implies \exs B_3 \in \mcal B,\: x \in B_3 \subseteq B_1 \cap B_2 \big)$.
	\end{enumerate}
	The topology $\mcal T$ generated by $\mcal B$ is the collection defined by \[
		\mcal T \coloneqq \{\, U \subseteq X \mid \forall x \in U,\: \exs B \in \mcal B,\: x \in B \subseteq U \,\}\text{.}
	\]
}
\nt{If $\mcal B$ is a basis for $X$ and $\mcal T$ is the topology generated by $\mcal B$, then $\mcal B \subseteq \mcal T$.}

\mlemma[topByBIsTop]{}{
	If $\mcal T$ is the topology generated by basis $\mcal B$ for $X$, then $\mcal T$ is a topology on $X$.
}
\pf{Proof}{
    \hfill
	\begin{enumerate}[label=(\roman*)]
		\ii $\varnothing \in \mcal T$ by vacuous truth, and $X \in \mcal T$ follows directly from (i) in \Cref{def:defBasis}. \checkmark
		\ii Let $\mcal U \coloneqq \{U_\alpha\}_{\alpha \in J} \subseteq \mcal T$.
            Then, $x \in \mcal \bigcup\, \mcal U$ implies $\exs \alpha \in J,\: x \in U_\alpha$.
            Since $U_\alpha \in \mcal T$, there is $B \in \mcal B$ such that $x \in B \subseteq U_\alpha \subseteq \bigcup \mcal U$.
            This means $\bigcup \mcal U \subseteq \mcal T$. \checkmark
		\ii It is enough to prove it for two sets $U_1$ and $U_2$ in $\mcal T$.
            Let $x \in U_1 \cap U_2$. (If $U_1 \cap U_2 = \varnothing$, then it is done.)
            By the definition of $\mcal T$, there are $B_1$ and $B_2$ in $\mcal B$ such that
		$x \in B_1 \subseteq U_1$ and $x \in B_2 \subseteq U_2$.
            Since $x \in B_1 \cap B_2$, there is $B_3 \in \mcal B$ such that $x \in B_3 \subseteq B_1 \cap B_2 \subseteq U_1 \cap U_2$.
            Thus, it implies $U_1 \cap U_2 \in \mcal T$. \checkmark
	\end{enumerate}
}

\mlemma[topByBIsUnionsOfB]{}{
	If $\mcal T$ is the topology generated by basis $\mcal B$ for $X$,
	then $\mcal T$ is the collection of all unions of elements of $\mcal B$.
	In other words, $\mcal T = \left\{\, \bigcup \mcal U \:\big|\: \mcal U \subseteq \mcal B \,\right\}$.
}
\pf{Proof}{
	Let $\mcal T' \coloneqq \left\{\, \bigcup \mcal U \:\big|\: \mcal U \subseteq \mcal B \,\right\}$.
	Since $\mcal B \subseteq \mcal T$ and $\mcal T$ is a topology by \Cref{lem:topByBIsTop},
	$\mcal T' \subseteq \mcal T$ follows. (See (ii) in \Cref{def:defTop}.)
	Now, we shall prove $\mcal T \subseteq \mcal T'$.

	Take any $U \in \mcal T$. Then, for each $x \in U$, there is $B_x \in \mcal B$ such that $x \in B_x \subseteq U$.
	Then, $U = \bigcup_{x \in U} B_x \in \mcal T'$, hence $\mcal T \subseteq \mcal T'$.
}

\mlemma[topToBasis]{}{
	Let $(X, \mcal T)$ be a topological space.
	If $\mcal C$ is a subset of $\mcal T$ such that \[
		\forall U \in \mcal T,\: (x \in U \implies \exs C \in \mcal C,\: x \in C \subseteq U)\text{,}
	\] then $\mcal C$ is a basis for $X$ and $\mcal T$ is the topology generated by $\mcal C$.
}
\pf{Proof}{
	We shall prove first $\mcal C$ is a basis for $X$.
	\begin{enumerate}[nolistsep, label=(\roman*)]
		\ii Since $X \in \mcal T$, $\forall x \in X,\: \exs C \in \mcal C,\: x \in C$. \checkmark
		\ii Let $C_1, C_2 \in \mcal C$ and suppose $x \in C_1 \cap C_2$.
		Since $C_1 \cap C_2 \in \mcal T$, there is $C_3 \in \mcal C$ such that $x \in C_3 \subseteq C_1 \cap C_2$. \checkmark
	\end{enumerate}
	Now let $\mcal T'$ be the topology generated by $\mcal C$. We want to show $\mcal T = \mcal T'$.

	For $\mcal T' \subseteq \mcal T$, take any $U \in \mcal T'$.
	Then, by \Cref{lem:topByBIsUnionsOfB}, $U = \bigcup_{\alpha \in J} C_\alpha$ where each $C_\alpha$ is in $\mcal C$.
	Now, $U = \bigcup_{\alpha \in J} C_\alpha \in \mcal T$ directly follows.
	The last inclusion is due to (ii) in \Cref{def:defTop} and $\mcal C \subseteq \mcal T$. \checkmark

	For $\mcal T \subseteq \mcal T'$, take any $U \in \mcal T$.
	Then, for any $x \in U$, there is $C \in \mcal C$ such that $x \in C \subseteq U$, therefore $U \in \mcal T'$
	by \Cref{def:defBasis}.
}

\mlemma[finerIff]{}{
	Let $\mcal T$ and $\mcal T'$ are topologies genereated by bases $\mcal B$ and $\mcal B'$, respectively. Then, \[
		\mcal T \subseteq \mcal T' \iff \forall B \in \mcal B,\: \big(x \in B \implies \exs B' \in \mcal B',\: x \in B' \subseteq B\big)\text{.}
	\]
}
\pf{Proof}{
	($\Leftarrow$) Take any $U \in \mcal T$ and $x \in U$. Since $\mcal B$ generates $\mcal T$,
	there is $B \in \mcal B$ such that $x \in B \subseteq U$.
	By the supposition, there is $B' \in \mcal B'$ such that $x \in B' \subseteq B \subseteq U$.
	This implies we can find $B' \in \mcal B'$ such that $x \in B' \subseteq U$, by definition, $U \in \mcal T'$. \checkmark

	($\Rightarrow$) Take any $B \in \mcal B$ and $x \in B$. Since $B \in \mcal T \subseteq \mcal T'$,
	by definition of $\mcal T'$, there is $B' \in \mcal B'$ such that $x \in B' \subseteq B$. \checkmark
}

\exmp{}{
	Let $\mcal B$ be a set of open region inside a disk, and $\mcal B'$ be a set of open region inside a rectangle.
	They are bases for $\RR[2]$, and topologies generated by them are the same by \Cref{lem:finerIff}.
}

\dfn{Common Topologies on $\RR$}{
	Define
	\begin{itemize}[nolistsep, label=$-$]
		\ii $\mcal B_{\RR} \coloneqq \{\, (a, b) \subseteq \RR \mid a < b\,\}$
		\ii $\mcal B_\ell \coloneqq \{\, [a, b) \subseteq \RR \mid a < b\,\}$
	\end{itemize}
	$\mcal B$ and $\mcal B'$ are bases for $\RR$. Then,
	\begin{itemize}[nolistsep]
		\ii $\mcal T_{\RR}$, the topology generated by $\mcal B$, is called the \textit{standard topology} on $\RR$, and
		\ii $\mcal T_\ell$, the topology generated by $\mcal B_\ell$, is called the \textit{lower limit topology} on $\RR$.
	\end{itemize}
	Let $K \coloneqq \{\, 1/n \mid n \in \ZZ_+ \,\}$ and $\mcal B_K \coloneqq \mcal B_{\RR} \cup \{\, (a, b) \setminus K \mid a < b \,\}$
	Then, $\mcal B''$ is a basis for $\RR$ and
	\begin{itemize}[nolistsep]
		\ii $\mcal T_K$, the topology generated by $\mcal B_K$, is called the \textit{K-topology} on $\RR$.
	\end{itemize}
}

\mlemma{Comparison Among the Common Topologies on $\RR$}{
	The following holds.
	\begin{enumerate}[noitemsep, label=(\roman*)]
		\ii $\mcal T_{\RR} \subsetneq \mcal T_\ell$ ($\mcal T_\ell$ is strictly finer than $\mcal T_{\RR}$.)
		\ii $\mcal T_{\RR} \subsetneq \mcal T_K$ ($\mcal T_K$ is strictly finer than $\mcal T_{\RR}$.)
		\ii $\mcal T_\ell$ and $\mcal T_K$ are not comparable.
	\end{enumerate}
}
\pf{Proof}{
	\hfill
	\begin{enumerate}[noitemsep, label=(\roman*)]
		\ii For any $(a, b) \in \mcal B_{\RR}$ and $x \in (a, b)$, $[x, b) \in \mcal B_\ell$ and $x \in [x, b) \subseteq (a, b)$.
            Therefore, by \Cref{lem:finerIff}, $\mcal T_{\RR} \subseteq \mcal T_\ell$. \checkmark
            \par Take any $a \in \RR$. $a$ is in the interval $[a, b) \in \mcal B_\ell$
            but there are no open interval $(c, d) \in \mcal B_{\RR}$ such that $a \in (c, d) \subseteq [a, b)$.
            Therefore, by \Cref{lem:finerIff}, $\mcal T_\ell \not\subseteq \mcal T_{\RR}$. \checkmark
		\ii $\mcal T_{\RR} \subseteq \mcal T_K$ directly follows from $\mcal B_{\RR} \subseteq \mcal B_K$. \checkmark
            \par Although $0 \in (-1, 1) \setminus K \in \mcal T_K$,
            there is no $(c, d) \in \mcal B_{\RR}$ such that $0 \in (c, d) \in (-1, 1) \setminus K$.
            Therefore, by \Cref{lem:finerIff}, $\mcal T_K \not\subseteq \mcal T_{\RR}$. \checkmark
		\ii The logics in (i) and (ii) can directly imported to prove (iii). \checkmark
	\end{enumerate}
}

\dfn{Subbasis}{
	A \textit{subbasis} $\mcal S$ for $X$ is a subset of $\mcal P(X)$ whose union is $X$, i.e., $\bigcup \mcal S = X$.

	The \textit{topology generated by the subbasis} $\mcal S$
	is defined to be the collection of all unions of finite intersections of elements of $\mcal S$.
}

\mlemma[topBySIsTop]{}{
	Let $\mcal S$ be a subbasis for $X$. Then, the topology generated by $\mcal S$ is a topology on $X$.
}
\pf{Proof}{
	By \Cref{lem:topByBIsUnionsOfB}, it is enough to show that
	$\mcal B \coloneqq \big\{\, \bigcap_{i=1}^n S_i \mid S_i \in \mcal S \,\big\}$ is a basis.

	\begin{enumerate}[nolistsep, label=(\roman*)]
		\ii Since $\mcal S \subseteq \mcal B$, $X = \bigcup \mcal S \subseteq \bigcup \mcal B \subseteq X$. \checkmark
        \ii Let $B_1, B_2 \in \mcal B$ and $x \in B_1 \cap B_2$. Then,
            $B_1 = \bigcap_{i=1}^n S_i$ and $B_2 = \bigcap_{i=1}^m S_i'$ where $S_i, S_i' \in \mcal S$.
            Then, $B_1 \cap B_2 = \big(\bigcap_{i=1}^n S_i\big) \cap \big(\bigcap_{i=1}^m S_i'\big) \in \mcal B$. \checkmark
	\end{enumerate}
}

\section{The Order Topology}

\dfn{Intervals}{
    Let $X$ be a set with an order $<$ and $a, b \in X$ with $a < b$ are given.
    \begin{itemize}[nolistsep]
        \ii $(a, b) \coloneqq \{\, x \in X \mid a < x < b \,\}$ (open interval)
        \ii $[a, b) \coloneqq \{\, x \in X \mid a \le x < b \,\}$ (half-open interval)
        \ii $(a, b] \coloneqq \{\, x \in X \mid a < x \le b \,\}$ (half-open interval)
        \ii $[a, b] \coloneqq \{\, x \in X \mid a \le x \le b \,\}$ (closed interval)
    \end{itemize}
}

\dfn{Order Topology}{
    Let $X$ has more than one element. Let $\mcal B$ be collection of
    \begin{itemize}[nolistsep]
        \ii all open intervals $(a, b)$ in $X$,
        \ii all half-open intervals $[a_0, b)$ where $a_0$ is the smallest element (if $a_0$ exists), and
        \ii all half-open intervals $(a, b_0]$ where $b_0$ is the largest element (if $b_0$ exists).
    \end{itemize}
    Then, $\mcal B$ is a basis and the topology generated by $\mcal B$ is called the \textit{order topology}.
}

\mlemma{}{
    The set $\mcal B$ above is a basis.
}
\pf{Proof}{
    \hfill
    \begin{enumerate}[nolistsep, label=(\roman*)]
        \ii Take any $x \in X$.
            \begin{itemize}[nolistsep]
                \ii If $x$ is the smallest, then $x \in [x, b)$ where $b$ is some element in $X \setminus \{x\}$.
                \ii If $x$ is the largest, then $x \in (a, x]$ where $a$ is some element in $X \setminus \{x\}$.
                \ii Otherwise, there are some $a, b \in X \setminus \{x\}$ such that $a < x < b$ so $x \in (a, b)$. \checkmark
            \end{itemize}
        \ii A nonempty intersection of two basis with different types of interval is an open interval.
            An intersection of two basis with the same type of interval still belongs to the type of interval. \checkmark
    \end{enumerate}
}

\exmp{}{
    The order topology on $\ZZ_+$ is the discrete topology.
    $n \in (n-1, n+1) = \{n\}$ if $n > 1$ and $1 \in [1, 2) = \{1\}$.
}

\exmp{}{
    The order topology on $\RR$ is the standard topology on $\RR$.
}

\dfn{Ray}{
    Let $X$ be an order set and $a \in X$. There are four types of rays.
    \begin{itemize}[nolistsep]
        \ii $(a, \infty) \coloneqq \{\, x \in X \mid x > a \,\}$ (open ray)
        \ii $(-\infty, a) \coloneqq \{\, x \in X \mid x < a \,\}$ (open ray)
        \ii $[a, \infty) \coloneqq \{\, x \in X \mid x \ge a \,\}$ (closed ray)
        \ii $(-\infty, a] \coloneqq \{\, x \in X \mid x \le a \,\}$ (closed ray)
    \end{itemize}
}
\nt{
    Open rays are open in the order topology.
    \begin{itemize}[nolistsep]
        \ii If $X$ has a largest element $b_0$, then $(a, \infty) = (a, b_0]$.
        \ii Otherwise, $(a, \infty) = \bigcup_{a < b} (a, b)$.
    \end{itemize}
    Thus, $(a, \infty)$ is open. Similarly, $(-\infty, a)$ is open.
}
\nt{
    Open rays form a subbasis that generates the order topology.
}

\section{The Product Topology on $X \times Y$}
\dfn{Product Topology}{
    Let $X$, $Y$ be topological spaces.
    The \textit{product topology} on $X \times Y$ is the topology generated by a basis \[
        \mcal B \coloneqq \{\, U \times V \mid U \subseteq X \text{ and } V \subseteq Y \text{ are open} \,\}\text{.}
    \]
}

\thm[prodBasis]{}{
    Let $\mcal B$ be a basis for $X$ nd $\mcal C$ be a basis for $Y$. Then \[
        \mcal D \coloneqq \{\, B \times C \mid B \in \mcal B \text{ and } C \in \mcal C \,\}
    \] is a basis for the product topology of $X \times Y$.
}
\pf{Proof}{
    We will exploit \Cref{lem:topToBasis}.
    Take any open set $W \subseteq X \times Y$ and $x \times y \in W$.
    Then, there is a basis element $U \times V$ of the product topology $X \times Y$
    such that $x \times y \in U \times V \subseteq W$.
    Since $U$ and $V$ are open in $X$ and $Y$, respectively, and $x \in U$ and $y \in V$,
    there are $B \in \mcal B$ and $C \in \mcal C$ such that $x \in B \subseteq U$ and $y \in C \subseteq V$.

    Here, we find that $x \times y \in B \times C \subseteq U \times V \subseteq W$ while $B \times C \in \mcal D$.
    Therefore, by \Cref{lem:topToBasis}, $\mcal D$ generates the product topology.
}

\dfn{Projection}{
    Let $\pi_1 \colon X \times Y \to X$ and $\pi_2 \colon X \times Y \to Y$ defined by the equations \[
        \begin{aligned}[t]
            \pi_1(x, y) &= x \\
            \pi_2(x, y) &= y
        \end{aligned}
    \] The maps $\pi_1$ and $\pi_2$ are called the \textit{projections} of $X \times Y$
    onto its first and second factors, respectively.
}
\nt{
    If $U \subseteq X$ is open, then $\pi_1\inv(U) = U \times Y$ is open.
    Similarly, if $V \subseteq Y$ is open, then $\pi_2\inv(V) = X \times V$ is open.
}

\thm{}{
    The collection \[
        \mcal S \coloneqq \{\, \pi_1\inv(U) \mid U \subseteq X \text{ is open} \,\}
                     \cup \{\, \pi_2\inv(V) \mid V \subseteq Y \text{ is open} \,\}
    \] is a subbasis for the product topology of $X \times Y$.
}
\pf{Proof}{
    Let $\mcal T$ be the product topology and $\mcal T'$ be the topology generated by $\mcal S$.
    \begin{itemize}[nolistsep]
        \ii Since $\mcal S \subseteq \mcal T$, every union of finite intersections in $\mcal S$ is in $\mcal T$.
            Thus, $\mcal T' \subseteq \mcal T$. \checkmark
        \ii Every open set of $\mcal T$ is a union of elements in
            $\mcal B \coloneqq \{\, U \times V \mid U \subseteq X \text{ and } V \subseteq Y \text{ are open} \,\}$.
            Noting that each $U \times V$ can be expressed as $\pi_1\inv(U) \cap \pi_2\inv(V)$,
            which is a finite intersection of elements in $\mcal S$,
            we may conclude $\mcal T \subseteq \mcal T'$. \checkmark
    \end{itemize}
}

\section{The Subspace Topology}

\dfn{Subspace Topology}{
    Let $(X, \mcal T)$ be a topological space. If $Y \subseteq X$, then \[
        \mcal T_Y \coloneqq \{\, Y \cap U \mid U \in \mcal T \,\}
    \]  is called the \textit{subspace topology} of $Y$
    and $(Y, \mcal T_Y)$ is called a \textit{subspace} of $(X, \mcal T)$.
}

\mlemma{}{
    $(Y, \mcal T_Y)$ is a topological space.
}
\pf{Proof}{
    \hfill
    \begin{enumerate}[nolistsep, label=(\roman*)]
        \ii $\varnothing = Y \cap \varnothing$ and $Y = Y \cap X$. \checkmark
        \ii If $U_{\alpha} \in \mcal T_Y$,
            $\bigcup_{\alpha \in J} (Y \cap U_\alpha) = Y \cap \left(\bigcup_{\alpha \in J} U_\alpha\right) \in \mcal T_Y$. \checkmark
        \ii If $U_i \in \mcal T_Y$, $\bigcap_{i=1}^n (Y \cap U_i) = Y \cap \left(\bigcap_{i=1}^n U_i\right) \in \mcal T_Y$. \checkmark
    \end{enumerate}
}

\mlemma[subspaceBasis]{}{
    If $\mcal B$ is a basis for $(X, \mcal T)$, then \[
        \mcal B_Y \coloneqq \{\, Y \cap B \mid B \in \mcal B \,\}
    \] is a basis for the subspace topology on $Y$.
}
\pf{Proof}{
    We will exploit \Cref{lem:topToBasis}.

    Take any $U \in \mcal T$ and $y \in Y \cap U$.
    Since $y \in U$, $\exs B \in \mcal B,\: y \in B \subseteq U$,
    which implies $y \in Y \cap B \subseteq Y \cap U$.
}

\nt{
    Not all open sets in $Y$ are open in $X$.

    For instance, if $X = \RR$ and $Y = [0, 1]$, $Y$ is open in $Y$ but not open in $X$.
}

\mlemma[subspaceOfOpenSet]{}{
    All the open sets in $Y$ are open in $X$ if and only if $Y$ is open in $X$.
}
\pf{Proof}{
    ($\Rightarrow$) $Y$ is open in $Y$. Hence, $Y$ is open in $X$.

    ($\Leftarrow$) Let $U$ be any open set in $Y$.
    Then, $U = Y \cap V$ for some open set $V$ in $X$. Since $Y$ is open in $X$, $U$ is open in $X$.
}

\thm[subOfProdIsProdOfSub]{}{
    If $A$ is a subspace of $X$ and $B$ is a subspace of $Y$, then the product topology on $A \times B$
    is the same as the the topology $A \times B$ inherits as a subspace of $X \times Y$. In other words,
    the following two topologies are the same. \[
        \setlength\arraycolsep{1pt}
        \begin{array}{r@{\qquad}c}
            \text{(i)} & X, Y \xrightarrow{\text{subspace}} A \subseteq X,\, B \subseteq Y \xrightarrow{\text{product}} A \times B \\
            \text{(ii)} & X, Y \xrightarrow{\text{product}} X \times Y \xrightarrow{\text{subspace}} A \times B \subseteq X \times Y
        \end{array}
    \]
}
\pf{Proof}{
    By \Cref{th:prodBasis}, \[
        \{\, U \times V \mid U \in \mcal B_X \text{ and } V \in \mcal B_Y \,\}
    \] is a basis for $X \times Y$. Thus, \[
        \mcal B \coloneqq \{\, (A \times B) \cap (U \times V) \mid U \in \mcal B_X \text{ and } V \in \mcal B_Y \,\}
    \] is a basis for (ii) by \Cref{lem:subspaceBasis}.

    Note that $(A \times B) \cap (U \times V) = (A \cap U) \times (B \cap V)$. 
    Also, $\{\, A \cap U \mid U \in \mcal B_X \,\}$ and $\{\, B \cap V \mid V \in \mcal B_Y \,\}$
    are bases for $A$ and $B$.
    Thus, $\mcal B$ is also a basis for (i) by \Cref{th:prodBasis}.
}

\wc{Order Topology and Subspace Topology}{
    Unlike product topology and subspace topology, order topology and subspace topology are not associative.
    Let $X$ be an ordered set and $Y \subseteq X$. \[
        \setlength\arraycolsep{1pt}
        \begin{array}{r@{\qquad}c}
            \text{(i)} & Y \xrightarrow{\text{order}} Y \\
            \text{(ii)} & X \xrightarrow{\text{order}} X \xrightarrow{\text{subspace}} Y \subseteq X
        \end{array}
    \] Then, will those be the same?
    \paragraph*{Example 1.} Consider $X = \RR$ and $Y = [0, 1]$.
    Then, the subspace topology of the order topology $X$ has a basis of \[
        \mcal B_{[0,1]} = \{\, [0, 1] \cap (a, b) \mid a < b \,\}\text{,}
    \] which is in fact the order topology on $Y$. In this case, $\text{(i)}=\text{(ii)}$.
    \paragraph*{Example 2.} Consider $X = \RR$ and $Y = [0, 1) \cup \{2\}$.
    Then, $\{2\}$ is an open in (ii) since $\{2\} = Y \cap (1.5, 2.5)$.
    But, there is no basis of the order topology on $Y$ such that contains $2$ and is a subset of $\{2\}$.
    Thus, in this case, $\text{(i)} \neq \text{(ii)}$.
    \paragraph*{Example 3.} Consider $X = \RR[2]$ and $Y = I^2$ where $I = [0, 1]$.
    Then, $\{1/2\} \times (1/2, 1]$ is an open set in (ii) since it is $\big(\{1/2\} \times (1/2,3/2)\big) \cap I^2$.
    But it is not an open set in (i) since there is no basis that contain $(1/2, 1)$ and is a subset of $\{1/2\} \times (1/2, 1]$.
}

\dfn{Convex Subset}{
    Given an ordered set $X$ and $Y \subseteq X$,
    $Y$ is called \textit{convex} if \[
        \forall a, b \in Y,\: \big(a < b \implies (a, b) \subseteq Y\big)\text{.}
    \]
}

\thm[convexThenSubspaceIsOrder]{}{
    Let $X$ be an ordered set with the ordered topology. If $Y \subseteq X$ is convex,
    then the order topology on $Y$ is the same as the subspace topology.
}
\pf{Proof}{
    We will make use of the fact that open rays form a subbasis that generates the order topology.
    
    First, every open ray of (i) is an open ray of the subspace (ii). \[
        \{\, x \in Y \mid x > a \,\} = \{\, x \in X \cap Y \mid x > a \,\}\text{,}
    \] for example. Therefore, (ii) is finer than (i).

    Now, take any open ray in $X$, $(a, \infty)_X = \{\, x \in X \mid x > a \,\}$, for instance.
    Then, let \[
        \begin{aligned}[t]
            R &\triangleq (a, \infty)_X \cap Y \\
              &= \{\, y \in Y \mid y > a \,\} = (a, \infty)_Y\text{.}
        \end{aligned}
    \]

    If $a \in Y$, then $R$ is an open ray in $Y$.

    Now consider the case $a \notin Y$. If $R$ is nonempty then there is some $y_0 \in R$.
    Take any $y \in Y$. If $y_0 < y$, then $y \in R$ since $a < y_0 < y$.
    If $y < y_0$, it implies $a < y < y_0$ because $y < a < y_0$ with $y, y_0 \in Y$ implies
    $a \in Y$ by the convexity of $Y$. Therefore, $y \in R$.
    So, if $a \notin Y$, it is either $R = \varnothing$ or $R = Y$.

    Combining the cases, we get the fact that the intersection of $Y$ and an arbitrary open ray in $X$
    is an open ray in $Y$, an empty set, or the whole $Y$.

    This is the final step. Take any open set $U$ in the ordered topology $X$.
    Then, $U = \bigcup_{\alpha \in J} U_\alpha$ where $U_\alpha \neq \varnothing$ is a finite intersection of open rays in $X$.
    Noting that $U \cap Y$ is a general form of an open set in $Y$,
    we get $U \cap Y = \bigcup_{\alpha \in J} (U_\alpha \cap Y)$, which implies either $U \cap Y = Y$ or
    $U \cap Y$ is a union of finite intersections of an open ray in $Y$.
}

\cor{}{
    Let $X$ be an ordered set with the ordered topology.
    The subspace topology of $Y \subseteq X$ is finer than the order topology on $Y$.
}

\section{Closed Sets and Limit Points}
\subsection{Closed Sets}

\dfn{Closed Set}{
    Let $X$ be a topological space.
    A subset $A \subseteq X$ is \textit{closed} if $X \setminus A$ is open.
}

\exmp{}{
    \begin{itemize}[nolistsep]
        \ii $[a, b] \subseteq \RR$ is closed since $(-\infty, a) \cup (b, \infty)$ is open.
        \ii $[a, b] \times [c, d] \subseteq \RR[2]$ is closed.
        \ii In discrete topology on $X$, every subset of $X$ is closed.
        \ii If $Y = [0, 1] \cup (2, 3) \subseteq \RR$, $[0, 1]$ and $(2, 3)$ are both open and closed in $Y$.
    \end{itemize}
}

\thm[closeIsDualOfOpen]{}{
    Let $X$ be a topological space. Then the following conditions hold.
    \begin{enumerate}[nolistsep, label=(\roman*)]
        \ii $\varnothing$ and $X$ are closed.
        \ii Arbitrary intersections of closed sets are closed.
        \ii Finite unions of closed sets are closed.
    \end{enumerate}
}
\pf{Proof}{
    \hfill
    \begin{enumerate}[nolistsep, label=(\roman*)]
        \ii $X \setminus \varnothing = X$ and $X \setminus X = \varnothing$ are open. \checkmark
        \ii Let $\{A_{\alpha}\}_{\alpha \in J}$ be a collection of closed sets. Then, \[
                \textstyle X \setminus \bigcap_{a \in J} A_\alpha = \bigcup_{\alpha \in J} (X \setminus A_\alpha)\text{.}
            \] is open since each $X \setminus A_\alpha$ is open. \checkmark
        \ii Let $\{A_i\}_{i=1}^n$ be a collection of closed sets. Then, \[
                \textstyle X \setminus \bigcup_{i=1}^n A_i = \bigcap_{i=1}^n (X \setminus A_i)\text{.}
            \] is open since it is a finite intersection of open sets. \checkmark
    \end{enumerate}
}

\thm[closedIffYCapB]{}{
    Let $X$ be a topological space and $Y \subseteq X$.
    Then $A \subseteq Y$ is closed in $Y$ if and only if there is a closed set $B$ in $X$ such that $A = Y \cap B$.
}
\pf{Proof}{
    ($\Leftarrow$) Let $B$ be a closed set of $X$ such that $A = Y \cap B$.
    Then, $X \setminus B$ is open in $X$ and $Y \cap (X \setminus B) = Y \setminus A$ is open in $Y$.
    Thus, $A$ is closed in $Y$.

    ($\Rightarrow$) Since $Y \setminus A$ is open in $Y$,
    $Y \setminus A = Y \cap U$ for some open set $U$ in $X$.
    Then, $A = Y \cap (X \setminus U)$ where $X \setminus U$ is closed in $X$.
}

\thm[subspaceOfClosedSet]{}{
    If $Y$ is closed in $X$, then every closed sets of $Y$ are closed in $X$ if and only if $Y$ is closed in $X$.
}
\pf{Proof}{
    Proof is analogous to the proof of \Cref{lem:subspaceOfOpenSet}.
}

\dfn{Interior and Closure of a Set}{
    Given a subset $A$ of a topological space $(X, \mcal T)$,
    \begin{itemize}
        \ii the \textit{interior} of $A$ is $\inter A \triangleq \bigcup \{\, U \subseteq X \mid U \in \mcal T \text{ and } U \subseteq A \,\}$, and
        \ii the \textit{closure} of $A$ is $\cl A \triangleq \bigcap \{\, V \subseteq X \mid X \setminus V \in \mcal T \text{ and } A \subseteq V \}$.
    \end{itemize}
}
\nt{
    \begin{itemize}[noitemsep]
        \ii $\inter A \subseteq A \subseteq \cl A$
        \ii $\inter A$ is open, and $\cl A$ is closed.
        \ii $\inter A$ is the largest open set contained in $A$, and $\cl A$ is the smallest closed set containing $A$.
    \end{itemize}
}

\thm[closureSubspace]{}{
    Let $Y$ be a subspace of $X$ and $A \subseteq Y$.
    Let $\cl A$ and $\cl A_Y$ denote the closures of $A$ in $X$ and $Y$, respectively.
    Then, \[
        \cl A \cap Y = \cl A_Y\text{.}
    \]
}
\pf{Proof}{
    ($\supseteq$) $\cl A \cap Y$ is closed in $Y$ by \Cref{th:closedIffYCapB}.
    Thus, $\cl A_Y \subseteq \cl A \cap Y$.

    ($\subseteq$) $\cl A_Y = B \cap Y$ for some closed set $B$ in $X$ by \Cref{th:closedIffYCapB}.
    Also, $\cl A \subseteq B$ holds. Therefore, $\cl A_Y = B \cap Y \subseteq \cl A \cap Y$.
}

\dfn{Intersection and Neighborhood}{
    \begin{itemize}[nolistsep]
        \ii Given two sets $A$ and $B$, we say $A$ and $B$ \textit{intersect} if $A \cap B \neq \varnothing$.
        \ii An open set containing $x \in X$ is called an open \textit{neighborhood} of $x$.
    \end{itemize}
}

\thm[inClosureIffNeighCapANonempty]{}{
    Let $A \subseteq X$ where $X$ is a topological space. The following hold.
    \begin{enumerate}[nolistsep, label=(\roman*)]
        \ii $x \in \cl A$ if and only if every neighborhood of $x$ intersects $A$.
        \ii Let $\mcal B$ be a basis for $X$.
            Then, $x \in \cl A$ if and only if every $B \in \mcal B$ containing $x$ intersects $A$.
    \end{enumerate}
}
\pf{Proof}{
    \hfill
    \begin{enumerate}[nolistsep, label=(\roman*)]
        \ii We will prove the contrapositive
            ``$x \notin \cl A \iff \exs \text{ neighborhood } U \text{ of } X,\: U \cap A = \varnothing$''.
            \par ($\Rightarrow$) $U \triangleq X \setminus \cl A$ is a neighborhood of $x$.
            We find that $U \cap A = \varnothing$ since $A \subseteq \cl A$. \checkmark
            \par ($\Leftarrow$) Suppose a neighborhood $U$ of $x$ satisfies $U \cap A = \varnothing$.
            It implies $A \subseteq X \setminus U$. Since $X \setminus U$ is closed, $\cl A \subseteq X \setminus U$ also holds.
            Since $x \in U$, $x \in \cl A$ may never hold. \checkmark
        \ii ($\Rightarrow$) A basis element that contains $x$ is a neighborhood of $x$. \checkmark
            \par ($\Leftarrow$) Follows from the definition of basis. (See \Cref{def:defBasis}.) \checkmark
    \end{enumerate}
}

\exmp{}{
    \begin{itemize}[nolistsep]
        \ii If $A = (0, 1/2) \subseteq \RR$, then $\cl A = [0, 1/2]$.
        \ii If $A = \{\, 1/n \mid n \in \ZZ_+ \,\} \subseteq \RR$, then $\cl A = A \cup \{0\}$.
        \ii If $A = \QQ \subseteq \RR$, then $\cl A = \RR$.
        \ii If $A = \ZZ \subseteq \RR$, then $\cl A = \ZZ$.
    \end{itemize}
}

\subsection{Limit Points}

\dfn{Limit Point}{
    Let $A \subseteq X$ and $x \in X$.
    The point $x$ is a \textit{limit point} of $A$ if every neighborhood of $x$ intersects $A$
    in some point other than $x$.
    The set of limit points of $A$ is denoted by $A'$.
}

\nt{
    Equivalently, $x$ is a limit point of $A$ if $x \in \cl{A \setminus \{x\}}$ thanks to \Cref{th:inClosureIffNeighCapANonempty}.
}

\thm[closureIsAcupAprime]{}{
    Let $A \subseteq X$ where $X$ is a topological space. Then \[
        \cl A = A \cup A'\text{.}
    \]
}
\pf{Proof}{
    ($\supseteq$) We only need to show $A' \subseteq \cl A$.
    For every $x \in A'$, $x \in \cl A$ due to \Cref{th:inClosureIffNeighCapANonempty}. \checkmark

    ($\subseteq$) Let $x \in \cl A \setminus A$.
    By definition, every neighborhood of $x$ intersects $A$ while $x$ cannot be in the intersection
    since $x \notin A$. Thus, $x \in A'$. \checkmark
}

\cor[closedIffContainsLimPts]{}{
    Let $A \subseteq X$ where $X$ is a topological space.
    Then $A$ is closed if and only if $A' \subseteq A$.
}
\pf{Proof}{
    ($\Rightarrow$) $A = \cl A = A \cup A'$ and it implies $A' \subseteq A$. \checkmark
    \par ($\Leftarrow$) $\cl A = A \cup A' = A$ and $\cl A$ is closed. \checkmark
}

\dfn{Convergence of a Sequence}{
    Let $X$ be a topological space.
    Then, a sequence $\{x_n\}$ in $X$ converges to $x \in X$
    if, for every neighborhood $U$ of $x$,
    there exists $N \in \ZZ_+$
    such that $x_n \in U$ for all $n \ge N$.
}

\nt{
    The point to which a sequence converges may not be unique in general.
    If $X = \{a, b, c\}$ and $\mcal T = \{\varnothing, X, \{b\}, \{a, b\}, \{b, c\}\}$,
    the sequence $x_n = b$ may converge to $a$, $b$, or $c$
    as any neighborhood of $a$ or $c$ contains $b$.
}

\subsection{Hausdorff Spaces}

\dfn{Housdorff Space}{
    A topological space $(X, \mcal T)$ is called a \textit{Hausdorff space} if for each pair $x_1$ and $x_2$
    of distinct points of $X$, there exist neighborhoods $U_1$ and $U_2$ of $x_1$ and $x_2$, respectively,
    that are disjoint. In other words, \[
        \forall x_1, x_2 \in X,\: \big(x_1 \neq x_2 \implies
        \exs U_1, U_2 \in \mcal T,\: x_1 \in U_1 \land x_2 \in U_2 \land U_1 \cap U_2 = \varnothing\big)\text{.}
    \]
}

\thm[finiteSetIsClosedInHaus]{}{
    Every finite point set in a Hausdorff space $X$ is closed.
}
\pf{Proof}{
    It suffices to prove that every singleton of $X$ is closed
    since closedness of finite point set will be naturally driven by \Cref{th:closeIsDualOfOpen}.

    If $|X| \le 1$, then it is done.
    Now, let $x$ and $y$ be distinct elements in $X$.
    Then, there are disjoint open sets $U$ and $V$ such that
    $x \in U$ and $y \in V$.
    Therefore, $x$ and $y$ are not limit points of each other.
    Thus, there are at most one limit point of $\{x\}$. (If it exists, it must be $x$.)
    Thus, $\{x\}' \subseteq \{x\}$; $\{x\}$ is closed by \Cref{cor:closedIffContainsLimPts}.
}

\dfn{$T_1$ Axiom}{
    A topological space $X$ is said to satisfy $T_1$ \textit{axiom} if
    every singleton in $X$ is closed.
}
\nt{
    \Cref{th:finiteSetIsClosedInHaus} implies that
    every Hausdorff space satisfies $T_1$ axiom.
}

\nt{
    $T_1$ axiom is strictly weaker than being a Hausdorff space.
    \begin{itemize}[nolistsep]
        \ii $\RR$ in the finite complement topology satisfies $T_1$ axiom.
            Every singleton $\{x\}$ is closed since $\RR \setminus \{x\}$ is open.
        \ii However, it is not a Hausdorff space.
            Suppose $x, y \in \RR$ with $x \neq y$
            and there are disjoint open set $U$ and $V$
            such that $x \in U$ and $y \in V$.
            Then, since $U \cap V = \varnothing$,
            $\RR = \RR \setminus (U \cap V) = (X \setminus U) \cup (X \setminus V)$,
            which is impossible siunce $X \setminus U$ and $X \setminus V$ are finite.
    \end{itemize}
}

\thm[t1LimPtIffDelNeiIntersectsInf]{}{
    Let $X$ be a space satisfying the $T_1$ axiom; let $A \subseteq X$.
    Then $x \in A'$ if and only if every neighborhood of $x$
    contains infinitely many points of $A$.
}
\pf{Proof}{
    ($\Rightarrow$)
    Let $x \in A'$ and suppose some neighborhood $U$ of $x$ intersects $A$
    in finitely many points.
    Then, it also intersects $A \setminus \{x\}$ in finitely many points;
    let us denote them $x_1, x_2, \cdots, x_m$.
    Noting that $\{x_1, x_2, \cdots, x_m\}$ is closed as $X$ satisfies $T_1$ axiom,
    $X \setminus \{x_1, x_2, \cdots, x_m\}$ is a neighborhood of $x$
    but does not intersect $A \setminus \{x\}$,
    contradicting that $x$ is a limit point of $A$.

    ($\Leftarrow$)
    Let $U$ be any neighborhood of $x$.
    Then, $U$ intersects $A$ in infinitely many points by assumption, and thus
    it intersects $A \setminus \{x\}$ in infinitely many points.
    Therefore, $x$ is a limit point of $A$.
}

\thm[seqConvUniqueInHaus]{}{
    If $X$ is a Hausdorff space,
    then there is at most one point of $X$ to which a sequence of points of $X$ converges.
}
\pf{Proof}{
    Suppose $\{x_n\}$ is a sequence in $X$ that converges to $x$.
    If $y \neq x$, we may find disjoint neighborhoods $U$ and $V$ of $x$ and $y$, respectively.
    Then, $U$ has all but finitely many points of $x_n$, but $V$ cannot.
    Therefore, $y$ cannot be a point that $\{x_n\}$ converges to.
}

\nt{
    The finite complement topology on $\RR$ is not a Hausdorff.

    Let $\{x_n\}$ be a sequence that has
    no points infinitely repeated in $\{x_n\}$.
    Then, $\{x_n\}$ converges to every point in $\RR[n]$.
}

\section{Continuous Functions}
\subsection{Continuity of a Function}

\dfn{Continuity of a Function}{
    Let $X$ and $Y$ be topological spaces.
    A function $f \colon X \to Y$ is said to be \textit{continuous}
    if for each open subset $V$ of $Y$, $f\inv(V)$ is open in $X$.
}

\nt{
    To prove a function $f \colon X \to Y$ is continuous,
    it is enough to prove that every basis of $Y$
    has an open preimage in $X$.
    Then, for every open $V = \bigcup_{\alpha \in J} B_{\alpha} \subseteq Y$,
    it follows that
    \[
        f\inv(V) = \bigcup_{\alpha \in J}f\inv(B_\alpha)
    \] is open in $X$.

    If the topology on $Y$ is given by a subbasis,
    it is even sufficient to prove every preimage of subbasis element is open.
    Then, for every basis $B = \bigcap_{i=1}^n S_i$, it follows that
    \[
        f\inv(B) = \bigcap_{i=1}^n f\inv(S_i)
    \] is open in $X$.
}

\thm[contiTFAE]{}{
    Let $X$ and $Y$ be topological spaces. \textsf{TFAE}
    \begin{enumerate}[noitemsep, label=(\roman*)]
        \ii $f$ is continuous.
        \ii For every subset $A$ of $X$, $f(\cl A) \subseteq \cl{f(A)}$.
        \ii For every closed set $B$ of $Y$, the set $f\inv(B)$ is closed in $X$.
        \ii For each $x \in X$ and each neighborhood $V$ of $f(x)$,
            there is a neighborhood $U$ of $x$ such that $f(U) \subseteq V$.
    \end{enumerate}
}
\pf{Proof}{
    ((i) $\Longrightarrow$ (ii))
    Take any $x \in \cl A$.
    Let $V$ be any neighborhood of $f(x)$.
    Then, $f\inv(V)$ is a neighborhood of $x$.
    Since $x \in \cl A$, by \Cref{th:inClosureIffNeighCapANonempty},
    $f\inv(V)$ intersects $A$; $A \cap f\inv(V) \neq \varnothing$.
    Therefore, since $\varnothing \neq f(A \cap f\inv(V)) = f(A) \cap f(f\inv(V)) \subseteq f(A) \cap V$,
    $V$ intersects $f(A)$; by \Cref{th:inClosureIffNeighCapANonempty},
    $f(x) \in \cl{f(A)}$ as $V$ was arbitrary.
    Therefore, $f(\cl A) \subseteq \cl{f(A)}$.

    ((ii) $\Longrightarrow$ (iii))
    Let $B$ be closed in $Y$ and let $A \triangleq f\inv(B)$.
    Then, $f(A) = f(f\inv(B)) \subseteq B$.
    Therefore, if $x \in \cl A$,
    $f(x) \in f(\cl A) \subseteq \cl{f(A)} \subseteq \cl B = B$;
    which implies $x \in f\inv(B) = A$.
    This means $\cl A \subseteq A$, thus $A$ is closed.

    ((iii) $\Longrightarrow$ (i))
    Let $V$ be an open set of $Y$. Let $B \triangleq Y \setminus B$. Then
    \[
        f\inv(B) = f\inv(Y) \setminus f\inv(V) = X \setminus f\inv(V)
    \]
    is closed as $B$ is closed. Thus, $f\inv(V) = X \setminus f\inv(B)$ is open.

    ((i) $\Longrightarrow$ (iv))
    For every neighborhood $V$ of $f(x)$,
    $U=f\inv(V)$ is the neighborhood of $x$ that satisfies $f(U) \subseteq V$.

    ((iv) $\Longrightarrow$ (i))
    Let $V$ be an open set of $Y$. Then, for each $x \in f\inv(V)$,
    since $V$ is a neighborhood of $f(x)$,
    there exists a neighborhood $U_x$ of $x$ that satisfies $f(U_x) \subseteq V$.
    Then, $U_x \subseteq f\inv(f(U_x)) \subseteq f\inv(V)$.
    Therfore, $f\inv(V) = \bigcup_{x \in f\inv(V)} U_x$ is open in $X$.
}

\subsection{Homeomorphisms}

\dfn{Homeomorphism}{
    Let $X$ and $Y$ be topological spaces $f \colon X \to Y$ be a bijection.
    $f$ is called a \textit{homeomorphism} if both $f$ and $f\inv$ are continuous.
}

\nt{
    Since the inverse image under $f\inv$ is exactly the image under $f$,
    ``$f\inv$ is continuous'' implies ``$f(U)$ is open for all open $U$ in $X$.''
    So, $f$ is a homeomorphism if and only if
    it is a bijection such that $U \subseteq X$ is open in $X$
    if and only if $f(U)$ is open in $Y$.
}

\nt{
    If $f$ is a homeomorphism between $X$ and $Y$,
    then $\mcal T_Y = \{\,f(U) \mid U \in \mcal T_X\,\}$ and
    $\mcal T_X = \{\,f\inv(V) \mid V \in \mcal T_X\,\}$.

    Therefore, any property of $X$ that is entirely expressed
    in terms of the topology of $X$ yields,
    via the correspondence $f$,
    the corresponding property for the space $Y$.
    Such a property of $X$ is called \textit{topological property} of $X$.\\[1em]
    \centerline{\fbox{\textit{Homeomorphism preserves topological properties.}}}
}

\dfn{Open Map and Closed Map}{
    Let $X$ and $Y$ be topological spaces $f \colon X \to Y$ be a function.
    \begin{itemize}[nolistsep]
        \ii $f$ is said to be an \textit{open map}
            if $f(U)$ is open for all open $U \subseteq X$ in $X$.
        \ii $f$ is said to be a \textit{closed map}
            if $f(U)$ is closed for all closed $U \subseteq X$ in $X$.
    \end{itemize}
}

\dfn{Topological Imbedding}{
    Let $X$ and $Y$ be topological spaces $f \colon X \hookrightarrow Y$ be an injection.
    Then, $f' \colon X \to f(X)$ obtained by restriction is a bijection.
    If $f'$ is a homeomorphism in which the topology of $\Img f$ is given as the subspace topology,
    $f$ is said to be a \textit{topological imbedding}, or simply an \textit{imbedding}, of $X$ in $Y$.
}

\subsection{Constructing Continuous Functions}

\thm[constructConti]{Rules for Constructing Continuous Functions}{
    Let $X$, $Y$, and $Z$ be topological spaces.
    \begin{enumerate}[label=(\roman*)]
        \ii (\textit{Constant Function})
            If $f \colon X \to Y$ has a singleton $f(X)$,
            $f$ is continuous.
        \ii (\textit{Inclusion})
            If $A$ is a subspace of $X$,
            the inclusion function $j \colon A \to X$ is continuous.
        \ii (\textit{Composites})
            If $f \colon X \to Y$ and $g \colon Y \to Z$ are continuous,
            then the map $g \circ f$ is continuous.
        \ii (\textit{Restricting the Domain})
            If $f \colon X \to Y$ is continuous,
            and if $A$ is a subspace of $X$,
            then the restricted function $f \big|_A \colon A \to Y$ is continuous.
        \ii (\textit{Restricting or Expanding the Codomain})
            Let $f \colon X \to Y$ be continuous.
            If $Z$ is a subspace of $Y$ and $f(X) \subseteq Z$,
            then the function $g \colon X \to Z$ obtained by restricting the range of $f$ is continuous.
            If $Z$ is a space having $Y$ as a subspace,
            then the function $h \colon X \to Z$ obtained by expanding the range of $f$ is continuous.
        \ii (\textit{Local Formulation of Continuity})
            The map $f \colon X \to Y$ is continuous
            if $X$ is a union of open sets $U_\alpha$
            such that $f \big|_{U_\alpha}$ is continuous for each $\alpha$.
    \end{enumerate}
}
\pf{Proof}{
    \hfill
    \begin{enumerate}[label=(\roman*)]
        \ii Let $f(x) = y_0$ for every $x \in X$ for some fixed $y_0 \in Y$.
            Then, for each (open) set $V \subseteq Y$,
            \[
                f\inv(V) = \begin{cases}
                    X           & \text{if } y_0 \in V    \\
                    \varnothing & \text{if } y_0 \notin V
                \end{cases}
            \]
            is always open in $X$.
        \ii If $U$is open in $X$, then $f\inv(U) = U \cap A$ is open in $A$ (by definition).
        \ii If $U$ is open in $Z$, then $g\inv(U)$ is open in $Y$,
            and thus $(g \circ f)\inv(U) = f\inv(g\inv(U))$ is open in $X$.
        \ii $f \big|_A = f \circ j$ where $j \colon A \to X$ is the inclusion function.
            Therefore, $f \big|_A$ is continuous by (ii) and (iii).
        \ii First, suppose $f(X) \subseteq Z \subseteq Y$.
            Take any open set $W \subseteq Z$ of $Z$.
            Then, $W = V \cap Z$ for some open set $V$ in $Y$.
            Because $f(X) \subseteq Z$ and $f(x) = g(x)$ for all $x \in X$,
            \[
                f\inv(V) = f\inv(V \cap Z) = f\inv(W) = g\inv(W)\text{.}
            \] Thus, $g\inv(W)$ is open in $X$ as $f$ is continuous.

            We get $h$ is continuous from noting that
            $h = j \circ f$ where $j \colon Y \to Z$ is the inclusion function.
        \ii Let $X = \bigcup_{\alpha \in J} U_\alpha$ in which,
            for each $\alpha \in J$, $U_\alpha$ is an open set in $X$
            such that $f \big|_{U_\alpha}$ is continuous.
            Let $V$ be an open set in $Y$. Then
            \[
                f\inv(V) \cap U_\alpha = \big(f\big|_{U_\alpha}\big)\inv(V)
            \] for each $\alpha \in J$;
            $f\inv(V) \cap U_\alpha$ is open in $X$
            since $f \big|_{U_\alpha}$ is continuous.
            Therefore,
            \[
                f\inv(V) = f\inv(V) \cap X
                = f\inv(V) \cap \left(\bigcup_{\alpha \in J} U_\alpha\right)
                = \bigcup_{\alpha \in J} \big(f\inv(V) \cap U_\alpha\big)
            \] is open in $X$.
        
    \end{enumerate}
}

\thm[pasting]{The Pasting Lemma}{
    Let $X = A \cup B$ be a topological space, where $A$ and $B$ are closed in $B$.
    Let $f \colon A \to Y$ and $g \colon B \to Y$ be continuous.
    If $f(x) = g(x)$ for every $x \in A \cap B$,
    then the function $h \colon X \to Y$ defined by
    \[
        h(x) \triangleq \begin{cases}
            f(x) & \text{if } x \in A \\
            g(x) & \text{if } x \in B
        \end{cases}
    \] is continuous.
}
\pf{Proof}{
    Let $C$ be a closed subset of $Y$. Now
    \[
        h\inv(C) = f\inv(C) \cup g\inv(C)\text{.}
    \] Since $f$ and $g$ are continuous and $C$ is closed,
    $f\inv(C)$ and $g\inv(C)$ are closed by \Cref{th:contiTFAE}.
    Thus, $h\inv(C)$ is closed.
    Hence, $h$ is continuous.
}

\nt{
    \Cref{th:pasting} holds if $A$ and $B$ are both open.
    It is, nonetheless, a special case of (vi) of \Cref{th:constructConti}.
}

\nt{
    \Cref{th:pasting} does not hold if $A$ is open and $B$ is closed.
    For instance, the function $h \colon A \cup B \to \RR$,
    where $A = (-\infty, 0)$ and $B = [0, \infty)$,
    defined by \[
        h(x) \triangleq \begin{cases}
            x - 2 & \text{if } x \in A \\
            x + 2 & \text{if } x \in B
        \end{cases}
    \] is not continuous
    since $h\inv((1, 3)) = [0, 1)$ is not open.
}

\thm[contiIffCoordsConti]{Maps Into Products}{
    Let $f \colon A \to X \times Y$ be given by
    \[
        f(a) = f_1(a) \times f_2(b)\text{.}
    \]
    Then $f$ is continuous if and only if the functions
    \[
        f_1 \colon A \to X \quad \text{and} \quad f_2 \colon A \to Y
    \]
    are continuous.
}
\pf{Proof}{
    ($\Rightarrow$)
    We first show that the projections
    $\pi_1 \colon X \times Y \to X$ and $\pi_2 \colon X \times Y \to Y$
    are continuous.
    For each open sets $U \subseteq X$ and $V \subseteq Y$,
    $\pi_1\inv(U) = U \times Y$ and $\pi_2\inv(V) = X \times V$
    are open; $\pi_1$ and $\pi_2$ are continuous.

    Then, noting that $f_1 = \pi_1 \circ f$ and $f_2 = \pi_2 \circ f$,
    we conlcude $f_1$ and $f_2$ are continuous.

    ($\Leftarrow$)
    For any basis element $U \times V$ in $X \times Y$,
    \[
        \begin{aligned}[t]
            f\inv(U \times V) &= \{\,a \in A \mid f(a) \in U \times V\,\} \\
                              &= \{\,a \in A \mid f_1(a) \in U \text{ and } f_2(a) \in V\,\} \\
                              &= f_1\inv(U) \cap f_2\inv(V)\text{.}
        \end{aligned}
    \]
    Thus, $f\inv(U \times V)$ is open since $f_1\inv(U)$ and $f_2\inv(V)$ are open.
}

\section{The Product Topology}

\dfn[boxDef]{Box Topology}{
    Let $\{X_{\alpha}\}_{\alpha \in J}$ be an indexed family of topological spaces.
    The topology generated by the basis
    \[
        \mcal B = \left\{\,\prod_{\alpha \in J} U_\alpha \:\bigg|\:
                           \forall \alpha \in J,\: U_\alpha \text{ is open in } X_\alpha \,\right\}
    \]
    for the product $\prod_{\alpha \in J} X_\alpha$
    is called the \textit{box topology}.
}
\nt{
    The collection $\mcal B$ is indeed a basis for $\prod_{\alpha \in J} X_\alpha$.
    $\bigcup \mcal B = \prod_{\alpha \in J} X_\alpha$ holds since
    $\prod_{\alpha \in J} X_\alpha \in \mcal B$.
    Also, an intersection of two basis elements is another basis element.
    This can be shown by
    \[
        \left(\prod_{\alpha \in J} U_\alpha\right) \cap \left(\prod_{\alpha \in J} V_\alpha\right)
        = \prod_{\alpha \in J} \big(U_\alpha \cap V_\alpha\big)\text{.}
    \]
}

\dfn{Projection}{
    Let $\{X_{\alpha}\}_{\alpha \in J}$ be an indexed family of sets. Let
    \[
        \pi_{\beta} \colon \prod_{\alpha \in J} X_\alpha \to X_\beta
    \]
    be defined by
    \[
        (x_\alpha)_{\alpha \in J} \mapsto x_\beta
    \]
    for each $\beta \in J$.
    Then, $\pi_\beta$ is called the \textit{projection mapping}
    associated with the index $\beta$.
}

\dfn[prodDef]{Product Topology}{
    Let $\{X_{\alpha}\}_{\alpha \in J}$ be an indexed family of topological spaces.
    Let $\mcal S_\beta$ denote the collection
    \[
        \mcal S_\beta = \{\,\pi_\beta\inv(U_\beta) \mid U_\beta \text{ is open in } X_\beta\,\}
    \] and let
    \[
        \mcal S = \bigcup_{\alpha \in J} S_\alpha\text{.}
    \]
    The topology generated by the subbasis $\mcal S$ for $\prod_{\alpha \in J} X_\alpha$
    is called the \textit{product topology}.
    In this topology, $\prod_{\alpha \in J} X_\alpha$ is called a \textit{product space}.
}

\nt{
    A typical basis of the product topology has a form of
    \[
        B = \pi_{\beta_1}\inv(U_{\beta_1}) \cap \pi_{\beta_2}\inv(U_{\beta_2}) \cap \cdots \cap \pi_{\beta_n}\inv(U_{\beta_n})
    \]
    where $\beta_i \in J$ and $U_{\beta_i}$ is open in $X_{\beta_i}$ for each $i \in [n]$.
    Since $\pi_\beta\inv(U_2) \cap \pi_\beta\inv(U_2) = \pi_\beta\inv(U_1 \cap U_2)$,
    without loss of generality, $\beta_i$'s are mutually different.
    This means,
    \[
        B = \prod_{\alpha \in J} U_\alpha
    \]
    where $U_\alpha = \begin{cases}
        U_{\beta_i} & \text{if }\alpha = \beta_i \text{ for some } i \in [n] \\
        X_\alpha    & \text{otherwise.}
    \end{cases}$
    In other words, a basis element is a product of $U_\alpha$'s
    where $U_\alpha$ is an open set of $X_\alpha$ for finitely many indices
    and $U_\alpha = X_\alpha$ for the remaining indices.
}

\nt{
    \begin{itemize}
        \ii For finite products, i.e., for finite $J$,
            the box topology and the product topology on $\prod_{\alpha \in J} X_\alpha$ are the same.
        \ii In general, the box topology is finer than the product topology
            since the basis of the box topology contains the basis of the product topology.
    \end{itemize}
}

\thm[basisOfProdFromBases]{}{
    Suppose the topology on each space $X_{\alpha}$ is given by a basis $\mcal B_{\alpha}$.
    Then,
    \[
        \mcal B_1 = \left\{\,\prod_{\alpha \in J} B_{\alpha} \:\bigg|\:
                 \forall \alpha \in J,\: B_\alpha \in \mcal B_{\alpha}\,\right\}
    \] is a basis for the box topology on $\prod_{\alpha \in J} X_\alpha$.

    Moreover,
    \[
        \mcal B_2 = \left\{\,\prod_{\alpha \in J} B_{\alpha} \:\bigg|\:
        B_\alpha \in \mcal B_\alpha \text{ for finitely many } \alpha\text{'s and }
        B_\alpha = X_\alpha \text{ for remaining indices}\,\right\}
    \] is a basis for the product topology on $\prod_{\alpha \in J} X_{\alpha}$.
}
\pf{Proof}{
    The basis for the box topology in \Cref{def:boxDef}
    has $B_1$ has a subset. Thus, the box topology is finer than
    the topology generated by $B_1$.
    
    Also, for any basis element $\prod_{\alpha \in J} U_\alpha$ of the box topology
    and $x \in \prod_{\alpha \in J} U_\alpha$,
    since $x_{\alpha} \in U_\alpha$,
    there exists some $B_{\alpha} \in \mcal B_{\alpha}$ such that
    $x_{\alpha} \in B_{\alpha} \subseteq U_{\alpha}$.
    Thus, $x \in \prod_{\alpha \in J} B_{\alpha} \subseteq \prod_{\alpha \in J} U_{\alpha}$;
    the topology generated by $\mcal B_1$ is finer than the box topology by \Cref{lem:finerIff}.

    Every element in $\mcal B_2$ is a basis element of the product topology.
    Thus, $\mcal B_2$ generates a product which is coarser than the product topology.

    Let $B = \prod_{\alpha \in J} U_\alpha$ be a basis of the product topology and $x \in B$.
    Then, $U_\alpha = X_\alpha$ for all but finitely many many indices;
    let $\alpha_1, \alpha_2, \cdots, \alpha_n$ denote indices where $U_\alpha \neq X_\alpha$.
    Then, for each $i \in [n]$, since $x_{\alpha_i} \in U_{\alpha_i}$,
    there exists bais element $B_{\alpha_i} \in \mcal B_{\alpha_i}$
    such that $x_{\alpha_i} \in B_{\alpha_i} \subseteq U_{\alpha_i}$.
    Thus, $x \in \prod_{\alpha \in J} B_{\alpha} \subseteq B$
    where $B_\alpha = X_\alpha$ if $\alpha \notin \{\, \alpha_1, \alpha_2, \cdots, \alpha_n \,\}$.
}

\thm[prodIsSubspIfEachIsSubsp]{}{
    Let $A_{\alpha}$ be a subspace of $X_{\alpha}$ for each $\alpha \in J$.
    Then $\prod_{\alpha \in J} A_{\alpha}$ is a subspace of $\prod_{\alpha \in J} X_{\alpha}$,
    if both products are given in the box topology,
    or if both products are given in the product topology.
}
\pf{Proof}{
    (\textit{For box topology})
    The box topology on $\prod_{\alpha \in J} A_{\alpha}$ has a basis of
    \[
        \textstyle \big\{\,\prod_{\alpha \in J} (A_{\alpha} \cap U_{\alpha}) \:\big|\:
            U_{\alpha} \text{ is open in } X_{\alpha}\,\big\}\text{,}
    \]
    which is exactly equal to the subspace topology of $\prod_{\alpha \in J} A_{\alpha}$,
    \[
        \textstyle \big\{\,\big(\prod_{\alpha \in J} A_{\alpha}\big) \cap \big(\prod_{\alpha \in J} U_\alpha\big) \:\big|\:
            U_{\alpha} \text{ is open in } X_{\alpha}\,\big\}\text{.}
    \]

    (\textit{For product topology})
    It is analogous; the theorem comes inherently from the fact that
    $\prod (A_\alpha \cap U_\alpha) = \big(\prod A_\alpha\big) \cap \big(\prod U_\alpha\big)$.
}

\thm[prodIsHausIfEachIsHaus]{}{
    If each space $X_{\alpha}$ is a Hausdorff space,
    then $\prod_{\alpha \in J} X_{\alpha}$ is a Hausdorff space
    in both the box and the product topologies.
}
\pf{Proof}{
    Let $x, y \in \prod_{\alpha \in J} X_{\alpha}$ with $x \neq y$.
    Then, there is some index $\alpha_0 \in J$ such that $x_{\alpha_0} \neq y_{\alpha_0}$.
    Then, since $X_{\alpha_0}$ is Hausdorff,
    there are disjoint neighborhoods $U$ and $V$
    in $X_{\alpha_0}$ of $x_{\alpha_0}$ and $y_{\alpha_0}$, respectively.
    Then, $x \in \prod_{\alpha \in J} U_{\alpha}$ and $y \in \prod_{\alpha \in J}$
    where \[
        U_\alpha \triangleq \begin{cases}
            U          & \text{if } \alpha = \alpha_0 \\
            X_{\alpha} & \text{otherwise}
        \end{cases} \quad \text{and} \quad
        V_\alpha \triangleq \begin{cases}
            V          & \text{if } \alpha = \alpha_0 \\
            X_{\alpha} & \text{otherwise.}
        \end{cases}
    \]
    As $\prod_{\alpha \in J} U_{\alpha}$ and $\prod_{\alpha \in J} V_{\alpha}$
    are open in both topologies,
    they are disjoint neighborhoods of $x$ and $y$ in both topologies.
}

\thm[prodOfClosureIsClosureOfProd]{}{
    Let $\{X_{\alpha}\}_{\alpha \in J}$ be an indexed family of spaces
    and $A_{\alpha} \subseteq X_{\alpha}$ for each $\alpha \in J$.
    Then
    \[
        \textstyle \prod_{\alpha \in J} \cl{A_{\alpha}} = \cl{\prod_{\alpha \in J} A_{\alpha}}
    \] in both the box and the product topologies.
}
\pf{Proof}{
    ($\subseteq$)
    Let $x \in \prod_{\alpha \in J} \cl{A_{\alpha}}$.
    Let $U = \prod_{\alpha \in J} U_{\alpha}$ be a basis element
    (for either the box or the product topology) that contains $x$.
    For each $\alpha \in J$, since $x_{\alpha} \in \cl{A_{\alpha}}$ and $U_{\alpha}$ is a neighborhood of $x$,
    $U_\alpha \cap A_\alpha \neq \varnothing$ by \Cref{th:inClosureIffNeighCapANonempty}.
    This implies
    \[ \textstyle
        \big(\prod_{\alpha \in J} A_{\alpha}\big) \cap U
        = \big(\prod_{\alpha \in J} A_{\alpha}\big) \cap \big(\prod_{\alpha \in J} U_{\alpha}\big)
        = \prod_{\alpha \in J} (A_{\alpha} \cap U_{\alpha}) \neq \varnothing
    \]
    Since the choice of $U$ was arbitrary, by \Cref{th:inClosureIffNeighCapANonempty},
    $x \in \cl{\prod_{\alpha \in J} A_{\alpha}}$.

    ($\supseteq$)
    Let $x \in \cl{\prod_{\alpha \in J} A_{\alpha}}$.
    Fix any $\alpha_0 \in J$, and let $U_{\alpha_0}$ be a neighborhood of $x_{\alpha_0}$ in $X_{\alpha_0}$.
    Since $\pi_{\alpha_0}\inv(U_{\alpha_0})$ is a neighborhood of $x$ (in both topologies),
    $\pi_{\alpha_0}\inv(U_{\alpha_0}) \cap \prod_{\alpha \in J} A_{\alpha} \neq \varnothing$
    by \Cref{th:inClosureIffNeighCapANonempty}.
    In particular, at the $\alpha_0^{\>\text{th}}$ index,
    $U_{\alpha_0} \cap A_{\alpha_0} \neq \varnothing$.
    Thus, $x_{\alpha_0} \in \cl{A_{\alpha_0}}$.

    Therefore, $x \in \prod_{\alpha \in J} \cl{A_{\alpha}}$.
}

\nt{
    \Cref{th:prodIsSubspIfEachIsSubsp}, \Cref{th:prodIsHausIfEachIsHaus}, and \Cref{th:prodOfClosureIsClosureOfProd}
    illustrate the common property of the box and the product topologies.
    We are now going to investigate the \textit{differences} that makes the product topology more useful.
}

\thm[contiIffCoordsContiGen]{}{
    Let $f \colon A \to \prod_{\alpha \in J} X_{\alpha}$ be given by the equation
    \[
        f(a) = \big(f_{\alpha}(a)\big)_{\alpha \in J}\text{,}
    \] where $f_{\alpha} \colon A \to X_{\alpha}$ for each $\alpha$.
    Let $\prod_{\alpha \in J} X_{\alpha}$ have the product topology.
    Then $f$ is continuous if and only if each $f_\alpha$ is continuous.
}
\pf{Proof}{
    ($\Rightarrow$)
    For each $\alpha \in J$, since $\pi_{\alpha}$ is continuous,
    $f_{\alpha} = \pi_{\alpha} \circ f$ is continuous by (iii) of \Cref{th:constructConti}.

    ($\Leftarrow$)
    Let $\pi_{\alpha}\inv(U_{\alpha})$ be any subbasis element of the product topology.
    Since $\pi_{\alpha} \circ f = f_{\alpha}$,
    $f\inv(\pi_\alpha\inv(U_\alpha)) = f_\alpha\inv(U_\alpha)$ is open.
    Thus, $f$ is contiuous.
}

\nt{
    It still holds in the box topology that, if $f$ is continuous,
    then each $f_{\alpha}$ is continuous.
    The proof is exactly the same.

    However, the converse does not hold.
    If we let $f \colon \RR \to \RR[\omega]$ (where $\RR$ is in the standard topology) defined by
    \[
        f(t) = (t, t, t, \cdots)\text{,}
    \]
    the coordinate functions $f_n \colon \RR \to \RR$ defined by $f_n(t) = t$ are continuous.
    However, $f$ is not continuous. The set
    \[
        U = \prod_{n \in \ZZ_+} \left(-\frac{1}{n}, \frac{1}{n}\right)
    \] is open in $\RR[\omega]$ endowed with the box topology.
    However, its inverse image $f\inv(U) = \{0\}$ is not open in $\RR$.
}

\section{The Metric Topology}

\dfn{Metric}{
    A \textit{metric} on a set $X$ is a function
    \[
        d \colon X \times X \to \RR
    \] having the following properties.
    \begin{enumerate}[noitemsep, label=(\roman*)]
        \ii (\textit{Positive Definiteness}) $d(x, y) \ge 0$ for all $x, y \in X$; equality holds if and only if $x = y$.
        \ii (\textit{Symmetry}) $d(x, y) = d(y, x)$ for all $x, y \in X$.
        \ii (\textit{Triangle Inequality}) $d(x, z) \le d(x, y) + d(y, z)$ for all $x, y, z \in X$.
    \end{enumerate}
}

\dfn{Epsilon-Ball}{
    Given a metrid $d$ on $X$ and $\veps \in \RR_+$, the set
    \[
        B_d(x, \veps) = \{\,y \in X \mid d(x, y) < \veps\,\}
    \]
    is called the $\veps$\textit{-ball centered at} $x$.
    Sometimes, we write $B(x, \veps)$ if no confusion arises.
}

\mlemma[smallerBall]{}{
    Let $d$ be a metric on a set $X$.
    If $y \in B(x, \veps)$, then there is some $\delta \in \RR_+$
    such that $y \in B(y, \delta) \subseteq B(x, \veps)$.
}
\pf{Proof}{
    Let $\delta = \veps - d(x, y)$. ($\delta \in \RR_+$, indeed.)
    Then, if $z \in B(y, \delta)$,
    $d(x, z) \le d(x, y) + d(y, z) < d(x, y) + (\veps - d(x, y)) = \veps$.
    Thus, $B(y, \delta) \subseteq B(x, \veps)$.
}

\dfn{Metric Topology}{
    If $d$ is a metric on the set $X$, then the topology generated by the basis
    \[
        \mcal B = \{\,B_d(x, \veps) \mid x \in X \text{ and } \veps \in \RR_+\:\}
    \]
    is called the \textit{metric topology induced by} $d$.
}

\nt{
    $\mcal B$ is actually a basis for $X$.
    The first condition can be easily check by noting that $x \in B(x, 1)$ for every $x \in X$.

    To check the second condition, let $y \in B(x_1, \veps_1) \cap B(x_2, \veps_2)$.
    Then, by \Cref{lem:smallerBall},
    there are $\delta_1, \delta_2 \in \RR_+$
    such that $B(y, \delta_1) \subseteq B(x_1, \veps_1)$
    and $B(y, \delta_2) \subseteq B(x_2, \veps_2)$.
    If we take $\delta_0 \triangleq \min \{\delta_1, \delta_2\}$,
    $y \in B(y, \delta_0) \subseteq B(x_1, \veps_1) \cap B(x_2, \veps_2)$.
}

\dfn{Metrizability and Metric Space}{
    If $X$ is a topological space, $X$ is said to be \textit{metrizable}
    if there exists a metric $d$ on $X$ that induces the topology of $X$.
    A \textit{metric space} is a metrizable space $X$ together with a specific metric $d$
    that gives the topology of $X$.
}

\dfn{Boundedness}{
    Let $(X, d)$ be a metric space.
    A subset of $A$ of $X$ is said to be \textit{bounded} if
    \[
        \exs M \in \RR,\: \forall a_1, a_2 \in A,\: d(a_1, a_2) \le M\text{.}
    \]
}

\nt{
    Boundedness is not a topological property
    as it depends on the metric.
    For instance, $\RR$ can be metrizable by two metrics:
    \[
        d_1(x, y) = |x - y| \quad\text{and}\quad d_2(x, y) = \min\{|x-y|, 1\}\text{.}
    \]
    (Both are metrics and induce the standard topology on $\RR$.)
    However, $\RR$ is not bounded with respect to $d_1$,
    but is bounded with respect to $d_2$.
}

\dfn{Diameter}{
    Let $(X, d)$ be a metric space.
    if $\varnothing \neq A \subseteq X$,
    the \textit{diameter} of $A$ is defined to be
    \[
        \diam A \triangleq \sup \{\,d(a_1, a_2) \mid a_1, a_2 \in A\,\}\text{.}
    \]
}

\thm[stdBddMetricInducesSame]{}{
    Let $(X, d)$ be a metric space.
    Define $\ol d \colon X \times X \to \RR$ by
    \[
        \ol d(x, y) = \min \{d(x, y), 1\}\text{.}
    \]
    Then $\ol d$ is a metric on $X$ that induces the same topology as $d$.
}
\pf{Proof}{
    The positive definiteness and the symmetry is direct.
    Let us check the triangle inequality.

    Take any $x, y, z \in X$.
    Since $\ol d(x, z) \le 1$ always holds,
    we get the triangle inequality
    in the case of $\ol d(x, y) \ge 1$ or $\ol d(y, z) \ge 1$.
    
    In the other case, i.e., $\ol d(x, y) < 1$ and $\ol d(y, z) < 1$,
    it holds that $\ol d(x, y) = d(x, y)$ and $\ol d(y, z) = d(y, z)$.
    This implies
    \[
        \ol d(x, z) \le d(x, z) \le d(x, y) + d(y, z) = \ol d(x, y) + \ol d(y, z)\text{,}
    \]
    which completes the proof that $\ol d$ is a metric on $X$.

    Now, note that, in any metric space,
    \[
        \{\,B_d(x, \veps) \mid x \in X \text{ and } \veps \in \RR_+\:\}
    \] and \[
        \{\,B_d(x, \veps) \mid x \in X \text{ and } \veps \in (0, 1)\:\}
    \] generates the same topology.
    Therefore, it follows that $d$ and $\ol d$ generates the same opology on $X$,
    because the collections of $\veps$-balls with $\veps < 1$ under
    these two metrics are the same.
}

\dfn{Standard Bounded Metric}{
    Let $(X, d)$ be a metric space.
    Define $\ol d \colon X \times X \to \RR$ by
    \[
        \ol d(x, y) = \min \{d(x, y), 1\}\text{.}
    \]
    Then, $\ol d$ is a metric on $X$ and is called
    the \textit{standard bounded metric corresponding to} $d$.
}

\dfn{Norm, Euclidean Metric and Square Metric}{
    Given $\vec x = (x_1, x_2, \cdots, x_n) \in \RR[n]$,
    we define the \textit{norm} of $\vec x$ by the equation.
    \[
        \|\vec x\| = (x_1^2 + x_2^2 + \cdots + x_n^2)^{1/2}\text{;}
    \]
    and we define the \textit{euclidean metric} $d$ on $\RR[n]$ by the equation
    \[
        d(\vec x, \vec y) = \|\vec x - \vec y\| = \big[(x_1-y_1)^2+\cdots+(x_n-y_n)^2\big]^{1/2}\text{.}
    \]
    We define the \textit{square metric} $\rho$ on $\RR[n]$ by the equation
    \[
        \rho(\vec x, \vec y) = \max \{\,|x_1-y_1|, \cdots, |x_n-y_n|\,\}\text{.}
    \]
}

\nt{
    The proof that $\rho$ is a metric is trivial but for the triangle inequality.

    Since, for each $i \in [n]$,
    \[
        |x_i - z_i| \le |x_i - y_i| + |y_i - z_i| \le \rho(\vec x, \vec y) + \rho(\vec y, \vec z)\text{,}
    \]
    it holds that
    \[
        \rho(\vec x, \vec z) \le \rho(\vec x, \vec y) + \rho(\vec y, \vec z)\text{.}
    \]
}

\mlemma[metricFinerIff]{}{
    Let $d$ and $d'$ be two metrics on the set $X$;
    let $\mcal T$ and $\mcal T'$ be the topologies they induce, respectively.
    Then,
    \[
        \mcal T \subseteq \mcal T' \iff
        \forall (x, \veps) \in X \times \RR_+,\: \exs \delta \in \RR_+,\:
        B_{d'}(x, \delta) \subseteq B_d(x, \veps)\text{.}
    \]
}
\pf{Proof}{
    ($\Rightarrow$)
    Take any $x \in X$ and $\veps \in \RR_+$
    Since $B_d(x, \veps)$ is a basis element of $\mcal T$,
    by \Cref{lem:finerIff}, there is a basis element $B'$ of $\mcal T'$
    such that $x \in B' \subseteq B_d(x, \veps)$.
    By \Cref{lem:smallerBall}, there is some $B_{d'}(x, \delta)$
    such that $x \in B_{d'}(x, \delta) \subseteq B'$.

    ($\Leftarrow$)
    Let $x \in X$; let $B$ be any basis element of $\mcal T$ that contains $x$.
    By \Cref{lem:smallerBall}, there is some $B_d(x, \veps)$
    such that $B_d(x, \veps) \subseteq B$.
    By supposition, there exists $\delta \in \RR_+$
    such that $x \in B_{d'}(x, \delta) \subseteq B_d(x, \veps)$.
    Thus, by \Cref{lem:finerIff}, $\mcal T'$ is finer than $\mcal T$.
}

\thm[eucliAndSqareMetricAreProduct]{}{
    The topologies on $\RR[n]$ induced by $d$ and $\rho$ are the same
    as the product topology on $\RR[n]$.
}
\pf{Proof}{
    Let $\mcal T_d$ and $\mcal T_{\rho}$ be the topologies
    induced by $d$ and $\rho$, respectively.
    Let $\mcal T_{\RR[n]}$ be the product topology on $\RR[n]$.

    ($\mcal T_d = \mcal T_{\rho}$)
    Let $\vec x = (x_1, \cdots, x_n)$ and $\vec y = (y_1, \cdots, y_n)$.
    Let $M \in [n]$ such that $|x_M - y_M| = \rho(\vec x, \vec y)$.

    Then,
    \[
        \begin{aligned}[t]
            \rho(\vec x, \vec y)^2 = |x_M - y_M|^2 &\le \sum_{i=1}^{n} (x_i - y_i)^2 = d(\vec x, \vec y)^2 \\
                                                   &\le \sum_{i=1}^{n} (x_M-y_M)^2 = n \rho(\vec x, \vec y)^2\text{;}
        \end{aligned}
    \]
    thus
    \[
        \rho(\vec x, \vec y) \le d(\vec x, \vec y) \le \sqrt{n} \rho(\vec x, \vec y)\text{.}
    \]

    Therefore, we get, for every $\vec x \in \RR[n]$ and $\veps \in \RR_+$,
    \[
        B_d(\vec x, \veps) \subseteq B_{\rho}(\vec x, \veps) \quad \text{and} \quad
        B_{\rho}(\vec x, \veps/ \sqrt{n}) \subseteq B_d(\vec x, \veps)\text{.}
    \]
    By \Cref{lem:metricFinerIff}, one is finer than the other; $\mcal T_d = \mcal T_\rho$.
    
    ($\mcal T_{\rho} = \mcal T_{\RR[n]}$)
    $\mcal T_{\rho} \subseteq \mcal T_{\RR[n]}$ is direct since
    every basis element
    \[
        B_{\rho}(\vec x, \veps)
        = (x_1 - \veps, x_1 + \veps) \times \cdots \times (x_n - \veps, x_n + \veps)
    \]
    of $\mcal T_\rho$ is a basis element of $\mcal T_{\RR[n]}$, by \Cref{lem:finerIff},
    $\mcal T_{\rho} \subseteq \mcal T_{\RR[n]}$.

    To prove the other containment, take any $\vec x \in \RR[n]$
    and let $B = \prod_{i=i}^n (a_i, b_i)$ be a basis element of $\mcal T_{\RR[n]}$
    that contains $x$.
    For each $i \in [n]$, let $\veps_i = \min \{\,x_i - a_i, b_i - x_i\,\}$.
    Then, $(x_i - \veps_i, x_i + \veps_i) \subseteq (a_i, b_i)$ for all $i \in [n]$.
    Thus, it follows that $\vec x \in B_{\rho}(\vec x, \min_{i=1}^n \veps_i) \subseteq B$;
    $\mcal T_{\RR[n]} \subseteq \mcal T_{\rho}$ by \Cref{lem:finerIff}.
}

\cor{}{
    The product topology on $\RR[n]$ is metrizable.
}

\thm[unifMetric]{}{
    Given an index set $J$ and given points $\vec x = (x_\alpha)_{\alpha \in J}$
    and $\vec y = (y_\alpha)_{\alpha \in J}$ of $\RR[J]$,
    let us define $\ol \rho \colon \RR[J] \times \RR[J] \to \RR$ by
    \[
        \ol \rho(\vec x, \vec y) = \sup \{\,\ol d(x_\alpha, y_\alpha) \mid \alpha \in J \,\}
    \]
    where $\ol d$ is the standard bounded metric on $\RR$.
    Then, $\ol \rho$ is a metric on $\RR[J]$.
}
\pf{Proof}{
    The positive definiteness and the symmetry is direct.
    Let us check the triangle inequality.

    Let $\vec x, \vec y, \vec z \in \RR[J]$.
    For each $\alpha \in J$, it holds that
    \[
        \ol d(x_\alpha, z_\alpha) \le \ol d(x_\alpha, y_\alpha) + \ol d(y_\alpha, z_\alpha)
        \le \ol \rho(\vec x, \vec y) + \ol \rho(\vec y, \vec z)\text{.}
    \]
    Therefore, $\ol \rho(\vec x, \vec z) \le \ol \rho(\vec x, \vec y) + \ol \rho(\vec y, \vec z)$.
}

\dfn{Uniform Metric and Uniform Topology}{
    Given an index set $J$, $\ol \rho$ in the \Cref{th:unifMetric}
    is called the \textit{unifrom metric} on $\RR[J]$,
    and the topology it induces is called the \textit{uniform topology}.
}

\thm{}{
    The uniform topology on $\RR[J]$ is finer than the product topology
    and coarser than the box topology.
    Moreover, they are all strict when $J$ is infinite. In other words,
    \[
        \mcal T_\text{product} \subseteq \mcal T_\text{uniform} \subseteq \mcal T_\text{box}
    \]
    They are strict if $J$ is infinite.
}
\pf{Proof}{
    ($\mcal T_\text{product} \subseteq \mcal T_\text{uniform}$)
    Let $B = \prod_{\alpha \in J} U_\alpha$ be a basis element of the product topology
    and $\vec x \in B$.
    Let $\alpha_1, \alpha_2, \cdots, \alpha_n$ be the indices
    such that $U_{\alpha_i} \neq \RR$.
    Then, for each $i \in [n]$, there exists $\veps_i \in \RR_+$
    such that $B_{\ol d}(x_{\alpha_i}, \veps_i) \subseteq U_{\alpha_i}$.
    Let $\veps \triangleq \min_{i=1}^n \veps_i$.
    Then, $B_{\ol \rho}(\vec x, \veps) \subseteq B$.
    The result follows from \Cref{lem:finerIff}. \checkmark

    ($\mcal T_\text{uniform} \subseteq \mcal T_\text{box}$)
    Let $B$ be any basis element of the uniform topoloy and $\vec x \in B$.
    Then, \Cref{lem:smallerBall} implies that there is some $\veps$-ball centered at $\vec x$
    such that $B_{\ol \rho}(\vec x, \veps) \subseteq B$.
    Then, $\prod_{\alpha \in J} (x_\alpha - \veps/2, x_\alpha + \veps/2)$
    is an open neighborhood of $\vec x$ which is contained in $B$. \checkmark

    ($\mcal T_\text{product} \nsupseteq \mcal T_\text{uniform}$ if $J$ is infinite)
    Let $0 < \veps < 1$ and $\vec x \in \RR[J]$.
    Then, $\vec x \in B_{\ol \rho}(\vec x, \veps)$ but
    there is no basis element of the product topology that is contained in $B_{\ol \rho}(\vec x, \veps)$.
    By \Cref{lem:finerIff}, the product topology is not finer than the uniform topology. \checkmark

    ($\mcal T_\text{uniform} \nsupseteq \mcal T_\text{box}$ if $J$ is infinite.)
    Let $U \triangleq \prod_{\alpha \in J} (0, 2)$,
    which is a basis element of the box topology
    There is an injective function $f \colon \ZZ_+ \hookrightarrow J$ by \Cref{th:InfSetTFAE}.
    Let $\vec x \in U$ where
    \[
        x_\alpha = \begin{cases}
            1/n & \text{if } \exs n \in \NN_+,\: f(n) = \alpha \\
            1   & \text{otherwise.}
        \end{cases}
    \]
    Then, no basis element that contains $\vec x$ can be contained in $U$.
    If otherwise, there is an $B_{\ol \rho}(\vec x, \veps') \subseteq U$ by \Cref{lem:smallerBall}.
    However, there exists $\alpha_0 \in J$ such that $f(n) = \alpha_0$
    where $n \veps' > 2$, which implies $x_{\alpha_0} = 1/n < \veps'/2$. \checkmark

    Let $\vec x' \in \RR[J]$ defined by
    \[
        x'_\alpha = \begin{cases}
            x_{\alpha_0} - \veps'/2 & \text{if } \alpha = \alpha_0 \\
            x_\alpha & \text{otherwise.}
        \end{cases}
    \]
    Then, $\vec x' \in B_{\ol \rho}(\vec x, \veps')$
    but $x'_{\alpha_0} - \veps'/2 < 0$; $\vec x' \notin U$.
    This contradicts $B_{\ol \rho}(\vec x, \veps') \subseteq U$. \checkmark
}

\thm{Countable Product of Metrizable Spaces Is Metrizable}{
    Let $X_n$ be a metric space with metric $d_n$ for each $n \in \ZZ_+$.
    Let $\ol {d}_n$ be the standard bounded metric corresponding to $d_n$.
    If $\vec x, \vec y \in \prod_{i \in \ZZ_+} X_i$, define
    \[
        D(\vec x, \vec y) = \sup \left\{\,\frac{\ol d_i(x_i, y_i)}{i} \:\bigg|\: i \in \ZZ_+\,\right\}\text{.}
    \]
    Then $D$ is a metric that induces the product topology on $\prod_{i \in \ZZ_+} X_i$.
}
\pf{Proof}{
    ($D$ is a metric on $\prod_{i \in \ZZ_+} X_i$.)
    The positive definiteness and the symmetry of $D$ is direct.
    Note that, for each $i \in \ZZ_+$,
    \[
        \frac{\ol d_i(x_i, z_i)}{i} \le \frac{\ol d_i(x_i, y_i)}{i} + \frac{\ol d_i(y_i, z_i)}{i}
        \le D(\vec x, \vec y) + D(\vec y, \vec z)\text{.}
    \]
    Thus,
    \[
        D(\vec x, \vec z) = \sup \left\{\,\frac{\ol d_i(x_i, z_i)}{i} \:\bigg|\: i \in \ZZ_+\,\right\}
        \le D(\vec x, \vec y) + D(\vec y, \vec z)\text{. \checkmark}
    \]

    ($\mcal T_\text{metric} \subseteq \mcal T_\text{product}$)
    Let $B$ be any $\veps'$-ball in the metric topology and let $\vec x \in B$.
    Then, by \Cref{lem:smallerBall}, there exists $\veps \in \RR_+$ such that
    $B_D(\vec x, \veps) \subseteq B$.
    Take $N \in \ZZ_+$ such that $\veps N > 1$.
    Let $V$ be the basis element for the product topology defined by
    \[
        V \triangleq B_{\ol d_1}(x_1, \veps) \times \cdots \times B_{\ol d_N}(x_N, \veps)
        \times X_{n+1} \times X_{n+2} \times \cdots\text{.}
    \]
    Note that, given any $\vec y \in \RR[\omega]$ and $i \ge N$,
    $\displaystyle \frac{\ol d_i(x_i, y_i)}{i} \le \frac{1}{N}$.
    Thus, when $\vec y \in V$,
    \[
        D(\vec x, \vec y) \le \max \left\{\,\frac{\ol d_1(x_1, y_1)}{1}, \frac{\ol d_2(x_2, y_2)}{2}, \cdots
        \frac{\ol d_N(x_{N}, y_{N})}{N}, \frac{1}{N}\right\} < \veps\text{.}
    \]
    Thus, $\vec x \in V \subseteq B_D(\vec x, \veps) \subseteq B$.
    Now, \Cref{lem:finerIff} tells the result. \checkmark

    ($\mcal T_\text{metric} \supseteq \mcal T_\text{product}$)
    Let $B = \prod_{i \in \ZZ_+} U_i$ be a basis element of the product topology and $\vec x \in B$.
    Let $i_1, i_2, \cdots, i_n$ be the indices such that $U_{i_k} \neq X_{i_k}$ for each $k \in [n]$.

    For each $k \in [n]$, since $U_{i_k}$ is open,
    there exists $\veps_k \in (0,1)$ such that $B_{\ol d_{i_k}}(x_{i_k}, \veps_k) \subseteq U_{i_k}$.
    Let $\veps \triangleq \min_{k=1}^n (\veps_k/i_k)$.

    Now we claim that $B_D(\vec x, \veps) \subseteq U$.
    Let $\vec y \in B_D(\vec x, \veps)$.
    Then, for all $k \in [n]$,
    \[
        \ol d_{i_k}(x_{i_k}, y_{i_k}) \le i_k \cdot D(\vec x, \vec y) < i_k \veps \le \veps_k < 1\text{.}
    \]
    It follows that $y_{i_k} \in B_{\ol d_{i_k}}(x_{i_k}, \veps_k)$; therefore $\vec y \in B$. \checkmark
}

\cor{}{
    $\RR[\omega]$ with the product topology is metrizable.
}

\section{The Metric Topology (continued)}

\thm[contiIffED]{The $\veps$-$\delta$ Definition of Continuity}{
    Let $f \colon X \to Y$; let $X$ and $Y$ be metrizable with metrics $d_X$ and $d_Y$, respectively.
    Then, $f$ is continuous if and only if
    \[
        \forall x \in X,\: \forall \veps \in \RR_+, \exs \delta \in \RR_+,\:
        \forall y \in Y,\: \big(d_X(x, y) < \delta \implies d_Y(f(x), f(y)) < \veps\big)\text{.}
    \]
}
\pf{Proof}{
    ($\Rightarrow$)
    Given $x \in X$ and $\veps \in \RR_+$,
    the set $f\inv\big(B(f(x), \veps)\big)$ is open and contains $x$.
    Thus, there is some $\delta$-ball $B(x, \delta)$ centered at $x$
    such that $x \in B(x, \delta) \subseteq f\inv\big(B(f(x), \veps)\big)$. \checkmark
    
    ($\Leftarrow$)
    Let $V$ be open in $Y$; we claim that $f\inv(V)$ is open in $X$.
    Let $x \in f\inv(V)$. Since $f(x) \in V$, there is some $\veps$-ball
    $B(f(x), \veps)$ such that $B(f(x), \veps) \subseteq V$.
    By the supposition, there is some $\delta \in \RR_+$
    such that $f\big(B(x, \delta)\big) \subseteq B(f(x), \veps)$.
    Thus, $x \in B(x, \delta) \subseteq f\inv(B(f(x), \veps)) \subseteq f\inv(V)$.
    This implies $f\inv(V)$ is open by definition. \checkmark
}

\dfn{Local Basis}{
    A space $X$ is said to have a \textit{local basis at the point} $x \in X$
    if there is a countable collection $\mcal U$ of open neighborhoods of $x$ such that
    any neighborhood $U$ of $x$ contains at least one of element of $\mcal U$.
}

\pagebreak
\dfn[1stCt]{First Countable Axiom}{
    A space $X$ satisfies the \textit{first countable axiom}
    if it has countable local basis at each point.
}

\exmp[metIs1stCt]{Every Metrizable Space IS First-Countable}{
    Any metrizable space satisfies the first countable axiom.
    For each $x \in X$, $\{\,B_d(x, 1/n)\mid n \in \ZZ_+\}$ is a countable local basis at $x$.
}

\mlemma[seqLemma]{The Sequence Lemma}{
    Let $X$ be a topological space; let $A \subseteq X$.
    If there is a sequence of points in $A$ converging to $x$, then $x \in \cl A$.
    Moreover, the converse holds if $X$ satisfies the first countable axiom.
}
\pf{Proof}{
    ($\Rightarrow$)
    Suppose $x_n \to x$ and $x_n \in A$. This means every neighborhood $U$ of $x$
    intersects $A$, so $x \in \cl A$ by \Cref{th:inClosureIffNeighCapANonempty}. \checkmark

    ($\Leftarrow$)
    Let $\{U_n\}_{n \in \ZZ_+}$ be a local basis for $x$.
    Set $B_n \triangleq \bigcap_{i=1}^n U_i$
    so that $B_1 \supseteq B_2 \supseteq \cdots$.
    Since $x \in \cl A$ and $x \in B_n$ is open,
    we may take $x_n \in A \cap B_n$.

    We want to show that $x_n \to x$.
    Take any neighborhood $U$ of $x$.
    Then, it contains $U_{n_0}$ for some $n_0 \in \ZZ_+$.
    Then, for all $n \ge n_0$, $x_n \in U_{n_0} \in U$. \checkmark
}

\mlemma[seqConvIffImgConv]{}{
    Let $X$ and $Y$ be topological spaces.
    If $f \colon X \to Y$ is continuous,
    then for every convergent sequence $x_n \to x$,
    the sequence $f(x_n)$ converges to $f(x)$.
    The converse also holds if $X$ satisfies the first countable axiom.
}
\pf{Proof}{
    ($\Rightarrow$)
    Let $V$ be a neighborhood of $f(x)$ in $Y$.
    Then, $f\inv(V)$ is a neighborhood of $x$ in $X$
    since $f$ is continuous.
    Since $x_n \to x$, there is some $n_0 \in \ZZ_+$
    such that $x_n \in f\inv(V)$ whenever $n \ge n_0$,
    i.e., $f(x_n) \in V$ whenever $n \ge n_0$. \checkmark

    ($\Leftarrow$)
    We claim that $f(\cl A) \subseteq \cl{f(A)}$ for any $A \subseteq X$,
    and thus $f$ is continuous by \Cref{th:contiTFAE}.
    Let $x \in \cl A$. Then, by \Cref{lem:seqLemma},
    there is a sequence $\{x_n\}_{n \in \ZZ_+} \subseteq A$ that converges to $x$.
    Then, by assumption, the sequence $\{f(x_n)\}_{n \in \ZZ_+}$ in $f(A)$ converges to $f(x)$.
    By \Cref{lem:seqLemma}, $f(x) \in \cl{f(A)}$. \checkmark
}

\mlemma{}{
    The addition, subtraction, and multiplication operations are continuous functions
    from $\RR \times \RR$ into $\RR$; and the quotient operation is a continuous function
    from $\RR \times (\RR \setminus \{0\})$ into $\RR$.
}

\thm{}{
    If $X$ is a topological space, and if $f, g \colon X \to \RR$ are continuous,
    then $f + g$, $f - g$, and $f \cdot g$ are continuous.
    If $g(x) \neq 0$ for all $x$, then $f/g$ is continuous.
}
\pf{Proof}{
    The map $h \colon X \to \RR \times \RR$ defined by
    \[
        h(x) = f(x) \times g(x)
    \] is continuous by \Cref{th:contiIffCoordsContiGen}.
    The function $f + g$ equals the composite of $h$ and the addition operation
    \[
        + \colon \RR \times \RR \to \RR\text{;}
    \]
    therefore $f + g$ is continuous by (iii) of \Cref{th:constructConti}.
    It is similar for $f-g$, $f \cdot g$, and $f/g$.
}

\dfn{Uniform Convergence}{
    Let $\{f_n\} \subseteq X \to Y$ be a sequence of functions
    from the set $X$ to the metric space $Y$.
    Let $d$ be the metric for $Y$.
    We say that the sequence $\{f_n\}$ converges uniformly
    to the function $f \colon X \to Y$ if
    \[
        \forall \veps \in \RR_+,\: \exs N \in \ZZ_+, \forall n\in \ZZ_+\:
        \big( n \ge N \implies \forall x \in X,\: d(f_n(x), f(x)) < \veps \big)\text{.}
    \]
}

\nt{
    Uniformity of convergence depends not only on the topology of $Y$
    but also on its metric.
}

\thm[unifLimThm]{Uniform Limit Theorem}{
    Let $\{f_n\} \subseteq X \to Y$ be a sequence of continuous functions
    from the topological space $X$ to the metric space $Y$.
    If $\{f_n\}$ converges uniformly to $f$, then $f$ is continuous.
}
\pf{Proof}{
    Let $V$ be open in $Y$. We want to show that $f\inv(V)$ is open.
    Take any $x_0 \in f\inv(V)$. Let $y_0 \triangleq f(x_0) \in V$.
    Since $f\inv(V)$ is open, there exists $\veps \in \RR_+$
    such that $B(y_0, \veps) \subseteq f\inv(V)$.
    By uniform convergence,
    \[
        \exs N \in \ZZ_+, \forall x \in X,\: d(f_N(x), f(x)) < \veps/4\text{.}
    \]
    where $d$ is the metric on $Y$.
    Moreover, since $f_N$ is continuous,
    $U = f_N\inv(B(f_N(x_0), \veps/2))$ is a neighborhood of $x_0$.
    
    Thus, for each $x \in U$,
    \[
        \begin{aligned}[t]
            d(y_0, f(x)) &\le d(f(x_0), f_N(x_0)) + d(f_N(x_0), f_N(x)) + d(f_N(x), f(x)) \\
                         &< \veps/4 + \veps/2 + \veps/4 = \veps\text{.}
        \end{aligned}
    \]
    Thus, we have $x_0 \in U \subseteq f\inv(V)$; $f\inv(V)$ is open.
}

\thm{}{
    $\{f_n\} \subseteq X \to \RR$ converges uniformly to $f \colon X \to \RR$
    if and only if $\{f_n\}$ converges to $f$ in the uniform topology.
}
\pf{Proof}{
    ($\Rightarrow$)
    Let $U$ be any neighborhood of $f$ in the uniform topology.
    Then, there is an $\veps$-ball $B_{\ol \rho}(f, \veps)$ centered at $f$ which is contained in $U$.
    By the uniform convergence, there is some $N \in \ZZ_+$ such that
    \[
        \forall n \in \ZZ_+,\: \big(n \ge N \implies \forall x \in X,\: d(f_n(x), f(x)) < \veps/2\big)\text{.}
    \]
    Thus, for all $n \ge N$, $\ol \rho(f_n, f) \le \veps/2 < \veps$,
    i.e., $f_n \in B_{\ol \rho}(f, \veps) \subseteq U$. \checkmark

    ($\Leftarrow$)
    Take any $\veps \in \RR_+$.
    By the convergence in the uniform topology,
    there exists some $N \in \ZZ_+$ such that
    \[
        \forall n \in \ZZ_+,\: \big(n \ge N \implies f_n \in B_{\ol\rho}(f, \veps)\big)\text{.}
    \]
    This implies, whenever $n \ge N$, $\forall x \in X,\: d(f_n(x), f(x)) < \veps$. \checkmark
}

\cor{}{
    $\RR[\omega]$ with the box topology is not metrizable.
}
\pf{Proof}{
    Let $A = (\RR_+)^\omega$ be a subset of $\RR[\omega]$.
    Then, $\vec 0$ is a limit point of $A$. To see this, let
    \[
        B = (a_1, b_1) \times (a_2, b_2) \times \cdots
    \]
    be any basis element that contains $\vec 0$. Then,
    \[
        (b_1/2, b_2/2, \cdots) \in A \cap B\text{.}
    \]
    However, there is no sequence of points of $A$ that converge to $\vec 0$.
    To see this, let $\{\vec a_n\}_{n \in \ZZ_+}$ be a sequence of points in $A$ where
    \[
        \vec a_n = (a_{n1}, a_{n2}, \cdots, a_{in}, \cdots)\text{.}
    \]
    Let $B' = \prod_{n \in \ZZ_+} (-a_{nn}, a_{nn})$ is a neighborhood of $\vec 0$
    but no $\vec a_n$ is in $B'$; $\{\vec a_n\}$ does not converge to $\vec 0$.
    
    Thus, by \Cref{lem:seqLemma}, $\RR[\omega]$ does not satisfy
    the first countable axiom, and thus is not metrizable.
}

\cor{}{
    $\RR[J]$ with uncountable $J$ in the product topology is not metrizable.
}
\pf{Proof}{
    Let $A = \{\,(x_{\alpha})_{\alpha \in J} \mid x_\alpha = 1
    \text{ for all but finitely many }\alpha\text{'s}\,\}$.

    Let $\prod_{\alpha \in J} U_\alpha$ be a basis that contains $\vec 0$
    and suppose $U_\alpha \neq \RR$ for $\alpha \in \{\,\alpha_1, \alpha_2, \cdots, \alpha_n\,\}$.
    Define $(y_{\alpha})_{\alpha \in J}$ by
    \[
        y_{\alpha} \triangleq \begin{cases}
            0 & \text{if } \alpha = \alpha_i \text{ for some } i \in [n] \\
            1 & \text{otherwise.}
        \end{cases}
    \]
    Then, $(y_{\alpha})_{\alpha \in J} \in A \cap \prod_{\alpha \in J} U_\alpha$.
    Hence, $\vec 0 \in \cl A$ by \Cref{th:inClosureIffNeighCapANonempty}.

    Now, we shall prove that no sequence in $A$ converges to $\vec 0$.
    Let $\{\vec a_n\}_{n \in \ZZ_+}$ be a sequence in $A$.
    For each $n \in \ZZ_+$, let
    \[
        J_n \triangleq \{\,\alpha \in J \mid (\vec a_n)_{\alpha} \neq 1\,\}\text{.}
    \]
    Since each $J_n$ is finite, and since $\bigcup_{n \in \ZZ_+} J_n$ is thus countable,
    we may take $\beta \in J \setminus \big(\bigcup_{n \in \ZZ_+} J_n\big)$.
    For such $\beta$, it is $(\vec a_n)_{\beta} \neq 1$ for all $n \in \ZZ_+$.
    This implies that $\vec a_n \notin \pi_{\beta}\inv\big((-1, 1)\big)$ for each $n \in \ZZ_+$
    while $\pi_{\beta}\inv\big((-1, 1)\big)$ is a neighborhood of $\vec 0$;
    $\{\vec a_n\}_{n \in \ZZ_+}$ does not converge to $\vec 0$.
    Thus, $\RR[J]$ is not metrizable by \Cref{lem:seqLemma}.
}

\section{The Quotient Topology}

\dfn[defQuotientMap]{Quotient Map}{
    Let $X$ and $Y$ be topological spaces.
    A map $p \colon X \to Y$ is called a \textit{quotient map} if
    \begin{enumerate}[nolistsep, label=(\roman*)]
        \ii $p$ is surjective and
        \ii $V \subseteq Y$ is open in $Y$ $\iff$ $p\inv(V)$ is open in $X$.
    \end{enumerate}
}

\nt{
    A quotient map is continuous.
}

\nt{
    (ii) of \Cref{def:defQuotientMap} is equivalent to
    \[
        C \subseteq Y \text{ is closed in } Y \iff p\inv(C) \text{ is closed in } X\text{.}
    \]
    as
    \[
        \begin{aligned}[t]
            C \text{ is closed in } Y &\iff Y \setminus C \text{ is open in }Y \quad\text{and} \\
            f\inv(C) \text{ is closed in } X &\iff X \setminus f\inv(C) \text{ is closed in }X
        \end{aligned}
    \]
}

\dfn{Saturated Set}{
    A subset $C$ of $X$ is \textit{saturated} (with respect to
    the map $p \colon X \to Y$) if
    \[
        \forall y \in Y,\: \big( p\inv(\{y\}) \cap C \neq \varnothing \implies f\inv(\{y\}) \subseteq C \big)\text{.}
    \]
    In other words, $C$ is saturated if $C = p\inv(V)$ for some $V \subseteq Y$.
}

\nt{
    Here is the proof of their equivalence.
    \begin{itemize}[nolistsep]
        \ii Suppose $C = p\inv(V)$ for some $V \subseteq Y$.
            Let $y \in Y$ and suppose it satisfies $p\inv(\{y\}) \cap C \neq \varnothing$.
            Thus, \[
                p\inv(\{y\}) \cap p\inv(V) = p\inv(V \cap \{y\}) \neq \varnothing\text{;}
            \] $y \in V$.
            Hence, $p\inv(\{y\}) \subseteq p\inv(V) = C$.
        \ii For the converse, let
            \[
                \begin{aligned}[t]
                    V &\triangleq \{\, y \in V \mid p\inv(\{y\}) \cap C \neq \varnothing\,\} \\
                      &= \{\, y \in V \mid p\inv(\{y\}) \subseteq C\,\}
                \end{aligned}
            \]
            The second equality follows from the hypothesis.

            If $p(x) \in V$ where $x \in X$,
            by definition of $V$, $x \in p\inv(p(\{x\})) = p\inv(\{p(x)\}) \subseteq C$.
            This proves $p\inv(V) \subseteq C$.
            
            For the other containment, let $x \in C$.
            Then, $\{p(x)\} \cap p(C) \neq \varnothing$,
            and thus \[
                \varnothing \neq p\inv\big(\{p(x)\} \cap p(C)\big)
                = p\inv\big(\{p(x)\}\big) \cap p\inv\big(p(C)\big)
                \subseteq p\inv\big(\{p(x)\}\big) \cap C
            \] is nonempty; $p(x) \in V$ by definition of $V$.
            This proves $C \subseteq p\inv(V)$. \qed
    \end{itemize}
}

\mlemma[quotMapIff]{}{
    Let $X$ and $Y$ be topological spaces.
    A surjective, continuous map $p \colon X \to Y$ is a quotient map if and only if
    $p(C)$ is open for every saturated open set $C \subseteq X$.
}
\pf{Proof}{
    The continuity is equivalent to $\Rightarrow$ of \Cref{def:defQuotientMap} (ii),
    and `sending every saturated open set to an open set' is equivalent to $\Leftarrow$
    of \Cref{def:defQuotientMap} (ii).
}

\mlemma[satSetFinvF]{}{
    If $p \colon X \to Y$ is a map and $A$ is saturated with respect to $p$,
    then $p\inv(p(A)) = A$.
}
\pf{Proof}{
    It is already $p\inv(p(A)) \supseteq A$ by \Cref{exmp:compFandFinvAndSurjInj}.
    
    There exists $V \subseteq Y$ such that $A = p\inv(V)$.
    Then, $p(A) = p(p\inv(V)) \subseteq V$; and it implies
    $p\inv(p(A)) \subseteq p\inv(V) = A$.
}

\mlemma[surjContiOpenThenQuotMap]{}{
    Let $X$ and $Y$ be topological spaces and $p \colon X \to Y$
    be surjective and continuous.
    Then, if $p$ is an open map or is a closed map, $p$ is a quotient map.
}
\pf{Proof}{
    If $p$ is open, then the result follows directly from \Cref{lem:quotMapIff}.

    Suppose $p$ is closed and let $V \subseteq Y$ such that $p\inv(V)$ is open in $X$.
    Then, $X \setminus p\inv(V)$ is closed,
    and thus, 
    \[
        p(X \setminus p\inv(V)) = p(X) \setminus p(p\inv(V)) = Y \setminus V
    \] is closed in $X$.
    The last equality comes from \Cref{exmp:compFandFinvAndSurjInj}.
    Thus, $V$ is open in $X$.
}

\wc{The Converses Do Not Hold}{
    Let $A = \big([0, \infty) \times \RR\big) \cup \big(\RR \times \{0\}\big)$
    be a subspace of $X = \RR^2$ endowed with the standard topology.
    Let $\pi \colon A \to \RR$ be the projection onto its first factor, i.e.,
    \[
        \pi(x \times y) = x\text{.}
    \]
    Since $\pi$ is surjective and $\pi\inv(V) = (V \times \RR) \cap A$ for each $V \subseteq \RR$,
    $\pi$ is a quotient map when $\RR$ is endowed with the standard topology.

    However, it is not open as $\pi \big((\RR \times (0, 1)) \cap A\big) = [0, \infty)$ is not open.
    It is also not closed as, if we let $C = \{\,x \times 1/x\mid x \in \RR_+\,\}$, $p(C) = (0, \infty)$
    is not closed although $C$ is closed in $A$.

    This shows that the converses of \Cref{lem:surjContiOpenThenQuotMap} are not true.
}

\wc{Subspaces and Quotient Map}{
    \centerline{\textit{
        A restriction on a subspace of a quotient map need not be a quotient map.
    }}

    Let $X$ be the subspace $[0, 1] \cup [2, 3]$ of $\RR$,
    and let $Y$ be the subspace $[0, 2]$ of $\RR$.
    Let $p \colon X \to Y$ be defined by
    \[
        p(x) = \begin{cases}
            x     & \text{if } x \in [0, 1] \\
            x - 1 & \text{if } x \in [2, 3]\text{.}
        \end{cases}
    \]

    $p$ is continuous since
    \[
        p\inv \big((a, b) \cap Y\big) = \begin{cases}
            (a, b) \cap X & \text{if } b \le 1 \\
            (a+1, b+1) \cap X & \text{if } a \ge 1 \\
            (a, b+1)\cap X & \text{if } a < 1 < b
        \end{cases}
    \]
    implies $p\inv(V)$ is open in $X$ if $V$ is open in $Y$.
    
    Also, since $\mrm{id}$ and $g \colon \RR \to \RR$
    defined by $g(x) = x - 1$ are closed (homeomorphisms, actually),
    if $C$ is closed in $X$,
    \[
        p(C) = p(C \cap [0, 1]) \cup p(C \cap [2, 3])
        = (C \cap [0, 1]) \cup g(C \cap [2, 3])
    \]
    is closed.
    
    $p$ is surjective, indeed;
    thus $p$ is a quotient map by \Cref{lem:surjContiOpenThenQuotMap}.

    Let $A$ be the subspace $[0, 1) \cup [2, 3]$.
    Then, the map $q \colon A \to Y$ obtained by restricting $p$ is continuous and surjective,
    but it is not a quotient map as $f\inv([1, 2]) = [2, 3]$ is open in $A$
    but $[1, 2]$ is not open in $Y$.
}

\thm[quotTopUnique]{}{
    If $X$ is a space and $A$ is a set and if $p \colon X \to A$ is a surjective map,
    then there exists a unique topology $\mcal T$ on $A$ relative to which $p$ is a quotient map.
    Moreover, \[
        \mcal T = \{\,V \subseteq A \mid p\inv(V) \text{ is open in }X\,\}\text{.}
    \]
}
\pf{Proof}{
    First, we shall prove that $\mcal T$ is a topology.
    \begin{enumerate}[nolistsep, label=(\roman*)]
        \ii $p\inv(\varnothing) = \varnothing$ and $p\inv(A) = X$ are open in $X$;
            thus $\varnothing, A \in \mcal T$. \checkmark
        \ii For any $\{V_\alpha\}_{\alpha \in J} \subseteq \mcal T$,
            $p\inv \big(\bigcup_{\alpha \in J} V_\alpha\big) = \bigcup_{\alpha \in J}p\inv(V_\alpha)$ is open in $X$.
            Thus, $\bigcup_{\alpha \in J} U_\alpha \in \mcal T$. \checkmark
        \ii For any $\{V_i\}_{i=1}^n \subseteq \mcal T$,
            $p\inv \big(\bigcup_{i=1}^n V_\alpha\big) = \bigcup_{i=1}^np\inv(V_i)$ is open in $X$.
            Thus, $\bigcup_{i=1}^n V_i \in \mcal T$. \checkmark
    \end{enumerate}

    $p$ is a quotient map relative to $\mcal T$.
    The surjectivity is given by definition, and the continuity is direct from the definition.
    Moreover, if $p\inv(U)$ is open in $X$ where $U \subseteq A$, by the definition of $\mcal T$,
    $U \in \mcal T$. \checkmark

    To prove the uniqueness, let $\mcal T'$ be a topology on $A$
    relative to which $p$ is a quotient map.
    Then,
    \[
        V \in \mcal T \iff p\inv(V) \text{ is open in } X \iff V \in \mcal T'\text{;}
    \]
    thus $\mcal T = \mcal T'$. \checkmark
}

\dfn{Quotient Topology}{
    Let $X$ be a space and $A$ be a set. Let $p \colon X \to A$ be a surjective map.
    Then, according to \Cref{th:quotTopUnique},
    \[
        \mcal T = \{\,V \subseteq A \mid p\inv(V) \text{ is open in }X\,\}
    \]
    is a unique topology on $A$ relative to which $p$ is a quotient map.
    Here, $\mcal T$ is called the \textit{quotient topology induced by} $p$.
}

\dfn{Quotient Space}{
    Let $X$ be a topological space,
    and let $X^\ast \subseteq \mcal P(X)$ be a partition of $X$.
    Let $p \colon X \to X^\ast$ be a function that maps each $x \in X$
    to the unique $U \in X$ such that $x \in U$.
    Then, $p$ is surjective.
    $X^\ast$ endowed with the quotient topology induced by $p$
    is called a \textit{quotient space} of $X$.
}

\nt{
    Since $U \subseteq X^\ast$ is a collection of equivalence classes,
    it is just $p\inv(U) = \bigcup U$.
}

\mlemma[lemSubspaceQuot]{}{
    Let $X$ and $Y$ be any sets, and let $p \colon X \to Y$ be a map.
    Let $A$ be a subset of $X$ that is saturated with respect to $p$.
    Let $q \colon A \to p(A)$ be the map obtained by restricting $p$.
    Then, the following hold.
    \begin{enumerate}[nolistsep, label=(\roman*)]
        \ii If $V \subseteq p(A)$, then $p\inv(V) = q\inv(V)$.
        \ii If $U \subseteq X$, then $p(U \cap A) = p(U) \cap p(A)$.
    \end{enumerate}
}
\pf{Proof}{
    \hfill
    \begin{enumerate}[nolistsep, label=(\roman*)]
        \ii It is direct that
            \[
                q\inv(V) = \{\,x \in A \mid q(x) \in V\,\}
                = \{\,x \in A \mid p(x) \in V\,\}
                \subseteq \{\, x \in X \mid p(x) \in V\,\} = p\inv(V)\text{,}
            \]
            and it does not require $A$ to be saturated.

            For the other direction, let $x \in p\inv(V)$.
            Since $A$ is saturated, $x \in p\inv(V) \subseteq p\inv(p(A)) = A$
            by \Cref{lem:satSetFinvF}.
            Thus, $x \in q\inv(V)$.
        \ii It is already $p(U \cap A) \subseteq p(U) \cap p(A)$ since
            $p(U \cap A) \subseteq p(U)$ and $p(U \cap A) \subseteq p(A)$.

            For the reverse inclusion,
            let $y \in p(U) \cap p(A)$.
            There exists $u \in U$ and $a \in A$ such that $y = p(u) = p(a)$.
            Then, $u \in p\inv(\{p(u)\}) = p\inv(\{p(a)\}) \subseteq A$
            since $A$ is saturated.
            Thus, $u \in U \cap A$; $y = p(u) \in p(U \cap A)$.
    \end{enumerate}
}

\thm[subspaceQuot]{}{
    Let $X$ and $Y$ be topological spaces, and let $p \colon X \to Y$ be a quotient map.
    Let $A$ be a subspace of $X$ that is saturated with respect to $p$.
    Let $q \colon A \to p(A)$ be the map obtained by restricting $p$.
    \begin{enumerate}[nolistsep, label=(\roman*)]
        \ii If $A$ is either open or closed in $X$, then $q$ is a quotient map.
        \ii If $p$ is either an open map or a closed map, then $q$ is a quotient map.
    \end{enumerate}
}
\pf{Proof}{
    Note that, $q$ is already surjective and continuous by \Cref{th:constructConti}.
    Let $V \subseteq p(A)$ and assume $q\inv(V)$ is open in $A$.
    $q\inv(V) = p\inv(V)$ by \Cref{lem:lemSubspaceQuot}.
    \begin{enumerate}[nolistsep, label=(\roman*)]
        \ii Suppose $A$ is open.
            Then, $q\inv(V) = p\inv(V)$, which is open in $A$, is open in $X$.
            Since $p$ is a quotient map, $V$ is open in $X$.
            Thus, $V = V \cap p(A)$ is also open in $p(A)$.
        \ii Suppose $p$ is open.
            Since $p\inv(V)$ is open in $A$,
            $p\inv(V) = U \cap A$ for some open set $U$ in $X$.
            Since $p$ is surjective,
            \[
                V = p(p\inv(V)) = p(U \cap A) = p(U) \cap p(A)\text{.}
            \]
            The last equation comes from \Cref{lem:lemSubspaceQuot}.
            Since $p(U)$ is open in $Y$, $V$ is also open in $p(A)$.
    \end{enumerate}
    Replace ``open'' with ``closed'' to get the proof for closed $A$ and closed $p$.
}

\thm[compQuotIsQuot]{}{
    Let $X$, $Y$, and $Z$ be topological spaces, and let
    $p \colon X \to Y$ and $q \colon Y \to Z$ be quotient maps.
    Then, $q \circ p \colon X \to Z$ is a quotient map.
}
\pf{Proof}{
    $q \circ p$ is surjective and continuous by \Cref{th:constructConti}.
    Also, if $(q\circ p)\inv(V)$ is open in $X$,
    since $(q\circ p)\inv(V) = p\inv(q\inv(V))$,
    $q\inv(V)$ is open, and thus $V$ is open.
}

\wc{Products and Quotient Map}{
    \centerline{\textit{
        The product of two quotient maps need not be a quotient map.
    }}
    
    Let $X = \RR$ and $X^\ast$ be obtained by
    \[
        X^\ast = \big\{\,\{x\} \:\big|\: x \in \RR \setminus \ZZ_+\,\big\} \cup \{\ZZ_+\}\text{,}
    \]
    i.e., identifying $\ZZ_+$ to one point $b = \ZZ_+$.
    Let $p \colon X \to X^\ast$ be the quotient map.
    Let $\QQ$ be the subspace of $\RR$ endowed with the standard topology;
    let $i \colon \QQ \to \QQ$ be the identity map.
    We show that
    \[
        p \times i \colon X \times \QQ \to X^\ast \times \QQ
    \]
    it not a quotient map.
    
    Let $c_n = \sqrt{2}/n$ where $n \in \ZZ_+$.
    For each $n \in \ZZ_+$, let
    \[
        U_n \triangleq \left\{\,(x, q) \in X \times \QQ \:\big|\:
        |x - n| < 1/4 \text{ and } |q-c_n| > |x-n| \,\right\}\text{.}
    \]
    \begin{center}
    \begin{tikzpicture}
        \footnotesize
        \begin{axis}[
            scale = 1,
            x=2cm, y=2cm,
            xmin=-.5, xmax=6.7,
            ymin=-.5, ymax=2,
            xlabel={$X$},
            ylabel={$\QQ$},
            minor tick num=1,
            % ymajorticks=false,
            yminorticks=false,
            xtick distance=1,
            ytick distance=.5,
            clip=true
            ]
        \foreach \n in {1,...,6}{
            \addplot[draw=none,name path=axis1, domain=\n-.25:\n+.25, mark=none] {-.5};
            \addplot[draw=none,name path=axis2, domain=\n-.25:\n+.25, mark=none] {2};
            \addplot[name path=f1,domain=\n-.25:\n+.25,black, dashed] {-abs(x-\n)+sqrt(2)/(\n)};
            \addplot[name path=f2,domain=\n-.25:\n+.25,black, dashed] {abs(x-\n)+sqrt(2)/(\n)};
            \addplot[black, dashed] coordinates {({(\n)-.25},{sqrt(2)/(\n)+.25}) ({(\n)-.25},2)};
            \addplot[black, dashed] coordinates {({(\n)+.25},{sqrt(2)/(\n)+.25}) ({(\n)+.25},2)};
            \addplot[black, dashed] coordinates {({(\n)-.25},{sqrt(2)/(\n)-.25}) ({(\n)-.25},-.5)};
            \addplot[black, dashed] coordinates {({(\n)+.25},{sqrt(2)/(\n)-.25}) ({(\n)+.25},-.5)};
            \addplot[fill=gray, fill opacity=0.2]
                fill between[
                    of=f1 and axis1,
                ];
            \addplot[fill=gray, fill opacity=0.2]
                fill between[
                    of=f2 and axis2,
                ];
            \ifthenelse{\n<4}{
                \edef\temp{\noexpand\node at (axis cs:\n,{sqrt(2)/(\n)-.25}) {$U_\n$};}\temp
            }{
                \edef\temp{\noexpand\node at (axis cs:\n,{sqrt(2)/(\n)+.25}) {$U_\n$};}\temp
            }

        }
        \addplot[domain=.1:6.7,black] {sqrt(2)/x};
        \node at (axis cs:.5,1.41) {$q=\sqrt{2}/x$};
        \end{axis}
    \end{tikzpicture}
    \end{center}
    Then, it is easy to see that each $U_n$ is open; so
    \[
        U \triangleq \bigcup_{n \in \ZZ_+} U_n
    \]
    is open.
    Moreover, $U$ is saturated with respect to $p \times i$ as $\ZZ_+ \times \{q\} \subseteq U$
    (a potential source that makes $U$ not saturated) for all $q \in \QQ$.

    Suppose $U' \triangleq (p \times i)(U)$ is open for the sake of contradiction.
    Since $\ZZ_+ \times \{0\} \subseteq U$, $b \times 0 \in U'$ by definition.
    Hence, $U'$ contains an open set $W \times I_\delta$
    where $W$ is a neighborhood of $b$ in $X^\ast$
    and $I_\delta = (-\delta, \delta) \cap \QQ$.
    Then, we have
    \[
        p\inv(W) \times I_{\delta} = (p \times i)\inv(W \times I_\delta)
        \subseteq (p \times i)\inv(U') = U\text{.}
    \]
    (The last equation follows from \Cref{lem:satSetFinvF}.)

    There exists $N \in \ZZ_+$ such that $c_N = \sqrt{2}/N < \delta$.
    Since $p$ is continuous,
    $p\inv(W)$ is open in $X$ and contains $\ZZ_+$.
    Thus, there exists $\veps \in (0, 1/4)$ so that $(N - \veps, N + \veps) \subseteq p\inv(W)$.
    This implies
    \[
        (N - \veps, N + \veps) \times I_\delta \subseteq U\text{,}
    \]
    but this is impossible since,
    if we let $c_N' \in (c_N - \veps/2, c_N + \veps/2) \cap I_\delta$,
    \[
        (N + \veps/2) \times c_N' \in (N - \veps, N + \veps) \times I_\delta
    \]
    but $(N + \veps/2) \times c_N' \notin U$, \#.
    Thus, $U' = (p \times i)(U)$ is not open while $U$ is saturated;
    $p \times i$ is not a quotient map.
    \begin{center}
    \begin{tikzpicture}
        \footnotesize
        \begin{axis}[
            scale = 1,
            x=2cm, y=2cm,
            xmin=-.5, xmax=4,
            ymin=-1, ymax=1.5,
            xlabel={$X$},
            ylabel={$\QQ$},
            ymajorticks=false,
            yminorticks=false,
            xmajorticks=false,
            xminorticks=false,
            clip=true
            ]
        \foreach \n in {1, 3}{
            \addplot[draw=none,name path=axis1, domain=\n-.25:\n+.25, mark=none] {-1};
            \addplot[draw=none,name path=axis2, domain=\n-.25:\n+.25, mark=none] {1.5};
            \addplot[name path=f1,domain=\n-.25:\n+.25,lightgray, dashed] {-abs(x-\n)+sqrt(2)/(3*\n)};
            \addplot[name path=f2,domain=\n-.25:\n+.25,lightgray, dashed] {abs(x-\n)+sqrt(2)/(3*\n)};
            \addplot[lightgray, dashed] coordinates {({(\n)-.25},{sqrt(2)/(3*\n)+.25}) ({(\n)-.25},1.5)};
            \addplot[lightgray, dashed] coordinates {({(\n)+.25},{sqrt(2)/(3*\n)+.25}) ({(\n)+.25},1.5)};
            \addplot[lightgray, dashed] coordinates {({(\n)-.25},{sqrt(2)/(3*\n)-.25}) ({(\n)-.25},-1)};
            \addplot[lightgray, dashed] coordinates {({(\n)+.25},{sqrt(2)/(3*\n)-.25}) ({(\n)+.25},-1)};
            \addplot[fill=lightgray, fill opacity=0.2]
                fill between[
                    of=f1 and axis1,
            ];
            \addplot[fill=lightgray, fill opacity=0.2]
                fill between[
                    of=f2 and axis2,
            ];
        }
        \foreach \n in {2}{
            \addplot[draw=none,name path=axis1, domain=\n-.25:\n+.25, mark=none] {-1};
            \addplot[draw=none,name path=axis2, domain=\n-.25:\n+.25, mark=none] {1.5};
            \addplot[name path=f1,domain=\n-.25:\n+.25,black, dashed] {-abs(x-\n)+sqrt(2)/(3*\n)};
            \addplot[name path=f2,domain=\n-.25:\n+.25,black, dashed] {abs(x-\n)+sqrt(2)/(3*\n)};
            \addplot[black, dashed] coordinates {({(\n)-.25},{sqrt(2)/(3*\n)+.25}) ({(\n)-.25},1.5)};
            \addplot[black, dashed] coordinates {({(\n)+.25},{sqrt(2)/(3*\n)+.25}) ({(\n)+.25},1.5)};
            \addplot[black, dashed] coordinates {({(\n)-.25},{sqrt(2)/(3*\n)-.25}) ({(\n)-.25},-1)};
            \addplot[black, dashed] coordinates {({(\n)+.25},{sqrt(2)/(3*\n)-.25}) ({(\n)+.25},-1)};
            \addplot[fill=gray, fill opacity=0.2]
                fill between[
                    of=f1 and axis1,
            ];
            \addplot[fill=gray, fill opacity=0.2]
                fill between[
                    of=f2 and axis2,
            ];
        }
        \node[lightgray] at (axis cs:1,1) {$U_{N-1}$};
        \node at (axis cs:2,1) {$U_{N}$};
        \node[lightgray] at (axis cs:3,1) {$U_{N+1}$};
        \addplot[black, dotted] coordinates {(-.5, .4) (4, .4)};
        \addplot[black, dotted] coordinates {(-.5, -.4) (4, -.4)};
        \node at (axis cs:-.1,.5) {$\delta$};
        \node[blue] at (axis cs:2.7,{sqrt(2)/6}) {\scriptsize $(N-\veps, N+\veps) \times I_\delta$};
        \path[fill=blue, fill opacity=.2] (1.875,-.4) rectangle (2.125,.4);
        \draw[blue, thin, dashed] (1.875,-.4) rectangle (2.125,.4);
        \end{axis}
    \end{tikzpicture}
    \end{center}
}

\thm[quotInduceFtn]{}{
    Let $p \colon X \to Y$ be quotient map.
    Let $Z$ be a space and let $g \colon X \to Z$ be a map
    that is constant on each set $p\inv(\{y\})$, $y \in Y$.
    Then, $g$ induces a map $f \colon Y \to Z$ such that $f \circ p = g$.
    Moreover, the following hold.
    \begin{enumerate}[nolistsep, label=(\roman*)]
        \ii $f$ is continuous if and only if $g$ is continuous.
        \ii $f$ is a quotient map if and only if $g$ is a quotient map.
    \end{enumerate}
}
\pf{Proof}{
    For each $y \in Y$, the set $g(p\inv(\{y\}))$ is a one-point set in $Z$
    as we assumed $g$ is constant on $p\inv(\{y\})$.
    Define $f(y)$ to be the only element of it.
    Then, $f(p(x))$ is the only elemnt of $A = g\big(p\inv(p(\{x\}))\big)$
    while $g(x) \in A$. Thus, $f(p(x)) = g(x)$ for each $x \in X$; $f \circ p = g$.

    \begin{enumerate}[nolistsep, label=(\roman*)]
        \ii If $f$ is continuous, $g = f \circ p$ is continuous by \Cref{th:constructConti}.
            Suppose $g$ is continuous.
            Let $V$ be open in $Z$.
            Then, $g\inv(V)$ is open in $X$ as $g$ is continuous.
            Noting that $g\inv(V) = p\inv(f\inv(V))$ and $p$ is a quotient map,
            we get $f\inv(V)$ is also open in $Y$. \checkmark
        \ii If $f$ is a quotient map, $g = f \circ p$ is a quotient map by \Cref{th:compQuotIsQuot}.
            Suppose $g$ is a quotient map.
            $f$ is already surjective by basic set theory and continuous by (i).
            Let $V$ be open in $Z$ and suppose $f\inv(V)$ is open in $Y$.
            $p\inv(f\inv(V)) = g\inv(V)$ is open since $p$ is continuous.
            Because $g$ is a quotient map, $V$ is open. Thus, $f$ is a quotient map.
    \end{enumerate}
}

\cor[quotInduceHomeo]{}{
    Let $g \colon X \to Z$ be a surjective contiuous map.
    Let $X^\ast$ be defined by
    \[
        X^\ast \triangleq \{\,g\inv(\{z\}) \subseteq X \mid z \in Z\,\}\text{.}
    \]
    Give $X^\ast$ the quotient topology. Then, the following hold.
    \begin{enumerate}[nolistsep, label=(\roman*)]
        \ii The map $g$ induces a bijective continuous map $f \colon X^\ast \to Z$,
            which is a homeomorphism if and only if $g$ is a quotient map.
        \ii If $Z$ is Hausdorff, so is $X^\ast$.
    \end{enumerate}
}
\pf{Proof}{
    \hfill
    \begin{enumerate}[nolistsep, label=(\roman*)]
        \ii Let $p \colon X \to X^\ast$ be the quotient map
            that induces the quotient topology on $X^\ast$.
            Then, by \Cref{th:quotInduceFtn},
            the induced $f \colon X^\ast \to Z$ is continuous.
            $f$ is surjective since $g$ and $p$ are surjective.
            $f$ is injective since $f(g\inv(\{z\})) = z$ for each $z \in Z$. \checkmark

            Suppose $f$ is a homeomorphism.
            Then both $f$ and $p$ are quotient maps; thus $g = f \circ p$ is a quotient map.
            Suppose $g$ is a quotient map.
            Then, by \Cref{th:quotInduceFtn}, $f$ is a quotient map.
            Since $f$ is already bijective, $f$ is a homeomorphism. \checkmark

        \ii Suppose $Z$ is Hausdorff.
            Given distinct points $a, b \in X^\ast$,
            $f(a) \neq f(b)$ since $f$ is injective.
            Thus, there are disjoint neighborhoods $U$ and $V$ in $Z$
            of $f(a)$ and $f(b)$, respectively.
            Then, $f\inv(U)$ and $f\inv(V)$ are disjoint neighborhoods of
            $a$ and $b$ as $f$ is continuous. Thus, $X^\ast$ is Hausdorff. \checkmark
    \end{enumerate}
}

\end{document}
