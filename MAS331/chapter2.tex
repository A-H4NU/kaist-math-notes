\documentclass[MAS331_Note.tex]{subfiles}

\begin{document}
\chapter{Topological Spaces and Continuous Functions}
\section{Topological Spaces}
\dfn[defTop]{Topology and Topological Space}{
	A \textit{topology} on a set $X$ is a collection $\mcal{T}$ of subsets of $X$ such that
	\begin{enumerate}[label=(\roman*)]
		\ii $\varnothing, X \in \mcal{T}$
		\ii $\{\, U_i \mid i \in J \,\} \subseteq \mcal T \implies \bigcup_{i \in J} U_i \in \mcal T$
		\ii $\{\, U_1, U_2, \cdots, U_n \,\} \subseteq \mcal T \implies \bigcap_{i=1}^n U_i \in \mcal T$
	\end{enumerate}
	We say $(X, \mcal T)$ is a \textit{topological space}, and each element $U \in \mcal T$ is called an \textit{open set}.
}
\exmp{Discrete Topology and Trivial Topology}{
	\begin{itemize}[nolistsep]
		\ii If $X$ is any set, the collection of all subsets of $X$, $\mcal P(X)$, is a topology on $X$;
            it is called the \textit{discrete topology}.
		\ii $\{\,\varnothing, X\,\}$ is also an topology on $X$; we shall call it the \textit{trivial topology}.
	\end{itemize}
}
\exmp{Finite Complement Topology}{
	Let $X$ be any set.
	Then, $\mcal T \coloneqq \{\, U \subseteq X \mid X \setminus U \text{ is finite}\,\} \cup \{\varnothing\}$ is a topology.
	\begin{enumerate}[label=(\roman*)]
		\ii $\varnothing, X \in \mcal{T}$ \checkmark
		\ii If $\{U_\alpha\}_{\alpha \in J} \subseteq \mcal T$,
            then $X \setminus \bigcup_{\alpha \in J} U_\alpha = \bigcap_{\alpha \in J} (X - U_\alpha)$ is finite. \checkmark
		\ii If $\{\,U_1, U_2, \cdots, U_n\,\} \subseteq \mcal T$,
            $X \setminus \bigcap_{i=1}^n U_\alpha = \bigcup_{i=1}^n (X \setminus U_\alpha)$ is finite by \Cref{exer:finUandCOfFinSetIsFin}. \checkmark
	\end{enumerate}
	The topology is called the \textit{finite complement topology}.
}
\exmp{}{
	If $X = \{\,a,b,c\,\}$, then $\mcal T = \{\,\varnothing, X, \{a\}, \{a, b\}\,\}$ is a topology on $X$.
}

\dfn{Finer and Coarser Topology}{
	Let $\mcal T$ and $\mcal T'$ be topologies of a set $X$. If $\mcal T \subseteq \mcal T'$, then we say
	\begin{itemize}
		\ii $\mcal T'$ is \textit{finer} than $\mcal T$ and
		\ii $\mcal T$ is \textit{coarser} than $\mcal T'$.
	\end{itemize}
	Also, $\mcal T$ is \textit{comparable} to $\mcal T'$ if either $\mcal T \supseteq \mcal T'$ or $\mcal T \subseteq \mcal T'$.
}

\section{Basis for a Topology}
\dfn[defBasis]{Basis and Toplogy Generated by a Basis}{
	A \textit{basis} for $X$ is a collection $\mcal B$ of subsets of $X$ such that:
	\begin{enumerate}[label=(\roman*)]
		\ii $\forall x \in X,\: \exs B \in \mcal B,\: x \in B$ (i.e., $X = \bigcup \mcal B$) and
		\ii $\forall B_1, B_2 \in \mcal B,\: \big(x \in B_1 \cap B_2 \implies \exs B_3 \in \mcal B,\: x \in B_3 \subseteq B_1 \cap B_2 \big)$.
	\end{enumerate}
	The topology $\mcal T$ generated by $\mcal B$ is the collection defined by \[
		\mcal T \coloneqq \{\, U \subseteq X \mid \forall x \in U,\: \exs B \in \mcal B,\: x \in B \subseteq U \,\}\text{.}
	\]
}
\nt{If $\mcal B$ is a basis for $X$ and $\mcal T$ is the topology generated by $\mcal B$, then $\mcal B \subseteq \mcal T$.}

\mlemma[topByBIsTop]{}{
	If $\mcal T$ is the topology generated by basis $\mcal B$ for $X$, then $\mcal T$ is a topology on $X$.
}
\pf{Proof}{
	$ $\\[-1em]
	\begin{enumerate}[label=(\roman*)]
		\ii $\varnothing \in \mcal T$ by vacuous truth, and $X \in \mcal T$ follows directly from (i) in \Cref{def:defBasis}. \checkmark
		\ii Let $\mcal U \coloneqq \{U_\alpha\}_{\alpha \in J} \subseteq \mcal T$.
            Then, $x \in \mcal \bigcup\, \mcal U$ implies $\exs \alpha \in J,\: x \in U_\alpha$.
            Since $U_\alpha \in \mcal T$, there is $B \in \mcal B$ such that $x \in B \subseteq U_\alpha \subseteq \bigcup \mcal U$.
            This means $\bigcup \mcal U \subseteq \mcal T$. \checkmark
		\ii It is enough to prove it for two sets $U_1$ and $U_2$ in $\mcal T$.
            Let $x \in U_1 \cap U_2$. (If $U_1 \cap U_2 = \varnothing$, then it is done.)
            By the definition of $\mcal T$, there are $B_1$ and $B_2$ in $\mcal B$ such that
		$x \in B_1 \subseteq U_1$ and $x \in B_2 \subseteq U_2$.
            Since $x \in B_1 \cap B_2$, there is $B_3 \in \mcal B$ such that $x \in B_3 \subseteq B_1 \cap B_2 \subseteq U_1 \cap U_2$.
            Thus, it implies $U_1 \cap U_2 \in \mcal T$. \checkmark
	\end{enumerate}
}

\mlemma[topByBIsUnionsOfB]{}{
	If $\mcal T$ is the topology generated by basis $\mcal B$ for $X$,
	then $\mcal T$ is the collection of all unions of elements of $\mcal B$.
	In other words, $\mcal T = \left\{\, \bigcup \mcal U \:\big|\: \mcal U \subseteq \mcal B \,\right\}$.
}
\pf{Proof}{
	Let $\mcal T' \coloneqq \left\{\, \bigcup \mcal U \:\big|\: \mcal U \subseteq \mcal B \,\right\}$.
	Since $\mcal B \subseteq \mcal T$ and $\mcal T$ is a topology by \Cref{lem:topByBIsTop},
	$\mcal T' \subseteq \mcal T$ follows. (See (ii) in \Cref{def:defTop}.)
	Now, we shall prove $\mcal T \subseteq \mcal T'$.

	Take any $U \in \mcal T$. Then, for each $x \in U$, there is $B_x \in \mcal B$ such that $x \in B_x \subseteq U$.
	Then, $U = \bigcup_{x \in U} B_x \in \mcal T'$, hence $\mcal T \subseteq \mcal T'$.
}

\mlemma[topToBasis]{}{
	Let $(X, \mcal T)$ be a topological space.
	If $\mcal C$ is a subset of $\mcal T$ such that \[
		\forall U \in \mcal T,\: (x \in U \implies \exs C \in \mcal C,\: x \in C \subseteq U)\text{,}
	\] then $\mcal C$ is a basis for $X$ and $\mcal T$ is the topology generated by $\mcal C$.
}
\pf{Proof}{
	We shall prove first $\mcal C$ is a basis for $X$.
	\begin{enumerate}[nolistsep, label=(\roman*)]
		\ii Since $X \in \mcal T$, $\forall x \in X,\: \exs C \in \mcal C,\: x \in C$. \checkmark
		\ii Let $C_1, C_2 \in \mcal C$ and suppose $x \in C_1 \cap C_2$.
		Since $C_1 \cap C_2 \in \mcal T$, there is $C_3 \in \mcal C$ such that $x \in C_3 \subseteq C_1 \cap C_2$. \checkmark
	\end{enumerate}
	Now let $\mcal T'$ be the topology generated by $\mcal C$. We want to show $\mcal T = \mcal T'$.

	For $\mcal T' \subseteq \mcal T$, take any $U \in \mcal T'$.
	Then, by \Cref{lem:topByBIsUnionsOfB}, $U = \bigcup_{\alpha \in J} C_\alpha$ where each $C_\alpha$ is in $\mcal C$.
	Now, $U = \bigcup_{\alpha \in J} C_\alpha \in \mcal T$ directly follows.
	The last inclusion is due to (ii) in \Cref{def:defTop} and $\mcal C \subseteq \mcal T$. \checkmark

	For $\mcal T \subseteq \mcal T'$, take any $U \in \mcal T$.
	Then, for any $x \in U$, there is $C \in \mcal C$ such that $x \in C \subseteq U$, therefore $U \in \mcal T'$
	by \Cref{def:defBasis}.
}

\mlemma[finerIff]{}{
	Let $\mcal T$ and $\mcal T'$ are topologies genereated by bases $\mcal B$ and $\mcal B'$, respectively. Then, \[
		\mcal T \subseteq \mcal T' \iff \forall B \in \mcal B,\: \big(x \in B \implies \exs B' \in \mcal B',\: x \in B' \subseteq B\big)\text{.}
	\]
}
\pf{Proof}{
	($\Leftarrow$) Take any $U \in \mcal T$ and $x \in U$. Since $\mcal B$ generates $\mcal T$,
	there is $B \in \mcal B$ such that $x \in B \subseteq U$.
	By the supposition, there is $B' \in \mcal B'$ such that $x \in B' \subseteq B \subseteq U$.
	This implies we can find $B' \in \mcal B'$ such that $x \in B' \subseteq U$, by definition, $U \in \mcal T'$. \checkmark

	($\Rightarrow$) Take any $B \in \mcal B$ and $x \in B$. Since $B \in \mcal T \subseteq \mcal T'$,
	by definition of $\mcal T'$, there is $B' \in \mcal B'$ such that $x \in B' \subseteq B$. \checkmark
}

\exmp{}{
	Let $\mcal B$ be a set of open region inside a disk, and $\mcal B'$ be a set of open region inside a rectangle.
	They are bases for $\RR[2]$, and topologies generated by them are the same by \Cref{lem:finerIff}.
}

\dfn{Common Topologies on $\RR$}{
	Define
	\begin{itemize}[nolistsep, label=$-$]
		\ii $\mcal B_{\RR} \coloneqq \{\, (a, b) \subseteq \RR \mid a < b\,\}$
		\ii $\mcal B_\ell \coloneqq \{\, [a, b) \subseteq \RR \mid a < b\,\}$
	\end{itemize}
	$\mcal B$ and $\mcal B'$ are bases for $\RR$. Then,
	\begin{itemize}[nolistsep]
		\ii $\mcal T_{\RR}$, the topology generated by $\mcal B$, is called the \textit{standard topology} on $\RR$, and
		\ii $\mcal T_\ell$, the topology generated by $\mcal B_\ell$, is called the \textit{lower limit topology} on $\RR$.
	\end{itemize}
	Let $K \coloneqq \{\, 1/n \mid n \in \ZZ_+ \,\}$ and $\mcal B_K \coloneqq \mcal B_{\RR} \cup \{\, (a, b) \setminus K \mid a < b \,\}$
	Then, $\mcal B''$ is a basis for $\RR$ and
	\begin{itemize}[nolistsep]
		\ii $\mcal T_K$, the topology generated by $\mcal B_K$, is called the \textit{K-topology} on $\RR$.
	\end{itemize}
}

\mlemma{Comparison Among the Common Topologies on $\RR$}{
	The following holds.
	\begin{enumerate}[noitemsep, label=(\roman*)]
		\ii $\mcal T_{\RR} \subsetneq \mcal T_\ell$ ($\mcal T_\ell$ is strictly finer than $\mcal T_{\RR}$.)
		\ii $\mcal T_{\RR} \subsetneq \mcal T_K$ ($\mcal T_K$ is strictly finer than $\mcal T_{\RR}$.)
		\ii $\mcal T_\ell$ and $\mcal T_K$ are not comparable.
	\end{enumerate}
}
\pf{Proof}{
	$ $\\[-1em]
	\begin{enumerate}[noitemsep, label=(\roman*)]
		\ii For any $(a, b) \in \mcal B_{\RR}$ and $x \in (a, b)$, $[x, b) \in \mcal B_\ell$ and $x \in [x, b) \subseteq (a, b)$.
            Therefore, by \Cref{lem:finerIff}, $\mcal T_{\RR} \subseteq \mcal T_\ell$. \checkmark
            \par Take any $a \in \RR$. $a$ is in the interval $[a, b) \in \mcal B_\ell$
            but there are no open interval $(c, d) \in \mcal B_{\RR}$ such that $a \in (c, d) \subseteq [a, b)$.
            Therefore, by \Cref{lem:finerIff}, $\mcal T_\ell \not\subseteq \mcal T_{\RR}$. \checkmark
		\ii $\mcal T_{\RR} \subseteq \mcal T_K$ directly follows from $\mcal B_{\RR} \subseteq \mcal B_K$. \checkmark
            \par Although $0 \in (-1, 1) \setminus K \in \mcal T_K$,
            there is no $(c, d) \in \mcal B_{\RR}$ such that $0 \in (c, d) \in (-1, 1) \setminus K$.
            Therefore, by \Cref{lem:finerIff}, $\mcal T_K \not\subseteq \mcal T_{\RR}$. \checkmark
		\ii The logics in (i) and (ii) can directly imported to prove (iii). \checkmark
	\end{enumerate}
}

\dfn{Subbasis}{
	A \textit{subbasis} $\mcal S$ for $X$ is a subset of $\mcal P(X)$ whose union is $X$, i.e., $\bigcup \mcal S = X$.

	The \textit{topology generated by the subbasis} $\mcal S$
	is defined to be the collection of all unions of finite intersections of elements of $\mcal S$.
}

\mlemma[topBySIsTop]{}{
	Let $\mcal S$ be a subbasis for $X$. Then, the topology generated by $\mcal S$ is a topology on $X$.
}
\pf{Proof}{
	By \Cref{lem:topByBIsUnionsOfB}, it is enough to show that
	$\mcal B \coloneqq \big\{\, \bigcap_{i=1}^n S_i \mid S_i \in \mcal S \,\big\}$ is a basis.

	\begin{enumerate}[nolistsep, label=(\roman*)]
		\ii Since $\mcal S \subseteq \mcal B$, $X = \bigcup \mcal S \subseteq \bigcup \mcal B \subseteq X$. \checkmark
        \ii Let $B_1, B_2 \in \mcal B$ and $x \in B_1 \cap B_2$. Then,
            $B_1 = \bigcap_{i=1}^n S_i$ and $B_2 = \bigcap_{i=1}^m S_i'$ where $S_i, S_i' \in \mcal S$.
            Then, $B_1 \cap B_2 = \big(\bigcap_{i=1}^n S_i\big) \cap \big(\bigcap_{i=1}^m S_i'\big) \in \mcal B$. \checkmark
	\end{enumerate}
}

\section{The Order Topology}

\dfn{Intervals}{
    Let $X$ be a set with an order $<$ and $a, b \in X$ with $a < b$ are given.
    \begin{itemize}[nolistsep]
        \ii $(a, b) \coloneqq \{\, x \in X \mid a < x < b \,\}$ (open interval)
        \ii $[a, b) \coloneqq \{\, x \in X \mid a \le x < b \,\}$ (half-open interval)
        \ii $(a, b] \coloneqq \{\, x \in X \mid a < x \le b \,\}$ (half-open interval)
        \ii $[a, b] \coloneqq \{\, x \in X \mid a \le x \le b \,\}$ (closed interval)
    \end{itemize}
}

\dfn{Order Topology}{
    Let $X$ has more than one element. Let $\mcal B$ be collection of
    \begin{itemize}[nolistsep]
        \ii all open intervals $(a, b)$ in $X$,
        \ii all half-open intervals $[a_0, b)$ where $a_0$ is the smallest element (if $a_0$ exists), and
        \ii all half-open intervals $(a, b_0]$ where $b_0$ is the largest element (if $b_0$ exists).
    \end{itemize}
    Then, $\mcal B$ is a basis and the topology generate by $\mcal B$ is called the \textit{order topology}.
}

\mlemma{}{
    The set $\mcal B$ above is a basis.
}
\pf{Proof}{
    $ $\\[-1em]
    \begin{enumerate}[nolistsep, label=(\roman*)]
        \ii Take any $x \in X$.
            \begin{itemize}[nolistsep]
                \ii If $x$ is the smallest, then $x \in [x, b)$ where $b$ is some element in $X \setminus \{x\}$.
                \ii If $x$ is the largest, then $x \in (a, x]$ where $a$ is some element in $X \setminus \{x\}$.
                \ii Otherwise, there are some $a, b \in X \setminus \{x\}$ such that $a < x < b$ so $x \in (a, b)$. \checkmark
            \end{itemize}
        \ii A nonempty intersection of two basis with different types of interval is an open interval.
            An intersection of two basis with the same type of interval still belongs to the type of interval. \checkmark
    \end{enumerate}
}

\exmp{}{
    The order topology on $\ZZ_+$ is the discrete topology.
    $n \in (n-1, n+1) = \{n\}$ if $n > 1$ and $1 \in [1, 2) = \{1\}$.
}

\exmp{}{
    The order topology on $\RR$ is the standard topology on $\RR$.
}

\dfn{Ray}{
    Let $X$ be an order set and $a \in X$. There are four types of rays.
    \begin{itemize}[nolistsep]
        \ii $(a, \infty) \coloneqq \{\, x \in X \mid x > a \,\}$ (open ray)
        \ii $(-\infty, a) \coloneqq \{\, x \in X \mid x < a \,\}$ (open ray)
        \ii $[a, \infty) \coloneqq \{\, x \in X \mid x \ge a \,\}$ (closed ray)
        \ii $(-\infty, a] \coloneqq \{\, x \in X \mid x \le a \,\}$ (closed ray)
    \end{itemize}
}
\nt{
    Open rays are open in the order topology.
    \begin{itemize}[nolistsep]
        \ii If $X$ has a largest element $b_0$, then $(a, \infty) = (a, b_0]$.
        \ii Otherwise, $(a, \infty) = \bigcup_{a < b} (a, b)$.
    \end{itemize}
    Thus, $(a, \infty)$ is open. Similarly, $(-\infty, a)$ is open.
}
\nt{
    Open rays form a subbasis that generates the order topology.
}

\section{The Product Topology on $X \times Y$}
\dfn{Product Topology}{
    Let $X$, $Y$ be topological spaces.
    The \textit{product topology} on $X \times Y$ is the toplogy generated by a basis \[
        \mcal B \coloneqq \{\, U \times V \mid U \subseteq X \text{ and } V \subseteq Y \text{ are open} \,\}\text{.}
    \]
}

\thm[prodBasis]{}{
    Let $\mcal B$ be a basis for $X$ nd $\mcal C$ be a basis for $Y$. Then \[
        \mcal D \coloneqq \{\, B \times C \mid B \in \mcal B \text{ and } C \in \mcal C \,\}
    \] is a basis for the product topology of $X \times Y$.
}
\pf{Proof}{
    We will exploit \Cref{lem:topToBasis}.
    Take any open set $W \subseteq X \times Y$ and $x \times y \in W$.
    Then, there is a basis element $U \times V$ of the product topology $X \times Y$
    such that $x \times y \in U \times V \subseteq W$.
    Since $U$ and $V$ are open in $X$ and $Y$, respectively, and $x \in U$ and $y \in V$,
    there are $B \in \mcal B$ and $C \in \mcal C$ such that $x \in B \subseteq U$ and $y \in C \subseteq V$.

    Here, we find that $x \times y \in B \times C \subseteq U \times V \subseteq W$ while $B \times C \in \mcal D$.
    Therefore, by \Cref{lem:topToBasis}, $\mcal D$ generates the product topology.
}

\dfn{Projection}{
    Let $\pi_1 \colon X \times Y \to X$ and $\pi_2 \colon X \times Y \to Y$ defined by the equations \[
        \begin{aligned}[t]
            \pi_1(x, y) &= x \\
            \pi_2(x, y) &= y
        \end{aligned}
    \] The maps $\pi_1$ and $\pi_2$ are called the \textit{projections} of $X \times Y$
    onto its first and second factors, respectively.
}
\nt{
    If $U \subseteq X$ is open, then $\pi_1\inv(U) = U \times Y$ is open.
    Similarly, if $V \subseteq Y$ is open, then $\pi_2\inv(V) = X \times V$ is open.
}

\thm{}{
    The collection \[
        \mcal S \coloneqq \{\, \pi_1\inv(U) \mid U \subseteq X \text{ is open} \,\}
                     \cup \{\, \pi_2\inv(V) \mid V \subseteq Y \text{ is open} \,\}
    \] is a subbasis for the product topology of $X \times Y$.
}
\pf{Proof}{
    Let $\mcal T$ be the product topology and $\mcal T'$ be the toplogy generated by $\mcal S$.
    \begin{itemize}[nolistsep]
        \ii Since $\mcal S \subseteq \mcal T$, every union of finite intersections in $\mcal S$ is in $\mcal T$.
            Thus, $\mcal T' \subseteq \mcal T$. \checkmark
        \ii Every open set of $\mcal T$ is a union of elements in
            $\mcal B \coloneqq \{\, U \times V \mid U \subseteq X \text{ and } V \subseteq Y \text{ are open} \,\}$.
            Noting that each $U \times V$ can be expressed as $\pi_1\inv(U) \cap \pi_2\inv(V)$,
            which is a finite intersection of elements in $\mcal S$,
            we may conclude $\mcal T \subseteq \mcal T'$. \checkmark
    \end{itemize}
}

\section{The Subspace Topology}

\dfn{Subspace Topology}{
    Let $(X, \mcal T)$ be a topological space. If $Y \subseteq X$, then \[
        \mcal T_Y \coloneqq \{\, Y \cap U \mid U \in \mcal T \,\}
    \]  is called the \textit{subspace toplogy} of $Y$
    and $(Y, \mcal T_Y)$ is called a \textit{subspace} of $(X, \mcal T)$.
}

\mlemma{}{
    $(Y, \mcal T_Y)$ is a topological space.
}
\pf{Proof}{
    $ $\\[-1em]
    \begin{enumerate}[nolistsep, label=(\roman*)]
        \ii $\varnothing = Y \cap \varnothing$ and $Y = Y \cap X$. \checkmark
        \ii If $U_{\alpha} \in \mcal T_Y$,
            $\bigcup_{\alpha \in J} (Y \cap U_\alpha) = Y \cap \left(\bigcup_{\alpha \in J} U_\alpha\right) \in \mcal T_Y$. \checkmark
        \ii If $U_i \in \mcal T_Y$, $\bigcap_{i=1}^n (Y \cap U_i) = Y \cap \left(\bigcap_{i=1}^n U_i\right) \in \mcal T_Y$. \checkmark
    \end{enumerate}
}

\mlemma[subspaceBasis]{}{
    If $\mcal B$ is a basis for $(X, \mcal T)$, then \[
        \mcal B_Y \coloneqq \{\, Y \cap B \mid B \in \mcal B \,\}
    \] is a basis for the subspace topology on $Y$.
}
\pf{Proof}{
    We will exploit \Cref{lem:topToBasis}.

    Take any $U \in \mcal T$ and $y \in Y \cap U$.
    Since $y \in U$, $\exs B \in \mcal B,\: y \in B \subseteq U$,
    which implies $y \in Y \cap B \subseteq Y \cap U$.
}

\nt{
    Not all open sets in $Y$ are open in $X$.

    For instance, if $X = \RR$ and $Y = [0, 1]$, $Y$ is open in $Y$ but not open in $X$.
}

\mlemma[subspaceOfOpenSet]{}{
    All the open sets in $Y$ are open in $X$ if and only if $Y$ is open in $X$.
}
\pf{Proof}{
    ($\Rightarrow$) $Y$ is open in $Y$. Hence, $Y$ is open in $X$.

    ($\Leftarrow$) Let $U$ be any open set in $Y$.
    Then, $U = Y \cap V$ for some open set $V$ in $X$. Since $Y$ is open in $X$, $U$ is open in $X$.
}

\thm[subOfProdIsProdOfSub]{}{
    If $A$ is a subspace of $X$ and $B$ is a subspace of $Y$, then the product topology on $A \times B$
    is the same as the the topology $A \times B$ inherits as a subspace of $X \times Y$. In other words,
    the following two topologies are the same. \[
        \setlength\arraycolsep{1pt}
        \begin{array}{r@{\qquad}c}
            \text{(i)} & X, Y \xrightarrow{\text{subspace}} A \subseteq X,\, B \subseteq Y \xrightarrow{\text{product}} A \times B \\
            \text{(ii)} & X, Y \xrightarrow{\text{product}} X \times Y \xrightarrow{\text{subspace}} A \times B \subseteq X \times Y
        \end{array}
    \]
}
\pf{Proof}{
    By \Cref{th:prodBasis}, \[
        \{\, U \times V \mid U \in \mcal B_X \text{ and } V \in \mcal B_Y \,\}
    \] is a basis for $X \times Y$. Thus, \[
        \mcal B \coloneqq \{\, (A \times B) \cap (U \times V) \mid U \in \mcal B_X \text{ and } V \in \mcal B_Y \,\}
    \] is a basis for (ii) by \Cref{lem:subspaceBasis}.

    Note that $(A \times B) \cap (U \times V) = (A \cap U) \times (B \cap V)$. 
    Also, $\{\, A \cap U \mid U \in \mcal B_X \,\}$ and $\{\, B \cap V \mid V \in \mcal B_Y \,\}$
    are bases for $A$ and $B$.
    Thus, $\mcal B$ is also a basis for (i) by \Cref{th:prodBasis}.
}

\wc{Order Topology and Subspace Topology}{
    Unlike product topology and subspace topology, order topology and subspace topology are not associative.
    Let $X$ be an ordered set and $Y \subseteq X$. \[
        \setlength\arraycolsep{1pt}
        \begin{array}{r@{\qquad}c}
            \text{(i)} & Y \xrightarrow{\text{order}} Y \\
            \text{(ii)} & X \xrightarrow{\text{order}} X \xrightarrow{\text{subspace}} Y \subseteq X
        \end{array}
    \] Then, will those be the same?
    \paragraph*{Example 1.} Consider $X = \RR$ and $Y = [0, 1]$.
    Then, the subspace topology of the order topology $X$ has a basis of \[
        \mcal B_{[0,1]} = \{\, [0, 1] \cap (a, b) \mid a < b \,\}\text{,}
    \] which is in fact the order topology on $Y$. In this case, $\text{(i)}=\text{(ii)}$.
    \paragraph*{Example 2.} Consider $X = \RR$ and $Y = [0, 1) \cup \{2\}$.
    Then, $\{2\}$ is an open in (ii) since $\{2\} = Y \cap (1.5, 2.5)$.
    But, there is no basis of the order topology on $Y$ such that contains $2$ and is a subset of $\{2\}$.
    Thus, in this case, $\text{(i)} \neq \text{(ii)}$.
    \paragraph*{Example 3.} Consider $X = \RR[2]$ and $Y = I^2$ where $I = [0, 1]$.
    Then, $\{1/2\} \times (1/2, 1]$ is an open set in (ii) since it is $\big(\{1/2\} \times (1/2,3/2)\big) \cap I^2$.
    But it is not an open set in (i) since there is no basis that contain $(1/2, 1)$ and is a subset of $\{1/2\} \times (1/2, 1]$.
}

\dfn{Convex Subset}{
    Given an ordered set $X$ and $Y \subseteq X$,
    $Y$ is called \textit{convex} if \[
        \forall a, b \in Y,\: \big(a < b \implies (a, b) \subseteq Y\big)\text{.}
    \]
}

\thm{}{
    Let $X$ be an ordered set with the ordered topology. If $Y \subseteq X$ is convex,
    then the order topology on $Y$ is the same as the subspace topology.
}
\pf{Proof}{
    We will make use of the fact that open rays form a subbasis that generates the order topology.
    
    First, every open ray of (i) is an open ray of the subspace (ii). \[
        \{\, x \in Y \mid x > a \,\} = \{\, x \in X \cap Y \mid x > a \,\}\text{,}
    \] for example. Therefore, (ii) is finer than (i).

    Now, take any open ray in $X$, $(a, \infty)_X = \{\, x \in X \mid x > a \,\}$, for instance.
    Then, let \[
        \begin{aligned}[t]
            R &\triangleq (a, \infty)_X \cap Y \\
              &= \{\, y \in Y \mid y > a \,\} = (a, \infty)_Y\text{.}
        \end{aligned}
    \]

    If $a \in Y$, then $R$ is an open ray in $Y$.

    Now consider the case $a \notin Y$. If $R$ is nonempty then there is some $y_0 \in R$.
    Take any $y \in Y$. If $y_0 < y$, then $y \in R$ since $a < y_0 < y$.
    If $y < y_0$, it implies $a < y < y_0$ because $y < a < y_0$ with $y, y_0 \in Y$ implies
    $a \in Y$ by the convexity of $Y$. Therefore, $y \in R$.
    So, if $a \notin Y$, it is either $R = \varnothing$ or $R = Y$.

    Combining the cases, we get the fact that the intersection of $Y$ and an arbitrary open ray in $X$
    is an open ray in $Y$, an empty set, or the whole $Y$.

    This is the final step. Take any open set $U$ in the ordered topology $X$.
    Then, $U = \bigcup_{\alpha \in J} U_\alpha$ where $U_\alpha \neq \varnothing$ is a finite intersection of open rays in $X$.
    Noting that $U \cap Y$ is a general form of an open set in $Y$,
    we get $U \cap Y = \bigcup_{\alpha \in J} (U_\alpha \cap Y)$, which implies either $U \cap Y = Y$ or
    $U \cap Y$ is a union of finite intersections of an open ray in $Y$.
}

\cor{}{
    Let $X$ be an ordered set with the ordered topology.
    The subspace topology of $Y \subseteq X$ is finer than the order topology on $Y$.
}

\section{Closed Sets and Limit Points}
\subsection{Closed Sets}

\dfn{Closed Set}{
    Let $X$ be a topological space.
    A subset $A \subseteq X$ is \textit{closed} if $X \setminus A$ is open.
}

\exmp{}{
    \begin{itemize}[nolistsep]
        \ii $[a, b] \subseteq \RR$ is closed since $(-\infty, a) \cup (b, \infty)$ is open.
        \ii $[a, b] \times [c, d] \subseteq \RR[2]$ is closed.
        \ii In discrete topology on $X$, every subset of $X$ is closed.
        \ii If $Y = [0, 1] \cup (2, 3) \subseteq \RR$, $[0, 1]$ and $(2, 3)$ are both open and closed in $Y$.
    \end{itemize}
}

\thm{}{
    Let $X$ be a topological space. Then the following conditions hold.
    \begin{enumerate}[nolistsep, label=(\roman*)]
        \ii $\varnothing$ and $X$ are closed.
        \ii Arbitrary intersections of closed sets are closed.
        \ii Finite unions of closed sets are closed.
    \end{enumerate}
}
\pf{Proof}{
    $ $\\[-1em]
    \begin{enumerate}[nolistsep, label=(\roman*)]
        \ii $X \setminus \varnothing = X$ and $X \setminus X = \varnothing$ are open. \checkmark
        \ii Let $\{A_{\alpha}\}_{\alpha \in J}$ be a collection of closed sets. Then, \[
                \textstyle X \setminus \bigcap_{a \in J} A_\alpha = \bigcup_{\alpha \in J} (X \setminus A_\alpha)\text{.}
            \] is open since each $X \setminus A_\alpha$ is open. \checkmark
        \ii Let $\{A_i\}_{i=1}^n$ be a collection of closed sets. Then, \[
                \textstyle X \setminus \bigcup_{i=1}^n A_i = \bigcap_{i=1}^n (X \setminus A_i)\text{.}
            \] is open since it is a finite intersection of open sets. \checkmark
    \end{enumerate}
}

\thm[closedIffYCapB]{}{
    Let $X$ be a topological space and $Y \subseteq X$.
    Then $A \subseteq Y$ is closed in $Y$ if and only if there is a closed set $B$ in $X$ such that $A = Y \cap B$.
}
\pf{Proof}{
    ($\Leftarrow$) Let $B$ be a closed set of $X$ such that $A = Y \cap B$.
    Then, $X \setminus B$ is open in $X$ and $Y \cap (X \setminus B) = X \setminus Y$ is open in $Y$.
    Thus, $A$ is closed in $Y$.

    ($\Rightarrow$) Since $Y \setminus A$ is open in $Y$,
    $Y \setminus A = Y \cap U$ for some open set $U$ in $X$.
    Then, $A = Y \cap (X \setminus U)$ where $X \setminus U$ is closed in $X$.
}

\thm[subspaceOfClosedSet]{}{
    If $Y$ is closed in $X$, then every closed sets of $Y$ are closed in $X$ if and only if $Y$ is closed in $X$.
}
\pf{Proof}{
    Proof is analogous to the proof of \Cref{lem:subspaceOfOpenSet}.
}

\dfn{Interior and Closure of a Set}{
    Given a subset $A$ of a topological space $(X, \mcal T)$,
    \begin{itemize}
        \ii the \textit{interior} of $A$ is $\inter A \triangleq \bigcup \{\, U \subseteq X \mid U \in \mcal T \text{ and } U \subseteq A \,\}$, and
        \ii the \textit{closure} of $A$ is $\cl A \triangleq \bigcap \{\, V \subseteq X \mid X \setminus V \in \mcal T \text{ and } A \subseteq V \}$.
    \end{itemize}
}
\nt{
    \begin{itemize}[noitemsep]
        \ii $\inter A \subseteq A \subseteq \cl A$
        \ii $\inter A$ is open, and $\cl A$ is closed.
        \ii $\inter A$ is the largest open set contained $A$, and $\cl A$ is the smallest closed set containing $A$.
    \end{itemize}
}

\thm[closureSubspace]{}{
    Let $Y$ be a subspace of $X$ and $A \subseteq Y$.
    Let $\cl A$ and $\cl A_Y$ denote the closures of $A$ in $X$ and $Y$, respectively.
    Then, \[
        \cl A \cap Y = \cl A_Y\text{.}
    \]
}
\pf{Proof}{
    ($\supseteq$) $\cl A \cap Y$ is closed in $Y$ by \Cref{th:closedIffYCapB}.
    Thus, $\cl A_Y \subseteq \cl A \cap Y$.

    ($\subseteq$) $\cl A_Y = B \cap Y$ for some closed set $B$ in $X$ by \Cref{th:closedIffYCapB}.
    Also, $\cl A \subseteq B$ holds. Therefore, $\cl A_Y = B \cap Y \subseteq \cl A \cap Y$.
}

\dfn{Intersection and Neighborhood}{
    \begin{itemize}[nolistsep]
        \ii Given two sets $A$ and $B$, we say $A$ and $B$ \textit{intersect} if $A \cap B \neq \varnothing$.
        \ii An open set containing $x \in X$ is called an open \textit{neighborhood} of $x$.
    \end{itemize}
}

\thm[closureIffNeighCapA]{}{
    Let $A \subseteq X$ where $X$ is a topological space. The following hold.
    \begin{enumerate}[nolistsep, label=(\roman*)]
        \ii $x \in \cl A$ if and only if every neighborhood of $x$ intersects $A$.
        \ii Let $\mcal B$ be a basis for $X$.
            Then, $x \in \cl A$ if and only if every $B \in \mcal B$ containing $x$ intersects $A$.
    \end{enumerate}
}
\pf{Proof}{
    $ $\\[-1em]
    \begin{enumerate}[nolistsep, label=(\roman*)]
        \ii We will prove the contrapositive
            ``$x \notin \cl A \iff \exs \text{ neighborhood } U \text{ of } X,\: U \cap A = \varnothing$''.
            \par ($\Rightarrow$) $U \triangleq X \setminus \cl A$ is a neighborhood of $x$.
            We find that $U \cap A = \varnothing$ since $A \subseteq \cl A$. \checkmark
            \par ($\Leftarrow$) Suppose a neighborhood $U$ of $x$ satisfies $U \cap A = \varnothing$.
            It implies $A \subseteq X \setminus U$. Since $X \setminus U$ is closed, $\cl A \subseteq X \setminus U$ also holds.
            Since $x \in U$, $x \in \cl A$ may never hold. \checkmark
        \ii ($\Rightarrow$) A basis element that contains $x$ is a neighborhood of $x$. \checkmark
            \par ($\Leftarrow$) Follows from the definition of basis. (See \Cref{def:defBasis}.) \checkmark
    \end{enumerate}
}

\exmp{}{
    \begin{itemize}[nolistsep]
        \ii If $A = (0, 1/2) \subseteq \RR$, then $\cl A = [0, 1/2]$.
        \ii If $A = \{\, 1/n \mid n \in \ZZ_+ \,\} \subseteq \RR$, then $\cl A = A \cup \{0\}$.
        \ii If $A = \QQ \subseteq \RR$, then $\cl A = \RR$.
        \ii If $A = \ZZ \subseteq \RR$, then $\cl A = \ZZ$.
    \end{itemize}
}

\subsection{Limit Points}

\dfn{Limit Point}{
    Let $A \subseteq X$ and $x \in X$.
    The point $x$ is a \textit{limit point} of $A$ if every neighborhood of $x$ intersects $A$
    in some point other than $x$.
    The set of limit points of $A$ is denoted by $A'$.
}

\nt{
    Equivalently, $x$ is a limit point of $A$ if $x \in \cl{A \setminus \{x\}}$ thanks to \Cref{th:closureIffNeighCapA}.
}

\thm[closureIsAcupAprime]{}{
    Let $A \subseteq X$ where $X$ is a topological space. Then \[
        \cl A = A \cup A'\text{.}
    \]
}
\pf{Proof}{
    ($\supseteq$) We only need to show $A' \subseteq \cl A$.
    For every $x \in A'$, $x \in \cl A$ due to \Cref{th:closureIffNeighCapA}. \checkmark

    ($\subseteq$) Let $x \in \cl A \setminus A$.
    By definition, every neighborhood of $x$ intersects $A$ while $x$ cannot be in the intersection
    since $x \notin A$. Thus, $x \in A'$. \checkmark
}

\cor[closedIffAprimeSubA]{}{
    Let $A \subseteq X$ where $X$ is a topological space.
    Then $A$ is closed if and only if $A' \subseteq A$.
}
\pf{Proof}{
    ($\Rightarrow$) $A = \cl A = A \cup A'$ and it implies $A' \subseteq A$. \checkmark
    \par ($\Leftarrow$) $\cl A = A \cup A' = A$ and $\cl A$ is closed. \checkmark
}

\subsection{Hausdorff Spaces}
\dfn{Housdorff Space}{
    A topological space $X$ is called a \textit{Hausdorff space} if for each pair $x_1$ and $x_2$
    of distinct points of $X$, there exist neighborhoods $U_1$ and $U_2$ of $x_1$ and $x_2$, respectively,
    that are disjoint. In other words, \[
        \forall x_1, x_2 \in X,\: \big(x_1 \neq x_2 \implies
        \exs U_1, U_2 \in \mcal T,\: x_1 \in U_1 \land x_2 \in U_2 \land U_1 \cap U_2 = \varnothing\big)\text{.}
    \]
}

\end{document}
