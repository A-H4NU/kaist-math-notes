\documentclass[../complex_variables_1.tex]{subfiles}

\begin{document}

\section{Random Variables and Their Distributions}

\begin{Definition}{Random Variable}[rv]
    A \emph{random variable} on \((\Omega, \mcal{F})\) is any mapping \(X \colon \Omega \to \ol{\RR}\)
    such that for all \(a \in \RR\), \(\{\,X \le a\,\} \triangleq \{\,\omega \in \Omega \mid
    X(\omega) \le a\,\} \in \mcal{F}\). Here, \(\ol{\RR} = \RR \cup \{\pm\infty\}\).

    \begin{itemize}
        \ii
        If \(X\) only takes finite values, \(X\) is called a
        \emph{real random variable}.
        \ii
        If \(X\) only takes only a countable set of values \(\{\,a_n\,\}_{n \in \ZZ_{\ge 0}}\),
        \(X\) is called a \emph{discrete random variable}.
    \end{itemize}
\end{Definition}

\begin{Definition}{Cumulative Distribution Function}[]
    The \emph{cumulative distribution function} (CDF) of a random variable \(X\) is the function
    \(F \colon \RR \to [0, 1]\) defined by
    \[
        F(x) = P(X \le x) \triangleq P(\{\,X \le x\,\})\text.
    \]
\end{Definition}

\begin{Lemma}{}[basicCDF]
    Let \(F\) be a cumulative distribution function of a random variable \(X\).
    \begin{enumerate}[label=(\arabic*), ref=\protect{\Cref{lem:basicCDF} (\arabic*)}]
        \ii
        \(F\) is monotone increasing.
        \ii
        \(F\) is right-continuous.
        \ii
        If we define \(F(\infty) \coloneqq \lim_{x\to\infty}F(x)\) and \(F(-\infty) = \lim_{x \to
        -\infty} F(x)\), then \(1 - F(\infty) = P(X = \infty)\) and \(F(-\infty) = P(X =
        -\infty)\).
    \end{enumerate}
\end{Lemma}
\begin{myclaim}[Proof]\hfill
\begin{enumerate}[label=(\arabic*)]
    \ii
    Take any \(x, y \in \RR\) with \(x \le y\).
    Then, \(\{\,X \le x\,\} \subseteq \{\,X \le y\,\}\).
    Hence, \(F(x) = P(X \le x) \le P(X \le y) \le F(y)\).

    \ii
    Take any decreasing nonnegative sequence \(\lang \veps_n \rang_{n \in \ZZ_+}\)
    of real numbers converging to zero and a real number \(x\).
    Let \(C_n \coloneqq \{\,X \le x + \veps_n \,\}\)
    so that \(\lang C_n \rang_{n \in \ZZ_+}\) is a decreasing sequence of events.
    Note also that \(\{\,X \le x\,\} = \bigcap_{n=1}^\infty C_n\)
    Then, by \ref{itm:seqCont.2},
    \[
        F(x) = P(X \le x) = \lim_{n \to \infty} P(X \le x + \veps_n) = \lim_{n \to \infty} F(x +
        \veps_n)\text.
    \]

    \ii
    Let \(B_n \coloneqq \{\,X \le n\,\}\) for each \(n \in \ZZ_+\)
    so that \(\bigcup_{n=1}^\infty B_n = \{\,X < \infty\,\}\)
    and \(\lang B_n \rang_{n \in \ZZ_+}\) is an increasing sequence of events.
    By \ref{itm:seqCont.1},
    \[
        1 - P(X = \infty) = P(X < \infty)
        = P \left( \bigcup_{n=1}^\infty B_n \right)
        = \lim_{n\to\infty} P(B_n) = \lim_{n\to\infty} F(n) = F(\infty)\text.
    \]
    The last equality is due to (1).
    \qed
\end{enumerate}
\end{myclaim}

\begin{Definition}{Probability Density}[]
    If a real random variable \(X\) admits a cumulative distribution function \(F\) such that
    \[
        F(x) = \int_{-\infty}^x f(y) \d y
    \]
    for some nonnegative function \(f\), then \(X\) is said to admit the
    \emph{probability density} \(f\).
\end{Definition}

\begin{note}
    Note that the probability density \(f\) satisfies
    \[
        \int_{-\infty}^\infty f(y) \d y = 1\text.
    \]
\end{note}

\end{document}
