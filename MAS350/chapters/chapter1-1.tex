\documentclass[../complex_variables_1.tex]{subfiles}

\begin{document}

\section{Events and Probability}

\begin{Definition}{Probability Space}[]
    A \emph{probability space} contains of a triple \((\Omega, \mcal{F}, P)\)
    where
    \begin{itemize}
        \ii
        \(\Omega\) is the sample space,
        \ii
        \(\mcal{F} \subseteq 2^{\Omega}\) (each \(A \in \mcal{F}\) is called an \emph{event}), and
        \ii
        \(P \colon \mcal{F} \to [0, 1]\) maps each event \(A \in \mcal{F}\) to the \emph{probability} of \(A\)
    \end{itemize}
    which satisfies the following conditions:
    \paragraph*{Axioms Relative to the Events}
    The family \(\mcal{F}\) of events must be a \(\sigma\)-field on \(\Omega\):
    \begin{enumerate}[label=(\arabic*)]
        \ii \(\Omega \in \mcal{F}\);
        \ii If \(A \in \mcal{F}\), then \(\cmpl A \in \mcal{F}\) (where \(\cmpl A\) is the complement
        of \(A\));
        \ii If \(\lang A_n \rang_{n \in \ZZ_+}\) is a sequence on \(\mcal{F}\),
        then \(\bigcup_{n=1}^\infty A_n \in \mcal{F}\).
    \end{enumerate}
    \paragraph*{Axioms Relative to the Probability}
    The function \(P\) must satisfy the following conditions:
    \begin{enumerate}[label=(\arabic*)]
        \ii \(P(\Omega) = 1\);
        \ii \(\sigma\)-additivity holds: if \(\lang A_n \rang_{n \in \ZZ_+}\) is a sequence of
        pairwise disjoint events, then
        \[
            P\left( \bigcup_{n=1}^\infty A_n \right) = \sum_{n=1}^\infty P(A_n).
        \]
    \end{enumerate}
\end{Definition}

\begin{note}
    Here are immediate properties of probability:
    \begin{itemize}
        \ii \(P(\cmpl A) = 1 - P(A)\);
        \ii \(\OO = \cmpl\Omega \in \mcal{F}\) and \(P(\OO) = 0\);
        \ii If \(\lang A_n \rang_{n \in \ZZ_+}\) is a sequence of events, then
        \(\bigcap_{n=1}^\infty A_n\) is also an event;
        \ii \(A, B \in \mcal{F}\) and \(A \subseteq B\) implies \(P(A) \le P(B)\).
    \end{itemize}
\end{note}

\begin{Lemma}{sub-\(\sigma\)-additivity}[subAdd]
    If \(\lang A_n \rang_{n \in \ZZ_+}\) is a sequence of events, then
    \[
        P\left( \bigcup_{n=1}^\infty A_n \right) \le \sum_{n=1}^\infty P(A_n)\text.
    \]
\end{Lemma}
\begin{myproof}[Proof]
    Let \(B_n = A_n \setminus \bigcup_{i=1}^{n-1} A_i\)
    for each \(n \ge 1\) and use \(\sigma\)-additivity.
\end{myproof}

\begin{Lemma}{Inclusion-Exclusion Principle}[]
    If \(A_1, \cdots, A_n\) are events, then
    \[
        P \left( \bigcup_{i=1}^n A_i \right)
        = \sum_{\OO \neq I \subseteq [n]} (-1)^{|I| - 1} P \left( \bigcap_{i \in I} A_i
        \right)\text.
    \]
\end{Lemma}
\begin{myproof}[Proof]
    Classic.
\end{myproof}

\begin{Theorem}{Sequential Continuity of Probability}[seqCont]
    \begin{enumerate}[label=(\arabic*), ref=\protect{\Cref{th:seqCont} (\arabic*)}]
        \ii\label{itm:seqCont.1}
        Let \(\lang B_n \rang_{n \in \ZZ_+}\) be a sequence of events such that
        \(B_n \subseteq B_{n+1}\) for all \(n \ge 1\). Then,
        \[
            P\left( \bigcup_{n=1}^\infty B_n \right) = \lim_{n \to \infty} P(B_n)\text.
        \]
        \ii\label{itm:seqCont.2}
        Let \(\lang C_n \rang_{n \in \ZZ_+}\) be a sequence of events such that
        \(C_n \supseteq C_{n+1}\) for all \(n \ge 1\). Then,
        \[
            P\left( \bigcap_{n=1}^\infty C_n \right) = \lim_{n \to \infty} P(C_n)\text.
        \]
    \end{enumerate}
\end{Theorem}
\begin{myclaim}[Proof]\hfill
\begin{enumerate}[label=(\arabic*)]
    \ii
    Let \(B_n' \coloneqq B_n \setminus B_{n-1}\) for each \(n \ge 2\) and \(B_1' \coloneqq B_1\).
    so that \(B_m = \bigcup_{n=1}^m B_n'\)
    and \(B_i'\)'s are pairwise disjoint.
    Hence, by \(\sigma\)-additivity, we have
    \[
        P\left( \bigcup_{n=1}^\infty B_n \right) =
        P\left( \bigcup_{n=1}^\infty B_n' \right) =
        \sum_{n=1}^\infty P(B_n') =
        P(B_1) + \sum_{n=1}^\infty \bigl( P(B_n) - P(B_{n-1}) \bigr) =
        \lim_{n \to \infty} P(B_n)\text.
    \]
    \ii
    Let \(C_n' \coloneqq \cmpl{C_n}\) for each \(n \ge 1\)
    so that \(C_n' \subseteq C_{n+1}'\) for all \(n\).
    Hence, by (1), we have \(P\left( \bigcup_{n=1}^\infty C_n' \right) = \lim_{n \to \infty}
    P(C_n')\). The result follows from the fact that
    \(\bigcup_{n=1}^\infty C_n' = \Omega \setminus \bigcap_{n=1}^\infty C_n\).
    \qed
\end{enumerate}
\end{myclaim}

\end{document}
