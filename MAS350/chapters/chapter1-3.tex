\documentclass[../complex_variables_1.tex]{subfiles}

\begin{document}

\section{Conditional Probability and Independence}

\begin{Definition}{Conditional Probability}[]
    Let \(B\) be an event with \(P(B) > 0\).
    For any event \(A\), we define
    \[
        P(A \mid B) \coloneqq \frac{P(A \cap B)}{P(B)}
    \]
    and it is called the \emph{probability of \(A\) given \(B\)}.
\end{Definition}

\begin{Definition}{Independent Events}[]
\begin{enumerate}[label=(\arabic*)]
    \ii
    Two events \(A\) and \(B\) are said to be \emph{indepenent} if
    \(P(A \cap B) = P(A)P(B)\).

    \ii
    Let \(\mcal{A}\) be a nonempty family of events.
    \(\mcal{A}\) is said to be a \emph{family of independent events} if
    for any finite subfamily \(\lang A_1, \cdots, A_n \rang\) of \(\mcal{A}\),
    \[
        P\left( \bigcap_{i=1}^n A_i \right) = \prod_{i=1}^n P(A_i)\text.
    \]
\end{enumerate}
\end{Definition}

\begin{note}
    When \(P(B) > 0\), \(A\) and \(B\) are indepenent
    if and only if \(P(A \mid B) = P(A)\).
\end{note}

\end{document}
