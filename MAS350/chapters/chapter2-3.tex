\documentclass[../probability.tex]{subfiles}

\begin{document}

\section{Independence}

\begin{Definition}{Independence of Discrete Random Elements}[indepDRV]
    Let \(X\) and \(Y\) be two discrete random elements
    with values in the denumerable spaces \(E\) and \(F\), respectively.
    Now, one can define another random element \(Z\) on \(G \coloneqq E \times F\)
    by \(Z(\omega) = (X(\omega), Y(\omega))\).
    We say \(X\) and \(Y\) are \emph{independent} if
    \[
        P(X = x, Y = y) \coloneqq P(Z = (x, y)) = P(X = x) P(Y = y)
    \]
    for all \(x \in E\) and \(y \in F\).
    This can be ge
\end{Definition}

\begin{Lemma}{Product Formula}[prodFomula]
    Let \(X\) and \(Y\) be two discrete random elements
    with values in the denumerable spaces \(E\) and \(F\), respectively.
    If \(f \colon E \to \RR\) and \(g \colon F \to \RR\)
    satisfy \eqref{eq:expAssum}, and if \(X\) and \(Y\)
    are independent, then \(\mbb{E}[f(X)g(Y)]\) is well-defined and
    \[
        \mbb{E}[f(X)g(Y)] = \mbb{E}[f(X)]\cdot \mbb{E}[g(Y)]\text.
    \]
\end{Lemma}

\nt{%
    \Cref{dfn:indepDRV,lem:prodFomula} can readily be generalized to
    finite number of discrete random elements.
}

% TODO: Add some exercises.

\end{document}
