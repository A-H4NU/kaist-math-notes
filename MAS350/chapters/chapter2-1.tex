\documentclass[../probability.tex]{subfiles}

\begin{document}

\section{Discrete Random Elements}

\begin{Definition}{Discrete Random Element}[dre2]
    Let \(E\) be a denumerable set and let \((\Omega, \mcal{F}, P)\) be a
    probability space. Any function \(X \colon \Omega \to E\) such that
    \[
        \fall x \in E,\: \{\,\omega \mid X(\omega) = x\,\} \in \mcal{F}
    \]
    is called a \emph{discrete random element} of \(E\).
    When \(E \subseteq \RR\), we refer to \(X\) as a \emph{discrete random variable}.
    This allows us to define
    \[
        p(x) \coloneqq P(X = x)
    \]
    for \(x \in E\).
    The collection \(\{\,p(x) \mid x \in E\,\}\)
    is the \emph{distribution} of \(X\). It satisfies
    \[
        0 \le p(x) \le 1\quad\text{and}\quad
        \sum_{x \in E} p(x) = 1\text.
    \]
\end{Definition}

\nt{%
    \(E\) being denumerable enables us to define in such way.
    Note the difference from \Cref{dfn:rv}.
}

\end{document}
