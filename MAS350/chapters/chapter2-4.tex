\documentclass[../probability.tex]{subfiles}

\begin{document}

\section{Mean and Variance}

\begin{Definition}{Mean and Variance of Discrete Random Variable}[meanVar]
    If \(X\) is a discrete random variable, the quantities
    \[
        m \triangleq \mbb{E}[X]\quad\text{and}\quad
        \sigma^2 \triangleq \Var[X] \triangleq \mbb{E}[(X - m)^2]
    \]
    are called the \emph{mean} and \emph{variance} of \(X\), respectively.
    The quantity \(\sigma \triangleq \sqrt{\sigma^2}\) is called the
    \emph{standard deviation} of \(X\).
\end{Definition}

\begin{note}
    Some properties of mean and variance:
    \begin{itemize}
        \ii
        \(\Var[aX] = a^2 \Var[X]\).
        \ii
        \(\sigma^2 = 0\) implies that
        \(p(x) = 0\) for all \(x \neq m\).
        \ii
        If \(X_1, \cdots, X_n\) are independent discrete random variables,
        then \(\Var \bigl[ \sum_{i=1}^n X_i \bigr] = \sum_{i=1}^n \Var[X_i]\).
    \end{itemize}
\end{note}

\begin{Theorem}{Chebyshev's Inequality}[chebyshev]
    Let \(X\) be a discrete random variable. Then,
    for any \(\veps > 0\), we have
    \[
        P(|X-m| \ge \veps) \le \frac{\sigma^2}{\veps^2}\text.
    \]
\end{Theorem}
\begin{myproof}[Proof]
    Apply \nameref{th:markov} to \(X\) with \(f(x) = (x-m)^2\) and \(a = \veps^2\) to get
    \begin{align*}
        P(|X-m| \ge \veps)
        &= P((X-m)^2 \ge \veps^2) \\
        &\le \frac{\mbb{E}[|X-m|^2]}{\veps^2}
        = \frac{\sigma^2}{\veps^2}\text. \qedhere
    \end{align*}
\end{myproof}

\begin{Theorem}{Weak Law of Large Numbers}[wlln]
    Let \(\lang X_n \rang_{n \in \ZZ_{>0}}\) be a sequence of discrete random variables,
    identically distributed with common mean \(m\) and common variance \(\sigma^2\).
\end{Theorem}

% TODO: Two exercises from lecture note.



\end{document}
