\documentclass[../modern_algebra.tex]{subfiles}

\begin{document}

\section{Alternating Groups}

\dfn[mCycle]{\(\bs{m}\)-Cycle}{
    Permutations of the form \(\cycle[\,]{a_1,a_2,\cdots,a_m}\)
    is called \textit{\(m\)-cycles}.
}

\nt{
Some basic facts:
\begin{itemize}[nolistsep, leftmargin=*, listparindent=\parindent]
    \ii
    \(S_1, S_2, S_3\) consist of cycles while
    \(S_4\) has a non-cycle \(\cycle[\,]{1,2}\cycle[\,]{3,4}\).

    \ii
    \(\cycle[\,]{a_1,a_2,\cdots,a_m}\inv = \cycle[\,]{a_m,a_{m-1},\cdots,a_1}\).

    \ii
    Every \(\sigma \in S_n\) admits a disjoint cycle decomposition.
    In other words, \[
        \sigma = \cycle[\,]{a_{i_{11}},\cdots,a_{i_{1m_1}}}
        \cycle[\,]{a_{i_{21}},\cdots,a_{i_{2m_2}}}
        \cdots \cycle[\,]{a_{i_{k1}},\cdots,a_{i_{km_k}}}
    \]
    where \(a_{i_{j\ell}}\)s are all different.
    Moreover, the cycle decomposition is unique up to
    permutation of the cycles.

    \ii
    If \(\sigma = \sigma_1 \sigma_2 \cdots \sigma_k\) is a disjoint cycle decomposition,
    then \(\sigma^n = \sigma_1^n \sigma_2^n \cdots \sigma_k^n\).
    Moreover, \(|\sigma| = \lcm(|\sigma_1|, |\sigma_2|, \cdots, \sigma_k)\).
\end{itemize}
}

\exmp[]{Center of Symmetric Group}{
    \(Z(S_2) = S_2\) since \(S_2\) is abelian.
    Fix \(n \ge 3\) and consider \(S_n\).
    Let \(\sigma \in Z(S_n) \setminus \{(1)\}\).
    Let \(\sigma = \cycle[\,]{a_1,a_2,\cdots,a_m}\sigma_2 \cdots \sigma_k\) be a disjoint cycle decomposition
    with \(m \ge 2\).
    Choose \(\tau \in S_n\) such that \(\tau(a_1) = a_1\) and \(\tau(a_2) \neq a_2\).
    Then, \(\sigma(a_1) = \tau \sigma \tau\inv (a_1) = \tau \sigma(a_1) = \tau(a_2) \neq a_2\),
    which is a contradiction. Hence, \(Z(S_n) = \{(1)\}\).
}

\dfn[transpotiion]{Transposition}{
    A \textit{transposition} is a \(2\)-cycle \(\cycle[\,]{a,b}\).
}
\nt{
\begin{itemize}[nolistsep, leftmargin=*, listparindent=\parindent]
    \ii \(\cycle[\,]{a_1,a_2,\cdots,a_m} = \cycle[\,]{a_1,a_m} \cycle[\,]{a_1,a_{m-1}} \cdots \cycle[\,]{a_1,a_2}\).
    \ii
    By the cyclic decomposition and the equation above, we get the fact that
    every \(\sigma \in S_n\) is a product of transpositions.
\end{itemize}
}

\dfn[parityPerm]{Parity of Permutation}{
    For each \(\sigma \in S_n\), define
    \(\sigma(\Delta) = \prod_{i \le i < j \le n} (x_{\sigma(i)} - x_{\sigma(j)})\)
    be a polynomial on independent variables \(x_1, \cdots, x_n\).
    Let \(\Delta \triangleq (1)(\Delta)\).
    Then, \(\sigma(\Delta) = \pm \Delta\).
    We define \(\veps \colon S_n \to \{1, -1\}\) by
    \[
        \veps(\sigma) \triangleq \begin{cases}
            1 & \text{if}~\sigma(\Delta) = \Delta \\
            -1 & \text{if}~\sigma(\Delta) = -\Delta.
        \end{cases}
    \]
}

\thm[]{}{
    \(\veps\) in \Cref{dfn:parityPerm} is a surjective group homomorphism.
}
\pf{Proof}{
    Take any \(\sigma, \tau \in S_n\).
    Suppose \(\sigma(\Delta)\) has exactly \(k\) factors of \((x_j - x_i)\) with \(j > i\)
    so that \(\veps(\sigma) = (-1)^k\).
    \(\veps(\tau \sigma) \Delta = (\tau \sigma)(\Delta) = \veps(\sigma) \prod_{i \le i < j \le n} (x_{\tau(i)} - x_{\tau(j)})
    = \veps(\sigma)\veps(\tau) \Delta\).
    Hence, \(\veps(\tau \sigma) = \veps(\sigma) \veps(\tau) = \veps(\tau) \veps(\sigma)\).
}

\dfn[alternatingGroup]{Alternating Group}{
    \[
        A_n \triangleq \ker (\veps \colon S_n \to \{\pm 1\})
    \]
}


\end{document}
