\documentclass[../modern_algebra.tex]{subfiles}

\begin{document}

\section{Stabilizers and Orbits}

\dfn[stabilizer]{Stabilizer}{
    Let \(G \actson A\).
    The \textit{stabilizer of \(a \in A\)} is the set
    \[
        G_a \triangleq \{\,g \in G \mid ga = a\,\}.
    \]
}

\dfn[kernelGroupAction]{Kernel of Group Action}{
    Let \(G \actson A\).
    The \textit{kernel of \(G \actson A\)} is the set
    \[
        K(G, A) \triangleq \{\,g \in G \mid \fall a \in A,\: ga = a\,\}
        = \textstyle \bigcap_{a \in A} G_a.
    \]
}

\nt{
    \(K(G, A)\) is the kernel of the permutation representation of the group action.
    Therefore, \(K(G, A) \nsub G\).
}

\thm[]{}{
    Let \(G \actson A\).
    Then, \(\fall a \in G,\: G_a \le G\).
}
\pf{Proof}{
    \(G_a \neq \OO\) since \(1 \in G_a\).
    If \(x, y \in G_a\),
    then \((xy\inv) a = (xy\inv)(ya) = xa = a\); thus \(xy\inv \in G_a\).
    Hence, \(G_a \le G\) by \Cref{th:subgroupTFAE}.
}

\exmp[]{}{
\begin{enumerate}[nolistsep, label=(\roman*), ref=\protect{(\roman*)}, listparindent=\parindent]
    \ii
    Let \(G\) be a group and let \(S \triangleq \mcal{P}(G)\).
    Define a group action of \(G\) on \(S\) by \((g, A) \mapsto gAg\inv\).
    Then, for each \(A \in \mcal{P}(G)\),
    \(G_A = \{\,g \in G \mid gAg\inv = A\,\} = N(A)\).

    \ii
    Let \(G\) be a group and let \(A \subseteq G\).
    Define a group action of \(N(A)\) on \(A\) by \((g, a) \mapsto gag\inv\).
    Then, \(K(N(A), A) = \{\,g \in N(A) \mid \fall a \in A,\: gag\inv = a\,\} = C(A)\).

    \ii
    Let \(G\) be a group
    and define a group action of \(G\) on \(G\) by \((g, a) \mapsto gag\inv\).
    Then, \(G_a = \{\,g \in G \mid gag\inv = a\,\} = C(a)\) for each \(a \in G\)
    and \(K(G, G) = \{\,g \in G \mid \fall a \in A,\: gag\inv = a\,\} = Z(G)\).
\end{enumerate}
}

\dfn[faithfulGroupAction]{Faithful Group Action}{
    If \(G \actson A\), we say the group action is \textit{faithful} if \(K(G, A) = \{1\}\).
}

\nt{
    Let \(\vphi \colon G \to S(A)\) be the permutation representation.
    Then, \(G/K(G, A) \cong \img(\vphi) \le S(A)\)
    so we may consider injective group homomorphism \(G/K(G, A) \injto S(A)\)
    so that \(G/K(G, A) \actson A\) is faithful.
}

\mlemma[HhRpyQeL]{}{
    Define \(a \sim b \iff \exs g \in G,\: a = g \cdot b\).
    Then, \(\sim\) is an equivalence relation.
}

\dfn[orbit]{Orbit}{
    Let \(G \actson A\).
    The \textit{orbit of \(a \in A\)} is the set
    \[
        Ga \triangleq \{\,g \cdot a \mid g \in G\,\}.
    \]
}
\nt{
    By \Cref{lem:HhRpyQeL}, the collection of orbits
    forms a partition of \(A\).
    Moreover, \(G \actson Ga\) for each \(a \in A\).
}

\thm[orbitStab]{Orbit-Stabilizer Theorem}{
    Let \(G \actson A\) and \(a \in A\).
    Then, the function
    \begin{align*}
        f\colon Ga &\longrightarrow \{\,\text{left cosets of}~G_a~\text{in}~G\,\} \\
        ga &\longmapsto g G_a
    \end{align*}
    is well-defined and is a bijection.
    In particular, if \(Ga\) is finite, then
    \(|Ga| = [G:G_a]\).
}
\pf{Proof}{
    For each \(g, g' \in G\), we have
    \[
        ga = g'a \iff a = g\inv g' a
        \iff g \inv g' \in G_a \iff g G_a = g' G_a
    \]
    Therefore, \(f\) is well-defined and is injective.
    The surjectivity of \(f\) is evident.
}

\dfn[transitiveGroupAction]{Transitive Group Action}{
    Let \(G \actson A\).
    The group action is \textit{transitive}
    if \(\fall a \in A,\: A = Ga\).
}

\nt{
    By \nameref{th:orbitStab} and \nameref{th:lagrange},
    if \(G\) and \(A\) are finite, and if the group action is transitive,
    then \(|A| \mid |G|\).
}

\dfn[]{}{
    Let \(G \actson A\).
    Then, for each \(g \in G\), we define
    \[
        A_g \triangleq \{\,a \in A \mid g \cdot a = a\,\}.
    \]
}

\exmp[]{}{
\begin{enumerate}[nolistsep, label=(\roman*), ref=\protect{(\roman*)}, listparindent=\parindent]
    \ii
    Let \(S_n \actson [n]\).
    Then, \((S_n)_i \cong S_{n-1}\) for each \(i \in [n]\).
    Moreover, \(K(S_n, [n]) = \bigcap_{i \in [n]} (S_n)_i = \{(1)\}\).
    By \nameref{th:orbitStab}, \(|S_n \cdot i| = |S_n| / |(S_n)_i| = n\);
    thus \(S_n \cdot i = [n]\).
\end{enumerate}
}

\thm[burnside]{Burnside's Lemma}{
    Let \(G \actson A\) and let \(|G|\) and \(|A|\) be finite.
    Then,
    \[
        (\text{\# of orbits of}~G) = \frac{1}{|G|} \sum_{a \in A} |G_a|
        = \frac{1}{|G|} \sum_{g \in G} |A_g|.
    \]
}
\pf{Proof}{
    Let \(S \triangleq \{\,(g, a) \in G \times A \mid g \cdot a = a\,\}\).
    Then, by double counting, \(|S| = \sum_{a \in A} |G_a| = \sum_{g \in G} |A_g|\).
    By \nameref{th:orbitStab},
    \[
        \sum_{a \in A} |G_a| = \sum_{a \in A} \frac{|G|}{|Ga|}
        = |G| \sum_{a \in A} \frac{1}{|Ga|}.
    \]
    Since \(\sum_{a' \in Ga} |Ga|\inv = 1\),
    we have \(\sum_{a \in A} \frac{1}{|Ga|} = (\text{\# of orbits of}~G)\).
    Therefore, we have
    \[
        (\text{\# of orbits of}~G) = \frac{1}{|G|} \sum_{a \in A} |G_a|
        = \frac{1}{|G|} \sum_{g \in G} |A_g|.
    \]
}

\end{document}
