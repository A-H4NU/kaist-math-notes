\documentclass[../modern_algebra.tex]{subfiles}

\begin{document}

\section{Group Actions by Conjugation}

\dfn[conjugate]{Conjugate}{
    Let \(G\) be a group.
    We say \(a, b \in G\) are \textit{conjugate}
    if
    \[\exs g \in G,\: b = gag\inv.\]
    In other words,
    if \(G\) acts on \(G\) by conjugation \(g \cdot a = g a g\inv\),
    \(a, b \in G\) are conjugate if they are in the same orbit.
    The orbit of \(a\) in this case
    is called \textit{conjugacy class} of \(a\).
}

\nt{
    Under conjugation, the stabilizer of \(a\) is
    the centralizer of \(a\).
}

\exmp[iUFrfCWG]{}{
\begin{enumerate}[nolistsep, label=(\roman*), ref=\protect{\Cref{exmp:iUFrfCWG} (\roman*)}, listparindent=\parindent]
    \ii\label{itm:iUFrfCWG.i}
    The conjugacy class of \(a\) is \(\{1\}\) if and only if \(a \in Z(G)\).
    \ii
    Let \(\sigma \in S_n\) has the \textit{cycle type} \((n_1, n_2, \cdots, n_r)\).
    Then, as \(\sigma\) and its conjugation have the same cycle type,
    the conjugacy class of \(\sigma\) is the collection of permutations
    with the same cycle type of \(\sigma\).
\end{enumerate}
}

\cor[XhaqBMFa]{}{
    Let \(G \actson A\) and let \(a \in A\).
    If \([G : C_G(a)]\) is finite, then
    \[
        |\text{conjugacy class of}~a| = [G : C_G(a)].
    \]
}
\pf{Proof}{
    Direct consequence of \nameref{th:orbitStab}.
}

\exmp[]{}{
Let \(1 \le m \le n\).
Let \(\sigma = \cycle[\,]{1,2,\cdots,m}\) be an \(m\)-cycle in \(S_n\).
Then, there are \(n(n-1)\cdots(n-m+1)/m\) number of \(m\)-cycles in \(S_n\).
Therefore, \(|C_{S_n}(\sigma)| = |G| / [n(n-1)\cdots(n-m+1)/m] = m \cdot (n-m)!\).
One may note that
\(C_{S_n}(\sigma) = \{\,\sigma^i \tau \mid 0 \le i \le m - 1 ~\text{and}~\tau \in S_{n-m}\,\}\).
}

\thm[classEquation]{Class Equation}{
    Let \(G\) be a finite group.
    If \(C_1, C_2, \cdots, C_r\) are all the distinct conjugacy classes of \(G\)
    such that \(\fall i \in [r],\: C_i \not\subseteq Z(G)\),
    and if \(a_i \in C_i\) for each \(i \in [r]\),
    then
    \[
        |G| = |Z(G)| + \sum_{i=1}^r [G : C_G(a_i)].
    \]
}
\pf{Proof}{
    \(Z(G)\) is the union of all singleton conjugacy classes by
    \ref{itm:iUFrfCWG.i}.
    The result follows from \Cref{cor:XhaqBMFa}.
}

\exmp[]{}{
\begin{itemize}[nolistsep, leftmargin=*, listparindent=\parindent]
    \ii \(|S_3| = 1 + 2 + 3\)
    \ii \(|Q_8| = 2 + 2 + 2 + 2\)
    \ii \(|D_4| = 2 + 2 + 2 + 2\)
\end{itemize}
}

\cor[pGroupCenter]{}{
    Let \(G\) be a group of order \(p^n\) where \(p\) is a prime number and \(n \ge 1\).
    Then, \(|Z(G)| = p^k\) for some \(k \le 1\).
}
\pf{Proof}{
    In \nameref{th:classEquation},
    each \([G : C_G(a_i)]\) is a multiple of \(p\).
    Therefore, we must have \(p \mid |Z(G)|\)
    while \(Z(G) \neq \OO\).
}

\cor[orderPSquare]{}{
    Let \(G\) be a group of order \(p^2\) where \(p\) is a prime number,
    then \(G \cong \ZZ_{p^2}\) or \(G \cong \ZZ_p \times \ZZ_p\).
}
\pf{Proof}{
    By \Cref{cor:pGroupCenter}, we have \(|Z(G)| = p^2\) or \(|Z(G)| = p\).

    If \(|Z(G)| = p^2\), then
    If \(G\) has an element of order \(p^2\), then \(G \cong \ZZ_{p^2}\).
    If every nonidentity element of \(G\) has order \(p\),
    then \(G \cong \ZZ_p \times \ZZ_p\).
    Then, \(f \colon \ZZ_p \times \ZZ_p \to G\)
    defined by \((i, j) \mapsto x^i y^j\)
    where \(x \in G \setminus \{1\}\) and \(y \in G \setminus \lang x \rang\)
    is a group isomorphism.

    Now, assume \(|Z(G)| = p\). Then, \(G/Z(G) \cong \ZZ_p\).
    By \Cref{th:modCenterCyclic}, we get \(Z(G) = G\), which is a contradiction.
}

\thm[normalIsUnionOfConjClass]{}{
    Let \(G\) be a group and let \(N \nsub G\).
    Let \(K\) be a conjugacy class of \(G\).
    Then,  we have \(K \subseteq N\) or \(K \cap N = \OO\).
    In particular, \(N\) is union of some conjugacy classes of \(G\).
}
\pf{Proof}{
    Assume \(K \cap N \neq \OO\) and take any \(x \in K \cap N\).
    Then, for any \(g \in G\), \(gxg\inv \in gNg\inv = N\); thus \(K \subseteq N\).
}

\exmp[]{}{
    There are four cycle types of \(A_5\);
    \(\cycle[\,]{1}, \cycle[\,]{1,2,3}, \cycle[\,]{1,2,3,4,5}, \cycle[\,]{1,2}\cycle[\,]{3,4}\).
    Note that, even if \(\sigma\) and \(\sigma'\) have the same cycle type
    so that \(\sigma' = \tau \sigma \tau\inv\) for some \(S_5\),
    \(\sigma\) and \(\sigma'\) may not be in the same conjugacy class
    since \(\tau\) may not be an element of \(A_5\).

    \begin{itemize}[nolistsep, leftmargin=*, listparindent=\parindent]
        \ii
        \(C_{S_5}(\cycle[\,]{1,2,3}) = \lang \cycle[\,]{1,2,3}, \cycle[\,]{4,5} \rang\) and
        \(C_{A_5}(\cycle[\,]{1,2,3}) = \lang \cycle[\,]{1,2,3} \rang \cong \ZZ_3\);
        thus the conjugacy class consists of \(20\) elements; which are all the \(3\)-cycles in \(A_5\).
        \ii
        \(C_{S_5}(\cycle[\,]{1,2,3,4,5}) = \lang \cycle[\,]{1,2,3,4,5} \rang\) and
        \(C_{A_5}(\cycle[\,]{1,2,3,4,5}) = \lang \cycle[\,]{1,2,3,4,5} \rang \cong \ZZ_5\);
        the conjugacy class of \(\cycle[\,]{1,2,3,4,5}\) consists of \(12\) elements
        while \(A_5\) has \(24\) \(5\)-cycles.
        The conjugacy class of \(\cycle[\,]{1,3,5,2,4}\) consists of \(12\) elements.
        \ii
        \(|C_{S_5}(\cycle[\,]{1,2}\cycle[\,]{3,4})| = 8\)
        and \(|C_{A_5}(\cycle[\,]{1,2}\cycle[\,]{3,4})| = 4\);
        the conjugacy class of \(\cycle[\,]{1,2}\cycle[\,]{3,4}\)
        consists of all \(15\) elements.
    \end{itemize}

    Therefore, the class equation of \(A_5\)
    is \(|A_5| = 1 + 12 + 12 + 15 + 20\);
    thus by \Cref{th:normalIsUnionOfConjClass},
    if there is a nontrivial normal subgroup
    then its order is sum of orders of some conjugacy classes but there is no way to
    make it divisible by \(|A_5| = 60\).
    Therefore, \(A_5\) is simple.
}

\cor[]{}{
    Let \(G \actson \mcal{P}(G)\) by conjugation;
    \((g, A) \mapsto g A g\inv\).
    We say \(A, B \subseteq G\) are \textit{conjugate}
    if \(A = gBg\inv\) for some \(g \in G\).
    Then, \([G:N_G(A)] = |G\cdot A| = |\text{orbit of}~A|\).
}
\pf{Proof}{
    \(N_G(A) = G_A\).
}

\end{document}
