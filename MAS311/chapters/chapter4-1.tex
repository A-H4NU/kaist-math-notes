\documentclass[../modern_algebra.tex]{subfiles}

\begin{document}

\section{Direct Products}

\dfn[directProduct2]{Direct Product}{
    (See \Cref{dfn:directProduct}.)

    Let \(G_1, G_2, \cdots, G_n\) be groups.
    Then, the operation on \(G_1 \times \cdots \times G_n\)
    given by \[(g_1, \cdots, g_n) \ast (g_1', \cdots, g_n') = (g_1 g_1', \cdots, g_n g_n')\]
    is a group operation.
    We call the group \((G_1 \times \cdots \times G_n, \ast)\)
    the \textit{direct product} of \(G_1, \cdots, G_n\).
}

\notat{}{
    Let \(G_1, G_2, \cdots, G_n\) be groups
    and consider their direct product \(G_1 \times G_2 \times \cdots, G_n\).
    For each \(i \in [n]\), define
    \[
        \tilde{G}_i \triangleq \{\,(1_{G_1}, \cdots, 1_{G_{i-1}}, g_i, 1_{G_{i+1}}, \cdots, 1_{G_n})
        \mid g_i \in G_i\,\} \le G_1 \times G_2 \times \cdots, G_n
    \]
    so that \(G_1 \cong \tilde{G}_i\) and
    \[
        (G_1 \times G_2 \times \cdots \times G_n)/\tilde{G}_i
        \cong G_1 \times \cdots \times G_{i-1} \times G_{i+1} \times \cdots \times G_n.
    \]
    Abusing the notation, we may write \(G_i\) instead of \(\tilde{G}_i\).
}

\nt{
    Let a group structure is given for \(G_1 \times G_2\).
    If both projections are group homomorphisms,
    then the group structure is the direct product.
}

\mlemma[normalTrivInter]{}{
    Let \(G\) be a group and let \(H, K \nsub G\) and \(H \cap K = \{1\}\).
    Then, \(\fall a \in M,\: \fall b \in N,\: ab = ba\).
}
\pf{Proof}{
    Take any \(h \in H\) and \(k \in K\).
    Then, \(h\inv k h \in K\) and \(k h k\inv \in H\) by normality; thus \(h\inv k h k\inv \in H \cap K\),
    which implies \(h\inv k h k\inv = 1\). Therefore, we have \(kh = hk\).
}

\thm[bijectiveProdIso]{}{
    Let \(G\) be a group and let \(N_1, N_2, \cdots, N_k\) be normal subgroups of \(G\).
    Let \(f \colon N_1 \times \cdots \times N_k \to G\) be defined by
    \((a_1, \cdots, a_k) \mapsto a_1\cdots a_k\).
    If \(f\) is bijective, then \(f\) is a group isomorphism.
}
\pf{Proof}{
    If \(\{1\} \subsetneq N_i \cap N_j\) for some \(i \neq j\), then 
    it contradicts the injectivity of \(f\).
    Hence, by \Cref{lem:normalTrivInter},
    \(a_ia_j = a_ja_i\) for all \(a_i \in N_i\) and \(a_j \in N_j\).

    Take any \((a_1, \cdots, a_k), (b_1, \cdots, b_k) \in N_1 \times \cdots \times N_k\).
    Then,
    \begin{align*}
        f((a_1, \cdots, a_k) (b_1, \cdots, b_k))
        &= f(a_1b_1, a_2b_2, \cdots, a_kb_k) \\
        &= a_1b_1a_2b_2 \cdots a_kb_k \\
        &= a_1 a_2 \cdots a_k b_1 b_2 \cdots b_k \\
        &= f(a_1, \cdots, a_k) f(a_2, \cdots, a_k).
    \end{align*}
    Hence, the result follows.
}

\cor[isoToProdIf]{}{
    Let \(G\) be a group and let \(N_1, N_2, \cdots, N_k\) be normal subgroups of \(G\).
    If
    \begin{enumerate}[nolistsep, label=(\roman*), ref=\protect{(\roman*)}, listparindent=\parindent]
        \ii \(G = N_1 N_2 \cdots N_k\) and
        \ii
        \(\fall i \in [k],\: N_i \cap (N_1 \cdots N_{i-1} N_{i+1} \cdots N_k) = \{1\}\),
    \end{enumerate}
    then \(G \cong N_1 \times N_2 \times \cdots \times N_k\).
}
\pf{Proof}{
    (i) essentially says that \(f\) in \Cref{th:bijectiveProdIso} is surjective.

    Suppose \(a_1a_2\cdots a_k = b_1 b_2 \cdots b_k\) but
    \((a_1, \cdots, a_k) \neq (b_1, \cdots, b_k)\).
    Then,
    \begin{align*}
        b_1\inv a_1
        &= (b_2 \cdots b_k)(a_2 \cdots a_k)\inv \\
        &= b_2 \cdots b_{k-1} b_ka_k\inv a_{k-1}\inv \cdots a_2\inv \\
        \intertext{%
            As \(b_k a_k\inv N_k \nsub G\),
            \((b_ka_k\inv) (a_{k-1}\inv \cdots a_2\inv) = (a_{k-1}\inv \cdots a_2\inv) n_k\)
            for some \(n_k \in N_k\). Therefore, this continues to
        }
        &= b_2 \cdots b_{k-1} a_{k-1}\inv \cdots a_2\inv n_k \\
        \intertext{%
            This continues and we yield
        }
        &= n_2n_3\cdots n_k \in N_2 N_3 \cdots N_{k-1}
    \end{align*}
    for some \(n_2, n_3, \cdots, n_{k-1}\) where \(n_i \in N_i\)
    for each \(i \in \{ 2,3,\cdots,k-1 \}\).
    By (ii), we have \(a_1 = b_1\); and thus \(a_2a_3 \cdots a_k = b_2 b_3 \cdots b_k\).
    We may repeat this and obtain \(a_i = b_i\) for all \(i \in [k]\).
    Hence, the function \(f\) in \Cref{th:bijectiveProdIso} is injective;
    the result follow from \Cref{th:bijectiveProdIso}.
}

\dfn[decomposableGroup]{Decomposable Group}{
    Let \(G\) be a group.
    We say \(G\) is \textit{decomposable}
    if \(G \cong M \times N\) for some nontrivial groups \(M\) and \(N\).
}

\nt{
    If \(G\) is decomposable, then \(G\) has at least four normal subgroups.
}

\cor[orderp2q]{}{
    Let \(G\) be a group of order \(p^2 q\) where \(p\) and \(q\) are distinct primes
    with \(q \not\equiv 1 \pmod{p}\) and \(p^2 \not\equiv 1 \pmod{q}\).
    Then, \(G \cong \ZZ_{p^2 q}\) or \(G \cong \ZZ_{pq} \times \ZZ_p\).
}
\pf{Proof}{
    By \ref{itm:sylow3},
    we have \(n_p \equiv 1 \pmod{p}\), \(n_p \mid q\),
    \(n_q \equiv 1 \pmod{q}\), and \(n_q \mid p^2\).
    By the constraints, we have \(n_p = 1\) and \(n_q = 1\).
    By \Cref{cor:npEqualsOneIff}, the unique \(P \in \Syl_p(G)\)
    and \(Q \in \Syl_q(G)\) are normal in \(G\).
    Moreover, \(P \cap Q = \{1\}\) by \nameref{th:lagrange}.
    By \Cref{th:HKCard}, \(PQ = G\).
    Hence, \(G \cong P \times Q\) by \Cref{cor:isoToProdIf}.
    By \Cref{cor:orderPSquare}, \(P \cong \ZZ_{p^2}\) and \(P \cong \ZZ_{p} \times \ZZ_{p}\).
    The result follows from \Cref{exmp:prodCycleIffGCD}.
}

\exmp[THtkKEBD]{}{
    \begin{enumerate}[nolistsep, label=(\roman*), ref=\protect{\Cref{exmp:THtkKEBD} (\roman*)}]
    \ii Suppose \(\ZZ \cong N \times H\) for some nontrivial normal subgroups \(N, H \nsub \ZZ\).
    However, any intersection of two nontrivial subgroups of \(\ZZ\)
    is nontrivial; thus \(\ZZ\) is indecomposable.

    \ii
    The image of the natural projection \(\ZZ \surjto \ZZ/6\ZZ\)
    is decomposable (\(\ZZ/6\ZZ \cong \ZZ_2 \times \ZZ_3\)) while
    \(\ZZ\) is indecomposable.

    \ii
    \(S_n\) for \(n \ge 5\) is indecomposable.

    \ii\label{itm:dihedralDecompose}
    Let \(n\) be an odd positive integer and consider \(D_{2n}\).
    Let \(M \triangleq \lang\,s, r_1^2\,\rang\) and \(N \triangleq \lang r_1^n \rang\).
    Then, they are nontrivial normal subgroups whose intersection is trivial
    and \(MN = D_{2n}\). Therefore, \(D_{2n} \cong D_n \times \ZZ_2\).
\end{enumerate}
}

\end{document}
