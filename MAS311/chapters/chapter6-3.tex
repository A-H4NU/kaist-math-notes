\documentclass[../modern_algebra.tex]{subfiles}

\begin{document}

\section{Prime and Maximal Ideals}

\begin{Definition}[colbacktitle=red!75!black]{Prime Ideal}{}
    Let \(R\) be a commutative ring.
    A proper ideal \(P\) in \(R\) is a \textit{prime ideal}
    if \(ab \in P\) implies \(a \in P \lor b \in P\).
\end{Definition}

\begin{Theorem}{\textsf{}}{}
    Let \(R\) and \(S\) be commutative rings with identity.
    Let \(f \colon R \to S\) be a ring homomorphism.
    If \(P \subseteq S\) is a prime ideal in \(S\), then \(f\inv(P)\) is a prime ideal in \(R\).
\end{Theorem}
\begin{myproof}[Proof]
    By \Cref{lem:inverseIdeal}, \(f\inv(P)\) is an ideal in \(R\).
    Moreover, as \(1 \notin P\) by \ref{itm:idealUnit}, \(1 \notin f\inv(P)\)
    by \ref{itm:basicRingHomo.i}, and thus \(f\inv(P) \subsetneq R\).

    Take any \(a, b \in R\) such that \(ab \in f\inv(P)\).
    Then, as \(f(a)f(b) = f(ab) \in P\), we have
    \(f(a) \in P\) or \(f(b) \in P\), i.e., \(a \in f\inv(P)\) or \(b \in f\inv(P)\).
\end{myproof}

\begin{Theorem}{\textsf{}}{primeIdealIff}
    Let \(R\) be a commutative ring with identity and let \(P\) be an ideal in \(R\).
    Then, \(P\) is a prime ideal if and only if \(R/P\) is an integral domain.
\end{Theorem}
\begin{myclaim}[Proof]\hfill
\begin{itemize}[nolistsep, wide=0pt, widest={(\(\Rightarrow\))}, leftmargin=*, listparindent=\parindent]
    \ii[(\(\Rightarrow\))]
    \(R/P\) is a commutative ring with identity. \(R/P\) is not trivial as \(P \subsetneq R\).
    Take any \(a, b \in R\) such that \((a + P)(b + P) = 0 + P\).
    Then, \(ab \in P\) and thus \(a \in P\) or \(b \in P\), i.e., \(a + P = 0 + P\) or \(b + P = 0 + P\).
    \ii[(\(\Leftarrow\))]
    \(P \subsetneq R\) as \(R/P\) is not trivial.
    Take any \(a, b \in R\) such that \(ab \in P\).
    Then, we have \((a + P)(b + P) = ab + P = 0 + P\).
    Hence, \(a + P = 0 + P\) or \(b + P = 0 + P\), i.e., \(a \in P\) or \(b \in P\).
    \qed
\end{itemize}
\end{myclaim}

\begin{Definition}[colbacktitle=red!75!black]{Maximal Ideal}{maximalIdeal}
    Let \(R\) be a ring.
    A proper ideal \(M\) in \(R\) is called a \textit{maximal ideal} if
    \(M\) is maximal with respect to inclusion among proper ideals in \(R\).
    In other words, if \(I\) is an ideal in \(R\) such that \(M \subseteq I\),
    then \(I = M\) or \(I = R\).
\end{Definition}

\begin{Theorem}{\textsf{}}{maximalIdealExists}
    Let \(R\) be a ring and let \(I\) be a proper ideal in \(R\).
    There exists a maximal ideal \(M\) of \(R\) such that \(I \subseteq M\).
\end{Theorem}
\begin{myproof}[Proof]
    Let
    \[
        \mcal{J} \triangleq \{\,J \subsetneq R \mid J~\text{is an ideal in}~R~\text{and}~I \subseteq J\,\}.
    \]
    Then, \((\mcal{J}, \subseteq)\) is a poset.
    Let \(\mcal{C}\) be a nonempty chain\footnotemark\ in \((\mcal{J}, \subseteq)\).
    Let \(M_{\mcal{C}} \triangleq \bigcup \mcal{C}\).
    \begin{Claim}{}{Bkvtgtrz}
        \(M_{\mcal{C}} \in \mcal{J}\)
        \tcblower
        It is clear that \(I \subseteq M_{\mcal{C}}\).
        Take any \(a, b \in M_{\mcal{C}}\).
        Then, there exists \(J_a, J_b \in \mcal{C}\) such that \(a \in J_a\) and \(b \in J_b\).
        \WLOG, \(J_a \subseteq J_b\). Then, \(a - b \in J_b \subseteq M_{\mcal{C}}\).

        Take any \(m \in M_{\mcal{C}}\) and \(r \in R\).
        Then, \(m \in J\) for some \(J \in \mcal{C}\)
        so that \(mr, rm \in J \subseteq M_{\mcal{C}}\).
        Hence, \(M_{\mcal{C}}\) is an ideal in \(R\).
        Moreover, \(M_{\mcal{C}}\) is proper since \(1 \notin M_{\mcal{C}}\).
        \qed
    \end{Claim}

    \Cref{clm:Bkvtgtrz} says that \(M_{\mcal{C}}\) is an upper bound of \(\mcal{C}\).
    Therefore, by Zorn's lemma, \(\mcal{J}\) has a maximal element \(M\) with
    respect to the inclusion, which is evidently a maximal ideal in \(R\)
    containing \(I\).
    \footnotetext{A \textit{chain} in a poset \((P, \le)\)
    is a totally ordered subset of \(P\).}
\end{myproof}

\begin{Theorem}{}{maximalIdealIff}
    Let \(R\) be a commutative ring with identity and let \(M\) be a ideal.
    Then, \(M\) is a maximal ideal if and only if \(R/M\) is a field.
\end{Theorem}
\begin{myclaim}[Proof]\hfill
\begin{itemize}[nolistsep, wide=0pt, widest={(\(\Rightarrow\))}, leftmargin=*, listparindent=\parindent]
    \ii[(\(\Rightarrow\))]
    As \(M\) is proper, \(R/M\) is nontrivial commutative ring with identity.
    Take any nonzero element \(a + M \in R/M\). Then, \(a \in R \setminus M\).
    Define
    \[
        J \triangleq \{\,m + ra \mid r \in R \land m \in M\,\}.
    \]
    Take any \(m + ra, m' + r'a \in J\). Then,
    \begin{gather*}
        (m + ra) - (m' + r'a) = (m - m') + (r - r')a \in J \\
        \shortintertext{and}
        r(m' + r'a) = rm' + (r r')a \in J, \\
        (m' + r'a)r = m'r + r'ra \in J.
    \end{gather*}
    Hence, \(J\) is an ideal such that \(M \subsetneq J\) as \(a \in J \setminus M\).
    As \(M\) is maximal, \(J = R\); thus \(1 \in J\).

    There exist \(m \in M\) and \(r \in R\) such that \(1 = m + ra\).
    Then,
    \[
        (r + M)(a + M) = ra + M = 1 + M;
    \]
    hence \(a + M\) is a unit.
    \ii[(\(\Leftarrow\))]
    As \(1 + M \neq 0 + M\), \(1 \notin M\), i.e., \(M\) is a proper ideal by \ref{itm:idealUnit}.
    Let \(J\) be an ideal in \(R\) such that \(M \subsetneq J\).
    There exists some \(a \in J \setminus M\) so that \(a + M \neq 0 + M\).
    Hence, there exists \(b + M \in R/M\) such that \(ab + M = (a + M)(b + M) = 1 + M\),
    i.e., \(m \triangleq ab - 1 \in M \subseteq J\). As \(ab \in J\) as \(a \in J\),
    we have \(1 = ab - m \in J\); hence \(J = R\) by \ref{itm:idealUnit}.
    \qed
\end{itemize}
\end{myclaim}

\begin{Corollary}{}{maximalThenPrime}
    Let \(R\) be a commutative ring with identity.
    Then, every maximal ideal in \(R\) is a prime ideal in \(R\).
\end{Corollary}
\begin{myproof}[Proof]
    Let \(M\) be a maximal ideal in \(R\).
    Then, \(R/M\) is a field by \Cref{th:maximalIdealIff}.
    In particular, \(R/M\) is an integral domain.
    Hence, by \Cref{th:primeIdealIff}, \(M\) is a prime ideal.
\end{myproof}

\begin{Corollary}{\textsf{}}{}
    Let \(R\) be a commutative ring with identity.
    Then, \((0)\) is a maximal ideal if and only if \(R\) is a field.
\end{Corollary}
\begin{myproof}[Proof]
    This directly follows from \(R \cong R/(0)\) and \Cref{th:maximalIdealIff}.
\end{myproof}

\end{document}
