\documentclass[../modern_algebra.tex]{subfiles}

\begin{document}

\section{Lagrange Theorem}

\dfn[]{Congruence}{
    Let \(K \le G\) and \(a, b \in G\).
    We say \textit{\(a\) is congruent to \(b\) modulo \(K\)}
    if \(ab\inv \in K\), and write \(a \equiv b \pmod{K}\).
}

\dfn[coset]{Coset}{
    Let \(K \le G\) and \(a \in G\).
    \begin{itemize}[nolistsep, leftmargin=*, listparindent=\parindent]
        \ii \(Ka \triangleq \{\,ka \mid k \in K\,\}\) is a \textit{right coset of \(K\) in \(G\)}.
        \ii \(aK \triangleq \{\,ak \mid k \in K\,\}\) is a \textit{left coset of \(K\) in \(G\)}.
    \end{itemize}
}

\nt{
    The relation \(\equiv \pmod{K}\) is reflexive, symmetric, and transitive;
    hence it is a equivalence relation.
    Then, the equivalence class of \(a \in G\) is
    \[
        [a]_K = \{\,b \in G \mid b \equiv a \pmod{K}\,\}
        = \{\,b \in G \mid \exs k \in K,\: b = ka\,\}
        = Ka.
    \]
    In other words, \(a \equiv b \pmod{K} \iff Ka = Kb\).

    One may define \(\equiv_l\) by \(a \equiv_l b\) iff \(a\inv b \in K\)
    so that \([a] = aK\).
}

\nt{
    One may note that, if \(K\) is just a nonempty subset of \(G\),
    then \(\equiv \pmod{K}\) is an equivalence relation if and only if \(K \le G\).
}

\dfn[]{}{
    Let \(K \le G\).
    \[
        G/K \triangleq \{\,Ka \mid a \in G\,\}.
    \]
}

\dfn[index]{Index}{
    The \textit{index of \(K\) in \(G\)} is
    \[
        [G \mathord{:} K] \triangleq |G/K|.
    \]
}

\exmp[]{}{
\begin{enumerate}[nolistsep, label=(\roman*), ref=\protect{(\roman*)}, listparindent=\parindent]
    \ii \(n \ZZ \le \ZZ\); \([\ZZ : n \ZZ] = n\).
    \ii \(\ZZ \le \QQ\); \([\QQ : \ZZ] = \infty\).
\end{enumerate}
}

\thm[bijLeftRightCoset]{}{
    Let \(K \le G\). Let \(L\) and \(R\) be sets of left and right cosets, respectively.
    Then, the map
    \begin{align*}
        \vphi\colon R &\longrightarrow L \\
        Ka &\longmapsto a\inv K
    \end{align*} is a (well-defined) bijection.
}
\pf{Proof}{
    Take any \(a, b \in G\) and assume \(Ka = Kb\).
    Then, we have \(b = ka\) for some \(k \in K\).
    Hence, \(a\inv = b\inv k\); thus we have \(a\inv K = b\inv K\).
    Therefore, the function is well-defined.
    Moreover, by a similar argument, \(a\inv K = b\inv K \implies Ka = Kb\);
    thus \(\vphi\) is injective.
    The surjectivity is evident.
}

\nt{
    \Cref{th:bijLeftRightCoset} implies that
    \([G:K] = |\{\,aK \mid a \in G\,\}|\).
}

\mlemma[cosetsSameSize]{}{
    Let \(K \le G\). For each \(a \in G\),
    the function
    \begin{align*}
        f\colon K &\longrightarrow Ka \\
        k &\longmapsto ka
    \end{align*}
    is a bijection.
}
\pf{Proof}{
    \(f\) is evidently surjective.
    If \(ka = f(k) = f(k') = k'a\),
    then we have \(k = k'\).
}

\thm[lagrange]{Lagrange Theorem}{
    Let \(K\) be a finite group and \(K \le G\).
    Then, \([G:K] = |G| / |K|\). (In particular, \(|K| \mid |G|\).)
}
\pf{Proof}{
    Let \(n = [G:K]\)
    and write \(G/K = \{\,Ka_1, Ka_2, \cdots, Ka_n\,\}\).
    By \Cref{lem:cosetsSameSize},
    \(|Ka_i| = |K|\) for all \(i \in [n]\).
    Therefore, \(|G| = \sum_{i=1}^n |Ka_i| = n|K| = [G:K] |K|\).
}

\exmp[]{}{
\(A_n \cycle[\,]{1,2} = \{\,\text{all odd permutations}\,\}\).
Therefore, \([S_n : A_n] = 2\); thus by \nameref{th:lagrange}, \(|A_n| = n!/2\).
}

\nt{
    The converse of \nameref{th:lagrange} (if \(d \mid |G|\), there exists a subgroup of order \(d\))
    does not hold.

    \(|A_4| = 12\).
    Suppose \(K \le A_4\) with \(|K| = 6\).
    Then, there are two right cosets \(K\) and \(Ka\) where \(a \in A_4 \setminus K\).
    (Note that \(Ka = A_4 \setminus K\).)
    Take any \(b \in A_4 \setminus K\).
    If \(b^2 \in Ka = Kb\), then \(b^2 = kb\) for some \(k \in K\), which implies \(b = k \in K\).
    Thus, \(b^2 \in K\). Therefore, \(\fall g \in G,\: g^2 \in K\).
    Hence, for all \(g \in G\) with \(|g| = 3\), then \(g = g^4 = (g^2)^2 \in K\)
    while there are \(8\) elements in \(A_4\) whose order is \(3\),
    which contradicts \(|K| = 6\).
}

\cor[orderDividesOrder]{}{
Let \(G\) be a finite group.
\begin{enumerate}[nolistsep, label=(\roman*), ref=\protect{(\roman*)}, listparindent=\parindent]
    \ii If \(a \in G\), then \(|a| \mid |G|\).
    \ii If \(a^{|G|} = 1\).
\end{enumerate}
}
\pf{Proof}{
    Direct from \nameref{th:lagrange}.
}

\cor[groupWithPrimeOrder]{}{
    Let \(p\) be a prime number.
    Then, every group of order \(p\) is cyclic.
}
\pf{Proof}{
    Fix any \(a \in G \setminus \{1\}\).
    Then, \(1 < |a| \mid p\); thus \(|a| = p\); thus \(G = \lang a \rang\).
}

\cor[lagrangeChain]{}{
    Let \(G\) be a finite group and let \(K \le H \le G\).
    Then, \([G:K] = [G:H][H:K]\).
}
\pf{Proof}{
    \([G:K]|K| = |G| = [G:H]|H| = [H:K] [G:H]|K|\).
}


\end{document}
