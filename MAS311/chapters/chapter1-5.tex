\documentclass[../modern_algebra.tex]{subfiles}

\begin{document}

\section{Cyclic Groups}

\dfn[orderOfElement]{Order}{
    Let \(G\) be a group and let \(a \in G\).
    If \(a^k = 1\) for some \(k \in \ZZ_+\), then we say
    \(a\) has a \textit{finite order} and the
    \textit{order of \(a\)} is
    \[
        |a| = \min \{\,n \in \ZZ_+ \mid a^n = 1\,\}.
    \]
    If \(a\) does not have a finite order,
    we write \(|a| = \infty\).
}

\exmp[]{}{
\begin{enumerate}[nolistsep, label=(\roman*), ref=\protect{(\roman*)}, listparindent=\parindent]
    \ii If \(f \colon G \isoto H\), then \(\fall a \in G,\: |a| = |f(a)|\).
    \ii \(\fall a \in G,\: |a| = |a\inv|\).
    \ii \(\fall a \in G,\: (|a| = 1 \iff a = 1)\).
    \ii \(\fall m \in \ZZ_n,\: |m| = n / \gcd(n, m)\).
    \ii In \(Q_8\), \(|1| = 1\), \(|-1| = 2\), \(|\pm i| = |\pm j| = |\pm k| = 4\).
    \ii In \(D_n\), \(|r_i| = n / \gcd(n, i)\) and \(|s| = 2\).
\end{enumerate}
Note that (v) and (vi) shows that \(Q_8 \ncong D_n\).
}

\thm[basicOrder]{}{
    Let \(G\) be a group. Let \(a, b \in G\).
    \begin{enumerate}[nolistsep, label=(\roman*), ref=\protect{\Cref{th:basicOrder} (\roman*)}, listparindent=\parindent]
        \ii
        \(|a| = \infty \iff \fall i, j \in \ZZ,\: (a^i = a^j \implies i = j)\).

        \ii\label{itm:basicOrder.ii}
        Assume \(|a| = n < \infty\).
        \begin{enumerate}[nolistsep, label=(\arabic*)]
            \ii \(a^k = 1 \iff n \mid k\).
            \ii \(a^i = a^j \iff i \equiv j \pmod{n}\)
            \ii If \(n = td\), then \(|a^t| = d\).
        \end{enumerate}

        \ii\label{itm:basicOrder.iii}
        Assume \(ab = ba\), \(|a| < \infty\), \(|b| < \infty\), and \(\gcd(a, b) = 1\).
        Then, \(|ab| = |a|\,|b|\).
    \end{enumerate}
}
\mclm{Proof}{\hfill
\begin{enumerate}[nolistsep, label=(\roman*), leftmargin=*, listparindent=\parindent]
    \ii Trivial.
    \ii Basic number theory.
    \ii
    Let \(\alpha \triangleq |a|\), \(\beta \triangleq |b|\), and \(\ell = \alpha \beta\).
    Since \((ab)^\ell = 1\), we have \(|ab| \le \ell\).

    Suppose \((ab)^m < 1\) for some \(0 < m < \ell\) for the sake of contradiction.
    Then, we have \(1 = a^{m \alpha} = b^{-m \alpha}\); thus \(\beta \mid m\)
    as \(\gcd(a, b) = 1\). Similarly, we have \(\alpha \mid m\), which
    implies \(\ell = \alpha \beta \mid m\).
    This contradicts \(m < \ell\).
    \qed
\end{enumerate}
}

\nt{
    \noindent
    We do not have \(|ab| = \lcm(|a|, |b|)\).
    In \(D_3\), \(|r_1s| = 2 \neq 6 = \lcm(|r_1|, |s|)\).
}

\cor[]{}{
    Let \(f \colon G \to H\) be a group homomorphism.
    If \(g \in G\) has a finite order,
    then \(|f(g)| \mid |g|\).
}

\cor[allDivideLargestOrder]{}{
    Let \(G\) be an abelian group in which all elements have finite order.
    If \(c \in G\) has the largest order, then \(\fall a \in G,\: |a| \mid |c|\).
}
\pf{Proof}{
    Suppose there exists \(a \in G\) such that \(|a| \nmid |c|\) for the sake of contradiction.
    Then, we may write \(|a| = p^r m\) and \(|c| = p^s n\)
    where \(p\) is a prime number, \(\gcd(m, p) = \gcd(n, p) = 1\), and \(r > s\).
    Then, by \ref{itm:basicOrder.ii},
    \(|a^m| = p^r\) and \(|c^{p^s}| = n\).
    Therefore, by \ref{itm:basicOrder.iii},
    \(|a^m c^{p^s}| = |a^m|\,|c^{p^s}| = p^r n > |c|\),
    which contradicts the maximality of \(|c|\).
}

\dfn[cyclicGroup]{}{
    Let \(G\) be a group.
    Then, a subgroup of \(G\) of the form
    \[
        \lang a \rang = \lang \{a\}\rang = \{\,a^n \mid n \in \ZZ\,\}
    \]
    is called a \textit{cyclic subgroup generated by \(a\)}.
    If \(G = \lang a \rang\), then we say \(G\) is a cyclic group.
}

\nt{
    Every cyclic group is abelian, but the converse is not true.
    (e.g. \ref{itm:finGenExmp.iii})
}

\cor[cycleClassify]{}{
    Let \(G\) be a group and let \(a \in G\).
    \begin{enumerate}[nolistsep, label=(\roman*), ref=\protect{(\roman*)}, listparindent=\parindent]
        \ii If \(|a| = \infty\), then \(\lang a \rang \cong \ZZ\).
        \ii If \(|a| = n\), then \(\lang a \rang \cong \ZZ_n\).
    \end{enumerate}
    This gives the complete classification of cyclic groups.
}

\cor[]{}{
    Let \(G = \lang a \rang\) be a cyclic group.
    Let \(H\) be a nontrivial subgroup of \(G\).
    \begin{enumerate}[nolistsep, label=(\roman*), ref=\protect{(\roman*)}, listparindent=\parindent]
        \ii
        \(H = \lang a^k \rang\) where \(k = \min \{\,n \mid a^n \in H\,\}\).

        \ii
        If \(|a| = \infty\),
        then \(\lang 1 \rang, \lang a \rang, \lang a^2 \rang, \cdots\)
        are all the distinct subgroups of \(G\).

        \ii
        If \(|a| = n < \infty\),
        then \(\min \{\,n \mid a^n \in H\,\} \mid n\).
    \end{enumerate}
}

\mclm{Proof}{\hfill
\begin{enumerate}[nolistsep, label=(\roman*), leftmargin=*, listparindent=\parindent]
    \ii
    As \(a^i \in H\) for some \(i \neq 0\),
    we may let \(k = \min \{\,n \mid a^n \in H\,\}\).

    Take any \(h \in H\). Then, \(h = a^m\) for some \(m \in \ZZ\).
    There exists \(q, r \in \ZZ\) such that \(0 \le r < k\)
    and \(m = kq + r\).
    Then, \(a^r = a^m (a^k)^{-q} \in H\); thus \(r = 0\) by
    minimality of \(k\). Hence, \(H = \lang a^k \rang\).

    \ii
    Trivial.

    \ii
    Let \(d = \gcd(k, n)\). As \(d \mid k\), we have \(\lang a^k \rang \subseteq \lang a^d \rang\).
    There exist \(u, v \in \ZZ\) such that \(d = mu + nv\).
    Then, \(a^d = (a^m)^u (a^n)^v = (a^m)^u\); thus \(\lang a^d \rang \subseteq \lang a^k \rang\).
    Hence, \(k = d \mid n\).
    \qed
\end{enumerate}
}

\exmp[prodCycleIffGCD]{}{
Let \(m, n \in \ZZ_+\). Then, \(\ZZ_m \times \ZZ_n \cong \ZZ_{mn} \iff \gcd(m, n) = 1\).
\begin{itemize}[nolistsep, wide=0pt, widest={(\(\Rightarrow\))}, leftmargin=*, listparindent=\parindent]
    \ii[(\(\Rightarrow\))]
    Suppose \(\gcd(m, n) > 1\) for the sake of contradiction.
    Take any \((a, b) \in \ZZ_m \times \ZZ_n\).
    Then, \(|(a, b)| \mid \lcm(m, n) = mn/\gcd(m,n) < mn\).
    Hence, \(\ZZ_m \times \ZZ_n\) has no element of order \(mn\);
    thus \(\ZZ_m \times \ZZ_n \not\cong \ZZ_{mn}\).

    \ii[(\(\Leftarrow\))]
    As \(|(1, 0)| = m\) and \(|(0, 1)| = n\) in \(\ZZ_m \times \ZZ_n\),
    \(|(1, 1)| = |(1, 0)(0, 1)| = mn\) by \ref{itm:basicOrder.iii}.
    Therefore, \(\ZZ_m \times \ZZ_n = \lang(1,1)\rang \cong \ZZ_{mn}\).
    \qed
\end{itemize}
}

\end{document}
