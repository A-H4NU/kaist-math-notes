\documentclass[../modern_algebra.tex]{subfiles}

\begin{document}

\section{Quotient Rings and Ring Homomorphisms}

\begin{Definition}[colbacktitle=red!75!black]{Quotient Ring}{quotientRing}
    Let \(R\) be a ring and let \(I \subseteq R\) be an ideal in \(R\).
    Then, \(R/I\) equipped with operations
    \begin{gather*}
        (a + I) + (b + I) = (a + b) + I \\
        (a+I) \cdot (b+I) = ab + I
    \end{gather*}
    is a ring and is called the \textit{quotient ring of \(R\) by \(I\)}.
    This is justified by \Cref{lem:idealIff}.
    \vspace*{1em}
    If \(R\) is commutative, then so is \(R/I\).
    If \(R\) has a multiplicative identity, then \(1 + I\) is
    the multiplicative identity of \(R/I\).
    \vspace*{1em}
    There is a surjective ring homomorphism
    \begin{align*}
        \pi \colon R &\longrightarrow R/I \\
        r &\longmapsto r + I
    \end{align*}
    which is called the \textit{natural projection from \(R\) to \(R/I\)}.
\end{Definition}

\begin{Lemma}{}{trivialRingKerIffInj}
    Let \(R\) and \(S\) be rings.
    Let \(f \colon R \to S\) be a ring homomorphism.
    Then, \(\ker(f) = \{0\}\) if and only if \(f\) is injective.
\end{Lemma}
\begin{myproof}[Proof]
    This is a special case of \Cref{th:trivialKerIffInj}
    noting that \(f\) is a group homomorphism from \((R, +)\) to \((S, +)\).
\end{myproof}

\begin{Theorem}{\textsf{First Isomorphism Theorem}}{firstIsoRing}
    Let \(R\) and \(S\) be rings.
    Let \(f \colon R \to S\) be a ring homomorphism.
    Then, \(R/\ker(f) \cong \img(f)\).
\end{Theorem}
\begin{myproof}[Proof]
    Let \(K \triangleq \ker(f)\).
    Define a function
    \begin{align*}
        \vphi \colon R/K &\longrightarrow \img(f) \\
        r + K &\longmapsto f(r).
    \end{align*}
    For each \(r, r' \in R\), we have \(r + K = r' + K\) if and only if \(f(r) = f(r')\)
    as \(f\) is a ring homomorphism.
    Hence, \(\vphi\) is well-defined and injective.
    \(\vphi\) is evidently surjective.
    Therefore, \(\vphi\) is a bijective ring isomorphism.
\end{myproof}

\begin{Definition}[colbacktitle=red!75!black]{}{}
    Let \(R\) be a ring and let \(I\) and \(J\) be ideals of \(R\).
    Then, we define
    \begin{gather*}
        I + J \triangleq \{\,i + j \mid i \in I \land j \in J\,\} \\
        IJ \triangleq \{\,i_1j_1 + i_2j_2 + \cdots + i_nj_n \mid n \in \NN \land \fall k \in [n],\: (i_k \in I \land j_k \in J)\,\}.
    \end{gather*}
\end{Definition}

\begin{Lemma}{\textsf{}}{}
    Let \(R\) be a ring and let \(I\) and \(J\) be ideals of \(R\).
    Then, \(I + J\) and \(IJ\) are ideals of \(R\).
\end{Lemma}
\begin{myclaim}[Proof]\hfill
\begin{enumerate}[nolistsep, label=(\roman*), listparindent=\parindent]
    \ii
    Take any \(i + j, i' + j' \in I + J\) and \(r \in R\). Then,
    \begin{gather*}
        (i + j) - (i' + j') = (i - i') + (j + j') \in I + J \\
        \shortintertext{and}
        r(i + j) = ri + rj \in I + J, \\
        (i + j)r = ir + jr \in I + J.
    \end{gather*}
    Hence, \(I + J\) is an ideal in \(R\).

    \ii
    Take any \(i_1j_1 + \cdots + i_mj_m, i'_1j'_1 + \cdots + i'_nj'_n \in IJ\) and \(r \in J\).
    Then,
    \begin{gather*}
        (i_1j_1 + \cdots + i_mj_m) - (i'_1j'_1 + \cdots + i'_nj'_n)
        = i_1j_1 + \cdots + i_mj_m + (-i'_1)j'_1 + \cdots + (-i'_n) j'_n \in IJ \\
        \intertext{and}
        r(i_1j_1 + \cdots + i_mj_m) = (ri_1)j_1 + \cdots (ri_m)j_m \in IJ, \\
        (i_1j_1 + \cdots + i_mj_m)r = i_1(j_1r) + \cdots i_m(j_mr) \in IJ
    \end{gather*}
    Hence, \(IJ\) is an ideal in \(R\).
    \qed
\end{enumerate}
\end{myclaim}

\begin{Theorem}{\textsf{Second Isomorphism Theorem}}{secondIsoRing}
    Let \(R\) be a ring and let \(I\) and \(J\) be ideals in \(R\).
    Then, \(I \cap J\) is an ideal in \(I\), \(J\) is an ideal in \(I + J\), and
    \(I/(I \cap J) \cong (I + J)/J\).
\end{Theorem}
\begin{myproof}[Proof]
    \(J\) is clearly an ideal in \(I + J\).
    Define a ring homomorphism
    \begin{align*}
        \vphi \colon I &\longrightarrow (I+J)/J \\
        i &\longmapsto i + J.
    \end{align*}
    Then, for any \(i + j \in I + J\), we have \((i + j) + J = i + (j + J) = i + J = \vphi(i)\);
    hence \(\vphi\) is surjective. We also have \(\ker(\vphi) = I \cap J\);
    \(I \cap J\) is an ideal in \(I\).
    Hence, by \nameref{th:firstIsoRing}, \(I/(I \cap J) \cong (I+J)/J\).
\end{myproof}

\begin{Theorem}{\textsf{Third Isomorphism Theorem}}{thirdIsoRing}
    Let \(R\) be a ring and let \(I\) and \(J\) be ideals in \(R\) such that \(J \subseteq I\).
    Then, \(I/J\) is an ideal in \(R/J\). Furthermore, \((R/J)/(I/J) \cong R/I\).
\end{Theorem}
\begin{myproof}[Proof]
    Define a function
    \begin{align*}
        \vphi \colon R/J &\longrightarrow R/I \\
        r + J &\longmapsto r + I.
    \end{align*}
    For each \(r, r' \in R\) such that \(r + J = r' + J\), then \(r - r' \in J \subseteq I\);
    thus \(r + I = r' + I\), hence \(\vphi\) is well-defined.
    It is evident that \(\vphi\) is a surjective ring homomorphism.
    Simply computing the kernel, we have \(\ker(\vphi) = I/J\) and \(I/J\) is an ideal in \(R/J\).
    Hence, by \nameref{th:firstIsoRing}, \((R/J)/(I/J) \cong R/I\).
\end{myproof}

\begin{Lemma}{}{inverseIdeal}
    Let \(R\) and \(S\) be rings.
    Let \(f \colon R \to S\) be a ring homomorphism.
    If \(I \subseteq S\) is an ideal in \(S\), then \(f\inv(I)\) is an ideal in \(R\).
\end{Lemma}
\begin{myproof}[Proof]
    Take any \(a, b \in f\inv(I)\).
    Then, \(f(a - b) = f(a) - f(b) \in I\); hence \(a - b \in f\inv(I)\).
    Moreover, for any \(r \in R\),
    we have \(f(ra) = f(r)f(a) \in I\) and \(f(ar) = f(a)f(r) \in I\); hence \(ar, ra \in f\inv(I)\).
    Hence, \(f\inv(I)\) is an ideal in \(R\).
\end{myproof}

\begin{Theorem}{\textsf{Fourth Isomorphism Theorem}}{fourthIsoRing}
    Let \(R\) be a ring and let \(I\) be an ideal in \(R\).
    Let \(\pi \colon R \surjto R/I\) be the natural projection.
    Then, there is a natural one-to-one correspondence between
    \[
        \{\,\text{ideals of}~R~\text{containing}~I\,\} \xleftrightarrow{\text{1:1}}
        \{\,\text{ideals of}~R/I\,\}
    \]
    with \(K \mapsto K/I\).
\end{Theorem}
\begin{myproof}[Proof]
    Define a function
    \begin{align*}
        \vphi \colon \{\,\text{ideals of}~R~\text{containing}~I\,\} &\longrightarrow \{\,\text{ideals of}~R/I\,\} \\
        K &\longmapsto K/I.
    \end{align*}
    By \nameref{th:thirdIsoRing}, if \(K \subseteq R\) is an ideal containing \(I\),
    then \(\vphi(K) = K/I\) is an ideal in \(R/I\). Hence, \(\vphi\) is well-defined.

    Let \(K, K' \subseteq R\) be ideals in \(R\) containing \(I\) such that \(K \neq K'\).
    Then, there exists \(k \in K \setminus K'\).
    If \(k + I = k' + I\) for some \(k' \in K'\), then \(k = k' + i\) for some \(i \in I\),
    which implies \(k \in K'\), which is a contradiction.
    Hence, \(k + I \neq k' + I\) for all \(k' \in K'\), i.e., \(k + I \in \vphi(K) \setminus \vphi(K')\).
    Therefore, \(\vphi\) is injective.

    Let \(\ol{K}\) be an ideal in \(R/I\).
    then, by \Cref{lem:inverseIdeal}, \(K \triangleq \vphi\inv(\ol{K})\) is an ideal in \(R\).
    Clearly, \(I = \ker(\vphi) = \vphi\inv(\{0\}) \subseteq K\) and \(\vphi(K) = K/I = \ol{K}\).
    Hence, \(\vphi\) is surjective.
\end{myproof}

\end{document}
