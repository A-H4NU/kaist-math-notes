\documentclass[../modern_algebra.tex]{subfiles}

\begin{document}

\section{Group Homomorphisms}

\dfn[groupHomomorphism]{Group Homomorphism}{
    Let \(G\) and \(H\) be groups.
    A \textit{group homomorphism} between \(G\) and \(H\) is a function \(f \colon G \to H\)
    such that \(\fall a, b \in G,\: f(ab) = f(a)f(b)\).
}

\dfn[groupIsomorphism]{Group Isomorphism}{
    Let \(G\) and \(H\) be groups.
    A \textit{group isomorphism} is a bijective group homomorphism
    between \(G\) and \(H\). (This means that \(G\) and \(H\) have the same group structure.)
    We write \(G \cong H\).
}

\thm[]{}{
    Let \(f \colon G \to H\) be a group homomorphism.
    \begin{enumerate}[nolistsep, label=(\roman*), ref=\protect{(\roman*)}, listparindent=\parindent]
        \ii \(f(1_G) = 1_H\).
        \ii \(\fall a \in G,\: f(a\inv) = f(a)\inv\).
        \ii \(\Img f\) is a group under the group operation under \(H\).
        \ii If \(f\) is injective, then \(G \cong \Img f\).
    \end{enumerate}
}
\mclm{Proof}{\hfill
\begin{enumerate}[nolistsep, label=(\roman*), leftmargin=*, listparindent=\parindent]
    \ii
    \(f(1_G) f(1_G) = f(1_G 1_G) = f(1_G) = f(1_G) 1_H\). Hence, we have \(f(1_G) = 1_H\) from
    \ref{itm:basicGroup.i}.

    \ii
    \(f(a\inv) f(a) = f(a\inv a) = f(1_G) = 1_H\) by (i).
    Hence, \(f(a\inv) = f(a)\inv\).

    \ii
    Direct from definition.
    \ii
    Direct from definition.
    \qed
\end{enumerate}
}

\nt{
    There is only one way---the direct product---to give a group structure on \(G \times H\)
    such that both projections are group homomorphisms.
}

\dfn[groupAutomorphism]{Group Automorphism}{
    An \textit{automorphism} of \(G\) is an isomorphism \(G \bijto G\) between \(G\) and itself.
    Then, the collection of all automorhpisms of \(G\),
    \(\Aut(G) \triangleq \{\,\text{automorphisms of}~G\,\}\),
    equipped with \(\circ\), is a group.
    Moreover, \(\Aut(G) \curvearrowleft G\) in the natural way (\((\sigma, g) \mapsto \sigma(g)\)).
}

\exmp[]{}{
    Fix any \(c \in G\) and define \(i_C \colon G \to G\)
    by \(g \mapsto cgc\inv\).
    Then, \(i_C \in \Aut(G)\).
}

\mlemma[actionInducesPermRep]{}{
    Let \(G \actson A\).
    Then, every \(g \in G\) induces a map
    \begin{align*}
       \vphi_g : A &\longrightarrow A \\
        a &\longmapsto ga
    .\end{align*}
    Then, \(\vphi_g \in S(A)\) and \(\vphi \colon G \to S(A)\) defined by \(g \mapsto \vphi_g\)
    is a group homomorphism,
    which is called the \textit{permutation representation of the group action of \(G\) on \(A\)}.
}
\pf{Proof}{
    For each \(a \in A\),
    \((\vphi_{g\inv} \circ \vphi_g)(a) = g\inv(ga) = (g\inv g) a = 1a = a\).
    Thus, \(\vphi_{g\inv} \circ \vphi_g = \vphi_g \circ \vphi_{g\inv} = \id\).
    Therefore, \(\vphi_g \in S(A)\).
    It is easy to show that \(\vphi\) is a group homomorphism.
}

\mlemma[permRepInducesAction]{}{
    Let \(G\) be a group and let \(A\) be a set.
    If \(\vphi \colon G \to S(A)\) is a group homomorphism,
    Then, the map \(G \times A \to A\) defined by
    \((g, a) \mapsto \vphi(g)(a)\) is a group action of \(G\) on \(A\).
}
\pf{Proof}{
    Direct from \Cref{dfn:groupAction}.
}

\thm[actionCorresHoms]{}{
    Let \(G\) be a group and let \(A\) be a nonempty set.
    Then, there exists one-to-one correspondence
    \[
        \text{\{all group actions of \(G\) on \(A\)\}}
        \xleftrightarrow{\text{1-1}}
        \text{\{all group homomorphisms \(G \to S(A)\)\}.}
    \]
}
\pf{Proof}{
    Direct from \Cref{lem:actionInducesPermRep,lem:permRepInducesAction}.
}

\end{document}
