\documentclass[../modern_algebra.tex]{subfiles}

\begin{document}

\section{Definitions and Examples of Rings}

\begin{Definition}[colbacktitle=red!75!black]{Ring}{ring}
    A \textit{ring} is a nonempty set equipped with two binary operations
    ``\(+\)'' and ``\(\cdot\)'' such that for all \(a, b, c \in R\), the following are
    satisfied:
    \begin{enumerate}[nolistsep, label=(\roman*)]
        \ii \((R, +)\) is an abelian group.
        \ii \((a \cdot b) \cdot c = a \cdot (b \cdot c)\).
        \ii \(a \cdot (b + c) = a \cdot b + a \cdot c\) and \((a + b) \cdot c = a \cdot c + b \cdot c\).
    \end{enumerate}
    The additive identity of ring \(R\) is usually denoted \(0\),
    and the additive inverse of \(a \in R\) is usually denoted \(-a\).

    \vspace*{1em}
    A \textit{commutative ring} is a ring \((R, +, \cdot)\) such that
    the following condition is additionally satisfied.
    \begin{enumerate}[nolistsep, label=(\roman*)]
        \setcounter{enumi}{3}
        \ii \(a \cdot b = b \cdot a\) for all \(a, b \in R\).
    \end{enumerate}
    \vspace*{1em}
    A \textit{ring with identity} is a ring \((R, +, \cdot)\) such that
    the following condition is additionally satisfied.
    \begin{enumerate}[nolistsep, label=(\roman*)]
        \setcounter{enumi}{4}
        \ii There exists \(1 \in R\) such that \(1 \cdot a = a \cdot 1 = a\) for all \(a \in R\).
    \end{enumerate}
    \vspace*{1em}
    A \textit{commutative ring with identity} is a ring \((R, +, \cdot)\)
    such that (iv) and (v) are both satisfied.
\end{Definition}

\begin{Theorem}{\textsf{}}{basicRing}
    Let \(R\) be a ring. Then, the following hold.
    \begin{enumerate}[nolistsep, label=(\roman*), ref=\protect{\Cref{th:basicRing} (\roman*)}]
        \ii\label{itm:basicRing.i}
        \(0 \cdot a = a \cdot 0 = 0\) for all \(a \in R\).
        \ii\label{itm:basicRing.ii}
        \(a \cdot (-b) = (-a) \cdot b = -ab\) for all \(a, b \in R\).
        \ii\label{itm:basicRing.iii}
        \((-a) \cdot (-b) = ab\) for all \(a, b \in R\).
        \ii\label{itm:basicRing.iv}
        If \(R\) is a ring with identity, then \((-1) \cdot a = -a\) for all \(a \in R\).
    \end{enumerate}
\end{Theorem}
\begin{myclaim}[Proof]\hfill
\begin{enumerate}[nolistsep, label=(\roman*), listparindent=\parindent]
    \ii
    We have \(0 \cdot a + 0 \cdot a = (0 + 0) \cdot a = 0 \cdot a\);
    hence \(0 \cdot a = 0\). We have \(a \cdot 0 = 0\) similarly.
    \ii
    \(a \cdot b + a \cdot (-b) = a \cdot (b - b) = a \cdot 0 = 0\) by (i).
    Hence, \(a \cdot (-b)\) is the additive inverse of \(a \cdot b\).
    Similarly, \((-a) \cdot b = -ab\).
    \ii
    \((-a)(-b) = -(-a)b = -(-ab) = ab\) by (ii).
    \ii
    \(a + (-1) \cdot a = 1 \cdot a + (-1) \cdot a = (1 - 1) \cdot a = 0 \cdot a = 0\) by (i).
    Hence, \((-1) \cdot a\) is the additive inverse of \(a\).
    \qed
\end{enumerate}
\end{myclaim}

\begin{Theorem}{\textsf{}}{}
    Let \(R\) be a commutative ring with identity.
    If \(1 = 0\), then \(R\) is the trivial ring \(\{0\}\).
\end{Theorem}
\begin{myproof}[Proof]
    For any \(a \in R\), then \(a = 1 \cdot a = 0 \cdot a = 0\) by \ref{itm:basicRing.i}.
\end{myproof}

\begin{Definition}[colbacktitle=red!75!black]{Unit}{unit}
    Let \(R\) be a ring with identity.
    An element \(a \in R\) is a \textit{unit} if \(a\) has a multiplicative inverse,
    i.e., there exists \(u \in R\) such that \(au = ua = 1\).
\end{Definition}

\begin{Definition}[colbacktitle=red!75!black]{Zero Divisor}{zeroDivisor}
    Let \(R\) be a ring.
    \begin{itemize}[nolistsep, leftmargin=*, listparindent=\parindent]
        \ii
        An element \(a \in R \setminus \{0\}\) is a \textit{zero divisor}
        if there exists \(b \in R\) such that \(ab = 0\) or \(ba = 0\).
        \ii
        An element \(a \in R \setminus \{0\}\) is a \textit{nonzero divisor}
        if \(a\) is not a zero divisor.
    \end{itemize}
\end{Definition}

\begin{Definition}[colbacktitle=red!75!black]{Integral Domain}{integralDomain}
    Let \(R\) be a nontrivial commutative ring with identity.
    If \(R\) has no zero divisor, then \(R\) is called an \textit{integral domain}.
\end{Definition}

\begin{Theorem}{\textsf{}}{basicRingId}
    Let \(R\) be a ring with identity. Then, the following hold.
    \begin{enumerate}[nolistsep, label=(\roman*), ref=\protect{\Cref{th:basicRingId} (\roman*)}]
        \ii\label{itm:basicRingId.i}
        If \(u \in R\) is a unit, then it is not a zero divisor.
        \ii\label{itm:basicRingId.ii}
        A multiplicative inverse \(u\inv\) of a unit \(u\) is unique.
        \ii\label{itm:basicRingId.iii}
        If \(a\) is a nonzero divisor and \(ab = ac\) (or \(ba = ca\)), then \(b = c\).
    \end{enumerate}
\end{Theorem}
\begin{myclaim}[Proof]\hfill
\begin{enumerate}[nolistsep, label=(\roman*), listparindent=\parindent]
    \ii
    There is an element \(w \in R\) such that \(uw = wu = 1\).
    Suppose \(uv = 0\) for some \(u \in R\).
    Then, \(0 = w0 = w(uv) = (wu)v = 1v = v\), which is a contradiction.
    It is similar for the case in which \(vu = 0\) for some \(u \in R\).

    \ii
    Assume \(vu = wu = 1\) for some \(v, w \in R\).
    Then, \(0 = vu - wu = (v - w)u\).
    By (i), \(u\) is not a zero divisor, hence \(v - w = 0\), i.e., \(v = w\).

    \ii
    We have \(a(b-c) = 0\) (or \((b - c)a = 0\)).
    As \(a\) is a nonzero divisor, we have \(b - c = 0\), i.e., \(b = c\).
    \qed
\end{enumerate}
\end{myclaim}

\begin{Theorem}{\textsf{}}{finiteCommRingId}
    Every element of a finite commutative ring with identity is
    \(0\), a unit, or a zero divisor.
\end{Theorem}
\begin{myproof}[Proof]
    Let \(R = \{a_1, \cdots, a_n\}\) be a finite commutative ring with identity.
    Take any \(a_t \in R \setminus \{0\}\) and assume \(a_t\) is a nonzero divisor.
    If \(a_i a_t = a_j a_t\), then \(a_i = a_j\) by \ref{itm:basicRingId.iii}, i.e., \(i = j\).
    Therefore, \(a_1 a_t, a_2 a_t, \cdots, a_n a_t\) are all distinct;
    hence \[
        R = \{a_1 a_t, a_2 a_t, \cdots, a_n a_t\}.
    \]
    Thus, there exists \(a_i \in R\) such that \(a_i a_t = 1\);
    hence \(a_t\) is a unit.
\end{myproof}

\begin{Corollary}{\textsf{}}{finiteIntegralDomain}
    A finite integral domain is a field\footnotemark.
    \footnotetext{A \textit{field} is a nontrivial commutative ring \((R, +, \cdot)\)
    with identity in which every nonzero element is a unit.}
\end{Corollary}
\begin{myproof}[Proof]
    Direct from \Cref{th:finiteCommRingId}.
\end{myproof}

\begin{Definition}[colbacktitle=red!75!black]{Subring}{subring}
    Let \(R\) be a ring and let \(S \subseteq R\) be nonempty.
    Then, \(S\) is a \textit{subring} of \(R\)
    if \(S\) is a ring under the binary operations \(+\) and \(\cdot\).
\end{Definition}

\begin{Theorem}{\textsf{}}{subringIff}
    Let \(R\) be a ring and let \(S \subseteq R\) be nonempty.
    Then, \(S\) is a subring of \(R\) if and only if
    \(S\) is closed under subtraction and multiplication.
\end{Theorem}

\end{document}
