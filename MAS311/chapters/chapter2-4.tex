\documentclass[../modern_algebra.tex]{subfiles}

\begin{document}

\section{Simple Groups and Jordan--H\"older Theorem}

\dfn[simpleGroup]{Simple Group}{
    A nontrivial group \(G\) is \textit{simple}
    if \(G\) has only two normal subgroups.
}

\exmp[]{}{
    Let \(G\) be a group and let \(M\) be a proper normal subgroup of \(G\).
    Then, \(M\) is a maximal normal subgroup if and only if \(G/M\) is simple.
    \begin{itemize}[nolistsep, wide=0pt, widest={(\(\Rightarrow\))}, leftmargin=*, listparindent=\parindent]
        \ii[(\(\Rightarrow\))]
        Let \(N \nsub G/M\).
        Let \(H \triangleq \{\,h \in G \mid Mh \in N\,\}\) so that \(M \le H \nsub G\).
        By maximality of \(M\), we have \(H = M\) or \(H = G\),
        that is to say \(N = \{M\}\) or \(N = G/M\).
        \ii[(\(\Leftarrow\))]
        Let \(M \nsub N \nsub G\).
        Then, by \nameref{th:thirdIso}, \(N/M \nsub G/M\);
        thus \(N/M = \{M\}\) or \(N/M = G/M\) as \(G/M\) is simple.
        Therefore, \(N = M\) or \(N = G\).
        \qed
    \end{itemize}
}

\dfn[compositionSeries]{Composition Series}{
    Let \(G\) be a group.
    A sequence of subgroups
    \[
        \{1\} = N_0 \nsub N_1 \nsub \cdots \nsub N_k = G
    \]
    of \(G\) is called a \textit{composition series of \(G\)}
    if \(N_{i}/N_{i-1}\) is simple for each \(i \in [k]\).
    Each \(N_{i+1} / N_i\) is called a \textit{composition factor of \(G\).}
}

\exmp[]{}{
\begin{enumerate}[nolistsep, label=(\roman*), ref=\protect{(\roman*)}, listparindent=\parindent]
    \ii
    \(\{r_0\} \nsub \lang s \rang \nsub \lang s, r_1^2 \rang \nsub D_4\)
    and \(\{r_0\} \nsub \lang r_1^2 \rang \nsub \lang s, r_1^2 \rang \nsub D_4\)
    are two composition series of \(D_4\).

    \ii
    \(\ZZ\) has no composition series because every proper subgroup of \(\ZZ\) is an infinite cyclic group.
\end{enumerate}
}

\thm[jordanHolder]{Jordan--H\"older Theorem}{
Let \(G\) be a nontrivial finite group.
\begin{enumerate}[nolistsep, label=(\roman*), ref=\protect{(\roman*)}, listparindent=\parindent]
    \ii \(G\) has a composition series.
    \ii
    If \((N_0, \cdots, N_r)\) and \((M_0, \cdots, M_s)\) are composition series of \(G\),
    then \(r = s\) and \(\exs \sigma \in S_r\) such that
    \(\fall i \in [r],\: M_{\sigma(i)}/M_{\sigma(i)-1} \cong N_i / N_{i-1}\).
\end{enumerate}
}
\mclm{Proof}{\hfill
\begin{enumerate}[nolistsep, label=(\roman*), leftmargin=*, listparindent=\parindent]
    \ii
    We prove (i) by induction on \(|G|\). It is trivial when \(|G| = 2\).
    Let \(G\) be a finite group with \(|G| \ge 3\).
    If \(G\) is simple, we are done; assume \(G\) is not simple.
    Then, \(G\) has a proper normal subgroup \(N\) which is maximal
    so that \(G/N\) is simple.
    By induction hypothesis, \(N\) admits a composition series.

    \ii
    \WLOG, \(s \ge r\). We proceed with induction on \(r\).
    Since \(r=1\) implies \(G\) is simple and \(s = 1\), we are done;
    hence assume \(r \ge 2\).
    If \(N_{r-1} = M_{s-1}\), then we are done by induction hypothesis.

    Now, assume \(N_{r-1} = M_{s-1}\).
    Then, \(N_{r-1}, M_{s-1} \nsub N_{r-1} M_{s-1} \le G\) by \Cref{cor:HsubNKThen}.
    Moreover, since \(g(nm)g\inv = (gng\inv)(gmg\inv) \in N_{r-1} M_{s-1}\) for all \(g \in G\),
    \(n \in N_{r-1}\), and \(m \in M_{s-1}\),
    we have \(N_{r-1}M_{s-1} \nsub G\).
    Hence, as \(N_{r-1}\) and \(M_{s-1}\) are maximal proper normal subgroups of
    \(G\), and as \(N_{r-1} \neq M_{s-1}\), we have \(N_{r-1}M_{s-1} = G\).
    Define \(H \triangleq H_{r-1} \cap M_{s-1}\) so that \(H \nsub N_{r-1}, M_{s-1}\).
    Then, by \nameref{th:secondIso}, \(G/N_{r-1} = N_{r-1}M_{s-1}/N_{r-1} \cong M_{s-1} / H\)
    and \(G/M_{s-1} = N_{r-1}M_{s-1}/M_{s-1} \cong N_{r-1}/H\),
    and they are simple groups.

    Let \(\{1\} = H_0 \nsub H_1 \nsub \cdots \nsub H_h = H\)
    be a composition series of \(H\). Then,
    \begin{gather*}
        \{1\} = H_0 \nsub H_1 \nsub \cdots \nsub H_h = H \nsub N_{r-1} \\
        \{1\} = H_0 \nsub H_1 \nsub \cdots \nsub H_h = H \nsub M_{s-1}
    \end{gather*}
    are composition series of \(N_{r-1}\) and \(M_{s-1}\), respectively.
    Therefore, by induction hypothesis, \(r - 1 = h + 1 = s - 1\); thus \(r = s\).
    By induction hypothesis again,
    \begin{align*}
        & H_1/H_0, H_2/H_1, \cdots, H_h/H_{h-1}, N_{r-1}/H_h \cong G/M_{s-1} \\
        \text{and}~& N_1/N_0, N_2/N_1, \cdots, N_{r-2}/N_{r-1}, N_{r-1}/N_{r-2}
    \end{align*}
    are the same up to permutation, and
    \begin{align*}
        & H_1/H_0, H_2/H_1, \cdots, H_h/H_{h-1}, M_{s-1}/H_h \cong G/N_{r-1} \\
        \text{and}~& M_1/M_0, M_2/M_1, \cdots, M_{s-2}/M_{s-1}, M_{s-1}/M_{s-2}
    \end{align*}
    are the same up to isomorphism.
    Hence, the result follows.
    \qed
\end{enumerate}
}

\thm[abelianSimple]{}{
    Let \(G\) be an abelian group.
    Then, \(G\) is simple if and only if \(G \cong \ZZ_p\) for some prime number \(p\).
}
\mclm{Proof}{\hfill
\begin{itemize}[nolistsep, wide=0pt, widest={(\(\Rightarrow\))}, leftmargin=*, listparindent=\parindent]
    \ii[(\(\Rightarrow\))]
    Take any \(a \in G \setminus \{1\}\).
    Then, \(\lang a \rang \nsub G\) since \(G\) is abelian.
    As \(G\) is simple, we have \(\lang a \rang = G\).
    Therefore, by \Cref{cor:cycleClassify}, \(\lang a \rang \cong Z_p\) for some prime \(p\).

    \ii[(\(\Leftarrow\))]
    Trivial.
    \qed
\end{itemize}
}

\thm[alterIsSimple]{}{
    \(A_n\) is simple for \(n \ge 5\).
}
\mclm{Proof}{
    \hfill
    \clm[jqiUsOPn]{
        For \(n \ge 3\), \(A_n\) is generated by \(3\)-cycles.
    }{
        There are three types of products of two transpositions. \hfill
        \begin{itemize}[nolistsep, leftmargin=*, listparindent=\parindent]
            \ii \(\cycle[\,]{a,b}\cycle[\,]{c,d} = \cycle[\,]{a,d,b}\cycle[\,]{a,d,c}\)
            \ii \(\cycle[\,]{a,b}\cycle[\,]{a,c} = \cycle[\,]{a,c,b}\)
            \ii \(\cycle[\,]{a,b}\cycle[\,]{a,b} = \cycle[\,]{1}\)
        \end{itemize}
        \noindent
        This is sufficient since every \(\sigma \in A_n\) is a product of even number of transpositions.
        \qed
    }

    \clm[EhVeEjzz]{
        Let \(n \ge 3\) and \(N \nsub A_n\)
        such that \(N\) contains a \(3\)-cycle. Then, \(N = A_n\).
    }{
        \WLOG, \(\cycle[\,]{1,2,3} \in N\). Then,
        \(\cycle[\,]{1,3,2} = \cycle[\,]{1,2,3}^2 \in N\).
        Take any \(k \ge 4\).
        Then,
        \begin{itemize}[nolistsep, leftmargin=*, listparindent=\parindent]
            \ii
            \(\cycle[\,]{1,2,k} = \cycle[\,]{2,k,1} = \tau \cycle[\,]{1,3,2} \tau\inv \in N\)
            where \(\tau = \cycle[\,]{1,2}\cycle[\,]{3,k}\), and
            \ii
            \(\cycle[\,]{2,1,k} = \cycle[\,]{1,k,2} = \tau' \cycle[\,]{1,2,3} (\tau')\inv \in N\)
            where \(\tau' = \cycle[\,]{3,2,k}\).
        \end{itemize}
        All other \(3\)-cycles can be generated by:
        \begin{itemize}[nolistsep, leftmargin=*, listparindent=\parindent]
            \ii
            \(\cycle[\,]{1,a,b} = \cycle[\,]{1,2,b}\cycle[\,]{1,2,a}\cycle[\,]{1,2,a} \in N\),
            \ii
            \(\cycle[\,]{2,a,b} = \cycle[\,]{2,1,b}\cycle[\,]{2,1,a}\cycle[\,]{2,1,a} \in N\), and
            \ii
            \(\cycle[\,]{a,b,c} = \cycle[\,]{1,2,a}\cycle[\,]{1,2,a}\cycle[\,]{1,2,c}
            \cycle[\,]{1,2,b}\cycle[\,]{1,2,b}\cycle[\,]{1,2,a} \in N\).
        \end{itemize}
        Therefore, by \Cref{clm:jqiUsOPn}, \(N = A_n\). \qed
    }

    Take any \(\{(1)\} \lneq N \nsub A_n\) and fix some \(\sigma \in N \setminus \{(1)\}\).
    Consider the cycle decomposition of \(\sigma\).
    There are three cases: (i) some cycle has length \(\ge 4\),
    (ii) the maximum length of cycle is \(3\), and (iii)
    every cycle has length \(\le 2\).
    \begin{enumerate}[nolistsep, label=(\roman*), listparindent=\parindent]
        \ii
        \WLOG, \(\sigma = \cycle[\,]{1,2,\cdots,r} \tau\) where \(r \ge 4\)
        where \(\tau(i) = i\) for each \(i \in [r]\).
        Let \(\delta = \cycle[\,]{1,2,3} \in A_n\).
        Then, we have \(\cycle[\,]{2,3,1,4,5,\cdots,r} \tau = \delta \sigma \delta\inv \in N\).
        Moreover, we have
        \[
            \sigma\inv \cycle[\,]{2,3,1,4,5,\cdots,r} \tau
            = \cycle[\,]{r,r-1,\cdots,1} \cycle[\,]{2,3,1,4,5,\cdots,r} \tau\inv \tau
            = \cycle[\,]{1,3,r} \in N;
        \]
        thus \(N = A_n\) by \Cref{clm:EhVeEjzz}.

        \ii
        We have two subcases: (1) there are (at least) two \(3\)-cycles and
        (2) there are only one \(3\)-cycle.
        \begin{enumerate}[nolistsep, label=(\arabic*), listparindent=\parindent]
            \ii
            \WLOG, \(\sigma = \cycle[\,]{1,2,3}\cycle[\,]{4,5,6} \tau\) where \(\tau\) fixes \([6]\).
            Let \(\delta = \cycle[\,]{1,2,4} \in A_n\).
            Then, \(\cycle[\,]{2,4,3}\cycle[\,]{1,5,6} \tau = \delta \sigma \delta\inv \in N\).
            Hence, we have
            \[
                \sigma\inv \cycle[\,]{2,4,3} \cycle[\,]{1,5,6} \tau
                = \cycle[\,]{3,2,1} \cycle[\,]{6,5,4} \cycle[\,]{2,4,3} \cycle[\,]{1,5,6} \tau\inv \tau = \cycle[\,]{1,4,2,6,3} \in N,
            \]
            which reduces to case (i). Hence, we have \(N = A_n\) in this case.

            \ii
            \WLOG, \(\sigma = \cycle[\,]{1,2,3} \tau\)
            where \(\tau\) fixes \([3]\) and \(\tau\) is a product of disjoint transpositions
            so that \(\tau^2 = 1\).
            Then, we have \(\sigma^2 = \cycle[\,]{1,3,2} \in N\);
            thus \(N = A_n\) by \Cref{clm:EhVeEjzz}.
        \end{enumerate}

        \ii
        \WLOG, \(\sigma = \cycle[\,]{1,2} \cycle[\,]{3,4} \tau\) where \(\tau\) fixes \([4]\)
        and \(\tau\) is a product of disjoint transpositions.
        Let \(\delta = \cycle[\,]{1,2,3} \in A_n\).
        Then, \(\cycle[\,]{2,3} \cycle[\,]{1,4} \tau = \delta \sigma \delta\inv \in N\).
        Therefore,
        \[
            \beta \triangleq \sigma\inv \cycle[\,]{2,3} \cycle[\,]{1,4} \tau
            = \cycle[\,]{1,2} \cycle[\,]{3,4}\cycle[\,]{2,3} \cycle[\,]{1,4} \tau\inv \tau
            = \cycle[\,]{1,3} \cycle[\,]{2,4} \in N.
        \]
        As \(n \ge 5\) we may fix \(5 \le k \le n\) and let \(\alpha = \cycle[\,]{1,3,k} \in A_n\).
        Then, \(\cycle[\,]{3,k} \cycle[\,]{2,4} = \alpha \beta \alpha\inv \in N\).
        Hence,
        \[
            \beta \cycle[\,]{3,k} \cycle[\,]{2,4}
            = \cycle[\,]{1,3} \cycle[\,]{2,4} \cycle[\,]{3,k} \cycle[\,]{2,4}
            = \cycle[\,]{1,3,k} \in N,
        \]
        which implies \(N = A_n\) by \Cref{clm:EhVeEjzz}.
        \qed
    \end{enumerate}
}

\nt{
\begin{itemize}[nolistsep, leftmargin=*, listparindent=\parindent]
    \ii
    \(A_4\) is not simple.

    \ii
    We have two infinite series of simple groups: \(\ZZ_p\)'s (\(p\) is prime) and \(A_n\)'s
    \(n \ge 5\).
\end{itemize}
}

\cor[symmetricThreeNSub]{}{
    For \(n \ge 5\), \(S_n\) has only three normal subgroups \(\{1\}\), \(A_n\), and \(S_n\).
}
\mclm{Proof}{
    By \Cref{lem:indexTwoSubgroup}, we have \(A_n \nsub S_n\).

    Let \(N \nsub S_n\) be a nontrivial normal subgroup of \(S_n\).
    Then, \(N \cap A_n \nsub A_n\).
    By \Cref{th:alterIsSimple}, we have (i) \(N \cap A_n = \{(1)\}\) or (ii) \(N \cap A_n = A_n\).
    \begin{enumerate}[nolistsep, label=(\roman*), ref=\protect{(\roman*)}, listparindent=\parindent]
        \ii
        If \(N \cap A_n = \{1\}\), then \(N \cong N/(N \cap A_n) \cong A_nN/A_n\)
        by \nameref{th:secondIso}.
        As \(|A_nN| | n!\) and \(|A_n| = n!/2\), we have
        \(|N| = |A_nN|/|A_n| = 2\) as we assumed \(N\) is nontrivial.
        Then, \(N = \{(1), \sigma\}\) where \(\sigma^2 = (1)\).
        By \Cref{th:normalSubgroupTFAE}, \(\tau N = N \tau\) for all \(\tau \in S_n\);
        that is to say \(\sigma \tau = \tau \sigma \in S_n\) for all \(\tau \in S_n\).
        This means \(N \le Z(S_n) = \{(1)\}\), which is a contradiction.

        \ii
        Assume \(N \cap A_n = A_n\), i.e., \(A_n \le N\).
        However, by \nameref{th:lagrange},
        \(n!/2 \mid |N| \mid n!\) so that \(N = A_n\) or \(N = S_n\). \qed
    \end{enumerate}
}

\dfn[solvable]{Solvable Group}{
    Let \(G\) be a group.
    We say \(G\) is \textit{solvable}
    if there is a sequence
    \[
        \{1\} = G_n \nsub G_{n-1} \nsub \cdots \nsub G_0 = G
    \]
    of subgroups of \(G\) such that \(G_{i-1}/G_i\) is abelian
    for each \(i \in [n]\).
}

\exmp[]{}{
\begin{itemize}[nolistsep, leftmargin=*, listparindent=\parindent]
    \ii Every abelian group is solvable. (\(G_0 = \{1\}, G_1 = G\))
    \ii \(\{1\} \nsub A_3 \nsub S_3\) and \(A_3\) is abelian; thus \(S_3\) is solvable.
    \ii \(\{1\} \nsub
    \{\,(1), \cycle[\,]{1,2}\cycle[\,]{3,4}, \cycle[\,]{1,3}\cycle[\,]{2,4}, \cycle[\,]{1,4}\cycle[\,]{2,3}\,\}
    \nsub A_4 \nsub S_4\); \(S_4\) is solvable.
    \ii \(S_n\) is not solvable for \(n \ge 5\).
\end{itemize}
}

\thm[GsolvableIff]{}{
    Let \(G\) be a group and \(N \nsub G\).
    Then, \(G\) is solvable if and only if \(N\) and \(G/N\) are solvable.
}
\mclm{Proof}{\hfill
\begin{itemize}[nolistsep, wide=0pt, widest={(\(\Rightarrow\))}, leftmargin=*, listparindent=\parindent]
    \ii[(\(\Rightarrow\))]
    There exists a sequence \(\{1\} = G_n \nsub G_{n-1} \nsub \cdots \nsub G_0 = G\)
    such that \(G_{i-1} / G_i\) is abelian for each \(i \in [n]\).
    Then, we have \(N \cap G_i \nsub G_{i-1}\) and thus \(N \cap G_i \nsub N \cap G_{i-1}\) for each \(i \in [n]\).
    Moreover,
    \[
        (N \cap G_{i-1})/(N \cap G_i) \le G_{i-1} / (N \cap G_i).
    \]
    By \nameref{th:thirdIso}, \(G_i/(N \cap G_i) \nsub G_{i-1}/(N \cap G_i)\) and
    \((G_{i-1}/(N \cap G_i))/(G_i/(N \cap G_i)) \cong G_{i-1}/G_i\).

    Considering the existence of natural projection
    \[
        G_{i-1}/(N \cap G_i) \surjto (G_{i-1}/(N \cap G_i))/(G_i/(N \cap G_i)) \cong G_{i-1}/G_i,
    \]
    there is a group homomorphism
    \[
        \vphi \colon (N \cap G_{i-1})/(N \cap G_i) \longrightarrow G_{i-1}/G_i
    \]
    whose kernel \(\ker(\vphi) = (N \cap G_{i-1})/(N \cap G_i) \cap G_i/(N \cap G_i) = (N \cap G_i)/(N \cap G_i)\)
    is trivial.
    Therefore, \(\vphi\) is injective by \Cref{th:trivialKerIffInj}.
    Hence, \((N \cap G_i)/(N \cap G_i)\) is isomorphic to a subgroup of
    \(G_{i-1}/G_i\), which is abelian.
    Therefore, the sequence
    \[
        \{1\} = N \cap G_n \nsub N \cap G_{n-1} \nsub \cdots \nsub N \cap G_0 = N
    \]
    witnesses that \(N\) is solvable.

    Let \(\pi \colon G \to G/N\) be the natural projection.
    Then, \(\pi(G_i) \nsub \pi(G_i)\) for all \(i \in [n]\).
    The map \(G_{i-1}/G_i \mapsto \pi(G_{i-1})/\pi(G_i)\) defined by \(G_i g_{i-1} \mapsto \pi(G_i) \pi(g_{i-1})\)
    is a surjective group homomorphism; thus \(\pi(G_{i-1})/\pi(G_i)\) is abelian.
    Hence, the sequence
    \[
        \{1\} = \pi(G_n) \nsub \pi(G_{n-1}) \nsub \cdots \nsub \pi(G_0) = G/N
    \]
    witnesses that \(G/N\) is solvable.
    \ii[(\(\Leftarrow\))]
    Let
    \begin{gather*}
        \{1\} = N_s \nsub N_{s-1} \nsub \cdots \nsub N_0 = N \\
        \shortintertext{and}
        \{N\} = \ol{G}_r \nsub \ol{G}_{r-1} \nsub \cdots \nsub \ol{G}_0 = G/N
    \end{gather*}
    be sequences that witnesses the solvability of \(N\) and \(G/N\).
    By \nameref{th:fourthIso}, for each \(j \in [r]\),
    there (uniquely) exists \(G_j \le G\) such that \(N \nsub G_j\) and \(G_j/N = \ol{G}_j\).
    Then, for each \(j \in [r]\), we have \(G_j \nsub G_{j-1}\) by \nameref{th:fourthIso}.
    By \nameref{th:thirdIso},
    \(G_{j-1}/G_j \cong (G_{j-1}/N)/(G_j/N) = \ol{G}_{j-1}/\ol{G}_j\) is abelian;
    thus
    \[
        \{1\} = N_s \nsub N_{s-1} \nsub \cdots \nsub N_0 = N
        = G_r \nsub G_{r-1} \nsub \cdots G_0 = G
    \]
    shows that \(G\) is solvable.
    \qed
\end{itemize}
}

\end{document}
