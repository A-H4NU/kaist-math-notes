\documentclass[../modern_algebra.tex]{subfiles}

\begin{document}

\section{Automorphisms}

\nt{
    Let \(G\) be a group and let \(N \nsub G\).
    We may let \(G \actson N\) by conjugation.
    Then, the permutation representation evaluated at \(g \in G\) is defined by
    \(\vphi_g \colon N \to N\) and \(n \mapsto gng\inv\)
}

\thm[permRepIsAut]{}{
    For each \(g \in G\),
    we have \(\vphi_g \in \Aut(N)\).
    Moreover, \(\ker(\vphi) = C_G(N)\).
    In particular, \(G/C_G(N)\) is isomorphic to a subgroup of \(\Aut(N)\).
}
\pf{Proof}{
    For each \(n_1, n_2 \in N\),
    we have \(\vphi_g(n_1n_2) = gn_1n_2g\inv = gn_1g\inv gn_2g\inv = \vphi_g(n_1) \vphi_g(n_2)\);
    thus \(\vphi_g\) is a group isomorphism as it is already \(\vphi_g \in S(N)\).

    We have
    \[
        \ker(\vphi)
        = \{\,g \in G \mid \fall n \in \NN,\:\vphi_g(n) = n\,\}
        = \{\,g \in G \mid \fall n \in \NN,\:ng = gn\,\} = C_G(N).
    \]
}

\cor[]{}{
    Let \(G\) be a group and let \(H \le G\).
    Then, \(N_G(H)/C_G(H)\) is isomorphic to a subgroup of \(\Aut(H)\).
    In particular, \(G/Z(G)\) is isomorphic to a subgroup of \(\Aut(G)\).
}
\pf{Proof}{
    We have \(H \nsub N_G(H)\), \(C_G(H) = C_{N_G(H)}(H)\), and
    \(N_G(H) = N_{N_G(H)}(H)\).
    The result follows from \Cref{th:permRepIsAut}.
}

\nt{
    Let \(\Inn(G)\) be the set of all inner automorphisms of \(G\).
    Then, \(\Inn(G) \nsub \Aut(G)\)
    since \(\fall \vphi \in \Aut(G),\: \vphi \circ i_c \circ \vphi\inv = i_{\vphi(c)}\).
    We call \(\Aut(G)/\Inn(G)\) the \textit{outer automorphism group}.
}

\cor[innIsoToGModZG]{}{
    Let \(G\) be a group.
    Then, \(\Inn(G) \cong G/Z(G)\).
}
\pf{Proof}{
    Let \(G \actson G\) by conjugation
    so that \(\vphi \colon G \surjto \Inn(G)\) is a permutation representation.
    Then, \(\ker(\vphi) = Z(G)\); the result follows from \nameref{th:firstIso}.
}

\exmp[]{}{
    \begin{itemize}[nolistsep, leftmargin=*, listparindent=\parindent]
        \ii \(\Inn(D_4) \cong \ZZ_2 \times \ZZ_2\)
        \ii \(\Inn(Q_8) \cong \ZZ_2 \times \ZZ_2\)
        \ii \(\Inn(S_n) \cong S_n\) for \(n \ge 3\).
    \end{itemize}
}

\dfn[]{}{
    For each integer \(n \ge 1\), define
    \[
        (\ZZ/n \ZZ)^\ast = \{\,k \in \ZZ_n \mid \gcd(k, n) = 1\,\}
    \]
    so that \((\ZZ/n \ZZ)^\ast\) is a group under usual multiplication.
}

\thm[]{}{
    For each \(n \in \ZZ_+\),
    \(\Aut(\ZZ_n) \cong (\ZZ/n \ZZ)^\ast\).
}
\pf{Proof}{
    Take any \(k \in \ZZ_+\) such that \(\gcd(k, n) = 1\).
    Consider the map \(f_k \colon \ZZ_n \to \ZZ_n\) by \(\ell \mapsto k\ell\).
    Then, clearly, \(f_k \in \Aut(\ZZ_n)\).

    Now, define \(\Phi \colon (\ZZ/n \ZZ)\ast \to \Aut(\ZZ_n)\)
    by \(k \mapsto f_k\).
    Then, it is easy to check \(\Phi\) is an injective group homomorphism.
    Take any \(f \in \Aut(\ZZ_n)\) and let \(k \triangleq f(1)\).
    Then, \(f = f_k\).
}

\nt{
\begin{itemize}[nolistsep, leftmargin=*, listparindent=\parindent]
    \ii \(\lnot (G~\text{is abelian} \implies \Aut(G)~\text{is abelian})\).
    \ii \(\lnot (G~\text{is cyclic} \implies \Aut(G)~\text{is cyclic})\).
\end{itemize}
}

\end{document}
