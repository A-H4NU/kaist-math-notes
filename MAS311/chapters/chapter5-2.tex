\documentclass[../modern_algebra.tex]{subfiles}

\begin{document}

\section{Ring Homomorphisms}

\begin{Definition}[colbacktitle=red!75!black]{Ring Homomorphism}{ringHomomorphism}
    Let \(R\) and \(S\) be groups.
    A \textit{ring homomorphism} between \(R\) and \(S\) is a function \(f \colon R \to S\)
    such that
    \[
        f(a + b) = f(a) + f(b) ~\text{and}~ f(ab) = f(a)f(b)
    \]
    for all \(a, b \in R\).
    The \textit{kernel} of a ring homomorphism \(f\)
    is the set
    \[
        \ker(f) \triangleq \{\,r \in R \mid f(r) = 0\,\}.
    \]
\end{Definition}

\begin{Definition}[colbacktitle=red!75!black]{Ring isomorphism}{ringIso}
    Let \(R\) and \(S\) be groups.
    A \textit{ring isomorphism} between \(R\) and \(S\)
    is a bijective ring homomorphism between \(R\) and \(S\).
    We write \(R \cong S\) if there is a ring isomorphism between \(R\) and \(S\).
    ``\(\cong\)'' is an equivalence relation.
\end{Definition}

\begin{Theorem}{}{basicRingHomo}
    Let \(R\) be a ring with identity and let \(S\) be a ring.
    Let \(f \colon R \surjto S\) be a surjective ring homomorphism.
    Then, the following hold.
    \begin{enumerate}[nolistsep, label=(\roman*), ref=\protect{\Cref{th:basicRingHomo} (\roman*)}]
        \ii\label{itm:basicRingHomo.i}
        \(f(1)\) is the multiplicative identity of \(S\).
        \ii\label{itm:basicRingHomo.ii}
        If \(u\) is a unit in \(R\), then \(f(u)\) is a unit in \(S\) and \(f(u)\inv = f(u\inv)\).
    \end{enumerate}
\end{Theorem}
\begin{myclaim}[Proof]\hfill
\begin{enumerate}[nolistsep, label=(\roman*), listparindent=\parindent]
    \ii
    Take any \(s \in S\). Then, there exists \(r \in R\) such that \(f(r) = s\).
    Then, \(sf(1) = f(r)f(1) = f(r) = s\) and \(f(1)s = f(1)f(r) = f(r) = s\).
    Hence, the result follows.
    \ii
    \(f(u) f(u\inv) = f(1) = 1\) and \(f(u\inv)f(u) = f(1) = 1\) by (i).
    Hence, \(f(u\inv) = f(u)\inv\).
    \qed
\end{enumerate}
\end{myclaim}

\begin{Theorem}{\textsf{}}{}
    Let \(R\) and \(S\) be groups and let \(f \colon R \to S\)
    be a group homomorphism.
    Then, \(\img(f)\) is a subring of \(S\) and \(\ker(f)\) is a subring of \(R\).
\end{Theorem}
\begin{myproof}[Proof]
    \(\img(f)\) and \(\ker(f)\) are a subgroup of \((R, +)\) and \((S, +)\), respectively.

    Take any \(s, s' \in \img(f)\). Then, there exist \(r, r' \in R\)
    such that \(f(r) = s\) and \(f(r') = s'\),
    then \(s s' = f(r) f(r') = f(r r') \in \img(f)\). Hence, \(\img(f)\) is closed under
    multiplication.

    Take any \(r, r' \in \ker(f)\).
    Then, \(f(r r') = f(r) f(r') = 0 \cdot 0 = 0\). Hence, \(\ker(f)\)
    is closed under multiplication. Ther result follows from \Cref{th:subringIff}.
\end{myproof}

\end{document}
