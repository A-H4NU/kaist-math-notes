\documentclass[../modern_algebra.tex]{subfiles}

\begin{document}

\section{Rings of Fractions}

\begin{Definition}[colbacktitle=red!75!black]{Multiplicative Set}{multiplicative}
    Let \(R\) be a commutative ring.
    Then, \(D \subseteq R\) is said to be \textit{multiplicative}
    if every element of \(D\) is a nonzero divisor and \(D\) is
    closed under multiplication.
\end{Definition}

\begin{Lemma}{}{ringOfFractionDef}
    Let \(R\) be a commutative ring and let \(D \subseteq R\) be a multiplicative set.
    Then, the relation \(\sim\) on \(R \times D\) defined by
    \[
        (r, d) \sim (s, e) \iff re = sd
    \]
    is an equivalence relation.
    Moreover, if \(Q \triangleq \{\,[a] \mid a \in R \times D\,\}\) is the set of
    equivalence classes, then the structure \((Q, +, \cdot)\)
    defined by
    \begin{gather*}
        \frac{a}{b} + \frac{c}{d} \triangleq \frac{ad + bc}{bd} \\
        \shortintertext{and}
        \frac{a}{b} \cdot \frac{c}{d} \triangleq \frac{ac}{bd}
    \end{gather*}
    where \(a/b\) denote the equivalence class \([(a, b)]\)
    is well-defined and is a commutative ring with identity such that every element of form \(d/d'\)
    where \(d, d' \in D\) is a unit.
\end{Lemma}
\begin{myproof}[Proof]
    
\end{myproof}

\end{document}
