\documentclass[../modern_algebra.tex]{subfiles}

\begin{document}

\section{Ideals}

\begin{Definition}[colbacktitle=red!75!black]{Congruence}{}
    Let \(R\) be a ring and let \(S\) be a subring of \(R\).
    For \(a, b \in R\),
    We say \textit{\(a\) is congruent to \(b\) modulo \(S\)}
    if \(a - b \in S\), and write \(a \equiv b \pmod{S}\).
\end{Definition}

\begin{Definition}[colbacktitle=red!75!black]{Coset}{}
    Let \(R\) be a ring and let \(S\) be a subring of \(R\).
    Let \(a \in R\).
    As \((R, +)\) is abelian, the left coset \(a + S\) equals the right coset \(S + a\).
    Hence, we call either of them just a \textit{coset} of \(S\).
\end{Definition}

\begin{Definition}[colbacktitle=red!75!black]{}{}
    Let \(R\) be a ring and let \(S\) be a subring of \(R\).
    We define \(R/S\) by
    \[
        R/S \triangleq \{\,a + S \mid a \in R\,\}.
    \]
\end{Definition}

\begin{Lemma}{\textsf{}}{idealIff}
    Let \(R\) be a ring and let \(S\) be a subring of \(R\). Then,
    \begin{multline*}
        \qquad\fall a, a', b, b' \in R,\: (a + S = a' + S \land b + S = b' + S \implies ab + S = a'b' + S) \\
        \iff \fall r \in R,\: \fall s \in S,\: (rs \in S \land sr \in S).\qquad
    \end{multline*}
\end{Lemma}
\begin{myclaim}[Proof]\hfill
\begin{itemize}[nolistsep, wide=0pt, widest={(\(\Rightarrow\))}, leftmargin=*, listparindent=\parindent]
    \ii[(\(\Rightarrow\))]
    Take any \(r \in R\) and \(s \in S\).
    Then, we have \(0 + S = s + S\) and \(r + S = r + S\).
    Hence, by assumption, \(0 + S = 0 \cdot r + S = sr + S\), i.e., \(sr \in S\).
    Similarly, \(rs \in S\).
    \ii[(\(\Leftarrow\))]
    Take any \(a, a', b, b' \in R\) such that \(a + S = a' + S\) and \(b + S = b' + S\).
    This means \(a - a' \in S\) and \(b - b' \in S\)
    so that
    \[
        (a - a')b' = ab' - a'b' \in S~\text{and}~a(b - b') = ab - ab' \in S,
    \]
    which implies \(ab - a'b' = (ab - ab') + (ab' - a'b') \in S\).
    Hence, \(ab + S = a'b' \in S\).
    \qed
\end{itemize}
\end{myclaim}

\begin{Definition}[colbacktitle=red!75!black]{Ideal}{ideal}
    Let \(R\) be a ring and let \(I \subseteq R\) be nonempty.
    Then, \(I\) is an \textit{ideal} of \(R\)
    if \(I\) is a subring of \(R\) and \(ir, ri \in I\) for all \(i \in I\) and \(r \in R\).
\end{Definition}

\begin{Example}{}{JUMqCIrL}
    \begin{enumerate}[nolistsep, label=(\roman*), ref=\protect{\Cref{exmp:JUMqCIrL} (\roman*)}]
    \ii
    For any ring \(R\), then the trivial subring \(\{0\}\)
    is an ideal in \(R\), which is called the \textit{trivial ideal} of \(R\).
    \ii\label{itm:idealUnit}
    For any ring \(R\) with identity and an ideal \(I\) in \(R\),
    \(I = R\) if and only if \(u \in I\) for some unit \(u \in R\).
    For if \(u \in I\) where \(u\) is a unit of \(R\), then \(r = (ru\inv)u \in I\)
    for all \(r \in R\).
\end{enumerate}
\end{Example}

\begin{Corollary}{}{interIdeal}
    Let \(R\) be a group and let \(\lang\,I_i \mid i \in I\,\rang\)
    be an indexed system of ideals of \(R\).
    Then, \(\bigcap_{i \in I} I_i\) is an ideal in \(R\).
\end{Corollary}
\begin{myproof}[Proof]
    Trivial.
\end{myproof}

\begin{Theorem}{\textsf{}}{generateIdeal}
    Let \(R\) be a commutative ring and let \(c_1, c_2, \cdots, c_n \in R\). Then,
    \[
        I \triangleq \{\,r_1c_1 + r_2c_2 + \cdots + r_nc_n \mid r_1, r_2, \cdots, r_n \in R\,\}
    \]
    is an ideal in \(R\).
\end{Theorem}
\begin{myproof}[Proof]
    Simply check.
\end{myproof}

\begin{Definition}[colbacktitle=red!75!black]{}{generatedIdeal}
    In the case of \Cref{th:generateIdeal},
    In this case, \(I\) is said to be \textit{(finitely) generated by} \(c_1, c_2, \cdots, c_n\)
    and is denoted by \((c_1, c_2, \cdots, c_n)\).
    When \(n = 1\), \(I\) is called a \textit{principal ideal} generated by \(c_1\).
\end{Definition}

\begin{note}
    The \textit{smallest ideal} of \(R\) containing \(a \in R\) is
    \[
        \{\,na + ra \mid n \in \ZZ \land r \in R\,\},
    \]
    which equals \((a)\) when \(R\) has an identity.
    If \(R\) a commutative ring without identity,
    then \(a \notin (a)\).
\end{note}

\end{document}
