\documentclass[../MAS242_Note.tex]{subfiles}

\begin{document}

\section{Preliminaries}

\dfn{Normed Vector Space}{
    Let \(V\) be a (real/complex) vector space
    equipped with a norm \(\|\cdot\|\), i.e.,
    the space \((V, \|\cdot\|)\) satisfies the following properties.
    \begin{enumerate}[nolistsep, label=(\roman*)]
        \ii \(0 \in V\)
        \ii \(\|\vec{x}\| \ge 0\) for all \(x \in V\) and \(\|\vec{x}\| = 0\) iff \(\vec{x=0}\).
            (\textit{positive definiteness})
        \ii \(\|\beta\vec{x}\| = |\beta| \cdot \|\vec{x}\|\) for all \(\vec{x} \in V\) and \(\beta \in \RR\).
            (\textit{absolute homogeneity})
        \ii \(\vec{\|x_1 + x_2\| \le \|x_1\| + \|x_2\|}\) for all \(\vec{x_1, x_2} \in V\).
            (\textit{triangle inequality})
    \end{enumerate}
}

\nt{
    \noindent
    Note that \((V, \|\cdot\|)\) is naturally a metric space
    with the metric function \(d(\vec{x}_1, \vec{x}_2) = \|\vec{x}_1 - \vec{x}_2\|\).
}

\dfn{Banach Space}{
    A normed vector space \((V, \|\cdot\|)\)
    is called a \textit{Banach space} if,
    for every Cauchy sequence \(\{x_j\}_{j \in \NN}\),
    there exists a unique \(\vec{x_\ast} \in V\)
    such that \(\lim_{n \to \infty} \|\vec{x}_n - \vec{x}_\ast\| = 0\).
}

\exmp{}{
    Let \(A\) be a compact subset of \(\RR[n]\).
    \((V, \|\cdot\|)\) where
    \(V = \{\,f \colon A \to \RR \mid f \text{ is continuous}\,\}\)
    and \(\|f\| = \sup_{x \in A} |f(x)|\)
    forms a Banach space.
}

\nt{
    A Banach space is a normed vector space
    whose naturally induced metric space is complete.
}

\dfn{Series}{
    Let \((V, \|\cdot\|)\) be a normed vector space.
    Given a sequence \(\{x_j\}_{j \in \NN} \subseteq V\),
    define \(S_k \triangleq \sum_{j=1}^{k} x_j\) for each \(k \in \NN\).
    Then, each \(S_k\) is called a \textit{partial sum} of \(\{x_j\}\).
    If \(\{S_k\}_{k \in \NN}\) converges to \(S_\ast\) with respect to \(\|\cdot\|\),
    then we write
    \[
        S_\ast = \sum_{j=1}^{\infty} x_j.
    \]
    If the limit \(S_\ast\) exists, we symbolically say that
    ``\(\sum_{j=1}^{\infty} x_j\) converges.''
}

\mlemma{}{
    Let \((V, \|\cdot\|)\) be a normed vector space.
    Let \(\{x_j\}_{j \in \NN} \subseteq V\) be a sequence.
    If a series \(\sum_{j=1}^{\infty} x_j\) converges,
    then \(\lim_{k \to \infty} \|\vec{x}_k\| = 0\).
}
\pf{Proof}{
    \(\{S_k\}_{k \in \NN}\) is a Cauchy sequence.
    Hence, \(\lim_{k \to \infty} \|\vec{x}_k\| = \lim_{k \to \infty} \|S_{k+1} - S_k\| = 0\).
}

\mlemma{}{
    Let \((V, \|\cdot\|)\) be a Banach space.
    Let \(\{x_j\}_{j \in \NN} \subseteq V\) be a sequence.
    A series \(\sum_{j=1}^{\infty} x_j\) converges
    if and only if \(\{S_k\}_{k \in \NN}\) is Cauchy.
}
\pf{Proof}{
    The definition of Banach spaces.
} 

\section{Finite Dimensional Banach Spaces}

\thm{Comparison Test}{
    \noindent
    Given two real sequence \(\{a_j\}\) and \(\{b_j\}\),
    suppose \(0 \le a_j \le b_j\) for all \(j \ge k_0\)
    where \(k_0 \in \NN\) is a fixed constant.
    Then, if \(\sum_{j=1}^{\infty} b_j\) converges, then
    \(\sum_{j=1}^{\infty} a_j\) converges.
}
\pf{Proof}{
    Let \(S_k = \sum_{j=k_0}^{k} a_j\) and \(T_k = \sum_{j=k_0}^{\infty} b_j\).
    Then, \(0 \le S_n - S_m = \sum_{j=m+1}^{n} a_j \le \sum_{j=m+1}^{n} b_j
    = T_n - T_m\) whenever \(n \ge m \ge k_0\).
    As \(\{T_k\}_{k \in \NN}\) is Cauchy, 
    \(\{S_k\}_{k \in \NN}\) is Cauchy as well.
    As \((\RR, \|\cdot\|)\) is a Banach space, \(\sum a_j\) converges.
}

\thm{Absolute Convergence Test}{
    \noindent
    Let \((V, \|\cdot\|)\) be a Banach space.
    Let \(\{\vec{x}_j\}_{j \in \NN} \subseteq V\) be a sequence.
    If \(\sum_{j=1}^{\infty} \|\vec{x}_j\|\) converges (in \(\RR\)),
    then \(\sum_{j=1}^{\infty} \vec{x}_j\) converges.
}
\pf{Proof}{
    Let \(S_k = \sum_{j=1}^{k} \vec{x}_j \in V\)
    and \(T_k = \sum_{j=1}^{k} \|\vec{x}_j\| \in \RR\).
    Then, \(\|S_n - S_m\| = \big\| \sum_{j=m+1}^{n} \vec{x}_j \big\|
    \le \sum_{j=m+1}^{n} \|\vec{x}_j\| = T_n - T_m\)
    whenever \(n \ge m\).
    As \(\{T_k\}\) is Cauchy, \(\{S_k\}\) is Cauchy as well.
    Hence, \(\sum \vec{x}_j\) converges.
}

\thm{Summation by Parts}{
    \noindent
    Let \(\{a_j\}\) and \(\{b_j\}\) be two real sequences.
    If \(\sum a_j\) converges and \(\{b_j\}\) is monotonic and convergent,
    then \(\sum_{j=1}^{\infty} a_j b_j\) converges.
}
\pf{Proof}{
    Let \(S_k = \sum_{j=1}^{k} a_j b_j \in V\)
    and \(A_k = \sum_{j=1}^{k} a_j \in \RR\). (\(A_0 = 0\).)
    Then, \(S_k = \sum_{j=1}^{k} (A_j - A_{j-1}) b_j
    = \sum_{j=1}^{k} A_j b_j - \sum_{j=0}^{k} A_{0}b_{j+1} + A_k b_{k+1}
    = A_k b_{k+1} - \sum_{j=1}^{k} A_j (b_{j+1} - b_j)\).

    % We already have \(|A_j| \le M\) for some \(M > 0\),
    % hence \(\big|\sum_{j=1}^{k} A_j(b_{j+1} - b_j)\big| \le M(b_{k+1} - b_1)\).

    Let \(T_k = \sum_{j=1}^{k} |A_j(b_{j+1} - b_j)|\).
    Then, whenever \(n < m\), we have
    \[
        0 \le T_m - T_n \le M \textstyle \sum_{j=n+1}^{m} |b_{j+1}-b_j|
        = M|b_{m+1}-b_{n+1}| \to 0,
    \]
    \(\{T_k\}\) is Cauchy, and thus converges; \(\{S_k\}\) converges as well.
}

\section{Conditional Convergence}

\dfn{Conditional Convergence}{
    Given a real sequence \(\{a_j\}_{j \in \NN} \subseteq \RR\),
    if \(\sum a_j\) converges, and if \(\sum |a_j|\) does not converge,
    then we say that \(\sum a_j\) \textit{converges conditionally}.
}

\thm{Alternating Series Test}{
    Let \(\{a_j\}_{j \in \NN} \subseteq \RR\) be a real sequence.
    If \(a_j \ge 0\) for all \(j \in \NN\),
    and if \(\lim_{j \to \infty} a_j = 0\),
    then \(\sum (-1)^j a_j\) converges.
}
\pf{Proof}{
    MAS101.
}

\exmp{}{
    \noindent
    \(\sum (-1)^j/j\) conditionally converges.
}

\nt{
    \centerline{\fbox{
    Given, a real sequence \(\{a_j\}\),
    we shall use the following definition for now.}}

    \vspace*{.5em}
    For \(j \in \NN\), define
    \[
        a_j^+ \triangleq \frac{|a_j| + a_j}{2} = \begin{cases}
            a_j & \text{if }a_j \ge 0 \\
            0 & \text{if }a_j < 0
        \end{cases} \quad\text{and}\quad
        a_j^- \triangleq \frac{|a_j| - a_j}{2} = \begin{cases}
            0 & \text{if }a_j \ge 0 \\
            -a_j & \text{if }a_j < 0
        \end{cases}\text{.}
    \]
    Then, \(a_j^+, a_j^- \ge 0\), \(|a_j| = a_j^+ + a_j^-\), and \(a_j = a_j^+ - a_j^-\).
}

\mlemma[condConvThenPosDiv]{}{
    Let \(\{a_j\}_{j \in \NN}\) be a real sequence.
    \begin{enumerate}[nolistsep, label=(\roman*)]
        \ii If \(\sum a_j\) converges absolutely,
            then both \(\sum a_j^+\) and \(\sum a_j^-\) converge.
            Moreover, \(\sum a_j = \sum a_j^+ - \sum a_j^-\).
        \ii If \(\sum a_j\) converges conditionally,
            then both \(\sum a_j^+\) and \(\sum a_j^-\) diverge.
    \end{enumerate}
}
\pf{Proof}{
    \hfill
    \begin{enumerate}[nolistsep, label=(\roman*)]
        \ii By the definition of \(a_j^+\) and \(a_j^-\).
        \ii If one of \(\sum a_j^+\) or \(\sum a_j^-\) converges,
            since \(a_j = a_j^+ - a_j^-\), the other converges as well.
            If they both converge, as \(|a_j| = a_j^+ + a_j^-\),
            \(\sum a_j\) converges absolutely.
    \end{enumerate}

}

\dfn{Rearrangement of Series}{
    Let \(\phi \colon \NN \to \NN\) be bijective.
    Given a sequence \(\{a_j\}_{j \in \NN}\),
    the series \(\sum a_{\phi(j)}\) is called a
    \textit{rearrangement} of \(\sum a_j\).
}

\thm[riemannRearrange]{Riemann's Rearrangement Theorem}{
    Let \(\{a_j\}_{j \in \NN}\) be a conditionally convergent real sequence.
    Then, for any given \(-\infty \le \alpha \le \beta \le \infty\) (\(\pm \infty\)
    is allowed for \(\alpha\) and \(\beta\)), there exists a rearrangement \(\phi \colon \NN \to \NN\)
    such that \(\liminf_{k \to \infty} \sum_{j=1}^{k} a_{\phi(j)} = \alpha\)
    and \(\limsup_{k \to \infty} \sum_{j=1}^{k} a_{\phi(j)} = \beta\).
}
\pf{Proof}{
    Let \(\{P_j\}_{j \in \NN}\) and \(\{Q_j\}_{j \in \NN}\) be nonnegative terms and
    absolute value of negative terms of \(\{a_j\}_{j \in \NN}\).
    Then, since they differ from \(\{a_j^+\}\) and \(\{a_j^-\}\) by zero terms,
    they are also divergent by \Cref{lem:condConvThenPosDiv}.

    Let \(\{\alpha_\ell\}_{\ell \in \NN}\) and \(\{\beta_\ell\}_{\ell \in \NN}\) be
    real sequences such that \(\lim_{\ell \to \infty} \alpha_\ell = \alpha\)
    and \(\lim_{\ell \to \infty} \beta_\ell = \beta\).
    Let \(k_1, m_1 \in \NN\) be the smallest integers such that
    \begin{itemize}[nolistsep]
        \ii \(S_1 \triangleq P_1 + \cdots + P_{k_1} > \beta_1\) and
        \ii \(T_1 \triangleq S_1 - (Q_1 + \cdots + Q_{m_1}) < \alpha_1\).
    \end{itemize}
    Inductively, define \(\{k_\ell\}_{\ell \in \NN}\) and \(\{m_\ell\}_{\ell \in \NN}\) by
    \begin{itemize}[nolistsep]
        \ii \(k_{\ell+1} \triangleq \min \big\{\, k \in \NN_{> k_\ell}
            \:\big|\: T_{\ell} + \sum_{j=k_\ell+1}^{k} P_j > \beta_{\ell + 1}\,\big\}\)
        \ii \(S_{\ell+1} \triangleq T_{\ell} + \sum_{j=k_\ell+1}^{k_{\ell+1}} P_j\)
        \ii \(m_{\ell+1} \triangleq \min \big\{\, m \in \NN_{> m_\ell}
            \:\big|\: S_{\ell+1} - \sum_{j=m_\ell+1}^{m} Q_j < \alpha_{\ell + 1}\,\big\}\)
        \ii \(T_{\ell+1} \triangleq S_{\ell+1} - \sum_{j=m_\ell+1}^{m_{\ell+1}} Q_j\)
    \end{itemize}
    for each \(\ell \in \NN\).
    As \(k_\ell \to \infty\) and \(m_\ell \to \infty\) as \(\ell \to \infty\),
    this construction gives the natural rearrangement \(\phi \colon \NN \to \NN\).

    By the construction, we have \(|S_\ell - \beta_\ell| \le P_{k_\ell}\)
    and \(|T_\ell - \alpha_\ell| \le Q_{m_\ell}\) for each \(\ell \in \NN\).
    As \(P_j, Q_j \to 0\) as \(j \to \infty\), we have \(S_\ell \to \beta\) and \(T_\ell \to \alpha\)
    as \(\ell \to \infty\); \(\alpha\) and \(\beta\) are cluster points of
    \(\big\{\sum_{j=1}^k a_{\phi(j)}\big\}_{k \in \NN}\) (as long as they are finite).

    Moreover, for every sufficiently large \(n \in \NN\), we have
    \(k_\ell + m_\ell \le n < k_{\ell+1} + m_{\ell+1}\) for some \(\ell \in \NN\), and thus
    \(\min \{T_\ell, T_{\ell+1}\} \le \sum_{j=1}^{n} a_{\phi(j)} \le S_{\ell+1}\).
    This, or some more rigorous explanation using arbitrary \(\veps \in \RR_+\),
    implies that there do not exist cluster points smaller than \(\alpha\) or greater than \(\beta\).
}

\section{The Cauchy Product}

\dfn{Cauchy Product}{
    Given two real sequences \(\{a_j\}_{j=0}^\infty\) and \(\{b_j\}_{j=0}^\infty\),
    define
    \[
        C_k \triangleq \sum_{j=0}^{k} a_j b_{k-j}.
    \]
    The series \(\sum_{k=1}^{\infty} C_k\) is called the
    \textit{Cauchy product} of \(\sum_{j=0}^{\infty} a_j\) and \(\sum_{j=0}^{\infty} b_j\).
}

\thm[]{}{
    Let \(\{a_j\}_{j=0}^\infty\) and \(\{b_j\}_{j=0}^\infty\) be two real sequences.
    Let \(\sum_{k=0}^{\infty} C_k\) be the Cauchy product of them.
    \begin{enumerate}[nolistsep, label=(\roman*)]
        \ii If \(\sum a_j\) converges absolutely, and if \(\sum b_j\) converges,
            then \(\sum C_k\) converges to \(\big(\sum a_j\big)\big(\sum b_j\big)\).
        \ii If both \(\sum a_j\) and \(\sum b_j\) converges absolutely,
            \(\sum C_k\) converges absolutely as well.
    \end{enumerate}
}
\pf{Proof}{
    (ii) directly follows from the inequality
    \(\sum_{k=0}^{n} |C_k| \le \big(\sum_{j=0}^{n} |a_j|\big) \big(\sum_{j=0}^{n} |b_j|\big)\)
    as long as (i) is proven.

    Let \(S_n \triangleq \sum_{k=0}^{n} C_k\), \(A_n \triangleq \sum_{j=0}^{n} a_j\),
    and \(B_n \triangleq \sum_{j=0}^{n} b_j\).
    Let \(B \triangleq \lim_{n \to \infty} B_n\) and \(\mu_n \triangleq B_n - B\).
    Then,
    \[\begin{aligned}[t]
        S_n &= \sum_{k=0}^{n} C_k = \sum_{k=0}^{n} \sum_{j=0}^{k} b_{k-j}
            = \sum_{j=0}^{n} a_j \sum_{k=j}^{n} b_{k-j} \\
            &= \sum_{j=0}^{n} a_j B_{n-j} = \sum_{j=0}^{n} a_j (B + \mu_{n-j})
            = B \sum_{j=0}^{n} a_j + \sum_{j=0}^{n} a_j \mu_{n-j}.
    \end{aligned}\]

    \mclm{Claim}{
        \(\lim_{n \to \infty} \sum_{j=0}^{n} a_j \mu_{n-j} = 0\).
    }

    Take any \(\veps \in \RR_+\) so there exists \(N \in \NN\) such that
    \begin{itemize}[nolistsep]
        \ii \(|\mu_n| < \veps\) for all \(n \ge N\) (by \(\mu_n \to 0\)) and
        \ii \(\sum_{j=n+1}^{m} |a_j| < \veps\) for all \(m > n \ge N\) (by \(\sum_{j=0}^{k} |a_j|\) being Cauchy).
    \end{itemize}
    
    As \(\mu_n\) converges, there exists \(\mu^\ast \triangleq \sup_{n \in \NN} |\mu_n|\).
    Let \(K_n \triangleq \sum_{j=0}^{n} a_j \mu_{n-j}\).
    Whenever \(n > 2N\),
    \[\begin{aligned}[t]
        |K_n| &\le \sum_{j=0}^{n} |a_j| \cdot |\mu_{n-j}|
        = \sum_{j=0}^{N-1} |a_j| \cdot |\mu_{n-j}| + \sum_{j=N}^{n} |a_j| \cdot |\mu_{n-j}| \\
              &\le \veps \sum_{j=0}^{N-1} |a_j| + \mu^\ast \sum_{j=N}^{n} |a_j|
              \le \veps \left[\sum |a_j| + \mu^\ast\right].
    \end{aligned}\]
    Hence, \(\lim_{n \to \infty} K_n = 0\); thus 
    \(\lim_{n \to \infty} S_n = \big(\sum a_j\big)\big(\sum b_j\big)\).
}

\end{document}
