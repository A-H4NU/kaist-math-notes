\documentclass[../MAS242_Note.tex]{subfiles}

\begin{document}

\section{Functions Defined by Improper Integrals}

\exmp{}{
    Fix a constantt \(r > 0\). On \(\RR\), define
    \[
        F(x) \triangleq \int_{0}^{\infty} e^{-r t} \frac{\sin xt}{t} \d t
        = \int_{0}^{\infty} f(t, x) \d t
    \]
    where \(f(t, x) = e^{-r t} \frac{\sin xt}{t}\).

    (\textit{Is it well-defined?}) We need to check if
    \(\lim_{R \to \infty} \int_{0}^{R} f(t, x) \d t\) exists for all \(x \in \RR\).
    As \(f(t, x)\) is continuous with respect to \(t\),
    we have \(F(x) = \lim_{n \to \infty} F_n(x)\)
    we may only consider the sequence \(F_n(x) = \int_{0}^{n} f(t, x) \d t\). (Proof?)
    For \(m, n \in \NN\) for \(m > n\),
    \[
        |F_m(x) - F_n(x)| \le \int_{n}^{m} \left|e^{r t} \frac{\sin xt}{t}\right| \d t
        \le |x| \int_{n}^{m} e^{r t} \d t \to 0
    \]
    as \(m, n \to \infty\). Hence, \(\{F_n(x)\}_{n \in \NN}\) is Cauchy, and thus is convergent
    for all \(x \in \RR\).

    (\textit{Is it continuous?})
    \[
        |F(x_1) - F(x_2)| \le \int_{0}^{\infty} \frac{e^{-r t}}{t} |\sin x_1t - \sin x_2t| \d t
        \le \frac{|x_1 - x_2|}{r}
    \]
    Hence, \(F\) is Lipschitz continuous (and thus uniformly continuous).

    (\textit{Is it differentiable?})
    If we have differentiability and uniform convergence of \(F_n\),
    by \Cref{th:limDiff}, we have differentiability of \(F\) and \(F' = \lim_{n \to \infty} F_n'\).
    \[
        \textstyle
        F_n'(x) \stackrel{?}{=} \int_{0}^{n} \frac{\partial}{\partial x} f(t, x) \d t
        = \int_{0}^{n} e^{-r t} \cos xt \d t
    \]

    Assuming this, we have, for all \(m > n\), \(|F_m'(x) - F_n'(x)| \le \int_n^m e^{-r t} \d t \to 0\),
    hence \(\{F_n'\}_{n \in \NN}\) is uniformly convergent.
    Therefore, by \Cref{th:limDiff},
    \[
        F'(x) = \lim_{n \to \infty} \left. \frac{-e^{-rt} \cos (xt) / r + x e^{-rt} \sin (xt) / r^2}{1 + (x/r)^2} \right|_{t=0}^n
            = \frac{r}{r^2 + x^2}.
    \]
    Moreover, \(F(0) = 0\); hence \(F(x) = \arctan (x/r)\).
}

\nt{
    If \(g_h(t) = \frac{f(t, x+h) - f(t, x)}{h}\) converges to \(\partial_x f(t, x)\)
    \textit{uniformly} with respect to \(t \in [0, n]\),
    then \(F'(x) = \int_{0}^{n} \partial_x f(t, x) \d t\).
}

\exmp{}{
    Fix \(x \in \RR\) and let \(G(r) = \int_{0}^{\infty} e^{-r t} \frac{\sin xt}{t} \d t\)
    for \(r > 0\).
    Then,
    \[
        \int_{0}^{\infty} \frac{\sin xt}{t} \d t = G(0) = \lim_{r \to 0^+} \arctan \left(\frac{x}{r}\right)
        = \begin{cases}
            \pi/2 & \text{if } x > 0 \\
            0 & \text{if } x = 0 \\
            -\pi/2 & \text{if } x < 0
        \end{cases}.
    \]
}

\exmp[gammaIntro]{}{
    Now, repeat with \(g(t, x) = t^{x-1} e^{-t}\) and \(G(x) = \int_{0}^{1} g(t, x) \d t\).
    Hence, define \(G_n(x) = \int_{1/n}^{n} g(t, x) \d t\).
    For \(n \in \NN\) and \(\sigma \in \RR_+\), we have
    \[
        \left|G_n(x) - \int_{\sigma}^{1} t^{x-1} e^{-t} \right| \le
        \left| \int_{1/n}^{\sigma} t^{x-1} e^{-t} \d x \right|
        = \frac{\sigma^{x}-(1/n)^{x}}{x} \to 0
    \]
    as \(n \to \infty\) and \(\sigma \to 0^+\). Hence, \(G(x) = \lim_{n \to \infty} G_n(x)\).
    \(G(x)\) is well-defined for \(0 < x < 1\).
    \[
        G_n'(x) \stackrel{?}{=} \int_{1/n}^{1} \partial_x g(t, x) \d t 
        = \int_{1/n}^{1} t^{x-1} \ln t e^{-t} \d t
    \]
    as \(\partial_x g(t, x)\) is uniformly continuous on \([1/n, 1]\).
    (The interchange of limit holds since \((g(t, x+h)-g(t,x))/h \rightrightarrows \partial_x g(t, x)\).)

    We claim that, for any fixed \(k \in \NN\) with \(k > 2\), \(\{G_n'(x)\}_{n \in \NN}\) is uniformly Cauchy on
    \(I_k = [2/k, 1)\).
    If the claim is proven, then \Cref{th:limDiff}, \(G'(x) = \int_{0}^{1} t^{x-1} \ln t e^{-t} \d t\)
    for all \(x \in [2/k, 1)\).

    Define an auxiliary function \(H_k(t) \triangleq kt^{-1/k} - |\ln t|\) for \(0 < t < 1\).
    Then, \(H_k'(t) = t^{-1}(1-1/t^{1/k}) < 0\).
    As \(H_k(1) = k\), \(H_k(t) > 0\).
    If \(x \in [2/k, 1)\), we have \(t^{x-1} |\ln t| e^{-t} \le t^{x-1} \cdot k t^{-1/k}
    = k t^{x-1/k-1} \le k t^{1/k - 1}\).
    Therefore, for all \(x \in I_k\),
    \[
        |G_n'(x) - G_m'(x)| \le \int_{1/n}^{1/m} k t^{1/k-1} \d t
        = k^2 \big\{ (1/m)^{1/k} - (1/n)^{1/k} \big\} \to 0
    \]
    as \(m, n \to \infty\). (\(\{G_n'(x)\}_{n \in \NN}\) is uniformly Cauchy on \(I_k\).)
}

\dfn{Gamma Function}{
    The function \(\Gamma \colon \RR_+ \to \RR\) defined by
    \[
        \Gamma(x) = \int_{0}^{\infty} t^{x-1} e^{-t} \d t
    \]
    is called the \textit{Gamma function}.
}

\nt{
    (\textit{Well-defined?})
    For \(x > 1\),
    \[
        |t^{x-1} e^{-t}| = t^{x-1} \cdot \frac{1}{\sum_{j=0}^{\infty} t^j/j!}
        \le t^{x-1} \cdot \frac{1}{t^{\lceil x \rceil + 1}/(\lceil x \rceil + 1)!}.
    \]
}

\thm{Properties of the Gamma Function}{
    Let \(\Gamma\) be the Gamma function.
    \begin{enumerate}[nolistsep, label=(\roman*)]
        \ii \(\Gamma(x+1) = x \Gamma(x)\) for each \(x \in \RR_+\).
        \ii \(\Gamma(n+1) = n!\) for each \(n \in \ZZ_{\ge 0}\).
        \ii \(\ln \Gamma(x)\) is a convex function.
    \end{enumerate}
}
\pf{Proof}{
\hfill
\begin{enumerate}[nolistsep, label=(\roman*)]
    \ii \[\begin{aligned}[t]
        \Gamma(x+1) &= \lim_{R \to \infty} \int_{0}^{R} t^x e^{-t} \d t \\
            &= \lim_{R \to \infty} \left[-t^x e^{-t} \bigg|_{t=0}^R + \int_{0}^{R} x t^{x-1} e^{-t} \d t\right]
            = x \Gamma(x)
        \end{aligned}\]
    \ii Corollary of (i).
    \ii H\"{o}lder's Inequality says that
        \(\int |fg| \d x \le \big(\int |f|^p\big)^{1/p} \big(\int |g|^q\big)^{1/q} \)    
        whenever \(1/p + 1/q = 1\).

        Now, take any \(x, y > 0\) and \(p, q > 1\) such that \(1/p + 1/p = 1\).
        \[\begin{aligned}[t]
            \Gamma \left(\frac{x}{p} + \frac{y}{q}\right)
            &= \int_{0}^{\infty} t^{\frac{x}{p} + \frac{y}{q} - \left(\frac{1}{p}+\frac{1}{q}\right)} e^{-t} \d t
            = \int_{0}^{\infty} \big( t^{\frac{x-1}{p}} e^{-t/p} \big)\big( t^{\frac{y-1}{q}} e^{-t/q} \big) \d t \\
            &\le \left[ \int_{0}^{\infty} t^{x-1} e^{-t} \d t \right]^{1/p}\left[ \int_{0}^{\infty} t^{y-1} e^{-t} \d t \right]^{1/q}
            = \Gamma(x)^{1/p} \Gamma(y)^{1/q},
        \end{aligned}\]
        Hence \(\ln \Gamma(x/p + y/q) \le (1/p) \Gamma(x) + (1/q) \Gamma(y)\).
\end{enumerate}
}

\section{The Laplace Transform}

\dfn[]{Laplace Transform}{
    For a function \(f \colon \RR_{\ge 0} \to \RR\) and for \(s \in \RR\),
    define
    \[
        \mcal L f(s) \triangleq \int_{0}^{\infty} e^{-st}f(t) \d t
        = \lim_{\substack{R \to \infty\\ \sigma \to 0^+}} \int_{\sigma}^{R} e^{-st}f(t)\d t
    \]
    be the \textit{Laplace transform of \(f\) evaluated at \(s\)}.
    The operator \(\mcal L \colon f \mapsto \mcal Lf\) is called the \textit{Laplace transform operator}.
}

\clearpage

\exmp{}{
    \noindent
    Take \(f(t) = 1\) for all \(t \in \RR_+\).
    Then,
    \[
        \mcal Lf(s) = \int_{0}^{\infty} e^{-st} \d t = \begin{cases}
            1/s & \text{if } s > 0 \\
            \text{undefined} & \text{if } s \le 0
        \end{cases}.
    \]
}

\exmp[expLaplace]{}{
    \noindent
    Take \(f(t) = e^{ct}\) for all \(t \in \RR_+\).
    Then,
    \[
        \mcal Lf(s) = \int_{0}^{\infty} e^{-st} \d t = \begin{cases}
            1/(s-c) & \text{if } s > c \\
            \text{undefined} & \text{if } s \le c
        \end{cases}.
    \]
}

\exmp[polyLaplace]{}{
    \noindent
    Take \(f(t) = t^x\) for \(x > -1\) and \(t > 0\).
    Then, for \(s > 0\),
    \[
        \mcal Lf(s) = \int_{0}^{\infty} e^{-st}t^x \d t
        = \frac{1}{s^{x+1}} \int_{0}^{\infty} e^{-u} u^x \d u
        = \frac{\Gamma(x+1)}{s^{x+1}}.
    \]
    \(\mcal Lf(s)\) is undefined for \(s \le 0\).
}

\notat{Translation}{
    For \(f \colon \RR_{\ge 0} \to \RR\) and \(c \in \RR_+\),
    we simply define
    \[
        \tilde f(t-c) = \begin{cases}
            f(t-c) & \text{if }t > c \\
            0 & \text{otherwise}
        \end{cases}.
    \]
}

\mlemma[]{}{
    Let \(f \colon \RR_{\ge 0} \to \RR\) be a continuous function.
    Suppose that \(\mcal Lf(s)\) is well-defined for \(s > r_0\) for some \(r_0 \in \RR\).
    Fix some \(c \in \RR\).
    \begin{enumerate}[nolistsep, label=(\roman*)]
        \ii \(\mcal L \big(e^{ct}f(t)\big)(s) = \mcal Lf(s-c)\) for \(s > r_0 + c\).
        \ii \(\mcal L\big(\tilde f(t-c)\big)(s) = e^{-cs} \mcal Lf(s)\) for \(s > r_0\).
        \ii For \(c > 0\), \(\mcal L \big(f(ct)\big)(s) = (1/c) \mcal Lf (s/c)\)
            for \(s > r_0\).
    \end{enumerate}
}
\pf{Proof}{
    Simple calculation.
}

\mlemma[]{}{
    Given two functions \(f_1, f_2 \in \RR_{\ge 0} \to \RR\),
    suppose that \(\mcal Lf_1(s)\) and \(\mcal Lf_2(s)\) are well-defined
    for \(s > r_0\) for some \(r_0 \in \RR\).
    Then, \(\mcal L \big( c_1f_1 + c_2f_2 \big)(s) = c_1 \mcal Lf_1(s) + c_2 \mcal Lf_2(s)\).
    That is, \(\mcal L\) is a linear operator.
}

\nt{
    Suppose that \(f \colon \RR_{\ge 0} \to \RR\) is \(k\) times differentiable
    and that \(\fall t \ge 0,\: |f^{(k)}(t)| \le A e^{Rt}\) for some \(A, R > 0\).
    Then,
    \[
        |f^{(k-1)}(t)| \le |f^{(k-1)}(0)| + \int_{0}^{t} A e^{R \tau} \d \tau.
    \]
    Thus, there exists \(\tilde A > 0\) such that \(|f^{(k-1)}(t)| \le \tilde{A} e^{Rt}\) for all \(t \ge 0\).
    By induction, we have, for each \(j \in \{\,0,1,\cdots,k-1\,\}\),
    there exists \(A_j \in \RR_+\) such that \(|f^{(j)}(t)| \le A_j e^{Rt}\) for all \(t \ge 0\).
    Hence, \(\mcal L(f^{(j)})(s)\) is well-defined for \(s > R\).
}

\mlemma[diffLaplaceOne]{}{
    Suppose that \(f \colon \RR_{\ge 0} \to \RR\) is differentiable
    and that \(\fall t \ge 0,\: |f'(t)| \le A e^{Rt}\) for some \(A, R > 0\).
    Then, we have \(\mcal L(f')(s) = s \mcal Lf(s) - f(0)\) for \(s > R\).
}
\pf{Proof}{
    Integration by parts.
}

\cor[diffLaplace]{}{
    Suppose that \(f \colon \RR_{\ge 0} \to \RR\) is \(k\) times differentiable
    and that \(\fall t \ge 0,\: |f^{(k)}(t)| \le A e^{Rt}\) for some \(A, R > 0\).
    Then, \(\mcal L(f^{(k)})(s) = s^k \mcal Lf(s) - \sum_{j=0}^{k-1} s^{k-1-j} f^{(j)}(0)\)
    for \(s > R\).
}
\pf{Proof}{
    Induction using \Cref{lem:diffLaplaceOne}.
}

\exmp[homOde]{}{
    \noindent
    Solve \(y'' - y' - 2y = 0\), \(y(0) = 2\), \(y'(0) = 3\) for \(y\).

    Let \(\eta(s) \triangleq \mcal Ly(s)\).
    Applying the Laplace transform to the both sides (without justifying the well-definedness), we get
    \[\begin{aligned}[t]
        0 &= \mcal L(y'') - \mcal L(y') - 2 \mcal L(y) \\
          &= s^2 \eta - (2s+3) - (s\eta - 2) - 2\eta.
    \end{aligned}\]
    Thus,
    \[
        \mcal Ly(s) = \eta(s) = \frac{2s+1}{(s-2)(s+1)}
        = \frac{5}{3} \cdot \frac{1}{s-2} + \frac{1}{3} \cdot \frac{1}{s+1}
    \]
    and it is well-defined for \(s > 2\).
    From \Cref{exmp:expLaplace},
    \[
        \mcal Ly(s) = \mcal L \left(\frac{5}{3}e^{2t} + \frac{1}{3}e^{-t}\right).
    \]
    Now, we shall discuss the \textit{injectivity} of \(\mcal L\).
}

\thm[laplaceIsInjective]{}{
    Given a continuous function \(f \in \RR_{\ge 0} \to \RR\),
    suppose that \(\mcal Lf(s) = 0\) for all \(s > R\) for some \(R \in \RR\).
    Then, \(f = 0\).
}
\pf{Proof}{
    The proof is so sophisticated that it is not discussed in MAS242. :(
}

\nt{
    Actually, the restrictions on the functions in \Cref{th:laplaceIsInjective}
    can be relaxed to not requiring continuity.
}

\dfn[]{Convolution}{
    \begin{itemize}[nolistsep]
        \ii Given two functions \(\phi, \psi \colon \RR \to \RR\),
            we define \(\phi \ast \psi\) by
            \[
                (\phi \ast \psi)(t) \triangleq \int_{-\infty}^{\infty} \phi(x) \psi(t-x) \d x.
            \]
            \(\phi \ast \psi\) is called the \textit{convolution of \(\phi\) and \(\psi\)}.

        \ii Given two functions \(\Phi, \Psi \colon \RR_{\ge 0} \to \RR\), 
            we define \(\Phi \ast \Psi \colon \RR_{\ge 0} \to \RR\) by
            \[
                (\Phi \ast \Psi)(t) \triangleq = \int_{0}^{t} \Phi(x) \Psi(t-x) \d x.
            \]
            \(\Phi \ast \Psi\) is called the \textit{convolution of \(\Phi\) and \(\Psi\)}.
            
    \end{itemize}
}

\nt{
    \noindent
    Convolution is commutative, i.e., \(\phi \ast \psi = \psi \ast \phi\).
}

\thm[convolLaplace]{}{
    Given two continuous functions \(f, g \colon \RR_{\ge 0} \to \RR\),
    suppose that \[\exs A, R > 0,\: \fall t \ge 0,\: \max \{\,|\phi(t)|, |\psi(t)|\,\} \le Ae^{Rt}.\]
    Then,
    \[
        \fall s > R,\: \mcal L (\phi \ast \psi)(s) = \mcal L \phi(s) \cdot \mcal L \psi(s).
    \]
}
\pf{Heuristic Proof}{
    Extend \(\phi, \psi\) to \(\RR\) where \(\phi(t) = \psi(t) = 0\) for \(t < 0\).
    \[\begin{aligned}[t]
        \mcal L(\phi \ast \psi)(s)
        &= \int_{0}^{\infty} e^{-st} (\phi \ast \psi)(t) \d t \\
        &= \int_{0}^{\infty} e^{-st} \int_{0}^{t} \phi(x) \psi(t-x) \d x\d t
         = \int_{0}^{\infty} e^{-st} \int_{0}^{\infty} \phi(x) \psi(t-x) \d x\d t \\
        &= \lim_{\kappa_1 \to \infty} \int_{0}^{\kappa_1} e^{-st} \lim_{\kappa_2 \to \infty} \int_{0}^{\kappa_2} \phi(x) \psi(t-x) \d x \d t \\
        &\,{\color{red}\stackrel{?}{=}}\, \lim_{\kappa_1 \to \infty} \lim_{\kappa_2 \to \infty} \int_{0}^{\kappa_1} e^{-st} \int_{0}^{\kappa_2} \phi(x) \psi(t-x) \d x \d t \\
        &=\lim_{\kappa_1 \to \infty} \lim_{\kappa_2 \to \infty} \int_{0}^{\kappa_2} \int_{0}^{\kappa_1} e^{-st} \phi(x) \psi(t-x) \d t \d x \\
        &=\lim_{\kappa_1 \to \infty} \lim_{\kappa_2 \to \infty} \int_{0}^{\kappa_2} e^{-sx} \phi(x) \int_{0}^{\kappa_1} e^{-s(t-x)} \psi(t-x) \d t \d x \\
        &=\lim_{\kappa_2 \to \infty} \int_{0}^{\kappa_2} e^{-sx} \phi(x) \lim_{\kappa_1 \to \infty} \int_{{\color{red}x}}^{\kappa_1} e^{-s(t-x)} \psi(t-x) \d t \d x \\
        &=\lim_{\kappa_2 \to \infty} \int_{0}^{\kappa_2} e^{-sx} \phi(x) \big[\mcal L \psi(s)\big] \d x = \mcal L \phi(s) \cdot \mcal L \psi(s)
    \end{aligned}\]
}

\exmp{}{
    \noindent
    Solve \(y'' - y' - 2y = f(t)\), \(y(0) = 2\), \(y'(0) = 3\) for \(y\).
    In the same way as in \Cref{exmp:homOde}, we have
    \[
        \mcal Ly(s) = \mcal L \left(\frac{5}{3} e^{2t} + \frac{1}{3} e^{-t}\right)
                    + \mcal Lf(s) \cdot \mcal L \left(\frac{1}{3} e^{2t} + \frac{1}{3} e^{-t}\right).
    \]
    By \Cref{th:convolLaplace}, we get
    \[
        y(t) = \frac{5}{3} e^{2t} + \frac{1}{3} e^{-t} + \int_{0}^{t} f(x) \left[\frac{1}{3} e^{2(t-x)} - \frac{1}{3}e^{-(t-x)}\right] \d x.
    \]
}

\section{Applications of Laplace Transforms}

\nt{
    \noindent
    Reference: \textit{Partial Differential Equations}, Walter Strauss
}

\dfn[]{Laplace Transform}{
    Let \(u \colon I \times \RR_{\ge 0} \to \RR\) where \(I\) is an interval on \(\RR\).
    We define
    \[
        \mcal L u(x, s) \triangleq \int_{0}^{\infty} e^{-st}u(x, t) \d t
    \]
    be the \textit{Laplace transform of \(u\)}.
}

\notat{}{
    \[U(x, s) = \mcal Lu(x, s)\]
}

\qs{1-D Heat Equation}{
    Solve the following partial differential equation
    \[
        \partial_t u = \partial_{xx} u \quad\text{for } 0<x<1 \text{ and } t > 0
    \]
    where
    \begin{itemize}[nolistsep]
        \ii \(u(0, t) = u(1, t) = 1\) for all \(t \ge 0\) and
        \ii \(u(x, 0) = 1 + \sin \pi x\) for all \(0 \le x \le 1\).
    \end{itemize}
}
\sol{
    By \Cref{lem:diffLaplaceOne},
    \[
        \mcal L(\partial_{x x}u)(x,s) = \mcal L(\partial_t u)(x, s) = s \mcal Lu(x, s) - u(x, 0) = sU(x,s) - (1 + \sin \pi x).
    \]
    Also,
    \[
        \mcal L(\partial_{x x}u)(x,s) = \int_{0}^{\infty} e^{-st} \partial_{xx} u(x, t) \d t
        \,{\color{red}\stackrel{?}{=}}\, \partial_{xx} \int_{0}^{\infty} e^{-st} u(x, t) \d t
        = \partial_{xx} U(x, s).
    \]
    From the boundary condition, we have \(U(0, s) = U(1, s) = 1/s\).
    We now notice that we are left with a non-homogeneous second-order ordinary differential equation for \(U(x, s)\)
    with respect to \(x\):
    \[
        U_{x x} - s U = -(1 + \sin \pi x).
    \]

    Let \(U^h\) be the homogeneous solution and \(U^p\) be the particular solution for the ODE (as usual)
    so \(U = U^h + U^p\) is the general solution. (Assume \(s > 0\). {\color{gray}\textit{Why?}})

    It is known that \(U^h = c_1 \cosh (\sqrt{s}x) + c_2 \sinh (\sqrt{s}x)\)
    where \(c_1, c_2 \in \RR\) is the homogeneous solution {\color{gray}from the functional analysis}.
    And, from some calculation, we get
    \[
        U^p = \frac{1}{s} + \frac{1}{\pi^2+s} \sin \pi x.
    \]
    Hence,
    \[
        U(x,s) = c_1 \cosh (\sqrt{s}x) + c_2 \sinh (\sqrt{s}x) + \frac{1}{s} + \frac{1}{\pi^2+s} \sin \pi x
    \]
    is the general solution.
    From the boundary condition, we have
    \[\begin{aligned}[t]
        \frac{1}{s} = U(0, s) &= c_1 + \frac{1}{s} \\
        \frac{1}{s} = U(1, s) &= c_2 \sinh (\sqrt{s}) + \frac{1}{s};
    \end{aligned}\]
    hence \(c_1 = 0\) and \(c_2 = 0\),
    we get
    \[
        U(x, s) = \frac{1}{s} + \frac{1}{\pi^2+s} \sin \pi x
        = \mcal L \left(1 + e^{-\pi^2 t} \sin \pi x\right)(s),
    \]
    i.e., \(u(x, t) = 1 + e^{-\pi^2 t} \sin \pi x\). \checkmark
}

\end{document}
