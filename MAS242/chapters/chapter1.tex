\documentclass[../MAS242_Note.tex]{subfiles}

\begin{document}
\section{Higher order partial derivatives}

\dfn{}{
    Given $f \colon U \to \RR$ where $U$ is an open set in $\RR[m]$,
    define $\partial_{ij} \triangleq \partial_i(\partial_j f)(x)$
    for each $i, j \in [m]$ to be \textit{2nd order partial derivatives}.
    Any higher order partial derivatives can be defined inductively.
}

\dfn{$C^k$-regularity}{
    $f \colon U \to \RR$ is $C^k$-regular if
    all partial derivatives up to order $k$ and they are continuous.
}

\thm[]{}{
    $f \colon U (\subseteq \RR[2]) \to \RR$ is $C^2$ at a point $c \in U$,
    i.e., $\exs \delta > 0$, $f$ is $C^2$ in $B_\delta(c)$.
    Then, $\partial_{12}f(c) = \partial_{21}f(c)$.
}

\pf{Proof}{
    Let $|h| < \delta$.
    Define $A(h) \triangleq f(c_1+h_1, c_2+h_2) - f(c_1+h_1, c_2) - f(c_1, c_2+h_2) + f(c_1, c_2)$.
    Define $u(x_1) \triangleq f(x_1, c_2+h_2) - f(x_1, c_2)$ and
    $v(x_2) \triangleq f(c_1 + h_1, x_2) - f(c_1, x_2)$.
    Note that $u$ and $v$ are differentiable.

    Then, $A(h) = u(c_1 + h_1) - u(c_1)$ and $A(h) = v(c_2+h) - v(c_2)$.
    By MVT, $\exs c_1^\ast \in (c_1, c_1+h_1)$ and $c_2^\ast \in (c_2, c_2+h_2)$ s.t. $A(h) = u'(c_1^\ast) h_1
    = h_1 \big( \partial_1f(c_1^\ast, c_2+h) - \partial_1 f(c_1^\ast, c_2) \big)
    = h_1h_2 \partial_{21}f(c_1^\ast, c_2^\ast)$

    Similarly, $\exs c_1^{\ast\ast}, c_2^{\ast\ast}$ such that $A(h) = h_1h_2 \partial_{12}f(c_1^{\ast\ast}, c_2^{\ast\ast})$.
    $\partial_{21}f(c_1^\ast, c_2^\ast) = \partial_{12}f(c_1^{\ast\ast}, c_2^{\ast\ast})$.
    Hence, as $|h| \to 0$, due to the continuity, $\partial_{21}(c) = \partial_{12}(c)$.
}

\cor{}{
    Suppose $f \colon U (\subseteq \RR[m]) \to \RR$ is $C^k$ at $c \in U$.
    Then $\partial_{j_1j_2\cdots j_k} f(c) = \partial_{j_1'j_2'\cdots j_k'}$
    where $j_1'\cdots$ are a permutation of $j_1\cdots$.
}

\section{Extreme Values of differentiable Functions}

\dfn{Hessian}{
    Let $f \colon U (\subseteq \RR[m]) \to \RR$ be $C_2$ in $U$.
    Suppose $p \in U$ is a critical point of $f$, i.e., $\nabla f(p) = 0$.
    Define \[
        \mcal H f(x) \triangleq \begin{pmatrix}
            \partial_{11} f(x) & \partial_{21}f(x) & \cdots & \partial_{m1}f(x)\\
            \partial_{12} f(x) & \partial_{22}f(x) & \cdots & \partial_{m2}f(x)\\
            \vdots & \vdots & \ddots & \vdots \\
            \partial_{1m} f(x) & \partial_{2m}f(x) & \cdots & \partial_{mm}f(x)\\
        \end{pmatrix}.
    \]
    (Sometimes $\mcal Hf(x) = D^2f(x)$.)

    Define $D(x) = \det \mcal Hf(x)$.
    (Note that $\mcal Hf(x)$ is symmetric when \(f\) is \(C^{2}\) by the theorem above.)
}

\thm[]{2nd-order derivative test for two variable functions.}{
    When \(m = 2\) and \(f\) is \(C^2\), a critical point $p$ is
    \begin{itemize}[nolistsep]
        \ii a local maximum if $D(p) > 0$ and $\partial_{11}f(p) > 0$ (or $\partial_{22}f(p) > 0$).
        \ii a local minimum if $D(p) > 0$ and $\partial_{11}f(p) < 0$ (or $\partial_{22}f(p) < 0$).
        \ii a saddle point if $D(p) < 0$.
    \end{itemize}
    The test fails when \(D(p) = 0\).
}
\pf{Proof}{
    Given a unit vector \(\vec u = (u_1, u_2) \in \RR[2]\),
    \(D_{\vec u} f = \nabla f \cdot \vec u = u_1 \partial_1 f + u_2 \partial_2 f\), and thus
    \[
        D^2_{\vec u} f = (u_1 \partial_1 + u_2 \partial_2) (u_1 \partial_1 f + u_2 \partial_2 f)
        = u_1^2 \partial_{11} f + u_1u_2 (2\partial_{12} f) + u_2^2 \partial_{22}f\text{.}
    \]

    \textsf{WLOG}, \(u_1 \neq 0\). Set \(z = u_2 / u_1\).
    Then,
    \[
        D^2_{\vec u} f(p) = u_1^2 \big(\partial_{11}f(p) +
        2\partial_{12} f(p) z + \partial_{22} f(p)z^2\big)\text{.}
    \]
    Note that, if \(D(p) > 0\), \(D^2_{\vec u} f(p)\) has no real root.
    \begin{itemize}[nolistsep]
        \ii If \(D(p) > 0\) and \(\partial_{11}f(p) < 0\),
            Then, \(D^2{\vec u} < 0\) for all unit vector \(\vec u\).
        \ii If \(D(p) > 0\) and \(\partial_{11}f(p) > 0\),
            Then, \(D^2{\vec u} > 0\) for all unit vector \(\vec u\).
        \ii If \(D(p) < 0\), \(D^2_{\vec u} f(p)\) has different signs depending on \(\vec u\).
    \end{itemize}

    For general \(m\)?
    \[\begin{aligned}[t]
        D_{\vec u} (D_{\vec u} f) = D_{\vec u} \sum_{j = 1}^{m} \partial_j f u_j
                                  = \sum_{j=1}^m \big((\nabla \partial_j f) \cdot \vec u\big) u_j
                                  = \sum_{j=1}^{m} \sum_{k=1}^{m} u_ku_j \partial_{kj}f\text{.}
    \end{aligned}\]
    Hence, \[
        D^2_{\vec u} f(p) = \vec u^\mrm{T} \cdot D^2f(p) \cdot \vec u
    \]
    Since \(D^2f(p)\) is symmetric, its eigenvalues \(\lambda_1, \cdots, \lambda_m\) exists
    and they are real numbers.
    Also, there exists an \(m \times m\) orthogonal matrix \(\mcal O\) such that
    \(D^2f(p) = \mcal O \Lambda(p) \mcal O^\mrm{T}\) where \(\Lambda(p)\) is the diagonal
    matrix with entries are the eigenvalues.
    
    Then, we can write \(D^2_{\vec u}f(p) = \vec u \mcal O \Lambda(p) \mcal O^\mrm{T} \vec u^\mrm{T}
    = (\vec u \mcal O) \Lambda(p) = (\vec u \mcal O) ^\mrm{T}\).
    Since \(\mcal O\) is orthogonal, \(\vec u \mcal O\) is another arbitrary unit vector.
}

\thm[]{Generalized 2nd order partial derivatives test}{
    When \(f\) is \(C^2\), a critical point $p$ is
    \begin{itemize}[nolistsep]
        \ii a local maximum if all eigenvalues of \(D^2f(p)\) are negative.
        \ii a local minimum if all eigenvalues of \(D^2f(p)\) are positive.
        \ii a saddle point if there are both negative eigenvalues and positive eigenvalues.
    \end{itemize}
    The test fails when there are zero eigenvalues.
}


\end{document}
