\documentclass[../MAS242_Note.tex]{subfiles}

\begin{document}
\section{Jacobian}

\dfn{Jacobian}{
    Let \(\vec f \colon U (\subseteq \RR[m]) \to \RR[n]\) be differentiable.
    The function \(J_f \colon U \to \RR\) defined by \[
        J_{\vec f}(\vec x) = \det \begin{bmatrix}
            \partial_1 f_1(\vec x) & \cdots & \partial_n f_1(\vec x) \\ 
            \vdots & \ddots & \vdots \\ 
            \partial_1 f_n(\vec x) & \cdots & \partial_n f_n(\vec x)
        \end{bmatrix}
    \]
    is called the \textit{Jacobian} of \(\vec f\) at \(\vec x\).
}

\mlemma{}{
    If \(f \colon V (\subseteq \RR[n]) - \RR\) and \(\vec g \colon U \to V\)
    are differentiable, then \[
        J_{f \circ \vec g}(\vec x) = J_f \big(\vec g(\vec x)\big) \cdot J_{\vec g}(\vec x)\text{.}
    \]
}

\nt{
    The linear mapping \(\d\vec f(c)\) is invertible if and only if \(J_{\vec f}(c)\) is nonzero.
}

\section{The Inverse Function Theorem}

\mlemma[contractionMappingP]{Contraction Mapping Principle}{
    Let \((X, d)\) be a complete metric space.
    Let \(\varphi \colon X \to X\). Suppose that there exists \(M \in [0, 1)\) such that
    \(d\big(\varphi(x_1), \varphi(x_2)\big) \le M d(x_1, x_2)\).
    (We call it a \textit{contraction mapping}.)

    Then, there uniquely exists \(x_{\ast} \in X\) such that
    \(\varphi(x_\ast) = x_\ast\).
}
\pf{Proof}{
    Fix any \(x_0 \in X\). Since \(\{x_j\}_{j \in \ZZ_+}\), where \(x_{j} = \varphi(x_{j-1})\)
    for each \(j \in \ZZ_+\), is continuous. It converges to some \(x_\ast\).
    As \(\varphi\) is continuous, we have \(\varphi(x_\ast) = x_\ast\).
    The uniqueness follows trivially.
}

\nt{
    \begin{itemize}[nolistsep]
        \ii For each \(v \in \RR[n] \setminus \{0\}\),
            \(|Av| = |v|\cdot \big|A \frac{v}{|v|}\big| \le \|A\|_{L} \cdot |v|\).
            The result is trivial when \(v = 0\).
        \ii For each \(u \in \RR[n]\) with \(|u| = 1\),
            \(|ABu| \le \|A\|_L |Bu| \le \|A\|_L \|B\|_L\).
            Hence, \(\|AB\|_L = \|A\| \|B\|\).
        \ii Given invertible \(A \in L(\RR[n]. \RR[n])\), \(A\inv \colon \RR[n] \to \RR[n]\)
            is linear. Moreover, \(\|A\|_L > 0\).
    \end{itemize}
}

\mlemma{}{
    Given two linear mappings \(A, B \colon \RR[n] \to \RR[n]\) with invertibility of \(A\),
    \[
        \|A-B\|_L \|A\inv\|_L < 1 \implies B \text{ is invertible.}
    \]
}
\pf{Proof}{
    (Hint: show that \(B\vec x = 0\) has only the trivial solution, i.e.,
    if \(\vec x \neq 0\), then \(B\vec x \neq 0\).)
}

\thm[]{Inverse Function Theorem}{
    Let \(\vec f \colon E (\subseteq \RR[n]) \to \RR[n]\) be \(C^1\) in \(U\),
    \(\vec a \in E\), and \(\vec b = f(a)\).
    Suppose that \(J_{\vec f}(a) \neq 0\). Then,
    \[
        \exs \delta \in \RR_+,\: \vec f \big|_{B_{\delta}(a)} \colon B_{\delta}(a) \to \vec f\big(B_{\delta}(a)\big) \text{ is invertible.}
    \]
    Moreover, \(\vec f \big(B_{\delta}(a)\big)\) is an open set, and
    \(\big(\vec f \big|_{B_{\delta}(a)}\big)\inv\) is \(C^1\).
}
\pf{Proof}{
    Let \(A \triangleq \d\vec f(\vec c)\).
    Define \(\lambda\) by
    \(\lambda \triangleq \dfrac{1}{2 \| A\inv \|_L} > 0\)
    so \(2 \lambda \big\|A\inv\big\|_L = 1\).
    Since \(\d\vec f\) is continuous, there exists \(\delta \in \RR_+\) such that
    \(\|\d\vec f(\vec x) - \d\vec f(\vec c)\|_L < \lambda\) for each \(B_{\delta}(\vec c)\).

    Given a point \(\vec y \in \RR[n]\), we define \(\varphi(\cdot; \vec y)\) by
    \begin{align*}
        \varphi(\cdot; \vec y): B_\delta(\vec c) &\longrightarrow \RR[n] \\
        \vec x &\longmapsto \varphi(\vec x; \vec y) = \vec x + A\inv(\vec y - \vec f(\vec x))
    \end{align*}
    Note that \(\vec x\) is a fixed point of \(\varphi(\cdot; \vec y)\) if and only if
    \(A\inv (\vec y - \vec f(\vec x)) = 0\), i.e., \(\vec y = \vec f(\vec x)\).
    Note also that \(\varphi\) is differentiable and
    \(\d\varphi(\vec x; \vec y) = \mrm{Id} - A\inv \d\vec f(\vec x) = A\inv \big(A - \d\vec f(\vec x)\big)\)
    for each \(x \in B_{\delta}(\vec c)\). Let \(U \triangleq B_{\delta}(\vec c)\) and
    \(V \triangleq \vec f(U)\).

    Hence, for all \(x \in U\),
    \[
        \|\d\varphi(\vec x; \vec y)\|_L
        = \big\|A\inv \big(A - \d\vec f(\vec x)\big)\big\|_L
        \le \|A\inv\|_L \cdot \|A - \d\vec f(\vec x)\|_L 
        < 1/(2\lambda) \cdot \lambda = 1/2\text{.}
    \]

    Now, fix any \(\vec y \in V\).
    Fix \(\vec x_1, \vec x_2 \in U\). Define \(\Psi \colon [0, 1] \to \RR\) by
    \(t \mapsto \varphi(t\vec x_1 + (1-t)\vec x_2; \vec y)\).
    \(\Psi(0) = \varphi(\vec x_2; \vec y)\) and \(\Psi(1) = \varphi(\vec x_1; \vec y)\).
    Note that \(\Psi\) is differentiable on \((0, 1)\).
    By MVT, there exists \(t_\ast \in (0, 1)\) such that
    \(\Psi(1) - \Psi(0) = \Psi'(t_\ast)\).
    The chain rule gives \[\Psi'(t_\ast) = \d \varphi(t_\ast \vec x_1 + (1-t_\ast)\vec x_2) (\vec x_1 - \vec x_2).\]
    Hence,
    \[
        |\varphi(\vec x_1) - \varphi(\vec x_2)|
        = |\d \varphi(t_\ast \vec x_1 + (1-t_\ast)\vec x_2)| \cdot |(\vec x_1 - \vec x_2)|
        \le |\vec x_1 - \vec x_2| / 2.
    \]

    We want to show that \(\vec f\) is locally invertible.
    It suffices to show that it is injective.
    Hence, \(\varphi\) has at most one fixed point,
    i.e., there exists at most one \(\vec x\) such that \(\vec y = \vec f(\vec x)\);
    thus \(\vec f\) is injective on \(U\).

    Let \(\vec x_0 \in U\) and \(\vec y_0 = \vec f (\vec x_0)\).
    Fix any \(r \in \RR_+\) such that \(\cl{B_r(\vec x_0)} \subseteq U\).
    Let \(B = B_r(\vec x_0)\).
    Take any \(\vec y \in B_{\lambda r}(\vec y_0)\).
    Then, \[
        |\varphi(\vec x_0; \vec y) - \vec x_0|
        = |A\inv(\vec y - \vec y_0)| < \|A\inv\|_L \lambda r = \frac{r}{2}\text{.}
    \]
    Moreover, for any \(\vec x \in \cl B\), \[
        |\varphi(\vec x; \vec y) - \vec x_0|
        \le |\varphi(\vec x; \vec y) - \varphi(\vec x_0; \vec y)| + |\varphi(\vec x_0) - \vec x_0|
        \le \frac{1}{2} |\vec x - \vec x_0| + \frac{r}{2} < r\text{.}
    \]

    This directly implies that \(\varphi(\cl B) \subseteq B \subseteq \cl B\).
    Hence, \(\varphi\) is a contraction mapping on a complete metric space \(\cl B\).
    By \Cref{lem:contractionMappingP}, there exists a fixed point \(\vec x \in \cl B\),
    which satisfies \(\vec y = \vec f(\vec x)\).
    Thus, \(\vec y \in f(\cl B) \subseteq f(U) = V\). Hence, \(B_{\lambda r}(\vec y_0) \subseteq V\), \(V\) is open.

    Now, let \(\vec g \colon V \to U\) be the local inverse of \(\vec f\).
    Take any \(\vec y \in V\) and \(\vec y + \vec k \in V\).
    There are unique \(\vec x \in U\) and \(\vec x + \vec h \in U\) such that
    \(\vec y = \vec f(\vec x)\) and \(\vec y + \vec k = \vec f(\vec x + \vec h)\).

    By \Cref{lem}
    Let \(T \triangleq \d\vec f(\vec x)\inv\).
    Then,
    \[
        
    \]
}

\end{document}
