\documentclass[../MAS242_Note.tex]{subfiles}

\begin{document}
\section{Jacobian}

\dfn{Jacobian}{
    Let \(\vec f \colon U (\subseteq \RR[m]) \to \RR[n]\) be differentiable.
    The function \(J_f \colon U \to \RR\) defined by \[
        J_{\vec f}(\vec x) = \det \begin{bmatrix}
            \partial_1 f_1(\vec x) & \cdots & \partial_n f_1(\vec x) \\ 
            \vdots & \ddots & \vdots \\ 
            \partial_1 f_n(\vec x) & \cdots & \partial_n f_n(\vec x)
        \end{bmatrix}
    \]
    is called the \textit{Jacobian} of \(\vec f\) at \(\vec x\).
}

\mlemma{}{
    If \(f \colon V (\subseteq \RR[n]) - \RR\) and \(\vec g \colon U \to V\)
    are differentiable, then \[
        J_{f \circ \vec g}(\vec x) = J_f \big(\vec g(\vec x)\big) \cdot J_{\vec g}(\vec x)\text{.}
    \]
}

\nt{
    The linear mapping \(\d\vec f(c)\) is invertible if and only if \(J_{\vec f}(c)\) is nonzero.
}

\section{The Inverse Function Theorem}

\mlemma[contractionMappingP]{Contraction Mapping Principle}{
    Let \((X, d)\) be a complete metric space.
    Let \(\varphi \colon X \to X\). Suppose that there exists \(M \in [0, 1)\) such that
    \(d\big(\varphi(x_1), \varphi(x_2)\big) \le M d(x_1, x_2)\).
    (We call it a \textit{contraction mapping}.)

    Then, there uniquely exists \(x_{\ast} \in X\) such that
    \(\varphi(x_\ast) = x_\ast\).
}
\pf{Proof}{
    Fix any \(x_0 \in X\). Since \(\{x_j\}_{j \in \ZZ_+}\), where \(x_{j} = \varphi(x_{j-1})\)
    for each \(j \in \ZZ_+\), is continuous. It converges to some \(x_\ast\).
    As \(\varphi\) is continuous, we have \(\varphi(x_\ast) = x_\ast\).
    The uniqueness follows trivially.
}

\nt{
    \begin{itemize}[nolistsep]
        \ii For each \(v \in \RR[n] \setminus \{0\}\),
            \(|Av| = |v|\cdot \big|A \frac{v}{|v|}\big| \le \|A\|_{L} \cdot |v|\).
            The result is trivial when \(v = 0\).
        \ii For each \(u \in \RR[n]\) with \(|u| = 1\),
            \(|ABu| \le \|A\|_L |Bu| \le \|A\|_L \|B\|_L\).
            Hence, \(\|AB\|_L = \|A\| \|B\|\).
        \ii Given invertible \(A \in L(\RR[n]. \RR[n])\), \(A\inv \colon \RR[n] \to \RR[n]\)
            is linear. Moreover, \(\|A\|_L > 0\).
    \end{itemize}
}

\mlemma[invertMapsIsOpen]{}{
    Given two linear mappings \(A, B \colon \RR[n] \to \RR[n]\) with invertibility of \(A\),
    \[
        \|A-B\|_L \cdot \|A\inv\|_L < 1 \implies B \text{ is invertible.}
    \]
}
\pf{Proof}{
    Let \(\|A\inv\|_L = 1 / \alpha\) and \(\|B - A\|_L = \beta\) so that \(\beta < \alpha\).
    Then, for every \(\vec x \in \RR[n]\),
    \[\begin{aligned}[t]
        \alpha |\vec x| &= \alpha |A\inv A \vec x| \le \alpha \|A\inv\| \cdot |A\vec x| \\
                        &= |A\vec x| \le |(A-B)\vec x| + |B\vec x| \le \beta|\vec x| + |B\vec x|\text{;}
    \end{aligned}\]
    hence \((\alpha - \beta)|\vec x| \le |B\vec x|\) where \(\vec x \in \RR[n]\) is arbitrary.
    As \(\alpha > \beta\), it holds that \(B\vec x = 0 \implies \vec x = 0\).
}

\cor{}{
    The set \(\Omega \subseteq L(\RR[n], \RR[n])\) of invertible linear transformations
    is open.
}

\mlemma[inverseIsConti]{}{
    The mapping from \(\Omega\) onto \(\Omega\) defined by \(A \mapsto A\inv\) is continuous.
}
\pf{Proof}{
    Let \(A\) and \(B\) be invertible linear transformations from \(\RR[n]\) to \(\RR[n]\).
    Let \(\|A\inv\| = 1 / \alpha\) and \(\|B-A\|_L = \beta\).
    We have \((\alpha - \beta)|\vec x| \le |B\vec x|\) by the same reasoning as in the proof of
    \Cref{lem:invertMapsIsOpen}.
    Hence, the following holds.
    \[
        \fall \vec y \in \RR[n],\: (\alpha - \beta)|B\inv\vec y| \le |BB\inv \vec y| = |\vec y|
    \]
    This shows that \(\|B\inv\|_L \le (\alpha - \beta)\inv\).
    
    Hence, we have
    \[
        \|B\inv - A\inv\|_L \le \|B\inv\|_L \|A - B\|_L \|A\inv\|_L \le \frac{\beta}{\alpha(\alpha-\beta)}.
    \]
    This implies that \(\|B\inv - A\inv\|_L \to 0\) as \(B \to A\).
}

\thm[ift]{Inverse Function Theorem}{
    Let \(\vec f \colon E (\subseteq \RR[n]) \to \RR[n]\) be \(C^1\) in \(E\) and
    \(\vec c \in E\).
    Suppose that \(J_{\vec f}(\vec c) \neq 0\). Then, the following hold.
    \begin{enumerate}[nolistsep, label=(\roman*)]
        \ii There exists a neighborhood \(U\) of \(\vec a\) such that
            \(\vec f \big|_U\) is bijective and \(V \triangleq \vec f(U)\) is open.
        \ii The inverse map of \(\vec f \big|_U\) is \(C^1\) in \(V\).
    \end{enumerate}
}
\pf{Proof}{
    Let \(A \triangleq \d\vec f(\vec c)\).
    Define \(\lambda \in \RR_+\) by \(2 \lambda \big\|A\inv\big\|_L = 1\).
    Since \(\d\vec f\) is continuous, there exists a neighborhood \(U\) of \(\vec c\) such that
    \(\|\d\vec f(\vec x) - A\|_L < \lambda\) for each \(\vec x \in U\).

    Given a point \(\vec y \in \RR[n]\), we define \(\varphi(\cdot; \vec y)\) by
    \begin{align*}
        \varphi(\cdot; \vec y): B_\delta(\vec c) &\longrightarrow \RR[n] \\
        \vec x &\longmapsto \vec x + A\inv(\vec y - \vec f(\vec x))
    \end{align*}
    Note that \(\vec x\) is a fixed point of \(\varphi(\cdot; \vec y)\) if and only if
    \(A\inv (\vec y - \vec f(\vec x)) = 0\), i.e., \(\vec y = \vec f(\vec x)\).
    Note also that \(\varphi\) is differentiable and
    \(\d\varphi(\vec x; \vec y) = \mrm{Id} - A\inv \d\vec f(\vec x) = A\inv \big(A - \d\vec f(\vec x)\big)\)
    for each \(\vec x \in U\).

    Hence, for all \(\vec x \in U\),
    \[
        \|\d\varphi(\vec x; \vec y)\|_L
        = \big\|A\inv \big(A - \d\vec f(\vec x)\big)\big\|_L
        \le \|A\inv\|_L \cdot \|A - \d\vec f(\vec x)\|_L 
        < 1/(2\lambda) \cdot \lambda = 1/2\text{.}
    \]

    Thus, MVT gives
    \[
        |\varphi(\vec x_1; \vec y) - \varphi(\vec x_2; \vec y)| \le \frac{1}{2} |\vec x_1 - \vec x_2|
    \]
    whenever \(\vec x_1, \vec x_2 \in U\).
    Note that this implies there is at most one fixed point of \(\varphi(\cdot; \vec y)\) in \(U\),
    i.e., \(\vec f \big|_U\) is bijective.

    Now, we shall show that \(V = \vec f(U)\) is open.
    Take any \(\vec y_0 \in V\). There (uniquely) exists \(\vec x_0 \in U\)
    such that \(\vec y_0 = \vec f(\vec x_0)\).
    Fix any \(r \in \RR_+\) such that \(\cl{B} \subseteq U\)
    where \(B = B_r(\vec x_0)\).
    Take any \(\vec y \in B_{\lambda r}(\vec y_0)\).
    Then, \[
        |\varphi(\vec x_0; \vec y) - \vec x_0|
        = |A\inv(\vec y - \vec y_0)| < \|A\inv\|_L \lambda r = \frac{r}{2}\text{.}
    \]
    Moreover, for any \(\vec x \in \cl B\), \[
        |\varphi(\vec x; \vec y) - \vec x_0|
        \le |\varphi(\vec x; \vec y) - \varphi(\vec x_0; \vec y)| + |\varphi(\vec x_0; \vec y) - \vec x_0|
        \le \frac{1}{2} |\vec x - \vec x_0| + \frac{r}{2} < r\text{.}
    \]

    This directly implies that \(\varphi(\cl B; \vec y) \subseteq B \subseteq \cl B\).
    Hence, \(\varphi(\cdot, \vec y)\) is a contraction mapping on a complete metric space \(\cl B\).
    By \Cref{lem:contractionMappingP}, there exists a fixed point \(\vec x \in \cl B\),
    which satisfies \(\vec y = \vec f(\vec x)\).
    Thus, \(\vec y \in f(\cl B) \subseteq f(U) = V\).
    Hence, \(B \subseteq V\), \(V\) is open. This proves (i).

    Now, let \(\vec g \colon V \to U\) be the local inverse map of \(\vec f \big|_U\).
    Take any \(\vec y \in V\) and \(\vec y + \vec k \in V\).
    There are unique \(\vec x \in U\) and \(\vec x + \vec h \in U\) such that
    \(\vec y = \vec f(\vec x)\) and \(\vec y + \vec k = \vec f(\vec x + \vec h)\).
    Then, we have
    \[
        \varphi(\vec x + \vec h; \vec y) - \varphi(\vec x; \vec y)
        = \vec h + A\inv \big(\vec f(\vec x) - \vec f(\vec x + \vec h)\big)
        = \vec h - A\inv\vec k\text{,}
    \]
    which implies \(|\vec h-A\inv\vec k| \le |h| / 2\).
    Hence, \(|A\inv \vec k| \ge |h| / 2\) is obtained by the triangle inequality;
    \(|\vec h| \le 2 \|A\inv\|_L |\vec k| = \lambda\inv |\vec k|\).

    Then, since \(\|\d\vec f(\vec x) - A\|_L \|A\inv\|_L < \lambda \cdot 1/(2\lambda) = 1/2\),
    \Cref{lem:invertMapsIsOpen} implies that \(\d\vec f(\vec x)\) is invertible.
    Let \(T \triangleq \d\vec f(\vec x)\).
    Then, we have \[
        \vec g(\vec y + \vec k) - \vec g(\vec y) - T\inv\vec k
        = \vec h - T\inv\vec k
        = -T\inv \big( \vec f(\vec x + \vec h) - \vec f(\vec x) - T\vec h \big),
    \]
    and thus
    \[
        \frac{|\vec g(\vec y + \vec k) - \vec g(\vec y) - T\inv\vec k|}{|\vec k|}
        \le \frac{\|T\inv\|_L}{\lambda} \cdot \frac{|\vec f(\vec x + \vec h) - \vec f(\vec x) - T\vec h|}{|\vec h|}.
    \]
    The equation implies that \(\vec g\) is differentiable on \(V\),
    and that \(\d\vec g(\vec y) = T\inv = \mrm d\vec f(\vec g(\vec y))\inv\).
    Since \(\d\vec g\) is a composition of continuous functions,
    \(\d\vec g\) itself is continuous.
}

\cor{}{
    Let \(\vec f \colon E (\subseteq \RR[n]) \to \RR[n]\) be \(C^1\) in \(E\)
    and \(J_{\vec f}(\vec x) \neq 0\) for all \(\vec x \in E\).
    Then, for every open set \(W \subseteq E\),
    \(\vec f(W)\) is open.
}
\pf{Proof}{
    This directly follows from (i) of \Cref{th:ift}.
}

\section{Implicit Function Theorem}

\dfn{}{
    \begin{itemize}[nolistsep]
        \ii If \(\vec x = (x_1, \cdots, x_n) \in \RR[n]\) and \(\vec y = (y_1, \cdots, y_m) \in \RR[m]\),
            let us write \((\vec x, \vec y)\) for the point
            \((x_1, \cdots, x_n, y_1, \cdots, y_m) \in \RR[n + m]\).

        \ii Every \(A \in L(\RR[n+m], \RR[n])\) can be split into \(A_x \in L(\RR[n])\) and
            \(A_y \in L(\RR[m], \RR[n])\) where
            \(A(\vec h, \vec k) = A_x\vec h + A_y\vec k\)
            for each \(\vec h \in \RR[n]\) and \(\vec k \in \RR[m]\).
    \end{itemize}
}

\mlemma[linearImft]{}{
    If \(A\in L(\RR[n + m], \RR[n])\) and if \(A_x\) is invertible, then
    \[
        \fall \vec k \in \RR[m],\: \exs! \vec h \in \RR[n],\: A(\vec h, \vec k) = \vec 0.
    \]
}
\pf{Proof}{
    \(A(\vec h, \vec k) = A_x \vec h + A_y \vec k = \vec 0\)
    if and only if \(\vec h = -(A_x)\inv A_y \vec k\).
}

\thm[imft]{Implicit Function Theorem}{
    Let \(\vec f \colon E \to \RR[n]\) be a \(C^1\) mapping
    where \(E\) is an open set in \(\RR[n+m]\).
    Let \((\vec a, \vec b) \in E\) satisfy \(\vec f(\vec a, \vec b) = \vec 0\).
    Let \(A = \mrm d\vec f(\vec a, \vec b)\) and suppose \(A_x\) is invertible.
    Then, there exist open sets \(U \subseteq \RR[n + m]\) and \(W \subseteq \RR[m]\)
    that satisfy the following.
    \begin{enumerate}[nolistsep, label=(\roman*)]
        \ii \((\vec a, \vec b) \in U\) and \(\vec b \in W\).
        \ii \(\fall \vec y \in W,\: \exs!\vec x \in \RR[n],\:
            (\vec x, \vec y) \in U \land \vec f(\vec x, \vec y) = 0\).
        \ii If the unique \(\vec x\) in (ii) is denoted by \(\vec g(\vec y)\),
            then \(\vec g \colon W \to \RR[n]\) is \(C^1\) on \(W\).
        \ii Moreover, \(\mrm d\vec g(\vec b) = -(A_x)\inv A_y\).
    \end{enumerate}
}
\pf{Proof}{
    Define \(\vec F \colon E \to \RR[n+m]\) by
    \(\vec F(\vec x, \vec y) \triangleq (\vec f(\vec x, \vec y), \vec y)\).
    Then, \(\vec F\) is \(C^1\).
    Since \(\vec f(\vec a, \vec b) = \vec 0\),
    if \(\vec r(\vec h, \vec k) \triangleq \vec{f(a+h,b+k)}-A(\vec h, \vec k)\),
    we have \(\lim_{\vec h, \vec k \to \vec 0} |\vec r(\vec h, \vec k)| / |(\vec h, \vec k)| = 0\).
    Hence, from
    \[
        \mathbf{F(a+h, b+k) - F(a, b) = (f(a+h, b+k), k)}
        = (A\mathbf{(h, k), k) + (r(h, k), 0)}\text{,}
    \]
    it is obtained that \(\mathrm d\mathbf{F(a, b)(h', k')} = (A\mathbf{(h', k'), k'})\)
    for each \((\vec{h'}, \vec k') \in \RR[n + m]\).
    If \(\mrm d\mathbf{F(a, b)(h', k') = 0}\),
    then \(\vec{k}' = 0\) and \(A(\vec{h', 0}) = \vec{0}\);
    thus \(\vec{h' = 0}\) as \(A_x\) is invertible.
    Hence, \(\mrm{d}\vec{F(a, b)}\) is invertible;
    \Cref{th:ift} can be applied to \(\vec{F}\) at \(\vec{(a, b)}\).

    By \Cref{th:ift}, there exists a neighborhood \(U \subseteq E\)
    of \(\vec{(a, b)}\) such that \(\vec{F}\big|_U\) is bijective,
    \(\vec{F}(U)\) is open, and its inverse is \(C^1\).
    Let \(W \triangleq \{\,\vec{y} \in \RR[m] \mid \vec{(0, y)} \in \vec{F}(U)\,\}\).
    \(W\) is open as \(\vec{F}(U)\) is open.
    Noting that \(\vec{b} \in W\), we finish the proof for (i).

    Take any \(\vec{y} \in W\). Then, there exists \((\vec{x, y}) \in U\)
    such that \(\vec{F(x, y) = (0, y)}\); thus \(\vec{f(x, y) = 0}\).
    If \(\vec{x, x'}\) are two such point corresponding to \(\vec{y}\),
    then \[\vec{F(x', y) = (f(x', y), y) = (0, y) = (f(x, y), y) = F(x, y)}.\]
    However, as \(\vec{F}\) being injective, \(\vec{x = x'}\).
    This proves (ii).

    Let \(V \triangleq \vec{F}(U)\).
    Let \(\vec{G} \colon V \to U\) be the inverse of \(\vec{F}\),
    which is \(C^1\) by \Cref{th:ift}.
    Hence, for each \(\vec{y} \in W\),
    from \(\vec{F(g(y), y) = (0, y)}\), we have \(\vec{(g(y), y) = G(0, y)}\).
    This directly shows that \(\vec{g}\) is \(C^1\) as well. This proves (iii).

    Let \(\vec{\Phi} \colon W \to U\) be defined by
    \(\vec{\Phi(y) = G(0, y) = (g(y), y)}\), which is \(C^1\), indeed.
    Then, \(\mrm{d}\vec{\Psi(y)} = (\mrm{d}\vec{g(y)}, I_m)\).
    Differentiating both sides of the equality \(\vec{f(\Phi(y)) = 0}\),
    we get
    \[
        \mrm{d}\vec{f(\Phi(y))} \d\vec{\Phi(y)} = 0.
    \]
    Putting \(\vec{y} \coloneqq \vec{b}\), as \(\vec{\Phi(b) = (a, b)}\),
    we get \(A \d\vec{\Phi(b)} = 0\), or
    \[
        A_x \d \vec{g(b)} + A_y = 0,
    \]
    i.e., \(\mrm{d}\vec{g(b)} = -(A_x)\inv A_y\).
}

\dfn{\(C^1\)-norm}{
    Suppose \(\varphi \colon \RR[n] \to \RR\) is \(C^1\).
    Then, \[\begin{aligned}[t]
        \|\varphi\|_{C^0(\cl\Omega)} &\triangleq \sup_{\vec x \in \Omega} |\varphi(x)| \\
        \|\varphi\|_{C^1(\cl\Omega)} &\triangleq \|\varphi\|_{C^0(\cl \Omega)}
            + \sum_{j=1}^{n} \|\partial_j \varphi\|_{C^0(\cl\Omega)}.
    \end{aligned}\]
    This is only for \Cref{exmp:lvlSetExmp}.
}

\exmp[lvlSetExmp]{Level Sets}{
    Define \(\Omega \triangleq \{\,(x_1, x_2) \in \RR[2] \mid |x_2| \le 1\,\}\).
    Given two constants, \(a, b \in \RR\) with \(a < b\),
    define \(\ol\varphi(x_1, x_2) = ax_1\) and \(\ol \psi(x_1, x_2) = bx_1\).
    Then, \(\Gamma_0 = \{\,\vec x \in \Omega \mid \ol\varphi(\vec x) - \ol\psi(\vec x) = 0\,\}
    = \{\,\vec x \in \Omega \mid x_1 = 0\,\}\).

    Suppose that \(\varphi, \psi \colon \Omega \to \RR\) satisfy
    \[
        \|\varphi-\ol\varphi\|_{C^1(\ol\Omega)} + \|\psi-\ol\psi\|_{C^1(\ol\Omega)}
        \le \frac{1}{4} |a - b|\text{.}
    \]
    Then, what would be the expression for \(\Gamma = \{\,\vec x \in \Omega \mid \varphi(\vec x) - \psi(\vec x) = 0\,\}\)?

    Observe that
    \((\varphi - \psi) = (\varphi - \ol\varphi) + (\ol\varphi - \ol \psi) + (\ol \psi - \psi)\)
    and thus
    \(|(\varphi-\psi)(x_1, x_2) - (a-b)x_1| \le |a-b|/4\).
    This implies \(\lim_{x_1 \to \pm\infty} (\varphi - \psi)(x_1, x_2) = \mp\infty\).
    Hence, for every \(x_2 \in [-1, 1]\), there exists \(x_1^\ast \in \RR\)
    such that \((\varphi-\psi)(x_1^\ast, x_2) = 0\).

    Moreover, 
    \(\partial_1(\varphi - \psi) = \partial_1(\varphi - \ol\varphi) + (a-b) + \partial_1(\ol \psi - \psi)\),
    and thus \(|\partial_1(\varphi - \psi)| \ge \frac{3}{4} |a-b| > 0\).
    Hence, the \(x_1^\ast\) in the previous paragraph is unique.
    This means that \(\Gamma = \{\,(f(x_2), x_2) \mid x_2 \in \RR\,\}\) for some \(f\).

    \((\varphi - \psi)(f(x_2), x_2) - (\ol\varphi - \ol \psi)(f(x_2), x_2)
    = - (\ol\varphi - \ol \psi)(f(x_2), x_2) = (b-a)f(x_2)\).
    Hence,
    \[
        f(x_2) = \frac{(\varphi - \ol\varphi)(f(x_2), x_2) - (\psi - \ol \psi)(f(x_2), x_2)}{b-a}.
    \]
    This is the implicit representation of \(f\).
    Moreover, \(|f(x_2)| = \dfrac{|b-a|/4}{|b-a|} = 1/4\).
}

\section{Applications of IMFT: Lagrange's Method}

\thm[optim]{Optimization Under Multiple Constraints}{
    Let \(f, g_1, g_2, \cdots, g_k \colon E \to \RR\) be \(C^1\)
    where \(E\) is an open set in \(\RR[n]\) and \(n > k\).
    Let \(Z \triangleq \cap_{j=1}^k \{\,\vec{z} \in \RR[n] \mid g_j(\vec{z}) = 0\,\}\).
    Suppose \(\vec{z}_0 \in Z\) is a local maximum point with respect to \(f\)
    on \(Z\). Suppose also that
    \[
        \Delta \triangleq \det \begin{bmatrix}
            \partial_1 g_1(\vec{z}_0) & \cdots & \partial_1 g_k(\vec{z}_0) \\
            \vdots & \ddots & \vdots \\
            \partial_k g_1(\vec{z}_0) & \cdots & \partial_k g_k(\vec{z}_0)
        \end{bmatrix} \neq 0.
    \]
    Then, there exists \(\lambda_1, \lambda_2, \cdots, \lambda_k \in \RR\)
    such that \(\nabla f(\vec{z}_0) = \sum_{m=1}^{k} \lambda_m \nabla g_m(\vec{z}_0)\).
}
\pf{Proof}{
    Since \(\Delta \neq 0\), there exists a unique solution
    \((\lambda_1, \cdots, \lambda_k) \in \RR[k]\) for the linear system
    \[
        \begin{bmatrix}
            \partial_1 g_1(\vec{z}_0) & \cdots & \partial_1 g_k(\vec{z}_0) \\
            \vdots & \ddots & \vdots \\
            \partial_k g_1(\vec{z}_0) & \cdots & \partial_k g_k(\vec{z}_0)
        \end{bmatrix} \begin{bmatrix}
            \lambda_1 \\ \vdots \\ \lambda_k
        \end{bmatrix} = \begin{bmatrix}
            \partial_1 f(\vec{z}_0) \\ \vdots \\ \partial_k f(\vec{z}_0)
        \end{bmatrix}.
    \]

    For each point \(\vec{z} = (z_1, \cdots, z_n) \in \RR[n]\),
    let \(\vec{x} = (z_1, \cdots, z_k)\) and \(\vec{y} = (z_{k+1}, \cdots, z_n)\).
    Let \(\vec{z}_0 = (\vec{x}_0, \vec{y}_0)\).
    Let \(\vec{g} \colon E \to \RR[k]\) be defined by
    \(\vec{g}(\vec{z}) = (g_1(\vec{z}), \cdots, g_k(\vec{z}))\).

    Since \(\vec{g}\) is \(C^1\), \(\vec{g}(\vec{z}_0) = 0\),
    and \(\big(\mrm{d}\vec{g}(\vec{z}_0)\big)_x\) is invertible,
    by \Cref{th:imft}, there exists an open neighborhood \(W \subseteq \RR[n-k]\)
    of \(\vec{y}_0\) and a \(C^1\) function \(\vec{s} \colon W \to \RR[k]\)
    such that \(\vec{g(s(y), y) = 0}\) for each \(\vec{y} \in W\).
    Note that \(\vec{s}(\vec{y}_0) = \vec{x}_0\).

    Define \(F \colon W \to \RR\) by \(\vec{y} \mapsto f\vec{(s(y), y)}\).
    As \(\vec{z}_0\) is a local maximum point, so is \(\vec{y}_0\).
    Hence, \(\nabla F(\vec{y}_0) = \vec{0}\).
    For each \(j \in [k]\), define \(G_j \colon W \to \RR\) by
    \(\vec{y} \mapsto g_j(\vec{s(y), y})\).
    As \((\vec{s(y), y}) \in Z\), we have \(G_j = 0\) for each \(j \in [k]\).
    Thus, \(\nabla G_j(\vec{y}) = \vec{0}\).
    
    Let \(\vec{s} = (s_1, s_2, \cdots, s_k)\) where each \(s_j \colon W \to \RR\).
    Since
    \[\begin{aligned}[t]
        \nabla F(\vec{y}) &= \mrm{d}f(\vec{s}(\vec{y}), \vec{y})\,\mrm{d}(\vec{s}(\vec{y}), \vec{y}) \\
        &= \begin{bmatrix}
            \partial_1 f(\vec{s}(\vec{y}), \vec{y}) & \cdots & \partial_n f(\vec{s}(\vec{y}), \vec{y})
        \end{bmatrix} \begin{bmatrix}
            \partial_1 s_1(\vec{y}) & \partial_2 s_1(\vec{y}) & \cdots & \partial_{n-k} s_1(\vec{y}) \\
            \partial_1 s_2(\vec{y}) & \partial_2 s_2(\vec{y}) & \cdots & \partial_{n-k} s_2(\vec{y}) \\
            \vdots & \vdots & \ddots & \vdots \\
            \partial_1 s_{k}(\vec{y}) & \partial_2 s_{k}(\vec{y}) & \cdots & \partial_{n-k} s_{k}(\vec{y}) \\
            1 & 0 & \cdots & 0 \\
            0 & 1 & \cdots & 0 \\
            \vdots & \vdots & \ddots & \vdots \\
            0 & 0 & \cdots & 1
        \end{bmatrix},
    \end{aligned}\]
    \(\nabla F(\vec{y}_0) = \vec{0}\) implies
    \[
        \partial_{k+j} f(\vec{z}_0)
        + \sum_{i=1}^{k} \partial_i f(\vec{z}_0) \partial_j s_i(\vec{y}_0) = 0
    \]
    for each \(j \in [n-k]\).
    Similarly, \(\nabla G_m(\vec{y}_0) = \vec{0}\) for each \(m \in [k]\) implies that
    \[
        -\lambda_m \left[ \partial_{k+j} g_m(\vec{z}_0)
        + \sum_{i=1}^{k} \partial_i g_m(\vec{z}_0) \partial_j s_i(\vec{y}_0)\right] = 0
    \]
    for each \(j \in [n - k]\) and \(m \in [k]\).

    Adding the \(k + 1\) equations together for each \(j \in [n - k]\),
    \[\begin{aligned}[t]
        0 &= \left[ \partial_{k+j} f(\vec{z}_0) - \sum_{m=1}^{k} \lambda_m \partial_{k+j} g_m(\vec{z}_0) \right]
        + \sum_{i=1}^{k} \left[
            \partial_i f(\vec{z}_0)
            - \sum_{m=1}^{k} \lambda_m \partial_i g_m(\vec{z}_0)
        \right] \partial_j s_i(\vec{y}_0).
    \end{aligned}\]
    By the definition of \(\lambda_1, \cdots, \lambda_n\),
    we are left with only
    \[
        \partial_{j} f(\vec{z}_0) = \sum_{m=1}^{k} \lambda_m \partial_{j} g_m(\vec{z}_0)
    \]
    for each \(j \in \{k+1, \cdots, n\}\).
    For \(j \in [k]\), the same equation holds by the definition of \(\lambda_1, \cdots, \lambda_k\).
    Hence, we have \(\nabla f(\vec{z}_0) = \sum_{m=1}^{k} \lambda_m \nabla g_m(\vec{z}_0)\).
}

\end{document}
