\documentclass[MAS241_Note.tex]{subfiles}


\begin{document}
\chapter{Differentiation}

\section{The Derivative}

\dfn{Derivative}{
    Let $f$ be defined on an interval $I$ on $\RR$. 
    Let $c$ be a point in $I$. The \textit{derivative} of $f$ at $c$ is defined
    to be
    \[
        \lim_{h \to 0} \frac{f(c + h) - f(c)}{h}
    \]
    provided this limit exists. The derivative of $f$ at $c$ is denoted by
    $f'(c)$. We say that $f$ is \textit{differentiable at} $c$ if $f'(c)$
    exists. We say that $f$ is \textit{differentiable on} $I$ if
    $f'(x)$ exists for each $x \in I$.
}

\dfn{Differential}{
    Let $f$ be a real-valued function defined on an interval $I$ in $\RR$.
    Suppose that $f$ is differentiable at a $c \in I$. The function
    $\mrm df(c; \cdot) \colon \RR \to \RR$ defined by
    \[
        \mrm df(c; t) = f'(c)t
    \]
    is called the \textit{differential} of $f$ at $c$.
}

\thm[]{}{
    Suppose $f$ is differentiable at $c$ in its domain.
    Then, for any $\veps \in \RR_+$, there exists a deleted neighborhood
    $N'(0)$ such that,
    \[
        |f(c+t)-f(c)-\mrm df(c; t)| < \veps|t|\text{,}
    \]
    for all $t \in N'(0)$.
}
\pf{Proof}{
    There is a deleted neighborhood $N'(0)$ such that
    \[
        \left|\frac{f(c+t)-f(c)}{t} - f'(c)\right| < \veps
    \]
    for all $t \in N'(0)$.
}

\thm[]{}{
    If $f$ is differentiable at $c$, then $f$ is continuous at $c$.
}
\pf{Proof}{
    \[
        \lim_{x \to c} \big(f(x) - f(c)\big) =
        \lim_{x \to c} \frac{f(x)-f(c)}{x-c} \cdot (x - c) = f'(c) \cdot 0 = 0
    \]
    Hence, $\lim_{x \to c} f(x) = f(c)$; $f$ is continuous at $c$.
}

\cor{}{
    If $f$ is differentiable on $I$, then $f \in C(I)$.
}

\thm[]{}{
    Suppose that $f$ and $g$ are two functions each differentiable at $c$
    and that $a \in \RR$.
    \begin{enumerate}[nolistsep, label=(\roman*)]
        \ii $f + g$ is differentiable at $c$ and
            \[
                (f+g)'(c) = f'(c) + g'(c)\text{.}
            \]
        \ii $af$ is differentiable at $c$ and
            \[
                (af)'(c) = af'(c)\text{.}
            \]
        \ii $fg$ is differentiable at $c$ and
            \[
                (fg)'(c) = f'(c)g(c) + f(c)g'(c)\text{.}
            \]
        \ii If $g(c) \neq 0$, then $1/g$ is differentiable at $c$ and
            \[
                \left(\frac{1}{g}\right)'(c) = -\frac{g'(c)}{g(c)^2}\text{.}
            \]
        \ii If $g(c) \neq 0$, then $f/g$ is differentiable at $c$ and
            \[
                \left(\frac{f}{g}\right)'(c)
                = \frac{f'(c)g(c) - f(c)g'(c)}{g(c)^2}\text{.}
            \]
    \end{enumerate}
}
\pf{Proof}{
    \hfill
    \begin{enumerate}[nolistsep, label=(\roman*)]
    \ii $\displaystyle \lim_{x \to c} \frac{(f+g)(x) - (f+g)(c)}{x-c}
        = \lim_{x \to c} \left[\frac{f(x)-f(c)}{x-c} + \frac{g(x)-g(c)}{x-c}\right]
        = f'(c) + g'(c)$.
    \ii $\displaystyle \lim_{x \to c} \frac{(af)(x) - (af)(c)}{x-c}
        = \lim_{x \to c} a \cdot \frac{f(x)-f(c)}{x-c} = af'(c)$.
    \ii First, note that
        \[\begin{aligned}[t]
            (fg)(x) - (fg)(c) &= f(x)g(x)-f(c)g(x)+f(c)g(x)-f(c)g(c) \\
                              &= \big(f(x)-f(c)\big)g(x) +
                                 \big(g(x)-g(c)\big)f(c)\text{.}
        \end{aligned}\]
        Since $g$ is continuous at $c$,
        \[\begin{aligned}[t]
            \lim_{x \to c} \frac{(fg)(x)-(fg)(c)}{x-c}
            &= \lim_{x \to c} \left[\frac{f(x)-f(c)}{x-c} g(x) +
                                   \frac{g(x)-g(c)}{x-c} f(c) \right] \\
            &= f'(c)g(c) + f(c)g'(c)\text{.}
        \end{aligned}\]
    \ii \[
        \lim_{x \to c} \frac{1/g(x)-1/g(c)}{x-c}
        = \lim_{x \to c} \left[\frac{g(x)-g(c)}{(x-c)} \cdot \frac{1}{g(x)g(c)}\right]
        = -\frac{g'(c)}{g(c)^2}\text{.}
        \]
    \ii \[
        \left(\frac{f}{g}\right)'(c)
        = f'(c)\left(\frac{1}{g}\right)(c) + f(c) \left(\frac{1}{g}\right)'(c)
        = \frac{f'(c)g(c)-f(c)g'(c)}{g(c)^2}\text{.}
        \]
    \end{enumerate}
}

\dfn{$k$th Derivative}{
    Suppose $f' = f^{(1)}$, $f'' = f^{(2)}$, $\cdots$, $f^{(k)}$ is defined
    on a neighborhood of $c$ and if
    \[
        f^{(k+1)}(c) \triangleq \lim_{h \to 0} \frac{f^{(k)}(c+h)-f^{(k)}(c)}{h}
    \]
    exists, then $f^{(k)}(c)$ is called the $k$th derivative of $f$ at $c$ and
    $f$ is said to be $k$ \textit{times differentiable} at $c$.

    If $f^{(k)}(c)$ exists for all $k \in \NN$, then $f$ is said to have
    \textit{derivatives of all orders} at $c$.
    If $f^{(k)}$ exists for all $k \in \NN$ and at all $x \in I$,
    then $f$ is said to have \textit{derivatives of all orders} on $I$.
}

\section{Composition of Functions: The Chain Rule}

\thm[chain]{The Chain Rule}{
    Suppose the following.
    \begin{itemize}[nolistsep]
        \ii $c$ is an interior point of an interval $I$.
        \ii $g$ is differentiable at $c$.
        \ii $d = g(c)$ is an interior point of $g(I)$.
        \ii $f$ is defined on $g(I)$.
        \ii $f$ is differentiable at $d$.
    \end{itemize}
    Then, $f \circ g$ is differentiable at $c$ and $(f \circ g)'(c) =
    f'(g(c))g'(c)$.
}
\pf{Proof}{
    For $y$ in $g(I)$, define
    \[
        h(y) = \begin{cases} \displaystyle
            \frac{f(y)-f(d)}{y-d} - f'(d) & \text{if } y \neq d \\
            0 & \text{if } y = d\text{.}
        \end{cases}
    \]
    Since $\lim_{y \to d} h(y) = 0 = h(d)$, $h$ is continuous at $d$.
    Also, since $h$ is defined on $g(I)$, we may composite it with $g$;
    \[
        (h \circ g)(x) = \begin{cases} \displaystyle
            \frac{f(g(x))-f(g(c))}{g(x)-g(c)}-f'(g(c)) & \text{if } g(x) \neq g(c) \\
            0 & \text{if } g(x) = g(c) = d\text{.}
        \end{cases}
    \]
    Multiplying by $g(x) - g(c)$, we get
    \[
        f(g(x)) - f(g(c))
        = \big\{(h \circ g)(x) + f'(g(c))\big\}
            \big\{g(x)-g(c)\big\}\text{.}
    \]
    Hence,
    \[\begin{aligned}[t]
        \lim_{x \to c} \frac{f(g(x))-f(g(c))}{x-c}
        &= \lim_{x \to c} \big\{(h \circ g)(x) + f'(g(c))\big\} \frac{g(x)-g(c)}{x-c}\\
        &= \big\{0+f'(g(c))\big\}g'(c) = f'(g(c))g'(c)\text{.}
    \end{aligned}\]
}

\section{The Mean Value Theorem}

\dfn{Local Minimum and Maximum}{
    Let $f$ be a real-valued function defined on the interval $I$ and $c$ is
    an interior point of $I$.
    \begin{itemize}[nolistsep]
        \ii $f$ has a \textit{local maximum} at $c$ if there exists a neighborhood
            $N(c)$ such that $\fall x \in N(c),\: f(x) \le f(c)$.
        \ii $f$ has a \textit{local minimum} at $c$ if there exists a neighborhood
            $N(c)$ such that $\fall x \in N(c),\: f(x) \ge f(c)$.
    \end{itemize}
}

\thm[locMinMaxZeroDeriv]{}{
    If $f$ has either a local maximum or a local minimum at an interior point
    $c$ of $I$ and if $f$ is differentiable at $c$, then $f'(c) = 0$.
}
\pf{Proof}{
    Let $N(c)$ be a neighborhood of $c$ such that $\fall x \in N(c),\: f(x) \ge f(c)$.
    Then, since
    \[
        \frac{f(x) - f(c)}{x-c} \ge 0 \quad\text{on } N(c) \cap (c, \infty)
        \text{ and}
    \]
    \[
        \frac{f(x) - f(c)}{x-c} \le 0 \quad\text{on } N(c) \cap (-\infty, c)\text{,}
    \]
    It follows that
    \[
        f'(c) = \lim_{x \to c^+} \frac{f(x)-f(c)}{x-c} \ge 0
        \quad{\text{and}}\quad
        f'(c) = \lim_{x \to c^-} \frac{f(x)-f(c)}{x-c} \le 0\text{.}
    \]
    Hence, $f'(c) = 0$.
}

\dfn{Critical Point}{
    If  a function $f$ is differentiable at a point $c$ and if $f'(c) = 0$,
    then $c$ is called a \textit{critical point} of $f$.
}

\thm[rolle]{Rolle's Theorem}{
    Suppose that $f$ is continuous on $[a, b]$ and is differentiable on $(a, b)$.
    Suppose further that $f(a) = f(b)$. Then there exists $c \in (a, b)$ such
    that $f'(c) = 0$.
}
\pf{Proof}{
    If $f(x) = f(a)$ for all $x \in [a, b]$, then $f' = 0$ on $[a, b]$, and
    it is done.

    Otherwise, $f$ has a maximum value or a minimum value at some point $c$
    in $(a, b)$ by \Cref{th:evt}. $c$ is either a local maximum or a local minimum of $f$
    as well. Hence, by \Cref{th:locMinMaxZeroDeriv}, $f'(c) = 0$.
}

\thm[mvt]{The Mean Value Theorem}{
    Suppose $f$ is continuous on $[a, b]$ and and is differentiable on $(a, b)$.
    Then,
    \[
        \exs c \in (a, b),\: \frac{f(b)-f(a)}{b-a} = f'(c)\text{.}
    \]
}
\pf{Proof}{
    Let
    \[
        h(x) = f(x) - \left[\frac{f(b)-f(a)}{b-a} (x-a) + f(a)\right]\text{.}
    \]
    Then, $h$ is continuous on $[a, b]$, differentiable on $(a, b)$, and
    $h(a) = h(b) = 0$.
    Hence, by \Cref{th:rolle}, there exists $c \in (a, b)$ such that
    \[
        h'(c) = f'(c) - \frac{f(b)-f(a)}{b-a} = 0\text{.}
    \]
}

\cor[zeroDerivThenConst]{}{
    Suppose $f$ is continuous on $[a, b]$ and is differentiable on $(a, b)$.
    If $f'(x) = 0$ for each $x \in (a, b)$, then $f$ is constant on $[a, b]$.
}
\pf{Proof}{
    Take any $x \in (a, b]$. By \Cref{th:mvt}, there exists $c \in (a, x)$
    such that
    \[
        f'(c) = \frac{f(x) - f(a)}{x-a}\text{.}
    \]
    Since $f'(c) = 0$, it follows that $f(x) = f(a)$. Since $x$ was arbitrary,
    $f$ is constant on $[a, b]$.
}

\cor[sameDerivThenDiffByConst]{}{
    Suppose $f$ and $g$ are continuous on $[a, b]$ and are differentiable
    on $(a, b)$. If $f'(x) = g'(x)$ for each $x \in (a, b)$, then
    $f$ and $g$ differ by a constant.
}
\pf{Proof}{
    Let $h(x) = f(x) - g(x)$ and apply \Cref{cor:zeroDerivThenConst}.
}

\cor[posDerivIncr]{}{
    Suppose $f$ is continuous on $[a, b]$ and is differentiable on $(a, b)$.
    \begin{itemize}[nolistsep]
        \ii If $f'(x) > 0$ for all $x \in (a, b)$, then $f$ is strictly increasing.
        \ii If $f'(x) < 0$ for all $x \in (a, b)$, then $f$ is strictly decreasing.
    \end{itemize}
}
\pf{Proof}{
    Suppose $f'$ is positive on $(a, b)$.
    Take any $x_1, x_2 \in [a, b]$ with $x_1 < x_2$. By \Cref{th:mvt},
    there exists $c \in (x_1, x_2)$ such that
    \[
        f'(c) = \frac{f(x_2)-f(x_1)}{x_2-x_1} > 0\text{.}
    \]
    Hence, $f(x_2) > f(x_1)$; $f$ is strictly increasing.
    It is analogous to show for negative $f'$.
}

\thm[mvtCauchy]{Cauchy's Generalized Mean Value Theorem}{
    Suppose $f$ and $g$ are continuous on $[a, b]$ and are differentiable
    on $(a, b)$. Then, there exists $c \in (a, b)$ such that
    \[
        f'(c) \big\{g(b)-g(a)\big\} = g'(c) \big\{f(b) - f(a)\big\}\text{.}
    \]
}
\pf{Proof}{
    Let $h(x) = f(x)\left\{g(b)-g(a)\right\} - g(x)\left\{f(b)-f(a)\right\}$.
    Then $h$ is continuous on $[a, b]$ and is differentiable on $(a, b)$.
    A calculation will verify that $h(a) = h(b)$.
    Hence, by \Cref{th:rolle}, there exists $c \in (a, b)$ such that
    \[
        h'(c) = f'(c)\left\{g(b)-g(a)\right\} - g'(c)\left\{f(b)-f(a)\right\}
        = 0\text{.}
    \]
}

\section{L'Hôpital's Rule}

\thm[lopital1]{L'Hôpital's Rule, Form I}{
    Suppose that $f$ and $g$ are continuous on $[a, b]$ and are differentiable
    on $(a, b)$. Suppose $c \in [a, b]$ and $f(c) = g(c) = 0$. Suppose also
    that, for some deleted neighborhood $N'(c)$,
    $g'(x) \neq 0$ for each $x \in N'(c)$. Then,
    \[
        \lim_{x \to c} \frac{f'(x)}{g'(x)} = L
        \implies \lim_{x \to c} \frac{f(x)}{g(x)} = L\text{.}
    \]
}
\pf{Proof}{
    Take any $\veps \in \RR_+$. Then, there exists some $\delta \in \RR_+$
    such that
    \[
        \forall x \in [a, b],\: \left(0 < |x-c| < \delta
        \implies \left|\frac{f'(x)}{g'(x)} - L\right| < \veps\right).
    \]
    Take any $x \in [a, b]$ such that $0 < |x-c| < \delta$.
    Then, by \Cref{th:mvtCauchy}, there exists $d$ between $x$ and $c$
    such that
    \[
        f'(d)g(x) = f'(d)\big\{g(x) - g(c)\big\}
        = g'(d)\big\{f(x) - f(c)\big\} = g'(d)f(x)\text{.}
    \]
    Hence, $f(x)/g(x) = f'(d)/g'(d)$. (Note that $g(x) \neq 0$ on
    $N'(c)$ since $g'(x) \neq 0$.) Therefore,
    \[
        \left|\frac{f(x)}{g(x)} - L\right|
        = \left|\frac{f'(d)}{g'(d)} - L\right| < \veps\text{.}
    \]
}

\thm[lopital2]{L'Hôpital's Rule, Form II}{
    Suppose $f$ and $g$ are differentiable on some neighborhood $N(\infty, M)$.
    Suppose $g'(x) \neq 0$ for each $x \in N(\infty, M)$ and
    $\lim_{x \to \infty} f(x) = \lim_{x \to \infty} g(x) = \infty$.
    Then,
    \[
        \lim_{x \to \infty} \frac{f'(x)}{g'(x)} = L
        \implies \lim_{x \to \infty} \frac{f(x)}{g(x)} = L\text{.}
    \]
}
\pf{Proof}{
    Take any $\veps \in (0, 1)$. Let $\veps' \triangleq \veps/(2+|L|)$.
    There exists $M_1 \ge M$ such that,
    \[
        \fall x \in N(\infty, M_1),\: \left|\frac{f'(x)}{g'(x)}-L\right| < \veps'\text{.}
    \]
    There exists $M_2 \ge M_1$ such that,
    \[
        \fall x \in N(\infty, M_2),\: f(x) > \max \{f(M_1), 1\} \text{ and } g(x) > g(M_1)\text{.}
    \]
    For each $x \in N(\infty, M_2)$, By \Cref{th:mvtCauchy}, there exists
    $c_x \in (M_1, x)$ such that
    \[
        \frac{f'(c_x)}{g'(c_x)} = \frac{f(x) - f(M_1)}{g(x) - g(M_1)}
        = \frac{f(x) \left(1 - \dfrac{f(M_1)}{f(x)}\right)}{g(x) \left(1 - \dfrac{g(M_1)}{g(x)}\right)}
    \]
    Let $\displaystyle h(x) = \frac{1-g(M_1)/g(x)}{1-f(M_1)/f(x)}$. Then,
    \[
        \frac{f(x)}{g(x)} = \frac{f'(c_x)}{g'(c_x)}h(x)\text{.}
    \]
    Since $\lim_{x \to \infty} h(x) = 1$, there exists $M_3 \ge M_2$ such that
    $\fall  x \in N(\infty, M_3),\: |h(x)-1| < \veps'$.
    Then, for any $x \in N(\infty, M_3)$,
    \[\begin{aligned}[t]
        \left|\frac{f(x)}{g(x)}-L\right|
        &= \left|\frac{f'(c_x)}{g'(c_x)}h(x) - L\right| \\
        &= \left|\frac{f'(c_x)}{g'(c_x)}h(x) - Lh(x) + Lh(x) - L\right| \\
        &\le \left|\frac{f'(c_x)}{g'(c_x)}-L\right| |h(x)| + |L| |h(x) - 1| \\
        &< \veps' (1+\veps') + |L| \veps' = (1 + |L|) \veps' + \veps'^2
        < (2 + |L|) \veps' = \veps\text{.}
    \end{aligned}\]
}

\section{Taylor's Theorem}

\dfn{$k^\text{th}$-degree Taylor Polynomial}{
    % Suppose that $f$ and its first $k$ derivatives exist and are continuous on
    % $[a, b]$ and that $f^{(k+1)}$ exists on $(a, b)$. Let $x_0 \in [a, b]$.
    Suppose $f$ is $k$ times differentiable at $x = x_0$.
    Then, a polynomial $p_k$ defined by
    \[
        p_k(x) \triangleq \sum_{j=0}^{k} \frac{f^{(j)}(x_0)}{j!}(x-x_0)^j
    \]
    is called the $k^\text{th}$-\textit{degree Taylor polynomial} of $f$
    at $x = x_0$.
}

\nt{
    \noindent
    $p_k$ is the only $k^\text{th}$ degre polynomial such that
    $p_k^{(j)}(x_0) = f^{(j)}(x_0)$ for each $j = 0, 1, \cdots, k$.
}

\thm[mvtCauchyGen]{}{
    Let $f$ and $g$ be continuous and $k$ times continuously differentiable on $[a, b]$.
    Suppose $f$ and $g$ also have $(k+1)$ times differentiable on $(a, b)$.
    Let $x_0 \in [a, b]$. Then, for every $x \in [a, b] \setminus \{x_0\}$,
    there exists $c$ between $x$ and $x_0$ such that
    \[
        f^{(k+1)}(c) \big\{g(x) - q_k(x)\big\} =
        g^{(k+1)}(c) \big\{f(x) - p_k(x)\big\}
    \]
    where $p_k$ and $q_k$ are $k^\text{th}$-degree Taylor polynomial at $x_0$
    of $f$ and $g$, respectively.
}
\pf{Proof}{
    \textsf{WLOG}, suppose $x_0 < x$. For each $t \in [x_0, x]$, define
    \[
        F(t) \triangleq f(t) + \sum_{j=1}^{k} \frac{f^{(j)}(t)}{j!} (x-t)^j
    \] and \[
        G(t) \triangleq g(t) + \sum_{j=1}^{k} \frac{g^{(j)}(t)}{j!} (x-t)^j
    \]
    Since $f^{(j)}$ and $g^{(j)}$ are continuous on $[x_0, x]$ for each
    $j = 0, 1, \cdots, k$, $F$ and $G$ are continuous on $[x_0, x]$.
    Since $f^{(j)}$ and $g^{(j)}$ are differentiable on $(x_0, x)$ for each
    $j = 0, 1, \cdots, k$, $F$ and $G$ are differentiable on $(x_0, x)$.

    Hence, by \Cref{th:mvtCauchy}, there exists $c \in (x_0, x)$ such that
    \[
        F'(c) \big\{G(x) - G(x_0)\big\} = G'(c)\left\{F(x) - F(x_0)\right\}\text{.}
    \]
    Note that $F(x) = f(x)$, $G(x) = g(x)$, $F(x_0) = p_k(x)$, and $G(x_0) = q_k(x)$.
    Moreover, \[
        F'(t) = \frac{f^{(k+1)}(t)}{k!}(x-t)^k \quad\text{and}\quad
        G'(t) = \frac{g^{(k+1)}(t)}{k!}(x-t)^k\text{.}
    \]
    Therefore, the result follows.
}

\thm[taylorThm]{Taylor's Theorem}{
    Suppose that $f$ is $k$ times continuously differentiable on $[a, b]$ 
    and is $(k+1)$ times differentiable on $(a, b)$. Let $x_0 \in [a, b]$.
    Then, for each $x \in [a, b] \setminus \{x_0\}$, there exists $c$
    between $x$ and $x_0$ such that
    \[
        f(x) = p_k(x) + \frac{f^{(k+1)}(c)}{(k+1)!} (x-x_0)^{k+1}\text{.}
    \]
    where $p_k$ is the $k^\text{th}$-degree Taylor polynomial of $f$ at $x_0$.
}
\pf{Proof}{
    Let $g(x) = (x-x_0)^{k+1}$. Apply \Cref{th:mvtCauchyGen} to $f$ and $g$.
    There exists $c$ between $x_0$ and $x$ such that
    \[
        f^{(k+1)}(c) \big\{g(x) - q_k(x)\big\} =
        g^{(k+1)}(c) \big\{f(x) - p_k(x)\big\}
    \]
    where $q_k$ is the $k^\text{th}$-degree Taylor polynomial of $g$ at $x_0$.
    Since $g^{(j)}(x_0) = 0$ for each $j = 0, 1, \cdots, k$, $q_k = 0$.
    Also, $g^{(k+1)}(x) = (k+1)!$. Hence, we get
    \[
        f^{(k+1)}(c) (x-x_0)^{k+1} = (k+1)!\big\{f(x) - p_k(x)\big\}\text{.}
    \]
    The result follows.
}

\thm{}{
    Let $f$ has derivatives of all orders on $[a, b]$. Suppose that there
    exists $M \in \RR_+$ such that $\|f^{(k)}\|_\infty^{1/k} \le M$ for all
    $k \in \NN$. Then, for any $x_0 \in [a, b]$,
    \[
        \lim_{k \to \infty} p_k = f \quad\text{[uniformly]}
    \]
    where $p_k$ is the $k^\text{th}$-degree Taylor polynomial of $f$ at $x_0$.
}
\pf{Proof}{
    Since $f^{(k)}$ is continuous on $[a, b]$ for each $k \in \NN$,
    $f^{(k)}$ is bounded on $[a, b]$ and thus $\|f^{(k)}\|_\infty$ is finite.
    Fix any $x_0 \in [a, b]$ and take $x \in [a, b]$. Then, for each $k \in \NN$,
    there exists some $c \in [a, b]$ such that
    \[
        f(x) - p_k(x) = \frac{f^{(k+1)}(c)}{(k+1)!} (x-x_0)^{k+1}\text{.}
    \]
    Hence,
    \[
        0 \le |f(x) - p_k(x)|
        \le \frac{\|f^{(k+1)}\|_\infty}{(k+1)!}|b-a|^{k+1}
        \le \frac{(M|b-a|)^{k+1}}{(k+1)!}\text{.}
    \]
    Since $\displaystyle \lim_{k \to \infty} \frac{(M|b-a|)^{k+1}}{(k+1)!} = 0$
    and the terms do not depend on $x$, $p_k$ uniformly converges to $f$ on
    $[a, b]$.
}

\end{document}
