\documentclass[MAS241_Note.tex]{subfiles}

\begin{document}
\chapter{Continuity}
\section{Limit and Continuity}

\dfn{Limit of a Function}{
    Let $f \colon S \to \RR$ where $S \subseteq \RR[n]$.
    Let $\vec c \in \cl S$.
    We say $f$ has \textit{limit} $L$ as $\vec x$ approaches $\vec c$
    provided that, for every neighborhood $N(L)$,
    there exists a deleted neighborhood $N'(\vec c)$ such that \[
        S \cap N'(\vec c) \subseteq f\inv(N(L))\text{.}
    \]
    We write $\lim_{\vec x \to \vec c} f(\vec x) = L$.
}

\nt{
    Limit is unique if it exists.
}

\nt{
    Note that $S \cap N'(\vec c; \delta) = \varnothing$ for sufficiently small $\delta$
    if $\vec c$ is an isolated point of $S$.
    This implies any real number can be a limit of $f$ as $\vec x$ approaches $\vec c$.
    Somehow, Douglass defined that $\lim_{\vec x \to \vec c} f(\vec x) = f(\vec c)$
    (since $\vec c \in S$ in this case).
    {\color{lightgray} Actually I do not think we should define limit for isolated points.}
}
\nt{
    This definition of limit is equivalent to the normal
    $\veps$-$\delta$ definition of limit,
    except that it defines a limit for isolated points.
}

\dfn{Continuity}{
    Let $f \colon S \to \RR$ where $S \subseteq \RR[n]$.
    Let $\vec c \in S$.
    We say $f$ is \textit{continuous at} $\vec c$ if \[
        \lim_{\vec x \to \vec c} f(\vec x) = f(\vec c)\text{.}
    \]
    In other words, for every neighborhood $N(f(\vec c))$,
    there exists a neighborhood $N(\vec c)$ such that \[
        S \cap N(\vec c) \subseteq f\inv \big(N(f(\vec c))\big)\text{.}
    \]
    If $f$ is continuous at every $\vec c \in S$,
    then $f$ is said to be \textit{continuous}.
}

\thm[limExsThenLocallyBdd]{}{
    Let $f \colon S \to \RR$ where $S \subseteq \RR[n]$.
    Let $\vec c \in \cl S$ where $\lim_{\vec x \to \vec c} f(\vec c) = L$ exists and.
    Then, $f$ is locally bounded on some deleted neighborhood of $\vec c$,
    that is, there are $M, \delta \in \RR_+$ such that \[
        \vec x \in S \cap N'(\vec c; \delta) \implies |f(\vec x)| \le M\text{.}
    \]
}
\pf{Proof}{
    There exists $\delta \in \RR_+$ such that $S \cap N'(\vec c; \delta) \subseteq f\inv(N(L; 1))$.
    Then, $|f(\vec x)| \le |L| + 1$ if $\vec x \in S \cap N'(\vec x; \delta)$.
}

\thm[non0LimThenBddAwayFrom0]{}{
    Let $f \colon S \to \RR$ where $S \subseteq \RR[n]$.
    Let $\vec c \in \cl S$ where $\lim_{\vec x \to \vec c} f(\vec c) = L$ exists and $L \neq 0$.
    Then, $f$ is locally bounded away from $0$ on some deleted neighborhood of $\vec c$,
    that is, there are $m, \delta \in \RR_+$ such that \[
        \vec x \in S \cap N'(\vec c; \delta) \implies |f(\vec x)| \ge m\text{.}
    \]
}
\pf{Proof}{
    There exists $\delta \in \RR_+$ such that $S \cap N'(\vec c; \delta) \subseteq f\inv(N(L; |L|/2))$.
    Then, $|f(\vec x)| \ge |L|/2$ if $\vec x \in S \cap N'(\vec x; \delta)$.
}

\thm[limAlgebraPt]{}{
    Let $f_1 \colon S \to \RR$ and $f_2 \colon S \to \RR$ where $S \subseteq \RR[n]$.
    Let $\vec c \in \cl S$, and suppose $\lim_{\vec x \to \vec c} f_1(\vec x) = L_1$
    and $\lim_{\vec x \to \vec c} f_2(\vec x) = L_2$. Then
    \begin{enumerate}[nolistsep, label=(\roman*)]
        \ii $\lim_{\vec x \to \vec c} \big(f_1(\vec x) + f_2(\vec x)\big) = L_1 + L_2$.
        \ii For any $a \in \RR$, $\lim_{\vec x \to \vec c} af(\vec x) = aL_1$.
        \ii $\lim_{\vec x \to \vec c} f_1(\vec x)\,f_2(\vec x) = L_1L_2$.
        \ii $\lim_{\vec x \to \vec c} f_1(\vec x)/f_2(\vec x) = L_1/L_2$ provided that $L_2 \neq 0$.
    \end{enumerate}
}
\pf{Proof}{
    Proved in MAS102 (Calculus \rom{2}).
}

\thm[squeezeFtn]{The Squeeze Play}{
    Let $f$, $g$, and $h$ be three real-valued functions
    sharing a common domain $S \subseteq \RR[n]$.
    Let $\vec c \in \cl C$ where $\lim_{\vec x \to \vec c} f(\vec x) = \lim_{\vec x \to \vec c} h(\vec x) = L$ exist.
    Suppose also that, for some $\delta_0 \in \RR_+$, \[
        \vec x \in S \cap N'(\vec c; \delta_0) \implies f(\vec x) \le g(\vec x) \le h(\vec x)
    \]
    Then, $\lim_{\vec x \to \vec c} g(\vec x) = L$.
}
\pf{Proof}{
    Proved in MAS102 (Calculus \rom{2}).
}

\thm[limIsOrderPrsvingFtn]{Limit is Order Preserving}{
    Let $f$ and $g$ be two real-valued functions
    sharing a common domain $S \subseteq \RR[n]$.
    Let $\vec c \in \cl C$ where $\lim_{\vec x \to \vec c} f(\vec x) = L_1$ and
    $\lim_{\vec x \to \vec c} g(\vec x) = L_2$ exist.
    Suppose also that, for some $\delta_0 \in \RR_+$, \[
        \vec x \in S \cap N'(\vec c; \delta_0) \implies f(\vec x) \le g(\vec x)
    \]
    Then, $L_1 \le L_2$.
}
\pf{Proof}{
    Proved in MAS102 (Calculus \rom{2}).
}

\thm[limIffEveryCauchyConv]{}{
    Let $S$ be a nonempty subset of $\RR[n]$, $\vec c \in S'$, and $f \colon S \to \RR$.
    $\lim_{\vec x \to \vec c} f(\vec x) = L$ if and only if,
    for every Cauchy sequence $\{\vec x_k\}$ in $S \setminus \{\vec c\}$
    such that $\lim_{k \to \infty} \vec x_k = \vec c$,
    it follows that $\lim_{k \to \infty} f(\vec x_k) = L$.
}
\pf{Proof}{
    ($\Rightarrow$) Let $\{\vec x_k\}$ be any of such Cauchy sequences.
    Take any $\veps \in \RR_+$.
    By continuity, there exists $\delta \in \RR_+$ such that
    $S \cap N'(\vec c; \delta) \subseteq f\inv(N(L; \veps))$.
    On the other hand, by convergence, there exists $k_0 \in \NN$ such that
    $\forall k \in \NN,\: (k \ge k_0 \implies \vec x_k \in N(\vec c; \delta))$.
    Since $\vec x_k \neq \vec c$ for each $k \in \NN$, we may say \[
        \forall k \in \NN,\: \big( k \ge k_0 \implies \vec x_k \in N'(\vec c; \delta)
        \implies \vec x_k \in f\inv(N(L; \veps)) \implies f(\vec x_k) \in N(L; \veps) \big)\text{.}
    \]
    Thus, $\lim_{k \to \infty} f(\vec x_k) = L$ holds.

    ($\Leftarrow$) Suppose it is not $\lim_{\vec x \to \vec c} f(\vec x) = L$.
    Then, it is equivalent to say that, there is some neighborhood $N(L; \veps_0)$
    such that $S \cap N'(\vec c; \delta) \not\subseteq f\inv(N(L; \veps_0))$
    for every deleted neighborhood $N'(\vec x; \delta)$.
    Construct a sequence $\{\vec x_k\}$ in $S \setminus \{\vec c\}$ as following.
    \begin{itemize}[nolistsep]
        \ii $\vec x_1 \in S \setminus \{\vec c\} \setminus f\inv(N(L; \veps_0))$.
        \ii For each $k \in \NN$, $\vec x_{k+1} \in S \cap N'(\vec x; |\vec x_k - \vec c|/2) \setminus f\inv(N(L; \veps_0))$.
    \end{itemize}
    Then, $\lim_{k \to \infty} \vec x_k = \vec c$ indeed holds,
    but it is not $\lim_{k \to \infty} f(\vec x_k) = L$
    since $f(\vec x_k) \notin (N(L; \veps_0))$ for each $k \in \NN$.
}

\thm[contiIffEveryCauchyConv]{}{
    Let $S$ be a nonempty subset of $\RR[n]$, $\vec c \in S$, and $f \colon S \to \RR$.
    $f$ is continuous at $\vec c$ if and only if,
    for every Cauchy sequence $\{\vec x_k\}$ in $S$
    such that $\lim_{k \to \infty} \vec x_k = \vec c$,
    it follows that $\lim_{k \to \infty} f(\vec x_k) = f(\vec c)$.
}
\pf{Proof}{
    ($\Rightarrow$)
    If at most finitely many $\vec x_k$ are distinct from $\vec c$,
    then $\exs k_0 \in \NN,\: \forall k \in \NN_{\ge k_0},\: \vec x_k = \vec c$;
    $\lim_{k \to \infty} f(\vec x_k) = \vec c$ is evident.

    If there are infinitely many $\vec x_k$ are distinct from $\vec c$,
    then we may extract a subsequence $\{\vec{x_{k_j}}\}_{j \in \NN}$
    such that each $\vec x_{k_j}$ is in $S \setminus \{c\}$.
    By \Cref{th:limIffEveryCauchyConv}, $\lim_{j \to \infty} f(\vec x_{k_j}) = \vec c$.
    This implies $\lim_{k \to \infty} f(\vec x_k) = \vec c$,
    regardless of the number of $\vec x_k$'s equal to $\vec c$.

    ($\Leftarrow$)
    If $\vec c \in S'$, then we may directly apply \Cref{th:limIffEveryCauchyConv}
    since every Cauchy sequence in $S \setminus \{\vec c\}$
    is a Cauchy sequence in $S$.

    If $\vec c \notin S'$, then $\vec c$ is an isolated point.
    Then, $\lim_{\vec x \to \vec c} f(\vec x) = f(\vec c)$ by definition.
}

\thm[limAlongAxes]{}{
    Let $S \subseteq \RR[n]$ and $f \colon S \to \RR$.
    Let $\vec c = (c_1, \cdots, c_n) \in \inter S$.
    For $j = 1, 2, \cdots, n$, let \[
        g_j(t) = f(c_1, c_2, \cdots, c_{j-1}, t, c_{j+1}, \cdots, c_n)\text{.}
    \]
    \begin{enumerate}[nolistsep, label=(\roman*)]
        \ii If $\lim_{\vec x \to \vec c} f(\vec x) = L$, then,
            for each $j \in [n]$, $\lim_{t \to c_j} g_j(t) = L$.
        \ii If $f$ is continuous at $\vec c$, then,
            for each $j \in [n]$, $g_j$ is continuous at $c_j$ and
            $\lim_{t \to c_j} g_j(t) = f(\vec c)$.
    \end{enumerate}
}
\pf{Proof}{
    $ $\\[-1em]
    \begin{enumerate}[nolistsep, label=(\roman*)]
        \ii Take any $j \in [n]$ and $\veps \in \RR_+$.
            By convergence, there exists $\delta_1 \in \RR_+$
            such that $S \cap N'(\vec c; \delta_1) \subseteq f\inv(N(L; \veps))$.
            Since $\vec x \in \inter S$, there exists $\delta_2 \in \RR_+$
            such that $N(\vec c; \delta_2) \subseteq S$.
            Let $\delta \triangleq \min \{\delta_1, \delta_2\}$.
            Then, $N'(\vec c; \delta) \subseteq f\inv(N(L; \veps))$ and $N(\vec c; \delta) \subseteq S$ hold.
            Hence, for any $t \in N'(c_j; \delta)$,
            \[
                g_j(t) = f(c_1, \cdots, c_{j-1}, t, c_{j+1}, \cdots, c_n) \in N(L; \veps)
            \]
            as $\|(c_1, \cdots, c_{j-1}, t, c_{j+1}, \cdots, c_n) - \vec c\| = |t - c_j| < \delta$.
        \ii Since $\lim_{\vec x \to \vec c} f(\vec x) = f(\vec c)$,
            by (a), for each $j \in [n]$, $\lim_{t \to c_j} g(t) = f(\vec c) = g(c_j)$.
    \end{enumerate}
}

\nt{
    The converse of \Cref{th:limAlongAxes} is not true.
}

\section{The Topological Description of Continuity}

\thm[contiTopDef]{}{
    A surjective function $f \colon S \to T$ where $S \subseteq \RR[n]$ and $T \subseteq \RR$
    is continuous if and only if $f\inv(U)$ is relatively open in $S$ for every relatively open set $U$ in $T$.
}
\pf{Proof}{
    ($\Rightarrow$)
    Let $U$ be a relatively open set in $T$ and $\vec c \in f\inv(U)$.
    Since $U$ is open and $f(\vec c) \in U$, there is a neighborhood $N(f(\vec c))$
    such that $T \cap N(f(\vec c)) \subseteq U$.
    By continuity, there is a neighborhood $N(\vec c)$ such that
    $S \cap N(\vec c) \subseteq f\inv(N(f(\vec c))) \subseteq f\inv(U)$.
    Therefore, $\vec c$ is a relative interior point of $f\inv(U)$.
    Since $\vec c$ was arbitrary, $f\inv(U)$ is relatively open in $S$.

    ($\Leftarrow$)
    Take any $\vec c \in S$ and a neighborhood $N(f(\vec c))$.
    Then, $f\inv(T \cap N(f(\vec c)))$ is relatively open in $S$.
    Since $\vec c \in f\inv(T \cap N(f(\vec c)))$,
    there is a neighborhood $N(\vec c)$ such that $S \cap N(\vec c) \subseteq f\inv(N(f(\vec c)))$.
}

\thm[contiSendsConToCon]{}{
    If $S$ is a connected subset of $\RR[n]$ and $f$ is continuous on $S$,
    then $T = f(S)$ is also connected.
}
\pf{Proof}{
    Suppose $T$ is disconnected for the sake of contradiction.
    There exists two disjoint open sets $U, V \subseteq \RR$ such that
    $T \subseteq U \cup V$, $T \cap U \neq \varnothing$, and $T \cap V \neq \varnothing$.
    Since $T \cap U$ and $T \cap V$ are relatively open in $T$,
    $U_1 = f\inv(T \cap U)$ and $V_1 = f\inv(T \cap V)$ are relatively open in $S$.
    Then, $S \subseteq U_1 \cup V_1 = S$, $U_1 \cap V_1 = \varnothing$,
    $S \cap U_1 \neq \varnothing$, and $S \cap V_1 \neq \varnothing$,
    which contradicts $S$ is connected, \#.
}

\thm[contiSendsCompToComp]{}{
    If $S$ is a compact subset of $\RR[n]$ and $f$ is continuous on $S$,
    then $T = f(S)$ is also compact.
}
\pf{Proof}{
    Let $\{U_\alpha\}_{\alpha \in J}$ be an open cover of $T$.
    Then, for each $\alpha \in J$, $f\inv(U_{\alpha})$ is relatively open in $S$
    since $U_\alpha$ is open and $f$ is continuous.
    Because \[
        S = f\inv(T) = f\inv \left(\bigcup_{\alpha \in J} U_\alpha\right)
        = \bigcup_{\alpha \in J} f\inv(U_\alpha)\text{,}
    \] $\{f\inv(U_\alpha)\}_{\alpha \in J}$ is a relative open cover of $S$.
    Since $S$ is compact, there is a finite subcover
    $\{\,f\inv(U_{\alpha_i}) \mid i \in [p], \alpha_i \in J\,\}$ of $S$.
    Then, \[
        T = f(S) = f\left(\bigcup_{i=1}^p f\inv(U_{\alpha_i})\right)
        = \bigcup_{i=1}^p f\big(f\inv(U_{\alpha_i})\big) \subseteq \bigcup_{i=1}^p U_{\alpha_i}\text{,}
    \] implying $\{U_{\alpha_i}\}_{i=1}^p$ is a finite subcover of $T$.
}

\thm[evt]{}{
    If $S$ is a compact subset of $\RR[n]$ and $f \colon S \to \RR$ is continous on $S$,
    then $f$ has a minimum and a maximum value on $S$.
}
\pf{Proof}{
    \Cref{th:contiSendsCompToComp} implies $T=f(S) \subseteq \RR$ is compact,
    and thus bounded and closed.
    Thus, $m = \inf T = \min T$ and $M = \sup T = \max T$ exist.
}

\thm[ivt]{The Intermediate Value Theorem}{
    If $f \colon [a, b] \to \RR$ is continuous
    and $c$ is any number between $f(a)$ and $f(b)$,
    then there exists an $x \in [a, b]$ such that $f(x) = c$.
}
\pf{Proof}{
    Since $[a, b]$ is connected and compact,
    \Cref{th:contiSendsConToCon} and \Cref{th:contiSendsCompToComp} imply
    that $f([a, b])$ is connected and compact.
    Thus, $f([a, b]) = [m, M]$ where \[
        m = \min f([a, b]) \le \min \{f(a), f(b)\}
    \] and \[
        M = \max f([a, b]) \ge \max \{f(a), f(b)\}\text{.}
    \]
    This implies $c \in [m, M] = f([a, b])$, i.e.,
    there exists $x \in [a, b]$ such that $f(x) = c$.
}

\thm[genIVT]{The General Intermediate Value Theorem}{
    If $S$ is any connected and compact subset of $\RR[n]$
    and $f \colon S \to \RR$ is continuous,
    if $f(\vec x_1)$ and $f(\vec x_2)$ are any two values of $f$ on $S$,
    and if $c$ is any number between them,
    then there exists a point $\vec x \in S$ such that $f(\vec x) = c$.
}
\pf{Proof}{
    Since $S$ is connected and compact,
    by \Cref{th:contiSendsConToCon} and \Cref{th:contiSendsCompToComp},
    $f(S)$ is an closed interval $[m, M]$ as in the proof of \Cref{th:ivt}.
    Since $m \le \min \{f(\vec x_1), f(\vec x_2)\}$ and
    $M \ge \max \{f(\vec x_1), f(\vec x_2)\}$,
    $c \in [m, M] = f(S)$, and thus $\exs \vec x \in S,\: f(\vec x) = c$.
}

\subsection{The Composition of Continuous Functions}

\thm[compOfContiIsContiPoint]{}{
    Let $S \subseteq \RR[n]$, $f \colon S \to \RR$, $f(S) \subseteq T \subseteq \RR$,
    and $g \colon T \to \RR$.
    If $f$ is continuous at $\vec c \in S$ and if $g$ is continuous at $f(\vec c) \in T$,
    then $g \circ f$ is continuous at $\vec c$.
}
\proof{}{
    Let $d = (g \circ f)(\vec c)$.
    Take any neighborhood $N(d)$ of $d$. By continuity of $g$ at $f(\vec c)$,
    there exists a neighborhood $N(f(\vec c))$ such that \[
        T \cap N(f(\vec c)) \subseteq g\inv(N(d))\text{.}
    \]
    By the continuity of $f$ at $\vec c$,
    there exists a neighborhood $N(\vec c)$ such that \[
        S \cap N(\vec c) \subseteq f\inv(N(f(\vec c)))\text{.}
    \]
    These imply $S \cap N(\vec c) \subseteq f\inv(g\inv(N(d))) = (g \circ f)\inv(N(d))$.
}

\cor{}{
    Let $S \subseteq \RR[n]$, $f \colon S \to \RR$, $f(S) \subseteq T \subseteq \RR$,
    and $g \colon T \to \RR$.
    If $f$ and $g$ are continuous, then $g \circ f$ is continuous.
}

\thm{}{
    If $f \colon [a, b] \to [c, d]$ is strictly monotone, continuous function,
    then the inverse function $f\inv$ is also strictly monotone, continuous, and bijective.
}
\pf{Proof}{
    All are immediate except for the continuity. Denote $f\inv$ by $g$.
    By \Cref{th:contiIffEveryCauchyConv}, it suffices to prove that
    whenever a Cauchy sequence $\{y_k\}$ in $f(S)$ converges to $y$,
    then $\{g(y_k)\}$ converges to $g(y)$ in $S$.

    Choose any such sequence and let $x_k \triangleq g(y_k)$ for each $k \in \NN$.
    Since $g$ is bijective, $\{\,y_k \mid k \in \NN\,\}$ is finite if and only if
    $\{\, x_k \mid k \in \NN\,\}$ is finite.
    If they are finite, then $\{y_k\}$ is eventually $y$, this implies
    $\{x_k\}$ is eventually $g(y)$, and it is done.

    If they are infinite, since domain and codomain are bounded and closed,
    by \Cref{th:BW}, $\{\, x_k \mid k \in \NN\,\}$ has a limit point $x$.
    But since $[a, b]$ is complete by \Cref{th:completeIffClosed},
    $x \in [a, b]$ by (ii) of \Cref{th:completeInX}.
    $x$ is a cluster point of $\{x_k\}$, thus
    there is a subsequence $\{x_{k_j}\}$ of $\{x_k\}$ such that
    $\lim_{j \to \infty} x_{k_j} = x$ by \Cref{th:clusterIffSubseqConv}.
    Now the continuity of $f$ guarantees that \[
        \lim_{j \to \infty} f(x_{k_j}) = f(x)\text{.}
    \]
    At the same time, since $f(x_{k_j}) = y_{k_j}$,
    $\{f(x_{k_j})\}$ is a subsequence of $\{y_k\}$.
    As $\{y_k\}$ converges to $y$, we get \[
        \lim_{j \to \infty} f(x_{k_j}) = y\text{.}
    \]
    By \Cref{th:limDistUnique}, $f(x) = y$, or $x = g(y)$.
    
    If there were another limit point $x'$ of $\{\, x_k \mid k \in \NN\,\}$,
    by the same procedure, we get $x = g(y) = x'$;
    $x = x'$; $x$ is the unique limit point of the set.
    Thus, $\{x_k\}$ converges to $x$, i.e., $\{g(y_k)\}$ converges to $g(y)$.
}

\subsection{Limiting Behavior at Infinity}

\dfn{Function Space $C(S)$ and $C_\infty(S)$}{
    Let $S \neq \varnothing$ be a subset of $\RR[n]$.
    \begin{itemize}[nolistsep]
        \ii $C(S)$ is the set of real-valued function on $S$ which is continuous on $S$.
        \ii $C_\infty(S)$ is the set of real-valued function on $S$ which is bounded and continuous on $S$.
    \end{itemize}
}

\nt{
    In general, $C_\infty(S) \subseteq C(S)$.
    If $\varnothing \neq S \subseteq \RR[n]$ is compact, then $C(S) = C_\infty(S)$.
}

\dfn{Neighborhood of $\infty$ and $-\infty$}{
    In $\RR$,
    \begin{itemize}[nolistsep]
        \ii $N(\infty; M) \triangleq (M, \infty) = \{\,x \in \RR \mid x > M\,\}$
        \ii $N(-\infty, -M) \triangleq (-\infty, -M) = \{\, x \in \RR \mid x < -M \,\}$
    \end{itemize}
    In $\RR[n]$,
    \begin{itemize}[nolistsep]
        \ii $N(\infty; M) \triangleq \{\,\vec x \in \RR[n] \mid \|\vec x\| > M\,\}$
    \end{itemize}
}

\dfn{Limit at Infinity}{
    \begin{enumerate}[nolistsep, label=(\roman*)]
        \ii Let $S$ be an unbounded set in $\RR$. Let $f \colon S \to \RR$.
            \begin{itemize}[nolistsep]
                \ii We say $f$ \textit{has limit} $L$ at $\infty$ if,
                    for all $\veps \in \RR_+$, there exists $M \in \RR_+$ such that
                    $S \cap N(\infty; M) \subseteq f\inv(N(L; \veps))$.
                    We write $\lim_{x \to \infty} f(x) = L$.
                \ii We say $f$ \textit{has limit} $L$ at $-\infty$ if,
                    for all $\veps \in \RR_+$, there exists $M \in \RR_+$ such that
                    $S \cap N(-\infty; -M) \subseteq f\inv(N(L; \veps))$.
                    We write $\lim_{x \to -\infty} f(x) = L$.
            \end{itemize}
        \ii Let $S$ be an unbounded set in $\RR[n]$. Let $f \colon S \to \RR$.
            We say that $f$ \textit{has limit} $L$ at $\infty$,
            if, for all $\veps \in \RR_+$, there exists $M \in \RR_+$ such that
            $S \cap N(\infty; M) \subseteq f\inv(N(L; \veps))$.
            We write $\lim_{\|\vec x\| \to \infty} f(\vec x) = L$.
    \end{enumerate}
}

\thm[squeezeInfty]{The Squeeze Play}{
    Let $f$, $g$, and $h$ be three real-valued functions
    sharing a common unbounded domain $S \subseteq \RR[n]$.
    Suppose $\lim_{\|\vec x\| \to \infty} f(\vec x) = \lim_{\|\vec x\| \to \infty} h(\vec x) = L$.
    Suppose also that, for some $M \in \RR_+$, \[
        \vec x \in S \cap N(\infty; M) \implies f(\vec x) \le g(\vec x) \le h(\vec x)
    \]
    Then, $\lim_{\|\vec x\| \to \infty} g(\vec x) = L$.
}

\thm[limExsAtInfWClosedUnbddDomImplBddF]{}{
    Let $S$ be a closed and unbounded set in $\RR[n]$
    and let $f \in C(S)$.
    Suppose $\lim_{\|\vec x\| \to \infty} f(\vec x) = L$ exists.
    Then $f \in C_\infty(S)$.
}
\pf{Proof}{
    There exists $M \in \RR_+$ such that, for $\vec x \in S \cap N(\infty; M)$,
    $|f(\vec x) - L| < 1$. Thus, for such $\vec x$, we have $|f(x)| < |L| + 1$.

    Since $S \cap \cl{N(\vec 0; M)}$ is a closed, bounded set in $\RR[n]$,
    it is compact by \Cref{th:compactIffClAndBdd}.
    Therefore the continuous $f$ is bounded on $S \cap \cl{N(\vec 0; M)}$
    by \Cref{th:contiSendsCompToComp}.
    In other words, there is some $K \in \RR_+$ such that,
    for $\vec x \in S \cap \cl{N(\vec 0; M)}$, we have $|f(x)| \le K$.
    Thus, $|f(\vec x)| \le \max \{\,K, |L|+1\,\}$ for all $\vec x \in S$.
}

\section{The Algebra of Continuous Functions}

\nt{
    Let $\varnothing \neq S \subseteq \RR[n]$.
    One can easily find that $C(S)$ is a commutative ring and is a vector space.
}

\thm[limAlgebraCS]{}{
    Let $\varnothing \neq S \subseteq \RR[n]$ and $f_1, f_2 \in C(S)$.
    Then, the following hold.
    \begin{enumerate}[nolistsep, label=(\roman*)]
        \ii $f_1 + f_2 \in C(S)$.
        \ii For any $a \in \RR$, $af \in C(S)$.
        \ii $f_1f_2 \in C(S)$.
        \ii $1/f_2 \in C(S)$, provided that $\forall \vec x \in S,\: f_2(\vec x) \neq 0$.
        \ii $f_1/f_2 \in C(S)$, provided that $\forall \vec x \in S,\: f_2(\vec x) \neq 0$.
    \end{enumerate}
}
\pf{Proof}{
    Directly import \Cref{th:limAlgebraPt}.
}

\thm[contiThenLocallyBdd]{}{
    Suppose $f$ is continous at a point $\vec c$ in $\RR[n]$.
    Then $f$ is locally bounded at $\vec c$.
    that is, there are $M, \delta \in \RR_+$ such that \[
        \vec x \in S \cap N(\vec c; \delta) \implies |f(\vec x)| \le M\text{.}
    \]
}

\thm[non0LimThenBddAwayFrom0]{}{
    Suppose $f$ is continous at a point $\vec c$ in $\RR[n]$ and $f(\vec c) \neq 0$.
    Then $f$ is locally bounded away from $0$ at $\vec c$.
    that is, there are $m, \delta \in \RR_+$ such that \[
        \vec x \in S \cap N(\vec c; \delta) \implies |f(\vec x)| \ge m\text{.}
    \]
}

\section{Uniform Continuity}

\dfn{Uniform Continuity}{
    A function $f \colon S \to \RR$ with $S \subseteq \RR[n]$
    is said to be \textit{uniformly continuous on} $S$ if, \[
        \forall \veps \in \RR_+,\: \exs \delta \in \RR_+,\:
        \forall \vec c \in S,\: S \cap N(\vec c; \delta) \subseteq f\inv(N(f(\vec c; \veps)))\text{.}
    \]
    Or, equivalently, \[
        \forall \veps \in \RR_+,\: \exs \delta \in \RR_+,\:
        \forall \vec x, \vec y \in S,\:
        \big( \|\vec x-\vec y\| < \delta \implies |f(x)-f(y)| < \veps \big)\text{.}
    \]
}

\exmp{}{
    $f \colon [0, b] \to \RR$ defined by $f(x) = x^2$ is uniformly continuous on $[0, b]$.
    Given any $\veps \in \RR_+$, let $\delta \triangleq \veps/2b$. Then,
    whenever $|x - y| < \delta$ where $x, y \in [0, b]$,
    $|x^2-y^2| = |x-y|\,|x+y| < \delta \cdot 2b = \veps$.
}

\exmp{}{
    $f \colon (0, M) \to \RR$ defined by $f(x) = 1/x$ is not uniformly continuous on $(0, M)$.
    Let any $\delta \in \RR_+$ is given.
    Let $a \in (0, \min \{\delta, 1/2, M/2\})$.
    Then, $|a - (2a)| = a < \delta$ but $|f(a) - f(2a)| = |1/a - 1/(2a)| = 1/(2a) > 1$.

    This is an example in which $f$ is continuous but the domain is not compact.
}

\exmp{}{
    $f \colon [-1, 1] \to \RR$ defined by $f(x) = \begin{cases}
        1 & \text{if } x \ge 0 \\
        0 & \text{if } x < 0
    \end{cases}$ is not uniformly continuous on $[-1, 1]$.
    Let any $\delta \in \RR_+$ is given.
    Let $a \in (0, \min \{\delta/2, 1\})$.
    Then, $|a - (-a)| = 2a < \delta$ but $|f(a) - f(-a)| = 1 > 0.5$.

    This is an example in which the domain is compact but $f$ is not continuous.
}

\thm{}{
    Suppose that $f$ is continuous on a compact subset $S$ of $\RR[n]$.
    Then $f$ is uniformly continuous on $S$.
}
\pf{Proof}{
    Let $\veps \in \RR_+$ be given.
    Since $f$ is continuous at each point of $S$,
    for each $\vec c$ of $S$, we may choose $\delta(\vec c) \in \RR_+$
    such that \[
        S \cap N(\vec c; \delta(\vec c)) \subseteq f\inv \left(N \left(f(\vec c); \frac{\veps}{2}\right)\right)\text{.}
    \] Then the set $\mcal C \triangleq \{\,N(\vec c; \delta(\vec c)/2) \mid \vec c \in S\,)\}$
    is an open cover of the compact set $S$.
    Since $S$ is compact, there is a finite subcover \[
        \mcal C_1 = \left\{ N \left(\vec c_1; \frac{\delta(\vec c_1)}{2}\right),
        N \left(\vec c_2; \frac{\delta(\vec c_2)}{2}\right), \cdots, N \left(\vec c_k; \frac{\delta(\vec c_k)}{2}\right) \right\}\text{.}
    \]
    Let $\delta_0 \triangleq \min_{i=1}^k \delta(\vec c_i)/2$.

    Now, take any $\vec c \in S$. Since $\mcal C_1$ is an open cover, \[
        \exs i \in [k],\: \vec c \in N \left(\vec c_i; \frac{\delta(\vec c_i)}{2}\right)\text{.}
    \]
    Then, for any $\vec x \in N(\vec c; \delta_0)$, \[
        \|\vec x-\vec c_i\| \le \|\vec x-\vec c\| + \|\vec c-\vec c_i\|
        < \delta_0 + \frac{\delta(\vec c_i)}{2} \le \delta(\vec c_i)\text{.}
    \] Thus, $N(\vec c; \delta_0) \subseteq N(\vec c_i; \delta(\vec c_i))$;
    or \[
        S \cap N(\vec c; \delta_0) \subseteq S \cap N(\vec c_i; \delta(\vec c_i))
        \subseteq f\inv\left(N \left(f(\vec c_i); \frac{\veps}{2}\right)\right)\text{.}
    \]
    Hence, for any $\vec x \in S \cap N(\vec c; \delta_0)$, \[
        |f(\vec x) - f(\vec c)| \le |f(\vec x)-f(\vec c_i)| + |f(\vec c_i) - f(\vec c)|
        < \frac{\veps}{2} + \frac{\veps}{2} = \veps
    \] as $\vec x, \vec c \in S \cap N(\vec c_i; \delta(\vec c_i))$.
}

\section{The Uniform Norm: Uniform Convergence}

\dfn{Function Space $B(S)$}{
    Let $S \neq \varnothing$ be a subset of $\RR[n]$.
    $B(S)$ denotes the vector space and ring of all bounded, real-valued functions on $S$.
}

\nt{
    \begin{itemize}[nolistsep]
        \ii For each $f \in B(S)$, $\sup \{\,|f(\vec x)| \mid \vec x \in S\,\}$ exists.
        \ii $C_\infty(S) = C(S) \cap B(S)$
    \end{itemize}
}

\dfn{Uniform Norm}{
    The \textit{uniform norm} of $f \in B(S)$ is defined to be \[
        \|f\|_\infty = \sup \{\,|f(\vec x)| \mid \vec x \in S\,\}\text{.}
    \]
}

\thm{}{
    The uniform norm is a norm.
}
\pf{Proof}{
    The positive definiteness and the absolute homogeneity is direct.

    Take any $f, g \in f$. Then, for any $\vec x \in S$,
    \[
        |(f+g)(\vec x)| \le |f(\vec x)| + |g(\vec x)| \le \|f\|_\infty + \|g\|_\infty\text{.}
    \]
    Thus, $\|f+g\|_\infty = \sup \{\,|(f+g)(\vec x)| \mid \vec x \in S\,\} \le \|f\|_\infty + \|g\|_\infty$;
    $\|\cdot\|_\infty$ satisfies the subadditivity.
}

\dfn{Uniform Metric}{
    The \textit{uniform metric} on $B(S)$ is \[
        d_\infty(f, g) = \|f - g\|_\infty\text{.}
    \]
}

\nt{
    The uniform metric is naturally a metric since the uniform norm is a norm.
}

\dfn{(Deleted) Uniform Neighborhood}{
    The \textit{(uniform) neighborhood} $N(f; r)$ of $f$ with radius $r$ is the set \[
        N(f; r) \triangleq \{\, g \in B(S) \mid d_\infty(f, g) < r \,\}\text{.}
    \]
    The \textit{deleted (uniform) neighborhood} $N'(f; r)$ of $f$ with radius $r$ is the set \[
        N'(f; r) \triangleq \{\, g \in B(S) \mid 0 < d_\infty(f, g) < r \,\}\text{.}
    \]
}

\dfn{Limit Point of a Set of Functions}{
    A function $f_0 \in B(S)$
    is said to be a \textit{(uniform) limit point} of a set $F \subseteq B(S)$ if
    \[
        \forall \veps \in \RR_+,\: F \cap N'(f_0; \veps) \neq \varnothing\text{.}
    \]
}

\dfn{Convergence of a Sequence of Functions}{
    \begin{itemize}[nolistsep]
        \ii A sequence $\{f_k\}_{k \in \NN}$ in $B(S)$ is said to \textit{converge uniformly}
            to $f_0 \in S \to \RR$ on $S$ if
            \[
                \forall \veps \in \RR_+,\: \exs k_0 \in \NN,\:
                \forall k \in \NN,\: \big(k \ge k_0 \implies f_k \in N(f_0; \veps)\big)\text{.}
            \]
            We write
            \[
                \lim_{k \to \infty} f_k = f_0 \text{ [uniformly].}
            \]
        \ii A sequence $\{f_k\}_{k \in \NN}$ in $B(S)$ is said to \textit{converge pointwise}
            to $f_0 \colon S \to \RR$ on $S$ if
            \[
                \forall (\vec c, \veps) \in S \times \RR_+,\: \exs k_0 \in \NN,\:
                \forall k \in \NN,\: \big(k \ge k_0 \implies f_k(\vec c) \in N(f_0(\vec c); \veps)\big)\text{.}
            \]
            We write
            \[
                \lim_{k \to \infty} f_k = f_0 \text{ [pointwise].}
            \]
        \ii A sequence $\{f_k\}_{k \in \NN}$ in $B(S)$ is said to be \textit{(uniformly) Cauchy} if
            \[
                \forall \veps \in \RR_+,\: \exs k_0 \in \NN,\:
                \forall k, m \in \NN,\: \big( k, m \ge k_0 \implies \|f_m - f_k\|_\infty < \veps \big)\text{.}
            \]
    \end{itemize}
}

\nt{
    A pointwise convergent sequence in $C_\infty(S)$
    may fail to have a limit that is in $C_\infty(S)$.
}

\thm[unifConvSeqContiFtnConvToConti]{}{
    Let $\varnothing \neq S \subseteq \RR[n]$.
    Suppose that $\{f_k\}$ is a sequence in $C(S)$ and
    it converges uniformly to $f_0 \colon S \to \RR$ on $S$.
    Then $f_0 \in C(S)$.
}
\pf{Proof}{
    Take any $\vec c \in S$ and $\veps \in \RR_+$.
    By uniform convergence, there exists $k \in \NN$ such that
    \[
        \|f_k - f_0\|_\infty < \frac{\veps}{4}.
    \]
    Since $f_k$ is continuous, there exists $\delta \in \RR_+$ such that
    \[
        S \cap N(\vec c; \delta) \subseteq f_k\inv \left(N \left(f_k(\vec c); \frac{\veps}{2}\right)\right)\text{.}
    \]
    Thus, for any $\vec x \in S \cap N(\vec c; \delta)$,
    \[
        \begin{aligned}[t]
            |f_0(\vec x) - f_0(\vec c)| &\le |f_0(\vec x) - f_k(\vec x)| + |f_k(\vec x) - f_k(\vec c)| + |f_k(\vec c) - f_0(\vec c)| \\
                                        &< \frac{\veps}{4} + \frac{\veps}{2} + \frac{\veps}{4} = \veps\text{.}
        \end{aligned}
    \]
    This exactly means that $f_0$ is continuous at $\vec c$.
    Since $\vec c$ is arbitrary, $f_0$ is continous on $S$.
}

\thm[cauchyBddFtn]{}{
    Let $\varnothing \neq S \subseteq \RR[n]$.
    A Cauchy sequence $\{f_k\}$ in $B(S)$ is bounded.
    That is, $\exs M \in \RR_+,\: \forall k \in \NN,\: \|f_k\|_\infty \le M$.
}
\pf{Proof}{
    Immitate the proof of \Cref{th:cauchyBdd}.
}

\thm[CinfIsComplete]{}{
    $C_\infty(S)$ is complete.
    That is, given any Cauchy sequence $\{f_k\}$ in $C_\infty(S)$,
    there exists $f_0 \in C_\infty(S)$ such that
    $\lim_{k \to \infty} f_k = f_0$ [uniformly] on $S$.
}
\pf{Proof}{

}

\end{document}
