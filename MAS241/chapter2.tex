\documentclass[MAS241_Note.tex]{subfiles}

\begin{document}
\chapter{Euclidean Spaces}
\section{Euclidean $n$-Space}

\dfn{Inner Product}{
	The \textit{inner product} of two vectors $\vec{x} = (x_1, x_2, \cdots, x_n)$ and $\vec{y} = (y_1, y_2, \cdots, y_n)$
	in $\RR[n]$ is \[
		\langle \vec{x}, \vec{y} \rangle = \sum_{j=1}^{n} x_jy_j\text{.}
	\]
}
\thm{}{
	If $\vec{x}$, $\vec{y}$, and $\vec{z}$ are arbitrary vectors in $\RR[n]$
	and if $a$ and $b$ are real numbers, then the following hold:
	\begin{enumerate}[label=(\roman*),nolistsep]
		\ii The inner product is \textit{additive} in both its variables: \[
			      \begin{aligned}[t]
				      \lang \vec{x} + \vec{y}, \vec{z} \rang & = \lang \vec{x}, \vec{z} \rang + \lang \vec{y}, \vec{z} \rang \\
				      \lang \vec{x}, \vec{y} + \vec{z} \rang & = \lang \vec{x}, \vec{y} \rang + \lang \vec{x}, \vec{z} \rang
			      \end{aligned}
		      \]
		\ii The inner product is \textit{symmetric}: $\lang \vec{x}, \vec{y} \rang = \lang \vec{y}, \vec{x} \rang$.
		\ii The inner product is \textit{homogeneous} in both its variables:
		      $\lang a\vec x, b \vec y \rang = ab \lang \vec x, \vec y \rang$.
	\end{enumerate}
}

\dfn{Euclidean Norm}{
	The \textit{Euclidean norm} of a vector $\vec x$ in $\RR[n]$ is \[
		\|\vec x\| = \sqrt{\lang \vec x, \vec x \rang}\text{.}
	\]
}

\thm[CSIneq]{The Cauchy-Schwarz Inequality}{
	If $\vec x, \vec y \in \RR[n]$, then \[
		|\langle \vec x, \vec y \rangle| \le \|\vec x\| \cdot \|\vec y\|\text{.}
	\]
}
\pf{Proof}{
	For any $t \in \RR$, $0 \le \|t\vec x + \vec y\|^2 = \|\vec x\|^2 t^2 + 2\lang \vec x, \vec y \rang t + \|\vec y\|^2$.
	Thus, the discriminant $|\lang \vec x, \vec y \rang|^2 - \|\vec x\|^2\|\vec y\|^2$ is nonpositive.
}

\thm{}{
	For vectors $\vec x$ and $\vec y$ in $\RR[n]$ and any $c \in \RR$, the Euclidean norm has the following properties.
	\begin{enumerate}[label=(\roman*), nolistsep]
		\ii $\|\vec x\| \ge 0$; $\|\vec x\| = 0$ if and only if $\vec x = \vec 0$. (\textit{Positive Definiteness})
		\ii $\|c\vec x\| = |c| \cdot \|\vec x\|$. (\textit{Absolute Homogeneity})
		\ii $\|\vec x + \vec y\| \le \|\vec x\| + \|\vec y\|$. (\textit{Subadditivity})
	\end{enumerate}
}
\pf{Proof of (iii)}{
	\[
		\begin{aligned}[t]
			0 \le \|\vec x + \vec y\|^2 & = \|\vec x\|^2 + 2\lang \vec x, \vec y \rang + \|\vec y\|^2                           \\
			                            & \le \|\vec x\|^2 + 2\|\vec x\|\|\vec y\| + \|\vec y\|^2 = (\|\vec x\| + \|\vec y\|)^2
		\end{aligned}
	\]
}

\dfn{Norm}{
	A \textit{norm} on $\RR[n]$ is any function $n \colon \RR[n] \to \RR$
	that is positive definite, absolutely homogeneous, and subadditive.
}

\dfn{Metric}{
	A \textit{metric} on $\RR[n]$ is a function from $\RR[n] \times \RR[n] \to \RR$
	having the following properties.
	\begin{enumerate}[label=(\roman*), nolistsep]
		\ii $\forall \vec x, \vec y \in \RR[n],\: d(\vec x, \vec y) \ge 0$; $d(\vec x, \vec y) = 0 \iff \vec x = \vec y$.
		      (\textit{Positive Definiteness})
		\ii $\forall \vec x, \vec y \in \RR[n],\: d(\vec x, \vec y) = d(\vec y, \vec y)$. (\textit{Symmetry})
		\ii $\forall \vec x, \vec y, \vec z \in \RR[n],\: d(\vec x, \vec z) \le d(\vec x, \vec y) + d(\vec y, \vec z)$.
		      (\textit{The Triangle Inequality})
	\end{enumerate}
}

\dfn{Euclidean Metric}{
	The \textit{Euclidean metric} on $\RR[n]$ is defined by \[
		d(\vec x, \vec y) = \|\vec x-\vec y\| = \left[\sum_{j=1}^{n} (x_j-y_j)^2\right]^{1/2}\text{.}
	\]
}

\thm{}{
	The Euclidean metric is a metric on $\RR[n]$.
}

\dfn{Orthogonality}{
	Two vectors $\vec x$ and $\vec y$ in $\RR[n]$ are said to be \textit{orthogonal}
	if $\lang \vec x, \vec y \rang = 0$.
}

\dfn{Neighborhood and Deleted Neighborhood}{
	A \textit{neighborhood} $N(\vec x; r)$ or $\vec x \in \RR[n]$ with radius $r$ is the set \[
		N(\vec x; r) = \{\, \vec y \in \RR[n] \mid \|\vec x - \vec y\| < r \,\}\text{.}
	\] A \textit{deleted neighborhood} $N'(\vec x, r)$ of $\vec x$ is $N'(\vec x; r) = N(\vec x; r) \setminus \{\vec x\}$.
}

\dfn{Limit Point}{
	Let $S$ be nonempty subset of $\RR[n]$.
	We say that $\vec x$ is a \textit{limit point} of $S$ if \[
		\forall \veps \in \RR_+,\: N'(\vec x; \veps) \cap S \neq \varnothing\text{.}
	\]
}

\thm{}{
	$\QQ[n]$ is dense in $\RR[n]$.
}
\pf{Proof}{
    Take any $\vec y = (y_1, y_2, \cdots, y_n) \in \RR[n]$ and $\veps \in \RR_+$.
    For each $j=1,2,\cdots,n$, choose a rational $x_j \in N(y_j; \veps/\sqrt{n})$
    and form $\vec x \triangleq (x_1, x_2, \cdots, x_n) \in \QQ[n]$. Then, \[
        \|\vec x - \vec y\|^2 = \sum_{j=1}^{n} (x_j-y_j)^2 < n(\veps/\sqrt{n})^2 = \veps^2\text{.}
    \] Therefore $\vec y$ is a limit point of $\QQ[n]$.
}

\dfn{Boundedness}{
	A subset $S$ of $\RR[n]$ is said to be \textit{bounded} if \[
		\exs M \in \RR_+,\: \forall\vec x \in S,\: \|\vec x\| \le M\text{.}
	\]
}

\subsection{Sequences in $\RR[n]$}

\dfn{Cluster Point}{
	$\vec c \in \RR[n]$ is a \textit{cluster point} of the sequence $\{\vec x_k\}$ if, \[
		\forall (\veps, k) \in \RR_+ \times \NN,\: \exists k_1 \in \NN_{>k},\: \vec x_{k_1} \in N(\vec c; \veps)\text{.}
	\]
}

\dfn{Convergence and Divergence of a Sequence}{
	The sequnce $\{\vec x_k\}$ \textit{converges} to $\vec x_0$ and $\vec x_0$ is the \textit{limit} of $\{\vec x_k\}$ if, \[
		\forall \veps \in \RR_+,\: \exists k_0 \in \NN,\: \forall k \in \NN_{\ge k_0},\: \vec x_k \in N(\vec x_0; \veps)\text{.}
	\] We write $\displaystyle \lim_{k \to \infty} \vec x_k = \vec x_0$.
	If there is no such $\vec x_0$, then $\{\vec x_k\}$ \textit{diverges}.
}

\thm[seqConvIffCoordsConv]{}{
    Let $\{\vec x_k\} = (x_1^{(k)}, x_2^{(k)}, \cdots, x_n^{(k)})$ for each $k \in \NN$.
    Let $\vec x_0 = (x_1^{(0)}, x_2^{(0)}, \cdots, x_n^{(0)})$.
    The sequence $\{\vec x_k\}$ converges to $\vec x_0$ if and only if,
    for each $j \in [n]$, the sequence $\{x_j^{(k)}\}$ converges to $\{x_j^{(0)}\}$.
}
\pf{Proof}{
    ($\Rightarrow$) Take any $\veps \in \RR_+$.
    There there is $k_0 \in \NN$ such that \[
        \forall k \in \NN_{\ge k_0}, \vec x_k \in N(\vec x_0; \veps)\text{.}
    \]
    Then, for each $j \in [n]$, \[
        \left(x_j^{(k)} - x_0^{(k)}\right)^2 \le \sum_{i=1}^{n} \left(x_i^{(k)}-x_0^{(k)}\right)^2 = \|\vec x_k - \vec x_0\|^2 < \veps\text{.}
    \]
    ($\Leftarrow$) Take any $\veps \in \RR_+$. Then, for each $j \in [n]$,
    there is some $k_j \in \NN$ such that \[
        \forall k \in \NN_{\ge k_j},\: x_j^{(k)} \in N(x_0^{(k)}; \veps/\sqrt{n})\text{.}
    \] Then, for all natural number $k$ not smaller than $\max_{j \in [n]} k_j$, \[
        \|\vec x_k - \vec x_0\|^2 = \sum_{j=1}^{n} \left(x_j^{(k)}-x_0^{(k)}\right)^2 < n(\veps/\sqrt{n})^2 = \veps^2\text{.}
    \]
}

\dfn{Cauchy Seqeunce}{
	A seqeunce $\{\vec x_k\}$ in $\RR[n]$ is called a \textit{Cauchy sequence} if \[
		\forall \veps \in \RR_+,\: \exs k_0 \in \NN,\: \forall k, m \in \NN_{\ge k_0},\: \|\vec x_k-\vec x_m\| < \veps\text{.}
	\]
}

\thm[convIffCauchyRn]{Cauchy's Completeness Theorem in $\RR[n]$}{
	A sequence $\{\vec x_k\}$ in $\RR[n]$ is Cauchy if and only if it converges.
	$\RR[n]$ is Cauchy complete.
}
\pf{Proof}{
    ($\Leftarrow$) The proof if similar to \Cref{th:convThenCauchy}. \par
    ($\Rightarrow$) Let some Cauchy sequence $\{\vec x_k\}$ in $\RR[n]$ be given. Take any $\veps \in \RR_+$.
    There is some $k_0 \in \NN$ such that for every natural number $k$ and $m$ not smaller than $k_0$,
    $\|\vec x_k - \vec x_m\| < \veps$.
    Then, for each $j \in [n]$, $|x_j^{(k)} - x_j^{(m)}| \le \|\vec x_k - \vec x_m\| < \veps$,
    which implies each $\{x_j^{(k)}\}_{k \in \NN}$ is Cauchy.
    By \Cref{th:convIffCauchy}, $\{x_j^{(k)}\}_{j \in \NN}$ converges to some number $x_j^{(0)}$.
    Then, \Cref{th:seqConvIffCoordsConv} ensures that $\lim_{k \to \infty} \vec x_k = (x_1^{(0)}, x_2^{(0)}, \cdots, x_n^{(0)})$.
}

\thm[BW2]{The Generalized Bolzano–Weierstrass Theorem}{
	Every bounded infinite set in $\RR[n]$ has a limit point in $\RR[n]$.
}
\pf{Proof}{
    Suppose that $S$ is any bounded, infinite set in $\RR[n]$.
    Being bounded, $S$ is contained in some $n$-cube $C(2M) = [-M, M]^n$ centered at $\vec 0$.
    Construct $C_1, C_2, \cdots$ as following.
    \begin{itemize}[nolistsep]
        \ii $C_1 \triangleq C(2M) = [a_1^{(1)}, b_1^{(1)}] \times \cdots \times [a_n^{(1)}, b_n^{(1)}]$
            \begin{itemize}[nolistsep]
                \ii Note that $C_1 \cap S = S$ is infinite.
            \end{itemize}
        \ii For each $k \in \NN$, $C_{k+1}$ is any cube of the form $[a_1^{(k+1)}, b_1^{(k+1)}] \times \cdots \times [a_n^{(k+1)}, b_n^{(k+1)}]$
            where each $[a_j^{(k+1)}, b_j^{(k+1)}]$ is either
            $[a_j^{(k)}, (a_j^{(k)}+b_j^{(k)})/2]$ or $[(a_j^{(k)}+b_j^{(k)})/2, b_j^{(k)}]$
            so that $C_{k+1} \cap S$ is infinite.
            \begin{itemize}[nolistsep]
                \ii This is possible since there is at least one cube among $2^n$ possible choices
                    that $C_{k+1} \cap S$ is infinite.
            \end{itemize}
    \end{itemize}
    Then, the main diagonal $d_k$ of $C_k$ equals to $Mn^{1/2}/2^{k-2}$.
    Also, note that $C_k \supseteq C_{k+1}$ for all $k \in \NN$.

    Now, we may construct a sequence $\{\vec x_k\}_{k \in \NN}$ as following.
    \begin{itemize}[nolistsep]
        \ii $\vec x_1$ is any element in $C_1 \cap S$.
        \ii For each $k \in \NN$, $\vec x_{k+1}$ is arbitrarily taken from
            $C_{k+1} \cap S \setminus \{\, \vec x_1, \vec x_2, \cdots, \vec x_k \,\}$.
    \end{itemize}
    We claim that $\{\vec x_k\}_{k \in \NN}$ is a Cauchy sequence. To show this, take any $\veps \in \RR_+$.
    There is some $k_0 \in \NN$ such that $d_{k_0} = Mn^{1/2}/2^{k_0-2} < \veps$ by \Cref{th:archi}.
    Then, for all $k, m \in \NN_{\ge k_0}$, $\|\vec x_k - \vec x_m\| \le d_{k_0} < \veps$.
    Therefore, since $\{\vec x_k\}$ is Cauchy, and therefore convergent by \Cref{th:convIffCauchyRn}.

    Clearly, $\vec x_0 \triangleq \lim_{k \to \infty} \vec x_k$ is a limit point of $S$
    since any deleted neighborhood $N'(\vec x_0)$ of $\vec x_0$
    intersects infinitely many points with $\{\vec x_k\}_{k \in \NN} \subseteq S$.
}

\dfn{Subsequence}{
    Let $\{\vec x_k\}$ be any sequence in $\RR[n]$.
    Choose any strictly monotone increasing sequnce $k_1 < k_2 < k_3 <\cdots$ of natural numbers.
    For each $j \in \NN$, let $\vec y_j \coloneqq \vec x_{k_j}$.
    The sequence $\{\vec y_j\}_{j=1}^\infty$ is called an \textit{subsequence} of $\{\vec x_k\}$.
}

\thm{}{
	The point $\vec c$ is a cluster point of $\{\vec x_k\}$
    if and only if there exists a subsequence of $\{\vec x_k\}$ that converges to $\vec c$.
}
\pf{Proof}{
    Analogous to \Cref{th:clusterIffSubseqConv}.
}

\thm[bddRnSeqHasCluster]{}{
	Any bounded sequence $\{\vec x_k\}$ has a cluster point.
}
\pf{Proof}{
    Analogous to \Cref{th:bddSeqHasCluster}.
}

\cor{}{
    If a sequence in $\RR[n]$ has no cluster point, then the sequence is unbounded.
}

\cor[bddRnSeqConvIffOneCluster]{}{
    Any bounded sequence in $\RR[n]$ converges if and only if it has exactly one cluster point.
}

\cor{}{
	A sequence $\{\vec x_k\}$ diverges if and only if at least one of the following conditions holds.
	\begin{itemize}[nolistsep]
		\ii $\{\vec x_k\}$ has two or more cluster points.
		\ii $\{\vec x_k\}$ is unbounded.
	\end{itemize}
}

\section{Open and Closed Sets}

\dfn{Interior/Boundary Point and Open/Closed Set}{
    Let $S$ be any subset of $\RR[n]$ and let $\vec x$ be any point in $\RR[n]$.
    \begin{enumerate}[label=(\roman*), nolistsep]
        \ii $\vec x$ is an \textit{interior point} of $S$ if $\exs r \in \RR_+,\: N(\vec x; r) \subseteq S$.
        \ii If every point of $S$ is an interior point of $S$, then $S$ is said to be \textit{open}.
        \ii We call $\vec x$ is a \textit{boundary point} of $S$ if
              $\forall r \in \RR_+,\: N(\vec x; r) \cap S \neq \varnothing \land N(\vec x; r) \setminus S \neq \varnothing$.
        \ii If $S$ continas all its boundary points, then $S$ is said to be \textit{closed}.
    \end{enumerate}
}

\dfn{}{
    Let $S \subseteq \RR[n]$.
    \begin{enumerate}[label=(\roman*), nolistsep]
        \ii The \textit{interior} of $S$, denoted $\inter S$, is the set of all interior points of $S$.
        \ii The \textit{boundary} of $S$, denoted $\bd S$, is the set of all boundary points of $S$.
        \ii The \textit{derived set} of $S$, denoted $S'$, is the set of all limit points of $S$.
        \ii The \textit{closure} of $S$, denoted $\cl S$, is the union of $S$ and $S'$.
        \ii The \textit{complement} of $S$, denoted $\cmpl S$, is the set $\RR[n] \setminus S$.
    \end{enumerate}
}

\nt{
    \begin{itemize}[nolistsep]
        \ii For $S \subseteq \RR[n]$, $\inter S \subseteq S \subseteq \cl S$.
        \ii For $S \subseteq \RR[n]$, $S$ is open if and only if $\inter S = S$.
        \ii For $S \subseteq \RR[n]$, $\inter S$ is open.
    \end{itemize}
}

\thm[UofOpenIsOpen]{}{
    The union of any collection of open sets in $\RR[n]$ is open.
    The intersection of any finite collection of open sets in $\RR[n]$ is also open.
}
\pf{Proof}{
    To prove the first assertion, suppose that $\{\,U_{\alpha} \mid \alpha \in J \,\}$
    is any collection of open sets in $\RR[n]$. Let $U \triangleq \bigcup_{a \in J} U_\alpha$.
    Take any $\vec x \in U$. Then, there is some $\alpha_0 \in J$ such that $\vec x \in U_{\alpha_0}$.
    Since $U_{\alpha_0}$ is open, there is some neighborhood $N(\vec x; \veps)$ such that
    $N(\vec x; \veps) \subseteq U_{\alpha_0}$, which, in turn, $N(\vec x; \veps) \subseteq U$.
    Therefore, $\vec x$ is an interior point of $U$; $U$ is open.

    To prove the second assertion, let $U$ be the intersection of any finite collection
    $\{\,U_1, U_2, \cdots, U_k\,\}$ of open sets and take any $\vec x \in U$.
    For each $j \in [k]$, since $\vec x \in U_j$, there is some $r_j \in \RR_+$ such that
    $N(\vec x; r_j) \subseteq U_j$.
    Then, take $r_0 \triangleq \min_{j \in [k]} r_j \in \RR_+$.
    Since, for all $j \in [k]$, $N(\vec x; r_0) \subseteq U_j$, it is implied that $N(\vec x; r_0) \subseteq U$.
    Therefore, $\vec x$ is an interior point of $U$; $U$ is open.
}

\nt{
    Intersection of infinitely many open sets may fail to be open. For instance, consider \[
        U_k \triangleq N(\vec 0; 1/k)\text{,}
    \] for each $k \in \NN$. Then, $\bigcap_{k \in \NN} U_k = \{\vec 0\}$, which is not open.
}

\thm[closedIffCmplOpen]{}{
    A set $C \subseteq \RR[n]$ is closed if and only if $C^{c}$ is open.
}
\pf{Proof}{
    ($\Rightarrow$) Take any $\vec x \in \cmpl C$. Since $C$ is closed and contains all of its boundary points,
    $\vec x$ is not a boundary point of $C$. Therefore, there is some neighborhood $N(\vec x)$ of $\vec x$
    such that $N(\vec x) C = \varnothing$ or $N(\vec x) \cap \cmpl C = \varnothing$.
    The second case is not possible since $\vec x \in N(\vec x) \cap \cmpl C$.
    Therefore, $N(\vec x) = \varnothing$, which implies $N(\vec x) \subseteq \cmpl C$;
    $\vec x$ is an interior point of $\cmpl C$. Therefore, $\cmpl C$ is open.

    ($\Leftarrow$) Take any bounddary point $\vec x$ of $C$. Assume $\vec x \in \cmpl C$ for the sake of contradiction.
    Since $\cmpl C$ is open, there is a neighborhood $N(\vec x)$ of $\vec x$ such that $N(\vec x) \subseteq \cmpl C$.
    However, that implies $N(\vec x) \cap C = \varnothing$, which contradicts $\vec x$ is a boundary point of $C$.
    Therefore, $\vec x \in C$; $C$ contains all of its boundary points.
}

\thm[IofClosedIsClosed]{}{
    The intersection of any collection of closed sets in $\RR[n]$ is closed.
    The union of any finite collection of closed sets in $\RR[n]$ is also closed.
}
\pf{Proof}{
    To prove the first assertion, let $\{C_\alpha\}_{\alpha \in J}$ be any collection of closed sets in $\RR[n]$.
    Then, each $\cmpl{C_\alpha}$ is open by \Cref{th:closedIffCmplOpen},
    and thus $\bigcup_{\alpha \in J} \cmpl{C_\alpha}$ is open by \Cref{th:UofOpenIsOpen}.
    Its complement $\cmpl{\left(\bigcup_{\alpha \in J} \cmpl{C_\alpha}\right)}$ is closed by \Cref{th:closedIffCmplOpen}.
    And note that $\cmpl{\left(\bigcup_{\alpha \in J} \cmpl{C_\alpha}\right)} = \bigcap_{\alpha \in J} C_\alpha$
    by De Morgan's law.

    To prove the second assertion, let $\{\,C_1,C_2,\cdots,C_k\,\}$ be a finite collection of closed sets in $\RR[n]$.
    Then, each $\cmpl{C_i}$ is open by \Cref{th:closedIffCmplOpen},
    and thus $\bigcap_{i=1}^k \cmpl{C_i}$ is open by \Cref{th:UofOpenIsOpen}.
    Its complement $\cmpl{\left(\bigcap_{i=1}^k \cmpl{C_i}\right)}$ is closed by \Cref{th:closedIffCmplOpen}.
    And note that $\cmpl{\left(\bigcap_{i=1}^k \cmpl{C_i}\right)} = \bigcup_{i=1}^k C_i$
    by De Morgan's law.
}

\thm[closedIffContainsLimPts]{}{
    $C \subseteq \RR[n]$ is closed if and only if $C' \subseteq C$.
}
\pf{Proof}{
    ($\Rightarrow$) Let $\vec x \in C'$. Assume $\vec x \in \cmpl C$ for the sake of contradiction.
    Since $\cmpl C$ is open by \Cref{th:closedIffCmplOpen},
    there is a neighborhood $N(\vec x)$ of $\vec x$ such that $N(\vec x) \subseteq \cmpl C$.
    Such $N(\vec x)$ satisfies $N(\vec x) \cap C = \varnothing$, which contradicts $\vec x \in C'$.
    Therefore, $\vec x \in C$; $C$ contains all its limit points.

    ($\Leftarrow$) It is enough to prove $\cmpl C$ is open by \Cref{th:closedIffCmplOpen}.
    Take any $\vec x \in \cmpl C$. $\vec x$ is not a limit point of $C$ by the hypothesis.
    Therefore, there is a deleted neighborhood $N'(\vec x)$ of $\vec x$ such that $N'(\vec x) \cap C = \varnothing$.
    Then, $N'(\vec x) \subseteq \cmpl C$, and thus $N(\vec x) \subseteq \cmpl C$,
    which implies $\vec x$ is an interior point of $\cmpl C$. Thus, $\cmpl C$ is open.
}

\cor{}{
    $C \subseteq \RR[n]$ is closed if and only if $\cl C = C$.
}

\thm[intIsUOfOpenSetsInS]{}{
    Let $S \subseteq \RR[n]$. The interior of $S$ is the union of all open sets contained in $S$.
}
\pf{Proof}{
    Let $\mcal U \triangleq \{\, U \subseteq S \mid U \text{ is open in } \RR[n]\}$. \par
    ($\subseteq$) Let $\vec x \in \inter S$. Then, there is an open neighborhood
    $N(\vec x)$ of $\vec x$ such that $N(\vec x) \subseteq S$.
    Noting that $\vec x \in N(\vec x) \in \mcal U$, we conclude $\inter S \subseteq \bigcup \mcal U$. \par
    ($\supseteq$) Take any $\vec x \in \bigcup \mcal U$.
    Then, there is an open set $U$ in $\RR[n]$ such that $x \in U \subseteq S$.
    There is a neighborhood $N(\vec x)$ of $\vec x$ such that $N(\vec x) \subseteq U$.
    Therefore, $N(\vec x) \subseteq S$; $\vec x$ is an interior point of $S$.
    Thus; $\inter S \supseteq \bigcup \mcal U$.
}

\thm{}{
    The closure of $S$ is the intersection of all closed sets that contain $S$.
}
\pf{Proof}{
    Let $\mcal C \triangleq \{\, C \subseteq \RR[n] \mid S \subseteq C \text{ and } C \text{ is closed}\,\}$. \par
    ($\subseteq$) Since $S \subseteq \bigcap \mcal C$ is obvious, we only need to show $S' \subseteq \bigcap \mcal C$.
    Let $\vec x \in S'$. Then, it is direct that $\forall C \in \mcal C,\: \vec x \in C'$ since each $C \in \mcal C$
    satisfies $S \subseteq C$.
    As $C$ is closed and thus $\vec x \in C' \subseteq C$ by \Cref{th:closedIffContainsLimPts},
    Consequently, $\vec x \in \bigcap \mcal C$; $\cl S \subseteq \bigcap \mcal C$. \par
    ($\supseteq$) It is enough to show that $\cl S$ is closed, which, in turn,
    is sufficient to show that $(\cl S)' \subseteq \cl S$ by \Cref{th:closedIffContainsLimPts}.
    Let $\vec y \in (\cl S)'$ and take any deleted neighborhood $N'(\vec y; \veps)$ of $\vec y$.
    Then, there is some element $\vec z$ in $N'(\vec y; \veps) \cap \cl S$. Then, $\vec z \in S$ or $\vec z \in S'$.

    If $\vec z \in S$, then $\vec z \in N'(\vec y; \veps) \cap S \neq \varnothing$.
    If $\vec z \in S'$, take $\veps' \triangleq \min \{\,\|\vec z - \vec y\|, \veps - \|\vec z - \vec y\|\}$.
    Then, $N(\vec z; \veps') \subseteq N'(\vec y; \veps)$.
    Since $\vec z \in S'$, there is some $\vec x$ in $N'(\vec z; \veps') \cap S$.
    Thus, $\vec x \in N'(\vec z; \veps') \cap S \subseteq N'(\vec y; \veps) \cap S \neq \varnothing$.
    
    In both cases, $N'(\vec y; \veps) \cap S \neq \varnothing$. Thus, we proved that $\vec y \in S' \subseteq \cl S$;
    $(\cl S)' \subseteq \cl S$.
}

\cor[closureIsClosed]{}{
    For any $S \subseteq \RR[n]$, the set $\cl{S}$ is closed.
}

\cor[closedIffClosureOfItself]{}{
    For any $C \subseteq \RR[n]$, $C$ is closed if and only if $\cl C = C$.
}

\thm{}{
    Let $S \subseteq \RR[n]$.
    \begin{enumerate}[label=(\roman*), nolistsep]
        \ii $\inter{\inter S}= \inter S$
        \ii $\cl{(\cl S)} = \cl S$
        \ii $\inter S \cap \bd S = \varnothing$
        \ii $\inter S \cup \bd S = \cl S$
        \ii $\cl S \cap \cl{\cmpl S} = \bd S$
    \end{enumerate}
}
\pf{Proof}{
    $ $\\[-1em]
    \begin{enumerate}[label=(\roman*), nolistsep]
        \ii $\inter S$ is open and an open set is the interior of itself.
        \ii $\cl S$ is closed and a closed set is the closure of itself.
            (See \Cref{cor:closureIsClosed} and \Cref{cor:closedIffClosureOfItself}).
        \ii Suppose there is some $\vec x \in \inter S \cap \bd S$.
            There is a neighborhood $N(\vec x)$ of $\vec x$ such that $N(\vec x) \subseteq S$.
            Then, $N(\vec x) \cap \cmpl C = \varnothing$, which contradicts $\vec x \in \bd S$.
        \ii ($\subseteq$) Since it is already $\inter S \subseteq S \subseteq \cl S$, we only need to show $\bd S \subseteq \cl S$.
            Let $\vec x \in \bd S$. If $\vec x \in S$, then it is done; so suppose $\vec x \in \cmpl S$.
            Take any neighborhood $N(\vec x; \veps)$ of $\vec x$. Then, $N(\vec x; \veps) \cap S \neq \varnothing$.
            Noting that $N'(\vec x; \veps) \cap S = N(\vec x; \veps) \cap S \neq \varnothing$,
            $\vec x \in S'$. \par
            ($\supseteq$) Let $\vec x \in \cl S$. If $\vec x \in S$, then it is either
            ``There is a neighborhood $N(\vec x)$ of $\vec x$ such that $N(\vec x) \subseteq S$.''
            or ``Every neighborhood $N(\vec x)$ of $\vec x$ satisfies $N(\vec x) \cap \cmpl S \neq \varnothing$.''
            The first case is $\vec x \in \inter S$ and the latter case is $\vec x \in \bd S$.

            Now the only left case if $\vec x \in S' \setminus S$.
            Take any deleted neighborhood $N'(\vec x)$ of $\vec x$.
            Then, $N(\vec x) \cap S = N'(\vec x) \cap S \neq \varnothing$.
            Also, $\vec x \in N(\vec x) \cap \cmpl S$. Thus, $\vec x \in \bd S$.
        \ii Using $\cl S = \inter S \cup \bd S$, we get \[
                \begin{aligned}[t]
                    \cl S \cap \cl{\cmpl S} & = (\inter S \cup \bd S) \cap (\inter{(\cmpl S)} \cup \bd\cmpl S) \\
                                            & = (\inter S \cap \inter{(\cmpl S)}) \cup (\inter S \cap \bd{\cmpl S}) \cup
                                                (\bd S \cap \inter{(\cmpl S)}) \cup (\bd S \cap \bd{\cmpl S})
                \end{aligned}
            \]
            $\inter S \cap \inter{(\cmpl S)} = \varnothing$ since
            $S \cap \cmpl S = \varnothing$ and $\inter S \subseteq S$ and $\inter{\cmpl S} \subseteq \cmpl S$.\par
            $\bd S = \bd \cmpl S$ is direct from their definitions. Thus, \[
                \begin{aligned}[t]
                    \inter S \cap \bd{\cmpl S}   & = \inter S \cap \bd S = \varnothing \\
                    \bd S \cap \inter{(\cmpl S)} & = \bd \cmpl S \cap \inter{(\cmpl S)} = \varnothing
                \end{aligned}
            \]
            by (iv). Therefore, $\cl S \cap \cl{\cmpl S} = \bd S \cap \bd{\cmpl S} = \bd S$.
    \end{enumerate}
}

\dfn{Diameter}{
    Let $\varnothing \neq S \subseteq \RR[n]$ be a bounded set.
    The \textit{diameter} of $S$ is defined to be \[
        d(S) \triangleq \sup \{\, \|\vec x - \vec y\| \mid \vec x, \vec y \in S \,\}\text{.}
    \]
}

\dfn{Distance}{
    Let $\varnothing \neq S \subseteq \RR[n]$ and $\vec x \in \RR[n]$.
    The distance from $\vec x$ to $S$ is defined to be \[
        d(\vec x, S) \triangleq \inf \{\, \|\vec x - \vec y\| \mid \vec y \in S \,\}\text{.}
    \]
}

\thm{}{
    Let $S$ be a nonempty set in $\RR[n]$ and let $\vec x$ be a point of $\RR[n]$.
    \begin{enumerate}[nolistsep, label=(\roman*)]
        \ii $d(\vec x, S) = 0$ if and only if $\vec x \in \cl S$.
        \ii $S$ is closed if and only if $d(\vec x, S) > 0$ for every $\vec x \in \cmpl S$.
        \ii If $S$ is closed, then there exists $\vec y_0 \in S$ such that $d(\vec x, S) = \|\vec x - \vec y_0\|$.
        \ii If $S$ is open and if $\vec x \in \cmpl S$, then there exists no $\vec y \in S$ such that $d(\vec x, S) = \|\vec x - \vec y\|$.
    \end{enumerate}
}
\pf{Proof}{
    $ $\\[-1em]
    \begin{enumerate}[nolistsep, label=(\roman*)]
        \ii ($\Rightarrow$) We shall show that if such $\vec x$ is not in $S$, then it is in $S'$.
            So, suppose $\vec x \notin S$.
            By \Cref{th:sup1}, for any $\veps \in \RR_+$, there is some $\vec y \in S$ such that
            $0 \le \|\vec x - \vec y\| < \veps$. Since $\vec x \notin S$, $\vec x \neq \vec y$, and thus
            $\vec y \in N'(\vec x; \veps) \cap S$, implying $\vec x$ is a limit point of $S$. \par
            ($\Leftarrow$) Conversely, if $\vec x \in S' \setminus S$, then for all $\veps \in \RR_+$,
            there is some $\vec z \in S$ such that $0 < \|\vec x - \vec z\| < \veps$.
            Therefore, $0 \le d(\vec x, S) < \veps$.
            Since $\veps$ is arbitrary, $d(\vec x, S) = 0$. \checkmark
        \ii ($\Rightarrow$) $d(\vec x, S) = 0$ if and only if $\vec x \in \cl S = S$.
            Therefore, $d(\vec x, S) > 0$ if and only if $\vec x \in \cmpl S$. \par
            ($\Leftarrow$) For every $\vec x \in \cmpl S$, $\vec x \notin \cl S$ by (i).
            Thus, if $\vec x \in \cl S$, then $\vec x \in S$, or, $\cl S \subseteq S$.
            $S$ is therefore closed. \checkmark
        \ii If $S$ is finite, then we can easily see $d(\vec x, S) = \min \{\,\|\vec x- \vec y\| \mid \vec y \in S\,\}$.

            Therefore, now suppose $S$ is infinite.
            Let $\{\veps_k\}_{k \in \NN}$ be a sequence defined by $\veps_k = 1/k$ for each $k \in \NN$.
            By \Cref{th:sup1}, for each $k \in \NN$, we can find $\vec y_k \in S$ that satisfies \[
                d(\vec x, S) \le \|\vec x-\vec y_k\| < d(\vec x, S) + \veps_k\text{.}
            \]
            If the set $\{\,\vec y_k \mid k \in \NN\,\}$ is finite,
            then there must be some $\vec y_k$ such that $\|\vec x-\vec y_k\| = d(\vec x, S)$, and we are done.

            Suppose $\{\,\vec y_k \mid k \in \NN\,\}$ is infinite.
            Since the set is also bounded ($\|x-y_k\| < d(\vec x, S) + \veps_1$ for each $k \in \NN$),
            by \Cref{th:BW2}, there is a convergent subsequence $\{\vec y_{k_j}\}_{j \in \NN}$ of $\{\vec y_k\}_{k \in \NN}$.
            Let $\vec y_0 \triangleq \lim_{j \to \infty} \vec y_{k_j}$. Since \[
                d(\vec x, S) \le \|\vec x-\vec y_{k_j}\| < d(\vec x, S) + \veps_{k_j}
            \] still holds, it follows that $\|\vec x-\vec y_0\| = d(\vec x, S)$ by \Cref{th:squeeze}.

            $\vec y_0 \in S$ since $S$ is closed and $\vec y_0 \in S$ is a limit point of $S$. \checkmark
        \ii Suppose there is some $\vec y \in S$ such that $d(\vec x, S) = \|\vec x-\vec y\|$.
            $\|\vec x - \vec y\| > 0$ since $\vec x \neq \vec y$.
            Since $S$ is open, there is some neighborhood $N(\vec y; r_0)$ of $\vec y$ such that $N(\vec y; r_0) \subseteq S$.
            It must be $r_0 \le \|\vec x - \vec y\|$. Let \[
                \vec z \triangleq \vec y + \frac{r_0}{2} \cdot \frac{\vec x - \vec y}{\|\vec x - \vec y\|}
            \]
            Then, \[
                \|\vec z - \vec y\| = \frac{r_0}{2} < r_0\text{,}
            \] thus $\vec z \in N(\vec y; r_0) \subseteq S$. However, \[
                \|\vec x - \vec y\| = \left| 1 - \frac{r_0}{2\|\vec x-\vec y\|} \right|\,\|\vec x-\vec y\| < \|\vec x - \vec y\|\text{,}
            \] contradicting the minimality of $\|\vec x- \vec y\|$, \#. \checkmark
    \end{enumerate}
}

\section{Completeness}

\dfn{Nested Sets}{
    A sequence $\{S_k\}$ of sets in $\RR[n]$ such that $S_k \supset S_{k+1}$ for each $k \in \NN$
    is said to be \textit{nested}.
}

\thm[nestedInterval]{Cantor's Nested Interval Theorem}{
    For each $k \in \NN$, let $I_k = [a_k, b_k]$ with $a_k < b_k$.
    Suppose that $\{I_k\}_{k \in \NN}$ is a nested sequence in $\RR$. Then \[
        \bigcap_{k=1}^\infty I_k = [\alpha, \beta]
    \] where $\alpha = \sup \{\,a_k \mid k \in \NN\,\}$ and $\beta = \inf \{\,b_k \mid k \in \NN\,\}$.
}
\pf{Proof}{
    Let $A \triangleq \{\,a_k \mid k \in \NN\,\}$ and $B \triangleq \{\,b_k \mid k \in \NN\,\}$.
    Then, $A$ is bounded above by any $b_k$ and $B$ is bounded below by any $a_k$.
    Thus, by \Cref{th:completeR}, $\alpha = \sup A$ and $\beta = \sup B$ exist.
    
    Any $a_k$ is a lower bound of $B$, therefore $a_k \le \beta$ for each $k \in \NN$,
    which implies $\beta$ is an upper bound of $A$. Hence $\alpha \le \beta$.
    
    To prove $\bigcap_{k=1}^\infty I_k \supseteq [\alpha, \beta]$, take any $x \in [\alpha, \beta]$.
    Then, for each $k \in \NN$, $a_k \le \alpha \le x \le \beta \le b_k$, which means $x \in I_k$.
    Thus, $[\alpha, \beta] \subseteq \bigcap_{k=1}^\infty I_k$.
    Now, to prove the reverse containment, take any $x \in \bigcap_{k=1}^\infty I_k$.
    This means $\forall k \in \NN,\: a_k \le x \le b_k$;
    $x$ is an upper bound of $A$ and is a lower bound of $B$ at the same time.
    Therefore, $\alpha \le x \le \beta$, hence $\bigcap_{k=1}^\infty I_k \subseteq [\alpha, \beta]$.
}
\pf{Another Proof}{
    Since the sequences $\{a_k\}$ and $\{b_k\}$ are bounded and monotone,
    there are limits $\alpha = \lim_{k \to \infty} a_k$ and $\beta = \lim_{k \to \infty} b_k$ by \Cref{th:bddMonotoneSeqConv}.
    By \Cref{th:limIsOrderPrsving}, $\alpha \le \beta$.

    Since $a_k \le \alpha \le \beta \le b_k$ for each $k \in \NN$, $[a, b] \subseteq \bigcap_{k=1}^n I_k$.

    Now, take any $x \in \bigcap_{k=1}^\infty I_k$.
    Then, for all $k \in \NN$, $a_k \le x \le b_k$.
    If it were $\alpha > x$, there is some $k_0 \in \NN$ such that $a_{k_0} > x$.
    It is similar for the case when $\beta < x$. Therefore, $\alpha \le x \le \beta$.
    We have proven that $\bigcap_{k=1}^\infty I_k \subseteq [\alpha, \beta]$.
}

\cor{}{
    If, in the notation of the previous theorem, $\lim_{k \to \infty} (b_k - a_k) = 0$,
    then $\bigcap_{k=1}^\infty I_k$ is a singleton.
}
\pf{Proof}{
    Take any $\veps \in \RR_+$. Then, there is some $k_0 \in \NN$ such that $b_{k_0} - a_{k_0} < \veps$. \[
        0 \le \beta - \alpha \le b_{k_0} - a_{k_0} < \veps
    \] holds. This implies that $0 \le \beta - \alpha < \veps$ for arbitrary $\veps \in \RR_+$; therfore $\alpha = \beta$.
}

\thm[CantorCriterionRn]{Cartor's Criterion}{
    If $\{C_k\}$ is a nested sequence of closed, bounded, nonempty subsets of $\RR[n]$, then \[
        \bigcap_{k=1}^\infty C_k \neq \varnothing\text{.}
    \]
    Furthermore, if $\lim_{k \to \infty} d(C_k) = 0$, then $\bigcap_{k=1}^\infty C_k$ is a singleton.
}
\pf{Proof}{
    If any of $C_k$ is finite, it is trivial. So, we suppose every $C_k$ is infinite.
    Construct a seqeunce $\{\vec x_k\}$ of points in $\RR[n]$ as following.
    \begin{itemize}[nolistsep]
        \ii Take any $\vec x_1$ in $C_1$.
        \ii For each $k \in \NN$, take any $\vec x_{k+1}$ in $C_{k+1} \setminus \{\,x_1, x_2, \cdots, x_k\,\}$.
    \end{itemize}
    Since $S \subseteq C_1$ is bounded and contains infinitely many points,
    by \Cref{th:BW2}, there is a limit point $\vec x_0$ of $S$ in $\RR[n]$.
    We now claim that $\vec x_0 \in \bigcap_{k=1}^\infty C_k$.

    Fix any $k \in \NN$ and choose any deleted neighborhood $N'(\vec x_0)$ of $\vec x_0$.
    Since $N'(\vec x_0) \cap S$ is infinite, there is some $k_1 \in \NN_{>k}$ such that $\vec x_{k_1} \in N'(\vec x_0)$.
    By the construction, $\vec x_{k_1} \in C_{k_1} \subseteq C_k$.
    This shows that every deleted neighborhood of $\vec x_0$ contains a point in $C_k$; $\vec x_0 \in C_k'$.
    As each $C_k$ is closed, $\vec x_0 \in C_k$, and thus $\vec x_0 \in \bigcap_{k=1}^\infty C_k$.

    Suppose, in addition, $\lim_{k \to \infty} d(C_k) = 0$.
    Assume $\bigcap_{k=1}^\infty C_k$ has two distinct points $\vec x$ and $\vec y$ for the sake of contradiction.
    Choose any $\veps \in (0, \|\vec x-\vec y\|)$ and then there is some $k \in \NN$ with $d(C_k) < \veps$.
    Nonetheless, $\veps < \|\vec x-\vec y\| \le d(C_k) < \veps$, \#.
}

\thm{Cantor's Criterion in $\RR[n]$ implies Cantor's Criterion in $\RR$}{
    Cantor's criterion in $\RR[n]$ implies Cantor's criterion also holds in $\RR$.
}
\pf{Proof}{
    $\RR \times \{0\} \times \cdots \times \{0\}$ is a closed subset of $\RR[n]$.
}

\thm[cantorCritImpliesSupExs]{}{
    Cantor's criterion in $\RR$ and Archimedes' principle implies the existence of supremum
    of any bounded above nonempty subset of $\RR$.
}
\pf{Proof}{
    Let $S$ be a nonempty, bounded above set in $\RR$.
    Let $B$ denote the set of upper bounds of $S$ and let $A = \cmpl B$.
    Since $x - 1 \in A$ for all $x \in S$, $A \neq \varnothing$.
    
    We first show that for all $a \in A$ and $b \in B$, $a < b$.
    If otherwise, i.e., $a \ge b$, $x \le b \le a$ for each $x \in S$,
    which implies $a \in B$, which is a contradiction.

    Moreover, $S \cap [a, b] \neq \varnothing$ for each $a \in A$ and $b \in B$.
    Assume $S \cap [a, b] = \varnothing$ for the sake of contradiction.
    Since $S \cap (b, \infty) = \varnothing$ as $b$ is an upper bound of $S$,
    then it follows $S \subseteq (-\infty, a)$, which implies $a$
    is an upper bound of $S$, which is a contradiction.

    Construct a nested sequence $\{[a_k, b_k]\}_{k \in \NN}$ of closed interval
    of which each $a_k$ is in $A$ and $b_k$ is in $B$.
    \begin{itemize}[nolistsep]
        \ii Take any $a_1$ in $A$ and $b_1$ in $B$.
        \ii For each $k \in \NN$, if $(a_k + b_k)/2 \in A$, then let $a_{k+1} \triangleq (a_k + b_k)/2$ and $b_{k+1} \triangleq b_k$.
            If $(a_k + b_k)/2 \in B$, then let $a_{k+1} \triangleq a_k$ and $b_{k+1} \triangleq (a_k + b_k)/2$.
    \end{itemize}
    Then it is immediate that $\lim_{k \to \infty} (b_k - a_k) = \lim_{k \to \infty} 2^{-k+1}(b_1-a_1) = 0$.
    Therefore, by Cantor's criterion in $\RR$, $\bigcap_{k=1}^\infty [a_k, b_k] = \{x_0\}$ for some $x_0 \in \RR$.

    We now show that $x_0$ is an upper bound of $S$.
    Assume not for the sake of contradiction, that is,
    there is some $x \in S$ such that $x > x_0$.
    Then, we may find some $k \in \NN$ such that $b_k - a_k < x - x_0$.
    Then it follows $b_k - x_0 \le b_k - a_k < x - x_0$, and therefore $b_k < x_0$.
    This contradicts that $b_k$ is an upper bound of $S$. Thus, $x_0 \in B$.

    We now claim that $x_0$ is the least upper bound.
    Assume to the contrary that there is some $b \in B$ such that $b < x_0$.
    Then, we may find some $k \in \NN$ such that $b_k - a_k < x_0 - b$.
    It follows $x_0 - a_k \le b_k - a_k < x_0 - b$, and therefore $b < a_k$.
    This contradicts that $a_k \in A$.
}

\thm{}{
    Assuming that Archimedes' principle holds in $\RR$, the following are equivalent.
    \begin{enumerate}[nolistsep, label=(\roman*)]
        \ii Eery nonempty supset in $\RR$ which is bounded above has a supremum in $\RR$.
        \ii Every bounded monotone sequence in $\RR$ converges.
        \ii $\RR$ has Bolzano–Weierstrass property.
        \ii $\RR$ is Cauchy complete.
        \ii $\RR[n]$ is Cauchy complete.
        \ii $\RR[n]$ has Bolzano–Weierstrass property.
        \ii Cantor's criterion holds in $\RR[n]$.
        \ii Cantor's criterion holds in $\RR$.
    \end{enumerate}
}
\pf{Proof}{
    \[
        \text{(i)} \overbrace{\implies}^{\text{\Cref{th:bddMonotoneSeqConv}}} \text{(ii)}
        \overbrace{\implies}^{\text{\Cref{th:nestedInterval}}} \text{(viii)}
        \overbrace{\implies}^{\text{\Cref{th:cantorCritImpliesSupExs}}} \text{(i)}
    \]
    \[
        \begin{aligned}[t]
            \text{(i)} \overbrace{\implies}^{\text{\Cref{th:BW}}} \text{(iii)}
            \overbrace{\implies}^{\text{\Cref{th:convIffCauchy}}} &\text{(iv)}
            \overbrace{\implies}^{\text{\Cref{th:convIffCauchyRn}}} \text{(v)}
            \overbrace{\implies}^{\text{\Cref{th:BW2}}} \text{(vi)} \\
            &\overbrace{\implies}^{\text{\Cref{th:CantorCriterionRn}}} \text{(vii)}
            \overbrace{\implies}^{\text{special case}} \text{(viii)}
            \overbrace{\implies}^{\text{\Cref{th:cantorCritImpliesSupExs}}} \text{(i)}
        \end{aligned}
    \]
}

\dfn{Completeness}{
    When the word \textit{complete} is applied to $\RR[n]$,
    it is assumed that it means any of these statements:
    \begin{itemize}[nolistsep]
        \ii the existence of least upper bounds in $\RR$,
        \ii the Monotone Convergence theorem in $\RR[n]$,
        \ii Cantor's criterion,
        \ii the Bolzano–Weierstrass property, or
        \ii Cauchy completeness.
    \end{itemize}
}

\section{Relative Topology and Connectedness}

\end{document}
