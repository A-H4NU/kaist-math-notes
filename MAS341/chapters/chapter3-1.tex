\documentclass[../complex_variables_1.tex]{subfiles}

\begin{document}

\section{Cauchy--Riemann Equation}

\begin{Definition}{Continuity}[]
    For a fixed point \(z_0 \in \CC\), a function
    \(f\) is said to be continuous at \(z_0\) if
    \[
        \lim_{\abs{z-z_0} \to 0} \abs{f(z)-f(z_0)} = 0\text.
    \]
\end{Definition}

\begin{Definition}{Differentiability}[]
    For a fixed point \(z_0 \in \CC\), a function
    \(f\) is said to be \emph{continuous at \(z_0\)} if
    \[
        \lim_{\substack{|\omega| \to 0 \\ \omega \in \CC}} \frac{f(z_0+\omega) - f(z_0)}{\omega}
    \]
    exists. If \(f\) is differentiable at \(z_0\), then define the \emph{derivative} of \(f\) at
    \(z_0\) by
    \[
        f'(z_0) \coloneqq \lim_{\substack{|\omega| \to 0 \\ \omega \in \CC}} \frac{f(z_0+\omega) -
        f(z_0)}{\omega}\text.
    \]
\end{Definition}

\begin{Example}{}[]
    For each \(n \in \NN\), one can derive that \(f'(z) = nz^{n-1}\) where \(f(z) = z^n\).
\end{Example}

\begin{Theorem}{}[]
    If \(f\) is differentiable at \(z_0\), then it is continuous at \(z_0\).
\end{Theorem}

\begin{Example}{}[]
    Let us determine differentiability of \(f(z) = |z|^2\).
    Write \(z = x+iy\) and \(\omega = p+iq\) for \(x,y,p,q\in\RR\).
    Then,
    \[
        \frac{f(z+\omega)-f(z)}{\omega} = \frac{2(xp+yq)+|\omega|^2}{\omega}
    \]
    As we know \(\lim_{\omega \to 0} \frac{|\omega|^2}{\omega} = 0\), we only need to care if
    \(\lim_{\omega \to 0} \frac{2(xp+yq)}{p+iq}\).
    Evaluating the limit along the real axis and the imaginary axis gives \(2x\) and \(-2yi\);
    hence \(f\) is not differentiable at \(z \in \CC \setminus \{0\}\).
    At the origin, we have \(f'(0) = \lim_{\omega \to 0} \frac{f(0+\omega)-f(0)}{\omega} = 0\).
\end{Example}

\begin{Theorem}{}[]
    Product, quotient, chain rule still holds in complex derivative.
    % TODO: Maybe list those derivative rules?
\end{Theorem}

\begin{Theorem}{Cauchy--Riemann Equation}[crEq]
    If \(f\) is differentiable at \(z\), then \(f_y(z) = if_x(z)\) at \(z\),
    or equivalently,
    \[
        \left\{\begin{aligned}[c]
                u_x &= v_y \\
                u_y &= -v_x
        \end{aligned}\right.
    \]
    where \(u(x, y) \coloneqq \Re f(x+iy)\) and \(v(x, y) \coloneqq \Im f(x+iy)\)
    for \(x, y \in \RR\).
\end{Theorem}
\begin{myproof}[Proof]
    \(\displaystyle f_x(z) = \lim_{\xi \to 0} \frac{f(z + \xi) - f(z)}{\xi} = f'(z)\) and
    \(\displaystyle -if_y(z) = \lim_{\eta \to 0} \frac{f(z + i\eta) - f(z)}{i\eta} = f'(z)\).
\end{myproof}

\begin{Example}{}[]
    Is \(e^z\) differentiable in \(\CC\)?
    \begin{align*}
        \lim_{h \to 0} \frac{e^h - 1}{h}
        &= \lim_{(\xi,\eta) \to 0} \frac{(e^\xi - 1) e^{i\eta} + (e^{i\eta} - 1)}{\xi+i\eta} \\
    \end{align*}
    % TODO: Fill the detail here.
\end{Example}

\nt{%
    We may write \(f(z) = u \left( \frac{z+\ol{z}}{2}, \frac{z-\ol{z}}{2i} \right) + iv \left(
    \frac{z+\ol{z}}{2}, \frac{z-\ol{z}}{2i} \right)\).
    If \(f\) is differentiable, we \emph{define}
    \begin{align*}
        \frac{\partial f}{\partial z} &\coloneqq
        \left( \frac{1}{2}\partial_x + \frac{1}{2i} \partial_y \right) u
        + i \left( \frac{1}{2} \partial_x + \frac{1}{2i} \partial_y \right) v
        = \left( \frac{1}{2} \partial_x + \frac{1}{2i} \partial_y \right) f \\
        \frac{\partial f}{\partial \ol{z}} &\coloneqq
        \left( \frac{1}{2}\partial_x - \frac{1}{2i} \partial_y \right) u
        + i \left( \frac{1}{2} \partial_x - \frac{1}{2i} \partial_y \right) v
        = \left( \frac{1}{2} \partial_x - \frac{1}{2i} \partial_y \right) f\text.
    \end{align*}
    So that \(\frac{\partial f}{\partial\ol{z}} = \frac{1}{2}(\partial_x + i\partial_y)f = 0\)
    if \(f\) is differentiable.
}

\begin{Definition}{Domain}[]
    A domain is an open and connected subset of \(\CC\).
\end{Definition}

\begin{Theorem}{}[domainPolyline]
    Any two points in a domain can be connected by polygonal lines
    parallel to the coordinate axes
    that lies in the domain.
\end{Theorem}
\begin{myproof}[Proof]
    Let \(D\) be a domain and let \(z_0 \in D\).
    Let \(A \subseteq D\) be the set of all points in \(D\)
    that can be connected from \(z_0\) by polygonal lines parallel to the coordinate axes.
    Let \(B \coloneqq D \setminus A\).
    If \(z \in A\) and \(r > 0\) satisfy \(B_r(z) \subseteq D\), then
    \(B_r(z) \subseteq A\); hence \(A\) is open.
    Similarly, \(B\) is open as well.
    As \(D\) is connected, \(A\) or \(B\) is empty but \(z_0 \in A\); hence, \(B = \OO\).
\end{myproof}

\begin{Theorem}{}[diffZero]
    If \(f'(z) \equiv 0\) in a domain \(D\), then \(f\) is constant on \(D\).
\end{Theorem}
\begin{myproof}[Proof]
    \(f_x \equiv f_y \equiv 0\); hence \(u_x \equiv v_x \equiv u_y \equiv u_x \equiv 0\) on \(D\).
    Thus, \(f\) is contant on every line segment in \(D\) parallel to coordinate axes.
    Hence, \(f\) is constant on \(D\).
\end{myproof}

\begin{Corollary}{}[diffConst]
    Let \(f\) be differentiable on a domain \(D\).
    \begin{enumerate}[label=(\arabic*), ref=\protect{\Cref{th:diffConst} (\arabic*)}]
        \ii\label{itm:diffConst.1}
        If \(\Re f(z)\) is constant on \(D\), then \(f\) is constant on \(D\).
        \ii\label{itm:diffConst.2}
        If \(\Im f(z)\) is constant on \(D\), then \(f\) is constant on \(D\).
        \ii\label{itm:diffConst.3}
        If \(\Arg f(z)\) is constant on \(D\), then \(f\) is constant on \(D\).
    \end{enumerate}
\end{Corollary}
\begin{myclaim}[Proof]\hfill
\begin{enumerate}[label=(\arabic*)]
    \ii
    There is \(\omega_0 \in \CC\) such that, when \(g\) is defined by \(g(z) \triangleq f(z) -
    \omega_0\), we have \(\Re g(z) \equiv 0\) and \(g\) is differentiable on \(D\).
    \[
        \lim_{\xi \to 0} \frac{f(z + \xi) - f(z)}{\xi} = f'(z)
        = \lim_{\eta \to 0} \frac{f(z + i\eta) - f(z)}{i\eta}
    \]
    where the left hand side is real and the right hand side is purely imaginary.
    Therefore, \(f'(z) = 0\) for all \(z \in D\).
    The result follows from \Cref{th:diffZero}.

    \ii
    Let \(g(z) = if(z)\) so that \(g\) is differentiable on \(D\) and \(\Re g(z)\) is constant.
    Therefore, by (1), \(g\) is constant and thus \(f\) is constant.

    \ii
    There is \(\omega_0 \in \RR\) such that, when \(g\) is defined by \(g(z) \triangleq f(z)e^{-i\omega_0}\),
    we have \(\Re g(z)\) is constant and \(g\) is differentiable on \(D\).
    Tehrefore, by (1), \(g\) is constant and thus \(f\) is constant.
    \qed
\end{enumerate}
\end{myclaim}

\end{document}
