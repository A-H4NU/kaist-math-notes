\documentclass[../complex_variables_1.tex]{subfiles}

\begin{document}

\section{Complex Exponents}

\begin{Definition}{Complex Exponents}[]
    For \(z \in \CC \setminus \{0\}\) and \(w \in \CC\), define
    \[
        z^w \coloneqq e^{w \ln z}\text.
    \]
\end{Definition}
\nt{%
    Complex exponentiation is not a function!
    If one considers the complex exponentiation as a set of possible values,
    then \(z^{\eta_1} \cdot z^{\eta_2} = z^{\eta_1+\eta_2}\) may easily fail!
}

\begin{Example}{}[]
    To solve \(z^{1-i} = 4\), write \(e^{(1-i)\ln z} = e^{\ln 4}\),
    i.e., \(\ln z = (1+i)(\ln 2 + k\pi i)\) for \(k \in \ZZ\).
    In other words, \(\ln \abs{z} + i \arg z = (\ln 2 - k\pi) + i(\ln 2 +k\pi)\).
    Hence, \(\abs{z} = e^{\ln 2 - k\pi}\) and \(\arg z = \ln 2 + k\pi \pmod{2\pi}\).
\end{Example}

\end{document}
