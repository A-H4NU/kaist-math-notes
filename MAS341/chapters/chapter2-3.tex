\documentclass[../complex_variables_1.tex]{subfiles}

\begin{document}

\section{Logarithmic ``Functions''}

\begin{Definition}{Logarithmic Function}[]
    For any \(z \in \CC \setminus \{0\}\),
    we define \(w = \ln z\) if and only if \(e^w = z\).
\end{Definition}

\begin{note}
    How to compute \(\ln z\)?
    Note that \(z = \abs{z} \cdot e^{i(\Arg z + 2k\pi)}\)
    for \(k \in \ZZ\).
    Let \(w = u + iv\) where \(u, v \in \RR\)
    so that \(e^w = e^u \cdot e^{iv} = \abs{z} \cdot e^{i(\Arg z + 2k\pi)}\).
    Hence, we have \(u = \ln\abs{z}\) and \(v = \Arg z + 2k\pi\).
    In other words, \(\ln z = \ln\abs{z} + i\arg z\). (Note that this is not a ``function''!)
\end{note}


\begin{Definition}{Principal Logarithmic Function}[]
    For any \(z \in \CC \setminus \{0\}\),
    we define \(\Ln z \coloneqq \ln\abs{z} + i \Arg z\)
    and it is called the \emph{principal value of \(\ln z\)}.
\end{Definition}

\end{document}
